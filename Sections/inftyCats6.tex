	\section{\texorpdfstring{$\infty$}{Infinity}-Category theory}\label{sec:InftyCategoryTheory}

Armed with Lurie's straightening equivalence and the quasi-categorical Yoneda lemma, we will spend \crefrange{subsec:Adjunctions}{subsec:KanExtensions} redeveloping the theory from \cref{sec:CategoryTheory} (and more) in the setting of quasi-categories. In \cref{subsec:EilenbergMacLane} we will see a first major application to topology and in \cref{subsec:Presentable} we will prove Lurie's adjoint functor theorem.

Even though, implicitly, we work with quasi-categories, our arguments in \crefrange{subsec:Adjunctions}{subsec:KanExtensions} will be almost entirely model-independent; the same is true, at least in large parts, for \cref{subsec:EilenbergMacLane} and \cref{subsec:Presentable}. So from now on, instead of quasi-categories, we'll simply write \emph{$\infty$-categories}. We'll consider ordinary categories as $\infty$-categories via the nerve construction, but we'll always suppress $\N$ in our notation. Furthermore, we'll write $\Fun(\Cc,\Dd)$ instead of $\F(\Cc,\Dd)$ for $\infty$-categories $\Cc$ and $\Dd$. We'll only switch back to the old terminology in the few instances where non-model-independent arguments are used. I believe these few exceptions could easily be treated in any other model of $\infty$-categories as well. Also recall from \cref{par:ModelIndependence} that many constructions and results so far (like cocartesian/left fibrations) can be reformulated in a model-independent fashion, and this is how we're going to use them.
%
%, which 
%
%the notions of cocartesian and left fibrations as well as Lurie's straightening/unstraightening equivalence can be reformulated in a model-independent way, and this is the way in which we're going to use them.

\subsection{Adjunctions}\label{subsec:Adjunctions}
\begin{defi}\label{def:Adjunction}
	Let $L\colon \Cc\rightarrow \Dd$ be a functor of $\infty$-categories.
	\begin{alphanumerate}
		\item Let $y\in \Dd$. An object $x\in \Cc$ is a \emph{right adjoint object to $y$ under $L$} if there exists an equivalence
		\begin{equation*}
			\Hom_\Cc\!\left(-,x\right)\simeq \Hom_\Dd\lef(L(-),y\righ)
		\end{equation*}
		in the functor category $\Fun\left(\Cc^\op,\cat{An}\right)$.
		\item A functor $R\colon \Dd\rightarrow \Cc$ is a \emph{right adjoint of $L$} if there exists an equivalence
		\begin{equation*}
			\Hom_\Cc\lef(-,R(-)\righ)\simeq \Hom_\Dd\lef(L(-),-\righ)
		\end{equation*}
		in the functor category $\Fun(\Cc^\op\times \Dd,\cat{Set})$. In this case we write $L\dashv R$.
	\end{alphanumerate}
\end{defi}
\begin{lem}[\enquote{Adjoints can be constructed pointwise}]\label{lem:Adjunction}
	A functor $L\colon \Cc\rightarrow \Dd$ has a right adjoint if and only if every $y\in \Dd$ has a right adjoint object $x\in \Cc$.
\end{lem}
\begin{proof}
	One implication is trivial: If $R\colon \Dd\rightarrow \Cc$ is a right adjoint of $L$, then $R(y)$ is a right adjoint object of $y$ for every $y\in \Dd$. For the other implication, consider $\Hom_\Dd(L(-),-)\colon \Cc^\op\times\Dd\rightarrow\cat{An}$ as a functor $\overline{R}\colon \Dd\rightarrow \Fun(\Cc^\op,\cat{An})$. Our assumption implies that $\overline{R}$ takes values in the image of the Yoneda embedding $\Yo_\Cc\colon \Cc\rightarrow\Fun(\Cc^\op,\cat{An})$; namely, $\overline{R}(y)\simeq \Hom_\Cc(-,x)$ if $x\in\Cc$ is a right adjoint object of $y\in\Dd$ under $L$. Since $\Yo_\Cc$ is an equivalence onto its image, we obtain a functor $R\colon \Dd\rightarrow \Cc$ with the required properties.
\end{proof}
\begin{exm}\label{exm:Adjunctions}
	It's clear that any adjunction of ordinary categories is also an adjunction of $\infty$-categories. Furthermore, we've already seen some non-trivial examples of adjunctions of $\infty$-categories:
	\begin{alphanumerate}
		\item The inclusion $\cat{An}\subseteq \cat{Cat}_\infty$ is fully faithful (by \cref{thm:CordierPorter} and \cref{cor:FIsKanComplex}) and has both adjoints: A right adjoint $\core\colon \cat{Cat}_\infty\rightarrow \cat{An}$ and a left adjoint $\left|\,\cdot\,\right|\colon \cat{Cat}_\infty\rightarrow \cat{An}$ sending $\Cc$ to $\left|\Cc\right|$, the localisation of $\Cc$ at all its morphisms.\label{enum:AnToCatInfty}
		\item For every $\infty$-category $\Cc$, the functor $-\times \Cc\colon \cat{Cat}_\infty\rightarrow \cat{Cat}_\infty$ has a right adjoint, which sends an $\infty$-category $\Dd$ to $\Fun(\Cc,\Dd)$.\label{enum:Currying}
	\end{alphanumerate}
	Both \cref{enum:AnToCatInfty} and \cref{enum:Currying} can easily be seen using \cref{lem:Adjunction}: For \cref{enum:AnToCatInfty}, it's enough to check that the functors $i\colon \core(\Dd)\rightarrow \Dd$ and $p\colon \Cc\rightarrow \left|\Cc\right|$ induce functorial equivalences
	\begin{equation*}
		i_*\colon \Hom_{\cat{An}}\lef(-,\core(\Dd)\righ)\overset{\simeq}{\Longrightarrow}\Hom_{\cat{Cat}_\infty}\!\left(-,\Dd\right)\,,\ \,p^*\colon \Hom_{\cat{An}}\lef(\left|\Cc\right|,-\righ)\overset{\simeq}{\Longrightarrow}\Hom_{\cat{Cat}_\infty}\!\left(\Cc,-\right)
	\end{equation*}
	via post- and precomposition, respectively. Indeed, equivalences can be checked pointwise by \cref{thm:EquivalencePointwise} and then we can apply \cref{lem:Localisation} (and a similar assertion for $\core$). For \cref{enum:Currying}, observe that we have an \emph{evaluation functor} $\ev\colon \Fun(\Cc,\Dd)\times\Cc\rightarrow \Dd$ for all $\infty$-categories $\Cc$ and $\Dd$. If we work with quasi-categories, this functor is simply given by the counit of the adjunction $-\times\Cc\colon\cat{sSet}\shortdoublelrmorphism \cat{sSet}\noloc \F(\Cc,-)$. Using \cref{lem:Adjunction}, it's enough to show that the composition
	\begin{equation*}
		\Hom_{\cat{Cat}_\infty}\lef(-,\Fun(\Cc,\Dd)\righ)\xRightarrow{-\times\Cc\vphantom{_y}}\Hom_{\cat{Cat}_\infty}\lef(-\times\Cc,\Fun(\Cc,\Dd)\times\Cc\righ)\overset{\ev_*\vphantom{_y}}{\Longrightarrow}\Hom_{\cat{Cat}_\infty}\!\left(-\times\Cc,\Dd\right)
	\end{equation*}
	is an equivalence. Again, \cref{thm:EquivalencePointwise} allows us to check this pointwise, and then \cref{thm:CordierPorter} reduces everything to the  adjunction $-\times\Cc\colon\cat{sSet}\shortdoublelrmorphism \cat{sSet}\noloc \F(\Cc,-)$ of ordinary categories.
	
	In particular, \cref{enum:Currying} allows us to define a functor $\Fun(\Cc,-)\colon \cat{Cat}_\infty\rightarrow \cat{Cat}_\infty$. We'd expect that these functors can be assembled into a two-argument functor
	\begin{equation*}
		\Fun(-,-)\colon \cat{Cat}_\infty^\op\times\cat{Cat}_\infty\longrightarrow \cat{Cat}_\infty\,.
	\end{equation*}
	This is indeed the case and we'll briefly explain the construction (not because we'll need it, but because it nicely illustrates these kinds of formal arguments are done). First, let $*\ \,*$ be the discrete category on two objects. In \cref{rem:ColimitFunctor} below, we'll construct a limit functor $\lim\colon \Fun(*\ \,*,\cat{Cat}_\infty)\rightarrow \cat{Cat}_\infty$. Under the identification $\Fun(*\ \,*,\cat{Cat}_\infty)\simeq \cat{Cat}_\infty\times\cat{Cat}_\infty$, this functor sends a pair $(\Cc,\Dd)$ of $\infty$-categories to the product $\Cc\times\Dd$. By \enquote{currying}, this functor corresponds to a functor $P\colon \cat{Cat}_\infty\rightarrow \Fun(\cat{Cat}_\infty,\cat{Cat}_\infty)$ sending $\Cc$ to $P(\Cc)\simeq -\times\Cc$. On a related note, this is how you construct the functor $-\times\Cc$ in \cref{enum:Currying}. As we've seen above, $P(\Cc)$ is a left adjoint, and so $P$ factors through the full sub-$\infty$-category $\Fun^\L(\cat{Cat}_\infty,\cat{Cat}_\infty)\subseteq \Fun(\cat{Cat}_\infty,\cat{Cat}_\infty)$ spanned by the left adjoint functors. By \cref{cor:ExtractingAdjoints} below, extracting adjoints induces an equivalence of $\infty$-categories $\Fun^\L(\cat{Cat}_\infty,\cat{Cat}_\infty)\simeq \Fun^\R(\cat{Cat}_\infty,\cat{Cat}_\infty)^\op$. Thus, $P^\op$ can be regarded as a functor $P^\op\colon \cat{Cat}_\infty^\op\rightarrow \Fun^\R(\cat{Cat}_\infty,\cat{Cat}_\infty)$, sending $\Cc$ to $P^\op(\Cc)\simeq \Fun(\Cc,-)$. \enquote{Currying} back, we obtain the desired functor $\Fun(-,-)\colon \cat{Cat}_\infty^\op\times\cat{Cat}_\infty\rightarrow \cat{Cat}_\infty$.
	
	It's true that the functor $\core\Fun(-,-)$ agrees with $\Hom_{\cat{Cat}_\infty}(-,-)$, but this is not so easy to see (and we won't need it). One way would be to turn $\F(-,-)\colon \cat{QCat}\times\cat{QCat}\rightarrow \cat{QCat}$ into a Kan-enriched functor and show that $\N_\Delta(\F(-,-))$ agrees with $\Fun(-,-)$. This is easy since $\N_\Delta$ turns Kan-enriched adjunctions into adjunctions of $\infty$-categories. Then one has to check that $\Hom_{\cat{Cat}_\infty}(-,-)$ agrees with $\N_\Delta$ applied to $\core\F(-,-)\colon\cat{QCat}\times\cat{QCat}\rightarrow \cat{Kan}$. At least for \cref{con:HomTwAr}, this is done in \cite[Proposition~\HAthm{5.2.1.11}]{HA}.
\end{exm}
Next, we'll characterise adjunctions in terms of unit and counit.
\begin{con}\label{con:Unit}
	Let $L\colon \Cc \shortdoublelrmorphism \Dd\noloc R$ be an adjunction. We obtain a natural transformation $u\colon \id_\Cc\Rightarrow RL$, called the \emph{unit} of the adjunction, as follows: Consider the natural transformations $\Hom_\Cc(-,-)\Rightarrow \Hom_\Dd(L(-),L(-))\simeq \Hom_\Dd(-,RL(-))$, where the first one is induced by functoriality of $L$ and the second by the given adjunction. We can consider these as a natural transformation $\Yo_\Cc\Rightarrow \Yo_\Cc\circ RL$ in $\Fun(\Cc,\Fun(\Cc^\op,\cat{An}))$. Since $\Yo_\Cc$ is fully faithful, we obtain a natural transformation $u\colon \id_\Cc\Rightarrow RL$, as desired. Dually, there is also a \emph{counit} $c\colon LR\Rightarrow \id_\Dd$, as usual.
\end{con}
\begin{lem}[Triangle identities]\label{lem:TriangleIdentities}
	Let $L\colon \Cc\shortdoublelrmorphism \Dd\noloc R$ be an adjunction of $\infty$-categories. Then there are commutative diagrams 
	\begin{equation*}
		\begin{tikzcd}
			L \doublear["Lu"{black,above=0.1em}]{r}\doublear["i_L"'{black}]{dr} & LRL\doublear["cL"{black,right=0.1em}]{d}\dar[phantom,""{name=A}]\arrow[from=1-1,to=A,commutes,pos=0.7]\\
			& L
		\end{tikzcd}\quad\text{and}\quad
		\begin{tikzcd}
			R \doublear["uR"{black,above=0.1em}]{r}\doublear["i_R"'{black}]{dr} & RLR\doublear["Rc"{black,right=0.1em}]{d}\dar[phantom,""{name=A}]\arrow[from=1-1,to=A,commutes,pos=0.7]\\
			& R
		\end{tikzcd}
	\end{equation*}
	where $i_L$ and $i_R$ are pointwise the identity \embrace{so they are equivalences by \cref{thm:EquivalencePointwise}}. Conversely, if $L$, $R$ are functors and $u:\id_\Cc\Rightarrow RL$, $c\colon LR\Rightarrow \id_\Dd$ are natural transformations that fit into diagrams as above, where $i_L$ and $i_R$ are equivalences \embrace{not necessarily pointwise the identity}, then $L$ and $R$ determine an adjunction. 
\end{lem}
\begin{proof}
	First observe that the composites
	\begin{align*}
		\Hom_\Dd\lef(L(-),-\righ)\overset{R}{\Longrightarrow}&\Hom_\Dd\lef(RL(-),R(-)\righ)\overset{u^*}{\Longrightarrow}\Hom_\Cc\lef(-,R(-)\righ)\,,\\
		\Hom_\Cc\lef(-,R(-)\righ)\overset{L}{\Longrightarrow}&\Hom_\Dd\lef(L(-),LR(-)\righ)\overset{c_*}{\Longrightarrow}\Hom_\Dd\lef(L(-),-\righ)
	\end{align*}
	agree pointwise with the adjunction equivalence $\Hom_\Dd(L(-),-)\simeq \Hom_\Cc(-,R(-))$ and its inverse, and are thus an equivalences themselves (\cref{thm:EquivalencePointwise}). Indeed, Yoneda's lemma (see the dual of \cref{thm:Yoneda}) tells us that for every fixed $x\in\Cc$, a natural transformation $\Hom_\Dd(L(x),-)\Rightarrow \Hom_\Dd(x,R(-))$ is determined up to contractible choice by the image of $\id_{L(x)}\colon L(x)\rightarrow L(x)$, which is a morphism $x\rightarrow RL(x)$. For the adjunction equivalence $\Hom_\Dd(L(-),-)\simeq \Hom_\Cc(-,R(-))$, that morphism is the unit $u_x\colon x\rightarrow RL(x)$ by definition. But the image of $\id_{L(x)}\colon L(x)\rightarrow L(x)$ under $u^*\circ R$ is also $u_x$. The same argument applies to show that $c_*\circ L$ agrees pointwise with $\Hom_\Cc(-,R(-))\simeq \Hom_\Dd(L(-),-)$.
	
	In the $1$-category case, this would be enough to show that they agree on the nose rather than pointwise, but in the $\infty$-category case it isn't. Nevertheless it should be true that they even agree on the nose , but I don't know any argument for this. If you do, please tell me. This would also allow us to replace $i_L$ and $i_R$ in the statement by $\id_L$ and $\id_R$, respectively.
	
	Now to prove the triangle identities, consider the diagram of natural transformations
	\begin{equation*}
		\begin{tikzcd}
			\Hom_\Dd\lef(L(-),-\righ)\doublear["R"{black,above=0.1em}]{r}\ar["(cL)^*"{black,swap},white,double, double equal sign distance,-{implies[black]},ddr,bend right]\ar[ddr,dash,shift right=0.1em,bend right,shorten >=0.35ex]\ar[ddr,dash,shift left=0.1em,bend right,shorten >=0.59ex,shorten <=0.35ex]\ar[ddr,commutes,pos=0.4]& \Hom_\Cc\lef(RL(-),R(-)\righ)\doublear["u^*"{black,above=0.1em}]{r}\doublear["L"{black,left=0.1em}]{d}\drar[commutes] & \Hom_\Cc\lef(-,R(-)\righ)\doublear["L"{black,right=0.1em}]{d}\\
			& \Hom_\Dd\lef(LRL(-),LR(-)\righ)\doublear["(Lu)^*"{black,above=0.1em}]{r}\doublear["c_*"{black,left=0.1em}]{d}\drar[commutes] & \Hom_\Dd\lef(L(-),LR(-)\righ)\doublear["c_*"{black,right=0.1em}]{d}\\
			& \Hom_\Dd\lef(LRL(-),-\righ)\doublear["(Lu)^*"{black,above=0.1em}]{r} & \Hom_\Dd\lef(L(-),-\righ)
		\end{tikzcd}
	\end{equation*}
	The top right square commutes by functoriality of $L$, the bottom right square commutes since pre- and postcomposition commute, and the left cell commuting is a consequence of $c\colon LR\Rightarrow \id_\Dd$ being a natural transformation (see \cref{lem:HomNaturalTransformation}). Now walking around the bottom part of the diagram shows that $(cL)^*\circ (Lu)^*\colon \Hom_\Dd(L(-),-)\Rightarrow \Hom_\Dd(L(-),-)$ agrees with $c_*\circ L\circ u^*\circ R$, which is pointwise the identity as seen above. This establishes the first triangle identity; the second one is analogous.
	
	Conversely, if $L$, $R$ are functors and $u:\id_\Cc\Rightarrow RL$, $c\colon LR\Rightarrow \id_\Dd$ are natural transformations satisfying the triangle identities, then the commutative diagram above (together with its dual) shows that $u^*\circ R$ and $c_*\circ L$ induce equivalences between $\Hom_\Dd(L(-),-)$ and $\Hom_\Cc(-,R(-))$ (which are pointwise inverse if $i_L$ and $i_R$ are pointwise the identity).
\end{proof}
\begin{cor}\label{cor:FunctorCategoryAdjunctions}
	Let $L\colon \Cc \shortdoublelrmorphism \Dd\noloc R$ be an adjunction and let $\Ii$ be another category. Then the pre- and postcomposition functors determine adjunctions
	\begin{align*}
		L\circ-\colon \Fun(\Ii,\Cc)&\doublelrmorphism \Fun(\Ii,\Dd)\noloc R\circ -\,,\\
		{-}\circ {R}\colon \Fun(\Cc,\Ii)&\doublelrmorphism \Fun(\Dd,\Ii)\noloc {-}\circ {L}\,.
	\end{align*}
\end{cor}
\begin{proof}
	The proof of \cref{cor:1FunctorCategoryAdjunctions} can be copied verbatim.
\end{proof}
To finish this subsection about adjunctions, we connect adjunctions to the theory of straightening/unstraightening. This won't be needed in the rest of this text (so feel free to skip it), but it's nice to know and a standard fact in other treatments of adjunctions of $\infty$-categories.
\begin{lem}\label{lem:AdjunctionBicartesian}
	Let $F\colon \Cc\rightarrow \Dd$ be a functor of $\infty$-categories, corresponding to a functor $\Delta^1\rightarrow \cat{Cat}_\infty$ \embrace{see \cref{exm:SimplicialNerve}}, which in turn corresponds to a cocartesian fibration $p\colon \Uu\rightarrow \Delta^1$ by \cref{thm:Straightening}\cref{enum:CocartesianStraightening}. Then the following are equivalent:
	\begin{alphanumerate}
		\item $F$ admits a right adjoint $G\colon \Dd\rightarrow \Cc$.\label{enum:BicartesianAdjoint}
		\item \!The cocartesian fibration $p\colon \Uu\rightarrow \Delta^1$ is also a cartesian fibration.\label{enum:Bicartesian}
	\end{alphanumerate}
	Furthermore, in this case $G$ agrees with the functor classified by the cartesian straightening $\operatorname{St}^{(\mathrm{cart})}(p)\colon (\Delta^1)^\op\rightarrow \cat{Cat}_\infty$.
\end{lem}
\begin{proof}
	The crucial observation is the following claim:
	\begin{alphanumerate}\itshape
		\item[\boxtimes] The functor $\Hom_\Dd(F(-),-)\colon \Cc^\op\times\Dd\rightarrow \cat{An}$ is equivalent to the composition\label{claim:HomInUnstraightening}
		\begin{equation*}
			\Cc^\op\times\Dd\xrightarrow{i_0^\op\times i_1} \Uu^\op\times\Uu\xrightarrow{\Hom_\Uu}\cat{An}\,.
		\end{equation*}
		Here the first arrow is given by $i_0\colon \Cc\simeq \{0\}\times_{\Delta^1}\Uu\rightarrow \Uu$ and $i_1\colon \Dd\simeq \{1\}\times_{\Delta^1}\Uu\rightarrow \Uu$.
	\end{alphanumerate}
	To prove \cref{claim:HomInUnstraightening}, first observe that $i_0\colon \Cc\rightarrow \Uu$ and $i_1\colon \Dd\rightarrow\Uu$ are fully faithful. Indeed, $\Hom$ animae in pullbacks are given as pullbacks of $\Hom$ animae in the respective factors (which is straightforward to see from \cref{par:HomInQuasiCategories} and we'll see a more general assertion in \cref{lem:HomInLimits}). So pullbacks of fully faithful functors are still fully faithful and it remains to observe that $\{0\}\rightarrow \Delta^1$ and $\{1\}\rightarrow \Delta^1$ are both fully faithful, which is obvious. Now consider the following commutative square in $\cat{Cat}_\infty$:
	\begin{equation*}
		\begin{tikzcd}
			\Cc\eqar[r]\eqar[d]\drar[commutes] & \Cc\dar["F"]\\
			\Cc\rar["F"] & \Dd
		\end{tikzcd}
	\end{equation*}
	It can be viewed as a natural transformation $\const \Cc\Rightarrow F$ in $\Fun(\Delta^1,\cat{Cat}_\infty)$. After cocartesian unstraightening, it thus induces a morphism $\varphi\colon \Delta^1\times\Cc\rightarrow \Uu$ in $\cat{Cocart}(\Delta^1)$. Consider the composite
	\begin{equation*}
		(\Delta^1\times\Cc)^\op\times\Dd\xrightarrow{\varphi^\op\times i_1}\Uu^\op\times\Uu\xrightarrow{\Hom_\Uu}\cat{An}\,.
	\end{equation*}
	By unravelling the definitions and using that $i_1\colon \Dd\rightarrow \Uu$ is fully faithful, this composite can be regarded as a natural transformation $\eta\colon \Hom_\Uu(i_0(-),i_1(-))\Rightarrow \Hom_\Uu(i_1F(-),i_1(-))$ in $\Fun(\Cc^\op\times\Dd,\cat{An})$. We wish to show that $\eta$ is an equivalence of functors. By \cref{thm:EquivalencePointwise} this can be checked pointwise. So fix $x\in \Cc$, $y\in\Dd$. By unravelling the constructions $\eta_{(x,y)}\colon \Hom_\Uu(i_0(x),i_1(y))\rightarrow \Hom_\Uu(i_1F(x),i_1(y))$ is given by precomposition with the morphism $\varphi_x\colon i_0(x)\rightarrow i_1F(x)$ in $\Uu$. As we've seen in \cref{par:StraighteningMotivation}, $\varphi_x$ is a cocartesian morphism. Since $\Hom_{\Delta^1}(1,1)\simeq \Hom_{\Delta^1}(0,1)$, \cref{lem:CocartesianMorphisms} implies that precomposition with $\varphi_x$ must be an equivalence. Thus $\eta$ is an equivalence of functors, as desired. To finish the proof of \cref{claim:HomInUnstraightening}, it remains to observe $\Hom_\Uu(i_1F(-),-)\simeq \Hom_\Dd(F(-),-)$ as we've checked above that $i_1\colon \Dd\rightarrow \Uu$ is fully faithful.
	
	Now assume that $p\colon \Uu\rightarrow \Delta^1$ is a cartesian fibration too and let $G\colon \Dd\rightarrow \Cc$ correspond to $\operatorname{St}^{(\mathrm{cart})}(p)\colon (\Delta^1)^\op\rightarrow \cat{Cat}_\infty$. Then \cref{claim:HomInUnstraightening} and its dual provide an equivalences of functors $\Hom_\Dd(F(-),-)\simeq \Hom_\Uu(i_0(-),i_1(-))\simeq \Hom_\Cc(-,G(-))$, so $F$ and $G$ are adjoints. This proves \cref{enum:Bicartesian} $\Rightarrow$ \cref{enum:BicartesianAdjoint}.
	
	Conversely, suppose $G\colon \Dd\rightarrow \Cc$ is a right adjoint of $F$. Fix $y\in \Dd$. Then \cref{claim:HomInUnstraightening} and the fact that $i_0$ is fully faithful shows
	\begin{equation*}
		\Hom_\Uu\lef(i_0(-),i_0G(y)\righ)\simeq \Hom_\Cc\lef(-,G(y)\righ)\simeq \Hom_\Dd\lef(F(-),y\righ)\simeq \Hom_\Uu\lef(i_0(-),i_1(y)\righ)\,.
	\end{equation*}
	The image of $\id_{i_0G(y)}$ defines a morphism $\psi_y\colon i_0G(y)\rightarrow i_1(y)$ in $\Uu$. Furthermore, Yoneda's lemma (or more precisely, the dual of \cref{thm:Yoneda}) shows that any natural transformation $\Hom_\Cc(-,G(y))\simeq \Hom_\Uu(i_0(-),i_0G(y))\Rightarrow \Hom_\Uu(i_0(-),i_1(y))$ is uniquely determined by the image of $\id_{G(y)}$. That uniqueness ensures that the chain of equivalences above is must be given by postcomposition with $\psi_y$. Hence the dual of \cref{lem:CocartesianMorphisms} shows that $\psi_y$ is a $p$-cartesian morphism and we have constructed a sufficient supply of $p$-cartesian lifts. This finishes the proof of \cref{enum:BicartesianAdjoint} $\Rightarrow$ \cref{enum:Bicartesian}.
\end{proof}
\begin{cor}[\enquote{Extracting adjoints is functorial}]\label{cor:ExtractingAdjoints}
	Let $\Fun^\L,\Fun^\R\subseteq \Fun$ denote the full sub-$\infty$-categories spanned by the left/right adjoint functors and let $\cat{Cat}_\infty^\L,\cat{Cat}_\infty^\R\subseteq \cat{Cat}_\infty$ be the non-full sub-$\infty$-categories \embrace{in the sense of \cref{par:SubQuasiCategories}} spanned by all objects but only the left/right adjoint functors.
	\begin{alphanumerate}
		\item For all $\infty$-categories $\Cc$ and $\Dd$, sending a left adjoint functor $L\colon \Cc\rightarrow \Dd$ to its right adjoint $R\colon \Dd\rightarrow \Cc$ can be turned into an equivalence of $\infty$-categories $\Fun^\L\!\left(\Cc,\Dd\right)^\op\simeq \Fun^\R\!\left(\Dd,\Cc\right)$.\label{enum:FunLFunR}
		\item \!There exists an equivalence of $\infty$-categories $\cat{Cat}_\infty^\L\simeq (\cat{Cat}_\infty^\R)^\op$ which is the identity on objects and sends morphisms in $\cat{Cat}_\infty^\L$, that is, left adjoint functors $L\colon \Cc\rightarrow \Dd$, to their right adjoints $R\colon \Dd\rightarrow \Cc$.\label{enum:CatLCatR}
	\end{alphanumerate}
\end{cor}
\begin{proof}[Proof sketch]
	For the equivalence in \cref{enum:FunLFunR}, it suffices show that the essential images of $\Fun^\L(\Cc,\Dd)$ and $\Fun^\R(\Dd,\Cc)^\op$ under the fully faithful Yoneda embeddings
	\begin{gather*}
		\Fun^\R(\Dd,\Cc)\xrightarrow{(\Yo_\Cc)_*}\Fun\lef(\Dd,\Fun(\Cc^\op,\cat{An})\righ)\simeq \Fun(\Cc^\op\times\Dd,\cat{An})\,,\\
		\Fun^\L(\Cc,\Dd)^\op\simeq \Fun^\R(\Cc^\op,\Dd^\op)\xrightarrow{(\Yo_{\Dd^\op})_*}\Fun\lef(\Cc^\op,\Fun(\Dd,\cat{An})\righ)\simeq \Fun(\Cc^\op\times\Dd,\cat{An})
	\end{gather*}
	coincide. Using \cref{lem:Adjunction} and the definition of the Yoneda embedding, it's straightforward to check that both essential images consist of those functors $H\colon \Cc^\op\times\Dd\rightarrow \cat{An}$ such that for every $x\in \Cc$ there exists a $y\in \Dd$ such that $H(x,-)\simeq \Hom_\Dd(y,-)$ and for every $y'\in\Dd$ there exists an $x'\in \Cc$ such that $H(-,y')\simeq \Hom_\Cc(-,x')$. This proves that there exists an equivalence $\Fun^\L(\Cc,\Dd)^\op\simeq \Fun^\R(\Dd,\Cc)$ as desired. Furthermore if $L\colon \Cc\shortdoublelrmorphism \Dd\noloc R$ is an adjunction, then the Yoneda embeddings above send both $R$ and $L$ to the functor $\Hom_\Cc(-,R(-))\simeq \Hom_\Dd(L(-),-)\colon \Cc^\op\times\Dd\rightarrow \cat{An}$. So the equivalence we've constructed is really given by extracting adjoints.
	
	To prove \cref{enum:CatLCatR}, we grossly neglect set theory and regard both $\cat{Cat}_\infty^\L$ and $\cat{Cat}_\infty^\R$ as objects in $\cat{Cat}_\infty$. This can be repaired by considering universes or, with some care, by imposing cardinality bounds (similar to the argument in \cref{lem:StraighteningFunctorial} below, where we do this in detail). We'll show that there exists a functorial bijection $\pi_0\Hom_{\cat{Cat}_\infty}(\Cc,\cat{Cat}_\infty^\L)\cong \pi_0\Hom_{\cat{Cat}_\infty}(\Cc,(\cat{Cat}_\infty^\R)^\op)$ for all $\infty$-categories $\Cc$; if we can do this, then the Yoneda lemma in the ordinary category $\operatorname{ho}(\cat{Cat}_\infty)$ will show that $\cat{Cat}_\infty^\L$ and $\cat{Cat}_\infty^\R$ are isomorphic in the homotopy category, hence equivalent as $\infty$-categories. We know $\Hom_{\cat{Cat}_\infty}(\Cc,\cat{Cat}_\infty^\L)\simeq \core\Fun(\Cc,\cat{Cat}_\infty^\L)$ by \cref{thm:CordierPorter} and $\Fun(\Cc,\cat{Cat}_\infty^\L)\simeq \cat{Cocart}(\Cc)$ by \cref{thm:Straightening}\cref{enum:CocartesianStraightening}. Let $F\colon \Cc\rightarrow \cat{Cat}_\infty$ be a functor and $p\colon \Uu\rightarrow \Cc$ be its cocartesian unstraightening. By \cref{lem:AdjunctionBicartesian}, $F$ factors through $\cat{Cat}_\infty^\L\rightarrow \cat{Cat}_\infty$ if and only if for all $\alpha\colon\Delta^1\rightarrow \Cc$, the pullback $p_{\alpha}\colon \Delta^1\times_{\alpha,\Cc}\Uu\rightarrow \Delta^1$ is not only a cocartesian, but also a cartesian fibration. In other words, $p$ is a \emph{locally cartesian fibration} in the sense of \cref{def:LocallyCocartesian}. Since right adjoints compose, it's clear that locally $p$-cartesian morphisms are closed under composition, and so $p$ is automatically a cartesian fibration. In summary, we obtain a bijection 
	\begin{equation*}
		\pi_0\Hom_{\cat{Cat}_\infty}(\Cc,\cat{Cat}_\infty^\L)\cong \pi_0\core\cat{Bicart}(\Cc)\,,
	\end{equation*}
	where we define $\cat{Bicart}(\Cc)\subseteq \cat{Cat}_{\infty/\Cc}$ as the non-full sub-$\infty$-category spanned by the \emph{bicartesian fibrations}. That is, objects of $\cat{Bicart}(\Cc)$ are those $p\colon \Uu\rightarrow \Cc$ that are both cocartesian and cartesian fibrations, and morphisms are those functors in $\cat{Cat}_{\infty/\Cc}$ that preserve both $p$-cocartesian and $p$-cartesian morphisms. In the same way, we find bijections 
	\begin{equation*}
		\pi_0\Hom_{\Cat_\infty}\bigl(\Cc,(\cat{Cat}_\infty^\R)^\op\bigr)\cong \pi_0\Hom_{\Cat_\infty}\bigl(\Cc^\op,\cat{Cat}_\infty^\R\bigr)\cong \pi_0\core\cat{Bicart}(\Cc)\,.
	\end{equation*}
	Hence $\pi_0\Hom_{\cat{Cat}_\infty}(\Cc,\cat{Cat}_\infty^\L)\cong \pi_0\Hom_{\cat{Cat}_\infty}(\Cc,(\cat{Cat}_\infty^\R)^\op)$ and so $\cat{Cat}_\infty^\L\simeq (\cat{Cat}_\infty^\R)^\op$, as argued above. By unravelling the cases $\Cc\simeq *$ and $\Cc\simeq \Delta^1$ (the latter using \cref{lem:AdjunctionBicartesian}), we find that this adjunction is really the identity on objects and given by extracting adjoints on morphisms.
\end{proof}

\subsection{Limits and colimits}
\begin{defi}\label{def:Colimits}
	Let $\Ii$ and $\Cc$ be $\infty$-categories.
	\begin{alphanumerate}
		\item Let $F\colon \Ii\rightarrow \Cc$ be a functor of $\infty$-categories. A \emph{colimit of $F$}, denoted $\colimit F$ (or sometimes $\colimit_{i\in\Ii}F(i)$), is a left adjoint object of $F$ under  $\operatorname{const}\colon\Ii\rightarrow \Fun(\Ii,\Cc)$ that sends $i\in\Ii$ to the constant functor with value $i$. Dually, a \emph{limit of $F$}, denoted $\limit F$ (or sometimes $\limit_{i\in\Ii}F(i)$), is a right adjoint object of $F$ under $\operatorname{const}$.\label{enum:Colimit}
		\item We say that \emph{$\Cc$ has all $\Ii$-shaped colimits} or \emph{all $\Ii$-shaped limits} if all functors $\Ii\rightarrow \Cc$ admit colimits or limits, respectively.\label{enum:ColimitFunctor}
	\end{alphanumerate}
\end{defi}
\begin{rem}\label{rem:ColimitFunctor}
	 If $\Cc$ has all $\Ii$-shaped colimits, then \cref{lem:Adjunction} implies that forming colimits assembles into a functor $\colimit\colon \Fun(\Ii,\Cc)\rightarrow \Cc$. The same is true for limits.
\end{rem}
\begin{lem}\label{lem:AdjointsPreserveColimits}
	Left adjoint functors between $\infty$-categories preserve colimits and right adjoint functors preserve limits.
\end{lem}
\begin{proof}
	The proof of \cref{lem:1AdjointsPreserveColimits} can be copied verbatim.
\end{proof}
\begin{lem}[\enquote{Colimits in functor $\infty$-categories are computed pointwise.}]\label{lem:ColimitsInFunctorCategories}
	Let $\Cc$, $\Dd$, and $\Ii$ be $\infty$-categories such that $\Dd$ has all $\Ii$-shaped colimits. Then $\Fun(\Cc,\Dd)$ has again all $\Ii$-shaped colimits and the evaluation functor 
	\begin{equation*}
		\ev_x\colon \Fun(\Cc,\Dd)\longrightarrow \Fun\lef(\{x\},\Dd\righ)\simeq \Dd
	\end{equation*}
	preserves $\Ii$-shaped colimits for all $x\in \Cc$. A dual assertion holds for limits.
\end{lem}
\begin{proof}
	The proof of \cref{lem:1ColimitsInFunctorCategories} can be copied verbatim.
\end{proof}
Our next goal is to analyse limits and colimits in the $\infty$-categories $\cat{An}$ and $\cat{Cat}_\infty$. We start with a procedure for computing pullbacks and pushouts which is very useful in practice.
\begin{numpar}[Pushouts and pullbacks in $\cat{An}$ and $\cat{Cat}_\infty$.]\label{par:HomotopyPushout}
	Pushouts and pullbacks in $\cat{An}$ or $\cat{Cat}_\infty$ can be computed using the following recipe:
	\begin{alphanumerate}
		\item Write down the diagram on the level of Kan complexes or quasi-categories.\label{enum:PushoutStepA}
		\item For pushouts, use \cref{lem:SmallObjectArgument} to replace one leg by a cofibration. For pullbacks, use \cref{lem:SmallObjectArgument} to replace one leg by a Kan fibration/isofibration (depending on whether you take the pullback in $\cat{An}$ or $\cat{Cat}_\infty$, respectively).\label{enum:PushoutStepB}
		\item Take the pushout or pullback in $\cat{sSet}$.\label{enum:PushoutStepC}
		\item For pushouts, the result of \cref{enum:PushoutStepC} will usually not be a Kan complex/quasi-category, so we need to use \cref{lem:SmallObjectArgument} once again to replace it by a Kan complex/quasi-category. For pullbacks, this step is unnecessary.\label{enum:PushoutStepD}
	\end{alphanumerate}
	We've already seen the case of pullbacks in \enquote{Definition}~\cref{def:HomotopyPullback} and model category fact~\cref{par:HomotopyPullback}. The procedure above is a consequence of the general model category fact that a pushout of cofibrant objects in a model category is automatically a homotopy pushout too if at least one leg is a cofibration, and a pullback of fibrant objects is a homotopy pullback if at least one leg is a fibration. See \cite[Corollary~{\href{https://cisinski.app.uni-regensburg.de/CatLR.pdf\#thm.2.3.28}{2.3.28}}]{Cisinski} for a proof of the general fact and \cite[Theorem~\HTTthm{4.2.4.1}, Remark~\HTTthm{A.3.3.14}]{HTT} or \cite[Theorem~X.21]{HigherCatsII} for a proof that homotopy colimits/limits in a simplicial model category agree with colimits/limits in the underlying $\infty$-category.
	
	The procedure above implies that many pullback constructions we've seen so far with simplicial sets are also pullbacks in $\cat{An}$ or $\cat{Cat}_\infty$ and can thus be reinterpreted as model-independent constructions. For example, the diagram from \cref{par:HomInQuasiCategories} defining $\Cc_{x/}$ and $\Hom_\Cc(x,y)$ is also a pullback in $\cat{Cat}_\infty$, because $(s,t)\colon \Ar(\Cc)\rightarrow \Cc\times\Cc$ is an isofibration (see the proof of \cref{lem:HomRealityCheck}). As another example, if $p\colon\Uu\rightarrow \Cc$ is a cocartesian fibration, then the fibre $p^{-1}\{x\}$, which computes the value of the associated functor $\operatorname{St}^{(\mathrm{cocart})}\colon \Cc\rightarrow\cat{Cat}_\infty$ at $x$, can also be identified with the $\infty$-categorical pullback $\{x\}\times_\Cc\Uu$, because any cocartesian fibration $p$ is automatically an isofibration. Indeed, $p$ is an inner fibration by \cref{def:Cocartesian} and cocartesian lifts of equivalences are easily checked to be equivalences again (for example, using \cref{lem:CocartesianMorphisms}). We'll often use these facts without mention. Let us also mention, and later use without mention, that $\cat{An}\subseteq\cat{Cat}_\infty$ preserves both pushouts and pullbacks; in fact, it preserves all limits and colimits by \cref{exm:Adjunctions}\cref{enum:AnToCatInfty} and \cref{lem:AdjointsPreserveColimits}.\hfill$\blacksquare$
\end{numpar}
%\begin{numpar}[Pushouts in $\cat{An}$ and $\cat{Cat}_\infty$.]
%	It turns out that pushouts can be computed analogously: Let 
%	\begin{equation*}
%		\begin{tikzcd}
%			X\rar\dar\drar[pushout] & X'\dar\\
%			Y\rar & \ov{Y}'
%		\end{tikzcd}\quad\text{and}\quad
%		\begin{tikzcd}
%			\Cc\rar\dar\drar[pushout] & \Cc'\dar\\
%			\Dd\rar & \ov{\Dd}'
%		\end{tikzcd}
%	\end{equation*}
%	be pushouts in $\cat{sSet}$ such that $X$, $X'$, $Y$ are Kan complexes and $\Cc$, $\Cc'$, $\Dd$ are quasi-categories. Assume furthermore that at least one leg is a cofibration in either case. Choose an anodyne map $\ov{Y}'\rightarrow Y'$ and an inner anodyne map $\ov{\Dd}'\rightarrow \Dd'$. Then $Y'$ and $\Dd'$ are the pushouts in the $\infty$-categories $\cat{An}$ and $\cat{Cat}_\infty$, respectively. For a proof of the general model category fact behind this see \cite[Definition~{\href{https://cisinski.app.uni-regensburg.de/CatLR.pdf\#thm.2.3.27}{2.3.27}}]{Cisinski} for example.
%	
%	So to compute a pushout in $\cat{An}$ or $\cat{Cat}_\infty$, write it down on the level of simplicial sets, replace at least one leg by a cofibration (via \cref{lem:SmallObjectArgument}), take the pushout in $\cat{sSet}$, and finally replace the result by a Kan complex or quasi-category, respectively. Note that we can skip the final replacement step for pullbacks, since then the result is already a Kan complex or a quasi-category.\hfill$\blacksquare$
%\end{numpar}
%
But there's also a description of limits and colimits that works in full generality and doesn't rely on the simplicial model (in fact, I'd like to see a proof of model category fact~\cref{par:HomotopyPushout} using only \cref{lem:ColimitsInAnima} below; I'm not sure if this works, so I'll leave it to you to figure out). For our model-independent description, we need a construction first.
\begin{con}
	Let $\Ii$ be an $\infty$-category and $p\colon \Uu\rightarrow \Ii$ a cocartesian fibration. Let $\Fun_\Ii(\Ii,\Uu)\coloneqq \Fun(\Ii,\Uu)\times_{\Fun(\Ii,\Ii)}\{\id_\Ii\}$, the pullback being taken in $\cat{Cat}_\infty$ (but we could take it in $\cat{sSet}$ as well by model category fact~\cref{par:HomotopyPullback}) and let
	\begin{equation*}
		\Fun_\Ii^{(\mathrm{cocart})}\!\left(\Ii,\Uu\right)\subseteq \Fun_\Ii\!\left(\Ii,\Uu\right)
	\end{equation*}
	be the full sub-$\infty$-category spanned by those $\Ii\rightarrow \Uu$ such that all morphisms in $\Ii$ are sent to $p$-cocartesian morphisms.
	
	Also note that if $p$ is a left fibration, then
	\begin{equation*}
		\Fun_\Cc^{(\mathrm{cocart})}\!\left(\Ii,\Uu\right)\simeq \Fun_\Cc\!\left(\Ii,\Uu\right)\simeq \Hom_{\Cat_{\infty/\Cc}}\!\left(\Ii,\Uu\right)\,.
	\end{equation*}
	Indeed, the first equivalence is clear since in this case all morphisms in $\Uu$ are $p$-cocartesian by \cref{lem:CocartesianLeft}. The second equivalence follows from \cref{cor:HomInSliceCategories} combined with the facts that $\core\colon \cat{An}\rightarrow \cat{Cat}_\infty$ preserves pullbacks (because it is a right adjoint by \cref{exm:Adjunctions}\cref{enum:AnToCatInfty}) and that $\Fun_\Cc(\Ii,\Uu)$ is already an anima (by \cref{cor:FKanFibration} and \cref{cor:LeftFibrationsOverAnima}).
\end{con}
\begin{lem}\label{lem:ColimitsInAnima}
	Let $F\colon \Ii\rightarrow \Cat_\infty$ be a functor and let $p\colon \Uu\rightarrow \Ii$ be its cocartesian unstraightening. Then the colimit and the limit of $F$ in $\cat{Cat}_\infty$ are given by.
	\begin{align*}
		\colimit_{i\in\Ii}F(i)&\simeq \Uu\left[\{\text{cocartesian morphisms}\}^{-1}\right]\,,\\
		\limit_{i\in\Ii}F(i)&\simeq \Fun_\Ii^\mathrm{cocart}(\Ii,\Uu)\,.
	\end{align*}
	In particular, if $F$ takes values in $\cat{An}$, then the colimit and the limit of $F$ in $\cat{An}$ are given by
	\begin{equation*}
		\colimit_{i\in\Ii}F(i)\simeq \left|\Uu\right|\quad\text{and}\quad \limit_{i\in\Ii}F(i)\simeq \Hom_{\Cat_{\infty/\Ii}}(\Ii,\Uu)\,.
	\end{equation*}
\end{lem}
\begin{proof}
	We only proof the case of a colimit in $\cat{An}$ and then indicate the necessary changes for the other cases. We need to furnish an equivalence $\Hom_{\cat{An}}\!\left(\left|\Uu\right|,-\right)\simeq \Hom_{\Fun(\Ii,\cat{An})}\!\left(F,\const(-)\right)$. We have $\Hom_{\cat{An}}\!\left(\left|\Uu\right|,-\right)\simeq \Hom_{\cat{Cat}_\infty}(\Uu,-)$ because $\left|\,\cdot\,\right|\colon \cat{Cat}_\infty\rightarrow\cat{An}$ is left adjoint to the inclusion $\cat{An}\subseteq\cat{Cat}_\infty$. Furthermore, $\Fun(\Ii,\cat{An})\simeq\cat{Left}(\Ii)$ by the straightening equivalence (\cref{thm:Straightening}\cref{enum:LeftStraightening}), hence
	\begin{equation*}
		\Hom_{\Fun(\Ii,\cat{An})}\lef(F,\const(-)\righ)\simeq \Hom_{\cat{Left}(\Ii)}\lef(\Uu,\operatorname{Un}^{(\mathrm{left})}\lef(\const(-)\righ)\righ)\,.
	\end{equation*}
	The unstraightening of $\const X\colon \Ii\rightarrow\cat{An}$ is the projection $X\times \Ii\rightarrow\Ii$ for all $X\in\cat{An}$. So we can compute
	\begin{align*}
		\Hom_{\cat{Left}(\Ii)}\!\left(\Uu,X\times\Ii\right)&\simeq \Hom_{\cat{Cat}_{\infty/\Ii}}\!\left(\Uu,X\times\Ii\right)\\
		&\simeq \Hom_{\Cat_\infty}\!\left(\Uu,X\times\Ii\right)\times_{\Hom_{\Cat_\infty}\!\left(\Uu,\Ii\right)}\{p\}\\
		&\simeq \bigl(\Hom_{\cat{Cat}_\infty}(\Uu,X)\times\Hom_{\cat{Cat}_\infty}(\Uu,\Ii)\bigr)\times_{\Hom_{\Cat_\infty}\!\left(\Uu,\Ii\right)}\{p\}\\
		&\simeq \Hom_{\cat{Cat}_\infty}(\Uu,X)\,.
	\end{align*}
	In the first step we use that $\cat{Left}(\Ii)\rightarrow \cat{Cat}_{\infty/\Ii}$ is fully faithful. In the second step we use \cref{cor:HomInSliceCategories}. In the third step, we use $\Hom_{\Cat_\infty}\!\left(\Uu,X\times\Ii\right)\simeq \Hom_{\cat{Cat}_\infty}(\Uu,X)\times\Hom_{\cat{Cat}_\infty}(\Uu,\Ii)$, which follows from \cref{thm:CordierPorter} (we do not yet know that $\Hom_{\Cat_\infty}(\Uu,-)$ commutes with limits). Finally, in the fourth step we use that $\Hom_{\cat{Cat}_\infty}(\Uu,\Ii)\times_{\Hom_{\Cat_\infty}\!\left(\Uu,\Ii\right)}\{p\}\simeq \{p\}$ is just a point. Since every step is functorial in $X$, this finishes the proof of $\colimit_{i\in\Ii}F(i)\simeq \left|\Uu\right|$.
	
	The limit case is analogous. When taking a colimit in $\cat{Cat}_\infty$, we can no longer argue that $\cat{Cocart}(\Ii)\rightarrow \cat{Cat}_{\infty/\Ii}$ is fully faithful. Instead, $\Hom_{\cat{Cocart}(\Ii)}\!\left(\Uu,\Cc\times\Ii\right)\subseteq \Hom_{\cat{Cat}_{\infty/\Ii}}\!\left(\Uu,\Cc\times\Ii\right)$ is a collection of path components by \cref{lem:NonFullSubcategory} and we have to check that it agrees with $\Hom_{\cat{Cat}_\infty}\!\left(\Uu[\{\text{cocartesian morphisms}\}^{-1}],\Cc\right)\subseteq \Hom_{\Cat_\infty}\!\left(\Uu,\Cc\right)$, which is also a collection of path components by \cref{lem:Localisation}. This is straightforward.
\end{proof}
\begin{cor}\label{cor:HomPreservesColimits}
	Let $F\colon \Ii\morphism\Cc$ be a functor of $\infty$-categories. A natural transformation $c_F\colon \const y\Rightarrow F$ exhibits $y\in \Cc$ as a limit of $F$ if and only if the natural map
	\begin{equation*}
		c_F^*\colon \Hom_\Cc\!\left(x,y\right)\overset{\simeq}{\longrightarrow}\limit_{i\in\Ii}\Hom_\Cc\lef(x,F(i)\righ)
	\end{equation*}
	is an equivalence for all $x\in \Cc$. A dual assertion holds for colimits.
\end{cor}
\begin{proof}
	The unstraightening of $\Hom_\Cc(x,F(-))\colon \Ii\rightarrow\cat{An}$ is the left fibration $F^*(\Cc_{x/})\rightarrow \Ii$, the pullback of the slice-$\infty$-category projection $t\colon \Cc_{x/}\rightarrow \Cc$ along $F\colon \Ii\rightarrow\Cc$. Hence, according to \cref{lem:ColimitsInAnima}, the limit on the right-hand side is given by
	\begin{align*}
		\Hom_{\Cat_{\infty/\Ii}}\bigl(\Ii,F^*(\Cc_{x/})\bigr)&\simeq \core\F\bigl(\Ii,F^*(\Cc_{x/})\bigr)\times_{t,\core\F(\Ii,\Ii)}\{\id_\Ii\}\\%\Hom_{\Cat_\infty{}_{/\Cc}}\bigl(\Ii,\Cc_{x/}\bigr)\\
		%&\simeq \core\F\bigl(\Ii,\Cc_{x/}\bigr)\times_{t,\core \F(\Ii,\Cc)}\{F\}\\
		&\simeq \core\F\bigl(\Ii,\{x\}\times_{\Cc,s}\F(\Delta^1,\Cc)\times_{t,\Cc,F}\Ii\bigr)\times_{t,\core \F(\Ii,\Ii)}\{\id_\Ii\}\\
		&\simeq \{\const x\}\times_{\core \F(\Ii,\Cc),s}\core \F\bigl(\Ii,\F(\Delta^1,\Cc)\bigr)\times_{t,\core \F(\Ii,\Cc)}\{F\}\\
		&\simeq \Hom_{\F(\Ii,\Cc)}\!\left(\const x,F\right)\,.
	\end{align*}
	In the first step we plug in \cref{cor:HomInSliceCategories} and \cref{thm:CordierPorter} to write $\Hom_{\Cat_{\infty/\Ii}}$ as a homotopy pullback, which is an ordinary pullback in $\cat{sSet}$ by model category fact~\cref{par:HomotopyPullback}\cref{enum:HomotopyPullbackOfKanComplexes} and the fact that $\core\F(\Ii,F^*(\Cc_{x/}))\rightarrow\core\F(\Ii,\Ii)$ is a Kan fibration. For the latter, we use \cref{cor:FKanFibration} and \cref{cor:HomAnima} to see that $\F(\Ii,F^*(\Cc_{x/}))\rightarrow\F(\Ii,\Ii)$ is a left fibration and then \cref{cor:LeftFibrationsOverAnima} to show that $\core$ transforms it into a Kan fibration. In the second step, we plug in the formula $F^*(\Cc_{x/})\cong\{x\}\times_{\Cc,s}\F(\Delta^1,\Cc)\times_{t,\Cc,F}\Ii$. In the third step, we use the universal property of pullbacks in $\cat{sSet}$. We also use that $\core$ preserves pullbacks if one leg is an isofibration (and left fibrations are isofibrations because $\{0\}\rightarrow N(J)$ is left anodyne). Finally, in the fourth step, we use this argument again together with the \enquote{currying} isomorphism $\F(\Ii,\F(\Delta^1,\Cc))\cong \F(\Delta^1,\F(\Ii,\Cc))$ and the definition of $\Hom_{\F(\Ii,\Cc)}(\const x,F)$.
	
	% used \cref{thm:CordierPorter} together with \cref{cor:HomInSliceCategories} and  In the second step, we plug in \cref{cor:HomInSliceCategories} to write $\Hom_{\Cat_{\infty/\Cc}}$ as a homotopy pullback, which is an ordinary pullback in $\cat{sSet}$ by model category fact~\cref{par:HomotopyPullback}\cref{enum:HomotopyPullbackOfKanComplexes} and the fact that $\core\F(\Ii,\Cc_{x/})\rightarrow\core\F(\Ii,\Cc)$ is a Kan fibration. For the latter, we use \cref{cor:FKanFibration} and \cref{cor:HomAnima} to see that $\F(\Ii,\Cc_{x/})\rightarrow\F(\Ii,\Cc)$ is a left fibration and then \cref{cor:LeftFibrationsOverAnima} to show that $\core$ transforms it into a Kan fibration. In the third step, we plug in the definition $\Cc_{x/}\coloneqq \{x\}\times_{\Cc,s}\F(\Delta^1,\Cc)$ of the slice-$\infty$-category. In the fourth step, we use the universal property of the pullback $\{x\}\times_{\Cc,s}\F(\Delta^1,\Cc)$ in $\cat{sSet}$. Finally, in the fifth step, we use the \enquote{currying} isomorphism $\F(\Ii,\F(\Delta^1,\Cc))\cong \F(\Delta^1,\F(\Ii,\Cc))$.
	
	Therefore, at least pointwise, $c_F^*$ takes the form $c_F^*\colon \Hom_\Cc\!\left(x,y\right)\rightarrow \Hom_{\Fun(\Ii,\Cc)}\!\left(\const x,F\right)$ for all $x\in\Cc$. Since a natural transformation is an equivalence if and only if it is a pointwise equivalence (\cref{thm:EquivalencePointwise}), we are done.
\end{proof}
\begin{cor}\label{cor:HomPreservesLimits}
	For every $\infty$-category $\Cc$, the functors $\Hom_\Cc(x,-)\colon \Cc\rightarrow\cat{An}$ and $\Hom_\Cc\!\left(-,y\right)\colon \Cc^\op\morphism\cat{An}$ preserve limits for all $x,y\in\Cc$ \embrace{note that limits in $\Cc^\op$ correspond to colimits in $\Cc$}. Likewise, the Yoneda embedding $\Yo_\Cc\colon \Cc\morphism\Fun\!\left(\Cc^\op,\cat{An}\right)$ preserves limits.
\end{cor}
\begin{proof}
	The first two assertions follow immediately from \cref{cor:HomPreservesColimits}. The last one follows from the first plus the fact that limits and equivalences in functor categories are pointwise by \cref{lem:ColimitsInFunctorCategories} and \cref{thm:EquivalencePointwise}.
\end{proof}

\subsection{Cofinality}
Our next goal is to develop a theory of cofinality for limits and colimits in $\infty$-categories. This is summarised by the following theorem due to Joyal, with a first written proof appearing in \cite[Theorem~\HTTthm{4.1.3.1}]{HTT}.
\begin{thm}[Joyal's version of Quillen's theorem A]\label{thm:JoyalsQuillenA}
	For a functor $\alpha\colon\Ii\rightarrow \Jj$ of $\infty$-categories, the following are equivalent:
	\begin{alphanumerate}
		\item For every $\infty$-category $\Cc$ and every $F\colon \Jj\rightarrow \Cc$, the functor $F$ has a colimit if and only if $F\circ \alpha$ has a colimit. Furthermore, in this case the natural map is an equivalence\label{enum:Cofinal}
		\begin{equation*}
			\colimit_{i\in\Ii}F\lef(\alpha(i)\righ)\overset{\simeq}{\longrightarrow}\colimit_{j\in\Jj}F(j)\,.
		\end{equation*}
		\item For every right fibration $f\colon X\rightarrow\Jj$, the natural map 
		is an equivalence\label{enum:RightAnodyne}
		\begin{equation*}
			\Hom_{\Cat_{\infty/\Jj}}\!\left(\Ii,X\right)\overset{\simeq}{\longrightarrow} \Hom_{\Cat_{\infty/\Jj}}\!\left(\Jj,X\right)\,.
		\end{equation*}
		\item For every $j\in\Jj$, the slice-$\infty$-category $\Ii_{j/}\coloneqq \Ii\times_{\Jj}\Jj_{j/}$ is weakly contractible. That is, we have $\mathopen|\Ii_{j/}\mathclose|\simeq *$.\label{enum:WeaklyContractible}
	\end{alphanumerate}
	A dual assertion holds for limits, left fibrations, and the slice-$\infty$-categories $\Ii_{/j}$.
\end{thm}
\begin{defi}
	If $\alpha\colon \Ii\rightarrow\Jj$ satisfies the equivalent conditions from \cref{thm:JoyalsQuillenA}, then $\alpha$ is called \emph{cofinal}. Dually, $\alpha$ is called \emph{final} if it satisfies the dual equivalent conditions for limits.
\end{defi}
\begin{exm}\label{exm:Cofinal}
	The following are examples of cofinal functors:
	\begin{alphanumerate}
		\item Right anodyne maps are cofinal. It's clear from \cref{cor:FKanFibration} and \cref{cor:HomInSliceCategories} that the condition from \cref{thm:JoyalsQuillenA}\cref{enum:RightAnodyne} is satisfied.\label{enum:RightAnodyneCofinal}
		\item Right-adjoint functors $\alpha\colon \Ii\rightarrow\Jj$ are cofinal. Indeed, if $\beta$ is a left adjoint, then $\alpha^*\colon \Fun(\Jj,\Cc)\shortdoublelrmorphism \Fun(\Ii,\Cc)\noloc \beta^*$ is an adjunction by \cref{cor:FunctorCategoryAdjunctions} and so to verify the condition ${\colimit_\Ii}\circ{\alpha^*}\simeq \colimit_\Jj$ from \cref{thm:JoyalsQuillenA}\cref{enum:Cofinal}, it's enough to check ${\beta^*}\circ{\const}\simeq\const$, which is clear.\label{enum:RightAdjointCofinal}
		\item Localisations $p\colon \Ii\rightarrow \Ii[W^{-1}]$ are cofinal. One way to see this is that localisations are right anodyne, since by construction, $p$ factors into $\Ii\rightarrow\ov{\Ii}\rightarrow \Ii[W^{-1}]$, where the second arrow is inner anodyne and the first arrow is right anodyne, because $\Delta^1\rightarrow J$ is right anodyne. Then \cref{enum:RightAnodyneCofinal} does it.\label{enum:LocalisationsCofinal}
		
		But of course there's also a synthetic way to see this. Since the precomposition functor $p^*\colon\Fun(\Ii[W^{-1}],\Cc)\rightarrow \Fun(\Ii,\Cc)$ is fully faithful by \cref{lem:Localisation}, we have
		\begin{equation*}
			\Hom_{\Fun(\Ii,\Cc)}\!\left(F\circ p,\const y\right)\simeq \Hom_{\Fun(\Ii[W^{-1}],\Cc)}\!\left(F,\const y\right)\,,
		\end{equation*}
		functorially in $F\colon \Ii[W^{-1}]\rightarrow \Cc$ and all $y\in \Cc$, which proves that the condition from \cref{thm:JoyalsQuillenA}\cref{enum:Cofinal} is satified.
	\end{alphanumerate}
\end{exm}
\begin{proof}[Proof of \cref{thm:JoyalsQuillenA}, \cref{enum:Cofinal} $\Leftrightarrow$ \cref{enum:RightAnodyne}]
	Let $F\colon \Jj\rightarrow \cat{An}$ be a functor with unstraightening $\Uu\rightarrow \Jj$. Then the pullback $\alpha^*(\Uu)\rightarrow \Ii$ is the unstraightening of $F\circ\alpha\colon \Ii\rightarrow\cat{An}$. \cref{lem:ColimitsInAnima} shows $\lim_{j\in\Jj}F(j)\simeq\Hom_{\Cat_{\infty/\Jj}}(\Jj,\Uu)$. Similarly,
	\begin{equation*}
		\lim_{i\in\Ii}F\lef(\alpha(i)\righ)\simeq\Hom_{\Cat_{\infty/\Ii}}\lef(\Ii,\alpha^*(\Uu)\righ)\simeq \Hom_{\Cat_{\infty/\Ii}}\!\left(\Ii,\Uu\right)\,,
	\end{equation*}
	where we also used \cref{cor:HomInSliceCategories} and \cref{cor:HomPreservesColimits} in the second step. This shows that \cref{enum:RightAnodyne} holds if and only if \cref{enum:Cofinal} holds for functors $F\colon \Jj\rightarrow \cat{An}$. Now let $F\colon \Jj\rightarrow \Cc$ be an arbitrary functor. By \cref{cor:HomPreservesColimits}, $\Yo_\Cc\colon \Cc\rightarrow \Fun(\Cc^\op,\cat{An})$ preserves limits and it is fully faithful, so \cref{enum:Cofinal} holds for $F\colon \Jj\rightarrow \Cc$ if and only if it holds for $\Yo_\Cc\circ F\colon \Jj\rightarrow \Fun(\Cc^\op,\cat{An})$. Finally, limits in $\Fun(\Cc^\op,\cat{An})$ are computed pointwise by \cref{lem:ColimitsInFunctorCategories} and equivalences can be checked pointwise by \cref{thm:EquivalencePointwise}, so \cref{enum:Cofinal} holds for functors into $\Fun(\Cc^\op,\cat{An})$ if and only if it holds for functors into $\cat{An}$. This finishes the proof of \cref{enum:Cofinal} $\Leftrightarrow$ \cref{enum:RightAnodyne}.
\end{proof}
Before we can prove \cref{enum:Cofinal}  $\Rightarrow$ \cref{enum:WeaklyContractible} $\Rightarrow$ \cref{enum:RightAnodyne}, we need another lemma.
\begin{lem}\label{lem:CartesianCofinal}
	A cartesian fibration $p\colon \Uu\rightarrow \Jj$ satisfies the conclusion of \cref{thm:JoyalsQuillenA}\cref{enum:RightAnodyne} if and only if the fibres $p^{-1}\{j\}$ of $p$ are weakly contractible, that is, $\left|p^{-1}\{j\}\right|\simeq *$ for all $j\in \Jj$.
\end{lem}
\begin{proof}
	Let $E\colon \Jj^\op\rightarrow\cat{Cat}_\infty$ be the straightening of $p\colon \Uu\rightarrow \Jj$ and let $f\colon X\rightarrow \Jj$ be a right fibration with straightening $F\colon \Jj^\op\rightarrow \cat{An}$. Then the cartesian straightening equivalence (the dual of \cref{thm:Straightening}\cref{enum:CocartesianStraightening}) shows
	\begin{equation*}
		\Hom_{\Cat_{\infty/\Jj}}\!\left(\Uu,X\right)\simeq \Hom_{\cat{Cart}(\Jj)}\!\left(\Uu,X\right)\simeq \Hom_{\Fun(\Jj^\op,\cat{Cat}_\infty)}\!\left(E,F\right)\,.
	\end{equation*}
	Note that the first equivalence holds even though $\cat{Cart}(\Jj)\rightarrow\cat{Cat}_{\infty/\Jj}$ is not fully faithful, since we're mapping into a right fibration where every morphism is cartesian (by the dual of \cref{lem:CocartesianLeft}). Now $\left|\,\cdot\,\right|\colon \cat{Cat}_\infty\rightarrow \cat{An}$ is left adjoint to the inclusion $\cat{An}\subseteq \cat{Cat}_\infty$ by \cref{exm:Adjunctions}\cref{enum:AnToCatInfty} and that adjunction persists to functor-$\infty$-categories by \cref{cor:FunctorCategoryAdjunctions}. Thus
	\begin{equation*}
		\Hom_{\Fun(\Jj^\op,\cat{Cat}_\infty)}\!\left(E,F\right)\simeq \Hom_{\Fun(\Jj^\op,\cat{Cat}_\infty)}\lef(\left|E\right|,F\righ)\,.
	\end{equation*}
	The cartesian straightening of $\id_\Jj\colon \Jj\rightarrow \Jj$ is $\const *\colon \Jj^\op\rightarrow\cat{An}$. By the same arguments as above we then obtain
	\begin{equation*}
		\Hom_{\Cat_{\infty/\Jj}}\!\left(\Jj,X\right)\simeq \Hom_{\Fun(\Jj^\op,\cat{Cat}_\infty)}\!\left(\const *,F\right)\,.
	\end{equation*}
	Putting everything together, we see that the condition from \cref{thm:JoyalsQuillenA}\cref{enum:RightAnodyne} is satisfied if and only if $p\colon\Uu\rightarrow \Jj$ induces an equivalence $\left|E\right|\Rightarrow \const *$ in $\Fun(\Jj^\op,\cat{An})$. Since equivalences can be checked pointwise (\cref{thm:EquivalencePointwise}), this becomes precisely the condition that all fibres of $p$ are weakly contractible.
\end{proof}
Note that \cref{lem:CartesianCofinal} and \cref{thm:JoyalsQuillenA}\cref{enum:WeaklyContractible} look very much alike, but are a priori two different criteria for a cartesian fibration $p\colon \Uu\rightarrow\Jj$ to be cofinal. As a reality check, let's see that they are indeed equivalent. This isn't necessary to complete our proof of \cref{thm:JoyalsQuillenA}, but we'll need it later.
\begin{lem}\label{lem:CartesianFibres}
	Let $p\colon \Uu\rightarrow \Jj$ be a cartesian fibration. Then for every $j\in\Jj$, the natural functor $p^{-1}\{j\}\rightarrow \Uu\times_\Jj\Jj_{j/}$ admits a right adjoint. In particular, we obtain a homotopy equivalence of animae $\mathopen|p^{-1}\{j\}\mathclose|\simeq \mathopen|\Uu\times_\Jj\Jj_{j/}\mathclose|$.
\end{lem}
\begin{proof}[Proof sketch]
	By \cref{lem:Adjunction}, right adjoints can be constructed pointwise. This can be done as follows: Fix an object $(u,\ov\varphi)\in \Uu\times_\Jj\Jj_{j/}$, given by an element $u\in \Uu$ and a morphism $\ov\varphi\colon j\rightarrow p(u)$ in $\Jj$. Let $\varphi\colon u'\rightarrow u$ be a $p$-cartesian lift of $\ov\varphi$. Then $u'\in p^{-1}\{j\}$ is a right adjoint object to $(u,\ov\varphi)$ under $p^{-1}\{j\}\rightarrow \Uu\times_\Jj\Jj_{j/}$. To see this, note that, by construction, we have a morphism $c\colon (u',\id_j)\rightarrow (u,\ov\varphi)$ in $\Uu\times_\Jj\Jj_{j/}$ (which will play the role of the counit); using \cref{thm:EquivalencePointwise}, we have to show that
	\begin{equation*}
		\Hom_{p^{-1}\{j\}}\!\left(u'',u'\right)\longrightarrow \Hom_{\Uu\times_\Jj\Jj_{j/}}\lef((u'',\id_j),(u',\id_j)\righ)\overset{c_*}{\longrightarrow}\Hom_{\Uu\times_\Jj\Jj_{j/}}\lef((u'',\id_j),(u,\ov\varphi)\righ)
	\end{equation*}
	is an equivalence for all $u''\in p^{-1}\{j\}$. Now use the characterisation of cartesian morphisms from the dual of \cref{lem:CocartesianMorphisms} together with \cref{cor:HomInSliceCategories} and the fact that Hom animae in pullbacks of $\infty$-categories are pullbacks of the respective Hom animae (which is straightforward to see; we'll prove a more general statement in \cref{lem:HomInLimits}) to show that both sides are equivalent to $\Hom_\Uu(u'',u)\times_{\Hom_\Jj(j,p(u))}\{\ov\varphi\}$ and that the morphism between them is equivalent to the identity. We'll leave the details to you.
\end{proof}

\begin{proof}[Proof of \cref{thm:JoyalsQuillenA}, \cref{enum:Cofinal}  $\Rightarrow$ \cref{enum:WeaklyContractible} $\Rightarrow$ \cref{enum:RightAnodyne}]
	Assume \cref{enum:Cofinal} holds true and consider the functor $\Hom_\Jj(j_0,-)\colon \Jj\rightarrow \cat{An}$ for some $j_0\in \Jj$. Its unstraightening is the slice-$\infty$-category projection $t\colon \Jj_{j_0/}\rightarrow\Jj$, hence $\colimit_{j\in\Jj} \Hom_\Jj(j_0,j)\simeq \mathopen|\Jj_{j_0/}\mathclose|$ by \cref{lem:ColimitsInAnima}. But $\Jj_{j_0/}$ has an initial element given by $\id_{j_0}$, and so $\{\id_{j_0}\}\shortdoublelrmorphism \Jj_{j_0/}$ is an adjunction. Since adjunctions induce homotopy equivalences after $\left|\,\cdot\,\right|$, we conclude $\mathopen|\Jj_{j_0/}\mathclose|\simeq *$.
	
	Now consider $\Hom_\Jj(j_0,\alpha(-))\colon \Ii\rightarrow \cat{An}$. Its unstraightening is $\Ii_{j_0/}\simeq \Ii\times_\Jj \Jj_{j_0/}$ (here we use that precomposition with $\alpha$ corresponds to pullback along $\alpha$ under the unstraightening equivalence, see \cref{thm:Straightening}). Combining \cref{lem:ColimitsInAnima} with condition \cref{enum:Cofinal}, we obtain
	\begin{equation*}
		\bigl|\Ii_{j_0/}\bigr|\simeq\colimit_{i\in\Ii}\Hom_\Jj\lef(j_0,\alpha(i)\righ)\simeq \colimit_{j\in\Jj}\Hom_\Jj\!\left(j_0,j\right)\simeq *\,,
	\end{equation*}
	as claimed. This finishes the proof of the implication \cref{enum:Cofinal}  $\Rightarrow$ \cref{enum:WeaklyContractible}.
	
	Now assume \cref{enum:WeaklyContractible}. We can factor $\alpha\colon \Ii\rightarrow \Jj$ into $\Ii\rightarrow \Ii\times_{\Jj,s}\Ar(\Jj)\rightarrow \Jj$, where the first functor sends $i\in \Ii$ to the pair $(i,\id_{\alpha(i)}\colon \alpha(i)\rightarrow \alpha(i))$ and the second functor is induced by the target projection $t\colon \Ar(\Jj)\rightarrow \Jj$. It's straightforward to verify that $\Ii\rightarrow \Ii\times_{\Jj,s}\Ar(\Jj)$ is right adjoint to the projection $s\colon \Ii\times_{\Jj,s}\Ar(\Jj)\rightarrow \Ii$ (for example, one could use \cref{lem:HomInArrowCategories}; alternatively, unit and counit as well as the triangle identities are easily constructed by hand). By \cref{exm:Cofinal}\cref{enum:RightAdjointCofinal}, we see that $\Ii\rightarrow \Ii\times_{\Jj,s}\Ar(\Jj)$ satisfies \cref{enum:Cofinal}, hence also \cref{enum:RightAnodyne}. Furthermore, a slight generalisation of \cref{exm:Straightening}\cref{enum:ArCocartesianFibration} (which can be proved by the same argument) shows that $t\colon\Ii\times_{\Jj,s}\Ar(\Jj)\rightarrow \Jj$ is a cocartesian fibration. Its fibres $t^{-1}\{j\}\simeq \Ii\times_\Jj \Jj_{j/}$ are weakly contractible. Hence $t\colon\Ii\times_{\Jj,s}\Ar(\Jj)\rightarrow \Jj$ satisfies \cref{enum:RightAnodyne} by \cref{lem:CartesianCofinal}. We conclude that $\alpha\colon \Ii\rightarrow \Jj$ must satisfy \cref{enum:RightAnodyne} as well. Indeed, if $f\colon X\rightarrow \Jj$ is a right fibration, then
	\begin{align*}
		\Hom_{\Cat_{\infty/\Jj}}\!\left(\Jj,X\right)&\simeq \Hom_{\Cat_{\infty/\Jj}}\lef(\Ii\times_{\Jj,s}\Ar(\Jj),X\righ)\\
		&\simeq  \Hom_{\Cat_{\infty/\Ii\times_{\Jj,s}\Ar(\Jj)}}\lef(\Ii\times_{\Jj,s}\Ar(\Jj),t^*(X)\righ)\,.
	\end{align*}
	In the first equivalence we use \cref{enum:RightAnodyne}, in the second equivalence we let $t^*(X)\rightarrow \Ii\times_{\Jj,s}\Ar(\Jj)$ is the pullback of $f$ along $t$ and use \cref{lem:KanExtensionForRight}\cref{enum:RightPullbackLeftAdjoint} below. Now a pullback of a right fibration is again a right fibration, whence
	\begin{align*}
		\Hom_{\Cat_{\infty/\Ii\times_{\Jj,s}\Ar(\Jj)}}\lef(\Ii\times_{\Jj,s}\Ar(\Jj),t^*(X)\righ)&\simeq \Hom_{\Cat_{\infty/\Ii\times_{\Jj,s}\Ar(\Jj)}}\lef(\Ii,t^*(X)\righ)\\
		&\simeq \Hom_{\Cat_{\infty/\Jj}}\!\left(\Ii,X\right)\,.
	\end{align*}
	In the first equivalence we use \cref{enum:RightAnodyne} and in the second we use \cref{lem:KanExtensionForRight}\cref{enum:RightPullbackLeftAdjoint} below again. This finishes the proof of the implication \cref{enum:WeaklyContractible} $\Rightarrow$ \cref{enum:RightAnodyne}.
\end{proof}
\subsection{Kan extensions}\label{subsec:KanExtensions}
We're now working towards an $\infty$-categorical analogue of \cref{thm:1PShFreeCocompletion}. %For an $\infty$-category $\Cc$, we let $\PSh(\Cc)\coloneqq\Fun(\Cc^\op,\cat{An})$ denote the \emph{$\infty$-category of presheaves on $\Cc$} (note that this is \emph{not} compatible with the previous definition if $\Cc$ is an ordinary category).
Our first goal is to construct left Kan extensions for presheaf categories. As it turns out, this is most easily done in the fibration picture.
\begin{lem}\label{lem:KanExtensionForRight}
	Let $F\colon \Cc\rightarrow \Dd$ be a functor of $\infty$-categories.
	\begin{alphanumerate}
		\item \!The pullback functor $F^*\colon \cat{Cat}_{\infty/\Dd}\rightarrow\cat{Cat}_{\infty/\Cc}$ has a left adjoint, namely the forgetful functor $\cat{Cat}_{\infty/\Cc}\rightarrow\cat{Cat}_{\infty/\Dd}$ that sends $f\colon \Cc'\rightarrow \Cc$ to $F\circ f\colon \Cc'\rightarrow \Dd$.\label{enum:ForgetfulFunctor}
		\item \!The inclusion $\cat{Right}(\Dd)\subseteq\cat{Cat}_{\infty/\Dd}$ has a left adjoint that sends $g\colon \Dd'\rightarrow \Dd$ to $q\colon Y\rightarrow \Dd$, where
		\begin{equation*}
			\Dd'\longrightarrow Y\overset{q}{\longrightarrow}\Dd
		\end{equation*}
		is any factorisation of $g$ into a cofinal functor followed by a right fibration.\label{enum:RightCofinalLeftAdjoint}
		\item \!The functor $F^*\colon \cat{Right}(\Dd)\rightarrow \cat{Right}(\Cc)$ has a left adjoint $F_!\colon \cat{Right}(\Cc)\rightarrow \cat{Right}(\Dd)$. On objects, $F_!$ is given as follows: Let $p\colon X\rightarrow \Cc$ be a right fibration and let\label{enum:RightPullbackLeftAdjoint}
		\begin{equation*}
			X\longrightarrow Y\overset{q}{\longrightarrow} \Dd
		\end{equation*}
		be any factorisation of $F\circ p$ into a cofinal functor followed by a right fibration. Then we have $F_!(p\colon X\rightarrow \Cc)\simeq (q\colon Y\rightarrow \Dd)$. In particular, all such factorisations are equivalent.
	\end{alphanumerate}
\end{lem}
\begin{proof}
	For \cref{enum:ForgetfulFunctor}, note that left adjoints can be constructed pointwise by \cref{lem:Adjunction}, so its enough to show that $F\circ f\colon \Cc'\rightarrow \Cc$ is a left adjoint object to $f\colon \Cc'\rightarrow\Cc$ under $F^*$. To this end, let $g\colon \Dd'\rightarrow \Dd$ be an element in $\cat{Cat}_{\infty/\Dd}$. We have a diagram
	\begin{equation*}
		\begin{tikzcd}
			\Hom_{\Cat_{\infty/\Cc}}\left(\Cc',F^*(\Dd')\righ)\dar\rar\drar[pullback]&  \Hom_{\Cat_\infty}\lef(\Cc',F^*(\Dd')\righ)\rar\dar\drar[pullback] & \Hom_{\Cat_\infty}\!\left(\Cc',\Dd'\right)\dar\\
			\{f\}\rar & \Hom_{\Cat_\infty}\!\left(\Cc',\Cc\right)\rar["F_*"] & \Hom_{\Cat_\infty}\!\left(\Cc',\Dd\right)
		\end{tikzcd}
	\end{equation*}
	in which the left square is a pullback by \cref{cor:HomInSliceCategories} and the right square is a pullback by \cref{cor:HomPreservesColimits}. Hence the outer rectangle is a pullback as well. By \cref{cor:HomInSliceCategories}, we obtain
	\begin{equation*}
		\Hom_{\Cat_{\infty/\Cc}}\left(\Cc',F^*(\Dd')\righ)\simeq\Hom_{\Cat_{\infty/\Dd}}\!\left(\Cc',\Dd'\right)\,.
	\end{equation*}
	Since every step in the argument can be made functorial in $g\colon \Dd'\rightarrow \Dd$, we have proved \cref{enum:ForgetfulFunctor}.
	
	For \cref{enum:RightCofinalLeftAdjoint}, note that \cref{enum:ForgetfulFunctor} combined with \cref{thm:JoyalsQuillenA}\cref{enum:RightAnodyne} immediately implies that $q\colon Y\rightarrow \Dd$ is a left adjoint object to $g\colon \Dd'\rightarrow \Dd$ under the inclusion $\cat{Right}(\Dd)\subseteq \cat{Cat}_{\infty/\Dd}$. Since left adjoints can be constructed pointwise by \cref{lem:Adjunction}, we only need to check that such a factorisation always exists. But that's easy! For example, we could choose $\Dd'\rightarrow Y$ to be right anodyne by \cref{lem:SmallObjectArgument} and \cref{exm:Cofinal}\cref{enum:RightAnodyneCofinal}. If you'd like to avoid simplicial sets, we could also argue as follows: Choose a factorisation $\Dd'\rightarrow Y'\rightarrow \Dd$ into a right adjoint functor followed by a cartesian fibration $g'\colon Y'\rightarrow \Dd$ as in the proof of \cref{thm:JoyalsQuillenA}. Then put
	\begin{equation*}
		(g\colon Y\rightarrow \Dd)\coloneqq \operatorname{Un}^{(\mathrm{right})}\left(\Dd\xrightarrow{\operatorname{St}^{(\mathrm{cart})}(g)}\cat{Cat}_\infty\overset{\left|\,\cdot\,\right|}{\longrightarrow}\cat{An}\right)\,.
	\end{equation*}
	Finally, \cref{enum:RightPullbackLeftAdjoint} follows from the combined powers of \cref{enum:ForgetfulFunctor} and \cref{enum:RightCofinalLeftAdjoint}.
\end{proof}
%Then \cref{lem:KanExtensionForRight} implies the following:
In the following, we use $\PSh(\Cc)\coloneqq\Fun(\Cc^\op,\cat{An})$ to denote the \emph{$\infty$-category of presheaves on $\Cc$}. Note that this is \emph{not} compatible with the previous definition for ordinary categories.
\begin{cor}\label{cor:f_!:PC->PD}
	Let $F\colon \Cc\rightarrow \Dd$ be a functor of $\infty$-categories. Then the precomposition functor $F^*\colon \PSh(\Dd)\morphism\PSh(\Cc)$ has a left-adjoint $F_!$ such that the diagram
	\begin{equation*}
		\begin{tikzcd}
			\Cc\rar["F"]\dar["\Yo_\Cc"']\drar[commutes]& \Dd\dar["\Yo_\Dd"]\\			\PSh(\Cc)\rar["F_!"]&\PSh(\Dd)
		\end{tikzcd}
	\end{equation*}
	commutes in the $\infty$-category $\Cat_\infty$.
\end{cor}
\begin{proof}
	It's clear from \cref{lem:KanExtensionForRight} and the straightening equivalence (the dual of \cref{thm:Straightening}\cref{enum:LeftStraightening}) that $F_!$ exists, so we only have to show that the diagram commutes. To this end, first note that the natural transformation $\Hom_\Cc\!\left(-,-\right)\Rightarrow\Hom_\Dd\!\left(F(-),F(-)\right)$ gets transformed into $\Yo_\Cc\Rightarrow F^*\circ \Yo_\Dd\circ F$ under the equivalence in $\Fun(\Cc^\op\times\Cc,\cat{An})\simeq \Fun(\Cc,\PSh(\Cc))$. Using the adjunction $F_!\dashv F^*$ as well as \cref{cor:FunctorCategoryAdjunctions}, this transformation is adjoint to a natural transformation $F_!\circ \Yo_\Cc\Rightarrow \Yo_\Dd\circ F$.
	
	So our diagram commutes up to natural transformation, and we have to show that said natural transformation is an equivalence. By \cref{thm:EquivalencePointwise}, this can be done pointwise. So choose $x\in\Cc$. Under the straightening equivalence, the functor $\Yo_\Cc(x)\simeq \Hom_\Cc\!\left(-,x\right)$ corresponds to the right fibration $\Cc_{/x}\rightarrow \Cc$. Likewise, $\Yo_\Dd(F(x))\simeq \Hom_\Dd\!\left(-,F(x)\right)$ corresponds to $\Dd_{/F(x)}\rightarrow \Dd$. Using \cref{lem:KanExtensionForRight}\cref{enum:RightPullbackLeftAdjoint}, we only have to show that the top horizontal arrow in the diagram
	\begin{equation*}
		\begin{tikzcd}
			\Cc_{/x}\dar\rar\drar[commutes]& \Dd_{/F(x)}\dar\\
			\Cc\rar["F"]&\Dd
		\end{tikzcd}
	\end{equation*}
	is cofinal. But that's easy! Both $\Cc_{/x}$ and $\Dd_{/F(x)}$ have terminal objects, hence there are adjunctions $\Cc_{/x}\shortdoublelrmorphism \{\id_x\}$ and $\Dd_{/F(x)}\shortdoublelrmorphism\{\id_{F(x)}\}$. Hence $*\rightarrow \Cc_{/x}$ and $*\rightarrow \Dd_{/F(x)}$ are both cofinal by \cref{exm:Cofinal}\cref{enum:RightAdjointCofinal}. Then $\Cc_{/x}\rightarrow \Dd_{/F(x)}$ must be cofinal too (for example, by the condition from \cref{thm:JoyalsQuillenA}\cref{enum:Cofinal}).
\end{proof}
%Before we move on to define and construct general Kan extensions in the $\infty$-categorical world, let us record a very pleasant consequence of \cref{cor:f_!:PC->PD}: We can compute Hom animae in functor $\infty$-categories!
\cref{cor:f_!:PC->PD} allows us to compute Hom animae in functor $\infty$-categories!
\begin{cor}\label{cor:HomInFunctorCats}
	Given functors $F,G\colon \Cc\rightarrow\Dd$ of $\infty$-categories, the anima of natural transformations $\Hom_{\Fun(\Cc,\Dd)}\!\left(F,G\right)$ can be computed as the following limit:
	\begin{equation*}
		\limit_{(x\rightarrow y)\in\TwAr(\Cc)}\Hom_\Dd\lef(F(x),G(y)\righ)\coloneqq \limit\!\left(\TwAr(\Cc)\xrightarrow{\!(s,t)\!}\Cc^\op\times\Cc\xrightarrow{\!F^\op\times G\!}\Dd^\op\times\Dd\xrightarrow{\!\Hom_\Dd\!}\cat{An}\right)\,.
	\end{equation*}
\end{cor}
\begin{proof}
	By \cref{lem:ColimitsInAnima}, the right-hand side can be computed as $\Hom_{\Cat_{\infty/\TwAr(\Cc)}}(\TwAr(\Cc),\Uu)$, where $\Uu$ denotes the unstraightening of ${\Hom_\Dd}\circ(F^\op\times G)\circ (s,t)\colon \TwAr(\Cc)\rightarrow\cat{An}$. Since unstraightening transforms compositions into pullbacks and the unstraightening of $\Hom_\Dd$ is $\TwAr(\Dd)\rightarrow \Dd^\op\times\Dd$ by \cref{con:HomInTwoVariables} or~\labelcref{con:HomTwAr}, we have a pullback diagram
	\begin{equation*}
		\begin{tikzcd}
			\Uu\rar\dar\drar[pullback] & \Uu'\rar\dar\drar[pullback] &[1em] \TwAr(\Dd)\dar\\
			\TwAr(\Cc)\rar["{(s,t)}"] & \Cc^\op\times\Cc\rar["F^\op\times G"] & \Dd^\op\times\Dd
		\end{tikzcd}
	\end{equation*}
	Using \cref{lem:KanExtensionForRight}\cref{enum:RightPullbackLeftAdjoint}, we see $\Hom_{\Cat_{\infty/\TwAr(\Cc)}}\!\left(\TwAr(\Cc),\Uu\right)\simeq\Hom_{\Cat_{\infty/\Cc^\op\times\Cc}}\!\left(\TwAr(\Cc),\Uu'\right)$. But these are both left fibrations over $\Cc^\op\times\Cc$, so the Hom anima on the right-hand side can be equivalently computed as $\Hom_{\Fun(\Cc^\op\times\Cc,\cat{An})}\!\left(\Hom_\Cc,{\Hom_\Dd}\circ(F^\op\times G)\right)$. Now the \enquote{currying} equivalence  $\Fun(\Cc^\op\times\Cc,\cat{An})\simeq\Fun(\Cc,\PSh(\Cc))$ sends $\Hom_\Cc$ to $\Yo_\Cc$ and ${\Hom_\Dd}\circ(F^\op\times G)$ to $F^*\circ \Yo_\Dd\circ G$, hence the Hom anima under consideration is given by
	\begin{align*}
		\Hom_{\Fun(\Cc,\PSh(\Cc))}\lef(\Yo_\Cc,F^*\circ \Yo_\Dd\circ G)\righ)&\simeq \Hom_{\Fun(\Cc,\PSh(\Dd))}\!\left(F_!\circ \Yo_\Cc,\Yo_\Dd\circ G\right)\\
		&\simeq \Hom_{\Fun(\Cc,\PSh(\Dd))}\!\left(\Yo_\Dd\circ F,\Yo_\Dd\circ G\right)\\
		&\simeq \Hom_{\Fun(\Cc,\Dd)}\!\left(F,G\right)\,,
	\end{align*}
	as claimed. For the first equivalence, we use that $F_!\circ -$ is an adjoint of $F^*\circ -$ by construction and \cref{cor:FunctorCategoryAdjunctions}, the second equivalence follows from \cref{cor:f_!:PC->PD}, and the third one since $\Yo_\Dd\colon \Dd\morphism\PSh(\Dd)$ is fully faithful by Yoneda's lemma (\cref{cor:YonedaEmbeddingFullyFaithful}).
\end{proof}
We'll now define and construct Kan extensions in the $\infty$-categorical world.
\begin{defi}\label{def:KanExtensions}
	Let $f\colon \Cc\rightarrow \Cc'$ and $F\colon \Cc\rightarrow \Dd$ be functors of $\infty$-categories. A \emph{left Kan extension of $F$ along $f$}, denoted $\Lan_fF\colon \Cc'\rightarrow\Dd$, is a left adjoint object to $F$ under $f^*\colon \Fun(\Cc',\Dd)\rightarrow\Fun(\Cc,\Dd)$. Dually, a \emph{right Kan extension of $F$ along $f$}, denoted $\Ran_fF\colon \Cc'\rightarrow\Dd$, is a left adjoint object to $F$ under $f^*$.
\end{defi}
Kan extensions in the $\infty$-categorical world can be computed by the same formula as in the ordinary case (\cref{lem:1KanExtensionFormula}):
\begin{lem}[Kan extension formula]\label{lem:KanExtensionFormula}
	In the situation of \cref{def:KanExtensions}, assume that for all $x'\in \Cc'$ the following colimits exist in $\Dd$:
	\begin{equation*}
		\colimit_{(x,f(x)\rightarrow x')\in \Cc_{/x'}}F(x)\coloneqq \colimit\lef(\Cc_{/x'}\longrightarrow \Cc\overset{F}{\longrightarrow}\Dd\righ)
	\end{equation*}
	Then $\Lan_fF$ exists and $\Lan_fF(x')$ is given by that colimit.
\end{lem}
To prove this, we first show that taking colimits is functorial in both the indexing $\infty$-category and the functor. As it will turn out during the proof, this is equivalent to constructing a partial left adjoint to the Yoneda embedding $\Yo_\Dd\colon \Dd\rightarrow\PSh(\Dd)$.
\begin{lem}[\enquote{Colimits are functorial}]\label{lem:ColimitsFunctorial}
	Let $\Dd$ be an $\infty$-category. $\Tt\subseteq\Cat_{\infty/\Dd}$ be spanned by those $\alpha\colon \Ii\rightarrow \Dd$ that admit a colimit. Consider the functor $\Dd_{/-}\colon \Dd\rightarrow \cat{Cat}_{\infty/\Dd}$ that sends $y\in \Dd$ to $\Dd_{/y}\rightarrow \Dd$. Then $\Dd_{/-}$ lands in $\Tt$ and admits a left adjoint $\colimit\colon \Tt\rightarrow \Dd$ that sends $\alpha\colon \Ii\rightarrow\Dd$ to $\colimit_{i\in\Ii}\alpha(i)\in\Dd$.
\end{lem}
\begin{proof}
	Formally, the functor $\Dd_{/-}\colon \Dd\rightarrow\cat{Cat}_{\infty/\Dd}$ is defined via
	\begin{equation*}
		\Dd\xrightarrow{\Yo_\Dd}\PSh\left(\Dd\right)\simeq\cat{Right}(\Dd)\longrightarrow \cat{Cat}_{\infty/\Dd}\,,
	\end{equation*}
	using the Yoneda embedding and the right straightening equivalence (the dual of \cref{thm:Straightening}\cref{enum:LeftStraightening}). It's clear that $\Dd_{/-}$ takes values in $\Tt$. Indeed, $\Dd_{/y}$ has a terminal object and so the colimit over $\Dd_{/y}\rightarrow \Dd$ is just $y$. To prove the second assertion, by \cref{lem:Adjunction}, it's enough to prove that for every $\alpha\colon \Ii\rightarrow \Dd$, the colimit $\colimit_{i\in\Ii}\alpha(i)\in\Dd$ is a left adjoint object to $\alpha$ under $\Dd_{/-}\colon \Dd\rightarrow\Tt$. This can be seen as follows: If $c\simeq \colimit_{i\in\Ii}\alpha(i)$, then the associated natural transformation $\alpha\Rightarrow\const c$ induces a functor $u_\alpha\colon \Ii\rightarrow \Dd_{/c}$ in $\cat{Cat}_{\infty/\Dd}$. We then get a natural transformation
	\begin{equation*}
		\Hom_\Dd(c,-)\xRightarrow{\Dd_{/-}}\Hom_{\Cat_{\infty/\Dd}}\bigl(\Dd_{/c},\Dd_{/-}\bigr)\xRightarrow{u_\alpha^*}\Hom_{\Cat_{\infty/\Dd}}\bigl(\Ii,\Dd_{/-}\bigr)\,.
	\end{equation*}
	Equivalences can be checked pointwise by \cref{thm:EquivalencePointwise}. So choose $y\in\Dd$. We compute
	\begin{align*}
		\Hom_{\Cat_{\infty/\Dd}}\bigl(\Ii,\Dd_{/y}\bigr)&\simeq \{\alpha\}\times_{\Hom_{\Cat_\infty}(\Ii,\Dd),s}\Hom_{\Cat_\infty}\lef(\Ii,\Ar(\Dd)\times_{t,\Dd}\{y\}\righ)\\
		&\simeq \{\alpha\}\times _{\Hom_{\Cat_\infty}(\Ii,\Dd),s}\Hom_{\Cat_\infty}\lef(\Ii,\Ar(\Dd)\righ)\times_{t,\Hom_{\Cat_\infty}(\Ii,\Dd)}\{\const y\}\\
		&\simeq \Hom_{\Fun(\Ii,\Dd)}\!\left(\alpha,\const y\right)\,,
	\end{align*}
	and this agrees with $\Hom_\Dd(c,y)$ by definition of $c$. In the first step we use \cref{cor:HomInSliceCategories} as well as $\Dd_{/y}\simeq \Ar(\Dd)\times_{t,\Dd}\{y\}$. In the second step we use \cref{cor:HomPreservesLimits}. In the third step, we use  $\Hom_{\Cat_\infty}(\Ii,\Ar(\Dd))\simeq \Hom_{\Cat_\infty}(\Delta^1,\Fun(\Ii,\Dd))$ and then plug in the definition of $\Hom_{\Fun(\Ii,\Dd)}$.
\end{proof}
\begin{proof}[Proof of \cref{lem:KanExtensionFormula}]
	Consider the chain of functors
	\begin{equation*}
		\begin{tikzcd}
			\Fun\!\left(\Cc,\Dd\right)\rar["{(\Yo_\Dd)_*}"] &  \Fun\lef(\Cc,\PSh(\Dd)\righ)\dar["\simeq"']\rar[dashed] &[2em]  \Fun\lef(\Cc',\cat{Right}(\Dd)\righ)\dar["\simeq"]\\
			& \cat{Right}\left(\Dd\times\Cc^\op\right)\rar["(\id_\Dd\times f^\op)_!"] & \cat{Right}\lef(\Dd\times(\Cc')^\op\righ)
		\end{tikzcd}
	\end{equation*}
	(the vertical equivalences follow from the right straightening equivalence, see the dual of \cref{thm:Straightening}\cref{enum:LeftStraightening}). Let $F'\colon \Cc'\rightarrow\cat{Right}(\Dd)$ denote the image of $F$ under the top row functors and let $\ov{\Tt}\coloneqq \Tt\cap\cat{Right}(\Dd)$, where $\Tt$ is defined as in \cref{lem:ColimitsFunctorial}. If we can show that $F'$ is contained in the full sub-$\infty$-category $\Fun(\Cc',\ov\Tt)\subseteq \Fun\lef(\Cc',\cat{Right}(\Dd)\righ)$, then we can define $\Lan_fF\coloneqq{\colimit}\circ F'\in \Fun(\Cc',\Dd)$. It's clear from the various equivalences and adjunctions involved (more precisely, from \cref{cor:YonedaEmbeddingFullyFaithful}, \cref{lem:KanExtensionForRight}\cref{enum:RightPullbackLeftAdjoint}, and \cref{lem:ColimitsFunctorial} combined with \cref{cor:FunctorCategoryAdjunctions}) that $\Lan_fF$ is indeed a left adjoint object of $F$ under the precomposition functor $f^*\colon \Fun(\Cc',\Dd)\rightarrow \Fun(\Cc,\Dd)$.
	
	So we have to check that $F'$ is indeed contained in $\Fun(\Cc',\ov\Tt)$. The image of $F$ under $(\Yo_\Dd)_*$ followed by the \enquote{currying} equivalence $\Fun(\Cc,\Fun(\Dd^\op,\cat{An}))\simeq \Fun(\Dd^\op\times\Cc,\cat{An})$ is $\Hom_\Dd(-,F(-))\colon \Dd^\op\times\Cc\rightarrow \cat{An}$. Its right unstraightening is
	\begin{equation*}
		\TwAr(\Dd)^\op\times_{t^\op,\Dd^\op,F^\op}\Cc^\op\longrightarrow \Dd\times\Cc^\op\,.
	\end{equation*}
	Indeed, the right unstraightening of $\Hom_\Dd\colon \Dd^\op\times\Dd\rightarrow\cat{An}$ is $(s^\op,t^\op)\colon \TwAr(\Dd)^\op\rightarrow \Dd\times\Dd^\op$ by definition (of either $\Hom_\Dd$ or $\TwAr(\Dd)$, see \cref{con:HomInTwoVariables,con:HomTwAr}), and precomposition with $F\colon \Cc\rightarrow\Dd$ corresponds to pullback along $F^\op$.
	
	By  \cref{lem:KanExtensionForRight}\cref{enum:RightPullbackLeftAdjoint}, the functor $(\id_\Dd\times f^\op)_!$ sends $\TwAr(\Dd)^\op\times_{t^\op,\Dd^\op,F^\op}\Cc^\op\rightarrow \Dd\times\Cc^\op$ to a cofinal replacement of $\TwAr(\Dd)^\op\times_{t^\op,\Dd^\op,F^\op}\Cc^\op\rightarrow \Dd\times\Cc^\op\rightarrow \Dd\times(\Cc')^\op$ by a right fibration. To figure out how such a cofinal replacement looks like, we claim the following:
	\begin{alphanumerate}\itshape
		\item[\boxtimes_1] In the diagram below, both vertical arrows are cofinal:\label{claim:TwArCofinal}
		\begin{equation*}
			\begin{tikzcd}[column sep=-7em]
				& \TwAr(\Dd)^\op\times_{t^\op,\Dd^\op,F\circ t^\op}\TwAr(\Cc)^\op\times_{f\circ s^\op,\Cc',s^\op}\TwAr(\Cc')^\op\dlar[start anchor=186]\drar[start anchor=-6]& \\
				\TwAr(\Dd)^\op\times_{t^\op,\Dd^\op,F^\op}\Cc^\op & & \Cc\times_{f,\Cc',s^\op}\TwAr(\Cc')^\op
			\end{tikzcd}
		\end{equation*} 
	\end{alphanumerate}
	To prove claim~\cref{claim:TwArCofinal}, we first observe that for every $\infty$-category $\Ii$, both the source projection $s^\op\colon \TwAr(\Ii)^\op\rightarrow \Ii$ and the target projection $t^\op\colon \TwAr(\Ii)^\op\rightarrow \Ii^\op$ are cofinal cartesian fibrations. Indeed, cartesianness is clear since $\TwAr(\Ii)^\op\rightarrow\Ii\times\Ii^\op$ is a right fibration and projection to either factor is cartesian. For cofinality, we use \cref{lem:CartesianCofinal}: The fibre of $t^\op$ over $i\in \Ii^\op$ is $(t^\op)^{-1}\{i\}\simeq \Ii_{/i}$; this follows from \cref{lem:HomRealityCheck}, regardless of which construction of $\TwAr(\Ii)$ you use. Now $\Ii_{/i}$ is weakly contractible since it has a terminal object. The same argument applies to $s^\op$. Now the conditions from \cref{lem:CartesianCofinal} are stable under base change, which proves that the arrows in the diagram above are indeed cofinal.
	
	So we may equivalently look for a cofinal replacement of $\Cc\times_{f,\Cc',s^\op}\TwAr(\Cc')^\op\rightarrow \Dd\times(\Cc')^\op$ by a right fibration. Once again, we won't do this directly; instead, we claim another claim:
	\begin{alphanumerate}\itshape
		\item[\boxtimes_2] In the diagram below, the vertical arrows are cartesian fibrations over $(\Cc')^\op$ and the horizontal arrows preserve cartesian lifts:\label{claim:CartesianDiagram}
		\begin{equation*}
			\begin{tikzcd}[column sep=large]
				\Cc\times_{\Cc',s^\op}\TwAr(\Cc')^\op\rar[r,"{({s^\op},\,{t^\op})}"]\drar["t^\op"']& \Cc\times(\Cc')^\op\dar["\pr_2"]\rar["F\times\id_{(\Cc')^\op}"]\dar[phantom,""{name=A}]\arrow[from=1-1,to=A,commutes,pos=0.7]\dar[phantom,""{name=A}]\arrow[from=A,to=1-3,commutes,pos=0.3] &[1em]\Dd\times(\Cc')^\op\dlar["\pr_2"]\\
				&(\Cc')^\op &
			\end{tikzcd}
		\end{equation*}
	\end{alphanumerate}
	Indeed, by definition of $\TwAr(\Cc')$, the arrow labelled $(s^\op,t^\op)$ is a right fibration, and both arrows labelled $\pr_2$ are cartesian fibrations (see \cref{exm:Straightening}\cref{enum:ProjectionsStraightenToConstantFunctors}). Hence $t^\op$, being a composition of cartesian fibrations, is cartesian too. Furthermore, $t^\op$-cartesian lifts are precisely the $(s^\op,t^\op)$-cartesian lifts of a $\pr_2$-cartesian lift, which immediately proves that $(s^\op,t^\op)$ preserves cartesian lifts. Finally, it's clear that $F\times\id_{(\Cc')^\op}$ preserves cartesian lifts, since these are given by those morphisms in $\Cc\times(\Cc')^\op$ and $\Dd\times(\Cc')^\op$ that are equivalences in the first component. This proves claim~\cref{claim:CartesianDiagram}.
	
	The cartesian straightening $\operatorname{St}^{(\mathrm{cart})}(t^\op)$ is a functor $\Cc'\rightarrow\cat{Cat}_\infty$. By the diagram above, it comes with a natural transformation $\operatorname{St}^{(\mathrm{cart})}(t^\op)\Rightarrow \const \Dd$, so that $\operatorname{St}^{(\mathrm{cart})}(t^\op)$ lifts to a functor $\Cc_{/-}\colon \Cc'\rightarrow \cat{Cat}_{\infty/\Dd}$. On objects, $\Cc_{/-}$ is given by sending $x'\in \Cc'$ to the slice category $\Cc_{/x'}$, which becomes an object in $\cat{Cat}_{\infty/\Dd}$ via
	\begin{equation*}
		\Cc_{/x'}\longrightarrow \Cc\overset{F}{\longrightarrow}\Dd\,.
	\end{equation*}
	Now that's something we've seen before! Our assumption precisely tells us that $\Cc_{/-}$ restricts to a functor $\Cc_{/-}\colon \Cc'\rightarrow\Tt$. To finish, let $c\colon \cat{Cat}_{\infty/\Dd} \rightarrow \cat{Right}(\Dd)$ denote the left adjoint to $\cat{Right}(\Dd)\subseteq \cat{Cat}_{\infty/\Dd}$, which exists due to \cref{lem:KanExtensionForRight}\cref{enum:RightCofinalLeftAdjoint}. It's clear from \cref{thm:JoyalsQuillenA}\cref{enum:Cofinal} that $c$ sends $\Tt$ to $\ov\Tt$, hence we obtain a functor $c\circ \Cc_{/-}\colon \Cc'\rightarrow\ov\Tt$. We claim that this finally allows us to compute the desired cofinal replacement:
	\begin{alphanumerate}[resume]\itshape
		\item[\boxtimes_3] If $p\colon X\rightarrow \Dd\times(\Cc')^\op$ is a cofinal replacement of $\Cc\times_{f,\Cc',s^\op}\TwAr(\Cc')^\op\rightarrow \Dd\times(\Cc')^\op$ by a right fibration, then the image of $p$ under $\cat{Right}(\Dd\times(\Cc')^\op)\simeq \Fun(\Cc',\cat{Right}(\Dd))$ will coincide with $c\circ \Cc_{/-}$.\label{claim:CofinalReplacement}
	\end{alphanumerate}
	To prove claim~\cref{claim:CofinalReplacement}, consider the following diagram, in which the dashed arrows are left adjoints (whose existence we're going to prove below):
	\begin{equation*}
		\begin{tikzcd}
			\bigl(\Cat_{\infty/(\Cc')^\op}\bigr)_{/\Dd\times(\Cc')^\op}\dar["\simeq"']\drar[commutes] & \cat{Cart}\!\left(\Cc'\right)_{/\Dd\times(\Cc')^\op}\rar["\simeq"]\lar\dar[dashed,shift right=0.2em,"c"']\drar[commutes] & \Fun\bigl(\Cc',\Cat_{\infty/\Dd}\bigr)\dar[dashed,shift right=0.2em,"c_*"']\\
			\Cat_{\infty/\Dd\times(\Cc')^\op}\rar[dashed, shift left=0.2em, "c"] & \cat{Right}\lef(\Dd\times(\Cc')^\op\righ)\rar["\simeq"]\lar[shift left=0.2em]\uar[shift right=0.2em] & \Fun\lef(\Cc',\cat{Right}(\Dd)\righ)\uar[shift right=0.2em]
		\end{tikzcd}
	\end{equation*}
	The horizontal equivalences as well as commutativity of the square on the right follow from the cartesian straightening equivalence (the dual of \cref{thm:Straightening}). Furthermore, once we know that the left adjoints exist, they will also form a commutative square on the right, since taking left adjoints is always compatible with equivalences. The vertical equivalence on the left follows by inspection (\enquote{a slice of a slice is a slice}). The vertical left adjoint $c_*$ exists by \cref{cor:FunctorCategoryAdjunctions}. The horizontal left adjoint $c\colon \Cat_{\infty/\Dd\times(\Cc')^\op}\rightarrow \cat{Right}\lef(\Dd\times(\Cc')^\op\righ)$ exists by \cref{lem:KanExtensionForRight}\cref{enum:RightCofinalLeftAdjoint}, and it we claim that it induces a left adjoint
	\begin{equation*}
		c\colon  \cat{Cart}\!\left(\Cc'\right)_{/\Dd\times(\Cc')^\op}\longrightarrow \cat{Right}\lef(\Dd\times(\Cc')^\op\righ)
	\end{equation*}
	to the forgetful functor $\cat{Right}\lef(\Dd\times(\Cc')^\op\righ)\rightarrow \cat{Cart}\!\left(\Cc'\right)_{/\Dd\times(\Cc')^\op}$. Indeed, if $\Uu\rightarrow \Dd\times(\Cc')^\op$ and $\Uu'\rightarrow \Dd\times(\Cc')^\op$ are elements in $\cat{Cart}\!\left(\Cc'\right)_{/\Dd\times(\Cc')^\op}$, then
	\begin{equation*}
		\Hom_{\cat{Cart}\left(\Cc'\right)_{/\Dd\times(\Cc')^\op}}(\Uu,\Uu')\longrightarrow \Hom_{\cat{Cat}_{\infty/\Dd\times(\Cc')^\op}}(\Uu,\Uu')
	\end{equation*}
	is usually \emph{not} an equivalence, only an inclusion of path components, since on the left-hand side, cartesian lifts need to be preserved. However, if $\Uu'\rightarrow \Dd\times(\Cc')^\op$ happens to be a right fibration, then it's straightforward to check that cartesian lifts are preserved automatically, so in this case we \emph{do} get an equivalence, which proves that $c$ is still a left adjoint when restricted along the non-fully faithful functor $\cat{Cart}\left(\Cc'\right)_{/\Dd\times(\Cc')^\op}\rightarrow \cat{Cat}_{\infty/\Dd\times(\Cc')^\op}$. So we've proved that the diagram above also commutes if we take the dashed left adjoints into account. This is precisely what we need to prove claim~\cref{claim:CofinalReplacement}.
	
	Using claim~\cref{claim:CofinalReplacement}, we've now succeeded in proving that $F\colon \Cc'\rightarrow\cat{Right}(\Dd)$ takes values in $\ov\Tt$, which proves that $\Lan_fF$ exists. Furthermore, for every $c'\in\Cc'$, the value $\Lan_fF(x')$ is given by a colimit over $c(\Cc_{/x'})$. Since the unit morphism $u_{\Cc_{/x'}}\colon \Cc_{/x'}\rightarrow c(\Cc_{/x'})$ is cofinal by \cref{lem:KanExtensionForRight}\cref{enum:RightPullbackLeftAdjoint}, we may as well take the colimit over $\Cc_{/x'}$. This proves that $\Lan_fF(x')$ is given by the desired formula and we're finally done!		
\end{proof}
\begin{cor}%[\enquote{Kan extensions along fully faithful functors behave nicely.}]
	\label{cor:KanExtensionAlongFullyFaithful}
	In the situation from \cref{def:KanExtensions}, assume that $f\colon \Cc\rightarrow \Cc'$ is fully faithful and that the colimits from \cref{lem:KanExtensionFormula} exist in $\Dd$. Then the natural transformation $u_F\colon F\Rightarrow \Lan_fF\circ f$ is an equivalence.
\end{cor}
\begin{proof}
	This follows from the same argument as in \cref{cor:1KanExtensionAlongFullyFaithful}, plus the fact that equivalences can be checked pointwise by \cref{thm:EquivalencePointwise}.
\end{proof}
We can now state the main result of this section: the $\infty$-categorical analogue of \cref{thm:1PShFreeCocompletion}!
\begin{thm}[\enquote{$\PSh(\Cc)$ arises by freely adding colimits to $\Cc$.}]\label{thm:PShFreeCocompletion}
	Let $\Cc$ and $\Dd$ be $\infty$-categories, where $\Dd$ has all colimits. Then restriction along the Yoneda embedding $\Yo_\Cc$ induces an equivalence
	\begin{equation*}
		\Yo_\Cc^*\colon \Fun^{\colimit}\lef(\PSh(\Cc),\Dd\righ)\overset{\simeq}{\longrightarrow}\Fun(\Cc,\Dd)\,.
	\end{equation*}
	Here $\Fun^{\colimit}(\PSh(\Cc),\Dd)\subseteq \Fun(\PSh(\Cc),\Dd)$ is the full sub-$\infty$-category spanned by the colimits-preserving functors. Furthermore, every colimits-preserving functor $\PSh(\Cc)\rightarrow \Dd$ admits a right adjoint.
\end{thm}
As it turns out, the proof will be exactly the same as for ordinary categories. Let's start with the two lemmas whose proofs where omitted in the ordinary case; at last, this will be rectified now!
\begin{lem}[\enquote{Every presheaf is a colimit of representables.}]\label{lem:PresheafColimitOfRepresentables}
	Let $\Cc$ be an $\infty$-category. For every $E\in \PSh(\Cc)$, the natural morphism
	\begin{equation*}
		\colimit_{(y,\Hom_\Cc(-,y)\rightarrow E)\in \Cc_{/E}}\Hom_\Cc(-,y)\overset{\simeq}{\longrightarrow}E
	\end{equation*}
	is an equivalence.
\end{lem}
\begin{proof}
	Since we get the natural transformation for free, we can check pointwise whether it is an equivalence (\cref{thm:EquivalencePointwise}). So fix $x\in\Cc$. Since colimits in $\PSh(\Cc)$ are computed pointwise (\cref{lem:ColimitsInFunctorCategories}), what we need to show is
	\begin{equation*}
		\colimit\left(\Cc_{/E}\overset{s}{\longrightarrow}\Cc\xrightarrow{\Hom_\Cc(x,-)}\cat{An}\right)\simeq E(x)\,.
	\end{equation*}
	By \cref{lem:ColimitsInAnima}, the colimit on the left-hand side is given by $\left|\Uu\right|$, where $\Uu$ is the unstraightening of $\Hom_\Cc(x,-)$. Since precomposition transforms into pullbacks under unstraightening, we find that $\Uu$ sits inside a pullback
	\begin{equation*}
		\begin{tikzcd}
			\Uu\rar\dar\drar[pullback] &\Cc_{/E}\rar \dar["s"]\drar[pullback]& \PSh(\Cc)_{/E}\dar["s"]\\
			\Cc_{x/}\rar & \Cc\rar["\Yo_\Cc"] & \PSh(\Cc)
		\end{tikzcd}
	\end{equation*}
	Since $\PSh(\Cc)_{/E}\rightarrow\PSh(\Cc)$ is a right fibration, $\Uu\rightarrow \Cc_{x/}$ is one too. In particular, it is a cartesian fibration. Hence \cref{lem:CartesianFibres} shows $\mathopen|\Uu\times_{\Cc_{x/}}\{\id_x\}\mathclose|\simeq \mathclose|\Uu\times_{\Cc_{x/}}(\Cc_{x/})_{\id_x/}|\simeq \left|\Uu\right|$; here we use $(\Cc_{x/})_{\id_x/}\simeq \Cc_{x/}$, since $\id_x\in \Cc_{x/}$ is an initial object. Now
	\begin{align*}
		\bigl|\Uu\times_{\Cc_{x/}}\{\id_x\}\bigr|\simeq \Uu\times_{\Cc_{x/}}\{\id_x\}&\simeq \PSh(\Cc)_{/E}\times_{\PSh(\Cc)}\lef\{\Yo_\Cc(x)\righ\}\\
		&\simeq \Hom_{\PSh(\Cc)}\lef(\Yo_\Cc(x),E\righ)\\
		&\simeq E(x)\,.
	\end{align*}
	In the first step, we use that the fibre $\Uu\times_{\Cc_{x/}}\{\id_x\}$ is already an anima, since $\Uu\rightarrow \Cc_{x/}$ is a right fibration. The second equivalence follows from the pullback diagram above. In the third step, we use the definition of $\Hom_{\PSh(\Cc)}$, and in the fourth step, we use Yoneda's lemma (\cref{thm:Yoneda}). In total, we find $\left|\Uu\right|\simeq E(x)$, which is exactly what we wanted to prove.
\end{proof}
\begin{lem}\label{lem:LanAlongYonedaHasRightAdjoint}
	For every $F\colon \Cc\rightarrow \Dd$, the left Kan extension $\Lan_{\Yo_\Cc}F\colon \PSh(\Cc)\rightarrow\Dd$ \embrace{which exists due to \cref{lem:KanExtensionFormula}} admits a right adjoint. The right adjoint sends $y\in \Dd$ to $\Hom_\Dd(F(-),y)\colon \Cc^\op\rightarrow\cat{Set}$.
\end{lem}
\begin{proof}
	Fix $y\in\Dd$. Since adjoints can be constructed pointwise (\cref{lem:Adjunction}), we only need to construct an equivalence
	\begin{equation*}
		\Hom_\Dd\bigl(\Lan_{\Yo_\Cc}F(-),y\bigr)\simeq \Hom_{\PSh(\Cc)}\bigl(-,\Hom_\Dd(F(-),y)\bigr)
	\end{equation*}
	of functors $\PSh(\Cc)^\op\rightarrow\cat{An}$. Restricting along $\Yo_\Cc^\op\colon \Cc^\op\rightarrow\PSh(\Cc)^\op$, both sides become $\Hom_\Dd(F(-),y)$: The left-hand side by \cref{cor:KanExtensionAlongFullyFaithful}, the right-hand side by Yoneda's lemma (\cref{thm:Yoneda}; see also \cref{par:YonedaFunctorial}). By the universal property of right Kan extension, we thus obtain natural transformations
	\begin{equation*}
		\Hom_\Dd\bigl(\Lan_{\Yo_\Cc}F(-),y\bigr)\Longrightarrow \Ran_{\Yo_\Cc^\op} \Hom_\Dd\!\left(F(-),y\right)\Longleftarrow\Hom_{\PSh(\Cc)}\bigl(-,\Hom_\Dd\!\left(F(-),y\right)\bigr)\,.
	\end{equation*}
	We claim that they're both equivalences. In either case, this can be checked pointwise by \cref{thm:EquivalencePointwise}. So plug in some $E\in \PSh(\Cc)$. We obtain a diagram
	\begin{equation*}
		\begin{tikzcd}[column sep=-0.5em]
			\Hom_\Dd\bigl(\Lan_{\Yo_\Cc}F(E),y\bigr)\rar\drar[bend right=15,end anchor=180,"\simeq"']\drar[commutes,pos=0.55]& \Ran_{\Yo_\Cc^\op} \Hom_\Dd\lef(F(E),y\righ)\dar["\simeq"] &\Hom_{\PSh(\Cc)}\bigl(E,\Hom_\Dd\!\left(F(-),y\right)\bigr)\dlar[commutes,pos=0.55]\lar\dlar[bend left=15,end anchor=0,"\simeq"]\\
			& \limit_{(x,\Yo_\Cc(x)\rightarrow E)\in (\Cc_{/E})^\op}\Hom_\Dd\lef(F(x),y\righ)
		\end{tikzcd}
	\end{equation*}
	The vertical arrow in the middle is an equivalence by the dual of \cref{lem:KanExtensionFormula}. For the vertical arrow on the left, we plug in the left Kan extension formula from \cref{lem:KanExtensionFormula} and use \cref{cor:HomPreservesColimits} to see that $\Hom_\Dd(-,y)$ transforms the colimit into a limit. For the vertical arrow on the right, we plug in \cref{lem:PresheafColimitOfRepresentables}, use \cref{cor:HomPreservesColimits} again to see that $\Hom_{\PSh(\Cc)}(-,\Hom_\Dd(F(-),y))$ transforms the colimit into a limit, and then use Yoneda's lemma. This proves that we obtain equivalences as desired.
\end{proof}
%The final ingredient in the proof of \cref{thm:PShFreeCocompletion} is the $\infty$-categorical analogue of \cref{lem:1FullyFaithfulConservativeAdjunction}.
\begin{lem}\label{lem:FullyFaithfulConservativeAdjunction}
	Let $\Cc$ and $\Dd$ be categories and let $L\colon \Cc\shortdoublelrmorphism \Dd\noloc R$ be an adjunction.
	\begin{alphanumerate}
		\item \!The left adjoint $L$ is fully faithful if and only if the unit transformation $u\colon \id_\Cc\Rightarrow RL$ is an equivalence.\label{enum:FullyFaithfulIffUnitEquivalence}
		\item Suppose the condition from \cref{enum:FullyFaithfulIffUnitEquivalence} is true. Furthermore, suppose that $R$ is conservative \embrace{that is, if $\alpha\colon x\rightarrow y$ is a morphism in $\Dd$ such that $R(\alpha)$ is an isomorphism, then $\alpha$ is an isomorphism too}. Then $L$ and $R$ are inverse equivalences of categories.\label{enum:Conservative}
	\end{alphanumerate}
\end{lem}
\begin{proof}
	The proof of \cref{lem:1FullyFaithfulConservativeAdjunction} can be copied verbatim.
\end{proof}
\begin{proof}[Proof of \cref{thm:PShFreeCocompletion}]
	By \cref{lem:LanAlongYonedaHasRightAdjoint} and \cref{lem:AdjointsPreserveColimits}, the adjunction $\Lan_{\Yo_\Cc}\dashv\Yo_\Cc^*$ restricts to an adjunction
	\begin{equation*}
		\Lan_{\Yo_\Cc}\colon \Fun(\Cc,\Dd)\doublelrmorphism\Fun^{\colimit}\lef(\PSh(\Cc),\Dd\righ)\noloc \Yo_\Cc^*\,.
	\end{equation*}
	By \cref{lem:FullyFaithfulConservativeAdjunction}\cref{enum:Conservative}, to prove that $\Lan_{\Yo_\Cc}$ and $\Yo_\Cc^*$ are inverse equivalences, we need to show that the unit $u\colon \id_{\Fun(\Cc,\Dd)}\Rightarrow\Yo_\Cc^*\circ \Lan_{\Yo_\Cc}$ is an equivalence and that $\Yo_\Cc^*$ is conservative. That $u$ is an equivalence can be checked object-wise by \cref{thm:EquivalencePointwise}, where it follows from \cref{cor:KanExtensionAlongFullyFaithful}, since the Yoneda embedding $\Yo_\Cc$ is fully faithful (\cref{cor:YonedaEmbeddingFullyFaithful}). To see that $\Yo_\Cc^*$ is conservative, we must show that a natural transformation $\eta\colon F\Rightarrow G$ between colimits-preserving functors $F,G\colon \PSh(\Cc)\rightarrow\Dd$ is an equivalence already if it is an equivalence when restricted to representable presheaves. But this is clear since every presheaf can be written as a colimit of representables (\cref{lem:PresheafColimitOfRepresentables}).
\end{proof}
\subsection{Digression: Homology, cohomology, Eilenberg--MacLane animae}\label{subsec:EilenbergMacLane}
\cref{thm:PShFreeCocompletion} is surprisingly powerful even in the special case $\Cc\simeq *$. In this case we have $\PSh(*)\simeq \cat{An}$ and so  \cref{thm:PShFreeCocompletion} says that a colimits-preserving functor $\cat{An}\rightarrow\Dd$ is uniquely determined by what it does on $*\in\cat{An}$. If you think about this, it's maybe not that surprising: \cref{thm:SimplicialApproximation} says that animae are essentially CW complexes and every CW complex is glued together from topological disks $D^n$. But $D^n\simeq *$. So $*$ generates all of $\cat{An}$ under colimits.

Using this observation, our goal in this subsection is to give a purely abstract proof of the Eilenberg--MacLane theorem (\cref{thm:EilenbergMacLane}). The first step is to construct an interesting $\infty$-category $\Dd$ with all colimits.

\begin{numpar}[Crash course in derived $\infty$-categories I: Basic definitions.]\label{con:DerivedCategoryI}
	Let $R$ be a (not necessarily commutative) ring. We'll give a very brief description of the \emph{derived $\infty$-category} $\Dd(R)$ and its variant $\Dd_{\geqslant 0}(R)$.%Of course, \enquote{very brief} means, unfortunately, that we won't give any proofs, only references.
	
	Let $\Ch(R)$ be the category of chain complexes of left $R$-modules and let $\Ch_{\geqslant 0}(R)\subseteq \Ch(R)$ be the full subcategory of those chain complexes $M_*=\left(\dotsb \rightarrow M_{n+1}\rightarrow M_n\rightarrow M_{n-1}\rightarrow\dotsb\right)$ that satisfy $M_n=0$ for $n<0$. Recall that a morphism $f\colon M_*\rightarrow N_*$ in $\Ch(R)$ is called a \emph{quasi-isomorphism} if $f\colon \H_i(M_*)\overset{\cong}{\longrightarrow} \H_i(N_*)$ is an isomorphism for all $i$. Then we put
	\begin{align*}
		\Dd(R)&\coloneqq \Ch(R)\lef[\left\{\text{quasi-isomorphisms}\right\}^{-1}\righ]\\
		\Dd_{\geqslant 0}(R)&\coloneqq \Ch_{\geqslant 0}(R)\lef[\left\{\text{quasi-isomorphisms}\right\}^{-1}\righ]\,,
	\end{align*}
	where the localisations are taken in the $\infty$-categorical sense (see \cref{con:Localisation}). The homotopy categories $\operatorname{ho}\Dd(R)$ and $\operatorname{ho}\Dd_{\geqslant 0}(R)$ are then the usual derived categories $D(R)$ and $D_{\geqslant 0}(R)$ from commutative algebra; see Corollary/Warning~\cref{cor:Localisation}. It's true, but not at all clear from the definition, that $\Dd_{\geqslant 0}(R)$ is a full sub-$\infty$-category of $\Dd(R)$. This will follow from the alternative descriptions of $\Dd(R)$ and $\Dd_{\geqslant 0}(R)$ that we give in crash course~\cref{con:DerivedCategoryIII} below. It's also true, but not obvious, that $\Dd(R)$ and $\Dd_{\geqslant 0}(R)$ have all colimits. We won't prove this, but it will become apparent through the proof of \cref{lem:Homology} below how this can be done.
	
	For a chain complex $M_*$, we often write $M$ for its image in $\Dd(R)$ to emphasise that this is no longer a \enquote{complex up to isomorphism}, but a \enquote{complex up to quasi-isomorphism}, so that for $M\in\Dd(R)$ there is no longer a well-defined notion of \enquote{$M_n$, the degree-$n$ part of $M$}. The homology of $M$, however, is still well-defined. Indeed, for all $n$ the functors $\H_n\colon \Ch(R)\rightarrow \cat{Mod}_R$ and $\H_n\colon \Ch_{\geqslant 0}(R)\rightarrow \cat{Mod}_R$ send quasi-isomorphisms to isomorphisms (by definition), hence by \cref{lem:Localisation} they define essentially unique functors
	\begin{equation*}
		\H_n\colon\Dd(R)\longrightarrow\cat{Mod}_R\quad\text{and}\quad \H_n\colon\Dd_{\geqslant 0}(R)\longrightarrow\cat{Mod}_R
	\end{equation*}
\end{numpar}
\begin{numpar}[Crash course in derived $\infty$-categories II: An alternative construction.]\label{con:DerivedCategoryIII}
	It's also possible to construct $\Dd(R)$ and $\Dd_{\geqslant 0}(R)$ as simplicial nerves of Kan-enriched categories (see \cref{con:SimplicialNerve}). We'll first explain how to obtain the Kan enrichment: Let $\Hhom_R(M_*,N_*)$ be the chain complex of abelian groups given by 
	\begin{equation*}
		\Hhom_R\!\left(M_*,N_*\right)_n\coloneqq\prod_{i\in\IZ}\Hom_R\!\left(M_i,N_{i+n}\right)\,.
	\end{equation*}
	The differentials send a family of morphisms $f=(f_i)_{i\in\IZ}\in\prod_{i\in\IZ}\Hom_R(M_i,N_{i+n})$ to the family $\partial f\coloneqq(\partial_M\circ f-(-1)^nf\circ \partial_N)_{i\in\IZ}$; here $\partial_M$ and $\partial_N$ denote the differentials of $M_*$ and $N_*$, respectively. By unravelling the definitions, we see that the $n$-cycles and $n$-boundaries of $\Hhom_R(M_*,N_*)$ are given by
	\begin{align*}
		\mathrm Z_n\lef(\Hhom_R(M_*,N_*)\righ)&\cong \Hom_{\Ch(R)}\lef(M_*,N[n]_*\righ)\\
		\mathrm B_n\lef(\Hhom_R(M_*,N_*)\righ)&\cong \left\{f\in\Hom_{\Ch(R)}\lef(M_*,N[n]_*\righ)\ \middle|\ f\text{ nullhomotopic}\right\}\,.
	\end{align*}
	Here $N[n]_*$ denotes the chain complex $N_*$ shifted by $i$. In particular, $\H_n\lef(\Hhom_R(M_*,N_*)\righ)$ is in bijection with the set of homotopy classes of maps $M_*\rightarrow N[n]_*$.
	
	The complexes $\Hhom_R(-,-)$ provide an enrichment of $\Ch(R)$ over $\Ch(\cat{
		Ab})$. To make this into a Kan enrichment, we let $\tau_{\geqslant 0}\Hhom_R(M_*,N_*)$ be the \emph{smart truncation of $\Hhom_R(M_*,N_*)$}. That is, we put
	\begin{equation*}
		\lef(\tau_{\geqslant 0}\Hhom_R(M_*,N_*)\righ)_n\coloneqq \begin{cases*}
			\Hhom_R(M_*,N_*)_n & if $n>0$\\
			\mathrm Z_0\lef(\Hhom_R(M_*,N_*)\righ) & if $n=0$\\
			0 & if $n<0$
		\end{cases*}\,,
	\end{equation*}
	so that $\H_n\lef(\tau_{\geqslant 0}\Hhom_R(M_*,N_*)\righ)\cong \H_n\lef(\Hhom_R(M_*,N_*)\righ)$ if $n\geqslant 0$ and $\H_n\lef(\tau_{\geqslant 0}\Hhom_R(M_*,N_*)\righ)\cong0$ if $n<0$. Then $\tau_{\geqslant 0}\Hhom_R(M_*,N_*)$ is a chain complex concentrated in nonnegative degrees. By the Dold-Kan correspondence (see \cite[Theorem~\HAthm{1.2.3.7}]{HA} for example), there is an equivalence of categories $\Ch_{\geqslant 0}(\IZ)\simeq \cat{sAb}$; we let $\F_{\Ch(R)}(M_*,N_*)$ denote the simplicial abelian group corresponding to $\Hhom_R(M_*,N_*)$ under this equivalence. The simplicial abelian groups $\F_{\Ch(R)}(-,-)$ provide an enrichment of $\Ch(R)$ in simplicial sets. This is automatically a Kan enrichment, since every simplicial abelian group is a Kan complex (see \cite[\stackstag{08NZ}]{Stacks}).
	
	A complex $P_*$ of $R$-modules is called \emph{$K$-projective} if $\Hhom_R(P_*,-)\colon \Ch(R)\rightarrow \Ch(R)$ preserves quasi-isomorphisms. It was shown by Spaltenstein \cite{KProjective} that every chain complex of $R$-modules $M_*$ admits a quasi-isomorphism $P_*\rightarrow M_*$ from a $K$-projective complex. If $P_*$ is $K$-projective, then it is \emph{degree-wise projective} in the sense that every $P_n$ is a projective $R$-module. Conversely, if $P_*$ is degree-wise projective and bounded below in the sense that $P_n\cong0$ for $n\lle 0$, then $P_*$ is $K$-projective. These statements can be found in \cite[Lemma~\href{https://people.math.rochester.edu/faculty/doug/otherpapers/hovey-model-cats.pdf\#page=52}{2.3.6}]{HoveyModelCategories}; the second statement also appears (in dual form) in \cite[\stackstag{070J}]{Stacks}.
	
	Let $K\mhyph\cat{Proj}(R)\subseteq\Ch(R)$ and $\cat{Proj}_{\geqslant 0}(R)\subseteq \Ch_{\geqslant 0}(R)$ be the full subcategories spanned by the $K$-projective complexes. Equip $K\mhyph\cat{Proj}(R)$ and $\cat{Proj}_{\geqslant 0}(R)$ with the Kan enrichment above. Then
	\begin{equation*}
		\Dd(R)\simeq \N_\Delta\lef(K\mhyph\cat{Proj}(R)\righ)\quad\text{and}\quad \Dd_{\geqslant 0}(R)\simeq \N_\Delta\lef(\cat{Proj}_{\geqslant 0}(R)\righ)\,.
	\end{equation*}
	The idea to prove this is, of course, similar to \cref{thm:AnAsALocalisation}: One can construct a simplicial model structure on $\Ch(R)$ in such a way that $K\mhyph\cat{Proj}(R)\simeq \Ch(R)^\circ$ are precisely the bifibrant objects. Then the above equivalences follow from a general result of Dwyer--Kan and Lurie. See \cite[\S\href{https://people.math.rochester.edu/faculty/doug/otherpapers/hovey-model-cats.pdf\#page=50}{2.3}]{HoveyModelCategories} for the construction of the model structure and \cite[Theorem~\HAthm{1.3.4.20}]{HA} for the general equivalence. Let us also mention that there is another simplicial model structure on $\Ch(R)$ in which the bifibrant objects are the \emph{$K$-injective} complexes, that is, those $I_*$ for which $\Hhom_R(-,I_*)$ preserves quasi-isomorphisms, and one has similar equivalences $\Dd(R)\simeq \N_\Delta(K\mhyph\cat{Inj}(R))$ and $\Dd_{\leqslant 0}(R)\simeq \N_\Delta(\cat{Inj}_{\leqslant 0}(R))$.
	
	This alternative construction is useful to compute $\Hom_{\Dd(R)}$. Let $M_*$ and $N_*$ be complexes and let $P_*\rightarrow M_*$ and $Q_*\rightarrow N_*$ be quasi-isomorphisms from $K$-projective complexes. Using \cref{thm:CordierPorter} and the fact that the Dold--Kan correspondence transforms homotopy groups of simplicial $R$-modules into homology groups of the associated chain complexes, we find
	\begin{equation*}
		\pi_n\Hom_{\Dd(R)}\!\left(M,N\right)\cong \pi_n\F_{\Ch(R)}\!\left(P_*,Q_*\right)\cong \H_n\lef(\Hhom_R(P_*,Q_*)\righ)
	\end{equation*}
	for all $n\geqslant 0$ and all basepoints. Furthermore, we have $\H_n\lef(\Hhom_R(P_*,Q_*)\righ)\cong \H_n\lef(\Hhom_R(P_*,N_*)\righ)$ by definition of $P_*$ being $K$-projective, so we only need to resolve $M_*$ by a $K$-projective complex. If you think at this point that $\Hom_{\Dd(R)}(M,N)$ looks suspiciously like the derived Hom functor $\RHom_R(M,N)$, you're right on the money: We'll see in \cref{cor:RHom} how exactly these two are related.\hfill$\blacksquare$
\end{numpar}
This covers the basics of derived $\infty$-categories. Before we move on to our applications, we need to introduce some convenient terminology.
\begin{defi}\label{def:Cofibre}
	Let $\Cc$ be an $\infty$-category with a terminal object $*$ and let $\alpha\colon x\rightarrow y$ be a morphism in $\Cc$. The \emph{cofibre of $\alpha$} is defined as the pushout
	\begin{equation*}
		\begin{tikzcd}
			x\rar["\alpha"]\dar\drar[pushout] & y\dar\\
			*\rar & \cofib(\alpha)
		\end{tikzcd}
	\end{equation*}
	(provided this exists in $\Cc$). We say that $x\overset{\alpha}{\longrightarrow}y\rightarrow z$ is a \emph{cofibre sequence in $\Cc$} if it exhibits $z$ as the cofibre of $\alpha$. There are dual notions of the \emph{fibre of $\alpha$} $\fib(\alpha)$  (given as the pullback against an initial object) and \emph{fibre sequences}.
\end{defi}
Now let's see \cref{thm:PShFreeCocompletion} in action!
\begin{con}\label{con:Homology}
	For all integers $n$ and all abelian groups $A$ define a chain complex
	\begin{equation*}
		A[n]_*\coloneqq \left(\dotsb\rightarrow 0\rightarrow 0\rightarrow A\rightarrow 0\rightarrow 0\rightarrow \dotsb\right)
	\end{equation*}
	with $A$ in degree $n$ and $0$ everywhere else. Consider the functor $\cat{Ab}\rightarrow\Ch_{\geqslant 0}(\IZ)\rightarrow\Dd_{\geqslant 0}(\IZ)$ that sends $A$ to $A[0]$. Since $\Dd_{\geqslant 0}(\IZ)$ has all colimits, so has $\Fun(\cat{Ab},\Dd_{\geqslant 0}(\IZ))$ by \cref{lem:ColimitsInFunctorCategories}. Hence, by \cref{thm:PShFreeCocompletion}, there exists a unique colimits-preserving functor $\cat{An}\rightarrow\Fun(\cat{Ab},\Dd_{\geqslant 0}(\IZ))$ that sends $*\in\cat{An}$ to the functor $\cat{Ab}\rightarrow\Ch_{\geqslant 0}(\IZ)\rightarrow\Dd_{\geqslant 0}(\IZ)$ discussed above. By \enquote{currying}, we obtain a functor
	\begin{equation*}
		\C(-,-)\colon \cat{An}\times\cat{Ab}\longrightarrow \Dd_{\geqslant 0}(\IZ)\,.
	\end{equation*} 
	For every abelian group $A$, $\C(-,A)\colon \cat{An}\rightarrow\Dd_{\geqslant 0}(\IZ)$ is the unique colimits-preserving functor that sends $*\in\cat{An}$ to $A[0]$ as above. For an anima $X$, we call $\C(X,A)$ the \emph{chains of $X$ with coefficients in $A$} and we call $\widetilde{\C}(X,A)\coloneqq \fib(\C(X,A)\rightarrow\C(*,A))$ the \emph{reduced chains of $X$ with coefficients in $A$} (using the fibre construction from \cref{def:Cofibre}). For all $n\geqslant 0$, we let
	\begin{align*}
		\H_n\!\left(X,A\right)&\coloneqq \H_n\lef(\C(X,A)\righ)\,,\\
		\widetilde{\H}_n\lef(X,A\righ)&\coloneqq\H_n\bigl(\widetilde{\C}(X,A)\bigr)
	\end{align*}
	denote the \emph{$n$\textsuperscript{th} homology of $X$ with coefficients in $A$} and the \emph{$n$\textsuperscript{th} reduced homology of $X$ with coefficients in $A$}; here $\H_n\colon \Dd_{\geqslant 0}(\IZ)\rightarrow\cat{Ab}$ is the functor from crash course~\cref{con:DerivedCategoryI}. Finally,
	\begin{align*}
		\H^n\!\left(X,A\right)&\coloneqq \pi_0\Hom_{\Dd_{\geqslant 0}(\IZ)}\lef(\C(X,\IZ),A[n]\righ)\,,\\ \widetilde{\H}^n\left(X,A\right)&\coloneqq \pi_0\Hom_{\Dd_{\geqslant 0}(\IZ)}\bigl(\widetilde{\C}\left(X,\IZ\right),A[n]\bigr)%\cong \H_{-n}\Bigl(\RHom_\IZ\lef(\C(X,\IZ),A[0]\righ)\Bigr)
	\end{align*}
	denote the \emph{$n$\textsuperscript{th} cohomology of $X$ with coefficients in $A$} and the \emph{$n$\textsuperscript{th} reduced cohomology of $X$ with coefficients in $A$}. We'll verify in \cref{lem:Homology} below that this definition of homology and cohomology is compatible with the one you are familiar with.
	
	But before we do that, let's see that with our definition of cohomology, the Eilenberg--MacLane theorem becomes almost a triviality! By \cref{thm:PShFreeCocompletion}, $\C(-,\IZ)\colon \cat{An}\rightarrow \Dd_{\geqslant 0}(\IZ)$ automatically acquires a right adjoint, which we denote $\K\colon \Dd_{\geqslant 0}(\IZ)\rightarrow\cat{An}$. For $M\in\Dd_{\geqslant 0}(\IZ)$ we call $\K(M)$ the \emph{generalised Eilenberg--MacLane anima of $M$}; in the case $M\simeq A[n]$ we say that $\K(A,n)\coloneqq \K(A[n])$ is the \emph{Eilenberg--MacLane anima of type $(A,n)$}. We'll justify this terminology in \cref{par:ComputationsWithEilenbergMacLane} below.
\end{con}
\begin{lem}\label{lem:KAdjunctionPointedAnima}
	The adjunction $\C(-,\IZ)\colon \cat{An}\shortdoublelrmorphism \Dd_{\geqslant 0}(\IZ)\noloc\K$ lifts to an adjunction
	\begin{equation*}
		\widetilde{\C}\!\left(-,\IZ\right)\colon \cat{An}_{*/}\doublelrmorphism \Dd_{\geqslant 0}(\IZ)\noloc \K\,.
	\end{equation*}
\end{lem}
\begin{proof}[Proof sketch]
	Since $\K$ is a right adjoint, it preserves limits. In particular, it preserves terminal objects, whence $\K(0)\simeq *$. But $0$ is also initial in $\Dd_{\geqslant 0}(\IZ)$. Hence $\Dd_{\geqslant 0}(\IZ)\simeq \Dd_{\geqslant 0}(\IZ)_{0/}$ and so $\K$ lifts to a functor $\K\colon \Dd_{\geqslant 0}(\IZ)\simeq \Dd_{\geqslant 0}(\IZ)_{0/}\rightarrow \cat{An}_{\K(0)/}\simeq \cat{An}_{*/}$, as desired.
	
	For a pointed anima $(X,x)\in \cat{An}_{*/}$, the canonical morphism $X\rightarrow*$ has a canonical section given by $\{x\}\rightarrow X$. Thus $\C(X,\IZ)\rightarrow\C(*,\IZ)$ has a section and we obtain
	\begin{equation*}
		\C\!\left(X,\IZ\right)\simeq \widetilde{\C}\left(X,\IZ\right)\oplus \C\lef(\{x\},\IZ\righ)\,.
	\end{equation*}
	Hence the reduced chains $\widetilde{\C}(X,\IZ)$ from \cref{con:Homology} can also be written as the cofibre $\widetilde{\C}(X,\IZ)\simeq\cofib(\C(\{x\},\IZ)\rightarrow \C(X,\IZ))$, functorially in the pointed anima $(X,x)$. Then
	\begin{equation*}
		\Hom_{\Dd_{\geqslant 0}(\IZ)}\bigl(\widetilde{\C}(X,\IZ),M\bigr)\simeq \Hom_{\cat{An}_{*/}}\lef((X,x),\K(M)\righ)\,,
	\end{equation*}
	follows easily using \cref{cor:HomPreservesColimits}, \cref{cor:HomInSliceCategories}, and the given adjunction $\C(-,\IZ)\dashv\K$. This proves that $\widetilde{\C}(-,\IZ)\colon \cat{An}_{*/}\rightarrow\cat{An}\rightarrow\Dd_{\geqslant 0}(\IZ)$ is indeed a left adjoint to $\K\colon \Dd_{\geqslant 0}(\IZ)\rightarrow\cat{An}_{*/}$.
\end{proof}
\begin{numpar}[Computations with Eilenberg--MacLane animae.]\label{par:ComputationsWithEilenbergMacLane}
	Let $S^n\in\cat{An}$ be the \emph{$n$-sphere}. There are many possible constructions. The most conceptual way would be to  define $S^n\coloneqq \Sigma^n(*\ \,*)$ as the $n$-fold suspension of two points, using the upcoming definition  \cref{def:Loop}. But there are also many possible simplicial models for $S^n$: For example, if $\partial D^{n+1}\subseteq D^{n+1}$ is the boundary of the topological $(n+1)$-disk, we could take $\Sing \partial D^{n+1}$ as our model for $S^n$. Alternatively, we could choose anodyne maps from $\square^n/\partial\square^n$ or $\Delta^n/\partial\Delta^n$ or $\partial\Delta^{n+1}$ into Kan complexes. All constructions are homotopy equivalent, so you can just choose your favourite option. Furthermore, in each case, the reduced homology of $S^n$ is given by $\widetilde{\H}_n(S^n,\IZ)\cong \IZ$ and $\widetilde{\H}_i(S^n,\IZ)\cong0$ for $i\neq n$; this follows from Lemmas~\labelcref{lem:HomologyEasier} or~\labelcref{lem:Homology} below. Thus $\widetilde{\C}(S^n,\IZ)\simeq \IZ[n]$.
	
	For every complex $M_*\in\Ch_{\geqslant 0}(\IZ)$,  we can now use \cref{lem:KAdjunctionPointedAnima} and the computation of $\Hom_{\Dd_{\geqslant 0}(\IZ)}$ from crash course~\cref{con:DerivedCategoryIII} to compute
	\begin{align*}
		\pi_n \K(M)\cong \pi_0\Hom_{\cat{An}_{*/}}\lef((S^n,*),\K(M)\righ)&\cong \pi_0\Hom_{\Dd_{\geqslant 0}(\IZ)}\bigl(\widetilde{\C}(S^n,\IZ),M\bigr)\\
		&\cong \H_0\Bigl(\Hhom_\IZ\lef(\IZ[n],M_*\righ)\Bigr)\\
		&\cong \H_n(M_*)\,.
	\end{align*}
	In the special case where $M_*=A[n]_*$, we conclude that the homotopy groups of the Eilenberg--MacLane anima $K(A,n)$ from \cref{con:Homology} are given by $\pi_n\K(A,n)\cong A$ and $\pi_i\K(A,n)\cong0$ for $i\neq n$, as we would expect. By the usual argument from topology, $\K(A,n)$ is uniquely determined by this property up to homotopy equivalence. So our construction of $\K(A,n)$ really coincides with the one you know. But we can say even more! For an anima $X$,
	\begin{equation*}
		\pi_0\Hom_{\cat{An}}\lef(X,\K(A,n)\righ)\cong \pi_0\Hom_{\Dd_{\geqslant 0}(\IZ)}\lef(\C(X,\IZ),A[n]\righ)\cong \H^n(X,A)\,.
	\end{equation*}
	using the adjunction $\C(-,\IZ)\colon\cat{An}\shortdoublelrmorphism \Dd_{\geqslant 0}(\IZ)\noloc \K$ and the definition of cohomology from \cref{con:Homology}. Analogously, for any pointed anima $(X,x)$, \cref{lem:KAdjunctionPointedAnima} can be used to show that  $\pi_0\Hom_{\cat{An}_{*/}}((X,x),\K(A,n))\cong \widetilde{\H}^n(X,A)$. In summary, we have just proved the following classical theorem, in a completely formal way and with a minimal amount of computations:
\end{numpar}
\begin{thm}[\enquote{Eilenberg--MacLane animae represent cohomology}]\label{thm:EilenbergMacLane}
	For every abelian group $A$ and all $n\geqslant 0$, there exists a unique \embrace{up to homotopy equivalence} anima $\K(A,n)$ satisfying $\pi_n\K(A,n)\cong A$ and $\pi_i\K(A,n)\cong0$ for $i\neq n$. Furthermore, $\K(A,n)$ represents cohomology with coefficients in $A$ \embrace{both unreduced and reduced} in the sense that the functors $\H^n(-,A)\colon \cat{An}\rightarrow \cat{Ab}$ and $\widetilde{\H}^n(-,A)\colon \cat{An}_{*/}\rightarrow \cat{Ab}$ are given by
	\begin{equation*}
		\lef[-,\K(A,n)\righ]\coloneqq\pi_0\Hom_{\cat{An}}\lef(-,\K(A,n)\righ)\quad\text{and}\quad \lef[-,\K(A,n)\righ]_*\coloneqq\pi_0\Hom_{\cat{An}_{*/}}\lef(-,\K(A,n)\righ)\,,
	\end{equation*}
	respectively.\hfill$\qedsymbol$
\end{thm}
To finish this subsection, we'll show that the definition of homology and cohomology from \cref{con:Homology} agrees with the definition that you are familiar with. But really, the position we take here is that \cref{con:Homology} is the better definition (and an even better one is \cref{cor:Homology})! To convince you further, we'll show that the usual properties of homology are straightforward to verify in our definition, much easier than for the usual definition, and we'll see that even though \cref{con:Homology} is very abstract, it's easy to extract concrete calculations.
\begin{lem}\label{lem:HomologyEasier}
	For all abelian groups $A$, the functors $\H_*(-,A)$ and $\widetilde{\H}_*(-,A)$ satisfy the Eilenberg--Steenrod axioms. For example, they are homotopy invariants and if $f\colon Y\rightarrow X$ is a morphism of animae with cofibre $X/Y\coloneqq\cofib(f)$, then there is a long exact sequence
	\begin{equation*}
		\dotsb\longrightarrow\H_n\!\left(Y,A\right)\longrightarrow\H_n\!\left(X,A\right)\longrightarrow\widetilde{\H}_n\lef(X/Y,A\righ)\overset{\partial}{\longrightarrow}\H_{n-1}\!\left(Y,A\right)\longrightarrow\dotsb
	\end{equation*}
	\embrace{so $\widetilde{\H}_*(X/Y,A)$ plays the role of the relative homology $\H_*(X,Y,A)$}. Furthermore, the suspension isomorphism is satisfied and pushouts of animae yield a long exact excision sequence.
\end{lem}
\begin{proof}[Proof sketch]
	Homotopy invariance follows from the definition of $\cat{An}$. The other assertions all follow from the fact that $\C(-,A)\colon \cat{An}\rightarrow\Dd_{\geqslant 0}(\IZ)$ preserves colimits, in particular, pushouts. To demonstrate these kinds of arguments, we'll show the long exact sequence; the rest will be left to you. We start with the pushout diagram
	\begin{equation*}
		\begin{tikzcd}
			\C(Y,A)\rar\dar\drar[pushout] & \C(*,A)\rar\dar\drar[pushout] & 0\dar\\
			\C(X,A)\rar & \C\lef(X/Y,A\righ)\rar & \widetilde{\C}\lef(X/Y,A\righ)
		\end{tikzcd}
	\end{equation*}
	(the left square is a pushout since $\C(-,A)$ preserves pushouts and the right square is a pushout by the proof of \cref{lem:KAdjunctionPointedAnima}). Hence $\C(Y,A)\rightarrow\C(X,A)\rightarrow\widetilde{\C}(X/Y,A)$ is a cofibre sequence in $\Dd_{\geqslant 0}(\IZ)$. So we have to unravel how cofibres in $\Dd_{\geqslant 0}(\IZ)$ are computed. We claim:
	\begin{alphanumerate}\itshape
		\item[\boxtimes] Let $i\colon M_*\rightarrow N_*$ be a morphism of chain complexes of $R$-modules for some ring $R$. Then the cofibre of $i$ in $\Dd(R)$ can be computed as the mapping cone\label{claim:CofibresInD}
		\begin{equation*}
			\cofib\left(i\colon M\rightarrow N\right)\simeq \cone\!\left(i\colon M_*\rightarrow N_*\right)
		\end{equation*}
		\embrace{the mapping cone is described in \cite[\stackstag{014E}]{Stacks}, but beware that they use cohomological indexing}. More generally, if $j\colon M_*\rightarrow M'_*$ is another morphism of complexes, then the pushout of the span $N_*\leftarrow M_*\rightarrow M'_*$ in $\Dd(R)$ is given by $\cone((i,-j)\colon M_*\rightarrow M'_*\oplus N_*)$. The same is true for bounded below complexes and $\Dd_{\geqslant 0}(R)$.
	\end{alphanumerate}
	Every sequence $M_*\rightarrow N_*\rightarrow \cone(M_*\rightarrow N_*)$ gives rise to a long exact sequence in homology. Hence \cref{claim:CofibresInD} implies that cofibre sequences in $\Dd_{\geqslant 0}(\IZ)$ give rise to long exact sequences in homology, which is exactly what we need.
	
	To prove \cref{claim:CofibresInD}, first choose quasi-isomorphisms $P_*\rightarrow M_*$ and $Q_*\rightarrow N_*$ from $K$-projective complexes and replace $i$ by a morphism $i'\colon P_*\rightarrow Q_*$. Since the mapping cone construction preserves quasi-isomorphisms (which follows from the long exact homology sequence of mapping cones and the five lemma), it suffices to show $\cofib(i')\simeq \cone(i')$. Using \cref{cor:HomPreservesColimits}, it's enough to show that
	\begin{equation*}
		\Hom_{\Dd(R)}\lef(\cone(i'),K\righ)\longrightarrow\Hom_{\Dd(R)}(Q,K)\longrightarrow\Hom_{\Dd(R)}(P,K)
	\end{equation*}
	is a fibre sequence of animae for all $K\in\Dd(R)$. Since $P_*\rightarrow Q_*\rightarrow\cone(i')$ is nullhomotopic, we get a map from $\Hom_{\Dd(R)}\lef(\cone(i'),K\righ)$ into the fibre. To prove that this map is an equivalence of animae, we can use \cref{thm:Whitehead}: Since $\cone(i')$ is $K$-projective, we can compute $\pi_n\Hom_{\Dd(R)}(\cone(i'),K)$ using crash course~\cref{con:DerivedCategoryIII}, and then by a simple comparison of long exact sequences, we get a bijection on $\pi_0$ and isomorphisms on $\pi_n$ for all $n\geqslant 1$ and all basepoints. This proves $\cofib(i)\simeq \cone(i')$. The assertion about pushouts follows formally from the assertion about cofibres. This finishes the proof of \cref{claim:CofibresInD}. 
\end{proof}
\begin{rem}
	So with our definition of homology, all the usual properties follow from a simple unravelling of colimits in $\Dd_{\geqslant 0}(\IZ)$. I find this much more enlightening and much less technical than the usual proof of the Mayer--Vietoris sequence via barycentric subdivision (see \cite[Proposition~\href{https://pi.math.cornell.edu/~hatcher/AT/AT.pdf\#page=128}{2.21}]{Hatcher} for example)!
	
	Moreover, for a given anima $X$, it's easy to write down an explicit complex that computes $\C(X,A)$ (and so homology is very computable). One way is \cref{lem:Homology} below, of course, but often it can be done by hand, and more efficiently. For example, assume we're given a \enquote{CW decomposition} of $X$; that is, a way to write $X$ as a sequence of pushouts along $S^n\rightarrow *$ (the $n$-disk is contractible, so we may as well use $*$ instead). Then, $\C(X,A)$ can be written as a similar sequence of pushouts in $\Dd_{\geqslant 0}(\IZ)$. Since we understand $\C(S^n,A)\simeq A[0]\oplus A[n]$ as well as $\C(*,A)\simeq A[0]$ and since we know how to compute pushouts in $\Dd_{\geqslant 0}(\IZ)$ by claim~\cref{claim:CofibresInD} in the proof of \cref{lem:HomologyEasier} above, we can compute $\C(X,A)$. If you think about this, the complex we end up with is precisely the \emph{cellular complex of $X$}, so we've just proved that homology agrees with cellular homology. Combining this with the classical fact that cellular and singular homology agree, we get an alternative proof of \cref{lem:Homology} below.
\end{rem}
Finally, here's the comparison result with the classical definition of homology.
\begin{lem}\label{lem:Homology}
	Let $X$ be an anima and $A$ an abelian group. Let $\C_*^\mathrm{sing}\left(\left|X\right|,A\right)$ be the usual singular chain complex of the geometric realisation $\left|X\right|\in\cat{Top}$. Then
	\begin{equation*}
		\C(X,A)\simeq \C^\mathrm{sing}_*\lef(|X|,\IZ\righ)\,.
	\end{equation*}
	In particular, the unreduced and reduced homology and cohomology of $X$, defined as in \cref{con:HomComposition}, agree with the usual ones for $\left|X\right|$.
\end{lem}
To prove \cref{lem:Homology}, we need the following fact about colimits (which is well-known for ordinary categories, but it's perhaps a bit surprising that this works for $\infty$-categories too, given that our colimits can be indexed by arbitrary $\infty$-categories, not just \enquote{discrete} ordinary categories):
\begin{lem}\label{lem:ColimitsIffCoproductsAndPushouts}
	An $\infty$-category $\Cc$ has all colimits if and only if $\Cc$ has pushouts and arbitrary coproducts. A functor $F\colon \Cc\rightarrow\Dd$ of $\infty$-categories preserves colimits if and only if it preserves pushouts and arbitrary coproducts. A dual assertion holds for limits.
\end{lem}
The crucial point, and the reason why we get away with \enquote{ordinary} colimits like pushouts and coproducts, is that $\{n\}\rightarrow \Delta^n$ is cofinal (in fact, right anodyne, so \cref{exm:Cofinal}\cref{enum:RightAnodyneCofinal} applies). Hence every functor $T\colon \Delta^n\rightarrow \Cc$ admits a colimit. For a general functor $T\colon\Ii\rightarrow \Cc$, we write $\Ii$ as a colimit of its skeleta to build $\colimit_{i\in\Ii}T(i)$ \enquote{simplex-by-simplex}: This needs pushouts (to attach $n$-simplices in the $n$\textsuperscript{th} step) and coproducts (to attach arbitrarily many $n$-simplices at the same time). 

To make this precise, we'll prove a lemma that will allow us to manipulate colimits: We can \enquote{slice a colimit into pieces} and \enquote{assemble colimits from subdiagrams}:
\begin{lem}\label{lem:ColimitManipulations}
	Let $\Ii$ and $\Cc$ be $\infty$-categories.
	\begin{alphanumerate}
		\item Suppose $p\colon \Uu\rightarrow \Ii$ is a cocartesian fibration and $T\colon \Uu\rightarrow \Cc$ is a functor such that $T|_{p^{-1}\{i\}}\colon p^{-1}\{i\}\rightarrow \Cc$ admits a colimit for all $i\in \Ii$. Then these colimits assemble into a functor $\ov T\colon \Ii\rightarrow \Cc$ satisfying $\ov T(i)\simeq \colimit_{u\in p^{-1}\{i\}}T(u)$. Furthermore,\label{claim:SliceColimits}
		\begin{equation*}
			\colimit_{u\in\Uu}T(u)\simeq \colimit_{i\in\Ii}\ov T(i)\,,
		\end{equation*}
		provided that at least one of these colimits exists in $\Cc$ \embrace{in which case the other exists as well}. Informally, we can rephrase this as $\colimit_{u\in\Uu}T(u)\simeq \colimit_{i\in\Ii}\colimit_{u\in p^{-1}\{i\}}T(u)$.
		\item Suppose $\Ii\simeq \colimit_{j\in\Jj}\Ii_j$ in $\Cat_\infty$. Let $T\colon \Ii\rightarrow \Cc$ be a functor such that the restrictions $T|_{\Ii_j}\colon \Ii_j\rightarrow \Cc$ admit colimits for all $j\in\Jj$. Then these colimits assemble into a functor $\ov T\colon \Jj\rightarrow \Cc$ satisfying $\ov T(j)\simeq \colimit_{i\in\Ii_j}T(i)$. Furthermore,
		\label{claim:AssembleColimits}
		\begin{equation*}
			\colimit_{i\in\Ii}T(i)\simeq \colimit_{j\in\Jj}\ov T(j)\,.
		\end{equation*}
		provided that at least one of these colimits exists in $\Cc$ \embrace{in which case the other exists as well}. Informally, we can rephrase this as $\colimit_{i\in\Ii}T(i)\simeq \colimit_{j\in\Jj}\colimit_{i\in\Ii_j}T(i)$.
	\end{alphanumerate}
	In particular, \enquote{colimits commute with colimits}: If $\Jj$ is an $\infty$-category and $T\colon \Ii\times\Jj\rightarrow \Cc$ is any functor, then
	\begin{equation*}
		\colimit_{i\in\Ii}\colimit_{j\in\Jj}T(i,j)\simeq\colimit_{(i,j)\in\Ii\times\Jj}T(i,j)\simeq \colimit_{j\in\Jj}\colimit_{i\in\Ii}T(i,j)\,.
	\end{equation*}
	%Dual assertions hold for limits \embrace{meaning that in \cref{claim:SliceColimits}, we consider a cartesian fibration, but in \cref{claim:AssembleColimits}, we still consider $\Ii\simeq\colimit_{j\in\Jj}\Ii_j$}.
\end{lem}
\begin{proof}
	To prove \cref{claim:SliceColimits}, first note that $\Lan_pT$ exists. Indeed, $p^{-1}\{i\}\rightarrow \Uu_{/i}$ is cofinal by the dual of \cref{lem:CartesianFibres} and \cref{exm:Cofinal}\cref{enum:RightAdjointCofinal}, so the existence of the colimits over $p^{-1}\{i\}$ implies that the condition from \cref{lem:KanExtensionFormula} is satisfied. So we can put $\ov T\coloneqq \Lan_pT$. Now $\colimit T$ corresponds to taking the left Kan extension of $T$ along $\Uu\rightarrow *$ (see \cref{exm:1ColimitAsKanExtension}). But we may as well first left Kan extend along $\Uu\rightarrow \Ii$ and then left Kan extend along $\Ii\rightarrow *$. This proves $\colimit T\simeq \colimit \Lan_pT$ and we've finished the proof of \cref{claim:SliceColimits}. In the special case where $p$ is the projection $\pr_1\colon \Ii\times\Jj\rightarrow \Jj$, we obtain the \enquote{in particular}.
	
	For \cref{claim:AssembleColimits}, let $p\colon \Uu\rightarrow \Jj$ be the unstraightening of the functor $\Jj\rightarrow \cat{Cat}_\infty$ of which $\Ii$ is the colimit. Then $\Ii$ is a localisation of $\Uu$ by \cref{lem:ColimitsInAnima}, so there's a natural functor $q\colon \Uu\rightarrow\Ii$. We have $p^{-1}\{j\}\simeq \Ii_j$, so we can apply \cref{claim:SliceColimits} to the functor $q\circ T\colon \Uu\rightarrow \Cc$. This allows us to construct $\ov T$ and we obtain $\colimit \ov T\simeq \colimit q\circ T$. But $q$, being a localisation, is cofinal by \cref{exm:Cofinal}\cref{enum:LocalisationsCofinal}, and so $\colimit q\circ T\simeq \colimit T$. This proves \cref{claim:AssembleColimits}.
\end{proof}
%\begin{rem}\label{rem:ColimitsCommute}
%	An important special case of \cref{lem:ColimitManipulations}\cref{claim:SliceColimits} is the case where $p$ is the projection $\pr_1\colon \Ii\times\Jj\rightarrow \Jj$: We obtain
%	\begin{equation*}
%		\colimit_{i\in\Ii}\colimit_{j\in\Jj}T(i,j)\simeq\colimit_{(i,j)\in\Ii\times\Jj}T(i,j)\simeq \colimit_{j\in\Jj}\colimit_{i\in\Ii}T(i,j)
%	\end{equation*}
%	for every functor $T(i,j)\rightarrow\Cc$. A dual assertion holds for limits. Informally, \enquote{colimits commute with colimits} and \enquote{limits commute with limits}.
%\end{rem}
\begin{proof}[Proof sketch of \cref{lem:ColimitsIffCoproductsAndPushouts}]
	In simplicial sets, we can write $\Ii\cong \colimit_{n\geqslant 0}\operatorname{sk}_n\Ii$, where $\operatorname{sk}_{n}\Ii$ is obtained from $\operatorname{sk}_{n-1}\Ii$ by attaching copies of $\Delta^n$; that is, we take a pushout along some coproduct of the form $\coprod \partial\Delta^n\rightarrow \coprod\Delta^n$. Up to replacing everything by $\infty$-categories (using \cref{lem:SmallObjectArgument}), we can thus write $\Ii\simeq \colimit_{n\geqslant 0}\Ii_n$ in $\cat{Cat}_\infty$ in such a way that $\Ii_n$ is obtained from $\Ii_{n-1}$ by a pushout along $\coprod \Bb^n\rightarrow \coprod \Delta^n$, where $\partial\Delta^n\rightarrow \Bb^n$ is an inner anodyne map into an $\infty$-category. By an inductive argument (in which \cref{lem:ColimitManipulations}\cref{claim:AssembleColimits} powers the inductive step), we find that $\colimit_{i\in\Ii_n}T(i)$ exists in $\Cc$ for all $n\geqslant 0$. Using \cref{lem:ColimitManipulations}\cref{claim:AssembleColimits} once again, it remains to show that $\colimit_{n\geqslant 0}\colimit_{i\in\Ii_n}T(i)$ exists in $\Cc$. But this colimit can be easily written as a suitable pushout of the disjoint union $\coprod_{n\geqslant 0}\colimit_{i\in\Ii_n}T(i)$.
\end{proof}
\begin{proof}[Proof of \cref{lem:Homology}]
	Let's first address the \enquote{in particular}: It's clear that the equivalence $\C(X,A)\simeq \C_*^\mathrm{sing}\left(\left|X\right|,A\right)$ implies that the homology of $X$ from \cref{con:Homology} agrees with the singular homology of $\left|X\right|$. This easily implies the same for reduced homology. For cohomology, recall $\H^n\!\left(X,A\right)\cong \pi_0\Hom_{\Dd_{\geqslant 0}(\IZ)}\lef(\C(X,\IZ),A[n]\righ)$; since $\C(X,A)$ is given by $\C_*^\mathrm{sing}\left(\left|X\right|,A\right)$, which is degree-wise free over $\IZ$ and bounded below, thus $K$-projective, we can rewrite $\H^n\!\left(X,A\right)$ as
	\begin{equation*}
		\pi_0\Hom_{\Dd_{\geqslant 0}(\IZ)}\Bigl(\C_*^\mathrm{sing}\lef(\left|X\right|,\IZ\righ),A[n]\Bigr)\cong \H_0\Hhom_\IZ\lef(\C_*^\mathrm{sing}\lef(\left|X\right|,\IZ\righ),A[n]\righ)
	\end{equation*}
	using the computation from crash course~\cref{con:DerivedCategoryIII}. Now $\Hhom_\IZ(\C_*^\mathrm{sing}(\left|X\right|,\IZ),A)\cong \C_\mathrm{sing}^{-*}(\left|X\right|,A)$ is the cochain complex that computes singular cohomology of $\left|X\right|$, placed in negative degrees (so that it becomes a chain complex). Taking the shifts into account, we get $\H^n(X,A)\cong \H^n(\left|X\right|,A)$, as claimed. This also implies the assertion about reduced cohomology.
	
	To prove $\C(X,A)\simeq \C_*^\mathrm{sing}\left(\left|X\right|,A\right)$, let's first make $\C_*^\mathrm{sing}(\left|\,\cdot\,\right|,A)$ into a functor $\cat{An}\rightarrow\Dd_{\geqslant 0}(\IZ)$. It clearly defines a functor $\cat{Kan}\rightarrow \Ch_{\geqslant 0}(\IZ)$, so by \cref{thm:AnAsALocalisation} and \cref{lem:Localisation}, all we need to check is that it sends homotopy equivalences of Kan complexes to quasi-isomorphisms. But that's a well-known property of the singular chain complex, since singular homology is a homotopy invariant.
	
	Clearly, $\C_*^\mathrm{sing}(\left|\,\cdot\,\right|,A)$ also sends $*$ to $A[0]$, so it remains to prove that $\C_*^\mathrm{sing}(\left|\,\cdot\,\right|,A)$ preserves colimits. By \cref{lem:ColimitsIffCoproductsAndPushouts}, it suffices to do the case of pushouts and coproducts.
	
	\emph{Preservation of Coproducts.} This case is easy: We have
	\begin{equation*}
		\C_*^\mathrm{sing}\lef(\left|\coprod X_i\right|,A\righ)\cong \bigoplus \C_*^\mathrm{sing}\lef(\left|X_i\right|,A\righ)
	\end{equation*}
	and direct sums are coproducts in $\Dd_{\geqslant 0}(\IZ)$ (which is straightforward to see from crash course~\cref{con:DerivedCategoryIII} and the fact that quasi-isomorphisms are preserved under arbitrary direct sums). 
	
	\emph{Preservation of Pushouts.} Suppose we're given a span $Y\leftarrow X\rightarrow X'$ in $\cat{An}$. By model category fact~\cref{par:HomotopyPushout}, the pushout in $\cat{An}$ can be computed as follows: First replace both arrows by cofibrations (using \cref{lem:SmallObjectArgument}); it would have sufficed to replace only one, but it'll be convenient later on to have both arrows cofibrations. Let $\ov Y'$ be the pushout in $\cat{sSet}$. Then the pushout $Y'$ in $\cat{An}$ can be obtained by choosing an anodyne map $\ov Y'\rightarrow Y'$ into a Kan complex (again using \cref{lem:SmallObjectArgument}). Since anodyne maps are sent to homotopy equivalences via $\left|\,\cdot\,\right|$, it will be enough to show that
	\begin{equation*}
		\begin{tikzcd}
			\C_*^\mathrm{sing}\lef(\left|X\right|,A\righ)\rar["i"]\dar["j"']\drar[commutes] & \C_*^\mathrm{sing}\lef(\mathopen|X'\mathclose|,A\righ)\dar\\
			\C_*^\mathrm{sing}\lef(\left|Y\right|,A\righ)\rar & \C_*^\mathrm{sing}\lef(\mathopen|\ov Y'\mathclose|,A\righ)
		\end{tikzcd}
	\end{equation*}
	is a pushout in $\Dd_{\geqslant 0}(\IZ)$. By claim~\cref{claim:CofibresInD} in the proof of \cref{lem:HomologyEasier}, the desired pushout is given by the mapping cone
	\begin{equation*}
		P_*\coloneqq \cone\left(\C_*^\mathrm{sing}\lef(\left|X\right|,A\righ)\xrightarrow{(i,-j)}\C_*^\mathrm{sing}\lef(\mathopen|X'\mathclose|,A\righ)\oplus \C_*^\mathrm{sing}\lef(\left|Y\right|,A\righ)\right)\,.
	\end{equation*}
	By the universal property of mapping cones, any choice of chain homotopy that makes the diagram above commute induces a map of chain complexes $P_*\rightarrow\C_*^\mathrm{sing}\lef(\left|\ov Y'\right|,A\righ)$; since the diagram commutes on the nose, we may as well use the trivial chain homotopy. So to finish the proof, it remains to show that the $P_*\rightarrow \C_*^\mathrm{sing}\lef(\left|\ov Y'\right|,A\righ)$ is a quasi-isomorphism. To prove this, we look at long exact homology sequences:
	\begin{equation*}
		\begin{tikzcd}[column sep=1.5em]
			\dotsb \rar & \H_i\lef(\left|X\right|,A\righ)\rar\eqar[d]\drar[commutes] & \H_i\lef(\mathopen|X'\mathclose|,A\righ)\oplus \H_i\lef(\left|Y\right|,A\righ)\eqar[d]\rar\drar[commutes] & \H_i(P_*)\dar\rar["\partial"]\drar[commutes] & \H_{i-1}\lef(\left|X\right|,A\righ)\rar\eqar[d] & \dotsb\\
			\dotsb \rar & \H_i\lef(\left|X\right|,A\righ)\rar & \H_i\lef(\mathopen|X'\mathclose|,A\righ)\oplus \H_i\lef(\left|Y\right|,A\righ)\rar & \H_i\lef(\mathopen|\ov Y'\mathclose|,A\righ)\rar["\partial"] & \H_{i-1}\lef(\left|X\right|,A\righ)\rar &\dotsb
		\end{tikzcd}
	\end{equation*}
	The top row is the long exact homology sequence of a mapping cone. The bottom row is the Mayer--Vietoris sequence for $\mathopen|\ov Y'\mathclose|=\left|X'\right|\cup\left|Y\right|$ (here it's convenient that $X\rightarrow X'$ and $X\rightarrow Y$ are both cofibrations; also note that Mayer--Vietoris holds for every decomposition of a CW complex into subcomplexes, which is the version we're using here). So the five lemma proves that $P_*\rightarrow \C_*^\mathrm{sing}\lef(\mathopen|\ov Y'\mathclose|,A\righ)$ is indeed a quasi-isomorphism.
\end{proof}

\subsection{Presentable \texorpdfstring{$\infty$}{Infinity}-categories}\label{subsec:Presentable}
Suppose $\Cc$ is an $\infty$-category with all colimits and let $F\colon \Cc\rightarrow\Dd$ be a colimits-preserving functor of $\infty$-categories. Then the only thing preventing $F$ from having a right adjoint is set theory. Indeed, the values of a hypothetical right adjoint $G\colon \Dd\rightarrow\Cc$ would be given by $G(y)\simeq \colimit(\Cc_{/y}\rightarrow \Cc)$ for all $y\in\Dd$ (as we'll see in the proof of \cref{thm:AdjointFunctorTheorem}\cref{enum:AdjointFunctorTheoremLeft}), except that this colimit usually doesn't exist, even though $\Cc$ has all colimits. The problem ist that $\Cc_{/y}$ is usually not an \emph{essentially small} $\infty$-category in the sense of \cref{def:KappaSmall}\cref{enum:Small} below. So far, we have ignored these smallness issues. Still, \crefrange{subsec:Adjunctions}{subsec:KanExtensions} can be made set-theoretically sound. As a rule of thumb, whenever a limit or colimit is considered, the indexing $\infty$-category should be assumed small (or at least admit a final/cofinal functor from an essentially small $\infty$-category) and whenever we consider $\PSh(\Cc)$, we should assume that $\Cc$ is essentially small. The only time this gets hairy is in the proof of \cref{lem:KanExtensionFormula}, where we should allow $\Dd$ to be large, but also consider $\PSh(\Dd)$. Nevertheless, this can be fixed too (for example, by using universes, but I think you can even get away with less).

However, a more thorough analysis is needed to save our adjoint functor argument. In fact, $\infty$-categories $\Cc$ with all colimits are very seldomly essentially small, and so neither is $\Cc_{/y}$. However, often there exists an essentially small sub-$\infty$-category $\Cc_0\subseteq\Cc$ that generates $\Cc$ under colimits, and in this case one can replace $\Cc_{/y}$ by a cofinal essentially small sub-$\infty$-category, so that the required colimits do exist. The theory of accessible and presentable $\infty$-categories makes these ideas precise and allows to prove an incredibly useful \emph{adjoint functor theorem}.

In this subsection, we'll give the necessary definitions, which will take some time, and then make a beeline for Lurie's adjoint functor theorem (see \cref{thm:AdjointFunctorTheorem} below). Naturally, this subsection is very technical and it can be safely skipped on first reading. For a much more detailed exposition you should consult \cite[\S\href{https://people.math.harvard.edu/~lurie/papers/HTT.pdf\#page=332}{5}]{HTT}. 

%Next we'll put cardinality bounds on $\infty$-categories.
\begin{defi}\label{def:KappaSmall}
	Let $\kappa$ be a regular cardinal and let $\Cc$ be an $\infty$-category.
	\begin{alphanumerate}
		\item If $\kappa=\aleph_0$, then $\Cc$ is called \emph{essentially $\aleph_0$-small} if it is contained in the full sub-$\infty$-category of $\cat{Cat}_\infty$ generated under pushouts by $\emptyset$ and $\Delta^n$ for all $n\geqslant 0$. If $\kappa$ is uncountable, then $\Cc$ is called \emph{essentially $\kappa$-small} if $\pi_0\core \Cc$ as well as $\pi_0\Hom_\Cc(x,y)$ and $\pi_n(\Hom_\Cc(x,y),\alpha)$ are sets of cardinality $<\kappa$ for all $x,y\in\Cc$, all $\alpha\colon x\rightarrow y$, and all $n\geqslant 1$.\label{enum:KappaSmallCategory}%$\pi_0\Hom_{\cat{Cat}_\infty}(\Delta^n,\Cc)$ has cardinality $<\kappa$ for all $n\geqslant 0$. \label{enum:KappaSmallCategory}
		\item $\Cc$ is called \emph{essentially small} if it is essentially $\kappa$-small for some regular cardinal $\kappa$, and \emph{large} otherwise. $\Cc$ is called \emph{locally small} if $\Hom_\Cc(x,y)$ is essentially small for all $x,y\in\Cc$.\label{enum:Small}
		\item A colimit or a limit over a functor $F\colon \Ii\rightarrow \Cc$ is called \emph{$\kappa$-small} if $\Ii$ is essentially $\kappa$-small. Instead of \emph{$\aleph_0$-small}, we often say that a limit or colimit is \emph{finite}.\label{enum:KappaSmallLimit}
	\end{alphanumerate}
\end{defi}
\begin{rem}\label{rem:FunLocallySmall}
	If $\Cc$ is a small $\infty$-category and $\Dd$ is locally small, then $\Fun(\Cc,\Dd)$ is again locally small, as can be seen by \cref{cor:HomInFunctorCats}. In particular, $\PSh(\Cc)$ and the $\infty$-categories $\cat{Ind}_\kappa(\Cc)$ from \cref{con:Ind} below will be locally small.
\end{rem}
For practical applications, it will, unfortunately, be necessary to translate our nice model independent \cref{def:KappaSmall}\cref{enum:KappaSmallCategory} into the language of simplicial sets.
%\cref{def:KappaSmall}\cref{enum:KappaSmallCategory} is rather cumbersome to work with. We chose it because we're striving for model-independent notions, but 
\begin{lem}\label{lem:KappaSmall}
	Let $\kappa$ be an uncountable regular cardinal and let $\Cc$ be an $\infty$-category. Then the following are equivalent:
	\begin{alphanumerate}
		\item $\Cc$ is essentially $\kappa$-small.\label{enum:KappaSmallA}
		\item $\Cc$ is equivalent to a quasi-category with $<\kappa$ simplices across all dimensions.\label{enum:KappaSmallB}
		\item \!There exists a simplicial set $K$ with $<\kappa$ simplices across all dimensions and a Joyal equivalence $K\rightarrow \Cc$ \embrace{that is, a weak equivalence in the Joyal model structure from \cref{exm:JoyalModelStructure}}.\label{enum:KappaSmallC}
	\end{alphanumerate}
	Furthermore, if $K$ is a finite simplicial set \embrace{that is, a simplicial set with only finitely many non-degenerate simplices} and $K\rightarrow \Cc$ is a Joyal equivalence, then $\Cc$ is $\aleph_0$-small.
\end{lem}
\begin{proof}[Proof sketch]
	The implications \cref{enum:KappaSmallB} $\Rightarrow$ \cref{enum:KappaSmallA} and \cref{enum:KappaSmallB} $\Rightarrow$ \cref{enum:KappaSmallC} are trivial. For \cref{enum:KappaSmallC} $\Rightarrow$ \cref{enum:KappaSmallB} let $K\rightarrow \Cc'$ be the inner anodyne map into a quasi-category provided by the proof of \cref{lem:SmallObjectArgument}. Then $\Cc'$ has again $<\kappa$ simplices across all dimensions, because we're attaching $<\kappa$ new simplices countably many times. For the additional assertion, use induction on the dimension and write $K$ as a sequence of pushouts against $\coprod\partial\Delta^n\rightarrow \coprod\Delta^n$, where the disjoint union is finite. Replacing everything by quasi-categories and using model category fact~\cref{par:HomotopyPushout}, we conclude that $\Cc$ is contained in the full sub-$\infty$-category of $\cat{Cat}_\infty$ generated under pushouts by $\emptyset$ and $\Delta^n$ for all $n\geqslant 0$, as desired.
	
	It remains to show \cref{enum:KappaSmallA} $\Rightarrow$ \cref{enum:KappaSmallB}. We build a sub-simplicial set $\Cc'\subseteq\Cc$ as follows: Start with $\Cc'=\emptyset$. Choose $<\kappa$ representatives for every equivalence class in $\pi_0\core (\Cc)$ and add them to $\Cc'$. For all $x,y\in\Cc'$ and every equivalence class in $\pi_0\Hom_\Cc(x,y)$, we add a representative $\alpha\colon x\rightarrow y$. Furthermore, for every $n\geqslant 1$ and every class in $\pi_n(\Hom_\Cc(x,y),\alpha)$, we choose a representative $\Delta^n/\partial\Delta^n\rightarrow \Hom_\Cc(x,y)$ and add the simplices in the image of the corresponding map $\Delta^n/\partial\Delta^n\times\Delta^1\rightarrow\Cc$ to $\Cc'$. Then $\Cc'$ still has $<\kappa$ simplices. Mimicking the proof of \cref{lem:SmallObjectArgument}, we can add $<\kappa$ further simplices to $\Cc'$ to ensure that $\Cc'$ is a quasi-category. By construction, $\Cc'\rightarrow\Cc$ is essentially surjective and the map $\Hom_{\Cc'}(x,y)\rightarrow\Hom_\Cc(x,y)$ is a surjection on all $\pi_n$ for all $x,y\in\Cc'$. To make it injective, for every class in the kernel,  choose a homotopy $\Delta^n/\partial\Delta^n\times\Delta^1\rightarrow \Hom_\Cc(x,y)$ to $\const\alpha$. This homotopy corresponds to a map $(\Delta^n/\partial\Delta^n\times\Delta^1)\times\Delta^1\rightarrow \Cc$ and we add its image to $\Cc'$. Then we add $<\kappa$ simplices to make $\Cc'$ into a quasi-category again. Clearly, $\Cc'\rightarrow\Cc$ is still essentially surjective; furthermore, all elements in the previous kernel of $\pi_n(\Hom_{\Cc'}(x,y),\alpha)\rightarrow\pi_n(\Hom_\Cc(x,y),\alpha)$ have been killed now. But there could be new ones. So we simply repeat this process countably many times. Then $\Cc'\rightarrow \Cc$ is fully faithful too and thus an equivalence by \cref{thm:EquivalenceFullyFaithfulEssentiallySurjective}.%
	%
	%By adding another $<\kappa$ simplices to $\Cc'$, we can ensure that for all $x,y\in\Cc'$, every equivalence class in $\pi_0\F(\partial\Delta^n,\Hom_\Cc(x,y))$ and $\pi_0\F(\Delta^n,\Hom_\Cc(x,y))$ has a representative in $\Cc'$. Indeed, this can be done by applying the above observation to $K=(\partial\Delta^n\times\Delta^1)/(\partial\Delta^n\times\{0,1\})$ and $K=(\Delta^n\times\Delta^1)/(\Delta^n\times\{0,1\})$. Mimicking the proof of \cref{lem:SmallObjectArgument}, we can add $<\kappa$ further simplices to $\Cc'$ to ensure that $\Cc'$ is a quasi-category. Then $\Cc'$ has $<\kappa$ simplices across all dimensions. Furthermore, $\Cc'\rightarrow\Cc$ is essentially surjective by construction and for all $x,y\in\Cc'$ the map $\Hom_{\Cc'}(x,y)\rightarrow\Hom_\Cc(x,y)$ is a bijection on $\pi_0$ and an isomorphism on $\pi_n$ for all $n\geqslant 1$ and all basepoints. Hence $\Cc'\rightarrow\Cc$ is fully faithful too and thus an equivalence by \cref{thm:EquivalenceFullyFaithfulEssentiallySurjective}.%First, a simple induction over the number of simplices shows that $\pi_0\core\F(K,\Cc)$ has cardinality $<\kappa$ for every finite simplicial set $K$. For the inductive step, write $K\cong K'\cup_{\partial\Delta^n}\Delta^n$, so that $\F(K,\Cc)\cong \F(K',\Cc)\times_{\F(\partial\Delta^n,\Cc)}\F(\Delta^n,\Cc)$. Let $S\subseteq \F(K',\Cc)_0$ be a set of $<\kappa$ representatives for every equivalence class in $\pi_0\core\F(K',\Cc)$ and define $T\subseteq \F(\Delta^n,\Cc)_0$ similarly. For every $\sigma\in S$ and $\tau\in T$ such that the images of $\sigma$ and $\tau$ in $\F(\partial\Delta^n,\Cc)$ are equivalent, we can find $\tau'\in \F(\Delta^n,\Cc)$ such that $\tau\simeq \tau'$ and the images of  $\sigma$ and $\tau'$ in $\F(\partial\Delta^n,\Cc)$ are equal. Indeed, claim~\cref{claim:Pullback} from the proof of \cref{thm:EquivalenceFullyFaithfulEssentiallySurjective} allows us to lift equivalences. Then $\{(\sigma,\tau')\}$ is a set of $<\kappa$ representatives for every equivalence class in $\pi_0\core\F(K,\Cc)$, finishing the induction.
\end{proof}
\begin{rem}\label{rem:KappaSmallClosedUnderPushouts}
	If $\kappa$ is an uncountable regular cardinal, then pushouts or pullbacks of essentially $\kappa$-small $\infty$-categories are essentially $\kappa$-small again. Indeed, this follows from \cref{lem:KappaSmall}\cref{enum:KappaSmallB} together with model category facts~\cref{par:HomotopyPullback} and~\cref{par:HomotopyPushout} and a cardinality bound on \cref{lem:SmallObjectArgument}: A functor between quasi-categories with $<\kappa$ simplices across all dimensions can be factored into a cofibration followed by a trivial fibration or into a Joyal equivalence followed by an isofibration in such a way that the new quasi-category in the middle has again $<\kappa$ simplices across all dimensions. Combining this observation with \cref{lem:KappaSmallColimits} below, we see that the full sub-$\infty$-category $\cat{Cat}_\infty^{<\kappa}$ of essentially $\kappa$-small $\infty$-categories is closed under $\kappa$-small limits and colimits.
	
	In the case $\kappa=\aleph_0$ it's obvious that $\aleph_0$-small $\infty$-categories are closed under pushouts and thus under finite colimits by \cref{lem:KappaSmallColimits} below. The same can be shown for finite products, but I don't know if it works for pullbacks too.
\end{rem}
\begin{lem}\label{lem:KappaSmallColimits}
	Let $\kappa$ be a regular cardinal. An $\infty$-category $\Cc$ has all $\kappa$-small colimits if and only if $\Cc$ has pushouts and $\kappa$-small coproducts. A functor $F\colon \Cc\rightarrow\Dd$ of $\infty$-categories preserves colimits if and only if it preserves pushouts and $\kappa$-small coproducts. A dual assertion holds for limits.
\end{lem}
\begin{proof}[Proof sketch]
	Repeat the proof of \cref{lem:ColimitsIffCoproductsAndPushouts} and use \cref{lem:KappaSmall} together with model category fact~\cref{par:HomotopyPushout} to see that pushouts of $\kappa$-small $\infty$-categories are still $\kappa$-small.
\end{proof}
Our next goal is to study filtered colimits in the $\infty$-setting. 
\begin{con}\label{con:ConeCategory}
	Let $\Ii$ be an $\infty$-category. We define the \emph{cone $\Ii^\triangleleft$ over $\Ii$} and the \emph{cocone $\Ii^\triangleright$ under $\Ii$} as the following pushouts in $\cat{Cat}_\infty$:
	\begin{equation*}
		\begin{tikzcd}
			\Ii\times\{0\}\rar\dar\drar[pushout] & \Ii\times\Delta^1\dar\\
			*\rar & \Ii^\triangleleft
		\end{tikzcd}\quad\text{and}\quad
		\begin{tikzcd}
			\Ii\times\{1\}\rar\dar\drar[pushout] &\Ii\times\Delta^1\dar\\
			*\rar & \Ii^\triangleright
		\end{tikzcd}
	\end{equation*}
	It's tempting to use the procedure from model category fact~\cref{par:HomotopyPushout} to compute these pushouts explicitly, but this is a little tricky. Steps \cref{enum:PushoutStepA}, \cref{enum:PushoutStepB}, and \cref{enum:PushoutStepC} are easy though: $\Ii\times \{0\}\rightarrow \Ii\times\Delta^1$ and $\Ii\times\{1\}\rightarrow \Ii\times\Delta^1$ are already cofibrations, so we can simply take the pushout on the nose. The tricky step, however, is \cref{enum:PushoutStepD}, in which one has to replace the pushout in   $\cat{sSet}$ by a quasi-category. One can show that the \emph{joins} $\{0\}\star\Ii$ and $\Ii\star\{1\}$, which we didn't introduce, are such replacements; see \cite[Proposition~\HTTthm{4.1.2.1}]{HTT} or \cite[Proposition~2.5.19]{Land}. We won't need this explicit description and work with the abstract construction exclusively.
	
	Note that $*\rightarrow \Ii^\triangleleft$ is an initial object and $*\rightarrow \Ii^\triangleright$ is a terminal object. This is obvious in the simplicial models, but there's also a model-independent argument: We must show that $*\rightarrow\Ii^\triangleleft$ is left adjoint to the unique functor $\Ii^\triangleleft\rightarrow *$. This can be done via \cref{lem:TriangleIdentities} by constructing the unit and counit by hand. The unit is clear, as there are not that many functors from $*$ to itself (in fact, there's only one). For the counit, we must construct a natural transformation $c\colon \Ii^\triangleleft\times\Delta^1\rightarrow\Ii^\triangleleft$ from $\const *$ to $\id_{\Ii^\triangleleft}$. Using that $-\times\Delta^1\colon \cat{Cat}_\infty\rightarrow \cat{Cat}_\infty$ commutes with pushouts (since $\Fun(\Delta^1,-)$ is a right adjoint by \cref{exm:Adjunctions}\cref{enum:Currying}), this boils down to constructing a natural transformation $\Delta^1\times\Delta^1\rightarrow\Delta^1$ from $\const 0$ to $\id_{\Delta^1}$, which is easy. In the same way, verifying the triangle identities reduces to a question about $\Delta^1$. In particular, we deduce $\left|\Ii^\triangleleft\right|\simeq *$ and $\left|\Ii^\triangleright\right|\simeq *$, as $\infty$-categories with an initial or terminal object are always weakly constractible.
	
	As in ordinary category theory, cones and cocones are closely related to limits and colimits, respectively. Concretely, if $F\colon \Ii\rightarrow \Cc$ is a functor and $y\in\Cc$ is an object, then an easy calculation using \cref{cor:HomPreservesLimits} shows
	\begin{equation*}
		\{F\}\times_{\Hom_{\cat{Cat}_\infty}(\Ii,\Cc)}\Hom_{\cat{Cat}_\infty}\!\left(\Ii^\triangleright,\Cc\right)\times_{\Hom_{\cat{Cat}_\infty}(*,\Cc)}\{y\}\simeq \Hom_{\Fun(\Ii,\Cc)}\!\left(F,\const y\right)\,.
	\end{equation*}
	Informally, an extension of $F$ to a functor $F^\triangleright \colon \Ii^\triangleright\rightarrow \Cc$ that sends the tip $*\in \Cc^\triangleright$ to $y$ is the same as a natural transformation $F\Rightarrow \const y$. If $F$ admits a colimit, such a natural transformation is the same as a morphsm $\colimit_{i\in\Ii}F(i)\rightarrow y$, and the right-hand side above is equivalent to $\Hom_\Cc(\colimit_{i\in\Ii}F(i),y)$.
\end{con}
%	\begin{con}\label{con:ConeCategory}
	%		For $\infty$-categories $\Cc$ and $\Dd$, we define the \emph{join $\Cc\star\Dd$} as the pushout
	%		\begin{equation*}
		%			\begin{tikzcd}
			%				\Cc\times\{0,1\}\times\Dd\rar\dar\drar[pushout] & \Cc\times\Delta^1\times\Dd\dar\\
			%				\Cc\sqcup\Dd\rar & \Cc\star\Dd
			%			\end{tikzcd}
		%		\end{equation*}
	%		in $\cat{Cat}_\infty$. Here the left vertical arrow is the coproduct of the projections $\Cc\times\{0\}\times\Dd\rightarrow \Cc$ and $\Cc\times\{1\}\times\Dd\rightarrow\Dd$. It can be shown that the eponymous simplicial set (which we didn't introduce) provides an explicit simplicial model for $\Cc\star\Dd$; see \cite[Proposition~\HTTthm{4.1.2.1}]{HTT} or \cite[Proposition~2.5.19]{Land}. It turns out that $\star$ is associative in the sense that there is a canonical equivalence $(\Cc\star\Dd)\star\Ee\simeq\Cc\star(\Dd\star\Ee)$. To prove this, one can use the colimit manipulation tricks from \cref{lem:ColimitManipulations} or verify that we even get an isomorphism of simplicial sets. Either way, it takes some fiddling and we'll skip the argument.
	%		
	%		For an $\infty$-category $\Cc$, we define the \emph{cone $\infty$-category $\Cc^\triangleleft$} as the join $\Cc^\triangleleft\coloneqq *\star\Cc$ and the \emph{cocone $\infty$-category $\Cc^\triangleright$} as join $\Cc^\triangleright \coloneqq \Cc\star *$. Explicitly, we have the following pushouts in $\cat{Cat}_\infty$:
	%		\begin{equation*}
		%			\begin{tikzcd}
			%				\Cc\times\{0\}\rar\dar\drar[pushout] & \Cc\times\Delta^1\dar\\
			%				*\rar & \Cc^\triangleleft
			%			\end{tikzcd}\quad\text{and}\quad
		%			\begin{tikzcd}
			%				\Cc\times\{1\}\rar\dar\drar[pushout] & \Cc\times\Delta^1\dar\\
			%				*\rar & \Cc^\triangleright
			%			\end{tikzcd}
		%		\end{equation*}
	%	\end{con}
%	\begin{lem}\label{lem:Cocone}
	%		Let $(\Cat_\infty)_{*/}^\mathrm{term}\subseteq(\cat{Cat}_\infty)_{*/}$ be the full sub-$\infty$-category spanned by those pointed $\infty$-categories $(\Dd,y)$ for which $y\in\Dd$ is a terminal object. Then the cocone functor $(-)^\triangleright\colon \cat{Cat}_\infty\rightarrow(\Cat_\infty)_{*/}^\mathrm{term}$ is left adjoint to the forgetful functor $(\Cat_\infty)_{*/}^\mathrm{term}\rightarrow\cat{Cat}_\infty$. A dual assertion holds for the cone $(-)^\triangleleft$.
	%	\end{lem}
%	\begin{proof}[Proof sketch]
	%		To show that $\Cc^\triangleright$ has a terminal object, we have to show that $\Cc^\triangleright\shortdoublelrmorphism *$ is an adjunction. This can be done via \cref{lem:TriangleIdentities} by constructing the unit and counit by hand. The counit is clear; for the unit, we must construct a natural transformation $u\colon \Cc^\triangleright\times\Delta^1\rightarrow\Cc^\triangleright$ from $\id_{\Cc^\triangleright}$ to $\const *$. Using that $-\times\Delta^1$ commutes with pushouts (by \cref{lem:AdjointsPreserveColimits}, since $\Fun(\Delta^1,-)$ is a right adjoint), this boils down to constructing a natural transformation $\Delta^1\times\Delta^1\rightarrow\Delta^1$ from $\id_{\Delta^1}$ to $\const 1$, which is easy.
	%		
	%		To show that $(-)^\triangleright$ is left adjoint to the forgetful functor, let $(\Dd,y)$ be a pointed $\infty$-category such that $y$ is terminal. Then $\Hom_{(\cat{Cat}_\infty)_{*/}}((\Cc^\triangleright,*),(\Dd,y))\simeq \Hom_{\cat{Cat}_\infty}(\Cc^\triangleright,\Dd)\times_{\Hom_{\cat{Cat}_\infty}(*,\Dd)}\{y\}$ by \cref{cor:HomInSliceCategories}. Using \cref{cor:HomPreservesColimits} on the pushout from \cref{con:ConeCategory}, this can be rewritten as
	%		\begin{multline*}
		%			\left(\Hom_{\cat{Cat}_\infty}\bigl(\Cc\times\Delta^1,\Dd\bigr)\times_{\Hom_{\cat{Cat}_\infty}(\Cc\times\{0\},\Dd)}\Hom_{\cat{Cat}_\infty}(*,\Dd)\right)\times_{\Hom_{\cat{Cat}_\infty}(*,\Dd)}\{y\}\\
		%			\begin{aligned}
			%				&\simeq \Hom_{\cat{Cat}_\infty}\lef(\Cc,\Ar(\Dd)\righ)\times_{t,\Hom_{\cat{Cat}_\infty}(\Cc,\Dd)}\Hom_{\cat{Cat}_\infty}\lef(\Cc,\{y\}\righ)\\
			%				&\simeq\Hom_{\cat{Cat}_\infty}\lef(\Cc,\Ar(\Dd)\times_{t,\Dd}\{y\}\righ)\,.
			%			\end{aligned}
		%		\end{multline*}
	%		In the first step we simplify the pullback, use that $-\times\Delta^1$ is left adjoint to $\Fun(\Delta^1,-)\simeq \Ar(-)$, and that $\{y\}\simeq\Hom_{\cat{Cat}_\infty}(\Cc,\{y\})$. In the second step we use \cref{cor:HomPreservesColimits} again. Now the right-hand side agrees with $\Hom_{\cat{Cat}_\infty}(\Cc,\Dd_{/y})\simeq \Hom_{\cat{Cat}_\infty}(\Cc,\Dd)$ since $y\in\Dd$ is terminal. This shows $\Hom_{(\cat{Cat}_\infty)_{*/}}((\Cc^\triangleright,*),(\Dd,y))\simeq\Hom_{\cat{Cat}_\infty}(\Cc,\Dd)$. Since all manipulations were functorial both in $\Cc$ and in $(\Dd,y)$, we're done.
	%	\end{proof}
\begin{defi}\label{def:KappaFiltered}
	Let $\kappa$ be a regular cardinal and let $\Jj$, $\Cc$ be $\infty$-categories.
	\begin{alphanumerate}
		\item $\Jj$ is called \emph{$\kappa$-filtered} if every functor $\Ii\rightarrow\Jj$ from an essentially $\kappa$-small $\infty$-category extends to a functor $\Ii^\triangleright\rightarrow\Jj$ from the cocone under $\Cc$, or in other words, if the restriction $\Fun(\Ii^\triangleright,\Jj)\rightarrow\Fun(\Ii,\Jj)$ is essentially surjective. In the case $\kappa=\aleph_0$, we usually just say $\Jj$ is \emph{filtered}.\label{enum:KappaFilteredCategory}
		\item A colimit over a functor $F\colon \Jj\rightarrow\Cc$ is called \emph{$\kappa$-filtered} if $\Jj$ is $\kappa$-filtered, and \emph{filtered} if $\Jj$ is filtered.\label{enum:KappaFilteredColimit}
		\item An object $x\in\Cc$ is called \emph{$\kappa$-compact} or \emph{compact} if $\Hom_\Cc(x,-)\colon \Cc\rightarrow\cat{An}$ commutes with $\kappa$-filtered or filtered colimits, respectively.\label{enum:KappaCompact} 
	\end{alphanumerate}
\end{defi}
\begin{rem}\label{rem:LuriesFilteredness}
	We'll explain why Lurie's definition of $\kappa$-filteredness in \cite[Definition~\HTTthm{5.3.1.7}]{HTT} is equivalent to ours. Let $\Jj$ be a $\kappa$-filtered quasi-category as in \cref{def:KappaFiltered}\cref{enum:KappaFilteredCategory}. Furthermore, let $\Ii$ be an essentially $\kappa$-small quasi-category and choose the simplicial model $\Ii\star\{1\}$ for $\Ii^\triangleright$ (as Lurie does). Then any functor $\Ii\rightarrow\Jj$ can not only be extended to $\Ii^\triangleright\rightarrow\Jj$ up to equivalence, but even on the nose. The reason is that $\Ii\rightarrow\Ii^\triangleright$ is a cofibration and thus $\core\F(\Ii^\triangleright,\Jj)\rightarrow \core\F(\Ii,\Jj)$ has lifting against $\{0\}\rightarrow\Delta^1$ by claim~\cref{claim:Pullback} in the proof of \cref{thm:EquivalenceFullyFaithfulEssentiallySurjective}. Then \cref{lem:KappaSmall} easily implies that Lurie's definition of $\kappa$-filteredness is equivalent to ours in the case where $\kappa$ is uncountable.
	
	If $\kappa=\aleph_0$, then \cref{lem:KappaSmall} shows that any filtered $\infty$-category $\Jj$ in the sense of \cref{def:KappaFiltered}\cref{enum:KappaFilteredCategory} is also filtered in Lurie's sense. The converse is true as well, but not as obvious (at least to me), since I don't know if the converse of the additional assertion in \cref{lem:KappaSmall} is true (I'd guess it's not). So here's a different argument: If $\Jj$ is filtered in Lurie's sense, then $\colimit\colon \Fun(\Jj,\cat{An})\rightarrow\cat{An}$ preserves finite limits (by \cite[Proposition~\HTTthm{5.3.3.3}]{HTT} or by observing that the proof of \cref{lem:FilteredColimitsPreserveFiniteLimits} still goes through). Hence \cref{lem:FilteredColimitsPreserveFiniteLimits}, which we'll prove next, implies that $\Jj$ is filtered in our sense. 
\end{rem}
\begin{lem}\label{lem:FilteredColimitsPreserveFiniteLimits}
	Let $\kappa$ be a regular cardinal. Then an $\infty$-category is $\kappa$-filtered if and only if the functor $\colimit\colon \Fun(\Jj,\cat{An})\rightarrow\cat{An}$ preserves $\kappa$-small limits.
\end{lem}
Before we can prove \cref{lem:FilteredColimitsPreserveFiniteLimits}, we need to send four more lemmas in advance.%First, we need a general cofinality result for $\kappa$-filtered $\infty$-categories.
\begin{lem}\label{lem:FilteredCofinal}
	Let $\kappa$ be a regular cardinal and let $\Jj$ be a $\kappa$-filtered $\infty$-category. Then $\left|\Jj\right|\simeq *$. Furthermore, for every $j\in\Jj$ the slice $\Jj_{j/}$ is $\kappa$-filtered again and $\Jj_{j/}\rightarrow \Jj$ is cofinal.
\end{lem}
\begin{proof}[Proof sketch]
	To see $\left|\Jj\right|\simeq *$, unfortunately, we need to use simplicial methods. It's enough to show that every map $\sigma\colon \partial\Delta^n\rightarrow\left|\Jj\right|$ is nullhomotopic, because then the same argument as in the proof of \cref{lem:ContractibleKanComplex} shows that $\left|\Jj\right|\rightarrow*$ is a trivial fibration. We'll show that for every $\sigma$ there is a functor $\alpha\colon\Ii\rightarrow \Jj$ from $\aleph_0$-small $\infty$-category $\Ii$ such that $\sigma$ factors through $\left|\alpha\right|\colon\left|\Ii\right|\rightarrow \left|\Jj\right|$. This will be enough since then $\sigma$ also factors through $*\simeq \left|\Ii^\triangleright\right|\rightarrow\left|\Jj\right|$ by filteredness of $\Jj$. To construct $\alpha$, recall that $\Jj\rightarrow\left|\Jj\right|$ can be constructed as an anodyne map into a Kan complex via \cref{lem:SmallObjectArgument}. Accordingly, as simplicial sets, $\left|\Jj\right|\cong \colimit_{i\geqslant 0}\Jj_i$, where $\Jj_0=\Jj$ and $\Jj_{i+1}$ is obtained from $\Jj_i$ by attaching solutions to horn filling problems. All the finitely many simplices in the image of $\sigma\colon\partial\Delta^n\rightarrow\left|\Jj\right|$ must already be contained in $\Jj$ or occur in $\Jj_i$ as a solution to some horn filling problem. If the latter is the case, all the finitely many simplices involved in that horn filling problem must already occur in $\Jj$ or in some $\Jj_k$ for $k<i$. Continuing in this way, we can trace back $\sigma$ to a finite number of simplices in $\Jj$. Completing these finitely many simplices to a sub-quasi-category $\Ii\subseteq\Jj$ as in the proof of \cref{lem:KappaSmall} yields the desired functor $\alpha\colon \Ii\rightarrow\Jj$. %use not only simplicial methods, but \cref{thm:SimplicialApproximation} (please tell me if you know a better argument): Let $X$ be the CW complex obtained as the geometric realisation of the Kan complex $\left|\Jj\right|$. Then $X$ is homotopy equivalent to the geometric realisation of $\Jj$, because $\Jj\rightarrow \left|\Jj\right|$ is anodyne by \cref{con:Localisation}. So it suffices to prove that every map $\alpha\colon \partial D^n\rightarrow X$ from the boundary of a topological $n$-disk extends all of $D^n$, up to homotopy. This follows from simplicial approximation in the form of \cite[Theorem~\href{https://pi.math.cornell.edu/~hatcher/AT/AT.pdf\#page=186}{2C.1}]{Hatcher}: Write $\partial D^n\cong \left|\partial \Delta^n\right|$; up to homotopy, we may assume that $\alpha$ is the geometric realisation of some map $\alpha_0\colon \operatorname{sd}^m(\partial\Delta^n)\rightarrow \Jj$ from some barycentric subdivision of $\partial\Delta^n$. Since $\Jj$ is filtered, $\alpha_0$ extends to $\alpha_0^\triangleright\colon\operatorname{sd}^m(\partial\Delta^n)^\triangleright\rightarrow\Jj$. Since $\left|\operatorname{sd}^m(\partial\Delta^n)^\triangleright\right|\simeq D^n$, we've proved that $\alpha$ extends to all of $D^n$ up to homotopy. This proves $\left|\Jj\right|\simeq *$.
	
	For the other assertions, let $\Ii\rightarrow\Jj_{j/}$ be a map from an essentially $\kappa$-small $\infty$-category. By unravelling the respective universal properties, such a map is equivalent to a map $\Ii^\triangleleft\rightarrow \Jj$ sending the tip of the cone to $j$. Since $\Ii^\triangleleft$ is still essentially $\kappa$-small, we get an extension $(\Ii^\triangleleft)^\triangleright\rightarrow\Jj$. Since $(\Ii^\triangleleft)^\triangleright\simeq (\Ii^\triangleright)^\triangleleft$, this defines a map $\Ii^\triangleright\rightarrow\Jj_{j/}$, proving that $\Jj_{j/}$ is $\kappa$-filtered. An analogous argument shows that $\Jj_{j/}\times_\Jj\Jj_{j'/}$ is $\kappa$-filtered for every $j'\in \Jj$. Hence $\mathopen|\Jj_{j/}\times_\Jj\Jj_{j'/}\mathclose|\simeq *$ by the first part. Thus $\Jj_{j/}\rightarrow\Jj$ is cofinal by \cref{thm:JoyalsQuillenA}\cref{enum:WeaklyContractible}.
\end{proof}
%Next, we need a result about limits and colimits in slice $\infty$-categories.
\begin{lem}\label{lem:ColimitsInSliceCategory}
	Let $\Dd$ be an $\infty$-category and $y\in\Dd$ an object.
	\begin{alphanumerate}
		\item $\Dd_{y/}\rightarrow\Dd$ preserves and detects arbitrary limits. That is, a diagram $\alpha\colon\Ii\rightarrow\Dd_{y/}$ has a limit in $\Dd_{y/}$ if and only if the underlying diagram $\ov\alpha\colon\Ii\rightarrow \Dd_{y/}\rightarrow\Dd$ has a limit in $\Dd$, in which case these limits coincide in $\Dd$.\label{enum:LimitsInSlice}
		\item $\Dd_{y/}\rightarrow\Dd$ preserves and detects $\Ii$-shaped colimits if $\left|\Ii\right|\simeq *$. In particular, this applies to pushouts \embrace{since $\left|\Lambda_0^2\right|\simeq*$} and filtered colimits \embrace{by \cref{lem:FilteredCofinal}}.\label{enum:ColimitsInSlice}
		\item In general, let $\alpha\colon \Ii\rightarrow \Dd_{y/}$ be a diagram in $\Dd_{y/}$ and let $\ov\alpha\colon \Ii\rightarrow \Dd_{y/}\rightarrow\Dd$ be the underlying diagram in $\Dd$. If the colimits $\colimit_{i\in\Ii}\ov\alpha(i)$ and $\colimit_{i\in\Ii}y$ as well as the pushout\label{enum:ColimitsInSliceGeneral}
		\begin{equation*}
			\begin{tikzcd}
				\colimit_{i\in\Ii}y\dar\rar\drar[pushout] & \colimit_{i\in\Ii}\ov\alpha(i)\dar\\
				y\rar & c
			\end{tikzcd}
		\end{equation*}
		exist in $\Dd$, then $(y\rightarrow c)\in\Dd_{y/}$ is the colimit of $\alpha\colon \Ii\rightarrow \Dd_{y/}$.
	\end{alphanumerate}
	%Dual assertions hold for the other slice projection $\Dd_{/y}\rightarrow\Dd$.
\end{lem}
\begin{proof}[Proof sketch]
	For \cref{enum:LimitsInSlice}, first consider the case where $\Dd$ has all limits. The functor $\alpha\colon\Ii\rightarrow\Dd_{y/}$ defines a natural transformation $\const y\Rightarrow \ov\alpha$, hence a morphism $y\rightarrow \lim_{i\in\Ii}\ov\alpha(i)$. We claim that $(y\rightarrow \limit_{i\in\Ii}\ov\alpha(i))$ is the limit of $\alpha$. Indeed, using \cref{cor:HomInSliceCategories} and the fact that limits commute with limits by the dual of \cref{lem:ColimitManipulations}, we immediately verify the condition from \cref{cor:HomPreservesColimits}. This concludes the case where $\Dd$ has all limits. The general case can be reduced to this special case by considering a fully faithful limits-preserving functor $i\colon \Dd\rightarrow\Dd'$ into an $\infty$-category with all limits; for example, $\Yo_\Dd\colon \Dd\rightarrow\Fun(\Dd^\op,\cat{An})$ does it by \cref{cor:HomPreservesLimits}.
	
	Assertion \cref{enum:ColimitsInSliceGeneral} is another straightforward application of \cref{cor:HomInSliceCategories}, combined with \cref{cor:HomPreservesColimits} and the fact that pullbacks can be commuted with arbitrary limits by the dual of \cref{lem:ColimitManipulations}. This proves \cref{enum:ColimitsInSliceGeneral}. To prove \cref{enum:ColimitsInSlice}, first assume that $\Dd$ has all colimits. Then the assumptions from \cref{enum:ColimitsInSliceGeneral} are satisfied and $\colimit_{i\in\Ii}\alpha(i)$ exists. If $\left|\Ii\right|\simeq *$, then \cref{lem:ContractibleColimit} below implies that the canonical morphism $\colimit_{i\in\Ii}y\rightarrow y$ is an equivalence. Hence the pushout from \cref{enum:ColimitsInSliceGeneral} becomes an equivalence $\colimit_{i\in\Ii}\ov\alpha(i)\simeq c$. This proves \cref{enum:ColimitsInSlice} in the case where $\Dd$ has all colimits. For the general case, choose a fully faithful colimits-preserving functor $i\colon \Dd\rightarrow\Dd'$ into an $\infty$-category $\Dd$ with all colimits; for example, the mutilated Yoneda embedding $(\Yo_{\Dd^\op})^\op\colon (\Dd^\op)^\op\rightarrow\Fun(\Dd,\cat{An})^\op$ does it by \cref{cor:HomPreservesLimits}.
	%
	%, we use a general fact: If $L\colon \Cc\shortdoublelrmorphism\Dd\noloc R$ is an adjunction of $\infty$-categories and the counit $c_y\colon LR(y)\rightarrow y$ is an equivalence, then we get an induced adjunction $L\colon \Cc_{R(y)/}\shortdoublelrmorphism \Dd_{y/}\noloc R$ on slice $\infty$-categories. The proof is another straightforward application of \cref{cor:HomInSliceCategories}.
	%
	%First assume that $\Dd$ has $\Ii$-shaped colimits. Applying the general fact to the adjunction $\colimit_\Ii\colon \Fun(\Ii,\Dd)\shortdoublelrmorphism \Dd\noloc \const$ and using $\Fun(\Ii,\Dd_{y/})\simeq \Fun(\Ii,\Dd)_{\const y/}$, we see that it suffices to check $\colimit_\Ii\const y\simeq y$. This follows from \cref{lem:ContractibleColimit} below. This settles the case where $\Dd$ has $\Ii$-shaped colimits. For the general case, choose a fully faithful colimits-preserving functor $i\colon \Dd\rightarrow\Dd'$ into an $\infty$-category $\Dd$ with all colimits; for example, the mutilated Yoneda embedding $(\Yo_{\Dd^\op})^\op\colon (\Dd^\op)^\op\rightarrow\Fun(\Dd,\cat{An})^\op$ does it by \cref{cor:HomPreservesLimits}.
\end{proof}
\begin{lem}\label{lem:ContractibleColimit}
	Let $\Dd$ be an $\infty$-category, $y\in\Dd$ an object, and $\Ii$ be an $\infty$-category satisfying $\left|\Ii\right|\simeq *$. Then $\colimit_{i\in\Ii} y\simeq y$; in particular, this colimit always exists.
\end{lem}
\begin{proof}
	Clearly, $\const y\colon \Ii\rightarrow \Dd$ factors through $\Ii\rightarrow\left|\Ii\right|$. This functor is cofinal by \cref{exm:Cofinal}\cref{enum:LocalisationsCofinal}, and since $ \left|\Ii\right|\simeq *$, it follows that the colimit is indeed given by $y$. 
\end{proof}
The following lemma is the crucial step in the proof of \cref{lem:FilteredColimitsPreserveFiniteLimits}:
\begin{lem}\label{lem:HomotopyGroupsFilteredColimits}
	The functor $\pi_0\colon \cat{An}\rightarrow \cat{Set}$ commutes with products and all colimits. The functors $\pi_1\colon \cat{An}_{*/}\rightarrow\cat{Grp}$, and $\pi_n\colon\cat{An}_{*/}\rightarrow\cat{Ab}$ for all $n\geqslant 2$ commute with products and filtered colimits.
\end{lem}
\begin{proof}[Proof sketch]
	Since $\pi_0\colon\cat{An}\rightarrow\cat{Set}$ is left adjoint to the inclusion $\cat{Set}\subseteq \cat{An}$, \cref{lem:AdjointsPreserveColimits} shows that $\pi_0$ preserves colimits. By a simple inspection $\pi_0$ also preserves products. This immediately implies that $\pi_n$ preserves products for all $n\geqslant 1$, since $\pi_n(X,x)\cong \pi_0\Hom_{\cat{An}_{*/}}((S^n,*),(X,x))$ and $\Hom_{\cat{An}_{*/}}((S^n,*),-)\colon \cat{An}_{*/}\rightarrow \cat{An}$ preserves limits by \cref{cor:HomPreservesLimits}.
	
	The assertion about $\pi_n$ needs simplicial methods (and two black boxes), unfortunately. Let $\Jj$ be a filtered $\infty$-category. For every ordinary category $\Cc$, we have $\Fun(\Jj,\Cc)\simeq \Fun(\operatorname{ho}(\Jj),\Cc)$ by \cref{lem:SimplicialHoNerveAdjunction}, and so $\Jj$-shaped colimits in $\Cc$ agree with $\operatorname{ho}(\Jj)$-shaped colimits. It's straightforward to see that $\operatorname{ho}(\Jj)$ is filtered in the usual sense. So for filtered colimits in an ordinary category we can replace the indexing diagram by an ordinary filtered category. But a stronger assertion is true, which we'll need later:
	\begin{alphanumerate}\itshape
		\item[\blacksquare_1] For every filtered $\infty$-category there exists a directed partially ordered set $J$ and a cofinal functor $J\rightarrow \Jj$.\label{blackbox:Cofinal}
	\end{alphanumerate}
	For a proof of \cref{blackbox:Cofinal} see \cite[Proposition~\HTTthm{5.3.1.18}]{HTT} or \cite[Tag~\href{https://kerodon.net/tag/02QA}{02QA}]{Kerodon} (the Kerodon proof is relatively short and only uses methods that we have already available).
	
	Next, observe that $\pi_1\colon \cat{Kan}_{*/}\rightarrow\cat{Grp}$ and $\pi_n\colon \cat{Kan}_{*/}\rightarrow \cat{Ab}$ for $n\geqslant 2$ commute with filtered colimits in the ordinary category $\cat{Kan}_{*/}$. This follows essentially from the fact that $\square^n$ and $\partial\square^n$ are finite simplicial sets, using an argument as near the end of the proof of \cref{lem:SmallObjectArgument}. It follows that for every filtered $\infty$-category $\Jj$, the functor $\colimit\colon \Fun(\Jj,\cat{Kan})\rightarrow\cat{Kan}$ sends pointwise homotopy equivalences to homotopy equivalences. Indeed, let $X_{(-)}\Rightarrow X'_{(-)}$ be a natural transformation in $\Fun(\Jj,\cat{Kan})$ such that $X_j\rightarrow X'_j$ is a homotopy equivalence for all $j\in\Jj$. We can check on homotopy groups whether $\colimit_{j\in\Jj}X_j\rightarrow\colimit_{j\in\Jj}X_j$ is a homotopy equivalence. By the argument above, we get a bijection on $\pi_0$. Now let $x\in \pi_0(\colimit_{j\in\Jj}X_j)$ be a point. Since $\pi_0$ commutes with colimits, we must have $x\in \pi_0(X_{j_0})$ for some $j_0\in \Jj$. By \cref{lem:FilteredCofinal}, we may replace $\Jj$ by $\Jj_{j_0/}$, so we may assume $j_0$ is initial in $\Jj$. Then $\{x\}\rightarrow X_{j_0}\rightarrow X_j$ for all $j\in \Jj$ turns $X_{(-)}$ into a functor $X_{(-)}\colon \Jj\rightarrow\cat{Kan}_{*/}$. The same works for $X'_{(-)}$. Since $X_{(-)}\Rightarrow X'_{(-)}$ is a pointwise homotopy equivalence and $\pi_n$ commutes with filtered colimits in $\cat{Kan}_{*/}$, we conclude that $\pi_n(\colimit_{j\in\Jj}X_j,x)\cong \pi_n(\colimit_{j\in\Jj}X_j',x)$. This finishes the proof that $\colimit\colon \Fun(\Jj,\cat{Kan})\rightarrow\cat{Kan}$ sends pointwise homotopy equivalences to homotopy equivalences. At this point, we need the second black box:
	\begin{alphanumerate}\itshape
		\item[\blacksquare_2] If $J$ is a directed partially ordered set, then there is an equivalence of $\infty$-categories\label{blackbox:Localisation}
		\begin{equation*}
			\Fun\!\left(J,\cat{Kan}\right)\left[\left\{\text{pointwise homotopy equivalences}\right\}^{-1}\right]\overset{\simeq}{\longrightarrow}\Fun\!\left(J,\cat{An}\right)\,.
		\end{equation*}
	\end{alphanumerate}
	The proof of \cref{blackbox:Localisation} is similar to that of \cref{thm:AnAsALocalisation}: First, one defines a simplicial model structure on $\Fun(J,\cat{sSet})$ in such a way that $\N_\Delta(\Fun(J,\cat{sSet})_\Delta^\mathrm{cf})\simeq \Fun(J,\cat{An})$.  
	In the proof of \cite[Proposition~\HTTthm{5.3.3.3}]{HTT}, Lurie explains how to do this. Then one uses the general result from \cite[Theorem~\HAthm{1.3.4.20}]{HA} to identify the simplicial nerve $\N_\Delta(\Fun(J,\cat{sSet})_\Delta^\mathrm{cf})$ with the localisation above.
	
	Now let $p\colon\cat{Kan}\rightarrow\cat{An}$ and $p_J\colon \Fun(J,\cat{Kan})\rightarrow\Fun(J,\cat{An})$ denote the canonical functors. By \cref{thm:AnAsALocalisation} and \cref{blackbox:Localisation}, both $p$ and $p_J$ are localisations. As we've seen above, $\colimit\colon \Fun(J,\cat{Kan})\rightarrow\cat{Kan}$ sends pointwise homotopy equivalences to homotopy equivalences. Hence $p\circ \colimit\colon \Fun(J,\cat{Kan})\rightarrow \cat{An}$ factors uniquely through the localisation $p_J$ by \cref{lem:Localisation}. Let $c\colon \Fun(J,\cat{An})\rightarrow\cat{An}$ be the induced functor; we claim that $c$ is simply the colimit functor. To this end, consider $\const\colon \cat{Kan}\rightarrow\Fun(J,\cat{Kan})$; since it sends homotopy equivalences to pointwise homotopy equivalences, the same argument as above shows that $p_J\circ \const$ factors uniquely through the localisation $p$. That factorisation is necessarily $\const\colon \cat{An}\rightarrow\Fun(J,\cat{An})$. It's straightforward to verify that the adjunction $\colimit\colon \Fun(J,\cat{Kan})\shortdoublelrmorphism \cat{Kan}\noloc \const$ descends to an adjunction $c\colon \Fun(J,\cat{An})\shortdoublelrmorphism \cat{An}\noloc \const$ on the localisations. Indeed, one can show using \cref{lem:Localisation} that the unit and counit transformations as well as the triangle identities get inherited, so we may appeal to \cref{lem:TriangleIdentities}. This shows that $c$ must be the colimit functor, as claimed.
	
	Finally, we can finish the proof. Let $(X_{(-)},x_{(-)})\colon\Jj\rightarrow \cat{An}_{*/}$ be a functor from a filtered $\infty$-category into pointed animae. By \cref{blackbox:Cofinal}, we may assume that $\Jj\simeq J$ is a directed partially ordered set. By \cref{lem:FilteredCofinal}, we may assume that $J$ contains an initial object $j_0$. Then for every $(X_j,x_j)$, the point $\{x_j\}\rightarrow X_j$ agrees with $\{x_{j_0}\}\rightarrow X_{j_0}\rightarrow X_j$. Since $\cat{An}_{*/}\rightarrow\cat{An}$ preserves filtered colimits by \cref{lem:ColimitsInSliceCategory}, it follows that the pointed anima $\colimit_{j\in J}(X_j,x_j)$ is given by the unpointed colimit $\colimit_{j\in J}X_j$ together with the point $x_{j_0}\rightarrow X_{j_0}\rightarrow \colimit_{j\in J}X_j$. By \cref{blackbox:Localisation}, $\Fun(J,\cat{Kan})\rightarrow\Fun(J,\cat{An})$ is essentially surjective. So we may assume that $X_{(-)}$ comes from a functor $X_{(-)}\colon I\rightarrow\cat{Kan}$. As argued above, we may then as well take the colimit in $\cat{Kan}$ instead of $\cat{An}$. So the fact that $\pi_n\colon \cat{An}_{*/}\rightarrow \cat{Set}$ preserves filtered colimits reduces to the same assertion about $\pi_n\colon\cat{Kan}_{*/}\rightarrow\cat{Set}$, which we already know.
\end{proof}
\begin{proof}[Proof of \cref{lem:FilteredColimitsPreserveFiniteLimits}]
	First assume $\Jj$ is $\kappa$-filtered. By \cref{lem:KappaSmallColimits}, it's enough to show that $\colimit\colon\Fun(\Jj,\cat{An})\rightarrow\cat{An}$ preserves pullbacks and $\kappa$-small products. Using \cref{lem:LongExactFibrationSequence} and the five lemma (plus \cref{rem:ExactnessInLowDegrees}), we can further reduce pullbacks to fibre sequences (in the sense of \cref{def:Cofibre}).
	
	Let's do $\kappa$-small products first. We have to show that for every set $I$ of cardinality $<\kappa$, every $\kappa$-filtered $\infty$-category $\Jj$, and every functor $X_{(-,-)}\colon I\times\Jj\rightarrow\cat{An}$, the natural map
	\begin{equation*}
		\colimit_{j\in\Jj}\prod_{i\in I}X_{i,j}\longrightarrow\prod_{i\in I}\colimit_{j\in\Jj}X_{i,j}
	\end{equation*}
	is an equivalence. This can be checked on homotopy groups. We get a bijection on $\pi_0$, since $\pi_0\colon\cat{An}\rightarrow\cat{Set}$ preserves products and colimits and in $\cat{Set}$, $\kappa$-filtered colimits commute with $\kappa$-small products. For higher homotopy groups, fix some $x\in \pi_0\bigl(\colimit_{j\in\Jj}\prod_{i\in I}X_{i,j}\bigr)$. Since $\pi_0$ commutes with colimits, we must have $x\in \pi_0\bigl(\prod_{i\in I}X_{i,j_0}\bigr)$ for some $j_0\in \Jj$. By \cref{lem:FilteredCofinal}, we may replace $\Jj$ by $\Jj_{j_0/}$ to assume that $j_0$ is initial in $\Jj$. Then $\{x\}\rightarrow \prod_{i\in I}X_{i,j_0}\rightarrow X_{i,j_0}\rightarrow X_{i,j}$ for all $(i,j)\in I\times\Jj$ turns $X_{(-,-)}$ into a functor $X_{(-,-)}\colon I\times\Jj\rightarrow\cat{An}_{*/}$. Then $\pi_n\bigl(\colimit_{j\in\Jj}\prod_{i\in I}X_{i,j},x\bigr)\cong\pi_n\bigl(\prod_{i\in I}\colimit_{j\in\Jj}X_{i,j},x\bigr)$ follows from \cref{lem:HomotopyGroupsFilteredColimits} and the fact that $\kappa$-filtered colimits in $\cat{Grp}$ or $\cat{Ab}$ commute with $\kappa$-small products.%; the latter is straightforward to verify, observing that $\operatorname{ho}(\Jj)$ is $\kappa$-filtered in the ordinary sense.
	
	The case of fibre sequences is similar. Let $F_{(-)}\Rightarrow X_{(-)}\Rightarrow Y_{(-)}$ be a fibre sequence in $\Fun(\Jj,\cat{An})$; by \cref{lem:ColimitsInFunctorCategories}, this is equivalent to $F_j\rightarrow X_j\rightarrow Y_j$ being a fibre sequences for every $j\in\Jj$. We must show that
	\begin{equation*}
		\colimit_{j\in \Jj}F_j\longrightarrow\fib\!\left(\colimit_{j\in\Jj}X_j\rightarrow \colimit_{j\in\Jj}Y_j\right)
	\end{equation*}
	is an equivalence. This follows from a comparison of long exact sequences, using \cref{lem:LongExactFibrationSequence} and the five lemma together with the fact that filtered colimits preserve exact sequences. This finishes the proof that $\colimit\colon \Fun(\Jj,\cat{An})\rightarrow\cat{An}$ preserves $\kappa$-small limits.
	
	Conversely, assume this is the case; we must show that $\Jj$ is $\kappa$-filtered. Let $\alpha\colon \Ii\rightarrow \Jj$ be a functor from a essentially $\kappa$-small $\infty$-category. Consider the composition
	\begin{equation*}
		\Ii^\op\xrightarrow{\alpha^\op} \Jj^\op\xrightarrow{\Yo_{\Jj^{\op}}} \Fun\!\left(\Jj,\cat{An}\right)\,.
	\end{equation*}
	Since $\Fun\!\left(\Jj,\cat{An}\right)$ has all limits by \cref{lem:ColimitsInFunctorCategories}, we can put $E\coloneqq \lim(\Ii^\op\rightarrow \Fun(\Jj,\cat{An}))$ and extend the functor above to a limit cone $(\alpha^\op)^\triangleleft\colon (\Ii^\op)^\triangleleft\rightarrow\Fun(\Jj,\cat{An})$. If there is an object $j\in\Jj^\op$ together with a natural transformation $\eta\colon \Hom_\Jj(j,-)\Rightarrow E$ in $\Fun(\Jj,\cat{An})$. We may view $(\alpha^\op)^\triangleleft$ as a natural transformation $\const E\Rightarrow \Yo_{\Jj^\op}\circ\alpha^\op$. Composing with $\const\eta\colon \const \Hom_\Jj(j,-)\Rightarrow \const E$ yields another natural transformation, which we may again view as a functor $(\beta^\op)^\triangleleft\colon (\Ii^\op)^\triangleleft\rightarrow\Fun(\Jj,\cat{An})$. Then $(\beta^\op)^\triangleleft$ lands in the essential image of $\Yo_{\Jj^\op}$. Since the Yoneda embedding is fully faithful by \cref{cor:YonedaEmbeddingFullyFaithful}, we obtain a functor $\beta^\triangleright\colon \Ii^\triangleright\rightarrow\Jj$, as desired.
	
	So assume on the contrary that there exists no $\eta\colon \Hom_\Jj(j,-)\Rightarrow E$ as above. By Yoneda's lemma, \cref{thm:Yoneda}, this implies $E(j)\simeq \emptyset$ for all $j\in \Jj$. Since initial objects are preserved under arbitrary colimits, $\colimit_{j\in\Jj}E(j)\simeq \emptyset$. On the other hand, \cref{lem:ColimitsInAnima} implies $\colimit_{j\in\Jj}\Hom_\Jj(j_0,j)\simeq \mathop|\Jj_{j_0/}\mathclose|\simeq*$ for every $j_0\in\Jj$. Since $\colimit\colon \Fun(\Jj,\cat{An})\rightarrow\cat{An}$ preserves $\kappa$-small limits by assumption, it follows that $\colimit_{j\in\Jj}E(j)\simeq \limit_{i\in\Ii^\op}*\simeq *$, as terminal objects are preserved under arbitrary limits. Since $\emptyset\not\simeq *$, we get a contradiction.
\end{proof}
Next, we'll introduce $\cat{Ind}$-categories in the world of $\infty$-categories and we'll finally define what a presentable $\infty$-category is supposed to be.
\begin{con}\label{con:Ind}
	Let $\kappa$ be a regular cardinal and let $\Cc$ be an essentially small $\infty$-category. We let $\cat{Ind}_\kappa(\Cc)\subseteq \PSh(\Cc)$ be the full sub-$\infty$-category spanned by those presheaves $E\colon \Cc^\op\rightarrow\cat{An}$ for which the unstraightening $\operatorname{Un}^{(\mathrm{right})}(E)$ is $\kappa$-filtered. In the case $\kappa=\aleph_0$, we often write $\cat{Ind}(\Cc)\coloneqq\cat{Ind}_{\aleph_0}(\Cc)$.
	
	Note that the Yoneda embedding $\Yo_\Cc\colon \Cc\rightarrow\PSh(\Cc)$ factors through $\cat{Ind}_\kappa(\Cc)$. Indeed, for every $x\in\Cc$, the unstraightening of $\Hom_\Cc(-,x)\colon \Cc^\op\rightarrow\Cc$ is the right fibration $\Cc_{/x}\rightarrow\Cc$ by the dual of \cref{exm:Straightening}\cref{enum:SliceLeftFibration}. Now $\Cc_{/x}$ has a final object, hence it is $\kappa$-filtered for any $\kappa$. Indeed, composing any functor $\Ii\rightarrow\Cc_{/x}$ with $\id_{\Cc_{/x}}\Rightarrow \const \id_x$ yields an extension $\Ii^\triangleright\rightarrow\Cc_{/x}$, as desired.  We'll denote the factorisation by $\Yo_\Cc^\kappa\colon \Cc\rightarrow\cat{Ind}_\kappa(\Cc)$. If no confusion can occur, we'll usually drop the superscript and just write $\Yo_\Cc$.
	
	More generally, we have $\operatorname{Un}^{(\mathrm{right})}(E)\simeq \Cc_{/E}$ for all $E\in \PSh(\Cc)$, so $E$ is contained in $\cat{Ind}_\kappa(\Cc)$ if and only if $\Cc_{/E}$ is filtered. Indeed, the right fibration $\PSh(\Cc)_{/E}\rightarrow\PSh(\Cc)$ is the unstraightening of $\Hom_{\PSh(\Cc)}(-,E)\colon \PSh(\Cc)^\op\rightarrow\cat{An}$. By Yoneda's lemma (and \cref{par:YonedaFunctorial}), we have an equivalence $E\simeq \Hom_{\PSh(\Cc)}(\Yo_\Cc(-),E)$ of presehaves. Hence the unstraightening of $E$ is the pullback of $\PSh(\Cc)_{/E}\rightarrow\PSh(\Cc)$ along $\Yo_\Cc\colon \Cc\rightarrow\PSh(\Cc)$, which is $\Cc_{/E}$.
\end{con}
\begin{defi}\label{def:Presentable}
	Let $\kappa$ be a regular cardinal. A \embrace{not necessarily essentially small} $\infty$-category $\Cc$ is \emph{$\kappa$-accessible} if $\Cc\simeq\cat{Ind}_\kappa(\Cc_0)$ for some small $\infty$-category $\Cc_0$. We call $\Cc$ \emph{accessible} if it is $\kappa$-accessible for some regular cardinal $\kappa$. We call $\Cc$ \emph{presentable} if it is accessible and has all colimits.
\end{defi}
\begin{lem}\label{lem:Ind}
	Let $\kappa$ be a regular cardinal and let $\Cc$ be a small $\infty$-category.
	\begin{alphanumerate}
		\item A presheaf $E\in\PSh(\Cc)$ belongs to $\cat{Ind}_\kappa(\Cc)$ if and only if $E$ can be written as a $\kappa$-filtered colimit of representable presheaves. Furthermore, $\cat{Ind}_\kappa(\Cc)\subseteq \PSh(\Cc)$ is closed under $\kappa$-filtered colimits.\label{enum:IndGeneratedUnderFilteredColimits}
		\item If $\Cc$ has $\kappa$-small colimits, then a presheaf $E\in\PSh(\Cc)$ belongs to $\cat{Ind}_\kappa(\Cc)$ if and only if $E\colon \Cc^\op\rightarrow\cat{An}$ preserves $\kappa$-small limits. \label{enum:IndLimits}
		\item If $\Dd$ is an $\infty$-category which has all $\kappa$-filtered colimits, then restriction along the Yoneda embedding induces an equivalence\label{enum:IndFreelyGenerated}
		\begin{equation*}
			\Yo_\Cc^*\colon \Fun^{\kappa\mhyph\mathrm{filt}}\lef(\cat{Ind}_\kappa(\Cc),\Dd\righ)\overset{\simeq}{\longrightarrow}\Fun\!\left(\Cc,\Dd\right)\,.
		\end{equation*}
		Here $\Fun^{\kappa\mhyph\mathrm{filt}}\lef(\cat{Ind}_\kappa(\Cc),\Dd\righ)\subseteq\Fun\lef(\cat{Ind}_\kappa(\Cc),\Dd\righ)$ is spanned by those functors that preserve $\kappa$-filtered colimits.
	\end{alphanumerate}
\end{lem}
\begin{proof}
	We begin with \cref{enum:IndGeneratedUnderFilteredColimits}. By \cref{lem:PresheafColimitOfRepresentables}, every presheaf $E$ can be written as a colimit of representables, with $\Cc_{/E}$ as indexing $\infty$-category. If $E\in\cat{Ind}_\kappa(\Cc)$, then $\Cc_{/E}$ is $\kappa$-filtered by \cref{con:Ind}, hence $E$ is a $\kappa$-filtered colimit of representables. Conversely, assume $E$ can be written as such a $\kappa$-filtered colimit, say, $E\simeq\colimit_{j\in\Jj}\Hom_\Cc(-,x_j)$. Since the unstraightening $\operatorname{Un}^{(\mathrm{right})}\colon \PSh(\Cc)\rightarrow\cat{Right}(\Cc)$ is an equivalence of $\infty$-categories, it preserves colimits. Recall from \cref{lem:KanExtensionForRight}\cref{enum:RightCofinalLeftAdjoint} that the inclusion $\cat{Right}(\Cc)\subseteq\Cat_{\infty/\Cc}$ has a left adjoint $c\colon \Cat_{\infty/\Cc}\rightarrow\cat{Right}(\Cc)$. Furthermore, by the dual of \cref{lem:ColimitsInSliceCategory}, $\cat{Cat}_{\infty/\Cc}\rightarrow\operatorname{Cat}_\infty$ preserves colimits. Hence a colimit in $\cat{Right}(\Cc)$ is computed by taking the colimit in $\cat{Cat}_\infty$ and then applying $c$. Therefore
	\begin{equation*}
		\operatorname{Un}^{(\mathrm{right})}(E)\simeq c\left(\colimit_{j\in\Jj}\Cc_{/x_j}\right)\,.
	\end{equation*}
	Now a $\kappa$-filtered colimit of $\kappa$-filtered $\infty$-categories is $\kappa$-filtered again, which follows by combining \cref{lem:ColimitManipulations}\cref{claim:AssembleColimits} with the characterisation of $\kappa$-filteredness from \cref{lem:FilteredColimitsPreserveFiniteLimits}. So $\colimit_{j\in\Jj}\Cc_{/x_j}$ is $\kappa$-filtered. By \cref{lem:FilteredColimitsPreserveFiniteLimits} again it's clear that being $\kappa$-filtered is preserved under cofinal functors. Since $\colimit_{j\in\Jj}\Cc_{/x_j}\rightarrow c(\colimit_{j\in\Jj}\Cc_{/x_j})$ is cofinal by \cref{lem:KanExtensionForRight}\cref{enum:RightCofinalLeftAdjoint}, we've shown that $\operatorname{Un}^{(\mathrm{right})}(E)$ is $\kappa$-filtered, as desired. The same argument shows that $\cat{Ind}_\kappa(\Cc)\subseteq \PSh(\Cc)$ is closed under $\kappa$-filtered colimits.
	
	For \cref{enum:IndLimits}, let's temporarily denote $\PSh^\kappa(\Cc)\subseteq\PSh(\Cc)$ the full sub-$\infty$-category of presheaves $E\colon \Cc^\op\rightarrow\cat{An}$ that preserve $\kappa$-small limits. Every representable presheaf preserves all limits by \cref{cor:HomPreservesLimits}, in particular, $\kappa$-small ones. By \cref{lem:FilteredColimitsPreserveFiniteLimits}, $\PSh^\kappa(\Cc)\subseteq \PSh(\Cc)$ is stable under $\kappa$-filtered colimits. By \cref{enum:IndGeneratedUnderFilteredColimits}, every $E\in\cat{Ind}_\kappa(\Cc)$ is a $\kappa$-filtered colimit of representables, hence $E\in \PSh^\kappa(\Cc)$. Conversely, assume $E\in\PSh^\kappa(\Cc)$. To show that $\Cc_{/E}$ is $\kappa$-filtered, we claim:
	\begin{alphanumerate}\itshape
		\item[\boxtimes] \!The restricted Yoneda embedding $\Yo_\Cc\colon \Cc\rightarrow\PSh^\kappa(\Cc)$ preserves $\kappa$-small colimits \embrace{note that this is completely false for the unrestricted Yoneda embedding}.\label{claim:YonedaPreservesColimits}
	\end{alphanumerate}
	If $\alpha\colon\Ii\rightarrow \Cc_{/E}$ is any functor, and $\ov\alpha\colon \Ii\rightarrow\Cc_{/E}\rightarrow\Cc$ denotes the composition of $\alpha$ with the projection to $\Cc$, then $x\coloneqq \colimit(\ov\alpha\colon\Ii\rightarrow\Cc)$ exists by assumption on $\Cc$. Now $\alpha$ corresponds to a natural transformation $\eta\colon\Yo_\Cc\circ\ov\alpha\Rightarrow\const E$ in $\Fun(\Ii,\PSh(\Cc))$. If \cref{claim:YonedaPreservesColimits} holds, then we can use the universal property of colimits to show that $\eta$ factors uniquely through $\Yo_\Cc\circ\ov\alpha\Rightarrow\const\Yo_\Cc(x)$. Since $\Yo_\Cc$ is fully faithful, this yields an extension $\alpha^\triangleright\colon \Ii\rightarrow\Cc_{/E}$, as desired.
	
	To prove \cref{claim:YonedaPreservesColimits}, let $x_{(-)}\colon \Ii\rightarrow\Cc$ be a functor from an essentially $\kappa$-small $\infty$-category $\Cc$ and let $E\in\PSh^\kappa(\Cc)$. Then $\Hom_{\PSh(\Cc)}(\Yo_\Cc(\colimit_{i\in\Ii}x_i),E)\simeq E(\colimit_{i\in\Ii}x_i)$ by Yoneda's lemma. Since $E$ preserves $\kappa$-small limits (and limits in $\Cc^\op$ correspond to colimits in $\Cc$), we can use Yoneda's lemma again to see
	\begin{equation*}
		E\!\left(\colimit_{i\in\Ii}x_i\right)\simeq \limit_{i\in\Ii}E\!\left(x_i\right)\simeq \limit_{i\in\Ii}\Hom_{\PSh(\Cc)}\lef(\Yo_\Cc(x_i),E\righ)\,.
	\end{equation*}
	Using \cref{cor:HomPreservesColimits}, this proves \cref{claim:YonedaPreservesColimits}.
	
	To prove \cref{enum:IndFreelyGenerated}, we can more or less copy the proof of \cref{thm:PShFreeCocompletion}: Let $F\colon \Cc\rightarrow\Dd$ be any functor. Since $\Dd$ has filtered colimits and $\Cc_{/E}$ is filtered for every $E\in\cat{Ind}_\kappa(\Cc)$, the Kan extension $\Lan_{\Yo_\Cc^\kappa}F\colon \cat{Ind}_\kappa(\Cc)\rightarrow\Dd$ exists by \cref{lem:KanExtensionFormula}. We must show that the Kan extension $\Lan_{\Yo_\Cc^\kappa}F$ preserves $\kappa$-filtered colimits. Let's first assume that $\Dd$ has all colimits. Consider the Kan extension $\Lan_{\Yo_\Cc}F\colon \PSh(\Cc)\rightarrow\Dd$. For formal reasons, $\Lan_{\Yo_\Cc}F$ is the left Kan extension of $\Lan_{\Yo_\Cc^\kappa}F$ along $\cat{Ind}_\kappa(\Cc)\subseteq\PSh(\Cc)$. Since the latter is fully faithful, \cref{cor:KanExtensionAlongFullyFaithful} shows $\Lan_{\Yo_\Cc^\kappa}F\simeq(\Lan_{\Yo_\Cc}F)|_{\cat{Ind}_\kappa(\Cc)}$. Now $\Lan_{\Yo_\Cc}F\colon\PSh(\Cc)\rightarrow\Dd$ preserves colimits by \cref{lem:LanAlongYonedaHasRightAdjoint} and $\cat{Ind}_\kappa(\Cc)\subseteq \PSh(\Cc)$ preserves $\kappa$-filtered colimits by \cref{enum:IndGeneratedUnderFilteredColimits}, so $\Lan_{\Yo_\Cc^\kappa}F$ preserves $\kappa$-filtered colimits, as desired. This concludes the case where $\Dd$ has all colimits.
	
	The general case can be reduced to this as follows: As in the proof of \cref{lem:ColimitsInSliceCategory}, we can choose a fully faithful colimits-preserving functor $i\colon \Dd\rightarrow\Dd'$ into an $\infty$-category with all colimits. The formula from \cref{lem:KanExtensionFormula} combined with \cref{thm:EquivalencePointwise} show that the canonical natural transformation $\Lan_{\Yo_\Cc^\kappa}(i\circ F)\Rightarrow i\circ \Lan_{\Yo_\Cc^\kappa}F$ is an equivalence, and so it suffices that $\Lan_{\Yo_\Cc^\kappa}(i\circ F)$ preserves $\kappa$-filtered colimits, which we did above.
	
	So the Kan extension functor $\Lan_{\Yo_\Cc^\kappa}\colon \Fun(\Cc,\Dd)\rightarrow\Fun(\cat{Ind}_\kappa(\Cc),\Dd)$ lands in the full sub-$\infty$-category $\Fun^\kappa(\cat{Ind}_\kappa(\Cc),\Dd)$. Therefore, we obtain an adjunction
	\begin{equation*}
		\Lan_{\Yo_\Cc}\colon \Fun\!\left(\Cc,\Dd\right)\doublelrmorphism \Fun^{\kappa\mhyph\mathrm{filt}}\lef(\cat{Ind}_\kappa(\Cc),\Dd\righ)\noloc \Yo_\Cc^*
	\end{equation*}
	By the same arguments as in the proof of \cref{thm:PShFreeCocompletion}, the unit $u\colon \id_{\Fun(\Cc,\Dd)}\Rightarrow\Yo_\Cc^*\circ \Lan_{\Yo_\Cc}$ is an equivalence and $\Yo_\Cc^*$ is conservative, hence the adjunction above is a pair of inverse equivalences by \cref{lem:FullyFaithfulConservativeAdjunction}.
\end{proof}
It's surprisingly common for an $\infty$-category to be of the form $\cat{Ind}_\kappa(\Cc)$ for some $\Cc$ (that is, to be \emph{$\kappa$-accessible} in the sense of \cref{def:Presentable}).
\begin{lem}\label{lem:KappaCompactlyGenerated}
	Let $\kappa$ be a regular cardinal and let $\Dd$ be a locally small $\infty$-category. Then the following are equivalent:
	\begin{alphanumerate}
		\item $\Dd$ is of the form $\Dd\simeq \cat{Ind}_\kappa(\Cc)$ for some small $\infty$-category $\Cc$.\label{enum:DIsIndC}
		\item $\Dd$ admits $\kappa$-filtered colimits and there exists a set $S$ of $\kappa$-compact objects such that every object from $\Dd$ can be written as a $\kappa$-filtered colimit of objects from $S$.\label{enum:DGeneratedUnderFilteredColimits}
	\end{alphanumerate}
	In this case automatically $\Dd\simeq \cat{Ind}_\kappa(\Dd^\kappa)$, where $\Dd^\kappa\subseteq \Dd$ is the full sub-$\infty$-category spanned by the $\kappa$-compact objects. Furthermore, if $\Dd$ has $\kappa$-small colimits, then there is another equivalent condition:
	\begin{alphanumerate}[resume]
		\item $\Dd$ admits $\kappa$-filtered colimits and has a set of $\kappa$-compact generators; that is, a set $S\subseteq \Dd$ of $\kappa$-compact objects such that the functors $\Hom_\Dd(s,-)\colon \Dd\rightarrow\cat{An}$ are jointly conservative.\label{enum:CompactGenerators}
	\end{alphanumerate}%
	In the case where $\Dd$ has $\kappa$-small colimits and \cref{enum:CompactGenerators} holds, $\Dd$ is even presentable.
\end{lem}
\begin{proof}
	Assume \cref{enum:DIsIndC} is true. Then $\Dd$ has $\kappa$-filtered colimits by \cref{lem:Ind}\cref{enum:IndGeneratedUnderFilteredColimits}. We claim that $S\coloneqq\left\{\Yo_\Cc(x)\ \middle|\ x\in\Cc\right\}$ generates $\Dd$ under $\kappa$-filtered colimits and forms a set of compact generators in the sense of \cref{enum:CompactGenerators}. The first assertion is \cref{lem:Ind}\cref{enum:IndGeneratedUnderFilteredColimits}. For the second assertion, Yoneda's lemma says $\Hom_{\PSh(\Cc)}(\Yo_\Cc(x),E)\simeq E(x)$ for every $E\in\cat{Ind}_\kappa(\Cc)$. Hence $\Hom_{\PSh(\Cc)}(\Yo_\Cc(x),-)$ for $x\in\Cc$ are jointly conservative by \cref{thm:EquivalencePointwise}. Furthermore, $\Yo_\Cc(x)$ is $\kappa$-compact since colimits in presheaf $\infty$-categories are computed pointwise by \cref{lem:ColimitsInFunctorCategories} and $\cat{Ind}_\kappa(\Cc)\subseteq\PSh(\Cc)$ preserves $\kappa$-filtered colimits by \cref{lem:Ind}\cref{enum:IndGeneratedUnderFilteredColimits}. This proves \cref{enum:DIsIndC} $\Rightarrow$ \cref{enum:DGeneratedUnderFilteredColimits} and \cref{enum:DIsIndC} $\Rightarrow$ \cref{enum:CompactGenerators} (even without the assumption that $\Dd$ has $\kappa$-small colimits).
	
	Now assume \cref{enum:DGeneratedUnderFilteredColimits}. Let's first sketch why $\Dd^\kappa$ is essentially small. We'll show that every $x\in\Dd^\kappa$ is a retract of some $s\in S$ and then leave it to you to verify that $S$ can't have \enquote{too many} retracts in the locally small $\infty$-category $\Dd$. Write $x\simeq \colimit_{j\in\Jj}s_j$ for some $s_j\in S$ and some $\kappa$-filtered $\infty$-category $\Jj$. Since $x$ is $\kappa$-compact and $\pi_0$ commutes with colimits by \cref{lem:HomotopyGroupsFilteredColimits}, we get $\colimit_{j\in\Jj}\pi_0\Hom_\Cc(x,s_j)\cong \pi_0\Hom_\Dd(x,x)$. Choosing a preimage of $\id_x$ yields a morphism $x\rightarrow s_j$ for some $j\in\Jj$, which exhibits $x$ as a retract of $s_j$, as desired.
	
	By \cref{lem:Ind}\cref{enum:IndFreelyGenerated}, the inclusion $\Dd^\kappa\subseteq \Dd$ extends uniquely to a functor $L^\kappa\colon \cat{Ind}_\kappa(\Dd^\kappa)\rightarrow\Dd$ that preserves $\kappa$-filtered colimits. Let's first construct a right adjoint $R^\kappa$. To this end, choose a fully faithful colimits-preserving functor $i\colon \Dd\rightarrow\Dd'$ into an $\infty$-category $\Dd'$ with all colimits; this can be done as in the proof of \cref{lem:Ind}\cref{enum:IndFreelyGenerated}. Furthermore, $i\circ L^\kappa$ extends uniquely to a colimits-preserving functor $L\colon \PSh(\Dd^\kappa)\rightarrow \Dd'$, which has a right adjoint $R$ by \cref{thm:PShFreeCocompletion}. We claim that $R\circ i\colon \Dd\rightarrow\PSh(\Dd)$ lands in $\cat{Ind}_\kappa(\Dd^\kappa)$. Indeed, let $y\in\Dd$ and write $y\simeq \colimit_{j\in\Jj}x_j$ where $\Jj$ is $\kappa$-filtered and $x_j\in\Dd^\kappa$; we could even choose $x_j\in S$. By the formula from \cref{lem:LanAlongYonedaHasRightAdjoint}, $R(i(y))$ is the presheaf $\Hom_\Dd(-,\colimit_{j\in\Jj}x_j)\colon (\Dd^\kappa)^\op\rightarrow\cat{An}$. By definition of $\Dd^\kappa$, this presheaf agrees with $\colimit_{j\in\Jj}\Hom_\Dd(-,x_j)$. Hence $R(i(y))$ is a $\kappa$-filtered colimit of representable presheaves and thus contained in $\cat{Ind}_\kappa(\Dd^\kappa)$ by \cref{lem:Ind}\cref{enum:IndGeneratedUnderFilteredColimits}. Thus, putting $R^\kappa\coloneqq R\circ i$, we obtain the desired adjunction $L^\kappa\colon \cat{Ind}_\kappa(\Dd^\kappa)\shortdoublelrmorphism\Dd\noloc R^\kappa$. Moreover, our argument shows that $R^\kappa$ commutes with $\kappa$-filtered colimits of objects from $\Dd^\kappa$. By inspection, the counit $c\colon L^\kappa\circ R^\kappa\Rightarrow \id_\Dd$ is an equivalence for objects from $\Dd^\kappa$. Since both sides commute with $\kappa$-filtered colimits of objects from $\Dd^\kappa$ and every object of $\Dd$ can be written as such a colimit, we see that $c$ is an equivalence. An analogous argument shows that $u\colon \id_{\cat{Ind}_\kappa(\Dd^\kappa)}\Rightarrow R^\kappa\circ L^\kappa$ is an equivalence. This proves \cref{enum:DGeneratedUnderFilteredColimits} $\Rightarrow$ \cref{enum:DIsIndC}.
	
	It remains to show \cref{enum:CompactGenerators} $\Rightarrow$ \cref{enum:DIsIndC}. First observe that if $\Dd$ has $\kappa$-small and $\kappa$-filtered colimits, then $\Dd$ has all colimits. Indeed, according to \cref{lem:ColimitsIffCoproductsAndPushouts}, we only need to check that $\Dd$ has arbitrary coproducts. This follows from the following claim:
	\begin{alphanumerate}\itshape
		\item[\boxtimes] Let $T$ be a discrete set and let $\Pp^{\kappa}(T)\subseteq \Pp(T)$ be the partially ordered set of all subsets $S\subseteq T$ of cardinality $\left|S\right|<\kappa$. Then $\Pp^{\kappa}(T)$ is $\kappa$-filtered and for every collection $(x_t)_{t\in T}$ of objects of $\Dd$ we have\label{claim:FilteredCoproduct}
		\begin{equation*}
			\colimit_{S\in\Pp^{\kappa}(T)}\coprod_{s\in S}x_s\overset{\simeq}{\longrightarrow}\coprod_{t\in T}x_t\,.
		\end{equation*}
	\end{alphanumerate}
	Using \cref{lem:SimplicialHoNerveAdjunction}, $\kappa$-filteredness of $\Pp^\kappa(T)$ reduces to a question about ordinary categories, which is easy. Now consider the tautological functor $U\colon \Pp^\kappa(T)\rightarrow\cat{Set}$ sending $S\mapsto S$ and let $\Uu$ be its unstraightening (which is an ordinary category and easy to describe). There is a tautological natural transformation $U\Rightarrow\const T$, which induces a functor $\Uu\rightarrow \Pp^\kappa(T)\times T$ on unstraightenings. Note that $\Uu\rightarrow \Pp^\kappa(T)\times T\rightarrow T$ is cofinal. indeed, for every $t\in T$, the slice $\Uu\times_TT_{t/}$ can be identified with the sub-partially ordered set $\Pp_t^\kappa(T)\subseteq \Pp^\kappa(T)$ of those $S$ such that $t\in S$. Then $\Pp_t^\kappa(T)$ has an initial object, namely $\{t\}$, and so $\mathopen|\Pp_t^\kappa(T)\mathclose|\simeq *$, whence \cref{thm:JoyalsQuillenA}\cref{enum:WeaklyContractible} is satisfied. Thus, if $T\rightarrow \Dd$ corresponds to the collection $(x_t)_{t\in T}$, then $\colimit(T\rightarrow\Dd)\simeq \colimit(\Uu\rightarrow T\rightarrow\Dd)$. Using \cref{lem:ColimitManipulations}\cref{claim:SliceColimits}, the right-hand side can be identified with $\colimit_{S\in\Pp^\kappa(T)}\coprod_{s\in S}x_s$. This finishes the proof of \cref{claim:FilteredCoproduct}
	
	Now let $S\subseteq \Dd$ be a set of $\kappa$-compact generators and let $\Cc\subseteq \Dd$ be the full sub-$\infty$-category generated by $S$ under $\kappa$-small colimits. Since $\Dd$ is locally small, one can verify that $\Cc$ is essentially small (there are \enquote{not too many} $\kappa$-small diagrams); we leave this to you. Since $\Dd$ has all colimits, we can apply \cref{thm:PShFreeCocompletion} to see that $\Cc\subseteq\Dd$ extends uniquely to a colimits-preserving functor $L\colon \PSh(\Cc)\rightarrow\Dd$, which has a right adjoint $R$. Observe that $R$ factors through $\cat{Ind}_\kappa(\Cc)$. Indeed, according to \cref{lem:LanAlongYonedaHasRightAdjoint}, for every $y\in\Dd$, the presheaf $R(y)$ is given by $\Hom_\Dd(-,y)\colon\Cc^\op\rightarrow\cat{An}$. This functor preserves arbitrary limits by \cref{cor:HomPreservesLimits}, in particular, $\kappa$-small ones, and so \cref{lem:Ind}\cref{enum:IndLimits} implies $R(y)\in\cat{Ind}_\kappa(\Cc)$. Restricting $L$, we thus obtain an adjunction $L\colon\cat{Ind}_\kappa(\Cc)\shortdoublelrmorphism\Dd\noloc R$. Observe that $R$ preserves $\kappa$-filtered colimits. Indeed, let $y_{(-)}\colon \Jj\rightarrow\Dd$ be a functor from a $\kappa$-filtered $\infty$-category. By \cref{thm:EquivalencePointwise} and \cref{lem:ColimitsInFunctorCategories}, it suffices to show that $\colimit_{j\in\Jj}\Hom_\Dd(x,y_j)\rightarrow \Hom_\Dd(x,\colimit_{j\in\Jj}y_j)$ is an equivalence for all $x\in\Cc$. But \cref{lem:FilteredColimitsPreserveFiniteLimits} easily implies that $\kappa$-compact objects are closed under $\kappa$-small colimits and so $x$ must be $\kappa$-compact, whence we get an equivalence as desired. Now we can apply the same argument as in the proof of \cref{enum:DGeneratedUnderFilteredColimits} $\Rightarrow$ \cref{enum:DIsIndC} to show that the unit $u\colon \id_{\cat{Ind}_\kappa(\Cc)}\Rightarrow R\circ L$ is an equivalence. Furthermore, $R$ is conservative. Indeed, if $\alpha\colon y\rightarrow z$ in $\Dd$ induces an equivalence $\alpha_*\colon \Hom_\Dd(-,y)\Rightarrow \Hom_\Dd(-,z)$ of presheaves, then, in particular, $\alpha_*\colon\Hom_\Dd(s,y)\rightarrow\Hom_\Dd(s,z)$ must be an equivalence for all $s\in S$. But $\Hom_\Dd(s,-)\colon \Dd\rightarrow\cat{An}$ for $s\in S$ are jointly conservative by assumption. Now \cref{lem:FullyFaithfulConservativeAdjunction}\cref{enum:Conservative} finishes the proof of the implication \cref{enum:CompactGenerators} $\Rightarrow$ \cref{enum:DIsIndC}.
\end{proof}
\begin{cor}\label{cor:AnPresentable}
	The $\infty$-categories $\cat{An}$ and $\cat{Cat}_\infty$ are presentable.
\end{cor}
\begin{proof}[Proof sketch]
	It's clear that $\cat{An}$ and $\cat{Cat}_\infty$ are locally small and they have all colimits by \cref{lem:ColimitsInAnima}. So it suffices to check that both are accessible. In fact, we'll show that both are $\aleph_0$-accessible, by verifying the condition from \cref{lem:KappaCompactlyGenerated}\cref{enum:CompactGenerators}. For $\cat{An}$, it's clear that $*$ is a compact generator as $\Hom_{\cat{An}}(*,X)\simeq X$ for all $X\in\cat{An}$. For $\cat{Cat}_\infty$, we claim that the $\infty$-categories $*$ and $\Delta^1$ are compact generators. 
	
	Let's first argue that $\Hom_{\cat{Cat}_\infty}(*,-)$ and $\Hom_{\cat{Cat}_\infty}(\Delta^1,-)$ are jointly conservative. To this end, recall from \cref{thm:CordierPorter} that $\Hom_{\cat{Cat}_\infty}(*,\Cc)\simeq \core(\Cc)$ and $\Hom_{\cat{Cat}_\infty}(\Delta^1,\Cc)\simeq \core\Ar(\Cc)$ for every $\infty$-category $\Cc$. Now if $F\colon \Cc\rightarrow\Dd$ is a functor such that $\core(F)\colon \core(\Cc)\rightarrow\core(\Dd)$ is an equivalence, then $F$ is essentially surjective. If furthermore $\core \Ar(\Cc)\rightarrow\core\Ar(\Dd)$ is an equivalence, then $F$ is fully faithful, because for all $x,y\in\Cc$ we can write $\Hom_\Cc(x,y)$ as a pullback of $\core\Ar(\Cc)\rightarrow\core(\Cc)\times\core(\Cc)$ by \cref{par:HomInQuasiCategories} (plus \cref{par:ModelIndependence}) combined with the fact that $\core\colon \cat{Cat}_\infty\rightarrow\cat{An}$ preserves pullbacks, since it is a right adjoint by \cref{exm:Adjunctions}\cref{enum:AnToCatInfty}. Then $F$ is an equivalence by \cref{thm:EquivalenceFullyFaithfulEssentiallySurjective}. This proves that $\Hom_{\cat{Cat}_\infty}(*,-)$ and $\Hom_{\cat{Cat}_\infty}(\Delta^1,-)$ are jointly conservative.
	
	We'll only sketch the argument why $*$ and $\Delta^1$ are compact in $\cat{Cat}_\infty$. The crucial observation is that equivalences of quasi-categories are preserved under filtered colimits in the ordinary category $\cat{QCat}$. Indeed, $\cat{QCat}\subseteq \cat{sSet}$ is closed under filtered colimits, because $\Lambda_i^n$ and $\Delta^n$ are finite simplicial sets and so every horn filling problem in a filtered colimit can be solved at some finite stage. So filtered colimits in $\cat{QCat}$ can be computed in $\cat{sSet}$ instead. Then it's straightforward to check that a filtered colimit of fully faithful and essentially surjective maps of quasi-categories is again fully faithful and essentially surjective. Now we can use the same arguments as in the proof of \cref{lem:HomotopyGroupsFilteredColimits} (including the black box \cref{blackbox:Cofinal} and an analogue of \cref{blackbox:Localisation}) to see that filtered colimits in $\cat{Cat}_\infty$ can be computed as ordinary filtered colimits in $\cat{QCat}$. So it remains to show that $ \colimit_{j\in J}\core \F(*,\Cc_j)\cong \core \F(*,\colimit_{j\in\Jj}\Cc_j)$ and $\colimit_{j\in J}\F(\Delta^1,\Cc_j)\cong \core\F(\Delta^1,\colimit_{j\in J}\Cc_j)$ holds for every filtered category $J$ and every diagram $\Cc_{(-)}\colon J\rightarrow\cat{QCat}$. This is straightforward.
\end{proof}
\begin{cor}\label{cor:FunctorCategoriesPresentable}
	If $\Dd$ is a presentable $\infty$-category, then for every $y\in\Dd$ the slice $\infty$-categories $\Dd_{y/}$ and $\Dd_{/y}$ are presentable again. Furthermore, if $\Cc$ is an essentially small $\infty$-category, then $\Fun(\Cc,\Dd)$ is presentable. In particular, $\PSh(\Cc)$ and $\Fun(\Cc,\cat{Cat}_\infty)$ are presentable by \cref{cor:AnPresentable}.
\end{cor}
The same results are true for accessible $\infty$-categories, but this requires significantly more effort. In practice, the results about presentable $\infty$-categories are usually sufficient and so we refer to \cite[\S\href{https://people.math.harvard.edu/~lurie/papers/HTT.pdf\#section.5.4}{5.4}]{HTT} for the accessible case.
\begin{proof}[Proof of \cref{cor:FunctorCategoriesPresentable}]
	It follows from \cref{lem:ColimitsInSliceCategory} and its dual that if $\Dd$ has all colimits, then $\Dd_{y/}$ and $\Dd_{/y}$ have all colimits again. \cref{lem:ColimitsInFunctorCategories} shows the same for $\Fun(\Cc,\Dd)$. So it's enough to check accessibility in each case.
	
	By \cref{lem:KappaCompactlyGenerated}\cref{enum:CompactGenerators}, we can choose a set $S$ of $\kappa$-compact generators for $\Dd$. Since $\Dd$ has coproducts, one easily verifies via \cref{lem:Adjunction} that $\Dd_{y/}\rightarrow\Dd$ has a left adjoint, sending $z\in\Dd$ to $(y\rightarrow y\sqcup z)\in\Dd_{y/}$. Then $\Hom_{\Dd_{y/}}(y\rightarrow y\sqcup s,y\rightarrow z)\simeq \Hom_\Dd(s,z)$ and so $\Hom_{\Dd_{y/}}(y\rightarrow y\sqcup s,-)\colon \Dd_{y/}\rightarrow\cat{An}$ for $s\in S$ are jointly conservative. Using the adjunction property plus the fact that $\Dd_{y/}\rightarrow \Dd$ preserves $\kappa$-filtered colimits by \cref{lem:ColimitsInSliceCategory}\cref{enum:ColimitsInSlice}, we see that every $(y\rightarrow y\sqcup s)$ is $\kappa$-compact again. So $\Dd_{y/}$ satisfies the condition from \cref{lem:KappaCompactlyGenerated}\cref{enum:CompactGenerators} and is therefore accessible.
	
	For $\Fun(\Cc,\Dd)$, consider the functors $F_{x,s}\coloneqq\Lan_{\{x\}\rightarrow \Cc}(\const s)\colon \Cc\rightarrow\Dd$, where $s\in S$ and $x$ runs through a set of representatives of the equivalence classes in $\Cc$. These Kan extensions exist by \cref{lem:KanExtensionFormula} since $\Dd$ has all colimits. The universal property of Kan extensions shows $\Hom_{\Fun(\Cc,\Dd)}(F_{x,s},G)\simeq \Hom_{\Fun(\{x\},\,\Cc)}(\const s,G|_{\{x\}})\simeq \Hom_\Dd(s,G(x))$ for every functor $G\in\Fun(\Cc,\Dd)$. Since colimits in functor categories are computed pointwise by \cref{lem:ColimitsInFunctorCategories} and $s$ is $\kappa$-compact by assumption, it follows that $F_{x,s}$ is $\kappa$-compact. Since equivalences of functors can be detected pointwise by \cref{thm:EquivalencePointwise}, it follows that $\Hom_{\Fun(\Cc,\Dd)}(F_{x,s},-)\colon \Fun(\Cc,\Dd)\rightarrow\cat{An}$ for $s\in S$ and $x$ running through all equivalence classes in $\Cc$ are jointly conservative. So $\Fun(\Cc,\Dd)$ satisfies the condition from \cref{lem:KappaCompactlyGenerated}\cref{enum:CompactGenerators} and is therefore accessible.
	
	For $\Dd_{/y}$, we will instead verify the condition from \cref{lem:KappaCompactlyGenerated}\cref{enum:DGeneratedUnderFilteredColimits}. First observe that $\Dd^\kappa_{/y}\simeq \Dd^\kappa\times_\Dd\Dd_{/y}$ is essentially small. Indeed, $\Dd^\kappa_{/y}$ is a full sub-$\infty$-category of $\Dd_{/y}$, hence locally small, because $\Dd_{/y}$ is locally small by the assumption on $\Dd$ and \cref{cor:HomInSliceCategories}. So its enough to show that $\pi_0\core(\Dd^\kappa_{/y})$ is a set. This follows from $\Dd^\kappa$ being essentially small (as we've seen in the proof of \cref{lem:KappaCompactlyGenerated}) and $\Dd$ being locally small, so that there can't be \enquote{too many} equivalence classes of morphisms $z\rightarrow y$ where $z\in\Dd^\kappa$. Since $\Dd_{/y}\rightarrow\Dd$ preserves arbitrary colimits by the dual of \cref{lem:ColimitsInSliceCategory}\cref{enum:LimitsInSlice} and $\kappa$-filtered colimits are preserved under pullbacks by \cref{lem:FilteredColimitsPreserveFiniteLimits}, we can use \cref{cor:HomInSliceCategories} to show that the objects in $\Dd^\kappa_{/y}$ are $\kappa$-compact in $\Dd_{/y}$. It remains to show that they generate $\Dd_{/y}$ under $\kappa$-filtered colimits. Pick some $(z\rightarrow y)\in\Dd_{/y}$ and write $z\simeq\colimit_{j\in\Jj}z_j$ for some $\kappa$-filtered $\infty$-category $\Jj$ and some diagram $z_{(-)}\colon\Jj\rightarrow\Dd^\kappa$. Composing the colimit transformation $u\colon z_{(-)}\Rightarrow \const z$ with $\const z\Rightarrow \const y$ yields a transformation $z_{(-)}\Rightarrow \const y$, which in turn defines a functor $(z_{(-)}\rightarrow y)\colon \Jj\rightarrow \Dd_{/y}$. Then $(z\rightarrow y)\simeq \colimit_{j\in\Jj}(z_j\rightarrow y)$ in $\Dd_{/y}$, as desired.
\end{proof}
\begin{lem}\label{lem:Presentable}
	For a locally small $\infty$-category $\Dd$, the following are equivalent:
	\begin{alphanumerate}
		\item $\Dd$ is presentable.\label{enum:DIsPresentable}
		\item $\Dd$ is $\kappa$-accessible and has $\kappa$-small colimits for some regular cardinal $\kappa$.\label{enum:DHasKappaSmallColimits}
		\item \!There exists an essentially small $\infty$-category $\Cc$ and an adjunction $L\colon \PSh(\Cc)\shortdoublelrmorphism \Dd\noloc R$ such that $R$ is fully faithful and preserves $\kappa$-filtered colimits for some regular cardinal $\kappa$.\label{enum:AccessibleLocalisation}
		\item $\Dd$ is of the form $\Dd\simeq\cat{Ind}_\kappa(\Cc)$ for some essentially small $\infty$-category $\Cc$ which has $\kappa$-small colimits.\label{enum:DCHasKappaSmallColimits}
	\end{alphanumerate}
	In this case, $\Dd^\kappa$ automatically has all $\kappa$-small colimits \embrace{so that we may choose $\Cc\simeq \Dd^\kappa$ in \cref{enum:DCHasKappaSmallColimits} by \cref{lem:KappaCompactlyGenerated}}.
\end{lem}
\begin{proof}
	The implication \cref{enum:DIsPresentable} $\Rightarrow$ \cref{enum:DHasKappaSmallColimits} is trivial and \cref{enum:DIsPresentable} $\Rightarrow$ \cref{enum:AccessibleLocalisation} follows from the proof of \cref{lem:KappaCompactlyGenerated}. For \cref{enum:DHasKappaSmallColimits} $\Rightarrow$ \cref{enum:DCHasKappaSmallColimits}, we use \cref{lem:KappaCompactlyGenerated} to see that $\Dd\simeq \cat{Ind}_\kappa(\Dd^\kappa)$. It follows easily from \cref{lem:FilteredColimitsPreserveFiniteLimits} and \cref{cor:HomPreservesLimits} that $\kappa$-compact objects are closed under $\kappa$-small colimits. Therefore, if $\Dd$ has all $\kappa$-small colimits, then so has $\Dd^\kappa$. This proves \cref{enum:DHasKappaSmallColimits} $\Rightarrow$ \cref{enum:DCHasKappaSmallColimits}.
	
	For \cref{enum:AccessibleLocalisation} $\Rightarrow$ \cref{enum:DIsPresentable}, first note that $\Dd$ has all colimits. Indeed, given a diagram $\alpha\colon \Ii\rightarrow\Dd$, we can form the colimit $c\simeq\colimit_{i\in\Ii}R(\alpha(i))$ in $\PSh(\Cc)$. Since $L$ preserves colimits by \cref{lem:AdjointsPreserveColimits} and $L\circ R\simeq \id_\Dd$ by the dual of \cref{lem:FullyFaithfulConservativeAdjunction}\cref{enum:FullyFaithfulIffUnitEquivalence}, we see that $L(c)\simeq \colimit_{i\in\Ii}\alpha(i)$, as desired. Now consider the objects $L(\Yo_\Cc(x))$, where $x$ runs through a set of representatives for every equivalence class in $\Cc$. Then $\Hom_\Dd(L(\Yo_\Cc(x)),-)\simeq \Hom_{\PSh(\Cc)}(\Yo_\Cc(x),R(-))$. By Yoneda's lemma, $\Hom_{\PSh(\Cc)}(\Yo_\Cc(x),-)\colon \PSh(\Cc)\rightarrow\cat{An}$ are jointly conservative, and they preserve all colimits by \cref{lem:ColimitsInFunctorCategories}. Since $R$ is fully faithful and preserves $\kappa$-filtered colimits, it follows that $\Hom_\Dd(L(\Yo_\Cc(x)),-)\colon \Dd\rightarrow\cat{An}$ are jointly conservative and preserve $\kappa$-filtered colimits. So the set $\{L(\Yo_\Cc(x))\}$ satisfies the conditions from \cref{lem:KappaCompactlyGenerated}\cref{enum:CompactGenerators} and it follows that $\Dd$ is presentable. This proves \cref{enum:AccessibleLocalisation} $\Rightarrow$ \cref{enum:DIsPresentable}
	
	It remains to show \cref{enum:DCHasKappaSmallColimits} $\Rightarrow$ \cref{enum:DIsPresentable}. We need to show that $\cat{Ind}_\kappa(\Cc)$ has all colimits. We know from \cref{lem:Ind}\cref{enum:IndGeneratedUnderFilteredColimits} that $\cat{Ind}_\kappa(\Cc)$ has $\kappa$-filtered colimits, so by claim~\cref{claim:FilteredCoproduct} in the proof of \cref{lem:KappaCompactlyGenerated}, it's enough to show that $\cat{Ind}_\kappa(\Cc)$ has $\kappa$-small colimits. By \cref{lem:KappaSmallColimits}, it's enough to construct pushouts and $\kappa$-small coproducts. Also, we've seen in the proof of \cref{lem:Ind} that $\Yo_\Cc\colon \Cc\rightarrow\cat{Ind}_\kappa(\Cc)$ preserves $\kappa$-small colimits. Since $\Cc$ itself has all $\kappa$-small colimits by assumption, we see that $\cat{Ind}_\kappa(\Cc)$ has $\kappa$-small colimits of representable presheaves.
	
	Let's first construct the coproduct $\coprod_{s\in S}y_s$ for a discrete set $S$ of cardinality $\left|S\right|<\kappa$ and $y_s\in\cat{Ind}_\kappa(\Cc)$. Write $y_s\simeq \colimit_{j\in \Jj_s}x_{j,s}$ for some filtered $\infty$-category $\Jj_s$ and representable presheaves $x_{j,s}$. Observe that arbitrary products of $\kappa$-filtered $\infty$-categories is $\kappa$-filtered again. Furthermore, if $\Ii$ is any $\kappa$-filtered $\infty$-category, then $\colimit_{i\in\Ii}y_s\simeq y_s$ by \cref{lem:ContractibleColimit} and \cref{lem:FilteredCofinal}. Hence  $\colimit_{(i,j)\in \Ii\times\Jj_s}x_{j,s}\simeq y_s$ by \cref{lem:ColimitManipulations}. So we may replace $\Jj_s$ by $\Ii\times\Jj_s$ for any $\kappa$-filtered $\Ii$. In particular, we may replace $\Jj_s$ by $\Jj\coloneqq \prod_{s\in S}\Jj_s$ and thus we may assume that the diagrams $\Jj_s$ coincide for all $s\in S$. Then \cref{lem:ColimitManipulations} shows
	\begin{equation*}
		\colimit_{j\in\Jj}\coprod_{s\in S}x_{j,s}\simeq \coprod_{s\in S}\colimit_{j\in\Jj}x_{j,s}\simeq \coprod_{s\in S}y_s\,,
	\end{equation*}
	provided any of these colimits exists. But $\coprod_{s\in S}x_{j,s}$ exists for all $j\in \Jj$ because $\cat{Ind}_\kappa(\Cc)$ has $\kappa$-small coproducts of representable presheaves, and then $\colimit_{j\in\Jj}\coprod_{s\in S}x_{j,s}$ exists because $\cat{Ind}_\kappa(\Cc)$ has all $\kappa$-filtered colimits. This shows that $\cat{Ind}_\kappa(\Cc)$ has $\kappa$-small coproducts.
	
	It remains to construct pushouts. Fix a span $y\leftarrow x\rightarrow z$. Let's first construct the pushout in the case where $x$ is representable. Write $y\simeq \colimit_{j\in\Jj} y_j$ and $z\simeq \colimit_{k\in\Kk}z_k$, where $\Jj$ and $\Kk$ are $\kappa$-filtered and $y_j$, $z_k$ are representable presheaves. Since $x$ is representable and thus $\kappa$-compact, \cref{lem:HomotopyGroupsFilteredColimits} implies  $\pi_0\Hom_{\cat{Ind}_\kappa(\Cc)}(x,y)\simeq \colimit_{j\in\Jj}\pi_0\Hom_{\cat{Ind}_\kappa(\Cc)}(x,y_j)$. Hence $x\rightarrow y$ factors through $x\rightarrow y_{j_0}$ for some $j_0\in \Jj$. By \cref{lem:FilteredCofinal}, we can replace $\Jj$ by $\Jj_{j_0/}$ and thus assume that $\Jj$ contains an initial element $j_0$ such that $x\rightarrow y$ is induced by a map $x\rightarrow y_{j_0}$. The same argument applies to $x\rightarrow z$. Furthermore, as above, we can replace $\Jj$ and $\Kk$ by $\Jj\times \Kk$ and thus assume $\Jj=\Kk$. Finally, we have $\colimit_{j\in\Jj}x\simeq x$ by \cref{lem:ContractibleColimit} and \cref{lem:FilteredCofinal}. Hence, using \cref{lem:ColimitManipulations}, we can construct the desired pushout as
	\begin{equation*}
		\colimit_{j\in\Jj}(y_j\cup_xz_j)\simeq \colimit_{j\in \Jj}y_j\cup_{\colimit_{j\in\Jj}x}\colimit_{j\in\Jj}z_j\simeq y\cup_xz\,.
	\end{equation*}
	Here $y_j\cup_xz_j$ exists since $\cat{Ind}_\kappa(\Cc)$ has pushouts of representable presheaves, as we've noted above, and then $\colimit_{j\in\Jj}(y_j\cup_xz_j)$ exists because $\cat{Ind}_\kappa(\Cc)$ has $\kappa$-filtered colimits. This finishes the case where $x$ is representable. In the general case, write $x\simeq\colimit_{j\in\Jj}x_j$, where $\Jj$ is $\kappa$-filtered and $x_j$ are representable presheaves. By an argument we've seen several times, $y\simeq \colimit_{j\in\Jj}y$ and $z\simeq \colimit_{j\in\Jj}z$. Then
	\begin{equation*}
		\colimit_{j\in\Jj}(y\cup_{x_j}z)\simeq \colimit_{j\in \Jj}y\cup_{\colimit_{j\in\Jj}x_j}\colimit_{j\in\Jj}z\simeq y\cup_xz\,.
	\end{equation*}
	Here the pushouts $y\cup_{x_j}z$ exist by the representable case and then $\colimit_{j\in\Jj}(y\cup_{x_j}z)$ exists because $\cat{Ind}_\kappa(\Cc)$ has $\kappa$-filtered colimits. This finishes the proof that $\cat{Ind}_\kappa(\Cc)$ has pushouts.
\end{proof}
Finally, we can state and prove the adjoint functor theorem. The original version is of course Lurie's \cite[Corollary~\HTTthm{5.5.2.9}]{HTT}. Our version is slightly more general and is taken from Markus Land's book \cite[Theorems~5.2.2 and~5.2.14]{Land}. %
\begin{satanicthm}[Adjoint functor theorem]\label{thm:AdjointFunctorTheorem}
	Let $F\colon \Cc\rightarrow\Dd$ be a functor between locally small $\infty$-categories.
	\begin{alphanumerate}
		\item Assume that $\Cc$ and $\Dd$ have all colimits and $\Cc$ is generated under colimits by an essentially small sub-$\infty$-category $\Cc_0\subseteq\Cc$. Then $F$ admits a right adjoint $G\colon \Dd\rightarrow\Cc$ if and only if $F$ preserves colimits.\label{enum:AdjointFunctorTheoremLeft}
		\item Assume that $\Cc$ and $\Dd$ have all limits, that $\Cc$ is accessible, and that for every object $y\in\Dd$ there exists a regular cardinal $\kappa_y$ such that $y$ is $\kappa_y$-compact. If there exists a regular cardinal $\kappa$ such that $F$ preserves limits as well as $\kappa$-filtered colimits, then $F$ admits a left adjoint $G\colon \Dd\rightarrow\Cc$. The converse is true as well provided that $\Dd$ is accessible too.\label{enum:AdjointFunctorTheoremRight}
	\end{alphanumerate}
	Furthermore, in both \cref{enum:AdjointFunctorTheoremLeft} and \cref{enum:AdjointFunctorTheoremRight}, the conditions on $\Cc$ and $\Dd$ are automatically satisfied if $\Cc$ and $\Dd$ are presentable.\hfill\smash{\GrothendieckRightDevil}
\end{satanicthm}

Before we embark on the proof, let's draw a somewhat surprising corollary.
\begin{cor}\label{cor:PresentableComplete}
	Let $\Cc$ be a locally small $\infty$-category.
	\begin{alphanumerate}
		\item If $\Cc$ has all colimits and is generated under colimits by an essentially small sub-$\infty$-category $\Cc_0\subseteq \Cc$, then $\Cc$ has all limits too. In particular, presentable $\infty$-categories have all limits.\label{enum:PresentableComplete}
		\item If $\Cc$ is accessible and has all limits, then $\Cc$ has all colimits too. In particular, $\Cc$ is presentable.\label{enum:AccessibleCocomplete}
	\end{alphanumerate}
\end{cor}
\begin{proof}
	Let $\Ii$ be an essentially small $\infty$-category. Then $\Fun(\Ii,\Cc)$ is locally small by \cref{rem:FunLocallySmall}. It follows from \cref{lem:ColimitsInFunctorCategories} that $\const\colon \Cc\rightarrow\Fun(\Ii,\Cc)$ preserves all limits and colimits. In the situation of \cref{enum:PresentableComplete}, we may apply \cref{thm:AdjointFunctorTheorem}\cref{enum:AdjointFunctorTheoremLeft} to see that $\const$ has a right adjoint $\limit_\Ii\colon \Fun(\Ii,\Cc)\rightarrow\Cc$, as desired.
	
	In the situation of \cref{enum:AccessibleCocomplete}, we only need to check that every element of $\Fun(\Ii,\Cc)$ is $\tau$-compact for some sufficiently large regular cardinal $\tau$, for then $\const$ will have a left adjoint $\colimit_\Ii\colon \Fun(\Ii,\Cc)\rightarrow\Cc$ by \cref{thm:AdjointFunctorTheorem}\cref{enum:AdjointFunctorTheoremRight}. Say $\Cc$ is $\kappa$-compact. Then every $x\in\Cc$ can be written as a colimit of $\kappa$-compact objects by \cref{lem:KappaCompactlyGenerated}\cref{enum:DGeneratedUnderFilteredColimits}. If $\kappa_x$ is larger than $\kappa$ and the cardinality of the indexing diagram, then $x$ will be $\kappa_x$-compact, because $\kappa_x$-compact objects are closed under $\kappa_x$-small colimits (we've seen this argument several times in the proofs of \cref{lem:KappaCompactlyGenerated,lem:Presentable}). Now let $\alpha\colon \Ii\rightarrow\Cc$ be a functor. Let $\tau_\alpha$ be a regular cardinal such that $\TwAr(\Cc)$ is $\tau_\alpha$-small and $\tau_\alpha\geqslant \kappa_{\alpha(i)}$ for every $i\in\Ii$. Using that $\tau_\alpha$-small limits commute with $\tau_\alpha$-filtered colimits by \cref{lem:FilteredColimitsPreserveFiniteLimits} and that colimits in functor categories are computed pointwise by \cref{lem:ColimitsInFunctorCategories}, the formula from \cref{cor:HomInFunctorCats} shows that $\alpha$ is $\tau_\alpha$-compact.
\end{proof}
We start off the proof with two preparatory lemmas.
\begin{lem}\label{lem:TerminalInSlice}
	Let $F\colon \Cc\rightarrow\Dd$ be a functor between $\infty$-categories and let $y\in\Dd$ be an object. Then $y$ admits a right adjoint object $x\in\Cc$ under $F$ if and only if the slice $\infty$-category $\Cc_{/y}\simeq \Cc\times_{\Dd}\Dd_{/y}$ has a terminal object.
\end{lem}
\begin{proof}[Proof sketch]
	Let $x\in \Cc$ be an object and $c\colon F(x)\rightarrow y$ a morphism in $\Cc$. Then $x$ is a right adjoint object to $y$ under $F$, with counit $c$, if and only if the composition
	\begin{equation*}
		\Hom_\Cc\!\left(-,x\right)\overset{F}{\Longrightarrow} \Hom_\Dd\lef(F(-),F(x)\righ)\overset{c_*}{\Longrightarrow}\Hom_\Dd\lef(F(-),y\righ)
	\end{equation*}
	is an equivalence of functors. By \cref{thm:EquivalencePointwise}, this can be checked on objects. So choose $x'\in\Cc$. To check that $\Hom_\Cc(x,x')\rightarrow \Hom_\Dd(F(x'),y)$, it's enough by \cref{thm:Whitehead} to check that the fibres over every $\alpha\in\Hom_\Dd(F(x'),y)$ are contractible. So fix some $\alpha\colon F(x')\rightarrow y$. Using the fact that $\Hom$ animae in pullbacks of $\infty$-categories are the pullbacks of the respective $\Hom$ animae (which we'll prove in more generality in \cref{lem:HomInLimits}), one easily computes that the fibre $\Hom_\Cc(x',x)\times_{\Hom_\Dd(F(x'),y)}\{\alpha\}$ is equivalent to $\Hom_{\Cc_{/y}}\lef((x',\alpha\colon F(x')\rightarrow y),(x,c\colon F(x)\rightarrow y)\righ)$. So the fibres are all contractible if and only if $(x,c\colon F(x)\rightarrow y)$ is a terminal object of $\Cc_{/y}$.
\end{proof}
\begin{lem}\label{lem:TerminalObjectColimit}
	Let $\Cc$ be any \embrace{possibly large} $\infty$-category. Then $\Cc$ has a terminal object if and only if $\id_\Cc\colon \Cc\rightarrow\Cc$ has a colimit, in which case the terminal object is that colimit.
\end{lem}
\begin{proof}[Proof sketch]
	If $y\in\Cc$ is terminal, then $\{y\}\rightarrow\Cc$ is a right adjoint, hence cofinal by \cref{exm:Cofinal}\cref{enum:RightAdjointCofinal}. Hence $\colimit(\id_\Cc\colon \Cc\rightarrow\Cc)\simeq y$; in particular, the colimit exists.
	
	Conversely, assume the colimit exists, and let $u\colon \id_\Cc\Rightarrow \const y$ be the natural transformation exhibiting $y$ has the colimit. We wish to prove that $\Cc\shortdoublelrmorphism \{y\}$ is an adjunction. To this end, by \cref{lem:TriangleIdentities}, it suffices to construct the unit and the counit as well as to verify the triangle identities. We take $u$ to be our unit. The counit as well as the first triangle identity come for free since $\Fun(\Cc,\{y\})\simeq *$ and $\Fun(\{y\},\{y\})\simeq *$. By a quick unravelling, the second triangle identity comes down to proving that $u_y\colon y\rightarrow y$ is the identity on $y$. To this end, consider $u$ as a functor $u\colon \Delta^1\times\Cc\rightarrow\Cc$ and consider the composition $\sigma\coloneqq u\circ(\id_{\Delta^1}\times\Cc)\colon \Delta^1\times(\Delta^1\times\Cc)\rightarrow \Delta^1\times\Cc\rightarrow \Cc$. By \enquote{currying}, $\sigma$ corresponds to a functor $\Delta^1\times\Delta^1\rightarrow\Fun(\Cc,\Cc)$, or in other words, to a commutative square in $\Fun(\Cc,\Cc)$. By a somewhat confusing unravelling, that commutative square is
	\begin{equation*}
		\begin{tikzcd}[column sep=large]
			\id_\Cc\doublear["u"{black,above=0.1em}]{r}\doublear["u"'{black,left=0.1em}]{d}\drar[commutes] & \const y\doublear["\const u_y"{black,right=0.1em}]{d}\\
			\const y\doublear["\id_{\const y}"{black,above=0.1em}]{r} & \const y
		\end{tikzcd}
	\end{equation*}
	Thus, in the equivalence $\Hom_\Cc(y,y)\simeq \Hom_\Cc(\colimit_\Cc\id_\Cc,y)\simeq \Hom_{\Fun(\Cc,\Cc)}(\id_\Cc,y)$, both $\id_y$ and $u_y$ are mapped to $u\in\Hom_{\Fun(\Cc,\Cc)}(\id_\Cc,y)$. This proves $\id_y\simeq u_y$, as desired.
\end{proof}

\begin{proof}[Proof sketch of \cref{thm:AdjointFunctorTheorem}\cref{enum:AdjointFunctorTheoremLeft}]
	If $F$ admits a right adjoint, then $F$ preserves colimits by \cref{lem:AdjointsPreserveColimits}. Conversely, assume $F$ preserves colimits. Adjoints can be constructed pointwise by \cref{lem:Adjunction}, and thus by \cref{lem:TerminalInSlice}, it's enough to show that the slice $\infty$-category $\Cc_{/y}\simeq \Cc\times_\Dd\Dd_{/y}$ has a terminal object for every $y\in\Dd$. A straightforward generalisation of the arguments in the proof of \cref{cor:FunctorCategoriesPresentable} shows that $\Cc_{/y}$ has again all (small) colimits and is generated under colimits by its full sub-$\infty$-category $(\Cc_0)_{/y}\simeq \Cc_0\times_\Cc\Cc_{/y}$; this is the only time we use that $F$ preserves colimits. So we can replace $\Cc$ by $\Cc_{/y}$ and are thus reduced to showing that $\Cc$ has a terminal object.		
	
	%$F$ preserves colimits, we can use a similar (but dual) argument as in the proof of \cref{lem:ColimitsInSliceCategory}\cref{enum:LimitsInSlice} to show that $\Cc_{/y}$ has all (small) colimits. Furthermore, as we'll now explain a variation of the argument shows that $\Cc_{/y}$ is generated under colimits by its full sub-$\infty$-category $(\Cc_0)_{/y}\simeq \Cc_0\times_\Cc\Cc_{/y}$. Indeed, pick some $(x,\alpha\colon F(x)\rightarrow y)\in\Cc_{/y}$ and write $x\simeq\colimit_{i\in\Ii}x_i$ for some diagram $x_{(-)}\colon\Ii\rightarrow\Cc_0$. Applying $F$ to the natural transformation $u\colon x_{(-)}\Rightarrow \const x$ and composing with $\const\alpha\colon \const F(x)\Rightarrow \const y$ yields a transformation $\const\alpha\circ F(u)\colon F(x_{(-)})\Rightarrow \const y$, which in turn defines functors $(F(x_{(-)})\rightarrow y)\colon \Ii\rightarrow \Dd_{y/}$ and $(x_{(-)},F(x_{(-)}\rightarrow y))\colon \Ii\rightarrow \Cc_0\times_\Dd\Dd_{y/}$. Then a straightforward argument shows $(x,\alpha\colon F(x)\rightarrow y)\simeq \colimit_{i\in\Ii}(x_i,F(x_i)\rightarrow y)$ in $\Cc_{/y}$, as desired. Finally, it's easy to show that $(\Cc_0)_{/y}$ is essentially small. Clearly, $(\Cc_0)_{/y}$ is locally small, because \cref{cor:HomInSliceCategories} and the upcoming \cref{lem:HomInLimits} show that $\Hom_{(\Cc_0)_{/y}}$ can be written as a pullback involving $\Hom_\Cc$ and $\Hom_\Dd$, both of which are essentially small by the assumption that $\Cc$ and $\Dd$ are locally small. So its enough to show that $\pi_0\core((\Cc _0)_{/y})$ is a set. This easily follows from $\Cc$ and $\Dd$ being locally small, so that there can't be \enquote{too many} equivalence classes of pairs $(x_0,\alpha_0\colon F(x_0)\rightarrow y)$ where $x_0\in\Cc_0$. We leave the argument to you.
	
	By \cref{lem:TerminalObjectColimit}, we must show that $\id_\Cc\colon \Cc\rightarrow\Cc$ admits a colimit. Since $\Cc$ has all small colimits, it will be enough to show that $\Cc$ admits a cofinal functor from a small $\infty$-category. Note that this step requires some set-theoretic care, since it's not so clear why \cref{thm:JoyalsQuillenA} would be applicable to colimits with potentially large indexing $\infty$-categories. This problem can be solved by considering universes, and with some more effort even in ZFC; we'll ignore it in the following.
	
	Since $\Cc$ has all small colimits, the colimit $t\coloneqq\colimit(\Cc_0\rightarrow\Cc)$ exists. Note that for every $x\in\Cc$, a morphism $x\rightarrow t$ exists. Indeed, by assumption, we can write $x$ as a colimit $x\simeq \colimit(\Ii\rightarrow\Cc_0\rightarrow\Cc)$ and then we can consider the morphism $x\simeq \colimit(\Ii\rightarrow\Cc_0\rightarrow\Cc)\rightarrow\colimit(\Cc_0\rightarrow\Cc)\simeq t$ using functoriality of colimits, see \cref{lem:ColimitsFunctorial}. Now let $\Tt\subseteq \Cc$ be the full sub-$\infty$-category spanned by $t$ (note that $\Tt$ is not just $\{t\}$, since we include all non-identity endomorphisms of $t$ as well). Since $\Cc$ is locally small, $\Tt$ must be essentially small. We claim that $\Tt\rightarrow\Cc$ is cofinal. To this end, we'll show that $\Tt\times_\Cc\Cc_{x/}$ is filtered; then \cref{lem:FilteredCofinal} will show that the condition from \cref{thm:JoyalsQuillenA}\cref{enum:WeaklyContractible} is satisfied. Let $\alpha\colon \Ii\rightarrow\Tt\times_\Cc\Cc_{x/}$ be a functor from any small $\infty$-category. If $\ov\alpha\colon \Ii\rightarrow\Tt\times_\Cc\Cc_{x/}\rightarrow\Cc$ denotes the underlying functor, then $\alpha$ is equivalently given by a natural transformation $\const x\Rightarrow \ov\alpha$ such that $\ov\alpha$ takes values in the full sub-$\infty$-category $\Tt\subseteq\Cc$. Since $\Cc$ has small colimits, $x\simeq\colimit_{i\in\Ii}\ov\alpha(i)$ exists in $\Cc$. As observed above, there exists a morphism $x\rightarrow t$. Composing the colimit transformation $\ov\alpha\Rightarrow \const x$ with $\const x\Rightarrow t$ yields a natural transformation $\ov\alpha\Rightarrow \const t$, or equivalently, a functor $\ov\alpha^\triangleright\colon \Ii^\triangleright \rightarrow \Cc$. By construction, $\ov\alpha^\triangleright$ takes values in the full sub-$\infty$-category $\Tt\subseteq\Cc$. Composing with $\const x\Rightarrow \ov\alpha$ provides a functor $\alpha^\triangleright\colon \Ii^\triangleright\rightarrow\Tt\times_\Cc\Cc_{x/}$, as desired. This finishes the proof that $\Tt\times_\Cc\Cc_{x/}$ is filtered.%So we're done!
\end{proof}
Our proof of \cref{thm:AdjointFunctorTheorem}\cref{enum:AdjointFunctorTheoremRight} will again be preceded by two preparatory lemmas.
\begin{lem}[\enquote{Right adjoints preserve sufficiently filtered colimits}]\label{lem:RightAdjointsAccessible}
	Let $G\colon \Dd\rightarrow\Cc$ be a functor between accessible $\infty$-categories. If $G$ admits a left adjoint $F$, then $G$ preserves $\tau$-filtered colimits for sufficiently large regular cardinals $\tau$.
\end{lem}
\begin{proof}
	Choose regular cardinals $\kappa$ and $\lambda$ such that $\Cc$ is $\kappa$-accessible and $\Dd$ is $\lambda$-accessible. By \cref{lem:KappaCompactlyGenerated}, we may identify $\Cc$ and $\Dd$ with $\cat{Ind}_\kappa(\Cc^\kappa)$ and $\cat{Ind}_\lambda(\Dd^\lambda)$, respectively. First note that for every $y\in\Dd$ there exists a regular cardinal $\lambda_y$ such that $y$ is $\lambda_y$-compact. Indeed, we may write $y$ has a colimit of $\lambda$-compact objects, and then it suffices to choose $\lambda_y\geqslant \lambda$ sufficiently large so that the indexing diagram of the colimit is $\lambda_y$-small. Since $\Cc^\kappa$ is essentially small, as we've seen in the proof of \cref{lem:KappaCompactlyGenerated}, we may choose a regular cardinal $\tau\geqslant \kappa$ such that $F(x)$ is $\tau$-compact for all $x\in\Cc^\kappa$. We claim that $G$ preserves $\tau$-filtered colimits. Since $\Cc\simeq\cat{Ind}_\kappa(\Cc^\kappa)\subseteq \PSh(\Cc^\kappa)$, the functors $\Hom_\Cc(x,-)\colon \Cc\rightarrow \cat{An}$ for $x\in\Cc^\kappa$ are jointly conservative and preserve $\kappa$-filtered and thus also $\tau$-filtered colimits. So it's enough to show that $\Hom_\Cc(x,G(-))$ preserves $\tau$-filtered colimits. But $\Hom_\Cc(x,G(-))\simeq \Hom_\Dd(F(x),-)$ and $F(x)$ is $\tau$-compact by construction.
\end{proof}
\begin{lem}\label{lem:Accessible}
	Let $\Cc$ be a $\kappa$-accessible $\infty$-category. Then $\Cc$ is also $\tau$-accessible for every sufficiently large regular cardinal $\tau$.
\end{lem}
\begin{proof}[Proof sketch]
	By \cref{lem:KappaCompactlyGenerated}\cref{enum:DGeneratedUnderFilteredColimits} will be enough to show that $\Cc$ is generated under $\tau$-filtered colimits by $\Cc^\tau$, where $\tau$ is a sufficiently large regular cardinal that will be chosen at the end of the proof. Every $x\in\Cc$ can be written as $x\simeq \colimit_{j\in\Jj}x_j$, where $\Jj$ is $\kappa$-filtered and $x_j\in\Cc^\kappa$. We'll rewrite this as a $\tau$-filtered colimit of $\tau$-compact objects. First, by \cref{blackbox:Cofinal} in the proof of \cref{lem:HomotopyGroupsFilteredColimits}, we find a cofinal functor $J\rightarrow\Jj$ from a directed partially ordered set $J$. Note that $J$ is automatically a $\kappa$-filtered $\infty$-category by the criterion from \cref{lem:FilteredColimitsPreserveFiniteLimits}. We'll show that $J$ can be written as a colimit $J\simeq \colimit_{i\in I}J_i$ in $\cat{Cat}_\infty$, where $I$ is a $\tau$-filtered directed partially ordered set and $J_i\subseteq J$ are essentially $\tau$-small $\kappa$-filtered partially ordered subsets. If we can do this, we're done. Indeed, by \cref{lem:ColimitManipulations}\cref{claim:AssembleColimits}, we may then write $x\simeq \colimit_{i\in I}\colimit_{j\in J_i}x_j$. Each $\colimit_{j\in J_i}x_j$ exists, as $\Cc$ admits $\kappa$-filtered colimits by \cref{lem:Ind}\cref{enum:IndGeneratedUnderFilteredColimits}. Furthermore, $\colimit_{j\in J_i}x_j$ is $\tau$-compact because each $x_j$ is $\kappa$-compact, hence $\tau$-compact, and $\tau$-compact objects are stable under $\tau$-small colimits by an easy application of \cref{lem:FilteredColimitsPreserveFiniteLimits}.
	
	To write $J$ as such a colimit, let $\Pp^\tau(J)$ be the partially ordered set of subsets $S\subseteq J$ of cardinality $\left|S\right|<\tau$. Note that $\Pp^\tau(J)$ is $\tau$-filtered as an $\infty$-category. Indeed, using \cref{lem:SimplicialHoNerveAdjunction}, it's enough to show that $\Pp^\tau(J)$ is $\tau$-filtered as an ordinary category, which is true since we can just take unions of $<\tau$ subsets of cardinality $<\tau$. Each $S\in\Pp^\tau(J)$ can be identified with the full subcategory $J[S]\subseteq J$ spanned by $S$ and we have $J\simeq \colimit_{S\in\Pp^\tau(J)}J[S]$ in $\cat{Cat}_\infty$. One way to prove this would be to use that filtered colimits in $\cat{Cat}_\infty$ can be computed on the level of simplicial sets (see the proof of \cref{cor:AnPresentable}); then the desired equivalence is straightforward. For an alternative, model-independent argument, let $\Uu$ be the unstraightening of the functor $\Pp^\tau(J)\rightarrow \cat{Cat}_\infty$ sending $S\mapsto J[S]$. Then $\Uu$ is an ordinary category and can be easily described explicitly. The same argument as in the proof of claim~\cref{claim:FilteredCoproduct} in the proof of~\cref{lem:KappaCompactlyGenerated} shows that $\Uu\rightarrow J$ is cofinal. By \cref{lem:ColimitsInAnima}, $\colimit_{S\in\Pp^\tau(J)}J[S]$ is a localisation of $\Uu$. Since localisations are cofinal by  \cref{exm:Cofinal}\cref{enum:LocalisationsCofinal}, we conclude that $\colimit_{S\in\Pp^\tau(J)}J[S]\rightarrow J$ is cofinal too. This is not quite what we wanted, but it's enough for our arguments to work. Now we claim:
	\begin{alphanumerate}\itshape
		\item[\boxtimes] \!There exists a partially ordered subset $I\subseteq \Pp^\tau(J)$ such that $J[S]$ is $\kappa$-filtered for every $S\in I$ and such that the inclusion $I\rightarrow\Pp^\tau(J)$ has a left adjoint $L\colon \Pp^\tau(J)\rightarrow I$.\label{claim:Filterification}
	\end{alphanumerate}
	Since right adjoints are cofinal by \cref{exm:Cofinal}\cref{enum:RightAdjointCofinal}, we also get $J\simeq \colimit_{S\in I}J[S]$. Furthermore, this cofinality implies that $I$ is $\tau$-filtered again, because it satisfies the criterion from \cref{lem:FilteredColimitsPreserveFiniteLimits}. So once we know \cref{claim:Filterification}, we're done.
	
	For every equivalence class of functors $\alpha\colon \Ii\rightarrow J$ from an essentially $\kappa$-small $\infty$-category $\Ii$, choose an extension $\alpha^\triangleright\colon \Ii^\triangleright\rightarrow J$. Let $S_0\in \Pp^\tau(J)$. Let $S_1\subseteq J$ be obtained from $S_0$ by adjoining the \enquote{tip of the cone} for every $\alpha^\triangleright \colon \Ii^\triangleright\rightarrow J$ such that $\alpha\colon \Ii \rightarrow J$ factors through $J[S_0]\rightarrow J$. If $\tau$ is larger than the set of equivalence classes of essentially $\kappa$-small $\infty$-categories, then $S_1$ will have cardinality $\left|S_1\right|<\tau$ again. By transfinite induction, we can repeat this construction $\kappa$ many times. The result is a subset $S_\kappa\subseteq J$ such that $\left|S_\kappa\right|<\tau$ and $J[S_\kappa]$ is $\kappa$-filtered. If we put $L(S_0)\coloneqq S_\kappa$, then $L\colon \Pp^\tau(J)\rightarrow \Pp^\tau(J)$ is a functor satisfying $L\circ L=L$ (we really get an equality, not just an equivalence). Thus, if $I\subseteq \Pp^\tau(J)$ is the image of $L$, then an easy argument shows that $L\colon \Pp^\tau(J)\rightarrow I$ is indeed left adjoint to the inclusion (bear in mind that we're working with ordinary categories here, so constructing functors and adjunctions can be done by hand). Therefore, the conditions from \cref{claim:Filterification} are satisfied.
\end{proof}
\begin{proof}[Proof sketch of \cref{thm:AdjointFunctorTheorem}\cref{enum:AdjointFunctorTheoremRight}]
	By the dual of \cref{lem:TerminalInSlice}, it's enough to show that the slice $\infty$-category $\Cc_{y/}\simeq \Cc\times_\Dd\Dd_{y/}$ has an initial object for all $y\in\Dd$. By the dual of \cref{lem:TerminalObjectColimit}, this is equivalent to showing that $\id_{\Cc_{y/}}\colon \Cc_{y/}\rightarrow\Cc_{y/}$ admits a limit. A straightforward generalisation of \cref{lem:ColimitsInSliceCategory}\cref{enum:LimitsInSlice} shows that $\Cc_{y/}$ has all (small) limits; this argument crucially uses that $F$ preserves limits. So it will be enough to construct a final functor from an essentially small $\infty$-category into $\Cc_{y/}$.
	
	By assumption and \cref{lem:Accessible} we may choose a sufficiently large regular cardinal $\kappa$ such that $\Cc$ is $\kappa$-accessible, $F$ preserves $\kappa$-filtered colimits, and $y$ is $\kappa$-compact. Let $\Tt\subseteq\Cc_{y/}$ be the full sub-$\infty$-category spanned by those $(x, \alpha\colon y\rightarrow F(x))$ where $x$ is $\kappa$-compact. By an easy argument, the likes of which we've seen several times by now, $\Tt$ is essentially small. We claim that for every $z\in \Cc_{y/}$ there is an element $t\in \Tt$ and a morphism $t\rightarrow z$ in $\Cc_{y/}$. If we can show this, then a similar argument as in the proof of \cref{enum:AdjointFunctorTheoremLeft} shows that $\Tt\rightarrow\Cc_{y/}$ is final. Indeed, we'll show that $\Tt\times_{\Cc_{y/}}(\Cc_{y/})_{/w}$ is cofiltered for every $w\in \Cc_{y/}$, which will imply finality by the dual of \cref{thm:JoyalsQuillenA}\cref{enum:WeaklyContractible} and the dual of \cref{lem:FilteredCofinal}. So let $\alpha\colon \Ii\rightarrow \Tt\times_{\Cc_{y/}}(\Cc_{y/})_{/w}$ be a functor from a small $\infty$-category $\Ii$. Since $\Cc_{y/}$ admits small limits, the underlying functor $\ov\alpha\colon \Ii\rightarrow \Cc_{y/}$ admits a limit $z\simeq \limit_{i\in\Ii}\ov\alpha(i)$. Choosing a morphism $t\rightarrow z$ for some $t\in \Tt$, we get natural transformations $\const t\Rightarrow\const z\Rightarrow\ov\alpha$. The composition $\const t\Rightarrow \ov\alpha$ induces an extension $\alpha^\triangleleft\colon \Ii^\triangleleft\rightarrow \Tt\times_{\Cc_{y/}}(\Cc_{y/})_{/w}$ of $\alpha$, as desired.
	
	It remains to show our claim that for every $z\in\Cc_{y/}$ there exists a moprhism $t\rightarrow z$ for some $t\in\Tt$. Write $z$ as a pair $(x,\beta\colon y\rightarrow F(x))$ for some $x\in \Cc$. Since $\Cc$ is $\kappa$-accessible, we can write $x$ as a $\kappa$-filtered colimit $x\simeq \colimit_{j\in\Jj}x_j$ for some $x_j\in \Cc^\kappa$. Since $F$ preserves $\kappa$-filtered colimits, $F(x)\simeq \colimit_{j\in\Jj}F(x_j)$. Since $y$ is $\kappa$-compact by assumption and $\pi_0$ commutes with colimits by \cref{lem:HomotopyGroupsFilteredColimits}, the canonical map
	\begin{equation*}
		\colimit_{j\in\Jj}\pi_0\Hom_\Dd\lef(y,F(x_j)\righ)\overset{\cong}{\longrightarrow}\pi_0\Hom_\Dd\lef(y,F(x)\righ)
	\end{equation*}
	is a bijection. Hence $\beta\colon y\rightarrow F(x)$ factors over a map $\beta_j\colon y\rightarrow F(x_j)$ for some $j\in\Jj$. Then $x_j\rightarrow\colimit_{j\in\Jj}x_j\simeq x$ induces a morphism $(x_j,\beta_j\colon y\rightarrow F( x_j))\rightarrow (x,\beta\colon y\rightarrow F(x))$ in $\Cc_{y/}$. Since $x_j$ is $\kappa$-compact, we see that $(x_j,\beta_j\colon y\rightarrow F( x_j))\in\Tt$. This proves that there exists a morphism $t\rightarrow z$ for some $t\in\Tt$ and thus we've finished the proof that $F\colon \Cc\rightarrow \Dd$ has a left adjoint.
	
	To prove the converse in the case where $\Cc$ and $\Dd$ are both accessible, just observe that if $F$ admits a left adjoint, then $F$ preserves all limits by \cref{lem:AdjointsPreserveColimits} and also all sufficiently filtered colimits by \cref{lem:RightAdjointsAccessible}. Finally, to show that the conditions in \cref{enum:AdjointFunctorTheoremLeft} and \cref{enum:AdjointFunctorTheoremRight} are satisfied in the case where $\Cc$ and $\Dd$ are presentable, the only non-obvious assertion is that for every $y\in\Dd$ there exists a regular cardinal $\kappa_y$ such that $y$ is $\kappa_y$-compact. But we've seen this in the proof of \cref{lem:RightAdjointsAccessible} already.
\end{proof}