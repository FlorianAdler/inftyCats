\section{\texorpdfstring{$\infty$}{Infinity}-Category theory}\label{sec:InftyCategoryTheory}

Armed with Lurie's straightening equivalence and the quasi-categorical Yoneda lemma, we will spend \crefrange{subsec:Adjunctions}{subsec:KanExtensions} redeveloping the theory from \cref{sec:CategoryTheory} (and more) in the setting of quasi-categories. In \cref{subsec:EilenbergMacLane} we will see a first major application to topology. After that, there will be a lengthy appendix (\crefrange{subsec:EssentiallySmall}{subsec:PrL}) in which we discuss presentable $\infty$-categories and prove Lurie's adjoint functor theorem.

Even though, implicitly, we work with quasi-categories, our arguments in \crefrange{subsec:Adjunctions}{subsec:KanExtensions} will be almost entirely model-independent; the same is true, at least in large parts, for \crefrange{subsec:EilenbergMacLane}{subsec:PrL}. So from now on, instead of \emph{quasi-categories}, we'll simply write \emph{$\infty$-categories}. We'll consider ordinary categories as $\infty$-categories via the nerve construction, but we'll always suppress $\N$ in our notation.\footnote{In particular, the partially ordered set $[n]$ is now identified with its nerve, the quasi-category $\Delta^n$. However, we'll continue to write $\Delta^n$. There are at least to reasons for this. First, I believe $\Delta^n$ is notationally easier to parse for a human brain. Second, once we reach \cref{sec:TowardsSpectra}, we'll consider simplicial objects in $\infty$-categories, and it will become necessary to keep the $\infty$-category $\Delta^n$ and the object $[n]\in\IDelta$ notationally separate.} Furthermore, we'll write $\Fun(\Cc,\Dd)$ instead of $\F(\Cc,\Dd)$ for $\infty$-categories $\Cc$ and $\Dd$. We'll only switch back to the old terminology in the few instances where non-model-independent arguments are used. I believe these few exceptions could easily be treated in any other model of $\infty$-categories as well. Also recall from \cref{par:ModelIndependence} that many constructions and results so far (like cocartesian/left fibrations) can be reformulated in a model-independent fashion, and this is how we're going to use them.
%
%, which 
%
%the notions of cocartesian and left fibrations as well as Lurie's straightening/unstraightening equivalence can be reformulated in a model-independent way, and this is the way in which we're going to use them.

\subsection{Adjunctions}\label{subsec:Adjunctions}
\begin{defi}\label{def:Adjunction}
	Let $L\colon \Cc\rightarrow \Dd$ be a functor of $\infty$-categories.
	\begin{alphanumerate}
		\item Let $y\in \Dd$. An object $x\in \Cc$ is a \emph{right adjoint object to $y$ under $L$} if there exists an equivalence
		\begin{equation*}
			\Hom_\Cc(-,x)\simeq \Hom_\Dd\bigl(L(-),y\bigr)
		\end{equation*}
		in the functor category $\Fun(\Cc^\op,\cat{An})$.
		\item A functor $R\colon \Dd\rightarrow \Cc$ is a \emph{right adjoint of $L$} if there exists an equivalence
		\begin{equation*}
			\Hom_\Cc\bigl(-,R(-)\bigr)\simeq \Hom_\Dd\bigl(L(-),-\bigr)
		\end{equation*}
		in the functor category $\Fun(\Cc^\op\times \Dd,\cat{Set})$. In this case we write $L\dashv R$.
	\end{alphanumerate}
\end{defi}
\begin{lem}[\enquote{Adjoints can be constructed pointwise}]\label{lem:Adjunction}
	A functor $L\colon \Cc\rightarrow \Dd$ has a right adjoint if and only if every $y\in \Dd$ has a right adjoint object $x\in \Cc$.
\end{lem}
\begin{proof}
	One implication is trivial: If $R\colon \Dd\rightarrow \Cc$ is a right adjoint of $L$, then $R(y)$ is a right adjoint object of $y$ for every $y\in \Dd$. For the other implication, consider $\Hom_\Dd(L(-),-)\colon \Cc^\op\times\Dd\rightarrow\cat{An}$ as a functor $\overline{R}\colon \Dd\rightarrow \Fun(\Cc^\op,\cat{An})$. Our assumption implies that $\overline{R}$ takes values in the image of the Yoneda embedding $\Yo_\Cc\colon \Cc\rightarrow\Fun(\Cc^\op,\cat{An})$; namely, $\overline{R}(y)\simeq \Hom_\Cc(-,x)$ if $x\in\Cc$ is a right adjoint object of $y\in\Dd$ under $L$. Since $\Yo_\Cc$ is an equivalence onto its image, we obtain a functor $R\colon \Dd\rightarrow \Cc$ with the required properties.
\end{proof}
\begin{exm}\label{exm:Adjunctions}
	It's clear that any adjunction of ordinary categories is also an adjunction of $\infty$-categories. Furthermore, we already know some non-trivial examples of adjunctions of $\infty$-categories:
	\begin{alphanumerate}
		\item The inclusion $\cat{An}\subseteq \cat{Cat}_\infty$ is fully faithful (by \cref{thm:CordierPorter} and \cref{cor:FIsKanComplex}) and has both adjoints: A right adjoint $\core\colon \cat{Cat}_\infty\rightarrow \cat{An}$ and a left adjoint $\abs*{\,\cdot\,}\colon \cat{Cat}_\infty\rightarrow \cat{An}$ sending $\Cc$ to $\abs*{\Cc}$, the localisation of $\Cc$ at all its morphisms.\label{enum:AnToCatInfty}
		\item For every $\infty$-category $\Cc$, the functor $-\times \Cc\colon \cat{Cat}_\infty\rightarrow \cat{Cat}_\infty$ has a right adjoint, which sends an $\infty$-category $\Dd$ to $\Fun(\Cc,\Dd)$.\label{enum:Currying}
	\end{alphanumerate}
	Both \cref{enum:AnToCatInfty} and \cref{enum:Currying} can easily be seen using \cref{lem:Adjunction}: For \cref{enum:AnToCatInfty}, it's enough to check that the functors $i\colon \core(\Dd)\rightarrow \Dd$ and $p\colon \Cc\rightarrow \abs*{\Cc}$ induce functorial equivalences
	\begin{equation*}
		i_*\colon \Hom_{\cat{An}}\bigl(-,\core(\Dd)\bigr)\overset{\simeq}{\Longrightarrow}\Hom_{\cat{Cat}_\infty}(-,\Dd)\,,\ \,p^*\colon \Hom_{\cat{An}}\bigl(\abs{\Cc},-\bigr)\overset{\simeq}{\Longrightarrow}\Hom_{\cat{Cat}_\infty}(\Cc,-)
	\end{equation*}
	via post- and precomposition, respectively. Indeed, equivalences can be checked pointwise by \cref{thm:EquivalencePointwise} and then we can apply \cref{lem:Localisation} (and a similar assertion for $\core$). For \cref{enum:Currying}, observe that we have an \emph{evaluation functor} $\ev\colon \Fun(\Cc,\Dd)\times\Cc\rightarrow \Dd$ for all $\infty$-categories $\Cc$ and $\Dd$. If we work with quasi-categories, this functor is simply given by the counit of the adjunction $-\times\Cc\colon\cat{sSet}\shortdoublelrmorphism \cat{sSet}\noloc \F(\Cc,-)$. Using \cref{lem:Adjunction}, it's enough to show that the composition
	\begin{equation*}
		\Hom_{\cat{Cat}_\infty}\bigl(-,\Fun(\Cc,\Dd)\bigr)\xRightarrow{-\times\Cc\vphantom{_y}}\Hom_{\cat{Cat}_\infty}\bigl(-\times\Cc,\Fun(\Cc,\Dd)\times\Cc\bigr)\overset{\ev_*\vphantom{_y}}{\Longrightarrow}\Hom_{\cat{Cat}_\infty}(-\times\Cc,\Dd)
	\end{equation*}
	is an equivalence. Again, \cref{thm:EquivalencePointwise} allows us to check this pointwise, and then \cref{thm:CordierPorter} reduces everything to the  adjunction $-\times\Cc\colon\cat{sSet}\shortdoublelrmorphism \cat{sSet}\noloc \F(\Cc,-)$ of ordinary categories.
	
	In particular, \cref{enum:Currying} allows us to define a functor $\Fun(\Cc,-)\colon \cat{Cat}_\infty\rightarrow \cat{Cat}_\infty$. With a little more work%
	\footnote{\label{footnote:Fun}Here's the argument: First, let $*\ \,*$ be the discrete category on two objects. In \cref{rem:ColimitFunctor} below, we'll construct a functor $\lim\colon \Fun(*\ \,*,\cat{Cat}_\infty)\rightarrow \cat{Cat}_\infty$. Under the identification $\Fun(*\ \,*,\cat{Cat}_\infty)\simeq \cat{Cat}_\infty\times\cat{Cat}_\infty$, this functor sends a pair $(\Cc,\Dd)$ of $\infty$-categories to the product $\Cc\times\Dd$. By \enquote{currying}, this functor corresponds to a functor $P\colon \cat{Cat}_\infty\rightarrow \Fun(\cat{Cat}_\infty,\cat{Cat}_\infty)$ sending $\Cc$ to $P(\Cc)\simeq -\times\Cc\colon \cat{Cat}_\infty\rightarrow \cat{Cat}_\infty$. By the way, this is also how you construct the functor $-\times\Cc$ in \cref{enum:Currying}. As we've seen above, $P(\Cc)$ is a left adjoint, and so $P$ factors through the full sub-$\infty$-category $\Fun^\L(\cat{Cat}_\infty,\cat{Cat}_\infty)\subseteq \Fun(\cat{Cat}_\infty,\cat{Cat}_\infty)$ spanned by the left adjoint functors. By \cref{cor:ExtractingAdjoints} below, extracting adjoints induces an equivalence of $\infty$-categories $\Fun^\L(\cat{Cat}_\infty,\cat{Cat}_\infty)\simeq \Fun^\R(\cat{Cat}_\infty,\cat{Cat}_\infty)^\op$. Thus, $P^\op$ can be regarded as a functor $P^\op\colon \cat{Cat}_\infty^\op\rightarrow \Fun^\R(\cat{Cat}_\infty,\cat{Cat}_\infty)$, sending $\Cc$ to $P^\op(\Cc)\simeq \Fun(\Cc,-)$. \enquote{Currying} back, we obtain the desired functor $\Fun(-,-)\colon \cat{Cat}_\infty^\op\times\cat{Cat}_\infty\rightarrow \cat{Cat}_\infty$.
	
	
	It's true that the functor $\core\Fun(-,-)$ agrees with $\Hom_{\cat{Cat}_\infty}(-,-)$, but this is not so easy to see (and we won't need it). One way would be to turn $\F(-,-)\colon \cat{QCat}\times\cat{QCat}\rightarrow \cat{QCat}$ into a Kan-enriched functor and show that $\N^\Delta(\F(-,-))$ agrees with $\Fun(-,-)$. This is easy since $\N^\Delta$ turns Kan-enriched adjunctions into adjunctions of $\infty$-categories. Then one has to check that $\Hom_{\cat{Cat}_\infty}(-,-)$ agrees with $\N^\Delta$ applied to $\core\F(-,-)\colon\cat{QCat}\times\cat{QCat}\rightarrow \cat{Kan}$. At least for \cref{con:HomTwAr}, this is done in \cite[Proposition~\HAthm{5.2.1.11}]{HA}.}, these functors can be assembled into a two-argument functor
	\begin{equation*}
		\Fun(-,-)\colon \cat{Cat}_\infty^\op\times\cat{Cat}_\infty\longrightarrow \cat{Cat}_\infty\,.
	\end{equation*}
\end{exm}
Next, we'll characterise adjunctions in terms of unit and counit.
\begin{con}\label{con:Unit}
	Let $L\colon \Cc \shortdoublelrmorphism \Dd\noloc R$ be an adjunction. We obtain a natural transformation $u\colon \id_\Cc\Rightarrow RL$, called the \emph{unit} of the adjunction, as follows: Consider the natural transformations $\Hom_\Cc(-,-)\Rightarrow \Hom_\Dd(L(-),L(-))\simeq \Hom_\Dd(-,RL(-))$, where the first one is induced by functoriality of $L$ and the second by the given adjunction. We can consider these as a natural transformation $\Yo_\Cc\Rightarrow \Yo_\Cc\circ RL$ in $\Fun(\Cc,\Fun(\Cc^\op,\cat{An}))$. Since $\Yo_\Cc$ is fully faithful, we obtain a natural transformation $u\colon \id_\Cc\Rightarrow RL$, as desired. Dually, there is also a \emph{counit} $c\colon LR\Rightarrow \id_\Dd$, as usual.
\end{con}
\begin{lem}[Triangle identities]\label{lem:TriangleIdentities}
	Let $L\colon \Cc\shortdoublelrmorphism \Dd\noloc R$ be an adjunction of $\infty$-categories. Then there are commutative diagrams 
	\begin{equation*}
		\begin{tikzcd}
			L \doublear["Lu"{black,above=0.1em}]{r}\doublear["i_L"'{black}]{dr} & LRL\doublear["cL"{black,right=0.1em}]{d}\dar[phantom,""{name=A}]\arrow[from=1-1,to=A,commutes,pos=0.7]\\
			& L
		\end{tikzcd}\quad\text{and}\quad
		\begin{tikzcd}
			R \doublear["uR"{black,above=0.1em}]{r}\doublear["i_R"'{black}]{dr} & RLR\doublear["Rc"{black,right=0.1em}]{d}\dar[phantom,""{name=A}]\arrow[from=1-1,to=A,commutes,pos=0.7]\\
			& R
		\end{tikzcd}
	\end{equation*}
	where $i_L$ and $i_R$ are pointwise\footnote{For ordinary categories, if two natural transformations agree pointwise, then they already agree. Indeed, in the ordinary world, a natural transformation is just pointwise data, subject to certain conditions. But this no longer works in $\infty$-land. Nevertheless, it should be true that $i_L$ and $i_R$ are just $\id_L$ and $\id_R$. If you know why, please tell me. In any case, this slightly weaker form of the triangle identities doesn't cause any problems.} the identity \embrace{so they are equivalences by \cref{thm:EquivalencePointwise}}. Conversely, if $L$, $R$ are functors and $u:\id_\Cc\Rightarrow RL$, $c\colon LR\Rightarrow \id_\Dd$ are natural transformations that fit into diagrams as above, where $i_L$ and $i_R$ are equivalences \embrace{not necessarily pointwise the identity}, then $L$ and $R$ determine an adjunction. 
\end{lem}
\begin{proof}
	First observe that the composites
	\begin{align*}
		\Hom_\Dd\bigl(L(-),-\bigr)\overset{R}{\Longrightarrow}&\Hom_\Dd\bigl(RL(-),R(-)\bigr)\overset{u^*}{\Longrightarrow}\Hom_\Cc\bigl(-,R(-)\bigr)\,,\\
		\Hom_\Cc\bigl(-,R(-)\bigr)\overset{L}{\Longrightarrow}&\Hom_\Dd\bigl(L(-),LR(-)\bigr)\overset{c_*}{\Longrightarrow}\Hom_\Dd\bigl(L(-),-\bigr)
	\end{align*}
	agree pointwise with the adjunction equivalence $\Hom_\Dd(L(-),-)\simeq \Hom_\Cc(-,R(-))$ and its inverse, and are thus an equivalences themselves (\cref{thm:EquivalencePointwise}). Indeed, Yoneda's lemma (see the dual of \cref{thm:Yoneda}) tells us that for every fixed $x\in\Cc$, a natural transformation $\Hom_\Dd(L(x),-)\Rightarrow \Hom_\Dd(x,R(-))$ is determined up to contractible choice by the image of $\id_{L(x)}\colon L(x)\rightarrow L(x)$, which is a morphism $x\rightarrow RL(x)$. For the adjunction equivalence $\Hom_\Dd(L(-),-)\simeq \Hom_\Cc(-,R(-))$, that morphism is the unit $u_x\colon x\rightarrow RL(x)$ by definition. But the image of $\id_{L(x)}\colon L(x)\rightarrow L(x)$ under $u^*\circ R$ is also $u_x$. The same argument applies to show that $c_*\circ L$ agrees pointwise with $\Hom_\Cc(-,R(-))\simeq \Hom_\Dd(L(-),-)$.
	
	Now to prove the triangle identities, consider the diagram of natural transformations
	\begin{equation*}
		\begin{tikzcd}
			\Hom_\Dd\bigl(L(-),-\bigr)\doublear["R"{black,above=0.1em}]{r}\ar["(cL)^*"{black,swap},white,double, double equal sign distance,-{implies[black]},ddr,bend right]\ar[ddr,dash,shift right=0.1em,bend right,shorten >=0.35ex]\ar[ddr,dash,shift left=0.1em,bend right,shorten >=0.59ex,shorten <=0.35ex]\ar[ddr,commutes,pos=0.4]& \Hom_\Cc\bigl(RL(-),R(-)\bigr)\doublear["u^*"{black,above=0.1em}]{r}\doublear["L"{black,left=0.1em}]{d}\drar[commutes] & \Hom_\Cc\bigl(-,R(-)\bigr)\doublear["L"{black,right=0.1em}]{d}\\
			& \Hom_\Dd\bigl(LRL(-),LR(-)\bigr)\doublear["(Lu)^*"{black,above=0.1em}]{r}\doublear["c_*"{black,left=0.1em}]{d}\drar[commutes] & \Hom_\Dd\bigl(L(-),LR(-)\bigr)\doublear["c_*"{black,right=0.1em}]{d}\\
			& \Hom_\Dd\bigl(LRL(-),-\bigr)\doublear["(Lu)^*"{black,above=0.1em}]{r} & \Hom_\Dd\bigl(L(-),-\bigr)
		\end{tikzcd}
	\end{equation*}
	The top right square commutes by functoriality of $L$, the bottom right square commutes since pre- and postcomposition commute, and the left cell commuting is a consequence of $c\colon LR\Rightarrow \id_\Dd$ being a natural transformation (see \cref{lem:HomNaturalTransformation}). Now walking around the bottom part of the diagram shows that $(cL)^*\circ (Lu)^*\colon \Hom_\Dd(L(-),-)\Rightarrow \Hom_\Dd(L(-),-)$ agrees with $c_*\circ L\circ u^*\circ R$, which is pointwise the identity as seen above. This establishes the first triangle identity; the second one is analogous.
	
	Conversely, if $L$, $R$ are functors and $u:\id_\Cc\Rightarrow RL$, $c\colon LR\Rightarrow \id_\Dd$ are natural transformations satisfying the triangle identities, then the commutative diagram above (together with its dual) shows that $u^*\circ R$ and $c_*\circ L$ induce equivalences between $\Hom_\Dd(L(-),-)$ and $\Hom_\Cc(-,R(-))$ (which are pointwise inverse if $i_L$ and $i_R$ are pointwise the identity).
\end{proof}
\begin{cor}\label{cor:FunctorCategoryAdjunctions}
	Let $L\colon \Cc \shortdoublelrmorphism \Dd\noloc R$ be an adjunction and let $\Ii$ be another category. Then the pre- and postcomposition functors determine adjunctions
	\begin{align*}
		L\circ-\colon \Fun(\Ii,\Cc)&\doublelrmorphism \Fun(\Ii,\Dd)\noloc R\circ -\,,\\
		{-}\circ {R}\colon \Fun(\Cc,\Ii)&\doublelrmorphism \Fun(\Dd,\Ii)\noloc {-}\circ {L}\,.
	\end{align*}
\end{cor}
\begin{proof}
	The proof of \cref{cor:1FunctorCategoryAdjunctions} can be copied verbatim.
\end{proof}
To finish this subsection about adjunctions, we connect adjunctions to the theory of straightening/unstraightening. This won't be needed in the rest of this text (so feel free to skip it), but it's nice to know and a standard fact in other treatments of $\infty$-categories.
\begin{lem}\label{lem:AdjunctionBicartesian}
	Let $F\colon \Cc\rightarrow \Dd$ be a functor of $\infty$-categories, corresponding to a functor $\Delta^1\rightarrow \cat{Cat}_\infty$ \embrace{see \cref{exm:SimplicialNerve}}, which in turn corresponds to a cocartesian fibration $p\colon \Uu\rightarrow \Delta^1$ by \cref{thm:Straightening}\cref{enum:CocartesianStraightening}. Then the following are equivalent:
	\begin{alphanumerate}
		\item $F$ admits a right adjoint $G\colon \Dd\rightarrow \Cc$.\label{enum:BicartesianAdjoint}
		\item \!The cocartesian fibration $p\colon \Uu\rightarrow \Delta^1$ is also a cartesian fibration.\label{enum:Bicartesian}
	\end{alphanumerate}
	Furthermore, in this case $G$ agrees with the functor classified by the cartesian straightening $\operatorname{St}^{(\mathrm{cart})}(p)\colon (\Delta^1)^\op\rightarrow \cat{Cat}_\infty$.
\end{lem}
\begin{proof}
	The crucial observation is the following claim:
	\begin{alphanumerate}\itshape
		\item[\boxtimes] The functor $\Hom_\Dd(F(-),-)\colon \Cc^\op\times\Dd\rightarrow \cat{An}$ is equivalent to the composition\label{claim:HomInUnstraightening}
		\begin{equation*}
			\Cc^\op\times\Dd\xrightarrow{i_0^\op\times i_1} \Uu^\op\times\Uu\xrightarrow{\Hom_\Uu}\cat{An}\,.
		\end{equation*}
		Here the first arrow is given by $i_0\colon \Cc\simeq \{0\}\times_{\Delta^1}\Uu\rightarrow \Uu$ and $i_1\colon \Dd\simeq \{1\}\times_{\Delta^1}\Uu\rightarrow \Uu$.
	\end{alphanumerate}
	To prove \cref{claim:HomInUnstraightening}, first observe that $i_0\colon \Cc\rightarrow \Uu$ and $i_1\colon \Dd\rightarrow\Uu$ are fully faithful. Indeed, $\Hom$ animae in pullbacks are given as pullbacks of $\Hom$ animae in the respective factors (which is straightforward to see from \cref{par:HomInQuasiCategories} and we'll see a more general assertion in \cref{lem:HomInLimits}\cref{enum:HomInLimits}). So pullbacks of fully faithful functors are still fully faithful and it remains to observe that $\{0\}\rightarrow \Delta^1$ and $\{1\}\rightarrow \Delta^1$ are both fully faithful, which is obvious. Now consider the following commutative square in $\cat{Cat}_\infty$:
	\begin{equation*}
		\begin{tikzcd}
			\Cc\eqar[r]\eqar[d]\drar[commutes] & \Cc\dar["F"]\\
			\Cc\rar["F"] & \Dd
		\end{tikzcd}
	\end{equation*}
	It can be viewed as a natural transformation $\const \Cc\Rightarrow F$ in $\Fun(\Delta^1,\cat{Cat}_\infty)$. After cocartesian unstraightening, it thus induces a morphism $\varphi\colon \Delta^1\times\Cc\rightarrow \Uu$ in $\cat{Cocart}(\Delta^1)$. Consider the composite
	\begin{equation*}
		(\Delta^1\times\Cc)^\op\times\Dd\xrightarrow{\varphi^\op\times i_1}\Uu^\op\times\Uu\xrightarrow{\Hom_\Uu}\cat{An}\,.
	\end{equation*}
	By unravelling the definitions and using that $i_1\colon \Dd\rightarrow \Uu$ is fully faithful, this composite can be regarded as a natural transformation $\eta\colon \Hom_\Uu(i_0(-),i_1(-))\Rightarrow \Hom_\Uu(i_1F(-),i_1(-))$ in $\Fun(\Cc^\op\times\Dd,\cat{An})$. We wish to show that $\eta$ is an equivalence of functors. By \cref{thm:EquivalencePointwise} this can be checked pointwise. So fix $x\in \Cc$, $y\in\Dd$. By unravelling the constructions $\eta_{(x,y)}\colon \Hom_\Uu(i_0(x),i_1(y))\rightarrow \Hom_\Uu(i_1F(x),i_1(y))$ is given by precomposition with the morphism $\varphi_x\colon i_0(x)\rightarrow i_1F(x)$ in $\Uu$. As we've seen in \cref{par:StraighteningMotivation}, $\varphi_x$ is a cocartesian morphism. Since $\Hom_{\Delta^1}(1,1)\simeq \Hom_{\Delta^1}(0,1)$, \cref{lem:CocartesianMorphisms} implies that precomposition with $\varphi_x$ must be an equivalence. Thus $\eta$ is an equivalence of functors, as desired. To finish the proof of \cref{claim:HomInUnstraightening}, it remains to observe $\Hom_\Uu(i_1F(-),-)\simeq \Hom_\Dd(F(-),-)$ as we've checked above that $i_1\colon \Dd\rightarrow \Uu$ is fully faithful.
	
	Now assume that $p\colon \Uu\rightarrow \Delta^1$ is a cartesian fibration too and let $G\colon \Dd\rightarrow \Cc$ correspond to $\operatorname{St}^{(\mathrm{cart})}(p)\colon (\Delta^1)^\op\rightarrow \cat{Cat}_\infty$. Then \cref{claim:HomInUnstraightening} and its dual provide an equivalences of functors $\Hom_\Dd(F(-),-)\simeq \Hom_\Uu(i_0(-),i_1(-))\simeq \Hom_\Cc(-,G(-))$, so $F$ and $G$ are adjoints. This proves \cref{enum:Bicartesian} $\Rightarrow$ \cref{enum:BicartesianAdjoint}.
	
	Conversely, suppose $G\colon \Dd\rightarrow \Cc$ is a right adjoint of $F$. Fix $y\in \Dd$. Then \cref{claim:HomInUnstraightening} and the fact that $i_0$ is fully faithful shows
	\begin{equation*}
		\Hom_\Uu\bigl(i_0(-),i_0G(y)\bigr)\simeq \Hom_\Cc\bigl(-,G(y)\bigr)\simeq \Hom_\Dd\bigl(F(-),y\bigr)\simeq \Hom_\Uu\bigl(i_0(-),i_1(y)\bigr)\,.
	\end{equation*}
	The image of $\id_{i_0G(y)}$ defines a morphism $\psi_y\colon i_0G(y)\rightarrow i_1(y)$ in $\Uu$. Furthermore, Yoneda's lemma (or more precisely, the dual of \cref{thm:Yoneda}) shows that any natural transformation $\Hom_\Cc(-,G(y))\simeq \Hom_\Uu(i_0(-),i_0G(y))\Rightarrow \Hom_\Uu(i_0(-),i_1(y))$ is uniquely determined by the image of $\id_{G(y)}$. That uniqueness ensures that the chain of equivalences above is must be given by postcomposition with $\psi_y$. Hence the dual of \cref{lem:CocartesianMorphisms} shows that $\psi_y$ is a $p$-cartesian morphism and we have constructed a sufficient supply of $p$-cartesian lifts. This finishes the proof of \cref{enum:BicartesianAdjoint} $\Rightarrow$ \cref{enum:Bicartesian}.
\end{proof}
\begin{cor}[\enquote{Extracting adjoints is functorial}]\label{cor:ExtractingAdjoints}
	Let $\Fun^\L,\Fun^\R\subseteq \Fun$ denote the full sub-$\infty$-categories spanned by the left/right adjoint functors and let $\cat{Cat}_\infty^\L,\cat{Cat}_\infty^\R\subseteq \cat{Cat}_\infty$ be the non-full sub-$\infty$-categories \embrace{in the sense of \cref{par:SubQuasiCategories}} spanned by all objects but only the left/right adjoint functors.
	\begin{alphanumerate}
		\item For all $\infty$-categories $\Cc$ and $\Dd$, sending a left adjoint functor $L\colon \Cc\rightarrow \Dd$ to its right adjoint $R\colon \Dd\rightarrow \Cc$ can be turned into an equivalence of $\infty$-categories $\Fun^\L(\Cc,\Dd)^\op\simeq \Fun^\R(\Dd,\Cc)$.\label{enum:FunLFunR}
		\item \!There exists an equivalence of $\infty$-categories $\cat{Cat}_\infty^\L\simeq (\cat{Cat}_\infty^\R)^\op$ which is the identity on objects and sends morphisms in $\cat{Cat}_\infty^\L$, that is, left adjoint functors $L\colon \Cc\rightarrow \Dd$, to their right adjoints $R\colon \Dd\rightarrow \Cc$.\label{enum:CatLCatR}
	\end{alphanumerate}
\end{cor}
\begin{proof}[Proof sketch]
	For the equivalence in \cref{enum:FunLFunR}, it suffices show that the essential images of $\Fun^\L(\Cc,\Dd)$ and $\Fun^\R(\Dd,\Cc)^\op$ under the fully faithful Yoneda embeddings
	\begin{gather*}
		\Fun^\R(\Dd,\Cc)\xrightarrow{(\Yo_\Cc)_*}\Fun\bigl(\Dd,\Fun(\Cc^\op,\cat{An})\bigr)\simeq \Fun\left(\Cc^\op\times\Dd,\cat{An}\right)\,,\\
		\Fun^\L(\Cc,\Dd)^\op\simeq \Fun^\R\left(\Cc^\op,\Dd^\op\right)\xrightarrow{(\Yo_{\Dd^\op})_*}\Fun\bigl(\Cc^\op,\Fun(\Dd,\cat{An})\bigr)\simeq \Fun(\Cc^\op\times\Dd,\cat{An})
	\end{gather*}
	coincide. Using \cref{lem:Adjunction} and the definition of the Yoneda embedding, it's straightforward to check that both essential images consist of those functors $H\colon \Cc^\op\times\Dd\rightarrow \cat{An}$ such that for every $x\in \Cc$ there exists a $y\in \Dd$ such that $H(x,-)\simeq \Hom_\Dd(y,-)$ and for every $y'\in\Dd$ there exists an $x'\in \Cc$ such that $H(-,y')\simeq \Hom_\Cc(-,x')$. This proves that there exists an equivalence $\Fun^\L(\Cc,\Dd)^\op\simeq \Fun^\R(\Dd,\Cc)$ as desired. Furthermore if $L\colon \Cc\shortdoublelrmorphism \Dd\noloc R$ is an adjunction, then the Yoneda embeddings above send both $R$ and $L$ to the functor $\Hom_\Cc(-,R(-))\simeq \Hom_\Dd(L(-),-)\colon \Cc^\op\times\Dd\rightarrow \cat{An}$. So the equivalence we've constructed is really given by extracting adjoints.
	
	To prove \cref{enum:CatLCatR}, we grossly neglect set theory and regard both $\cat{Cat}_\infty^\L$ and $\cat{Cat}_\infty^\R$ as objects in $\cat{Cat}_\infty$. This can be repaired by considering universes or, with some care, by imposing cardinality bounds (similar to the argument in \cref{lem:StraighteningFunctorial} below, where we do this in detail). We'll show that there exists a functorial bijection $\pi_0\Hom_{\cat{Cat}_\infty}(\Cc,\cat{Cat}_\infty^\L)\cong \pi_0\Hom_{\cat{Cat}_\infty}(\Cc,(\cat{Cat}_\infty^\R)^\op)$ for all $\infty$-categories $\Cc$; if we can do this, then the Yoneda lemma in the ordinary category $\operatorname{ho}(\cat{Cat}_\infty)$ will show that $\cat{Cat}_\infty^\L$ and $\cat{Cat}_\infty^\R$ are isomorphic in the homotopy category, hence equivalent as $\infty$-categories. We know $\Hom_{\cat{Cat}_\infty}(\Cc,\cat{Cat}_\infty^\L)\simeq \core\Fun(\Cc,\cat{Cat}_\infty^\L)$ by \cref{thm:CordierPorter} and $\Fun(\Cc,\cat{Cat}_\infty^\L)\simeq \cat{Cocart}(\Cc)$ by \cref{thm:Straightening}\cref{enum:CocartesianStraightening}. Let $F\colon \Cc\rightarrow \cat{Cat}_\infty$ be a functor and $p\colon \Uu\rightarrow \Cc$ be its cocartesian unstraightening. By \cref{lem:AdjunctionBicartesian}, $F$ factors through $\cat{Cat}_\infty^\L\rightarrow \cat{Cat}_\infty$ if and only if for all $\alpha\colon\Delta^1\rightarrow \Cc$, the pullback $p_{\alpha}\colon \Delta^1\times_{\alpha,\Cc}\Uu\rightarrow \Delta^1$ is not only a cocartesian, but also a cartesian fibration. In other words, $p$ is a \emph{locally cartesian fibration} in the sense of \cref{def:LocallyCocartesian}. Since right adjoints compose, it's clear that locally $p$-cartesian morphisms are closed under composition, and so $p$ is automatically a cartesian fibration by \cref{cor:LocallyCocartesianComposition}. In summary, we obtain a bijection 
	\begin{equation*}
		\pi_0\Hom_{\cat{Cat}_\infty}(\Cc,\cat{Cat}_\infty^\L)\cong \pi_0\core\cat{Bicart}(\Cc)\,,
	\end{equation*}
	where we define $\cat{Bicart}(\Cc)\subseteq \cat{Cat}_{\infty/\Cc}$ as the non-full sub-$\infty$-category spanned by the \emph{bicartesian fibrations}. That is, objects of $\cat{Bicart}(\Cc)$ are those $p\colon \Uu\rightarrow \Cc$ that are both cocartesian and cartesian fibrations, and morphisms are those functors in $\cat{Cat}_{\infty/\Cc}$ that preserve both $p$-cocartesian and $p$-cartesian morphisms. In the same way, we find bijections 
	\begin{equation*}
		\pi_0\Hom_{\Cat_\infty}\bigl(\Cc,(\cat{Cat}_\infty^\R)^\op\bigr)\cong \pi_0\Hom_{\Cat_\infty}\bigl(\Cc^\op,\cat{Cat}_\infty^\R\bigr)\cong \pi_0\core\cat{Bicart}(\Cc)\,.
	\end{equation*}
	Hence $\pi_0\Hom_{\cat{Cat}_\infty}(\Cc,\cat{Cat}_\infty^\L)\cong \pi_0\Hom_{\cat{Cat}_\infty}(\Cc,(\cat{Cat}_\infty^\R)^\op)$ and so $\cat{Cat}_\infty^\L\simeq (\cat{Cat}_\infty^\R)^\op$, as argued above. By unravelling the cases $\Cc\simeq *$ and $\Cc\simeq \Delta^1$ (the latter using \cref{lem:AdjunctionBicartesian}), we find that this adjunction is really the identity on objects and given by extracting adjoints on morphisms.
\end{proof}

\subsection{Limits and colimits}
\begin{defi}\label{def:Colimits}
	Let $\Ii$ and $\Cc$ be $\infty$-categories.
	\begin{alphanumerate}
		\item Let $F\colon \Ii\rightarrow \Cc$ be a functor of $\infty$-categories. A \emph{colimit of $F$}, denoted $\colimit F$ (or sometimes $\colimit_{i\in\Ii}F(i)$), is a left adjoint object of $F$ under  $\operatorname{const}\colon\Cc\rightarrow \Fun(\Ii,\Cc)$ that sends $x\in\Cc$ to the constant functor with value $x$. Dually, a \emph{limit of $F$}, denoted $\limit F$ (or sometimes $\limit_{i\in\Ii}F(i)$), is a right adjoint object of $F$ under $\operatorname{const}$.\label{enum:Colimit}
		\item We say that \emph{$\Cc$ has all $\Ii$-shaped colimits} or \emph{all $\Ii$-shaped limits} if all functors $\Ii\rightarrow \Cc$ admit colimits or limits, respectively.\label{enum:ColimitFunctor}
	\end{alphanumerate}
\end{defi}
\begin{rem}\label{rem:ColimitFunctor}
	 If $\Cc$ has all $\Ii$-shaped colimits, then \cref{lem:Adjunction} implies that forming colimits assembles into a functor $\colimit\colon \Fun(\Ii,\Cc)\rightarrow \Cc$. The same is true for limits.
\end{rem}
\begin{lem}\label{lem:AdjointsPreserveColimits}
	Left adjoint functors between $\infty$-categories preserve colimits and right adjoint functors preserve limits.
\end{lem}
\begin{proof}
	The proof of \cref{lem:1AdjointsPreserveColimits} can be copied verbatim.
\end{proof}
\begin{lem}[\enquote{Colimits in functor $\infty$-categories are computed pointwise.}]\label{lem:ColimitsInFunctorCategories}
	Let $\Cc$, $\Dd$, and $\Ii$ be $\infty$-categories such that $\Dd$ has all $\Ii$-shaped colimits. Then $\Fun(\Cc,\Dd)$ has again all $\Ii$-shaped colimits and the evaluation functor 
	\begin{equation*}
		\ev_x\colon \Fun(\Cc,\Dd)\longrightarrow \Fun\bigl(\{x\},\Dd\bigr)\simeq \Dd
	\end{equation*}
	preserves $\Ii$-shaped colimits for all $x\in \Cc$. A dual assertion holds for limits.
\end{lem}
\begin{proof}
	The proof of \cref{lem:1ColimitsInFunctorCategories} can be copied verbatim.
\end{proof}
Our next goal is to analyse limits and colimits in the $\infty$-categories $\cat{An}$ and $\cat{Cat}_\infty$. We start with a procedure for computing pullbacks and pushouts which is very useful in practice.
\begin{numpar}[Pushouts and pullbacks in $\cat{An}$ and $\cat{Cat}_\infty$.]\label{par:HomotopyPushout}
	Pushouts and pullbacks in $\cat{An}$ or $\cat{Cat}_\infty$ can be computed using the following recipe:
	\begin{alphanumerate}
		\item Write down the diagram on the level of Kan complexes or quasi-categories.\label{enum:PushoutStepA}
		\item For pushouts, use \cref{lem:SmallObjectArgument} to replace one leg by a cofibration. For pullbacks, use \cref{lem:SmallObjectArgument} to replace one leg by a Kan fibration/isofibration (depending on whether you take the pullback in $\cat{An}$ or $\cat{Cat}_\infty$, respectively).\label{enum:PushoutStepB}
		\item Take the pushout or pullback in $\cat{sSet}$.\label{enum:PushoutStepC}
		\item For pushouts, the result of \cref{enum:PushoutStepC} will usually not be a Kan complex/quasi-category, so we need to use \cref{lem:SmallObjectArgument} once again to replace it by a Kan complex/quasi-category. For pullbacks, this step is unnecessary.\label{enum:PushoutStepD}
	\end{alphanumerate}
	We've already seen the case of pullbacks in \enquote{Definition}~\cref{def:HomotopyPullback} and model category fact~\cref{par:HomotopyPullback}. The procedure above is a consequence of the general model category fact that a pushout of cofibrant objects in a model category is automatically a homotopy pushout too if at least one leg is a cofibration, and a pullback of fibrant objects is a homotopy pullback if at least one leg is a fibration. See \cite[Corollary~{\href{https://cisinski.app.uni-regensburg.de/CatLR.pdf\#thm.2.3.28}{2.3.28}}]{Cisinski} for a proof of the general fact and \cite[Theorem~\HTTthm{4.2.4.1}, Remark~\HTTthm{A.3.3.14}]{HTT} or \cite[Theorem~X.21]{HigherCatsII} for a proof that homotopy colimits/limits in a simplicial model category agree with colimits/limits in the underlying $\infty$-category.
	
	The procedure above implies that many pullback constructions we've seen so far with simplicial sets are also pullbacks in $\cat{An}$ or $\cat{Cat}_\infty$ and can thus be reinterpreted as model-independent constructions. For example, the diagram from \cref{par:HomInQuasiCategories} defining $\Cc_{x/}$ and $\Hom_\Cc(x,y)$ is also a pullback in $\cat{Cat}_\infty$, because $(s,t)\colon \Ar(\Cc)\rightarrow \Cc\times\Cc$ is an isofibration (see the proof of \cref{lem:HomRealityCheck}). As another example, if $p\colon\Uu\rightarrow \Cc$ is a cocartesian fibration, then the fibre $p^{-1}\{x\}$, which computes the value of the associated functor $\operatorname{St}^{(\mathrm{cocart})}\colon \Cc\rightarrow\cat{Cat}_\infty$ at $x$, can also be identified with the $\infty$-categorical pullback $\{x\}\times_\Cc\Uu$, because any cocartesian fibration $p$ is automatically an isofibration. We'll often use these facts without mention. Let us also mention, and later use without mention, that $\cat{An}\subseteq\cat{Cat}_\infty$ preserves both pushouts and pullbacks; in fact, it preserves all limits and colimits by \cref{exm:Adjunctions}\cref{enum:AnToCatInfty} and \cref{lem:AdjointsPreserveColimits}.\hfill$\blacksquare$
\end{numpar}
%\begin{numpar}[Pushouts in $\cat{An}$ and $\cat{Cat}_\infty$.]
%	It turns out that pushouts can be computed analogously: Let 
%	\begin{equation*}
%		\begin{tikzcd}
%			X\rar\dar\drar[pushout] & X'\dar\\
%			Y\rar & \ov{Y}'
%		\end{tikzcd}\quad\text{and}\quad
%		\begin{tikzcd}
%			\Cc\rar\dar\drar[pushout] & \Cc'\dar\\
%			\Dd\rar & \ov{\Dd}'
%		\end{tikzcd}
%	\end{equation*}
%	be pushouts in $\cat{sSet}$ such that $X$, $X'$, $Y$ are Kan complexes and $\Cc$, $\Cc'$, $\Dd$ are quasi-categories. Assume furthermore that at least one leg is a cofibration in either case. Choose an anodyne map $\ov{Y}'\rightarrow Y'$ and an inner anodyne map $\ov{\Dd}'\rightarrow \Dd'$. Then $Y'$ and $\Dd'$ are the pushouts in the $\infty$-categories $\cat{An}$ and $\cat{Cat}_\infty$, respectively. For a proof of the general model category fact behind this see \cite[Definition~{\href{https://cisinski.app.uni-regensburg.de/CatLR.pdf\#thm.2.3.27}{2.3.27}}]{Cisinski} for example.
%	
%	So to compute a pushout in $\cat{An}$ or $\cat{Cat}_\infty$, write it down on the level of simplicial sets, replace at least one leg by a cofibration (via \cref{lem:SmallObjectArgument}), take the pushout in $\cat{sSet}$, and finally replace the result by a Kan complex or quasi-category, respectively. Note that we can skip the final replacement step for pullbacks, since then the result is already a Kan complex or a quasi-category.\hfill$\blacksquare$
%\end{numpar}
%
But there's also a description of limits and colimits that works in full generality and doesn't rely on the simplicial model.\footnote{I'd like to see a proof of model category fact~\cref{par:HomotopyPushout} using only \cref{lem:ColimitsInAnima} below; I'm not sure if this works, so I'll leave it to you to figure out.} To formulate this, we need to introduce some notation. Let $\Ii$ be an $\infty$-category and let $p\colon \Uu\rightarrow \Ii$ a cocartesian fibration. Furthermore, let $\Fun_\Ii(\Ii,\Uu)\coloneqq \Fun(\Ii,\Uu)\times_{\Fun(\Ii,\Ii)}\{\id_\Ii\}$, the pullback being taken in $\cat{Cat}_\infty$ (but we could take it in $\cat{sSet}$ as well by model category fact~\cref{par:HomotopyPullback}) and let
\begin{equation*}
	\Fun_\Ii^{(\mathrm{cocart})}(\Ii,\Uu)\subseteq \Fun_\Ii(\Ii,\Uu)
\end{equation*}
be the full sub-$\infty$-category spanned by those $\Ii\rightarrow \Uu$ such that all morphisms in $\Ii$ are sent to $p$-cocartesian morphisms. Note that if $p$ is a left fibration, then
\begin{equation*}
	\Fun_\Ii^{(\mathrm{cocart})}(\Ii,\Uu)\simeq \Fun_\II(\Ii,\Uu)\simeq \Hom_{\Cat_{\infty/\Ii}}(\Ii,\Uu)\,.
\end{equation*}
Indeed, the first equivalence is clear since in this case all morphisms in $\Uu$ are $p$-cocartesian by \cref{lem:CocartesianLeft}. The second equivalence follows from \cref{cor:HomInSliceCategories} combined with the facts that $\core\colon \cat{An}\rightarrow \cat{Cat}_\infty$ preserves pullbacks (because it is a right adjoint by \cref{exm:Adjunctions}\cref{enum:AnToCatInfty}) and that $\Fun_\Ii(\Ii,\Uu)$ is already an anima (by \cref{cor:FKanFibration} and \cref{cor:LeftFibrationsOverAnima}).
\begin{lem}\label{lem:ColimitsInAnima}
	Let $F\colon \Ii\rightarrow \Cat_\infty$ be a functor and let $p\colon \Uu\rightarrow \Ii$ be its cocartesian unstraightening. Then the colimit and the limit of $F$ in $\cat{Cat}_\infty$ are given by
	\begin{equation*}
		\colimit_{i\in\Ii}F(i)\simeq \Uu\left[\{\text{cocartesian morphisms}\}^{-1}\right]\quad\text{and}\quad
		\limit_{i\in\Ii}F(i)\simeq \Fun_\Ii^{(\mathrm{cocart})}(\Ii,\Uu)\,.
	\end{equation*}
	In particular, if $F$ takes values in $\cat{An}$, then the colimit and the limit of $F$ in $\cat{An}$ are given by
	\begin{equation*}
		\colimit_{i\in\Ii}F(i)\simeq \abs*{\Uu}\quad\text{and}\quad \limit_{i\in\Ii}F(i)\simeq \Hom_{\Cat_{\infty/\Ii}}(\Ii,\Uu)\,.
	\end{equation*}
\end{lem}
For the proof, we need the following lemma. In \cref{cor:HomPreservesLimits}, a more general version of \cref{lem:HomPreservesPullbacks} is proved, but we need this special case as an input.
\begin{lem}\label{lem:HomPreservesPullbacks}
	For every $\infty$-category $\Cc$, the functor $\Hom_{\Cat_\infty}(\Cc,-)\colon \cat{Cat}_\infty\rightarrow \cat{An}$ preserves pullbacks.
\end{lem}
\begin{proof}[Proof sketch]
	As explained in footnote~\cref{footnote:Fun} in \cref{exm:Adjunctions}, Lurie constructs an equivalence of functors $\Hom_{\Cat_\infty}(-,-)\simeq \core\Fun(-,-)$ in \cite[Proposition~\HAthm{5.2.1.11}]{HA}, provided that you go with \cref{con:HomTwAr}. Thanks to \cref{lem:HomRealityCheck}, this proves $\Hom_{\Cat_\infty}(\Cc,-)\simeq \core\Fun(\Cc,-)$, no matter whether you use \cref{con:HomInTwoVariables} or~\labelcref{con:HomTwAr}. Now the claim is obvious, since both $\core\colon \cat{Cat}_\infty\rightarrow \cat{An}$ and $\Fun(\Cc,-)\colon \cat{Cat}_\infty\rightarrow\cat{Cat}_\infty$ are right adjoints by \cref{exm:Adjunctions}.
	
	In fact, to show preservation of pullbacks in this way, we can get away with a little less than Lurie's result: We only need that $\Hom_{\Cat_\infty}(\Cc,-)$ and  $\core\Fun(\Cc,-)$ agree on objects and morphisms. The former is clear by \cref{thm:CordierPorter}. Unfortunately, the latter still needs some care (and simplicial arguments): We know what $\core \Fun(-,-)$ does on morphisms, because it agrees with $\N^\Delta(\core\F(-,-))$ (the argument is in \cref{exm:Adjunctions}). For $\Hom_{\cat{Cat}_\infty}(\Cc,-)$, we need to unravel what straightening does on morphisms; this is quite nasty, but doable via \cref{par:StraighteningOnMorphisms}.
\end{proof}
\begin{proof}[Proof sketch of \cref{lem:ColimitsInAnima}]
	The idea in all of these statements is that the unstraightening of a constant functor $\const X$ is precisely the projection $\pr_2\colon X\times \Ii\rightarrow \Ii$. Let's first consider the case of colimits in $\cat{An}$ and see where this ideas takes us. To show that $\abs*{\Uu}$ is the desired colimit, we want an equivalence $\Hom_{\cat{An}}(\abs{\Uu},-)\simeq \Hom_{\Fun(\Ii,\cat{An})}(F,\const(-))$. Let's start manipulating the right-hand side. By \cref{thm:Straightening}\cref{enum:LeftStraightening}, $\Fun(\Ii,\cat{An})\simeq\cat{Left}(\Ii)$, hence
	\begin{equation*}
		\Hom_{\Fun(\Ii,\cat{An})}\bigl(F,\const(-)\bigr)\simeq \Hom_{\cat{Left}(\Ii)}\left(\Uu,\operatorname{Un}^{(\mathrm{left})}\bigl(\const(-)\bigr)\right)\,.
	\end{equation*}
	The unstraightening of $\const X\colon \Ii\rightarrow\cat{An}$ is the projection $X\times \Ii\rightarrow\Ii$, functorially in $X\in\cat{An}$ (this is a consequence of the fact that precomposition corresponds to pullback in \cref{thm:Straightening}\cref{enum:CocartesianStraightening}). So we can continue our manipulations as follows:
	\begin{align*}
		\Hom_{\cat{Left}(\Ii)}(\Uu,-\times\Ii)&\simeq \Hom_{\cat{Cat}_{\infty/\Ii}}(\Uu,-\times\Ii)\\
		&\simeq \Hom_{\Cat_\infty}(\Uu,-\times\Ii)\times_{\Hom_{\Cat_\infty}(\Uu,\Ii)}\{p\}\\
		&\simeq \bigl(\Hom_{\cat{Cat}_\infty}(\Uu,-)\times\Hom_{\cat{Cat}_\infty}(\Uu,\Ii)\bigr)\times_{\Hom_{\Cat_\infty}(\Uu,\Ii)}\{p\}\\
		&\simeq \Hom_{\cat{Cat}_\infty}(\Uu,-)\,.
	\end{align*}
	In the first step we use that $\cat{Left}(\Ii)\rightarrow \cat{Cat}_{\infty/\Ii}$ is fully faithful. In the second step we use \cref{cor:HomInSliceCategories}; by \cref{lem:ColimitsInFunctorCategories}, the pullback is automatically functorial  provided the square from \cref{cor:HomInSliceCategories} is functorial, which it clearly is by construction. In the third step, we use $\Hom_{\Cat_\infty}(\Uu,-\times\Ii)\simeq \Hom_{\cat{Cat}_\infty}(\Uu,-)\times\Hom_{\cat{Cat}_\infty}(\Uu,\Ii)$ by \cref{lem:HomPreservesPullbacks}. Finally, in the fourth step we use that $\Hom_{\cat{Cat}_\infty}(\Uu,\Ii)\times_{\Hom_{\Cat_\infty}(\Uu,\Ii)}\{p\}\simeq \{p\}$ is just a point.
	
	It remains to observe $\Hom_{\cat{An}}(\abs{\Uu},-)\simeq \Hom_{\cat{Cat}_\infty}(\Uu,-)$ because $\abs{\,\cdot\,}\colon \cat{Cat}_\infty\rightarrow\cat{An}$ is left adjoint to the inclusion $\cat{An}\subseteq\cat{Cat}_\infty$. Thus, we have proved $\colimit_{i\in\Ii}F(i)\simeq \abs{\Uu}$ by verifying that $\abs*{\Uu}$ satisfies the desired universal property.
	
	Let us now indicate the necessary changes to prove the other cases. For limits in $\cat{An}$, we can use a similar calculation; the crucial step is $\Hom_{\cat{Cat}_\infty}(-,\Fun_\Ii(\Ii,\Uu))\simeq \Hom_{\cat{Cat}_{\infty/\Ii}}((-)\times\Ii,\Uu)$, which uses \cref{lem:HomPreservesPullbacks}, the adjunction from \cref{exm:Adjunctions}\cref{enum:Currying}, and \cref{cor:HomInSliceCategories}. We leave the details to you. When taking colimits or limits in $\cat{Cat}_\infty$, we can no longer argue that $\cat{Cocart}(\Ii)\rightarrow \cat{Cat}_{\infty/\Ii}$ is fully faithful. Instead, in the colimit case, $\Hom_{\cat{Cocart}(\Ii)}(\Uu,-\times\Ii)\subseteq \Hom_{\cat{Cat}_{\infty/\Ii}}(\Uu,\Cc\times\Ii)$ is a collection of path components by \cref{lem:NonFullSubcategory} and we have to check that it agrees with $\Hom_{\cat{Cat}_\infty}(\Uu[\{\text{cocartesian morphisms}\}^{-1}],-)\subseteq \Hom_{\Cat_\infty}(\Uu,-)$, which is also a collection of path components by \cref{lem:Localisation}. This is straightforward. A similar argument applies in the limit case.
\end{proof}
\begin{cor}\label{cor:HomPreservesColimits}
	Let $F\colon \Ii\rightarrow\Cc$ be a functor of $\infty$-categories. A natural transformation $c_F\colon \const y\Rightarrow F$ exhibits $y\in \Cc$ as a limit of $F$ if and only if the natural map
	\begin{equation*}
		c_F^*\colon \Hom_\Cc(x,y)\overset{\simeq}{\longrightarrow}\limit_{i\in\Ii}\Hom_\Cc\bigl(x,F(i)\bigr)
	\end{equation*}
	is an equivalence for all $x\in \Cc$. A dual assertion holds for colimits.
\end{cor}
\begin{proof}
	The unstraightening of $\Hom_\Cc(x,F(-))\colon \Ii\rightarrow\cat{An}$ is the left fibration $F^*(\Cc_{x/})\rightarrow \Ii$, the pullback of the slice-$\infty$-category projection $t\colon \Cc_{x/}\rightarrow \Cc$ along $F\colon \Ii\rightarrow\Cc$. Hence, according to \cref{lem:ColimitsInAnima}, $\limit_{i\in\Ii}\Hom_\Cc(x,F(i))\simeq\Hom_{\Cat_{\infty/\Ii}}(\Ii,F^*(\Cc_{x/}))$. Let us now manipulate the right-hand side as follows:
	\begin{align*}
		\Hom_{\Cat_{\infty/\Ii}}\bigl(\Ii,F^*(\Cc_{x/})\bigr)&\simeq \Hom_{\cat{Cat}_\infty}\bigl(\Ii,F^*(\Cc_{x/})\bigr)\times_{t,\Hom_{\cat{Cat}_\infty}(\Ii,\Ii)}\{\id_\Ii\}\\%\Hom_{\Cat_\infty{}_{/\Cc}}\bigl(\Ii,\Cc_{x/}\bigr)\\
		%&\simeq \core\F\bigl(\Ii,\Cc_{x/}\bigr)\times_{t,\core \F(\Ii,\Cc)}\{F\}\\
		&\simeq \Hom_{\cat{Cat}_\infty}\bigl(\Ii,\{x\}\times_{\Cc,s}\Ar(\Cc)\times_{t,\Cc,F}\Ii\bigr)\times_{t,\Hom_{\cat{Cat}_\infty}(\Ii,\Ii)}\{\id_\Ii\}\\
		&\simeq \{\const x\}\times_{\Hom_{\cat{Cat}_\infty}(\Ii,\Cc),s}\Hom_{\cat{Cat}_\infty}\bigl(\Ii,\Ar(\Cc)\bigr)\times_{t,\Hom_{\cat{Cat}_\infty}(\Ii,\Cc)}\{F\}\\
		&\simeq \Hom_{\Fun(\Ii,\Cc)}(\const x,F)\,.
	\end{align*}
	In the first step we plug in \cref{cor:HomInSliceCategories} to write $\Hom_{\Cat_{\infty/\Ii}}$ as a pullback. In the second step, we plug in $F^*(\Cc_{x/})\cong\{x\}\times_{\Cc,s}\Ar(\Cc)\times_{t,\Cc,F}\Ii$. In the third step we use \cref{lem:HomPreservesPullbacks} and simplify the pullback. Finally, in the fourth step we write $\Hom_{\cat{Cat}_\infty}(\Ii,\Cc)\simeq \core \Fun(\Ii,\Cc)$ and $\Hom_{\Cat_\infty}(\Ii,\Ar(\Cc))\simeq \Hom(\Delta^1,\Fun(\Ii,\Cc))\simeq \core\Ar(\Fun(\Ii,\Cc))$ and use the definition of $\Hom_{\Fun(\Ii,\Cc)}(\const x,F)$ from \cref{par:HomInQuasiCategories}; as we've seen in model category fact~\cref{par:HomotopyPushout}, the pullbacks in $\cat{sSet}$ from \cref{par:HomInQuasiCategories} can be taken in $\cat{Cat}_\infty$ as well, and since $\Hom_{\Fun(\Ii,\Cc)}(\const x,F)$ is an anima anyway, it doesn't matter that we apply $\core$ everywhere.
	
	Therefore, at least pointwise, $c_F^*$ takes the form $c_F^*\colon \Hom_\Cc(x,y)\rightarrow \Hom_{\Fun(\Ii,\Cc)}(\const x,F)$ for all $x\in\Cc$. Since a natural transformation is an equivalence if and only if it is a pointwise equivalence (\cref{thm:EquivalencePointwise}), we are done.
\end{proof}
\begin{cor}\label{cor:HomPreservesLimits}
	For every $\infty$-category $\Cc$, the functors $\Hom_\Cc(x,-)\colon \Cc\rightarrow\cat{An}$ and $\Hom_\Cc(-,y)\colon \Cc^\op\rightarrow\cat{An}$ preserve limits for all $x,y\in\Cc$ \embrace{note that limits in $\Cc^\op$ correspond to colimits in $\Cc$}. Likewise, the Yoneda embedding $\Yo_\Cc\colon \Cc\rightarrow\Fun(\Cc^\op,\cat{An})$ preserves limits.
\end{cor}
\begin{proof}
	The first two assertions follow immediately from \cref{cor:HomPreservesColimits}. The last one follows from the first plus the fact that limits and equivalences in functor categories are pointwise by \cref{lem:ColimitsInFunctorCategories} and \cref{thm:EquivalencePointwise}.
\end{proof}

\subsection{Coinitial and initial functors}
Our next goal is to develop a general theory of diagrams that have the same limit or colimit. This is summarised by the following theorem due to Joyal, with a first written proof appearing in \cite[Theorem~\HTTthm{4.1.3.1}]{HTT}.
\begin{thm}[Joyal's version of Quillen's theorem A]\label{thm:JoyalsQuillenA}
	For a functor $\alpha\colon\Ii\rightarrow \Jj$ of $\infty$-categories, the following are equivalent:
	\begin{alphanumerate}
		\item For every $\infty$-category $\Cc$ and every $F\colon \Jj\rightarrow \Cc$, the functor $F$ has a colimit if and only if $F\circ \alpha$ has a colimit. Furthermore, in this case the following natural map is an equivalence:\label{enum:Cofinal}
		\begin{equation*}
			\colimit_{i\in\Ii}F\bigl(\alpha(i)\bigr)\overset{\simeq}{\longrightarrow}\colimit_{j\in\Jj}F(j)\,.
		\end{equation*}
		\item For every right fibration $f\colon X\rightarrow\Jj$, the following natural map 
		is an equivalence:\label{enum:RightAnodyne}
		\begin{equation*}
			\Hom_{\Cat_{\infty/\Jj}}(\Ii,X)\overset{\simeq}{\longrightarrow} \Hom_{\Cat_{\infty/\Jj}}(\Jj,X)\,.
		\end{equation*}
		\item For every $j\in\Jj$, the slice-$\infty$-category $\Ii_{j/}\coloneqq \Ii\times_{\Jj}\Jj_{j/}$ is weakly contractible. That is, we have $\abs{\Ii_{j/}}\simeq *$.\label{enum:WeaklyContractible}
	\end{alphanumerate}
	A dual assertion holds for limits, left fibrations, and the slice-$\infty$-categories $\Ii_{/j}$.
\end{thm}
\begin{defi}
	If $\alpha\colon \Ii\rightarrow\Jj$ satisfies the equivalent conditions from \cref{thm:JoyalsQuillenA}, then $\alpha$ is called \emph{coinitial}. Dually, $\alpha$ is called \emph{initial} if it satisfies the dual equivalent conditions for limits.%
	\footnote{Calling these functors \emph{\embrace{co-}initial} is slightly nonstandard: People usually use the terms \emph{\embrace{co-}final}, but there seems to be no universally agreed convention on which of these terms refers to limits and which refers to colimits. In any case, our naming convention is objectively the correct one. It is uniquely determined by the following two desiderata:
	\begin{alphanumerate}
		\item The concept for colimits should have the presyllable \emph{co-}, whereas the concept for limits should have no presyllable at all.
		\item The inclusion of an initial object should be \emph{initial}, or maybe \emph{cofinal}, but definitely not \emph{final}. Likewise for terminal objects.
		\end{alphanumerate}}
\end{defi}
\begin{exm}\label{exm:Cofinal}
	The following are examples of coinitial functors:
	\begin{alphanumerate}
		\item Right anodyne maps are coinitial. It's clear from \cref{cor:FKanFibration} and \cref{cor:HomInSliceCategories} that the condition from \cref{thm:JoyalsQuillenA}\cref{enum:RightAnodyne} is satisfied.\label{enum:RightAnodyneCofinal}
		\item Right adjoint functors $\alpha\colon \Ii\rightarrow\Jj$ are coinitial. Indeed, if $\beta$ is a left adjoint, then $\alpha^*\colon \Fun(\Jj,\Cc)\shortdoublelrmorphism \Fun(\Ii,\Cc)\noloc \beta^*$ is an adjunction by \cref{cor:FunctorCategoryAdjunctions} and so to verify the condition ${\colimit_\Ii}\circ{\alpha^*}\simeq \colimit_\Jj$ from \cref{thm:JoyalsQuillenA}\cref{enum:Cofinal}, it's enough to check ${\beta^*}\circ{\const}\simeq\const$, which is clear.\label{enum:RightAdjointCofinal}
		\item Localisations $p\colon \Ii\rightarrow \Ii[W^{-1}]$ are coinitial. One way to see this is that localisations are right anodyne, since by construction, $p$ factors into $\Ii\rightarrow\ov{\Ii}\rightarrow \Ii[W^{-1}]$, where the second arrow is inner anodyne and the first arrow is right anodyne, because $\Delta^1\rightarrow J$ is right anodyne. Then \cref{enum:RightAnodyneCofinal} does it.\label{enum:LocalisationsCofinal}
		
		But of course there's also a synthetic way to see this. Since the precomposition functor $p^*\colon\Fun(\Ii[W^{-1}],\Cc)\rightarrow \Fun(\Ii,\Cc)$ is fully faithful by \cref{lem:Localisation}, we have
		\begin{equation*}
			\Hom_{\Fun(\Ii,\Cc)}(F\circ p,\const y)\simeq \Hom_{\Fun(\Ii[W^{-1}],\Cc)}(F,\const y)\,,
		\end{equation*}
		functorially in $F\colon \Ii[W^{-1}]\rightarrow \Cc$ and all $y\in \Cc$, which proves that the condition from \cref{thm:JoyalsQuillenA}\cref{enum:Cofinal} is satified.
	\end{alphanumerate}
\end{exm}
\begin{proof}[Proof of \cref{thm:JoyalsQuillenA}, \cref{enum:Cofinal} $\Leftrightarrow$ \cref{enum:RightAnodyne}]
	Let $F\colon \Jj\rightarrow \cat{An}$ be a functor with unstraightening $\Uu\rightarrow \Jj$. Then the pullback $\alpha^*(\Uu)\rightarrow \Ii$ is the unstraightening of $F\circ\alpha\colon \Ii\rightarrow\cat{An}$. \cref{lem:ColimitsInAnima} shows $\lim_{j\in\Jj}F(j)\simeq\Hom_{\Cat_{\infty/\Jj}}(\Jj,\Uu)$. Similarly,
	\begin{equation*}
		\lim_{i\in\Ii}F\bigl(\alpha(i)\bigr)\simeq\Hom_{\Cat_{\infty/\Ii}}\bigl(\Ii,\alpha^*(\Uu)\bigr)\simeq \Hom_{\Cat_{\infty/\Ii}}(\Ii,\Uu)\,;
	\end{equation*}
	here the second equivalence is a quick calculation using \cref{cor:HomInSliceCategories} and \cref{cor:HomPreservesColimits}. This shows that \cref{enum:RightAnodyne} holds if and only if \cref{enum:Cofinal} holds for functors $F\colon \Jj\rightarrow \cat{An}$. Now let $F\colon \Jj\rightarrow \Cc$ be an arbitrary functor. By \cref{cor:HomPreservesColimits}, $\Yo_\Cc\colon \Cc\rightarrow \Fun(\Cc^\op,\cat{An})$ preserves limits and it is fully faithful, so \cref{enum:Cofinal} holds for $F\colon \Jj\rightarrow \Cc$ if and only if it holds for $\Yo_\Cc\circ F\colon \Jj\rightarrow \Fun(\Cc^\op,\cat{An})$. Finally, limits in $\Fun(\Cc^\op,\cat{An})$ are computed pointwise by \cref{lem:ColimitsInFunctorCategories} and equivalences can be checked pointwise by \cref{thm:EquivalencePointwise}, so \cref{enum:Cofinal} holds for functors into $\Fun(\Cc^\op,\cat{An})$ if and only if it holds for functors into $\cat{An}$. This finishes the proof of \cref{enum:Cofinal} $\Leftrightarrow$ \cref{enum:RightAnodyne}.
\end{proof}
Before we can prove \cref{enum:Cofinal}  $\Rightarrow$ \cref{enum:WeaklyContractible} $\Rightarrow$ \cref{enum:RightAnodyne}, we need another lemma.
\begin{lem}\label{lem:CartesianCofinal}
	A cartesian fibration $p\colon \Uu\rightarrow \Jj$ satisfies the conclusion of \cref{thm:JoyalsQuillenA}\cref{enum:RightAnodyne} if and only if the fibres $p^{-1}\{j\}$ of $p$ are weakly contractible, that is, $\abs*{p^{-1}\{j\}}\simeq *$ for all $j\in \Jj$.
\end{lem}
\begin{proof}
	Let $E\colon \Jj^\op\rightarrow\cat{Cat}_\infty$ be the straightening of $p\colon \Uu\rightarrow \Jj$ and let $f\colon X\rightarrow \Jj$ be a right fibration with straightening $F\colon \Jj^\op\rightarrow \cat{An}$. Then the cartesian straightening equivalence (the dual of \cref{thm:Straightening}\cref{enum:CocartesianStraightening}) shows
	\begin{equation*}
		\Hom_{\Cat_{\infty/\Jj}}(\Uu,X)\simeq \Hom_{\cat{Cart}(\Jj)}(\Uu,X)\simeq \Hom_{\Fun(\Jj^\op,\cat{Cat}_\infty)}(E,F)\,.
	\end{equation*}
	Note that the first equivalence holds even though $\cat{Cart}(\Jj)\rightarrow\cat{Cat}_{\infty/\Jj}$ is not fully faithful, since we're mapping into a right fibration where every morphism is cartesian (by the dual of \cref{lem:CocartesianLeft}). Now $\abs*{\,\cdot\,}\colon \cat{Cat}_\infty\rightarrow \cat{An}$ is left adjoint to the inclusion $\cat{An}\subseteq \cat{Cat}_\infty$ by \cref{exm:Adjunctions}\cref{enum:AnToCatInfty} and that adjunction persists to functor-$\infty$-categories by \cref{cor:FunctorCategoryAdjunctions}. Thus
	\begin{equation*}
		\Hom_{\Fun(\Jj^\op,\cat{Cat}_\infty)}(E,F)\simeq \Hom_{\Fun(\Jj^\op,\cat{Cat}_\infty)}\bigl(\abs{E},F\bigr)\,.
	\end{equation*}
	The cartesian straightening of $\id_\Jj\colon \Jj\rightarrow \Jj$ is $\const *\colon \Jj^\op\rightarrow\cat{An}$. By the same arguments as above we then obtain
	\begin{equation*}
		\Hom_{\Cat_{\infty/\Jj}}(\Jj,X)\simeq \Hom_{\Fun(\Jj^\op,\cat{Cat}_\infty)}(\const *,F)\,.
	\end{equation*}
	Putting everything together, we see that the condition from \cref{thm:JoyalsQuillenA}\cref{enum:RightAnodyne} is satisfied if and only if $p\colon\Uu\rightarrow \Jj$ induces an equivalence $\abs*{E}\Rightarrow \const *$ in $\Fun(\Jj^\op,\cat{An})$. Since equivalences can be checked pointwise (\cref{thm:EquivalencePointwise}), this becomes precisely the condition that all fibres of $p$ are weakly contractible.
\end{proof}
Note that \cref{lem:CartesianCofinal} and \cref{thm:JoyalsQuillenA}\cref{enum:WeaklyContractible} look very much alike, but are a priori two different criteria for a cartesian fibration $p\colon \Uu\rightarrow\Jj$ to be coinitial. As a reality check, let's see that they are indeed equivalent. This isn't necessary to complete our proof of \cref{thm:JoyalsQuillenA}, but we'll need it later.
\begin{lem}\label{lem:CartesianFibres}
	Let $p\colon \Uu\rightarrow \Jj$ be a cartesian fibration. Then for every $j\in\Jj$, the natural functor $p^{-1}\{j\}\rightarrow \Uu\times_\Jj\Jj_{j/}$ admits a right adjoint. In particular, we obtain a homotopy equivalence of animae $\abs{p^{-1}\{j\}}\simeq \abs{\Uu\times_\Jj\Jj_{j/}}$.
\end{lem}
\begin{proof}[Proof sketch]
	By \cref{lem:Adjunction}, right adjoints can be constructed pointwise. This can be done as follows: Fix an object $(u,\ov\varphi)\in \Uu\times_\Jj\Jj_{j/}$, given by an element $u\in \Uu$ and a morphism $\ov\varphi\colon j\rightarrow p(u)$ in $\Jj$. Let $\varphi\colon u'\rightarrow u$ be a $p$-cartesian lift of $\ov\varphi$. Then $u'\in p^{-1}\{j\}$ is a right adjoint object to $(u,\ov\varphi)$ under $p^{-1}\{j\}\rightarrow \Uu\times_\Jj\Jj_{j/}$. To see this, note that, by construction, we have a morphism $c\colon (u',\id_j)\rightarrow (u,\ov\varphi)$ in $\Uu\times_\Jj\Jj_{j/}$ (which will play the role of the counit); using \cref{thm:EquivalencePointwise}, we have to show that the composition
	\begin{equation*}
		\Hom_{p^{-1}\{j\}}(u'',u')\longrightarrow \Hom_{\Uu\times_\Jj\Jj_{j/}}\bigl((u'',\id_j),(u',\id_j)\bigr)\overset{c_*}{\longrightarrow}\Hom_{\Uu\times_\Jj\Jj_{j/}}\bigl((u'',\id_j),(u,\ov\varphi)\bigr)
	\end{equation*}
	is an equivalence for all $u''\in p^{-1}\{j\}$. Now use the characterisation of cartesian morphisms from the dual of \cref{lem:CocartesianMorphisms} together with \cref{cor:HomInSliceCategories} and the fact that Hom animae in pullbacks of $\infty$-categories are pullbacks of the respective Hom animae (which is straightforward to see; we'll prove a more general statement in \cref{lem:HomInLimits}\cref{enum:HomInLimits}) to show that both sides are equivalent to $\Hom_\Uu(u'',u)\times_{\Hom_\Jj(j,p(u))}\{\ov\varphi\}$ and that the morphism between them is equivalent to the identity. We'll leave the details to you.
\end{proof}

\begin{proof}[Proof of \cref{thm:JoyalsQuillenA}, \cref{enum:Cofinal}  $\Rightarrow$ \cref{enum:WeaklyContractible} $\Rightarrow$ \cref{enum:RightAnodyne}]
	Assume \cref{enum:Cofinal} holds true and consider the functor $\Hom_\Jj(j_0,-)\colon \Jj\rightarrow \cat{An}$ for some $j_0\in \Jj$. Its unstraightening is the slice-$\infty$-category projection $t\colon \Jj_{j_0/}\rightarrow\Jj$, hence $\colimit_{j\in\Jj} \Hom_\Jj(j_0,j)\simeq \abs{\Jj_{j_0/}}$ by \cref{lem:ColimitsInAnima}. But $\Jj_{j_0/}$ has an initial element given by $\id_{j_0}$, and so $\{\id_{j_0}\}\shortdoublelrmorphism \Jj_{j_0/}$ is an adjunction. Since adjunctions induce homotopy equivalences after $\abs{\,\cdot\,}$, we conclude $\abs{\Jj_{j_0/}}\simeq *$.
	
	Now consider $\Hom_\Jj(j_0,\alpha(-))\colon \Ii\rightarrow \cat{An}$. Its unstraightening is $\Ii_{j_0/}\simeq \Ii\times_\Jj \Jj_{j_0/}$ (here we use that precomposition with $\alpha$ corresponds to pullback along $\alpha$ under the unstraightening equivalence, see \cref{thm:Straightening}). Combining \cref{lem:ColimitsInAnima} with condition \cref{enum:Cofinal}, we obtain
	\begin{equation*}
		\bigl\lvert\Ii_{j_0/}\bigr\rvert\simeq\colimit_{i\in\Ii}\Hom_\Jj\bigl(j_0,\alpha(i)\bigr)\simeq \colimit_{j\in\Jj}\Hom_\Jj(j_0,j)\simeq *\,,
	\end{equation*}
	as claimed. This finishes the proof of the implication \cref{enum:Cofinal}  $\Rightarrow$ \cref{enum:WeaklyContractible}.
	
	Now assume \cref{enum:WeaklyContractible}. We can factor $\alpha\colon \Ii\rightarrow \Jj$ into $\Ii\rightarrow \Ii\times_{\Jj,s}\Ar(\Jj)\rightarrow \Jj$, where the first functor sends $i\in \Ii$ to the pair $(i,\id_{\alpha(i)}\colon \alpha(i)\rightarrow \alpha(i))$ and the second functor is induced by the target projection $t\colon \Ar(\Jj)\rightarrow \Jj$. It's straightforward to verify that $\Ii\rightarrow \Ii\times_{\Jj,s}\Ar(\Jj)$ is right adjoint to the projection $s\colon \Ii\times_{\Jj,s}\Ar(\Jj)\rightarrow \Ii$.\footnote{For example, one could use \cref{lem:HomInArrowCategories}; alternatively, unit and counit as well as the triangle identities are easily constructed by hand.} By \cref{exm:Cofinal}\cref{enum:RightAdjointCofinal}, we see that the functor $\Ii\rightarrow \Ii\times_{\Jj,s}\Ar(\Jj)$ satisfies \cref{enum:Cofinal}, hence also \cref{enum:RightAnodyne}. Furthermore, a slight generalisation of \cref{exm:Straightening}\cref{enum:ArCocartesianFibration} (which can be proved by the same argument) shows that $t\colon\Ii\times_{\Jj,s}\Ar(\Jj)\rightarrow \Jj$ is a cocartesian fibration. Its fibres $t^{-1}\{j\}\simeq \Ii\times_\Jj \Jj_{j/}$ are weakly contractible. Hence $t\colon\Ii\times_{\Jj,s}\Ar(\Jj)\rightarrow \Jj$ satisfies \cref{enum:RightAnodyne} by \cref{lem:CartesianCofinal}. We conclude that $\alpha\colon \Ii\rightarrow \Jj$ must satisfy \cref{enum:RightAnodyne} as well. Indeed, if $f\colon X\rightarrow \Jj$ is a right fibration, then
	\begin{align*}
		\Hom_{\Cat_{\infty/\Jj}}(\Jj,X)&\simeq \Hom_{\Cat_{\infty/\Jj}}\bigl(\Ii\times_{\Jj,s}\Ar(\Jj),X\bigr)\\
		&\simeq  \Hom_{\Cat_{\infty/\Ii\times_{\Jj,s}\Ar(\Jj)}}\bigl(\Ii\times_{\Jj,s}\Ar(\Jj),t^*(X)\bigr)\,.
	\end{align*}
	In the first equivalence we use \cref{enum:RightAnodyne} for $t\colon\Ii\times_{\Jj,s}\Ar(\Jj)\rightarrow \Jj$. In the second equivalence we let $t^*(X)\rightarrow \Ii\times_{\Jj,s}\Ar(\Jj)$ be the pullback of $f$ along $t$ and use \cref{lem:KanExtensionForRight}\cref{enum:ForgetfulFunctor} below. Now a pullback of a right fibration is again a right fibration, whence
	\begin{align*}
		\Hom_{\Cat_{\infty/\Ii\times_{\Jj,s}\Ar(\Jj)}}\bigl(\Ii\times_{\Jj,s}\Ar(\Jj),t^*(X)\bigr)&\simeq \Hom_{\Cat_{\infty/\Ii\times_{\Jj,s}\Ar(\Jj)}}\bigl(\Ii,t^*(X)\bigr)\\
		&\simeq \Hom_{\Cat_{\infty/\Jj}}(\Ii,X)\,.
	\end{align*}
	In the first equivalence we use \cref{enum:RightAnodyne} for $\Ii\rightarrow \Ii\times_{\Jj,s}\Ar(\Jj)$ and in the second we use \cref{lem:KanExtensionForRight}\cref{enum:ForgetfulFunctor} below again. This finishes the proof of the implication \cref{enum:WeaklyContractible} $\Rightarrow$ \cref{enum:RightAnodyne}.
\end{proof}
\subsection{Kan extensions}\label{subsec:KanExtensions}
We're now working towards an $\infty$-categorical analogue of \cref{thm:1PShFreeCocompletion}. %For an $\infty$-category $\Cc$, we let $\PSh(\Cc)\coloneqq\Fun(\Cc^\op,\cat{An})$ denote the \emph{$\infty$-category of presheaves on $\Cc$} (note that this is \emph{not} compatible with the previous definition if $\Cc$ is an ordinary category).
Our first goal is to construct left Kan extensions for presheaf categories. As it turns out, this is most easily done in the fibration picture.
\begin{lem}\label{lem:KanExtensionForRight}
	Let $F\colon \Cc\rightarrow \Dd$ be a functor of $\infty$-categories.
	\begin{alphanumerate}
		\item \!The pullback functor $F^*\colon \cat{Cat}_{\infty/\Dd}\rightarrow\cat{Cat}_{\infty/\Cc}$ has a left adjoint, namely the forgetful functor $\cat{Cat}_{\infty/\Cc}\rightarrow\cat{Cat}_{\infty/\Dd}$ that sends $f\colon \Cc'\rightarrow \Cc$ to $F\circ f\colon \Cc'\rightarrow \Dd$.\label{enum:ForgetfulFunctor}
		\item \!The inclusion $\cat{Right}(\Dd)\subseteq\cat{Cat}_{\infty/\Dd}$ has a left adjoint that sends $g\colon \Dd'\rightarrow \Dd$ to $q\colon Y\rightarrow \Dd$, where
		\begin{equation*}
			\Dd'\longrightarrow Y\overset{q}{\longrightarrow}\Dd
		\end{equation*}
		is any factorisation of $g$ into a coinitial functor followed by a right fibration.\label{enum:RightCofinalLeftAdjoint}
		\item \!The functor $F^*\colon \cat{Right}(\Dd)\rightarrow \cat{Right}(\Cc)$ has a left adjoint $F_!\colon \cat{Right}(\Cc)\rightarrow \cat{Right}(\Dd)$. On objects, $F_!$ is given as follows: Let $p\colon X\rightarrow \Cc$ be a right fibration and let\label{enum:RightPullbackLeftAdjoint}
		\begin{equation*}
			X\longrightarrow Y\overset{q}{\longrightarrow} \Dd
		\end{equation*}
		be any factorisation of $F\circ p$ into a coinitial functor followed by a right fibration. Then we have $F_!(p\colon X\rightarrow \Cc)\simeq (q\colon Y\rightarrow \Dd)$. In particular, all such factorisations are equivalent.
	\end{alphanumerate}
\end{lem}
\begin{proof}
	For \cref{enum:ForgetfulFunctor}, note that left adjoints can be constructed pointwise by \cref{lem:Adjunction}, so its enough to show that $F\circ f\colon \Cc'\rightarrow \Cc$ is a left adjoint object to $f\colon \Cc'\rightarrow\Cc$ under $F^*$. To this end, let $g\colon \Dd'\rightarrow \Dd$ be an element in $\cat{Cat}_{\infty/\Dd}$. We have a diagram
	\begin{equation*}
		\begin{tikzcd}
			\Hom_{\Cat_{\infty/\Cc}}\bigl(\Cc',F^*(\Dd')\bigr)\dar\rar\drar[pullback]&  \Hom_{\Cat_\infty}\bigl(\Cc',F^*(\Dd')\bigr)\rar\dar\drar[pullback] & \Hom_{\Cat_\infty}(\Cc',\Dd')\dar\\
			\{f\}\rar & \Hom_{\Cat_\infty}(\Cc',\Cc)\rar["F_*"] & \Hom_{\Cat_\infty}(\Cc',\Dd)
		\end{tikzcd}
	\end{equation*}
	in which the left square is a pullback by \cref{cor:HomInSliceCategories} and the right square is a pullback by \cref{cor:HomPreservesColimits}. Hence the outer rectangle is a pullback too. Combining this with \cref{cor:HomInSliceCategories}, we obtain
	\begin{equation*}
		\Hom_{\Cat_{\infty/\Cc}}\bigl(\Cc',F^*(\Dd')\bigr)\simeq\Hom_{\Cat_{\infty/\Dd}}(\Cc',\Dd')\,.
	\end{equation*}
	Since every step in the argument can be made functorial in $g\colon \Dd'\rightarrow \Dd$, we have proved \cref{enum:ForgetfulFunctor}.
	
	For \cref{enum:RightCofinalLeftAdjoint}, note that \cref{enum:ForgetfulFunctor} combined with \cref{thm:JoyalsQuillenA}\cref{enum:RightAnodyne} immediately implies that $q\colon Y\rightarrow \Dd$ is a left adjoint object to $g\colon \Dd'\rightarrow \Dd$ under the inclusion $\cat{Right}(\Dd)\subseteq \cat{Cat}_{\infty/\Dd}$. Since left adjoints can be constructed pointwise by \cref{lem:Adjunction}, we only need to check that such a factorisation always exists. But that's easy! For example, we could choose $\Dd'\rightarrow Y$ to be right anodyne by \cref{lem:SmallObjectArgument} and \cref{exm:Cofinal}\cref{enum:RightAnodyneCofinal}. If you'd like to avoid simplicial sets, we could also argue as follows: Choose a factorisation $\Dd'\rightarrow Y'\rightarrow \Dd$ into a right adjoint functor followed by a cartesian fibration $g'\colon Y'\rightarrow \Dd$ as in the proof of \cref{thm:JoyalsQuillenA}. Then put
	\begin{equation*}
		(g\colon Y\rightarrow \Dd)\coloneqq \operatorname{Un}^{(\mathrm{right})}\left(\Dd\xrightarrow{\operatorname{St}^{(\mathrm{cart})}(g)}\cat{Cat}_\infty\overset{\abs{\,\cdot\,}}{\longrightarrow}\cat{An}\right)\,.
	\end{equation*}
	Finally, \cref{enum:RightPullbackLeftAdjoint} follows from the combined powers of \cref{enum:ForgetfulFunctor} and \cref{enum:RightCofinalLeftAdjoint}.
\end{proof}
%Then \cref{lem:KanExtensionForRight} implies the following:
In the following, we let $\PSh(\Cc)\coloneqq\Fun(\Cc^\op,\cat{An})$ denote the \emph{$\infty$-category of presheaves on $\Cc$}.%Note that this is \emph{not} compatible with the previous definition for ordinary categories.
\begin{cor}\label{cor:f_!:PC->PD}
	Let $F\colon \Cc\rightarrow \Dd$ be a functor of $\infty$-categories. Then the precomposition functor $F^*\colon \PSh(\Dd)\rightarrow\PSh(\Cc)$ has a left adjoint $F_!$ such that the diagram
	\begin{equation*}
		\begin{tikzcd}
			\Cc\rar["F"]\dar["\Yo_\Cc"']\drar[commutes]& \Dd\dar["\Yo_\Dd"]\\			\PSh(\Cc)\rar["F_!"]&\PSh(\Dd)
		\end{tikzcd}
	\end{equation*}
	commutes in the $\infty$-category $\Cat_\infty$.
\end{cor}
\begin{proof}
	It's clear from \cref{lem:KanExtensionForRight} and the right straightening equivalence (the dual of \cref{thm:Straightening}\cref{enum:LeftStraightening}) that $F_!$ exists, so we only have to show that the diagram commutes. To this end, first note that the natural transformation $\Hom_\Cc(-,-)\Rightarrow\Hom_\Dd(F(-),F(-))$ gets transformed into $\Yo_\Cc\Rightarrow F^*\circ \Yo_\Dd\circ F$ under the equivalence in $\Fun(\Cc^\op\times\Cc,\cat{An})\simeq \Fun(\Cc,\PSh(\Cc))$. Using the adjunction $F_!\dashv F^*$ as well as \cref{cor:FunctorCategoryAdjunctions}, this transformation is adjoint to a natural transformation $F_!\circ \Yo_\Cc\Rightarrow \Yo_\Dd\circ F$.
	
	So our diagram commutes up to natural transformation, and we have to show that said natural transformation is an equivalence. By \cref{thm:EquivalencePointwise}, this can be done pointwise. So choose $x\in\Cc$. Under the straightening equivalence, the functor $\Yo_\Cc(x)\simeq \Hom_\Cc(-,x)$ corresponds to the right fibration $\Cc_{/x}\rightarrow \Cc$. Likewise, $\Yo_\Dd(F(x))\simeq \Hom_\Dd(-,F(x))$ corresponds to $\Dd_{/F(x)}\rightarrow \Dd$. Using \cref{lem:KanExtensionForRight}\cref{enum:RightPullbackLeftAdjoint}, we only have to show that the top horizontal arrow in the diagram
	\begin{equation*}
		\begin{tikzcd}
			\Cc_{/x}\dar\rar\drar[commutes]& \Dd_{/F(x)}\dar\\
			\Cc\rar["F"]&\Dd
		\end{tikzcd}
	\end{equation*}
	is coinitial. But that's easy! Both $\Cc_{/x}$ and $\Dd_{/F(x)}$ have terminal objects, hence there are adjunctions $\Cc_{/x}\shortdoublelrmorphism \{\id_x\}$ and $\Dd_{/F(x)}\shortdoublelrmorphism\{\id_{F(x)}\}$. Hence $*\rightarrow \Cc_{/x}$ and $*\rightarrow \Dd_{/F(x)}$ are both coinitial by \cref{exm:Cofinal}\cref{enum:RightAdjointCofinal}. Since being coinitial is closed under $2$-out-of-$3$ (for example, by the condition from \cref{thm:JoyalsQuillenA}\cref{enum:Cofinal}), $\Cc_{/x}\rightarrow \Dd_{/F(x)}$ must be coinitial too.
\end{proof}
%Before we move on to define and construct general Kan extensions in the $\infty$-categorical world, let us record a very pleasant consequence of \cref{cor:f_!:PC->PD}: We can compute Hom animae in functor $\infty$-categories!
\cref{cor:f_!:PC->PD} allows us to compute Hom animae in functor $\infty$-categories!
\begin{cor}\label{cor:HomInFunctorCats}
	Given functors $F,G\colon \Cc\rightarrow\Dd$ of $\infty$-categories, the anima of natural transformations $\Hom_{\Fun(\Cc,\Dd)}(F,G)$ can be computed as the following limit:
	\begin{equation*}
		\limit_{(x\rightarrow y)\in\TwAr(\Cc)}\Hom_\Dd\bigl(F(x),G(y)\bigr)\coloneqq \limit\left(\TwAr(\Cc)\xrightarrow{\!(s,t)\!}\Cc^\op\times\Cc\xrightarrow{\!F^\op\times G\!}\Dd^\op\times\Dd\xrightarrow{\!\Hom_\Dd\!}\cat{An}\right)\,.
	\end{equation*}
\end{cor}
\begin{proof}
	By \cref{lem:ColimitsInAnima}, the right-hand side can be computed as $\Hom_{\Cat_{\infty/\TwAr(\Cc)}}(\TwAr(\Cc),\Uu)$, where $\Uu$ denotes the unstraightening of ${\Hom_\Dd}\circ(F^\op\times G)\circ (s,t)\colon \TwAr(\Cc)\rightarrow\cat{An}$. Since unstraightening transforms compositions into pullbacks and the unstraightening of $\Hom_\Dd$ is $\TwAr(\Dd)\rightarrow \Dd^\op\times\Dd$ by \cref{con:HomInTwoVariables} or~\labelcref{con:HomTwAr}, we have a pullback diagram
	\begin{equation*}
		\begin{tikzcd}
			\Uu\rar\dar\drar[pullback] & \Uu'\rar\dar\drar[pullback] &[1em] \TwAr(\Dd)\dar\\
			\TwAr(\Cc)\rar["{(s,t)}"] & \Cc^\op\times\Cc\rar["F^\op\times G"] & \Dd^\op\times\Dd
		\end{tikzcd}
	\end{equation*}
	Using \cref{lem:KanExtensionForRight}\cref{enum:RightPullbackLeftAdjoint}, we see $\Hom_{\Cat_{\infty/\TwAr(\Cc)}}(\TwAr(\Cc),\Uu)\simeq\Hom_{\Cat_{\infty/\Cc^\op\times\Cc}}(\TwAr(\Cc),\Uu')$. But these are both left fibrations over $\Cc^\op\times\Cc$, so the Hom anima on the right-hand side can be equivalently computed as $\Hom_{\Fun(\Cc^\op\times\Cc,\cat{An})}(\Hom_\Cc,{\Hom_\Dd}\circ(F^\op\times G))$. Now the \enquote{currying} equivalence  $\Fun(\Cc^\op\times\Cc,\cat{An})\simeq\Fun(\Cc,\PSh(\Cc))$ sends $\Hom_\Cc$ to $\Yo_\Cc$ and ${\Hom_\Dd}\circ(F^\op\times G)$ to $F^*\circ \Yo_\Dd\circ G$, hence the Hom anima under consideration is given by
	\begin{align*}
		\Hom_{\Fun(\Cc,\PSh(\Cc))}\left(\Yo_\Cc,F^*\circ \Yo_\Dd\circ G\right)&\simeq \Hom_{\Fun(\Cc,\PSh(\Dd))}\left(F_!\circ \Yo_\Cc,\Yo_\Dd\circ G\right)\\
		&\simeq \Hom_{\Fun(\Cc,\PSh(\Dd))}\left(\Yo_\Dd\circ F,\Yo_\Dd\circ G\right)\\
		&\simeq \Hom_{\Fun(\Cc,\Dd)}(F,G)\,,
	\end{align*}
	as claimed. For the first equivalence, we use that $F_!\circ -$ is an adjoint of $F^*\circ -$ by construction and \cref{cor:FunctorCategoryAdjunctions}, the second equivalence follows from \cref{cor:f_!:PC->PD}, and the third one since $\Yo_\Dd\colon \Dd\rightarrow\PSh(\Dd)$ is fully faithful by Yoneda's lemma (\cref{cor:YonedaEmbeddingFullyFaithful}).
\end{proof}
We'll now define and construct Kan extensions in the $\infty$-categorical world.
\begin{defi}\label{def:KanExtensions}
	Let $f\colon \Cc\rightarrow \Cc'$ and $F\colon \Cc\rightarrow \Dd$ be functors of $\infty$-categories. A~\emph{left Kan extension of $F$ along $f$}, denoted $\Lan_fF\colon \Cc'\rightarrow\Dd$, is a left adjoint object to $F$ under $f^*\colon \Fun(\Cc',\Dd)\rightarrow\Fun(\Cc,\Dd)$. Dually, a \emph{right Kan extension of $F$ along $f$}, denoted $\Ran_fF\colon \Cc'\rightarrow\Dd$, is a right adjoint object to $F$ under $f^*$.
\end{defi}
Kan extensions in the $\infty$-categorical world can be computed by the same formula as in the ordinary case (\cref{lem:1KanExtensionFormula}):
\begin{lem}[Kan extension formula]\label{lem:KanExtensionFormula}
	In the situation of \cref{def:KanExtensions}, assume that for all $x'\in \Cc'$ the following colimits exist in $\Dd$:
	\begin{equation*}
		\colimit_{(x,f(x)\rightarrow x')\in \Cc_{/x'}}F(x)\coloneqq \colimit\left(\Cc_{/x'}\longrightarrow \Cc\overset{F}{\longrightarrow}\Dd\right)
	\end{equation*}
	Then $\Lan_fF$ exists and $\Lan_fF(x')$ is given by that colimit.
\end{lem}
To prove this, we first show that taking colimits is functorial in both the indexing $\infty$-category and the functor. As it will turn out during the proof, this is equivalent to constructing a partial left adjoint to the Yoneda embedding $\Yo_\Dd\colon \Dd\rightarrow\PSh(\Dd)$.
\begin{lem}[\enquote{Colimits are functorial}]\label{lem:ColimitsFunctorial}
	Let $\Dd$ be an $\infty$-category. Let $\Tt\subseteq\Cat_{\infty/\Dd}$ be spanned by those $\alpha\colon \Ii\rightarrow \Dd$ that admit a colimit. Consider the functor $\Dd_{/-}\colon \Dd\rightarrow \cat{Cat}_{\infty/\Dd}$ that sends $y\in \Dd$ to $\Dd_{/y}\rightarrow \Dd$. Then $\Dd_{/-}$ lands in $\Tt$ and admits a left adjoint $\colimit\colon \Tt\rightarrow \Dd$ that sends $\alpha\colon \Ii\rightarrow\Dd$ to $\colimit_{i\in\Ii}\alpha(i)\in\Dd$.
\end{lem}
\begin{proof}
	Formally, the functor $\Dd_{/-}\colon \Dd\rightarrow\cat{Cat}_{\infty/\Dd}$ is defined via
	\begin{equation*}
		\Dd\xrightarrow{\Yo_\Dd}\PSh(\Dd)\simeq\cat{Right}(\Dd)\longrightarrow \cat{Cat}_{\infty/\Dd}\,,
	\end{equation*}
	using the Yoneda embedding and the right straightening equivalence (the dual of \cref{thm:Straightening}\cref{enum:LeftStraightening}). It's clear that $\Dd_{/-}$ takes values in $\Tt$. Indeed, $\Dd_{/y}$ has a terminal object and so the colimit over $\Dd_{/y}\rightarrow \Dd$ is just $y$. To prove the second assertion, by \cref{lem:Adjunction}, it's enough to prove that for every $\alpha\colon \Ii\rightarrow \Dd$, the colimit $\colimit_{i\in\Ii}\alpha(i)\in\Dd$ is a left adjoint object to $\alpha$ under $\Dd_{/-}\colon \Dd\rightarrow\Tt$. This can be seen as follows: If $c\simeq \colimit_{i\in\Ii}\alpha(i)$, then the associated natural transformation $\alpha\Rightarrow\const c$ induces a functor $u_\alpha\colon \Ii\rightarrow \Dd_{/c}$ in $\cat{Cat}_{\infty/\Dd}$. We then get a natural transformation
	\begin{equation*}
		\Hom_\Dd(c,-)\xRightarrow{\Dd_{/-}}\Hom_{\Cat_{\infty/\Dd}}\bigl(\Dd_{/c},\Dd_{/-}\bigr)\xRightarrow{u_\alpha^*}\Hom_{\Cat_{\infty/\Dd}}\bigl(\Ii,\Dd_{/-}\bigr)\,.
	\end{equation*}
	Equivalences can be checked pointwise by \cref{thm:EquivalencePointwise}. So choose $y\in\Dd$. We compute
	\begin{align*}
		\Hom_{\Cat_{\infty/\Dd}}\bigl(\Ii,\Dd_{/y}\bigr)&\simeq \{\alpha\}\times_{\Hom_{\Cat_\infty}(\Ii,\Dd),s}\Hom_{\Cat_\infty}\bigl(\Ii,\Ar(\Dd)\times_{t,\Dd}\{y\}\bigr)\\
		&\simeq \{\alpha\}\times _{\Hom_{\Cat_\infty}(\Ii,\Dd),s}\Hom_{\Cat_\infty}\bigl(\Ii,\Ar(\Dd)\bigr)\times_{t,\Hom_{\Cat_\infty}(\Ii,\Dd)}\{\const y\}\\
		&\simeq \Hom_{\Fun(\Ii,\Dd)}(\alpha,\const y)\,,
	\end{align*}
	and this agrees with $\Hom_\Dd(c,y)$ by definition of $c$. In the first step we use \cref{cor:HomInSliceCategories} as well as $\Dd_{/y}\simeq \Ar(\Dd)\times_{t,\Dd}\{y\}$. In the second step we use \cref{cor:HomPreservesLimits}. In the third step, we use  \enquote{currying} in the form of $\Hom_{\Cat_\infty}(\Ii,\Ar(\Dd))\simeq \Hom_{\Cat_\infty}(\Delta^1,\Fun(\Ii,\Dd))$ and then plug in the definition of $\Hom_{\Fun(\Ii,\Dd)}$ as in \cref{par:HomInQuasiCategories}.
\end{proof}
\begin{proof}[Proof of \cref{lem:KanExtensionFormula}]
	Consider the diagram of functors
	\begin{equation*}
		\begin{tikzcd}
			\Fun(\Cc,\Dd)\rar["{(\Yo_\Dd)_*}"] &  \Fun\bigl(\Cc,\PSh(\Dd)\bigr)\dar["\simeq"']\rar[dashed] &[2em]  \Fun\bigl(\Cc',\cat{Right}(\Dd)\bigr)\dar["\simeq"]\\
			& \cat{Right}(\Dd\times\Cc^\op)\rar["(\id_\Dd\times f^\op)_!"] & \cat{Right}\bigl(\Dd\times(\Cc')^\op\bigr)
		\end{tikzcd}
	\end{equation*}
	(the vertical equivalences follow from the right straightening equivalence, see the dual of \cref{thm:Straightening}\cref{enum:LeftStraightening}). Let $F'\colon \Cc'\rightarrow\cat{Right}(\Dd)$ denote the image of $F$ under the top row functors and let $\ov{\Tt}\coloneqq \Tt\cap\cat{Right}(\Dd)$, where $\Tt$ is defined as in \cref{lem:ColimitsFunctorial}. If we can show that $F'$ is contained in the full sub-$\infty$-category $\Fun(\Cc',\ov\Tt)\subseteq \Fun(\Cc',\cat{Right}(\Dd))$, then we can define $\Lan_fF\coloneqq{\colimit}\circ F'\in \Fun(\Cc',\Dd)$. It's clear from the various equivalences and adjunctions involved (more precisely, from \cref{cor:YonedaEmbeddingFullyFaithful}, \cref{lem:KanExtensionForRight}\cref{enum:RightPullbackLeftAdjoint}, and \cref{lem:ColimitsFunctorial} combined with \cref{cor:FunctorCategoryAdjunctions}) that $\Lan_fF$ is indeed a left adjoint object of $F$ under the precomposition functor $f^*\colon \Fun(\Cc',\Dd)\rightarrow \Fun(\Cc,\Dd)$.
	
	So we have to check that $F'$ is indeed contained in $\Fun(\Cc',\ov\Tt)$. The image of $F$ under $(\Yo_\Dd)_*$ followed by the \enquote{currying} equivalence $\Fun(\Cc,\Fun(\Dd^\op,\cat{An}))\simeq \Fun(\Dd^\op\times\Cc,\cat{An})$ is $\Hom_\Dd(-,F(-))\colon \Dd^\op\times\Cc\rightarrow \cat{An}$. Its right unstraightening is
	\begin{equation*}
		\TwAr(\Dd)^\op\times_{t^\op,\Dd^\op,F^\op}\Cc^\op\longrightarrow \Dd\times\Cc^\op\,.
	\end{equation*}
	Indeed, the right unstraightening of $\Hom_\Dd\colon \Dd^\op\times\Dd\rightarrow\cat{An}$ is $(s^\op,t^\op)\colon \TwAr(\Dd)^\op\rightarrow \Dd\times\Dd^\op$ by definition (of either $\Hom_\Dd$ or $\TwAr(\Dd)$, see \cref{con:HomInTwoVariables,con:HomTwAr}), and precomposition with $F\colon \Cc\rightarrow\Dd$ corresponds to pullback along $F^\op$.
	
	By  \cref{lem:KanExtensionForRight}\cref{enum:RightPullbackLeftAdjoint}, the functor $(\id_\Dd\times f^\op)_!$ sends $\TwAr(\Dd)^\op\times_{t^\op,\Dd^\op,F^\op}\Cc^\op\rightarrow \Dd\times\Cc^\op$ to a coinitial replacement of $\TwAr(\Dd)^\op\times_{t^\op,\Dd^\op,F^\op}\Cc^\op\rightarrow \Dd\times\Cc^\op\rightarrow \Dd\times(\Cc')^\op$ by a right fibration. To figure out how such a coinitial replacement looks like, we claim the following:
	\begin{alphanumerate}\itshape
		\item[\boxtimes_1] In the diagram below, both vertical arrows are coinitial:\label{claim:TwArCofinal}
		\begin{equation*}
			\begin{tikzcd}[column sep=-7em]
				& \TwAr(\Dd)^\op\times_{t^\op,\Dd^\op,F\circ t^\op}\TwAr(\Cc)^\op\times_{f\circ s^\op,\Cc',s^\op}\TwAr(\Cc')^\op\dlar[start anchor=186]\drar[start anchor=-6]& \\
				\Cc\times_{f,\Cc',s^\op}\TwAr(\Cc')^\op & & \TwAr(\Dd)^\op\times_{t^\op,\Dd^\op,F^\op}\Cc^\op
			\end{tikzcd}
		\end{equation*} 
	\end{alphanumerate}
	To prove claim~\cref{claim:TwArCofinal}, we first observe that for every $\infty$-category $\Ii$, both the source projection $s^\op\colon \TwAr(\Ii)^\op\rightarrow \Ii$ and the target projection $t^\op\colon \TwAr(\Ii)^\op\rightarrow \Ii^\op$ are coinitial cartesian fibrations. Indeed, cartesianness is clear since $\TwAr(\Ii)^\op\rightarrow\Ii\times\Ii^\op$ is a right fibration and projection to either factor is cartesian. For coinitiality, we use \cref{lem:CartesianCofinal}: The fibre of $t^\op$ over $i\in \Ii^\op$ is $(t^\op)^{-1}\{i\}\simeq \Ii_{/i}$; this follows from \cref{lem:HomRealityCheck}, regardless of which construction of $\TwAr(\Ii)$ you use. Now $\Ii_{/i}$ is weakly contractible since it has a terminal object. The same argument applies to $s^\op$. To apply this observation, observe that in the diagram above, the left vertical arrow is a composition of a base change of $t^\op\colon \TwAr(\Cc)^\op\rightarrow \Cc^\op$ and a base change of $s^\op\colon \TwAr(\Cc')^\op\rightarrow \Cc'$. Since the conditions from \cref{lem:CartesianCofinal} are stable under base change, this proves that the left vertical arrow is indeed coinitial. Similarly, the right vertical arrow is a composition of a base change of $s^\op\colon \TwAr(\Cc)^\op\rightarrow \Cc$ and a base change of $t^\op\colon \TwAr(\Dd)^\op\rightarrow \Dd^\op$, whence the same argument applies.
	
	So we may equivalently look for a coinitial replacement of $\Cc\times_{f,\Cc',s^\op}\TwAr(\Cc')^\op\rightarrow \Dd\times(\Cc')^\op$ by a right fibration. Once again, we won't do this directly; instead, we claim another claim:
	\begin{alphanumerate}\itshape
		\item[\boxtimes_2] In the diagram below, the vertical arrows are cartesian fibrations over $(\Cc')^\op$ and the horizontal arrows preserve cartesian lifts:\label{claim:CartesianDiagram}
		\begin{equation*}
			\begin{tikzcd}[column sep=large]
				\Dd\times(\Cc')^\op\drar["\pr_2"']&[1em] \Cc\times(\Cc')^\op\dar["\pr_2"]\lar["F\times\id_{(\Cc')^\op}"']\dar[phantom,""{name=A}]\arrow[from=1-1,to=A,commutes,pos=0.7]\dar[phantom,""{name=A}]\arrow[from=A,to=1-3,commutes,pos=0.3] &\Cc\times_{\Cc',s^\op}\TwAr(\Cc')^\op \lar["{({s^\op},\,{t^\op})}"']\dlar["t^\op"] \\
				&(\Cc')^\op &
			\end{tikzcd}
		\end{equation*}
	\end{alphanumerate}
	Indeed, by definition of $\TwAr(\Cc')$, the arrow labelled $(s^\op,t^\op)$ is a right fibration, and it's clear that both arrows labelled $\pr_2$ are cartesian fibrations (see \cref{exm:Straightening}\cref{enum:ProjectionsStraightenToConstantFunctors}). Hence $t^\op$, being a composition of cartesian fibrations, is cartesian too. Furthermore, by a simple unravelling, we see that $t^\op$-cartesian lifts are precisely the $(s^\op,t^\op)$-cartesian lifts of $\pr_2$-cartesian lifts, which immediately proves that $(s^\op,t^\op)$ preserves cartesian lifts. Finally, it's clear that $F\times\id_{(\Cc')^\op}$ preserves cartesian lifts, since these are given by those morphisms in $\Cc\times(\Cc')^\op$ and $\Dd\times(\Cc')^\op$ that are equivalences in the first component. This proves claim~\cref{claim:CartesianDiagram}.
	
	The cartesian straightening $\operatorname{St}^{(\mathrm{cart})}(t^\op)$ is a functor $\Cc'\rightarrow\cat{Cat}_\infty$. By the diagram above, it comes with a natural transformation $\operatorname{St}^{(\mathrm{cart})}(t^\op)\Rightarrow \const \Dd$, so that $\operatorname{St}^{(\mathrm{cart})}(t^\op)$ lifts to a functor $\Cc_{/-}\colon \Cc'\rightarrow \cat{Cat}_{\infty/\Dd}$. On objects, $\Cc_{/-}$ is given by sending $x'\in \Cc'$ to the slice category $\Cc_{/x'}$, which becomes an object in $\cat{Cat}_{\infty/\Dd}$ via
	\begin{equation*}
		\Cc_{/x'}\longrightarrow \Cc\overset{F}{\longrightarrow}\Dd\,.
	\end{equation*}
	Now that's something we've seen before! Our assumption that the functor above admits a colimit precisely tells us that $\Cc_{/-}$ restricts to a functor $\Cc_{/-}\colon \Cc'\rightarrow\Tt$. To finish the proof, let $c\colon \cat{Cat}_{\infty/\Dd} \rightarrow \cat{Right}(\Dd)$ denote the left adjoint to $\cat{Right}(\Dd)\subseteq \cat{Cat}_{\infty/\Dd}$, which exists due to \cref{lem:KanExtensionForRight}\cref{enum:RightCofinalLeftAdjoint}. It's clear from \cref{thm:JoyalsQuillenA}\cref{enum:Cofinal} that $c$ sends $\Tt$ to $\ov\Tt$, hence we obtain a functor $c\circ \Cc_{/-}\colon \Cc'\rightarrow\ov\Tt$. We claim that this finally allows us to compute the desired coinitial replacement:
	\begin{alphanumerate}\itshape
		\item[\boxtimes_3] If $p\colon X\rightarrow \Dd\times(\Cc')^\op$ is a coinitial replacement of $\Cc\times_{f,\Cc',s^\op}\TwAr(\Cc')^\op\rightarrow \Dd\times(\Cc')^\op$ by a right fibration, then the image of $p$ under $\cat{Right}(\Dd\times(\Cc')^\op)\simeq \Fun(\Cc',\cat{Right}(\Dd))$ will coincide with $c\circ \Cc_{/-}$.\label{claim:CofinalReplacement}
	\end{alphanumerate}
	To prove claim~\cref{claim:CofinalReplacement}, consider the following diagram, in which the dashed arrows are left adjoints (whose existence we're going to prove below):
	\begin{equation*}
		\begin{tikzcd}
			\bigl(\Cat_{\infty/(\Cc')^\op}\bigr)_{/\Dd\times(\Cc')^\op}\dar["\simeq"']\drar[commutes] & \cat{Cart}\bigl((\Cc')^\op\bigr)_{/\Dd\times(\Cc')^\op}\rar["\simeq"]\lar\dar[dashed,shift right=0.2em,"c"']\drar[commutes] & \Fun\bigl(\Cc',\Cat_{\infty/\Dd}\bigr)\dar[dashed,shift right=0.2em,"c_*"']\\
			\Cat_{\infty/\Dd\times(\Cc')^\op}\rar[dashed, shift left=0.2em, "c"] & \cat{Right}\bigl(\Dd\times(\Cc')^\op\bigr)\rar["\simeq"]\lar[shift left=0.2em]\uar[shift right=0.2em] & \Fun\bigl(\Cc',\cat{Right}(\Dd)\bigr)\uar[shift right=0.2em]
		\end{tikzcd}
	\end{equation*}
	The horizontal equivalences as well as commutativity of the square on the right follow from the cartesian straightening equivalence (the dual of \cref{thm:Straightening}). Furthermore, once we know that the left adjoints exist, they will also form a commutative square on the right, since taking left adjoints is always compatible with equivalences. The vertical equivalence on the left follows by inspection (\enquote{a slice of a slice is a slice}). The vertical left adjoint $c_*$ exists by \cref{cor:FunctorCategoryAdjunctions}. The horizontal left adjoint $c\colon \Cat_{\infty/\Dd\times(\Cc')^\op}\rightarrow \cat{Right}(\Dd\times(\Cc')^\op)$ exists by \cref{lem:KanExtensionForRight}\cref{enum:RightCofinalLeftAdjoint}, and it we claim that it induces a left adjoint
	\begin{equation*}
		c\colon  \cat{Cart}\bigl((\Cc')^\op\bigr)_{/\Dd\times(\Cc')^\op}\longrightarrow \cat{Right}\bigl(\Dd\times(\Cc')^\op\bigr)
	\end{equation*}
	to the forgetful functor $\cat{Right}(\Dd\times(\Cc')^\op)\rightarrow \cat{Cart}((\Cc')^\op)_{/\Dd\times(\Cc')^\op}$. Indeed, if $\Uu\rightarrow \Dd\times(\Cc')^\op$ and $\Uu'\rightarrow \Dd\times(\Cc')^\op$ are objects in $\cat{Cart}((\Cc')^\op)_{/\Dd\times(\Cc')^\op}$, then
	\begin{equation*}
		\Hom_{\cat{Cart}((\Cc')^\op)_{/\Dd\times(\Cc')^\op}}(\Uu,\Uu')\longrightarrow \Hom_{\cat{Cat}_{\infty/\Dd\times(\Cc')^\op}}(\Uu,\Uu')
	\end{equation*}
	is usually \emph{not} an equivalence, only an inclusion of path components, since on the left-hand side, cartesian lifts need to be preserved. However, if $\Uu'\rightarrow \Dd\times(\Cc')^\op$ happens to be a right fibration, then cartesian lifts are preserved automatically\footnote{It's easy to get confused here: $\Uu'\rightarrow (\Cc')^\op$ need not be a right fibration, so we can't appeal to (the dual of) \cref{lem:CocartesianLeft} directly. But the argument is still straightforward: If $\Uu'\rightarrow \Dd\times (\Cc')^\op$ is a right fibration, then \emph{any} lift of a cartesian morphism in $\Dd\times(\Cc')^\op$ will be cartesian again, thanks to (the dual of) \cref{lem:CocartesianLeft}. So a morphism $\Uu\rightarrow\Uu'$ in $\cat{Cat}_{\infty/\Dd\times(\Cc')^\op}$ preserves cocartesian lifts if and only if $\Uu\rightarrow \Dd\times(\Cc')^\op$ does. But the latter is true by definition, since $\Uu\rightarrow \Dd\times(\Cc')^\op$ is a morphism in $\cat{Cart}((\Cc')^\op)$ if $\Uu$ is an object of the slice $\infty$-category $\cat{Cart}((\Cc')^\op)_{/\Dd\times(\Cc')^\op}$.}, so in this case we \emph{do} get an equivalence, which proves that $c$ is still a left adjoint when restricted along the non-fully faithful functor $\cat{Cart}((\Cc')^\op)_{/\Dd\times(\Cc')^\op}\rightarrow \cat{Cat}_{\infty/\Dd\times(\Cc')^\op}$. So we've proved that the diagram above also commutes if we take the dashed left adjoints into account. This is precisely what we need to prove claim~\cref{claim:CofinalReplacement}.
	
	Using claim~\cref{claim:CofinalReplacement}, we've now succeeded in proving that $F\colon \Cc'\rightarrow\cat{Right}(\Dd)$ takes values in $\ov\Tt$, which proves that $\Lan_fF$ exists. Furthermore, for every $c'\in\Cc'$, the value $\Lan_fF(x')$ is given by a colimit over $c(\Cc_{/x'})$. Since the unit morphism $u_{\Cc_{/x'}}\colon \Cc_{/x'}\rightarrow c(\Cc_{/x'})$ is coinitial by \cref{lem:KanExtensionForRight}\cref{enum:RightPullbackLeftAdjoint}, we may as well take the colimit over $\Cc_{/x'}$. This proves that $\Lan_fF(x')$ is given by the desired formula and we're finally done!		
\end{proof}
\begin{cor}%[\enquote{Kan extensions along fully faithful functors behave nicely.}]
	\label{cor:KanExtensionAlongFullyFaithful}
	In the situation from \cref{def:KanExtensions}, assume that $f\colon \Cc\rightarrow \Cc'$ is fully faithful and that the colimits from \cref{lem:KanExtensionFormula} exist in $\Dd$. Then the natural transformation $u_F\colon F\Rightarrow \Lan_fF\circ f$ is an equivalence.
\end{cor}
\begin{proof}
	This follows from the same argument as in \cref{cor:1KanExtensionAlongFullyFaithful}, plus the fact that equivalences can be checked pointwise by \cref{thm:EquivalencePointwise}.
\end{proof}
We can now state the main result of this section: the $\infty$-categorical analogue of \cref{thm:1PShFreeCocompletion}!
\begin{thm}[\enquote{$\PSh(\Cc)$ arises by freely adding colimits to $\Cc$.}]\label{thm:PShFreeCocompletion}
	Let $\Cc$ and $\Dd$ be $\infty$-categories, where $\Dd$ has all colimits. Then restriction along the Yoneda embedding $\Yo_\Cc$ induces an equivalence
	\begin{equation*}
		\Yo_\Cc^*\colon \Fun^{\colimit}\bigl(\PSh(\Cc),\Dd\bigr)\overset{\simeq}{\longrightarrow}\Fun(\Cc,\Dd)\,.
	\end{equation*}
	Here $\Fun^{\colimit}(\PSh(\Cc),\Dd)\subseteq \Fun(\PSh(\Cc),\Dd)$ is the full sub-$\infty$-category spanned by the colimits-preserving functors. Furthermore, every colimits-preserving functor $\PSh(\Cc)\rightarrow \Dd$ admits a right adjoint.
\end{thm}
As it turns out, the proof will be exactly the same as for ordinary categories. Let's start with the two lemmas whose proofs where omitted in the ordinary case.%; at last, this will be rectified now!
\begin{lem}[\enquote{Every presheaf is a colimit of representables.}]\label{lem:PresheafColimitOfRepresentables}
	Let $\Cc$ be an $\infty$-category. For every $E\in \PSh(\Cc)$, the natural morphism
	\begin{equation*}
		\colimit_{(y,\Hom_\Cc(-,y)\rightarrow E)\in \Cc_{/E}}\Hom_\Cc(-,y)\overset{\simeq}{\longrightarrow}E
	\end{equation*}
	 is an equivalence.
\end{lem}
\begin{proof}
	Since we get the natural transformation for free, we can check pointwise whether it is an equivalence (\cref{thm:EquivalencePointwise}). So fix $x\in\Cc$. Since colimits in $\PSh(\Cc)$ are computed pointwise (\cref{lem:ColimitsInFunctorCategories}), what we need to show is
	\begin{equation*}
		\colimit\left(\Cc_{/E}\overset{s}{\longrightarrow}\Cc\xrightarrow{\Hom_\Cc(x,-)}\cat{An}\right)\simeq E(x)\,.
	\end{equation*}
	By \cref{lem:ColimitsInAnima}, the colimit on the left-hand side is given by $\abs*{\Uu}$, where $\Uu$ is the unstraightening of $\Hom_\Cc(x,-)\circ s$. Since precomposition transforms into pullbacks under unstraightening, we find that $\Uu$ sits inside a pullback
	\begin{equation*}
		\begin{tikzcd}
			\Uu\rar\dar\drar[pullback] &\Cc_{/E}\rar \dar["s"]\drar[pullback]& \PSh(\Cc)_{/E}\dar["s"]\\
			\Cc_{x/}\rar & \Cc\rar["\Yo_\Cc"] & \PSh(\Cc)
		\end{tikzcd}
	\end{equation*}
	Since $\PSh(\Cc)_{/E}\rightarrow\PSh(\Cc)$ is a right fibration, $\Uu\rightarrow \Cc_{x/}$ is one too. In particular, it is a cartesian fibration. Hence \cref{lem:CartesianFibres} shows $\abs{\Uu\times_{\Cc_{x/}}\{\id_x\}}\simeq \abs{\Uu\times_{\Cc_{x/}}(\Cc_{x/})_{{(\id_x\colon x\rightarrow x)}/}}\simeq \abs{\Uu}$; here we use $(\Cc_{x/})_{{(\id_x\colon x\rightarrow x)}/}\simeq \Cc_{x/}$ (\enquote{a slice of a slice is a slice}). Now
	\begin{align*}
		\bigl\lvert\Uu\times_{\Cc_{x/}}\{\id_x\}\bigr\rvert\simeq \Uu\times_{\Cc_{x/}}\{\id_x\}&\simeq \PSh(\Cc)_{/E}\times_{\PSh(\Cc)}\left\{\Yo_\Cc(x)\right\}\\
		&\simeq \Hom_{\PSh(\Cc)}\bigl(\Yo_\Cc(x),E\bigr)\\
		&\simeq E(x)\,.
	\end{align*}
	In the first step, we use that the fibre $\Uu\times_{\Cc_{x/}}\{\id_x\}$ is already an anima, since $\Uu\rightarrow \Cc_{x/}$ is a right fibration. The second equivalence follows from the pullback diagram above. In the third step, we use the definition of $\Hom_{\PSh(\Cc)}$, and in the fourth step, we use Yoneda's lemma (\cref{thm:Yoneda}). In total, we find $\abs*{\Uu}\simeq E(x)$, which is exactly what we wanted to prove.
\end{proof}
\begin{lem}\label{lem:LanAlongYonedaHasRightAdjoint}
	For every $F\colon \Cc\rightarrow \Dd$, the left Kan extension $\Lan_{\Yo_\Cc}F\colon \PSh(\Cc)\rightarrow\Dd$ \embrace{which exists due to \cref{lem:KanExtensionFormula}} admits a right adjoint. The right adjoint sends $y\in \Dd$ to $\Hom_\Dd(F(-),y)\colon \Cc^\op\rightarrow\cat{Set}$.
\end{lem}
\begin{proof}
	Fix $y\in\Dd$. Since adjoints can be constructed pointwise (\cref{lem:Adjunction}), we only need to construct an equivalence
	\begin{equation*}
		\Hom_\Dd\bigl(\Lan_{\Yo_\Cc}F(-),y\bigr)\simeq \Hom_{\PSh(\Cc)}\bigl(-,\Hom_\Dd(F(-),y)\bigr)
	\end{equation*}
	of functors $\PSh(\Cc)^\op\rightarrow\cat{An}$. Restricting along $\Yo_\Cc^\op\colon \Cc^\op\rightarrow\PSh(\Cc)^\op$, both sides become $\Hom_\Dd(F(-),y)$: The left-hand side by \cref{cor:KanExtensionAlongFullyFaithful}, the right-hand side by Yoneda's lemma (\cref{thm:Yoneda}; see also \cref{par:YonedaFunctorial}). By the universal property of right Kan extension, we thus obtain natural transformations
	\begin{equation*}
		\Hom_\Dd\bigl(\Lan_{\Yo_\Cc}F(-),y\bigr)\Longrightarrow \Ran_{\Yo_\Cc^\op} \Hom_\Dd\bigl(F(-),y\bigr)\Longleftarrow\Hom_{\PSh(\Cc)}\bigl(-,\Hom_\Dd\bigl(F(-),y\bigr)\bigr)\,.
	\end{equation*}
	We claim that they're both equivalences. In either case, this can be checked pointwise by \cref{thm:EquivalencePointwise}. So plug in some $E\in \PSh(\Cc)$. We obtain a diagram
	\begin{equation*}
		\begin{tikzcd}[column sep=-0.5em]
			\Hom_\Dd\bigl(\Lan_{\Yo_\Cc}F(E),y\bigr)\rar\drar[bend right=15,end anchor=180,"\simeq"']\drar[commutes,pos=0.55]& \Ran_{\Yo_\Cc^\op} \Hom_\Dd\bigl(F(E),y\bigr)\dar["\simeq"] &\Hom_{\PSh(\Cc)}\bigl(E,\Hom_\Dd\bigl(F(-),y\bigr)\bigr)\dlar[commutes,pos=0.55]\lar\dlar[bend left=15,end anchor=0,"\simeq"]\\
			& \limit_{(x,\Yo_\Cc(x)\rightarrow E)\in (\Cc_{/E})^\op}\Hom_\Dd\bigl(F(x),y\bigr)
		\end{tikzcd}
	\end{equation*}
	The vertical arrow in the middle is an equivalence by the dual of \cref{lem:KanExtensionFormula}. For the vertical arrow on the left, we plug in the left Kan extension formula from \cref{lem:KanExtensionFormula} and use \cref{cor:HomPreservesColimits} to see that $\Hom_\Dd(-,y)$ transforms the colimit into a limit. For the vertical arrow on the right, we plug in \cref{lem:PresheafColimitOfRepresentables}, use \cref{cor:HomPreservesColimits} again to see that $\Hom_{\PSh(\Cc)}(-,\Hom_\Dd(F(-),y))$ transforms the colimit into a limit, and then use Yoneda's lemma. This proves that we obtain equivalences as desired.
\end{proof}
%The final ingredient in the proof of \cref{thm:PShFreeCocompletion} is the $\infty$-categorical analogue of \cref{lem:1FullyFaithfulConservativeAdjunction}.
\begin{lem}\label{lem:FullyFaithfulConservativeAdjunction}
	Let $\Cc$ and $\Dd$ be categories and let $L\colon \Cc\shortdoublelrmorphism \Dd\noloc R$ be an adjunction.
	\begin{alphanumerate}
		\item \!The left adjoint $L$ is fully faithful if and only if the unit transformation $u\colon \id_\Cc\Rightarrow RL$ is an equivalence.\label{enum:FullyFaithfulIffUnitEquivalence}
		\item Suppose the condition from \cref{enum:FullyFaithfulIffUnitEquivalence} is true. Furthermore, suppose that $R$ is conservative \embrace{that is, if $\alpha\colon x\rightarrow y$ is a morphism in $\Dd$ such that $R(\alpha)$ is an equivalence, then $\alpha$ is an equivalence too}. Then $L$ and $R$ are inverse equivalences of categories.\label{enum:Conservative}
	\end{alphanumerate}
\end{lem}
\begin{proof}
	The proof of \cref{lem:1FullyFaithfulConservativeAdjunction} can be copied verbatim.
\end{proof}
\begin{proof}[Proof of \cref{thm:PShFreeCocompletion}]
	By \cref{lem:LanAlongYonedaHasRightAdjoint} and \cref{lem:AdjointsPreserveColimits}, the adjunction $\Lan_{\Yo_\Cc}\dashv\Yo_\Cc^*$ restricts to an adjunction
	\begin{equation*}
		\Lan_{\Yo_\Cc}\colon \Fun(\Cc,\Dd)\doublelrmorphism\Fun^{\colimit}\bigl(\PSh(\Cc),\Dd\bigr)\noloc \Yo_\Cc^*\,.
	\end{equation*}
	By \cref{lem:FullyFaithfulConservativeAdjunction}\cref{enum:Conservative}, to prove that $\Lan_{\Yo_\Cc}$ and $\Yo_\Cc^*$ are inverse equivalences, we need to show that the unit $u\colon \id_{\Fun(\Cc,\Dd)}\Rightarrow\Yo_\Cc^*\circ \Lan_{\Yo_\Cc}$ is an equivalence and that $\Yo_\Cc^*$ is conservative. That $u$ is an equivalence can be checked object-wise by \cref{thm:EquivalencePointwise}, where it follows from \cref{cor:KanExtensionAlongFullyFaithful}, since the Yoneda embedding $\Yo_\Cc$ is fully faithful (\cref{cor:YonedaEmbeddingFullyFaithful}). To see that $\Yo_\Cc^*$ is conservative, we must show that a natural transformation $\eta\colon F\Rightarrow G$ between colimits-preserving functors $F,G\colon \PSh(\Cc)\rightarrow\Dd$ is an equivalence already if it is an equivalence when restricted to representable presheaves. But this is clear since every presheaf can be written as a colimit of representables (\cref{lem:PresheafColimitOfRepresentables}).
\end{proof}
\subsection{Homology, cohomology, Eilenberg--MacLane animae}\label{subsec:EilenbergMacLane}
\cref{thm:PShFreeCocompletion} is surprisingly powerful even in the special case $\Cc\simeq *$. In this case we have $\PSh(*)\simeq \cat{An}$ and so  \cref{thm:PShFreeCocompletion} says that a colimits-preserving functor $\cat{An}\rightarrow\Dd$ is uniquely determined by what it does on $*\in\cat{An}$.\footnote{If you think about this fact for a bit, it becomes very natural: \cref{thm:SimplicialApproximation} says that animae are essentially CW complexes and every CW complex is glued together from topological disks $D^n$. But $D^n\simeq *$. So it makes sense that $*$ should generate all of $\cat{An}$ under colimits. Another way to see this is via \cref{lem:ColimitsInAnima}: It's immediately clear that $X\simeq \colimit(\const*\colon X\rightarrow\cat{An})$ for all $X\in\cat{An}$.} Using this observation, our goal in this subsection is to give a purely abstract proof of the Eilenberg--MacLane theorem (\cref{thm:EilenbergMacLane}).

The first step is to construct an interesting $\infty$-category $\Dd$ with all colimits: For a ring $R$ (not necessarily commutative), we'll give a brief introduction to the \emph{derived $\infty$-category} $\Dd(R)$ and its variant $\Dd_{\geqslant 0}(R)$.

\begin{numpar}[Crash course in derived $\infty$-categories I: Basic definitions.]\label{con:DerivedCategoryI}%Of course, \enquote{very brief} means, unfortunately, that we won't give any proofs, only references.
	Let $\Ch(R)$ be the category of chain complexes
	\begin{equation*}
		M_*=\left(\dotsb \overset{\partial}{\longrightarrow} M_{n+1}\overset{\partial}{\longrightarrow} M_n\overset{\partial}{\longrightarrow} M_{n-1}\overset{\partial}{\longrightarrow}\dotsb\right)
	\end{equation*}
	of left $R$-modules and let $\Ch_{\geqslant 0}(R)\subseteq \Ch(R)$ be the full subcategory of those chain complexes that satisfy $M_n\cong 0$ for $n<0$. We usually write $\mathrm Z_n(M_*)\coloneqq \ker(\partial\colon M_n\rightarrow M_{n-1})$ and $\mathrm B_n(M_*)\coloneqq \im(\partial\colon M_{n+1}\rightarrow M_n)$. The quotient $\H_n(M_*)\coloneqq \mathrm Z_n(M_*)/\mathrm B_n(M_*)$ is called the \emph{$n$\textsuperscript{th} homology of $M_*$}. A morphism $\alpha\colon M_*\rightarrow N_*$ in $\Ch(R)$ is called a \emph{quasi-isomorphism} if $\H_n(\alpha)\colon \H_n(M_*)\overset{\cong}{\longrightarrow} \H_n(N_*)$ is an isomorphism for all $n$. Then we put
	\begin{align*}
		\Dd(R)&\coloneqq \Ch(R)\left[\{\text{quasi-isomorphisms}\}^{-1}\right]\\
		\Dd_{\geqslant 0}(R)&\coloneqq \Ch_{\geqslant 0}(R)\left[\{\text{quasi-isomorphisms}\}^{-1}\right]\,,
	\end{align*}
	where the localisations are taken in the $\infty$-categorical sense (see \cref{con:Localisation}). If you've seen the ordinary derived categories $D(R)$ and $D_{\geqslant 0}(R)$ before, then Corollary/Warning~\cref{cor:Localisation} will convince you that these are simply the homotopy categories $\operatorname{ho}\Dd(R)$ and $\operatorname{ho}\Dd_{\geqslant 0}(R)$.
	
	This definition of $\Dd(R)$ is easy to state, but just from that it's nearly impossible to say anything about colimits in $\Dd(R)$, which is why, we will describe a more explicit construction of $\Dd(R)$ in crash course~\cref{con:DerivedCategoryIII} below. However, already with the abstract definition one can get quite far. For example, let's show that $\Dd_{\geqslant 0}(R)$ is indeed a full sub-$\infty$-category of $\Dd(R)$. To this end, observe that the inclusion $\Ch_{\geqslant 0}(R)\subseteq \Ch(R)$ has a right adjoint $\tau_{\geqslant 0}\colon \Ch(R)\rightarrow \Ch_{\geqslant 0}(R)$ given by \emph{smart truncation}: For a chain complex $M_*$ and an integer $i\in \IZ$, we let $\tau_{\geqslant i}M_*$ be the chain complex given by
	\begin{equation*}
		(\tau_{\geqslant i}M_*)_n\coloneqq \ScaledBracesCases{\!\begin{plaincases*}
				M_n & if $n>i$\\
				\mathrm Z_i(M_*) & if $n=i$\\
				0 & if $n<i$
		\end{plaincases*}}\,,
	\end{equation*}
	so that $\H_n(\tau_{\geqslant i}M_*)\cong \H_n(M_*)$ if $n\geqslant i$ and $\H_n(\tau_{\geqslant i}M_*)\cong 0$ for $n<i$. It's clear that $\tau_{\geqslant i}$ preserves quasi-isomorphisms, hence it descends to a functor $\tau_{\geqslant i}\colon \Dd(R)\rightarrow\Dd_{\geqslant 0}(R)$ by \cref{lem:Localisation}. We claim that $\tau_{\geqslant 0}$ is a right adjoint to $\Dd_{\geqslant 0}(R)\rightarrow \Dd(R)$. By \cref{lem:TriangleIdentities}, it's enough to provide a unit and a counit transformation and to verify the triangle identities. But \cref{lem:Localisation} allows us to inherit all this data from the adjunction $i\colon \Ch_{\geqslant 0}(R)\shortdoublelrmorphism \Ch(R)\noloc \tau_{\geqslant 0}$.\footnote{We've seen a similar argument in \cref{rem:ModelCategoryUnderlyingInftyCategory}. The crucial observation to construct natural transformations via \cref{lem:Localisation} is the following: For every $\infty$-category $\Cc$ and every collection of morphisms $W$ in $\Cc$, the functor
	\begin{equation*}
		\bigl(\Cc\times\Delta^1\bigr)\Bigl[\bigl(W\times\{\id_0\}\cup W\times\{\id_1\}\bigr)^{-1}\Bigr]\overset{\simeq}{\longrightarrow}\Cc[W^{-1}]\times\Delta^1
	\end{equation*}
	(which is itself constructed via \cref{lem:Localisation}) is an equivalence of $\infty$-categories. Back in \cref{rem:ModelCategoryUnderlyingInftyCategory}, we appealed to the explicit simplicial construction, but there's also a model-independent way to see this fact. By Yoneda's lemma, \cref{thm:EquivalencePointwise}, and \cref{lem:Localisation}, it's enough to check for every $\infty$-category $\Dd$ that the morphism of animae $\Hom_{\cat{Cat}_\infty}(\Cc[W^{-1}]\times\Delta^1,\Dd)\rightarrow \Hom_{\Cat_\infty}(\Cc\times\Delta^1,\Dd)$ exhibits the left-hand side as the collection of path components of functors that send $W\times\{\id_0\}\cup W\times\{\id_1\}$ to equivalences. By \cref{exm:Adjunctions}\cref{enum:Currying}, we can rewrite the morphism in question as $\Hom_{\Cat_\infty}(\Cc[W^{-1}],\Ar(\Dd))\rightarrow \Hom_{\Cat_\infty}(\Cc,\Ar(\Dd))$ and then \cref{lem:Localisation} shows that, indeed, we get the correct inclusion of path components.} Now to show that $\Dd_{\geqslant 0}(R)\rightarrow \Dd(R)$ is fully faithful, it's enough to check that the unit is an equivalence (see \cref{lem:FullyFaithfulConservativeAdjunction}\cref{enum:FullyFaithfulIffUnitEquivalence}), which is obvious.
	
	Apart from $\tau_{\geqslant i}\colon \Dd(R)\rightarrow \Dd(R)$, there are some more useful functors that can be constructed directly using our definition of $\Dd(R)$ and \cref{lem:Localisation}. For example, if $M_*$ is a chain complex, its \emph{shift by $i$} is the chain complex $M[i]_*$ given by $M[i]_n\cong M_{n-i}$; the differentials are those of $M_*$, but multiplied by $(-1)^i$ (for technical reasons). It's clear that $(-)[i]\colon \Ch(R)\rightarrow \Ch(R)$ preserves quasi-isomorphisms and so it defines a functor $(-)[i]\colon \Dd(R)\rightarrow \Dd(R)$. For an even more obvious example, consider $\H_n\colon \Ch(R)\rightarrow \cat{Mod}_R$ and $\H_n\colon \Ch_{\geqslant 0}(R)\rightarrow \cat{Mod}_R$. These functors send quasi-isomorphisms to isomorphisms (by definition), hence they define essentially unique functors
	\begin{equation*}
		\H_n\colon\Dd(R)\longrightarrow\cat{Mod}_R\quad\text{and}\quad \H_n\colon\Dd_{\geqslant 0}(R)\longrightarrow\cat{Mod}_R\,.
	\end{equation*}
	by \cref{lem:Localisation}. It's probably clear to you, but let us mention that neither $\mathrm{Z}_n\colon \Ch(R)\rightarrow \cat{Mod}_R$ nor $\mathrm{B}_n\colon \Ch(R)\rightarrow \cat{Mod}_R$ preserves quasi-isomorphisms, so they don't extend to $\Dd(R)$, even though their quotient $\H_n\cong \mathrm{Z}_n/\mathrm{B}_n$ does.
\end{numpar}
	For a chain complex $M_*$, we often write $M$ for its image in $\Dd(R)$ to emphasise that this is no longer a \enquote{complex up to isomorphism}, but a \enquote{complex up to quasi-isomorphism}, so that for $M\in\Dd(R)$ there is no longer a well-defined notion of \enquote{$M_n$, the degree-$n$ part of $M$}.\footnote{On a related note, the inclusion $\Ch_{\geqslant 0}(R)\subseteq \Ch(R)$ also has a left adjoint, which simply replaces everything in negative degrees by $0$. This is called the \emph{stupid truncation}. It doesn't preserve quasi-isomorphisms (hence the name) and so we couldn't have used it to show that $\Dd_{\geqslant 0}(R)\rightarrow \Dd(R)$ is fully faithful.}
\begin{numpar}[Crash course in derived $\infty$-categories II: Simplicialities.]\label{con:DerivedCategoryII}
	The famous \emph{Dold--Kan correspondence} (see \cite[Theorem~\HAthm{1.2.3.7}]{HA} or \cite[\S \href{http://dodo.pdmi.ras.ru/~topology/books/goerss-jardine.pdf\#page=169}{III.2}]{GoerssJardine} for example) states that there is an equivalence
	\begin{equation*}
		\Ch_{\geqslant 0}(\IZ)\overset{\simeq}{\longrightarrow}\cat{sAb}
	\end{equation*}
	between the category of chain complexes in non-negative degrees and the category of simplicial abelian groups. We'll need following two facts about the Dold--Kan correspondence:
	\begin{alphanumerate}\itshape
		\item Every simplicial abelian groups is automatically a Kan complex and every degree-wise surjective morphism in $\cat{sAb}$ maps to a Kan fibration in $\cat{sSet}$.\label{enum:SimplicialAbelianGroupKanComplex}
		\item If $A$ is a simplicial abelian group and $M_*$ is the associated chain complex, then there are isomorphisms $\pi_n(A,a)\cong \H_n(M_*)$ for all $a\in A$ and all $n\geqslant 0$.\label{enum:DoldKanHomologyToHomotopy}
	\end{alphanumerate}
	Fact~\cref{enum:SimplicialAbelianGroupKanComplex} not particularly difficult, but not completely obvious either; see \cite[Tags~\href{https://stacks.math.columbia.edu/tag/08NZ}{08NZ} and~\href{https://stacks.math.columbia.edu/tag/08P0}{08P0}]{Stacks}. Let us sketch how to prove \cref{enum:DoldKanHomologyToHomotopy}. First, we may assume $a=0$, since $(-)+a\colon A\rightarrow A$ is an automorphism of $A$ as a simplicial set and induces an isomorphism $\pi_n(A,0)\cong \pi_n(A,a)$. Now $\pi_n(A,0)\cong [(\Delta^n,\partial\Delta^n),(A,0)]$ by \cref{lem:HomotopyGroupsSimplex}, where $[-,-]$ denotes homotopy classes of maps of pairs. Since $(A,0)$ is a group object, even in the homotopy category of pairs, $[(\Delta^n,\partial\Delta^n),(A,0)]$ inherits a group structure. Using the Eckmann--Hilton trick (see the proof of \cref{lem:HomotopyGroups}\cref{enum:EckmannHilton}), we see that this group structure agrees with the one on $\pi_n(A,0)$.
	
	Using \cref{cor:1FunctorCategoryAdjunctions}, the free-forgetful adjunction $\IZ[-]\colon \cat{Set}\shortdoublelrmorphism \cat{Ab}\noloc {\operatorname{forget}}$ induces a similar adjunction $\IZ[-]\colon \cat{sSet}\shortdoublelrmorphism \cat{sAb}\noloc {\operatorname{forget}}$. Then a map of pairs $(\Delta^n,\partial\Delta^n)\rightarrow (A,0)$ is the same as a morphism $\IZ[\Delta^n]/\IZ[\partial\Delta^n]\rightarrow A$ in $\cat{sAb}$. We are, however, not interested in maps, but \emph{homotopy classes} of maps. Our analysis of the group structure on $\pi_n(A,0)$ shows: Instead of quotienting out the equivalence relation generated by homotopies, we may as well quotient out the subgroup generated by the nullhomotopic maps. Using the results from \cref{sec:SimplicialHomotopyTheory}, it's straightforward to show that, for any pointed Kan complex $(X,x)$, a map of pairs $\sigma\colon (\Delta^n,\partial\Delta^n)\rightarrow (X,x)$ is nullhomotopic if and only if it can be extended to a map of pairs $\overline{\sigma}\colon (\Delta^{n+1},\Lambda_{n+1}^{n+1})\rightarrow (X,x)$ in such a way that $d_{n+1}^*(\overline{\sigma})=\sigma$. By the same reasoning as above, such a map is the same as a morphism $\IZ[\Delta^{n+1}]/\IZ[\Lambda_{n+1}^{n+1}]\rightarrow A$. In total, this proves:
	\begin{equation*}
		\pi_n(A,0)\cong \Hom_{\cat{sAb}}\bigl(\IZ[\Delta^n]/\IZ[\partial\Delta^n],A\bigr)/\Hom_{\cat{sAb}}\bigl(\IZ[\Delta^{n+1}]/\IZ[\Lambda_{n+1}^{n+1}],A\bigr)
	\end{equation*}
	A simple unravelling of the Dold--Kan correspondence shows that $\IZ[\Delta^n]/\IZ[\partial\Delta^n]$ is sent to $\IZ[n]_*$, the chain complex consisting of a single $\IZ$ in degree $n$ and zeros everywhere else. Hence $\Hom_{\cat{sAb}}(\IZ[\Delta^n]/\IZ[\partial\Delta^n],A)\cong \Hom_{\Ch_{\geqslant 0}(\IZ)}(\IZ[n]_*,M_*)\cong \mathrm{Z}_n(M_*)$. A similar analysis shows that $\Hom_{\cat{sAb}}(\IZ[\Delta^{n+1}]/\IZ[\Lambda_{n+1}^{n+1}],A)\cong \mathrm{B}_n(M_*)$. Hence $\pi_n(A,0)\cong \H_n(M_*)$, as desired.
\end{numpar}
\begin{numpar}[Crash course in derived $\infty$-categories III: Projective resolutions.]\label{con:DerivedCategoryIII}
	Recall the simplicial nerve from \cref{con:SimplicialNerve}. It's also possible to construct $\Dd(R)$ and $\Dd_{\geqslant 0}(R)$ in this way; this alternative construction will allow us to study colimits. We'll first explain how to equip $\Dd(R)$ and $\Dd_{\geqslant 0}(R)$ with a Kan enrichment: Let $\Hhom_R(M_*,N_*)$ be the chain complex of abelian groups given by
	\begin{equation*}
		\Hhom_R(M_*,N_*)_n\coloneqq\prod_{i\in\IZ}\Hom_R(M_i,N_{i+n}) \,.
	\end{equation*}
	The differentials send a family of morphisms $f=(f_i)_{i\in\IZ}\in\prod_{i\in\IZ}\Hom_R(M_i,N_{i+n})$ to the family $\partial f\coloneqq(\partial_N\circ f_i-(-1)^nf_{i-1}\circ \partial_M)_{i\in\IZ}$; here $\partial_M$ and $\partial_N$ denote the differentials of $M_*$ and $N_*$, respectively. By unravelling the definitions, we see that the $n$-cycles and $n$-boundaries of $\Hhom_R(M_*,N_*)$ are given by
	\begin{align*}
		\mathrm Z_n\bigl(\Hhom_R(M_*,N_*)\bigr)&\cong \Hom_{\Ch(R)}\bigl(M_*,N[-n]_*\bigr)\\
		\mathrm B_n\bigl(\Hhom_R(M_*,N_*)\bigr)&\cong  \left\{f\in\Hom_{\Ch(R)}\bigl(M_*,N[-n]_*\bigr)\ \middle|\ f\text{ nullhomotopic}\right\}\,.
	\end{align*}
	Here $N[-n]_*$ denotes the shift from crash course~\cref{con:DerivedCategoryI}. Since $\mathrm{H}_n\cong \mathrm{Z}_n/\mathrm{B}_n$, we deduce that $\H_n(\Hhom_R(M_*,N_*))$ is in bijection with the set of homotopy classes of maps $M_*\rightarrow N[-n]_*$.
	
	The complexes $\Hhom_R(-,-)$ provide an enrichment of $\Ch(R)$ over $\Ch(\cat{Ab})$ (in fact, even an enrichment of $\Ch(R)$ over itself). To make this into a Kan enrichment, we let $\tau_{\geqslant 0}\Hhom_R(M_*,N_*)$ be the smart truncation from crash course~\cref{con:DerivedCategoryI} and let $\F_{\Ch(R)}(M_*,N_*)$ denote the simplicial abelian group corresponding to $\tau_{\geqslant0}\Hhom_R(M_*,N_*)$ under the Dold--Kan correspondence. The simplicial abelian groups $\F_{\Ch(R)}(-,-)$ provide an enrichment of $\Ch(R)$ in simplicial sets, which is automatically a Kan enrichment by crash course~\cref{con:DerivedCategoryII}\cref{enum:SimplicialAbelianGroupKanComplex}.
	
	A complex $P_*$ of $R$-modules is called \emph{$K$-projective} if $\Hhom_R(P_*,-)\colon \Ch(R)\rightarrow \Ch(R)$ preserves quasi-isomorphisms. It was shown by Spaltenstein \cite{KProjective} that every chain complex of $R$-modules $M_*$ admits a quasi-isomorphism $P_*\rightarrow M_*$ from a $K$-projective complex. If $P_*$ is $K$-projective, then it is \emph{degree-wise projective} in the sense that every $P_n$ is a projective $R$-module. Conversely, if $P_*$ is degree-wise projective and bounded below in the sense that $P_n\cong0$ for $n\lle 0$, then $P_*$ is $K$-projective. These statements can be found in \cite[Lemma~\href{https://people.math.rochester.edu/faculty/doug/otherpapers/hovey-model-cats.pdf\#page=52}{2.3.6}]{HoveyModelCategories}; the second statement also appears (in dual form) in \cite[\stackstag{070J}]{Stacks}.
	
	Let $K\mhyph\cat{Proj}(R)\subseteq\Ch(R)$ and $\cat{Proj}_{\geqslant 0}(R)\subseteq \Ch_{\geqslant 0}(R)$ be the full subcategories spanned by the $K$-projective complexes. Equip $K\mhyph\cat{Proj}(R)$ and $\cat{Proj}_{\geqslant 0}(R)$ with the Kan enrichment above. Then
	\begin{equation*}
		\Dd(R)\simeq \N^\Delta\bigl(K\mhyph\cat{Proj}(R)\bigr)\quad\text{and}\quad \Dd_{\geqslant 0}(R)\simeq \N^\Delta\bigl(\cat{Proj}_{\geqslant 0}(R)\bigr)\,.
	\end{equation*}
	The idea to prove this is, of course, similar to \cref{thm:AnAsALocalisation}: One can construct a simplicial model structure on $\Ch(R)$ (and, by restriction, on $\Ch_{\geqslant 0}(R)$) in such a way that $K\mhyph\cat{Proj}(R)\simeq \Ch(R)^\mathrm{cf}$ are precisely the bifibrant objects, see \cite[\S\href{https://people.math.rochester.edu/faculty/doug/otherpapers/hovey-model-cats.pdf\#page=50}{2.3}]{HoveyModelCategories}. Then the above equivalences follow from \cref{rem:SimplicialModelCategory,rem:ModelCategoryUnderlyingInftyCategory}.\footnote{It's worth pointing out that the process of choosing a cofibrant replacement in $\Ch_{\geqslant 0}(R)$ precisely recovers the method of \emph{projective resolutions} that you may be familiar with from homological algebra. Indeed, the cofibrant objects in $\Ch_{\geqslant 0}(R)$ are precisely the degree-wise projective complexes in non-negative degrees. Now if $M$ is a left $R$-module and we think of $M$ as a complex $M[0]_*$ concentrated in degree~$0$ (see \cref{con:Homology}, then a cofibrant replacement of $M[0]_*$, that is, a quasi-isomorphism $P_*\rightarrow M[0]_*$ from a degree-wise projective complex, is precisely a projective resolution of $M$. This begs the question how \emph{injective resolutions} fit into the picture. There is another simplicial model structure on $\Ch(R)$ in which the bifibrant objects are the \emph{$K$-injective} complexes, that is, those $I_*$ for which $\Hhom_R(-,I_*)$ preserves quasi-isomorphisms. A $K$-injective complex is degree-wise an injective $R$-module and conversely any degree-wise injective and bounded above complex is $K$-injective. One then has similar equivalences $\Dd(R)\simeq \N^\Delta(K\mhyph\cat{Inj}(R))$ and $\Dd_{\leqslant 0}(R)\simeq \N^\Delta(\cat{Inj}_{\leqslant 0}(R))$.}
	
	This alternative construction is useful to compute $\Hom_{\Dd(R)}$. Let $M_*$ and $N_*$ be complexes and let $P_*\rightarrow M_*$ and $Q_*\rightarrow N_*$ be quasi-isomorphisms from $K$-projective complexes. Using \cref{thm:CordierPorter}, we get $\Hom_{\Dd(R)}(M,N)\simeq \F_{\Ch(R)}(P_*,Q_*)$. In particular, since the Dold--Kan correspondence transforms homotopy groups of simplicial abelian groups into homology groups of the associated chain complexes by crash course~\cref{con:DerivedCategoryII}\cref{enum:DoldKanHomologyToHomotopy}, we find
	\begin{equation*}
		\pi_n\Hom_{\Dd(R)}(M,N)\cong \pi_n\F_{\Ch(R)}(P_*,Q_*)\cong \H_n\bigl(\Hhom_R(P_*,Q_*)\bigr)
	\end{equation*}
	for all $n\geqslant 0$ and all basepoints. Furthermore, $\H_n(\Hhom_R(P_*,Q_*))\cong \H_n(\Hhom_R(P_*,N_*))$ by definition of $P_*$ being $K$-projective, so we only need to resolve $M_*$ by a $K$-projective complex. If you've seen derived categories before, you'll probably have noticed that $\Hom_{\Dd(R)}(M,N)$ looks suspiciously like the derived Hom functor $\RHom_R(M,N)$: We'll see in \cref{cor:RHom} how exactly these two are related.\hfill$\blacksquare$
\end{numpar}

To be able to apply \cref{thm:PShFreeCocompletion} to $\Dd(R)$ or $\Dd_{\geqslant 0}(R)$, we need to show that these $\infty$-categories have all colimits. In order to to this, we'll show some general results about colimits in $\infty$-categories; these results will also be very useful later on.

\begin{lem}\label{lem:ColimitsIffCoproductsAndPushouts}
	An $\infty$-category $\Cc$ has all colimits if and only if $\Cc$ has pushouts and arbitrary coproducts. A functor $F\colon \Cc\rightarrow\Dd$ of $\infty$-categories preserves colimits if and only if it preserves pushouts and arbitrary coproducts. A dual assertion holds for limits.
\end{lem}
The crucial point, and the reason why we get away with \enquote{ordinary} colimits like pushouts and coproducts, is that $\{n\}\rightarrow \Delta^n$ is coinitial (in fact, right anodyne, so \cref{exm:Cofinal}\cref{enum:RightAnodyneCofinal} applies). Hence every functor $T\colon \Delta^n\rightarrow \Cc$ admits a colimit. For a general functor $T\colon\Ii\rightarrow \Cc$, we write $\Ii$ as a colimit of its skeleta to build $\colimit_{i\in\Ii}T(i)$ \enquote{simplex-by-simplex}: This needs pushouts (to attach $n$-simplices in the $n$\textsuperscript{th} step) and coproducts (to attach arbitrarily many $n$-simplices at the same time). 

To make this precise, we'll prove a lemma that will allow us to manipulate colimits: We can \enquote{slice a colimit into pieces} and \enquote{assemble colimits from subdiagrams}:
\begin{lem}\label{lem:ColimitManipulations}
	Let $\Ii$ and $\Cc$ be $\infty$-categories.
	\begin{alphanumerate}
		\item Suppose $p\colon \Uu\rightarrow \Ii$ is a cocartesian fibration and $T\colon \Uu\rightarrow \Cc$ is a functor such that $T|_{p^{-1}\{i\}}\colon p^{-1}\{i\}\rightarrow \Cc$ admits a colimit for all $i\in \Ii$. Then these colimits assemble into a functor $\ov T\colon \Ii\rightarrow \Cc$ satisfying $\ov T(i)\simeq \colimit_{u\in p^{-1}\{i\}}T(u)$. Furthermore,\label{claim:SliceColimits}
		\begin{equation*}
			\colimit_{u\in\Uu}T(u)\simeq \colimit_{i\in\Ii}\ov T(i)\,,
		\end{equation*}
		provided that at least one of these colimits exists in $\Cc$ \embrace{in which case the other exists as well}. Informally, we can rephrase this as $\colimit_{u\in\Uu}T(u)\simeq \colimit_{i\in\Ii}\colimit_{u\in p^{-1}\{i\}}T(u)$.
		\item Suppose $\Ii\simeq \colimit_{j\in\Jj}\Ii_j$ in $\Cat_\infty$. Let $T\colon \Ii\rightarrow \Cc$ be a functor such that the restrictions $T|_{\Ii_j}\colon \Ii_j\rightarrow \Cc$ admit colimits for all $j\in\Jj$. Then these colimits assemble into a functor $\ov T\colon \Jj\rightarrow \Cc$ satisfying $\ov T(j)\simeq \colimit_{i\in\Ii_j}T(i)$. Furthermore,
		\label{claim:AssembleColimits}
		\begin{equation*}
			\colimit_{i\in\Ii}T(i)\simeq \colimit_{j\in\Jj}\ov T(j)\,.
		\end{equation*}
		provided that at least one of these colimits exists in $\Cc$ \embrace{in which case the other exists as well}. Informally, we can rephrase this as $\colimit_{i\in\Ii}T(i)\simeq \colimit_{j\in\Jj}\colimit_{i\in\Ii_j}T(i)$.
	\end{alphanumerate}
	In particular, \enquote{colimits commute with colimits}: If $\Jj$ is an $\infty$-category and $T\colon \Ii\times\Jj\rightarrow \Cc$ is any functor, then
	\begin{equation*}
		\colimit_{i\in\Ii}\colimit_{j\in\Jj}T(i,j)\simeq\colimit_{(i,j)\in\Ii\times\Jj}T(i,j)\simeq \colimit_{j\in\Jj}\colimit_{i\in\Ii}T(i,j)\,.
	\end{equation*}
	%Dual assertions hold for limits \embrace{meaning that in \cref{claim:SliceColimits}, we consider a cartesian fibration, but in \cref{claim:AssembleColimits}, we still consider $\Ii\simeq\colimit_{j\in\Jj}\Ii_j$}.
\end{lem}
\begin{proof}
	To prove \cref{claim:SliceColimits}, first note that $\Lan_pT$ exists. Indeed, $p^{-1}\{i\}\rightarrow \Uu_{/i}$ is coinitial by the dual of \cref{lem:CartesianFibres} and \cref{exm:Cofinal}\cref{enum:RightAdjointCofinal}, so the existence of the colimits over $p^{-1}\{i\}$ implies that the condition from \cref{lem:KanExtensionFormula} is satisfied. So we can put $\ov T\coloneqq \Lan_pT$. Now $\colimit T$ corresponds to taking the left Kan extension of $T$ along $\Uu\rightarrow *$ (see \cref{exm:1ColimitAsKanExtension}). But we may as well first left Kan extend along $\Uu\rightarrow \Ii$ and then left Kan extend along $\Ii\rightarrow *$. This proves $\colimit T\simeq \colimit \Lan_pT$ and we've finished the proof of \cref{claim:SliceColimits}. In the special case where $p$ is the projection $\pr_1\colon \Ii\times\Jj\rightarrow \Jj$, we obtain the \enquote{in particular}.
	
	For \cref{claim:AssembleColimits}, let $p\colon \Uu\rightarrow \Jj$ be the unstraightening of the functor $\Jj\rightarrow \cat{Cat}_\infty$ of which $\Ii$ is the colimit. Then $\Ii$ is a localisation of $\Uu$ by \cref{lem:ColimitsInAnima}, so there's a natural functor $q\colon \Uu\rightarrow\Ii$. We have $p^{-1}\{j\}\simeq \Ii_j$, so we can apply \cref{claim:SliceColimits} to the functor $q\circ T\colon \Uu\rightarrow \Cc$. This allows us to construct $\ov T$ and we obtain $\colimit \ov T\simeq \colimit q\circ T$. But $q$, being a localisation, is coinitial by \cref{exm:Cofinal}\cref{enum:LocalisationsCofinal}, and so $\colimit q\circ T\simeq \colimit T$. This proves \cref{claim:AssembleColimits}.
\end{proof}
%\begin{rem}\label{rem:ColimitsCommute}
%	An important special case of \cref{lem:ColimitManipulations}\cref{claim:SliceColimits} is the case where $p$ is the projection $\pr_1\colon \Ii\times\Jj\rightarrow \Jj$: We obtain
%	\begin{equation*}
	%		\colimit_{i\in\Ii}\colimit_{j\in\Jj}T(i,j)\simeq\colimit_{(i,j)\in\Ii\times\Jj}T(i,j)\simeq \colimit_{j\in\Jj}\colimit_{i\in\Ii}T(i,j)
	%	\end{equation*}
%	for every functor $T(i,j)\rightarrow\Cc$. A dual assertion holds for limits. Informally, \enquote{colimits commute with colimits} and \enquote{limits commute with limits}.
%\end{rem}
\begin{proof}[Proof sketch of \cref{lem:ColimitsIffCoproductsAndPushouts}]
	In simplicial sets, we can write $\Ii\cong \colimit_{n\geqslant 0}\operatorname{sk}_n\Ii$, where $\operatorname{sk}_{n}\Ii$ is obtained from $\operatorname{sk}_{n-1}\Ii$ by attaching copies of $\Delta^n$; that is, we take a pushout along some coproduct of the form $\coprod \partial\Delta^n\rightarrow \coprod\Delta^n$. Up to replacing everything by quasi-categories (using \cref{lem:SmallObjectArgument}), we can thus write $\Ii\simeq \colimit_{n\geqslant 0}\Ii_n$ in $\cat{Cat}_\infty$ in such a way that $\Ii_n$ is obtained from $\Ii_{n-1}$ by a pushout along $\coprod \Bb^n\rightarrow \coprod \Delta^n$, where $\Bb^n$ is defined by choosing an inner anodyne map $\partial\Delta^n\rightarrow \Bb^n$ into a quasi-category. By an inductive argument (in which \cref{lem:ColimitManipulations}\cref{claim:AssembleColimits} powers the inductive step), we find that $\colimit_{i\in\Ii_n}T(i)$ exists in $\Cc$ for all $n\geqslant 0$. Using \cref{lem:ColimitManipulations}\cref{claim:AssembleColimits} once again, it remains to show that $\colimit_{n\geqslant 0}\colimit_{i\in\Ii_n}T(i)$ exists in $\Cc$. But this colimit can be easily written as a suitable pushout of the disjoint union $\coprod_{n\geqslant 0}\colimit_{i\in\Ii_n}T(i)$.
\end{proof}

To study colimits in derived $\infty$-categories, we introduce the following convenient terminology.
\begin{defi}\label{def:Cofibre}
	Let $\Cc$ be an $\infty$-category with a terminal object $*$ and let $\alpha\colon x\rightarrow y$ be a morphism in $\Cc$. The \emph{cofibre of $\alpha$} is defined as the pushout
	\begin{equation*}
		\begin{tikzcd}
			x\rar["\alpha"]\dar\drar[pushout] & y\dar\\
			*\rar & \cofib(\alpha)
		\end{tikzcd}
	\end{equation*}
	(provided this exists in $\Cc$). We say that $x\overset{\alpha}{\longrightarrow}y\rightarrow z$ is a \emph{cofibre sequence in $\Cc$} if the induced morphism $x\rightarrow z$ can be factored through $*$ in such a way that it exhibits $z$ as the cofibre of $\alpha$. There are dual notions of the \emph{fibre of $\alpha$} $\fib(\alpha)$  (given as the pullback against an initial object) and \emph{fibre sequences}.
\end{defi}
\begin{lem}\label{lem:ColimitsInDR}
	Let $R$ be any ring \embrace{not necessarily commutative}. The $\infty$-category $\Dd(R)$ has all colimits. The full sub-$\infty$-caegory $\Dd_{\geqslant 0}(R)\subseteq \Dd(R)$ is closed under colimits in $\Dd(R)$ and therefore has all colimits too. Coproducts and pushouts in $\Dd(R)$ can be described as follows:
	\begin{alphanumerate}
		\item If $I$ is any set and $M_{i,*}$ are chain complexes of left $R$-modules, then the chain complex $\bigoplus_{i\in I}M_{i,*}$ defines a coproduct of the objects $M_i\in \Dd(R)$.\label{enum:CoproductsInDR}
		\item Let $\alpha\colon M_*\rightarrow N_*$ be a morphism of chain complexes of $R$-modules for some ring $R$. Then the cofibre of $\alpha$ in $\Dd(R)$ can be computed as the mapping cone\footnote{The mapping cone $\cone(\alpha)_*$ is the chain complex given by $\cone(\alpha)_n\cong N_n\oplus M_{n-1}$, with differentials given by the matrix
		\begin{equation*}
			\begin{pmatrix}
				\partial_N & \alpha\\
				0 & -\partial_M
			\end{pmatrix}\colon N_n\oplus M_{n-1}\rightarrow N_{n-1}\oplus M_{n-2}\,;
		\end{equation*}
		here $\partial_M$ and $\partial_N$ denote the differentials of $M_*$ and $N_*$, respectively. The mapping cone comes equipped with obvious maps $N_*\rightarrow \cone(\alpha)_*$ and $\cone(\alpha)_*\rightarrow M[1]_*$. The induced maps $\H_n(N_*)\rightarrow \H_n(\cone(\alpha)_*)$ and $\H_n(\cone(\alpha)_*)\rightarrow \H_{n-1}(M_*)$ on homology fit into a long exact sequence
		\begin{equation*}
			\dotsb\longrightarrow\H_{n+1}\bigl(\cone(\alpha)_*\bigr)\overset{\partial}{\longrightarrow} \H_n(M_*)\longrightarrow \H_n(N_*)\longrightarrow \H_n\bigl(\cone(\alpha)_*\bigr)\overset{\partial}{\longrightarrow}\H_{n-1}(M_*)\longrightarrow \dotsb
		\end{equation*}
		called the \emph{cone sequence}. In fact, this is nothing but the long exact homology sequence associated to the short exact sequence of complexes $0\rightarrow N_*\rightarrow \cone(\alpha)_*\rightarrow M[1]_*\rightarrow 0$.}\label{enum:CofibresInDR}
		\begin{equation*}
			\cofib\left(\alpha\colon M\rightarrow N\right)\simeq \cone\left(\alpha\colon M_*\rightarrow N_*\right)\,.
		\end{equation*}
		More generally, if $\beta\colon M_*\rightarrow M'_*$ is another morphism of complexes, then the pushout of the span $N_*\leftarrow M_*\rightarrow M'_*$ in $\Dd(R)$ is given by $\cone((\alpha,-\beta)\colon M_*\rightarrow M'_*\oplus N_*)$.
	\end{alphanumerate}
\end{lem}
\begin{proof}[Proof sketch]
	By \cref{lem:ColimitsIffCoproductsAndPushouts}, to show that $\Dd(R)$ has all colimits, it's enough to check that coproducts and pushouts in $\Dd(R)$ can be described as in~\cref{enum:CoproductsInDR} and~\cref{enum:CofibresInDR}. Furthermore, it's clear from these descriptions that $\Dd_{\geqslant 0}(R)\subseteq \Dd(R)$ is closed under formation of coproducts and pushouts, hence under all colimits.
	
	To prove \cref{enum:CoproductsInDR}, choose quasi-isomorphisms $P_{i,*}\rightarrow M_{i,*}$ from $K$-projective complexes. Since quasi-isomorphisms are preserved under direct sums, it's enough to show that $\bigoplus_{i\in I}P_i$ is a coproduct of the $P_i$. Using \cref{cor:HomPreservesColimits}, we must show that 
	\begin{equation*}
		\Hom_{\Dd(R)}\biggl(\bigoplus_{i\in I}P_i,T\biggr)\overset{\simeq}{\longrightarrow}\prod_{i\in I}\Hom_{\Dd(R)}(P_i,T)
	\end{equation*}
	is an equivalence of animae for all $K\in\Dd(R)$. But if $T_*$ is any chain complex representing $T$, then $\Hhom_R\bigl(\bigoplus_{i\in I}M_{P,*},T_*\bigr)\cong \prod_{i\in I}\Hhom_R(P_{i,*},T_*)$ holds in $\Ch(R)$. Both the functor $\tau_{\geqslant 0}\colon \Ch(R)\rightarrow \Ch_{\geqslant 0}(R)$ and the Dold--Kan equivalence $\Ch_{\geqslant 0}(R)\simeq \cat{sAb}$ preserve products. Hence we get an isomorphism of Kan complexes (in fact, of simplicial abelian groups) $\F_{\Ch(R)}\bigl(\bigoplus_{i\in I} P_{i,*},T_*\bigr)\cong \prod_{i\in I}\F_{\Ch(R)}(P_{i,*},T_*)$. By crash course~\cref{con:DerivedCategoryIII}, this is (stronger than) what we need.
	
	The proof of \cref{enum:CofibresInDR} is similar, but needs a little more care. First, the assertion about pushouts is a formal consequence of the assertion about cofibres; we leave this to you (just verify the universal property). To show the assertion about cofibres, choose quasi-isomorphisms $P_*\rightarrow M_*$ and $Q_*\rightarrow N_*$ from $K$-projective complexes and replace $\alpha$ by a morphism $\alpha'\colon P_*\rightarrow Q_*$. Since the mapping cone construction preserves quasi-isomorphisms (by the long exact cone sequence and the five lemma), it suffices to show $\cofib(\alpha')\simeq \cone(\alpha')$. To this end, first note that the composition $P_*\rightarrow Q_*\rightarrow \cone(\alpha')_*$ is nullhomotopic as a map of complexes. Any choice of nullhomotopy defines a morphism $\cofib(\alpha')\rightarrow \cone(\alpha')$ in $\Dd(R)$. To see that this morphism is an equivalence, we can appeal again to \cref{cor:HomPreservesColimits} and the Yoneda lemma: It's enough to show that
	\begin{equation*}
		\Hom_{\Dd(R)}\bigl(\cone(\alpha'),T\bigr)\longrightarrow\Hom_{\Dd(R)}(Q,T)\longrightarrow\Hom_{\Dd(R)}(P,T)
	\end{equation*}
	is a fibre sequence of animae for all $T\in\Dd(R)$. Now we claim:
	\begin{alphanumerate}\itshape
		\item[\boxtimes] Let $\varphi\colon K_*\rightarrow L_*$ be any morphism of chain complexes and consider the canonical sequence\label{enum:DoldKanFibreSequence}
		\begin{equation*}
			\cone(\varphi)[-1]_*\longrightarrow K_*\overset{\varphi}{\longrightarrow} L_*
		\end{equation*}
		in $\Ch(R)$. Upon applying $\tau_{\geqslant 0}\colon \Ch(R)\rightarrow \Ch_{\geqslant 0}(R)$ and the Dold--Kan correspondence, this sequence is sent to a fibre sequence of animae.
	\end{alphanumerate}
	Once we know \cref{enum:DoldKanFibreSequence}, we're done. Indeed, it's straightforward to check that the sequence $\Hhom_{\Dd(R)}(\cone(\alpha')_*,T_*)\rightarrow\Hhom_{R}(Q_*,T_*)\rightarrow\Hhom_{R}(P_*,T_*)$ is of the desired form. Then \cref{enum:DoldKanFibreSequence} ensures that $\F_{\Ch(R)}(\cone(\alpha')_*,T_*)\rightarrow\F_{\Ch(R)}(Q_*,T_*)\rightarrow\F_{\Ch(R)}(P_*,T_*)$ is a fibre sequence; by crash course~\cref{con:DerivedCategoryIII}, this is what we need.
	
	To prove \cref{enum:DoldKanFibreSequence}, we can restrict ourselves to the case where $\varphi\colon K_*\rightarrow L_*$ is degree-wise surjective. Indeed, we can always replace $K_*$ by $K_*\oplus \cone(\id_{L_*}\colon L_*\rightarrow L_*)[-1]_*$; this doesn't change anything since $\cone(\id_{L_*})_*\rightarrow 0$ is a quasi-isomorphism, $\tau_{\geqslant 0}$ preserves quasi-isomorphisms, and Dold--Kan sends quasi-isomorphisms to homotopy equivalences of Kan complexes (because it sends $\H_n$ to $\pi_n$ by crash course~\cref{con:DerivedCategoryII}\cref{enum:DoldKanHomologyToHomotopy}). If $\varphi\colon K_*\rightarrow L_*$ is surjective, a well-known fact from homological algebra states that there is a quasi-isomorphism $\ker(\varphi)_*\simeq \cone(\varphi)[-1]_*$.\footnote{To see this, the first step is to construct a morphism $\ker(\varphi)_*\rightarrow \cone(\varphi)[-1]_*$: This is straightforward from the construction of $\cone(\varphi)[-1]_*$. Alternatively, we can invoke a universal property: It can be shown that maps $T_*\rightarrow \cone(\varphi)[-1]_*$ are in bijection with pairs $(\alpha,\eta)$, where $\alpha\colon T_*\rightarrow K_*$ is a morphism and $\eta$ is a nullhomotopy of the composition $\varphi\circ\alpha\colon T_*\rightarrow L_*$. Since $\ker(\varphi)_*\rightarrow L_*$ is zero on the nose, we can just choose the trivial nullhomotopy. To show that the map $\ker(\varphi)_*\rightarrow \cone(\varphi)[-1]_*$ is a quasi-isomorphism, it's straightforward to check that this map induces a morphism between the long exact homology sequence associated to $0\rightarrow \ker(\varphi)_*\rightarrow K_*\rightarrow L_*\rightarrow 0$ and the cone sequence for $\cone(\varphi)_*$. Then the five lemma does the rest.} So it's enough to show that $\ker(\varphi)_*\rightarrow K_*\rightarrow L_*$ is sent to a fibre sequence.
	
	Since $\tau_{\geqslant 0}\colon \Ch(R)\rightarrow\Ch_{\geqslant 0}(R)$ is a right adjoint, we see that $\tau_{\geqslant 0}\ker(\varphi)_*$ is still the kernel of $\tau_{\geqslant 0}\varphi\colon \tau_{\geqslant 0}K_*\rightarrow \tau_{\geqslant 0}L_*$, but that map might not be surjective anymore in degree~$0$. So consider $\im(\tau_{\geqslant 0}\varphi)_*\rightarrow \tau_{\geqslant 0}L_*$.  Under the Dold--Kan correspondence, this map is sent to the inclusion of a collection of path components. Indeed, $\im(\tau_{\geqslant 0}\varphi)_*\rightarrow \tau_{\geqslant 0}L_*$ is an isomorphism on $\H_n$ for all $n\geqslant 1$ and injective on $\H_0$, so after Dold--Kan, we obtain an isomorphism on $\pi_n$ for all $n\geqslant 1$ and an injection on $\pi_0$. Therefore, it's enough to show that $\tau_{\geqslant 0}\ker(\varphi)_*\rightarrow \tau_{\geqslant 0}K_*\rightarrow \im(\tau_{\geqslant 0}\varphi)_*$ is sent to a fibre sequence. But that's a short exact sequence in $\Ch_{\geqslant 0}(R)$, and so it's sent to a short exact sequence in $\cat{sAb}$. As mentioned in crash course~\cref{con:DerivedCategoryII}\cref{enum:SimplicialAbelianGroupKanComplex}, surjections of simplicial abelian groups are Kan fibrations. By model category fact~\cref{par:HomotopyPushout}, this means that the kernel, that is, the fibre over $0$ taken in simplicial sets, agrees with the homotopy fibre. So we're done.
\end{proof}
With all the preparatory stuff about $\Dd(R)$ out of the way, we can now finally get to the actual subject of \cref{subsec:EilenbergMacLane}: Homology, cohomology, and Eilenberg--MacLane animae.
\begin{con}\label{con:Homology}
	For all integers $n$ and all abelian groups $A$ define a chain complex
	\begin{equation*}
		A[n]_*\coloneqq \left(\dotsb\rightarrow 0\rightarrow 0\rightarrow A\rightarrow 0\rightarrow 0\rightarrow \dotsb\right)
	\end{equation*}
	with $A$ in degree $n$ and $0$ everywhere else. Consider the functor $\cat{Ab}\rightarrow\Ch_{\geqslant 0}(\IZ)\rightarrow\Dd_{\geqslant 0}(\IZ)$ that sends $A$ to $A[0]$. Since $\Dd_{\geqslant 0}(\IZ)$ has all colimits, so has $\Fun(\cat{Ab},\Dd_{\geqslant 0}(\IZ))$ by \cref{lem:ColimitsInFunctorCategories}. Hence, by \cref{thm:PShFreeCocompletion}, there exists a unique colimits-preserving functor $\cat{An}\rightarrow\Fun(\cat{Ab},\Dd_{\geqslant 0}(\IZ))$ that sends $*\in\cat{An}$ to the functor $\cat{Ab}\rightarrow\Ch_{\geqslant 0}(\IZ)\rightarrow\Dd_{\geqslant 0}(\IZ)$ discussed above. By \enquote{currying}, we obtain a functor
	\begin{equation*}
		\C(-,-)\colon \cat{An}\times\cat{Ab}\longrightarrow \Dd_{\geqslant 0}(\IZ)\,.
	\end{equation*} 
	For every abelian group $A$, $\C(-,A)\colon \cat{An}\rightarrow\Dd_{\geqslant 0}(\IZ)$ is the unique colimits-preserving functor that sends $*\in\cat{An}$ to $A[0]$ as above. For an anima $X$, we call $\C(X,A)$ the \emph{chains of $X$ with coefficients in $A$} and we call $\widetilde{\C}(X,A)\coloneqq \fib(\C(X,A)\rightarrow\C(*,A))$ the \emph{reduced chains of $X$ with coefficients in $A$} (using the fibre construction from \cref{def:Cofibre}). For all $n\geqslant 0$, we let
	\begin{equation*}
		\H_n(X,A)\coloneqq \H_n\bigl(\C(X,A)\bigr)\,,\quad \widetilde{\H}_n(X,A)\coloneqq\H_n\bigl(\widetilde{\C}(X,A)\bigr)
	\end{equation*}
	denote the \emph{$n$\textsuperscript{th} homology of $X$ with coefficients in $A$} and the \emph{$n$\textsuperscript{th} reduced homology of $X$ with coefficients in $A$}; here $\H_n\colon \Dd_{\geqslant 0}(\IZ)\rightarrow\cat{Ab}$ is the functor from crash course~\cref{con:DerivedCategoryI}. Finally,
	\begin{equation*}
		\H^n(X,A)\coloneqq \pi_0\Hom_{\Dd_{\geqslant 0}(\IZ)}\bigl(\C(X,\IZ),A[n]\bigr)\,,\quad\widetilde{\H}^n(X,A)\coloneqq \pi_0\Hom_{\Dd_{\geqslant 0}(\IZ)}\bigl(\widetilde{\C}(X,\IZ),A[n]\bigr)
	\end{equation*}
	denote the \emph{$n$\textsuperscript{th} cohomology of $X$ with coefficients in $A$} and the \emph{$n$\textsuperscript{th} reduced cohomology of $X$ with coefficients in $A$}. We'll verify in \cref{lem:Homology} below that this definition of homology and cohomology is compatible with the one you are familiar with.
	
	But before we do that, let's give yet another unfamiliar formulation of a familiar definition! By \cref{thm:PShFreeCocompletion}, $\C(-,\IZ)\colon \cat{An}\rightarrow \Dd_{\geqslant 0}(\IZ)$ automatically acquires a right adjoint, which we denote $\K\colon \Dd_{\geqslant 0}(\IZ)\rightarrow\cat{An}$. For $M\in\Dd_{\geqslant 0}(\IZ)$ we call $\K(M)$ the \emph{generalised Eilenberg--MacLane anima of $M$}. In the special case $M\simeq A[n]$ we say that $\K(A,n)\coloneqq \K(A[n])$ is the \emph{Eilenberg--MacLane anima of type $(A,n)$}. Again, we'll justify in \cref{thm:EilenbergMacLane} below that this recovers the definition you're familiar with.
\end{con}
\begin{lem}\label{lem:Homology}
	Let $A$ be an arbitrary abelian group.
	\begin{alphanumerate}
		\item The functors $\H_*(-,A)$ and $\widetilde{\H}_*(-,A)$ from \cref{con:Homology} satisfy the Eilenberg--Steenrod axioms. In particular, they are homotopy invariants; if $f\colon Y\rightarrow X$ is a morphism of animae with cofibre $X/Y\coloneqq\cofib(f)$, then there is a long exact sequence\label{enum:HomologyEilenbergSteenrod}
		\begin{equation*}
			\dotsb\longrightarrow\H_n(Y,A)\longrightarrow\H_n(X,A)\longrightarrow\widetilde{\H}_n(X/Y,A)\overset{\partial}{\longrightarrow}\H_{n-1}(Y,A)\longrightarrow\dotsb
		\end{equation*}
		\embrace{so $\widetilde{\H}_*(X/Y,A)$ plays the role of the relative homology $\H_*(X,Y,A)$}; and $\H_*(-,A)$ sends disjoint unions to direct sums. Similar assertions hold for $\widetilde{\H}_*(-,A)$. Furthermore, the suspension isomorphism is satisfied and pushouts of animae yield long exact Mayer--Vietoris sequences.
		\item Let $X$ be a Kan complex with geometric realisation $\abs*{X}\in\cat{Top}$. Let $\C_*^\mathrm{sing}(\abs*{X},A)$ denote the singular chain complex of $\abs*{X}$ with coefficients in $A$. Then there is a natural quasi-somorphism
		\begin{equation*}
			\C(X,A)\overset{\simeq}{\longrightarrow} \C^\mathrm{sing}_*\bigl(\abs{X},\IZ\bigr)\,.
		\end{equation*}
		In particular, we get $\H_*(X,A)\cong \H_*^\mathrm{sing}(\abs*{X},A)$, as well as similar isomorphisms for reduced homology and for cohomology \embrace{both unreduced and reduced}.\label{enum:SingularHomology}
	\end{alphanumerate}
\end{lem}
\begin{proof}[Proof sketch]
	We begin with \cref{enum:HomologyEilenbergSteenrod}. Homotopy invariance follows from the definition of $\cat{An}$. The other Eilenberg--Steenrod axioms all follow from the fact that $\C(-,A)\colon \cat{An}\rightarrow\Dd_{\geqslant 0}(\IZ)$ preserves colimits. To demonstrate these kinds of arguments, we'll show the long exact sequence for $\H_*(-,A)$; the disjoint union axiom as well as the Eilenberg--Steenrod axioms for $\widetilde{\H}_*(-,A)$ will be left to you. We start with the following diagram:
	\begin{equation*}
		\begin{tikzcd}
			\C(Y,A)\rar\dar\drar[pushout] & \C(*,A)\rar\dar\drar[pushout] & 0\dar\\
			\C(X,A)\rar & \C(X/Y,A)\rar & \widetilde{\C}(X/Y,A)
		\end{tikzcd}
	\end{equation*}
	The left square is a pushout since $\C(-,A)$ preserves pushouts. To obtain the pushout square on the right, observe that $X/Y$ is canonically a pointed anima via $*\simeq Y/Y\rightarrow X/Y$. In general, for any pointed anima $(Z,z)\in\cat{An}_{*/}$, the canonical morphism $Z\rightarrow*$ has a preferred section given by $\{z\}\rightarrow Z$. Thus $\C(Z,\IZ)\rightarrow\C(*,\IZ)$ has a section and we obtain
	\begin{equation*}
		\C(Z,\IZ)\simeq \widetilde{\C}(Z,\IZ)\oplus \C\bigl(\{z\},\IZ\bigr)\,.
	\end{equation*}
	Hence the reduced chains $\widetilde{\C}(Z,\IZ)$ from \cref{con:Homology} can also be written as the cofibre $\widetilde{\C}(Z,\IZ)\simeq\cofib(\C(\{z\},\IZ)\rightarrow \C(Z,\IZ))$, functorially in $(Z,z)$. This explains the right square in the diagram above. Hence $\C(Y,A)\rightarrow\C(X,A)\rightarrow\widetilde{\C}(X/Y,A)$ is a cofibre sequence in $\Dd_{\geqslant 0}(\IZ)$ and thus in $\Dd(\IZ)$. By \cref{lem:ColimitsInDR}\cref{enum:CofibresInDR}, cofibre sequences in $\Dd(\IZ)$ can be represented by cone sequences, so the desired long exact sequence is simply the cone sequence.
	
	The suspension isomorphism and the Mayer--Vietoris sequence are formal consequences of the Eilenberg--Steenrod axioms. Alternatively, they can be checked by hand. For example, for Mayer--Vietoris, use that $\C(-,A)$ preserves pushouts and then apply the characterisation of pushouts in $\Dd(\IZ)$ from \cref{lem:ColimitsInDR}\cref{enum:CofibresInDR}.
	
	For~\cref{enum:SingularHomology}, let's first describe how to get a functor $\C^{\mathrm{sing}}(\abs*{\,-\,},A)\colon\cat{An}\rightarrow\Dd_{\geqslant 0}(\IZ)$. By \cref{thm:AnAsALocalisation} and \cref{lem:Localisation}, it's enough to check that $\C_*^\mathrm{sing}(\abs{\,-\,},A)\colon \cat{Kan}\rightarrow \Ch_{\geqslant 0}(\IZ)$ sends homotopy equivalences to quasi-isomorphisms, which is obviously true. So we get our desired functor. We know from \cref{lem:PresheafColimitOfRepresentables} and the Kan extension formula that $\C(-,A)\colon \cat{An}\rightarrow \Dd_{\geqslant 0}(\IZ)$ is the left Kan extension of its restriction to $\{*\}\subseteq \cat{An}$. By the universal property of Kan extensions, the equivalence $\C(*,A)\simeq A[0]\simeq \C^\mathrm{sing}(\abs*{*},A)$ extends to a natural transformation
	\begin{equation*}
		\C(-,A)\Longrightarrow \C^\mathrm{sing}\bigl(\abs*{\,-\,},A\bigr)\,.
	\end{equation*}
	Whether this is an equivalence can be checked pointwise and on homology groups. Now we can use the well-known fact that any unreduced homology theory $h_*$ with $h_0(*)\cong A$ and $h_n(*)\cong 0$ for $n\geqslant 1$ must necessarily coincide with singular homology with coefficients in $A$. See \cite[Theorem~\href{https://pi.math.cornell.edu/~hatcher/AT/AT.pdf\#page=408}{4.59}]{Hatcher}. Alternatively, we can directly show that $\C^{\mathrm{sing}}(\abs*{\,-\,},A)$ preserves colimits. Using \cref{lem:ColimitsInDR}, this is straightforward (preservation of coproducts is trivial and preservation of pushouts is, essentially, the Mayer--Vietoris sequence).
	
	The isomorphism $\H_*(X,A)\cong \H_*^\mathrm{sing}(\abs*{X},A)$ is an immediate consequence, as is its variant for reduced homology. For cohomology, we can argue as follows: We already know that $\C(X,\IZ)\simeq \C^\mathrm{sing}(\abs*{X},\IZ)$. The chain complex $\C_*^\mathrm{sing}(\abs{X},\IZ)$ is $K$-projective because it is degree-wise free over $\IZ$ and bounded below. Using the computation from crash course~\cref{con:DerivedCategoryIII}, we deduce
	\begin{equation*}
		\pi_0\Hom_{\Dd_{\geqslant 0}(\IZ)}\bigl(\C^\mathrm{sing}\bigl(\abs{X},\IZ\bigr),A[n]\bigr)\cong \H_0\Hhom_\IZ\bigl(\C_*^\mathrm{sing}\bigl(\abs{X},\IZ\bigr),A[n]\bigr)
	\end{equation*}
	Now $\Hhom_\IZ(\C_*^\mathrm{sing}(\abs{X},\IZ),A)\cong \C_\mathrm{sing}^{-*}(\abs{X},A)$ is the cochain complex that computes singular cohomology of $\abs{X}$, placed in negative degrees (so that it becomes a chain complex). Taking the shifts into account, we get $\H^n(X,A)\cong \H_\mathrm{sing}^n(\abs{X},A)$, as claimed. The same argument also shows the assertion about reduced cohomology.
\end{proof}
\begin{rem}
	Let's take a moment to appreciate the beauty of \cref{lem:Homology}\cref{enum:HomologyEilenbergSteenrod}. With our definition of homology, the Eilenberg--Steenrod axioms and all the usual properties of homology are completely formal. The only input we need is of algebraic nature: namely, the description of colimits in $\Dd(\IZ)$ from \cref{lem:ColimitsInDR}. Now compare that to the classical construction of singular homology: To prove the Mayer--Vietoris sequence (or equivalently excision), one has to take sufficiently fine barycentric subdivisions, apply Lebesgue's covering theorem, and construct a bunch of chain homotopies by hand (see \cite[Proposition~\href{https://pi.math.cornell.edu/~hatcher/AT/AT.pdf\#page=128}{2.21}]{Hatcher} for such an argument). Blargh! I find our approach much more enlightening and much less technical.\footnote{The careful traditionalist will---rightfully---object that our theory doesn't really avoid barycentric subdivision, it just moves it to the proof of \cref{thm:SimplicialApproximation}, conveniently hidden in a black box. Sure, but that doesn't undermine my point. Barycentric subdivision is a technical tool to compare animae to CW complexes---that's its natural place in the theory. Beyond that, it can be avoided.} In fact, I'd argue that \cref{con:Homology} is the better definition of homology (and an even better one is \cref{cor:Homology})!
	
	Although our definition is very abstract, for a given anima $X$, it's often easy to write down an explicit complex that computes $\C(X,A)$. For example, assume we're given a \enquote{CW decomposition} of $X$; that is, a way to write $X$ as a sequence of pushouts along $S^n\rightarrow *$ (the $n$-disk is contractible, so we may as well use $*$ instead). Then, $\C(X,A)$ can be written as a similar sequence of pushouts in $\Dd_{\geqslant 0}(\IZ)$. Since we understand $\C(S^n,A)\simeq A[0]\oplus A[n]$ as well as $\C(*,A)\simeq A[0]$ and since we know how to compute pushouts in $\Dd_{\geqslant 0}(\IZ)$ by \cref{lem:ColimitsInDR}\cref{enum:CofibresInDR}, we can compute $\C(X,A)$. If you think about this, the complex we end up with is precisely the \emph{cellular complex of $X$}, so we've just proved that homology agrees with cellular homology. Combining this with the classical fact that cellular and singular homology agree, we get an alternative proof of \cref{lem:Homology}\cref{enum:SingularHomology}.
\end{rem}
We finish this subsection by proving the classical Eilenberg--MacLane theorem. As we'll see, once again, the proof is entirely formal.
\begin{thm}[\enquote{Eilenberg--MacLane animae represent cohomology}]\label{thm:EilenbergMacLane}
	For every abelian group $A$ and all $n\geqslant 0$, the Eilenberg--MacLane anima $\K(A,n)$ from \cref{con:Homology} satisfies $\pi_n\K(A,n)\cong A$ and $\pi_i\K(A,n)\cong0$ for $i\neq n$. This condition determines $\K(A,n)$ uniquely up to homotopy equivalence. Furthermore, $\K(A,n)$ represents cohomology with coefficients in $A$ \embrace{both unreduced and reduced} in the sense that the functors $\H^n(-,A)\colon \cat{An}\rightarrow \cat{Ab}$ and $\widetilde{\H}^n(-,A)\colon \cat{An}_{*/}\rightarrow \cat{Ab}$ are given by
	\begin{equation*}
		\bigl[-,\K(A,n)\bigr]\coloneqq\pi_0\Hom_{\cat{An}}\bigl(-,\K(A,n)\bigr)\quad\text{and}\quad \bigl[-,\K(A,n)\bigr]_*\coloneqq\pi_0\Hom_{\cat{An}_{*/}}\bigl(-,\K(A,n)\bigr)\,,
	\end{equation*}
	respectively.
\end{thm}
\begin{proof}
	For every anima $X$ and every $M_*\in\Ch_{\geqslant 0}(\IZ)$, the adjunction $\C(-,\IZ)\colon \cat{An}\shortdoublelrmorphism\Dd_{\geqslant 0}(\IZ)\noloc \K$ from \cref{con:Homology} shows
	\begin{equation*}
		\Hom_{\cat{An}}\bigl(X,\K(M)\bigr)\cong \Hom_{\Dd_{\geqslant 0}(\IZ)}\bigl(\C(X,\IZ),M\bigr)\,.
	\end{equation*}
	In the case $M_*\simeq A[n]_*$, we immediately obtain $\pi_0\Hom_{\cat{An}}(X,\K(A,n))\simeq \H^n(X,A)$. This shows that $[-,\K(A,n)]\cong\H^n(-,A)$. The assertion about reduced cohomology follows analogously if we can show that the adjunction $\C(-,\IZ)\colon \cat{An}\shortdoublelrmorphism \Dd_{\geqslant 0}(\IZ)\noloc{\K}$ lifts to an adjunction
	\begin{equation*}
		\widetilde{\C}(-,\IZ)\colon \cat{An}_{*/}\doublelrmorphism \Dd_{\geqslant 0}(\IZ)\noloc {\K}\,.
	\end{equation*}
	In \cref{lem:SliceAdjunction} below, we'll show a general fact about passing adjunctions to slice $\infty$-categories. Let's explain how this applies in our situation: Since $\K$ is a right adjoint, it preserves terminal objects, whence $\K(0)\simeq *$. But $0$ is also an initial object in $\Dd_{\geqslant 0}(\IZ)$. Hence $\Dd_{\geqslant 0}(\IZ)\simeq \Dd_{\geqslant 0}(\IZ)_{0/}$ and so the induced functor $\K\colon \Dd_{\geqslant 0}(\IZ)\simeq \Dd_{\geqslant 0}(\IZ)_{0/}\rightarrow \cat{An}_{\K(0)/}\simeq \cat{An}_{*/}$ on slice $\infty$-categories is indeed of the form studied in \cref{lem:SliceAdjunction}. We've seen in the proof of \cref{lem:Homology}\cref{enum:HomologyEilenbergSteenrod} that for every pointed anima $(X,x)$ one has $\widetilde{\C}(X,\IZ)\simeq \cofib(\C(\{x\},\IZ)\rightarrow \C(X,\IZ))$, so $\widetilde{\C}(-,\IZ)$ agrees with the left adjoint constructed in \cref{lem:SliceAdjunction}.
	
	To compute the homotopy groups of $\K(A,n)$, let $S^i\in\cat{An}$ be the \emph{$i$-sphere}.\footnote{There are many possible constructions for $S^i$. The most conceptual way would be to define $S^i\coloneqq \Sigma^i(*\ \,*)$ as the $i$-fold suspension of two points, using the upcoming definition \cref{def:Loop}. But there are also many possible simplicial models. For example, if $\partial D^{i+1}\subseteq D^{i+1}$ is the boundary of the topological $(i+1)$-disk, we could take $\Sing \partial D^{i+1}$ as our model for $S^i$. Alternatively, we could choose anodyne maps from $\square^i/\partial\square^i$ or $\Delta^i/\partial\Delta^i$ or $\partial\Delta^{i+1}$ into Kan complexes. All constructions you could possibly come up with will be homotopy equivalent, so you can just choose your favourite option.} Plugging in $(S^i,*)$ for any choice of basepoint yields
	\begin{equation*}
		\pi_i\K(A,n)\cong \pi_0\Hom_{\cat{An}_{*/}}\bigl((S^i,*),\K(A,n)\bigr)\cong \widetilde{\H}^n(S^i,A)\,.
	\end{equation*}
	By \cref{lem:Homology}, one has $\widetilde{\H}^n(S^n,A)\cong A$ and $\widetilde{\H}^n(S^i,A)\cong 0$ for $i\neq n$, and the desired description of $\pi_*\K(A,n)$ follows. By the usual argument from topology, $\K(A,n)$ is uniquely determined by this property up to homotopy equivalence.
\end{proof}
The following lemma was used in the proof:
\begin{lem}\label{lem:SliceAdjunction}
	Let $L\colon \Cc\shortdoublelrmorphism \Dd\noloc R$ be an adjunction of $\infty$-categories and let $y\in \Dd$. If for every morphism $R(y)\rightarrow x$ in $\Cc$ the pushout
	\begin{equation*}
		\begin{tikzcd}
			LR(y)\rar\dar["c_y"']\drar[pushout] & L(x)\dar\\
			y\rar & y\sqcup_{LR(y)}L(x)
		\end{tikzcd}
	\end{equation*}
	exists in $\Dd$, then the functor $R\colon \Dd_{y/}\rightarrow \Cc_{R(y)/}$ on slice $\infty$-categories still has a left adjoint $L_y\colon \Cc_{R(y)/}\rightarrow \Dd_{y/}$. On objects, $L_y$ is given by $L_y(R(y)\rightarrow x)\simeq (y\rightarrow y\sqcup_{LR(y)}L(x))$, constructed via the pushout square above. Moreover, the pushout square can be made functorial in an obvious way and this recovers $L_y$ as a functor \embrace{not only pointwise}.
\end{lem}
\begin{proof}[Proof sketch]
	You can directly verify $\Hom_{\Dd_{y/}}(L_y(R(y)\rightarrow x),-)\simeq \Hom_{\Cc_{R(y)/}}(R(y)\rightarrow x,R(-))$. To do so, plug in \cref{cor:HomPreservesColimits}, \cref{cor:HomInSliceCategories}, and the given adjunction $L\dashv R$, then perform a formal manipulation of pullbacks. Since adjoints can be constructed pointwise (\cref{lem:Adjunction}), this proves the existence of $L_y$. With a little more care, one can make the pullback manipulation functorial in $R(y)\rightarrow x$ as well, and then the claimed description of $L_y$ follows.
\end{proof}

\newpage

\sectionappendix{Presentable \texorpdfstring{$\infty$}{Infinity}-categories}
%\addcontentsline{toc}{sectionappendix}{Appendix to \texorpdfstring{\cref{sec:InftyCategoryTheory}}{\S6}. Presentable \texorpdfstring{$\infty$}{Infinity}-categories}
%\sectionmark{Toast}

Suppose $\Cc$ is an $\infty$-category with all colimits and let $F\colon \Cc\rightarrow\Dd$ be a colimits-preserving functor of $\infty$-categories. Then the only thing preventing $F$ from having a right adjoint is set theory. Indeed, the values of a hypothetical right adjoint $G\colon \Dd\rightarrow\Cc$ would be given by $G(y)\simeq \colimit(\Cc_{/y}\rightarrow \Cc)$ for all $y\in\Dd$ (as we'll see in the proof of \cref{thm:AdjointFunctorTheorem}\cref{enum:AdjointFunctorTheoremLeft}), except that this colimit usually doesn't exist, even though $\Cc$ has all colimits. The problem ist that $\Cc_{/y}$ is usually not an \emph{essentially small} $\infty$-category in the sense of \cref{def:KappaSmall}\cref{enum:Small} below. So far, we have ignored these smallness issues. Still, \crefrange{subsec:Adjunctions}{subsec:KanExtensions} can be made set-theoretically sound. As a rule of thumb, whenever a limit or colimit is considered, the indexing $\infty$-category should be assumed small (or at least admit a coinitial/initial functor from an essentially small $\infty$-category) and whenever we consider $\PSh(\Cc)$, we should assume that $\Cc$ is essentially small. The only time this gets hairy is in the proof of \cref{lem:KanExtensionFormula}, where we should allow $\Dd$ to be large, but also consider $\PSh(\Dd)$. Nevertheless, this can be fixed too.\footnote{For example, by using universes, but Fabian proposed a trick to get away with ZFC: Instead of $\PSh(\Dd)$, consider the $\infty$-category of right fibrations $\Uu\rightarrow \Dd$, for which $\Uu$ admits a coinitial functor from an essentially small $\infty$-category. This $\infty$-category contains $\Dd_{/y}\rightarrow\Dd$ for all $y\in \Dd$, so all Yoneda arguments go through.}

However, a more thorough analysis is needed to save our adjoint functor argument. In fact, $\infty$-categories $\Cc$ with all colimits are very seldomly essentially small, and so neither is $\Cc_{/y}$. However, often there exists an essentially small sub-$\infty$-category $\Cc_0\subseteq\Cc$ that generates $\Cc$ under colimits, and in this case one can replace $\Cc_{/y}$ by a coinitial essentially small sub-$\infty$-category, so that the required colimits do exist. The theory of accessible and presentable $\infty$-categories makes these ideas precise and allows to prove an incredibly powerful \emph{adjoint functor theorem}.

In \crefrange{subsec:EssentiallySmall}{subsec:AdjointFunctorTheorem}, we'll give the necessary definitions and prove Lurie's adjoint functor theorem (\cref{thm:AdjointFunctorTheorem}). After that, we'll discuss some supplements in \cref{subsec:PrL}. Naturally, this means that \crefrange{subsec:EssentiallySmall}{subsec:PrL} will be very technical. If you're mainly interested in spectra and willing to take the adjoint functor theorem on faith, you can safely skip ahead to \cref{sec:TowardsSpectra} at this point. If instead you're looking for a much more detailed exposition, you should consult \cite[\S\href{https://people.math.harvard.edu/~lurie/papers/HTT.pdf\#page=332}{5}]{HTT}.


\subsection{Essentially small and locally small \texorpdfstring{$\infty$}{Infinity}-categories}\label{subsec:EssentiallySmall}

First we'll explain how to put cardinality bounds on $\infty$-categories.
\begin{defi}\label{def:KappaSmall}
	Let $\kappa$ be a regular cardinal and let $\Cc$ be an $\infty$-category.
	\begin{alphanumerate}
		\item If $\kappa=\aleph_0$, then $\Cc$ is called \emph{essentially $\aleph_0$-small} if it is contained in the full sub-$\infty$-category of $\cat{Cat}_\infty$ generated under pushouts by $\emptyset$ and $\Delta^n$ for all $n\geqslant 0$. If $\kappa$ is uncountable, then $\Cc$ is called \emph{essentially $\kappa$-small} if $\pi_0\core \Cc$ as well as $\pi_0\Hom_\Cc(x,y)$ and $\pi_n(\Hom_\Cc(x,y),\alpha)$ are sets of cardinality $<\kappa$ for all $x,y\in\Cc$, all $\alpha\colon x\rightarrow y$, and all $n\geqslant 1$.\label{enum:KappaSmallCategory}%$\pi_0\Hom_{\cat{Cat}_\infty}(\Delta^n,\Cc)$ has cardinality $<\kappa$ for all $n\geqslant 0$. \label{enum:KappaSmallCategory}
		\item $\Cc$ is called \emph{essentially small} if it is essentially $\kappa$-small for some regular cardinal $\kappa$, and \emph{large} otherwise. $\Cc$ is called \emph{locally small} if $\Hom_\Cc(x,y)$ is essentially small for all $x,y\in\Cc$.\label{enum:Small}
		\item A colimit or a limit over a functor $F\colon \Ii\rightarrow \Cc$ is called \emph{$\kappa$-small} if $\Ii$ is essentially $\kappa$-small. Instead of \emph{$\aleph_0$-small}, we often say that a limit or colimit is \emph{finite}.\label{enum:KappaSmallLimit}
	\end{alphanumerate}
\end{defi}
\begin{rem}\label{rem:FunLocallySmall}
	If $\Cc$ is a small $\infty$-category and $\Dd$ is locally small, then $\Fun(\Cc,\Dd)$ is again locally small, as can be seen by \cref{cor:HomInFunctorCats}. In particular, $\PSh(\Cc)$ and the $\infty$-categories $\cat{Ind}_\kappa(\Cc)$ from \cref{con:Ind} below will be locally small.
\end{rem}
For practical applications, it will, unfortunately, be necessary to translate our nice model independent \cref{def:KappaSmall}\cref{enum:KappaSmallCategory} into the language of simplicial sets.
%\cref{def:KappaSmall}\cref{enum:KappaSmallCategory} is rather cumbersome to work with. We chose it because we're striving for model-independent notions, but 
\begin{lem}\label{lem:KappaSmall}
	Let $\kappa$ be an uncountable regular cardinal and let $\Cc$ be an $\infty$-category. Then the following are equivalent:
	\begin{alphanumerate}
		\item $\Cc$ is essentially $\kappa$-small.\label{enum:KappaSmallA}
		\item $\Cc$ is equivalent to a quasi-category with $<\kappa$ simplices across all dimensions.\label{enum:KappaSmallB}
		\item \!There exists a simplicial set $K$ with $<\kappa$ simplices across all dimensions and a Joyal equivalence $K\rightarrow \Cc$ \embrace{that is, a weak equivalence in the Joyal model structure from \cref{exm:JoyalModelStructure}}.\label{enum:KappaSmallC}
	\end{alphanumerate}
	Furthermore, if $K$ is a finite simplicial set \embrace{that is, a simplicial set with only finitely many non-degenerate simplices} and $K\rightarrow \Cc$ is a Joyal equivalence, then $\Cc$ is $\aleph_0$-small.
\end{lem}
\begin{proof}[Proof sketch]
	The implications \cref{enum:KappaSmallB} $\Rightarrow$ \cref{enum:KappaSmallA} and \cref{enum:KappaSmallB} $\Rightarrow$ \cref{enum:KappaSmallC} are trivial. For \cref{enum:KappaSmallC} $\Rightarrow$ \cref{enum:KappaSmallB} let $K\rightarrow \Cc'$ be the inner anodyne map into a quasi-category provided by the proof of \cref{lem:SmallObjectArgument}. Then $\Cc'$ has again $<\kappa$ simplices across all dimensions, because we're attaching $<\kappa$ new simplices countably many times. For the additional assertion, use induction on the dimension and write $K$ as a sequence of pushouts against $\coprod\partial\Delta^n\rightarrow \coprod\Delta^n$, where the disjoint union is finite. Replacing everything by quasi-categories and using model category fact~\cref{par:HomotopyPushout}, we conclude that $\Cc$ is contained in the full sub-$\infty$-category of $\cat{Cat}_\infty$ generated under pushouts by $\emptyset$ and $\Delta^n$ for all $n\geqslant 0$, as desired.
	
	It remains to show \cref{enum:KappaSmallA} $\Rightarrow$ \cref{enum:KappaSmallB}. We build a sub-simplicial set $\Cc'\subseteq\Cc$ as follows: Start with $\Cc'=\emptyset$. Choose $<\kappa$ representatives for every equivalence class in $\pi_0\core (\Cc)$ and add them to $\Cc'$. For all $x,y\in\Cc'$ and every equivalence class in $\pi_0\Hom_\Cc(x,y)$, we add a representative $\alpha\colon x\rightarrow y$. Furthermore, for every $n\geqslant 1$ and every class in $\pi_n(\Hom_\Cc(x,y),\alpha)$, we choose a representative $\Delta^n/\partial\Delta^n\rightarrow \Hom_\Cc(x,y)$ and add the simplices in the image of the corresponding map $\Delta^n/\partial\Delta^n\times\Delta^1\rightarrow\Cc$ to $\Cc'$. Then $\Cc'$ still has $<\kappa$ simplices. Mimicking the proof of \cref{lem:SmallObjectArgument}, we can add $<\kappa$ further simplices to $\Cc'$ to ensure that $\Cc'$ is a quasi-category. By construction, $\Cc'\rightarrow\Cc$ is essentially surjective and the map $\Hom_{\Cc'}(x,y)\rightarrow\Hom_\Cc(x,y)$ is a surjection on all $\pi_n$ for all $x,y\in\Cc'$. To make it injective, for every class in the kernel,  choose a homotopy $\Delta^n/\partial\Delta^n\times\Delta^1\rightarrow \Hom_\Cc(x,y)$ to $\const\alpha$. This homotopy corresponds to a map $(\Delta^n/\partial\Delta^n\times\Delta^1)\times\Delta^1\rightarrow \Cc$ and we add its image to $\Cc'$. Then we add $<\kappa$ simplices to make $\Cc'$ into a quasi-category again. Clearly, $\Cc'\rightarrow\Cc$ is still essentially surjective; furthermore, all elements in the previous kernel of $\pi_n(\Hom_{\Cc'}(x,y),\alpha)\rightarrow\pi_n(\Hom_\Cc(x,y),\alpha)$ have been killed now. But there could be new ones. So we simply repeat this process countably many times. Then $\Cc'\rightarrow \Cc$ is fully faithful too and thus an equivalence by \cref{thm:EquivalenceFullyFaithfulEssentiallySurjective}.%
	%
	%By adding another $<\kappa$ simplices to $\Cc'$, we can ensure that for all $x,y\in\Cc'$, every equivalence class in $\pi_0\F(\partial\Delta^n,\Hom_\Cc(x,y))$ and $\pi_0\F(\Delta^n,\Hom_\Cc(x,y))$ has a representative in $\Cc'$. Indeed, this can be done by applying the above observation to $K=(\partial\Delta^n\times\Delta^1)/(\partial\Delta^n\times\{0,1\})$ and $K=(\Delta^n\times\Delta^1)/(\Delta^n\times\{0,1\})$. Mimicking the proof of \cref{lem:SmallObjectArgument}, we can add $<\kappa$ further simplices to $\Cc'$ to ensure that $\Cc'$ is a quasi-category. Then $\Cc'$ has $<\kappa$ simplices across all dimensions. Furthermore, $\Cc'\rightarrow\Cc$ is essentially surjective by construction and for all $x,y\in\Cc'$ the map $\Hom_{\Cc'}(x,y)\rightarrow\Hom_\Cc(x,y)$ is a bijection on $\pi_0$ and an isomorphism on $\pi_n$ for all $n\geqslant 1$ and all basepoints. Hence $\Cc'\rightarrow\Cc$ is fully faithful too and thus an equivalence by \cref{thm:EquivalenceFullyFaithfulEssentiallySurjective}.%First, a simple induction over the number of simplices shows that $\pi_0\core\F(K,\Cc)$ has cardinality $<\kappa$ for every finite simplicial set $K$. For the inductive step, write $K\cong K'\sqcup_{\partial\Delta^n}\Delta^n$, so that $\F(K,\Cc)\cong \F(K',\Cc)\times_{\F(\partial\Delta^n,\Cc)}\F(\Delta^n,\Cc)$. Let $S\subseteq \F(K',\Cc)_0$ be a set of $<\kappa$ representatives for every equivalence class in $\pi_0\core\F(K',\Cc)$ and define $T\subseteq \F(\Delta^n,\Cc)_0$ similarly. For every $\sigma\in S$ and $\tau\in T$ such that the images of $\sigma$ and $\tau$ in $\F(\partial\Delta^n,\Cc)$ are equivalent, we can find $\tau'\in \F(\Delta^n,\Cc)$ such that $\tau\simeq \tau'$ and the images of  $\sigma$ and $\tau'$ in $\F(\partial\Delta^n,\Cc)$ are equal. Indeed, claim~\cref{claim:Pullback} from the proof of \cref{thm:EquivalenceFullyFaithfulEssentiallySurjective} allows us to lift equivalences. Then $\{(\sigma,\tau')\}$ is a set of $<\kappa$ representatives for every equivalence class in $\pi_0\core\F(K,\Cc)$, finishing the induction.
\end{proof}
\begin{rem}\label{rem:KappaSmallClosedUnderPushouts}
	If $\kappa$ is an uncountable regular cardinal, then pushouts or pullbacks of essentially $\kappa$-small $\infty$-categories are essentially $\kappa$-small again. Indeed, this follows from \cref{lem:KappaSmall}\cref{enum:KappaSmallB} together with model category facts~\cref{par:HomotopyPullback} and~\cref{par:HomotopyPushout} and a cardinality bound on \cref{lem:SmallObjectArgument}: A functor between quasi-categories with $<\kappa$ simplices across all dimensions can be factored into a cofibration followed by a trivial fibration or into a Joyal equivalence followed by an isofibration in such a way that the new quasi-category in the middle has again $<\kappa$ simplices across all dimensions. Combining this observation with \cref{lem:KappaSmallColimits} below, we see that the full sub-$\infty$-category $\cat{Cat}_\infty^{<\kappa}$ of essentially $\kappa$-small $\infty$-categories is closed under $\kappa$-small limits and colimits.
	
	In the case $\kappa=\aleph_0$ it's obvious that $\aleph_0$-small $\infty$-categories are closed under pushouts and thus under finite colimits by \cref{lem:KappaSmallColimits} below. The same can be shown for finite products, but I don't know if it works for pullbacks too.
\end{rem}
\begin{lem}\label{lem:KappaSmallColimits}
	Let $\kappa$ be a regular cardinal. An $\infty$-category $\Cc$ has all $\kappa$-small colimits if and only if $\Cc$ has pushouts and $\kappa$-small coproducts. A functor $F\colon \Cc\rightarrow\Dd$ of $\infty$-categories preserves colimits if and only if it preserves pushouts and $\kappa$-small coproducts. A dual assertion holds for limits.
\end{lem}
\begin{proof}[Proof sketch]
	Repeat the proof of \cref{lem:ColimitsIffCoproductsAndPushouts} and use \cref{lem:KappaSmall} together with model category fact~\cref{par:HomotopyPushout} to see that pushouts of $\kappa$-small $\infty$-categories are still $\kappa$-small.
\end{proof}
%This finishes our discussion of cardinality bounds on $\infty$-categories. Our next goal is to study filtered colimits in the $\infty$-setting.
\subsection{Filtered colimits}

In this subsection, we'll study filtered colimits in $\infty$-categories and prove a version of the well-known fact that filtered colimits commute with finite limits (\cref{lem:FilteredColimitsPreserveFiniteLimits}).

\begin{con}\label{con:ConeCategory}
	Let $\Ii$ be an $\infty$-category. We define the \emph{cone $\Ii^\triangleleft$ over $\Ii$} and the \emph{cocone $\Ii^\triangleright$ under $\Ii$} as the following pushouts in $\cat{Cat}_\infty$:
	\begin{equation*}
		\begin{tikzcd}
			\Ii\times\{0\}\rar\dar\drar[pushout] & \Ii\times\Delta^1\dar\\
			*\rar & \Ii^\triangleleft
		\end{tikzcd}\quad\text{and}\quad
		\begin{tikzcd}
			\Ii\times\{1\}\rar\dar\drar[pushout] &\Ii\times\Delta^1\dar\\
			*\rar & \Ii^\triangleright
		\end{tikzcd}
	\end{equation*}
	It's tempting to use the procedure from model category fact~\cref{par:HomotopyPushout} to compute these pushouts explicitly, but this is a little tricky. Steps \cref{enum:PushoutStepA}, \cref{enum:PushoutStepB}, and \cref{enum:PushoutStepC} are easy though: $\Ii\times \{0\}\rightarrow \Ii\times\Delta^1$ and $\Ii\times\{1\}\rightarrow \Ii\times\Delta^1$ are already cofibrations, so we can simply take the pushout on the nose. The tricky step, however, is \cref{enum:PushoutStepD}, in which one has to replace the pushout in   $\cat{sSet}$ by a quasi-category. One can show that the \emph{joins} $\{0\}\star\Ii$ and $\Ii\star\{1\}$, which we didn't introduce, are such replacements; see \cite[Proposition~\HTTthm{4.1.2.1}]{HTT} or \cite[Proposition~2.5.19]{Land}. We won't need this explicit description and work with the abstract construction exclusively.
	
	Note that $*\rightarrow \Ii^\triangleleft$ is an initial object and $*\rightarrow \Ii^\triangleright$ is a terminal object. This is obvious in the simplicial models, but there's also a model-independent argument: We must show that $*\rightarrow\Ii^\triangleleft$ is left adjoint to the unique functor $\Ii^\triangleleft\rightarrow *$. This can be done via \cref{lem:TriangleIdentities} by constructing the unit and counit by hand. The unit is clear, as there are not that many functors from $*$ to itself (in fact, there's only one). For the counit, we must construct a natural transformation $c\colon \Ii^\triangleleft\times\Delta^1\rightarrow\Ii^\triangleleft$ from $\const *$ to $\id_{\Ii^\triangleleft}$. Using that $-\times\Delta^1\colon \cat{Cat}_\infty\rightarrow \cat{Cat}_\infty$ commutes with pushouts (since $\Fun(\Delta^1,-)$ is a right adjoint by \cref{exm:Adjunctions}\cref{enum:Currying}), this boils down to constructing a natural transformation $\Delta^1\times\Delta^1\rightarrow\Delta^1$ from $\const 0$ to $\id_{\Delta^1}$, which is easy. In the same way, verifying the triangle identities reduces to a question about $\Delta^1$. In particular, we deduce $\abs{\Ii^\triangleleft}\simeq *$ and $\abs{\Ii^\triangleright}\simeq *$, as $\infty$-categories with an initial or terminal object are always weakly constractible.
	
	As in ordinary category theory, cones and cocones are closely related to limits and colimits, respectively. Concretely, if $F\colon \Ii\rightarrow \Cc$ is a functor and $y\in\Cc$ is an object, then an easy calculation using \cref{cor:HomPreservesLimits} shows
	\begin{equation*}
		\{F\}\times_{\Hom_{\cat{Cat}_\infty}(\Ii,\Cc)}\Hom_{\cat{Cat}_\infty}(\Ii^\triangleright,\Cc)\times_{\Hom_{\cat{Cat}_\infty}(*,\Cc)}\{y\}\simeq \Hom_{\Fun(\Ii,\Cc)}(F,\const y)\,.
	\end{equation*}
	Informally, an extension of $F$ to a functor $F^\triangleright \colon \Ii^\triangleright\rightarrow \Cc$ that sends the tip $*\in \Cc^\triangleright$ to $y$ is the same as a natural transformation $F\Rightarrow \const y$. If $F$ admits a colimit, such a natural transformation is the same as a morphsm $\colimit_{i\in\Ii}F(i)\rightarrow y$, and the right-hand side above is equivalent to $\Hom_\Cc(\colimit_{i\in\Ii}F(i),y)$.
\end{con}

\begin{defi}\label{def:KappaFiltered}
	Let $\kappa$ be a regular cardinal and let $\Jj$, $\Cc$ be $\infty$-categories.
	\begin{alphanumerate}
		\item $\Jj$ is called \emph{$\kappa$-filtered} if every functor $\Ii\rightarrow\Jj$ from an essentially $\kappa$-small $\infty$-category extends to a functor $\Ii^\triangleright\rightarrow\Jj$ from the cocone under $\Cc$, or in other words, if the restriction $\Fun(\Ii^\triangleright,\Jj)\rightarrow\Fun(\Ii,\Jj)$ is essentially surjective. In the case $\kappa=\aleph_0$, we usually just say $\Jj$ is \emph{filtered}.\label{enum:KappaFilteredCategory}
		\item A colimit over a functor $F\colon \Jj\rightarrow\Cc$ is called \emph{$\kappa$-filtered} if $\Jj$ is $\kappa$-filtered, and \emph{filtered} if $\Jj$ is filtered.\label{enum:KappaFilteredColimit}
		\item An object $x\in\Cc$ is called \emph{$\kappa$-compact} or \emph{compact} if $\Hom_\Cc(x,-)\colon \Cc\rightarrow\cat{An}$ commutes with $\kappa$-filtered or filtered colimits, respectively.\label{enum:KappaCompact} 
	\end{alphanumerate}
\end{defi}
\begin{rem}\label{rem:LuriesFilteredness}
	We'll explain why Lurie's definition of $\kappa$-filteredness in \cite[Definition~\HTTthm{5.3.1.7}]{HTT} is equivalent to ours. Let $\Jj$ be a $\kappa$-filtered quasi-category as in \cref{def:KappaFiltered}\cref{enum:KappaFilteredCategory}. Furthermore, let $\Ii$ be an essentially $\kappa$-small quasi-category and choose the simplicial model $\Ii\star\{1\}$ for $\Ii^\triangleright$ (as Lurie does). Then any functor $\Ii\rightarrow\Jj$ can not only be extended to $\Ii^\triangleright\rightarrow\Jj$ up to equivalence, but even on the nose. The reason is that $\Ii\rightarrow\Ii^\triangleright$ is a cofibration and thus $\core\F(\Ii^\triangleright,\Jj)\rightarrow \core\F(\Ii,\Jj)$ has lifting against $\{0\}\rightarrow\Delta^1$ by claim~\cref{claim:Pullback} in the proof of \cref{thm:EquivalenceFullyFaithfulEssentiallySurjective}. Then \cref{lem:KappaSmall} easily implies that Lurie's definition of $\kappa$-filteredness is equivalent to ours in the case where $\kappa$ is uncountable.
	
	If $\kappa=\aleph_0$, then \cref{lem:KappaSmall} shows that any filtered $\infty$-category $\Jj$ in the sense of \cref{def:KappaFiltered}\cref{enum:KappaFilteredCategory} is also filtered in Lurie's sense. The converse is true as well, but not as obvious (at least to me), since I don't know if the converse of the additional assertion in \cref{lem:KappaSmall} is true (I'd guess it's not). So here's a different argument: If $\Jj$ is filtered in Lurie's sense, then $\colimit\colon \Fun(\Jj,\cat{An})\rightarrow\cat{An}$ preserves finite limits (by \cite[Proposition~\HTTthm{5.3.3.3}]{HTT} or by observing that the proof of \cref{lem:FilteredColimitsPreserveFiniteLimits} still goes through). Hence \cref{lem:FilteredColimitsPreserveFiniteLimits}, which we'll prove next, implies that $\Jj$ is filtered in our sense. 
\end{rem}
\begin{thm}\label{lem:FilteredColimitsPreserveFiniteLimits}
	Let $\kappa$ be a regular cardinal. Then an $\infty$-category is $\kappa$-filtered if and only if the functor $\colimit\colon \Fun(\Jj,\cat{An})\rightarrow\cat{An}$ preserves $\kappa$-small limits.
\end{thm}
Before we can prove \cref{lem:FilteredColimitsPreserveFiniteLimits}, we need to send four more lemmas in advance.%First, we need a general coinitiality result for $\kappa$-filtered $\infty$-categories.
\begin{lem}\label{lem:FilteredCofinal}
	Let $\kappa$ be a regular cardinal and let $\Jj$ be a $\kappa$-filtered $\infty$-category. Then $\abs*{\Jj}\simeq *$. Furthermore, for every $j\in\Jj$ the slice $\Jj_{j/}$ is $\kappa$-filtered again and $\Jj_{j/}\rightarrow \Jj$ is coinitial.
\end{lem}
\begin{proof}[Proof sketch]
	To see $\abs{\Jj}\simeq *$, unfortunately, we need to use simplicial methods. It's enough to show that every map $\sigma\colon \partial\Delta^n\rightarrow\abs{\Jj}$ is nullhomotopic, because then the same argument as in the proof of \cref{lem:ContractibleKanComplex} shows that $\abs{\Jj}\rightarrow*$ is a trivial fibration. We'll show that for every $\sigma$ there is a functor $\alpha\colon\Ii\rightarrow \Jj$ from $\aleph_0$-small $\infty$-category $\Ii$ such that $\sigma$ factors through $\abs{\alpha}\colon\abs{\Ii}\rightarrow \abs{\Jj}$. This will be enough since then $\sigma$ also factors through $*\simeq \abs{\Ii^\triangleright}\rightarrow\abs{\Jj}$ by filteredness of $\Jj$. To construct $\alpha$, recall that $\Jj\rightarrow\abs{\Jj}$ can be constructed as an anodyne map into a Kan complex via \cref{lem:SmallObjectArgument}. Accordingly, as simplicial sets, $\abs{\Jj}\cong \colimit_{i\geqslant 0}\Jj_i$, where $\Jj_0=\Jj$ and $\Jj_{i+1}$ is obtained from $\Jj_i$ by attaching solutions to horn filling problems. All the finitely many simplices in the image of $\sigma\colon\partial\Delta^n\rightarrow\abs{\Jj}$ must already be contained in $\Jj$ or occur in $\Jj_i$ as a solution to some horn filling problem. If the latter is the case, all the finitely many simplices involved in that horn filling problem must already occur in $\Jj$ or in some $\Jj_k$ for $k<i$. Continuing in this way, we can trace back $\sigma$ to a finite number of simplices in $\Jj$. Completing these finitely many simplices to a sub-quasi-category $\Ii\subseteq\Jj$ as in the proof of \cref{lem:KappaSmall} yields the desired functor $\alpha\colon \Ii\rightarrow\Jj$. %use not only simplicial methods, but \cref{thm:SimplicialApproximation} (please tell me if you know a better argument): Let $X$ be the CW complex obtained as the geometric realisation of the Kan complex $\abs*{\Jj}$. Then $X$ is homotopy equivalent to the geometric realisation of $\Jj$, because $\Jj\rightarrow \abs*{\Jj}$ is anodyne by \cref{con:Localisation}. So it suffices to prove that every map $\alpha\colon \partial D^n\rightarrow X$ from the boundary of a topological $n$-disk extends all of $D^n$, up to homotopy. This follows from simplicial approximation in the form of \cite[Theorem~\href{https://pi.math.cornell.edu/~hatcher/AT/AT.pdf\#page=186}{2C.1}]{Hatcher}: Write $\partial D^n\cong \abs*{\partial \Delta^n}$; up to homotopy, we may assume that $\alpha$ is the geometric realisation of some map $\alpha_0\colon \operatorname{sd}^m(\partial\Delta^n)\rightarrow \Jj$ from some barycentric subdivision of $\partial\Delta^n$. Since $\Jj$ is filtered, $\alpha_0$ extends to $\alpha_0^\triangleright\colon\operatorname{sd}^m(\partial\Delta^n)^\triangleright\rightarrow\Jj$. Since $\abs*{\operatorname{sd}^m(\partial\Delta^n)^\triangleright}\simeq D^n$, we've proved that $\alpha$ extends to all of $D^n$ up to homotopy. This proves $\abs*{\Jj}\simeq *$.
	
	For the other assertions, let $\Ii\rightarrow\Jj_{j/}$ be a map from an essentially $\kappa$-small $\infty$-category. By unravelling the respective universal properties, such a map is equivalent to a map $\Ii^\triangleleft\rightarrow \Jj$ sending the tip of the cone to $j$. Since $\Ii^\triangleleft$ is still essentially $\kappa$-small, we get an extension $(\Ii^\triangleleft)^\triangleright\rightarrow\Jj$. Since $(\Ii^\triangleleft)^\triangleright\simeq (\Ii^\triangleright)^\triangleleft$, this defines a map $\Ii^\triangleright\rightarrow\Jj_{j/}$, proving that $\Jj_{j/}$ is $\kappa$-filtered. An analogous argument shows that $\Jj_{j/}\times_\Jj\Jj_{j'/}$ is $\kappa$-filtered for every $j'\in \Jj$. Hence $\abs{\Jj_{j/}\times_\Jj\Jj_{j'/}}\simeq *$ by the first part. Thus $\Jj_{j/}\rightarrow\Jj$ is coinitial by \cref{thm:JoyalsQuillenA}\cref{enum:WeaklyContractible}.
\end{proof}
%Next, we need a result about limits and colimits in slice $\infty$-categories.
\begin{lem}\label{lem:ColimitsInSliceCategory}
	Let $\Dd$ be an $\infty$-category and $y\in\Dd$ an object.
	\begin{alphanumerate}
		\item $\Dd_{y/}\rightarrow\Dd$ preserves and detects arbitrary limits. That is, a diagram $\alpha\colon\Ii\rightarrow\Dd_{y/}$ has a limit in $\Dd_{y/}$ if and only if the underlying diagram $\ov\alpha\colon\Ii\rightarrow \Dd_{y/}\rightarrow\Dd$ has a limit in $\Dd$, in which case these limits coincide in $\Dd$.\label{enum:LimitsInSlice}
		\item $\Dd_{y/}\rightarrow\Dd$ preserves and detects $\Ii$-shaped colimits if $\abs{\Ii}\simeq *$. In particular, this applies to pushouts \embrace{since $\abs{\Lambda_0^2}\simeq*$} and filtered colimits \embrace{by \cref{lem:FilteredCofinal}}.\label{enum:ColimitsInSlice}
		\item In general, let $\alpha\colon \Ii\rightarrow \Dd_{y/}$ be a diagram in $\Dd_{y/}$ and let $\ov\alpha\colon \Ii\rightarrow \Dd_{y/}\rightarrow\Dd$ be the underlying diagram in $\Dd$. If the colimits $\colimit_{i\in\Ii}\ov\alpha(i)$ and $\colimit_{i\in\Ii}y$ as well as the pushout\label{enum:ColimitsInSliceGeneral}
		\begin{equation*}
			\begin{tikzcd}
				\colimit_{i\in\Ii}y\dar\rar\drar[pushout] & \colimit_{i\in\Ii}\ov\alpha(i)\dar\\
				y\rar & c
			\end{tikzcd}
		\end{equation*}
		exist in $\Dd$, then $(y\rightarrow c)\in\Dd_{y/}$ is the colimit of $\alpha\colon \Ii\rightarrow \Dd_{y/}$.
	\end{alphanumerate}
	%Dual assertions hold for the other slice projection $\Dd_{/y}\rightarrow\Dd$.
\end{lem}
\begin{proof}[Proof sketch]
	For \cref{enum:LimitsInSlice}, first consider the case where $\Dd$ has all limits. The functor $\alpha\colon\Ii\rightarrow\Dd_{y/}$ defines a natural transformation $\const y\Rightarrow \ov\alpha$, hence a morphism $y\rightarrow \lim_{i\in\Ii}\ov\alpha(i)$. We claim that $(y\rightarrow \limit_{i\in\Ii}\ov\alpha(i))$ is the limit of $\alpha$. Indeed, using \cref{cor:HomInSliceCategories} and the fact that limits commute with limits by the dual of \cref{lem:ColimitManipulations}, we immediately verify the condition from \cref{cor:HomPreservesColimits}. This concludes the case where $\Dd$ has all limits. The general case can be reduced to this special case by considering a fully faithful limits-preserving functor $i\colon \Dd\rightarrow\Dd'$ into an $\infty$-category with all limits; for example, $\Yo_\Dd\colon \Dd\rightarrow\Fun(\Dd^\op,\cat{An})$ does it by \cref{cor:HomPreservesLimits}.
	
	Assertion \cref{enum:ColimitsInSliceGeneral} follows from \cref{lem:SliceAdjunction}, using $\Fun(\Ii,\Dd_{y/})\simeq \Fun(\Ii,\Dd)_{\const y/}$. To prove \cref{enum:ColimitsInSlice}, first assume that $\Dd$ has all colimits. Then the assumptions from \cref{enum:ColimitsInSliceGeneral} are satisfied and $\colimit_{i\in\Ii}\alpha(i)$ exists. If $\abs{\Ii}\simeq *$, then \cref{lem:ContractibleColimit} below implies that the canonical morphism $\colimit_{i\in\Ii}y\rightarrow y$ is an equivalence. Hence the pushout from \cref{enum:ColimitsInSliceGeneral} becomes an equivalence $\colimit_{i\in\Ii}\ov\alpha(i)\simeq c$. This proves \cref{enum:ColimitsInSlice} in the case where $\Dd$ has all colimits. For the general case, choose a fully faithful colimits-preserving functor $i\colon \Dd\rightarrow\Dd'$ into an $\infty$-category $\Dd$ with all colimits; for example, the mutilated Yoneda embedding $(\Yo_{\Dd^\op})^\op\colon (\Dd^\op)^\op\rightarrow\Fun(\Dd,\cat{An})^\op$ does it by \cref{cor:HomPreservesLimits}.
	%
	%, we use a general fact: If $L\colon \Cc\shortdoublelrmorphism\Dd\noloc R$ is an adjunction of $\infty$-categories and the counit $c_y\colon LR(y)\rightarrow y$ is an equivalence, then we get an induced adjunction $L\colon \Cc_{R(y)/}\shortdoublelrmorphism \Dd_{y/}\noloc R$ on slice $\infty$-categories. The proof is another straightforward application of \cref{cor:HomInSliceCategories}.
	%
	%First assume that $\Dd$ has $\Ii$-shaped colimits. Applying the general fact to the adjunction $\colimit_\Ii\colon \Fun(\Ii,\Dd)\shortdoublelrmorphism \Dd\noloc \const$ and using $\Fun(\Ii,\Dd_{y/})\simeq \Fun(\Ii,\Dd)_{\const y/}$, we see that it suffices to check $\colimit_\Ii\const y\simeq y$. This follows from \cref{lem:ContractibleColimit} below. This settles the case where $\Dd$ has $\Ii$-shaped colimits. For the general case, choose a fully faithful colimits-preserving functor $i\colon \Dd\rightarrow\Dd'$ into an $\infty$-category $\Dd$ with all colimits; for example, the mutilated Yoneda embedding $(\Yo_{\Dd^\op})^\op\colon (\Dd^\op)^\op\rightarrow\Fun(\Dd,\cat{An})^\op$ does it by \cref{cor:HomPreservesLimits}.
\end{proof}
\begin{lem}\label{lem:ContractibleColimit}
	Let $\Dd$ be an $\infty$-category, $y\in\Dd$ an object, and $\Ii$ be an $\infty$-category satisfying $\abs*{\Ii}\simeq *$. Then $\colimit_{i\in\Ii} y\simeq y$; in particular, this colimit always exists.
\end{lem}
\begin{proof}
	Clearly, $\const y\colon \Ii\rightarrow \Dd$ factors through $\Ii\rightarrow\abs*{\Ii}$. This functor is coinitial by \cref{exm:Cofinal}\cref{enum:LocalisationsCofinal}, and since $ \abs*{\Ii}\simeq *$, it follows that the colimit is indeed given by $y$. 
\end{proof}
The following lemma is the crucial step in the proof of \cref{lem:FilteredColimitsPreserveFiniteLimits}:
\begin{lem}\label{lem:HomotopyGroupsFilteredColimits}
	The functor $\pi_0\colon \cat{An}\rightarrow \cat{Set}$ commutes with products and all colimits. The functors $\pi_1\colon \cat{An}_{*/}\rightarrow\cat{Grp}$, and $\pi_n\colon\cat{An}_{*/}\rightarrow\cat{Ab}$ for all $n\geqslant 2$ commute with products and filtered colimits.
\end{lem}
\begin{proof}[Proof sketch]
	Since $\pi_0\colon\cat{An}\rightarrow\cat{Set}$ is left adjoint to the inclusion $\cat{Set}\subseteq \cat{An}$, \cref{lem:AdjointsPreserveColimits} shows that $\pi_0$ preserves colimits. By a simple inspection $\pi_0$ also preserves products. This immediately implies that $\pi_n$ preserves products for all $n\geqslant 1$, since $\pi_n(X,x)\cong \pi_0\Hom_{\cat{An}_{*/}}((S^n,*),(X,x))$ and $\Hom_{\cat{An}_{*/}}((S^n,*),-)\colon \cat{An}_{*/}\rightarrow \cat{An}$ preserves limits by \cref{cor:HomPreservesLimits}.
	
	The assertion about $\pi_n$ needs simplicial methods (and two black boxes), unfortunately. Let $\Jj$ be a filtered $\infty$-category. For every ordinary category $\Cc$, we have $\Fun(\Jj,\Cc)\simeq \Fun(\operatorname{ho}(\Jj),\Cc)$ by \cref{lem:SimplicialHoNerveAdjunction}, and so $\Jj$-shaped colimits in $\Cc$ agree with $\operatorname{ho}(\Jj)$-shaped colimits. It's straightforward to see that $\operatorname{ho}(\Jj)$ is filtered in the usual sense. So for filtered colimits in an ordinary category we can replace the indexing diagram by an ordinary filtered category. But a stronger assertion is true, which we'll need later:
	\begin{alphanumerate}\itshape
		\item[\blacksquare_1] For every filtered $\infty$-category there exists a directed partially ordered set $J$ and a coinitial functor $J\rightarrow \Jj$.\label{blackbox:Cofinal}
	\end{alphanumerate}
	For a proof of \cref{blackbox:Cofinal} see \cite[Proposition~\HTTthm{5.3.1.18}]{HTT} or \cite[Tag~\href{https://kerodon.net/tag/02QA}{02QA}]{Kerodon} (the Kerodon proof is relatively short and only uses methods that we have already available).
	
	Next, observe that $\pi_1\colon \cat{Kan}_{*/}\rightarrow\cat{Grp}$ and $\pi_n\colon \cat{Kan}_{*/}\rightarrow \cat{Ab}$ for $n\geqslant 2$ commute with filtered colimits in the ordinary category $\cat{Kan}_{*/}$. This follows essentially from the fact that $\square^n$ and $\partial\square^n$ are finite simplicial sets, using an argument as near the end of the proof of \cref{lem:SmallObjectArgument}. It follows that for every filtered $\infty$-category $\Jj$, the functor $\colimit\colon \Fun(\Jj,\cat{Kan})\rightarrow\cat{Kan}$ sends pointwise homotopy equivalences to homotopy equivalences. Indeed, let $X_{(-)}\Rightarrow X'_{(-)}$ be a natural transformation in $\Fun(\Jj,\cat{Kan})$ such that $X_j\rightarrow X'_j$ is a homotopy equivalence for all $j\in\Jj$. We can check on homotopy groups whether $\colimit_{j\in\Jj}X_j\rightarrow\colimit_{j\in\Jj}X_j$ is a homotopy equivalence. By the argument above, we get a bijection on $\pi_0$. Now let $x\in \pi_0(\colimit_{j\in\Jj}X_j)$ be a point. Since $\pi_0$ commutes with colimits, we must have $x\in \pi_0(X_{j_0})$ for some $j_0\in \Jj$. By \cref{lem:FilteredCofinal}, we may replace $\Jj$ by $\Jj_{j_0/}$, so we may assume $j_0$ is initial in $\Jj$. Then $\{x\}\rightarrow X_{j_0}\rightarrow X_j$ for all $j\in \Jj$ turns $X_{(-)}$ into a functor $(X_{(-)},x)\colon \Jj\rightarrow\cat{Kan}_{*/}$. The same works for $X'_{(-)}$. Since $X_{(-)}\Rightarrow X'_{(-)}$ is a pointwise homotopy equivalence and $\pi_n$ commutes with filtered colimits in $\cat{Kan}_{*/}$, we conclude that $\pi_n(\colimit_{j\in\Jj}X_j,x)\cong \pi_n(\colimit_{j\in\Jj}X_j',x)$. This finishes the proof that $\colimit\colon \Fun(\Jj,\cat{Kan})\rightarrow\cat{Kan}$ sends pointwise homotopy equivalences to homotopy equivalences. At this point, we need the second black box:
	\begin{alphanumerate}\itshape
		\item[\blacksquare_2] If $J$ is a directed partially ordered set, then there is an equivalence of $\infty$-categories\label{blackbox:Localisation}
		\begin{equation*}
			\Fun(J,\cat{Kan})\left[\{\text{pointwise homotopy equivalences}\}^{-1}\right]\overset{\simeq}{\longrightarrow}\Fun(J,\cat{An})\,.
		\end{equation*}
	\end{alphanumerate}
	The proof of \cref{blackbox:Localisation} is similar to that of \cref{thm:AnAsALocalisation}: First, one defines a simplicial model structure on $\Fun(J,\cat{sSet})$ in such a way that $\N^\Delta((\Fun(J,\cat{sSet})^\Delta)^\mathrm{cf})\simeq \Fun(J,\cat{An})$.  
	In the proof of \cite[Proposition~\HTTthm{5.3.3.3}]{HTT}, Lurie explains how to do this. Then one uses \cref{rem:SimplicialModelCategory,rem:ModelCategoryUnderlyingInftyCategory} to identify the simplicial nerve $\N^\Delta(\Fun(J,\cat{sSet})_\Delta^\mathrm{cf})$ with the localisation above.
	
	Now let $p\colon\cat{Kan}\rightarrow\cat{An}$ and $p_J\colon \Fun(J,\cat{Kan})\rightarrow\Fun(J,\cat{An})$ denote the canonical functors. By \cref{thm:AnAsALocalisation} and \cref{blackbox:Localisation}, both $p$ and $p_J$ are localisations. As we've seen above, $\colimit\colon \Fun(J,\cat{Kan})\rightarrow\cat{Kan}$ sends pointwise homotopy equivalences to homotopy equivalences. Hence $p\circ \colimit\colon \Fun(J,\cat{Kan})\rightarrow \cat{An}$ factors uniquely through the localisation $p_J$ by \cref{lem:Localisation}. Let $c\colon \Fun(J,\cat{An})\rightarrow\cat{An}$ be the induced functor; we claim that $c$ is simply the colimit functor. To this end, consider $\const\colon \cat{Kan}\rightarrow\Fun(J,\cat{Kan})$; since it sends homotopy equivalences to pointwise homotopy equivalences, the same argument as above shows that $p_J\circ \const$ factors uniquely through the localisation $p$. That factorisation is necessarily $\const\colon \cat{An}\rightarrow\Fun(J,\cat{An})$. It's straightforward to verify that the adjunction $\colimit\colon \Fun(J,\cat{Kan})\shortdoublelrmorphism \cat{Kan}\noloc \const$ descends to an adjunction $c\colon \Fun(J,\cat{An})\shortdoublelrmorphism \cat{An}\noloc \const$ on the localisations. Indeed, one can show using \cref{lem:Localisation} that the unit and counit transformations as well as the triangle identities get inherited, so we may appeal to \cref{lem:TriangleIdentities}. This shows that $c$ is left adjoint to $\const$, hence it must be the colimit functor, as claimed.
	
	Finally, we can finish the proof. Let $(X_{(-)},x_{(-)})\colon\Jj\rightarrow \cat{An}_{*/}$ be a functor from a filtered $\infty$-category into pointed animae. By \cref{blackbox:Cofinal}, we may assume that $\Jj\simeq J$ is a directed partially ordered set. By \cref{lem:FilteredCofinal}, we may assume that $J$ contains an initial object $j_0$. Then for every $(X_j,x_j)$, the point $\{x_j\}\rightarrow X_j$ agrees with $\{x_{j_0}\}\rightarrow X_{j_0}\rightarrow X_j$. Since $\cat{An}_{*/}\rightarrow\cat{An}$ preserves filtered colimits by \cref{lem:ColimitsInSliceCategory}, it follows that the pointed anima $\colimit_{j\in J}(X_j,x_j)$ is given by the unpointed colimit $\colimit_{j\in J}X_j$ together with the point $x_{j_0}\rightarrow X_{j_0}\rightarrow \colimit_{j\in J}X_j$. By \cref{blackbox:Localisation}, $\Fun(J,\cat{Kan})\rightarrow\Fun(J,\cat{An})$ is essentially surjective. So we may assume that $X_{(-)}$ comes from a functor $X_{(-)}\colon J\rightarrow\cat{Kan}$. As argued above, we may then as well take the colimit in $\cat{Kan}$ instead of $\cat{An}$. So the fact that $\pi_n\colon \cat{An}_{*/}\rightarrow \cat{Set}$ preserves filtered colimits reduces to the same assertion about $\pi_n\colon\cat{Kan}_{*/}\rightarrow\cat{Set}$, which we already know.
\end{proof}
\begin{proof}[Proof of \cref{lem:FilteredColimitsPreserveFiniteLimits}]
	First assume $\Jj$ is $\kappa$-filtered. By \cref{lem:KappaSmallColimits}, it's enough to show that $\colimit\colon\Fun(\Jj,\cat{An})\rightarrow\cat{An}$ preserves pullbacks and $\kappa$-small products. Using \cref{lem:LongExactFibrationSequence} and the five lemma (plus \cref{rem:ExactnessInLowDegrees}), we can further reduce pullbacks to fibre sequences (in the sense of \cref{def:Cofibre}).
	
	Let's do $\kappa$-small products first. We have to show that for every set $I$ of cardinality $<\kappa$, every $\kappa$-filtered $\infty$-category $\Jj$, and every functor $X_{(-,-)}\colon I\times\Jj\rightarrow\cat{An}$, the natural map
	\begin{equation*}
		\colimit_{j\in\Jj}\prod_{i\in I}X_{i,j}\longrightarrow\prod_{i\in I}\colimit_{j\in\Jj}X_{i,j}
	\end{equation*}
	is an equivalence. This can be checked on homotopy groups. We get a bijection on $\pi_0$, since $\pi_0\colon\cat{An}\rightarrow\cat{Set}$ preserves products and colimits and in $\cat{Set}$, $\kappa$-filtered colimits commute with $\kappa$-small products. For higher homotopy groups, fix some $x\in \pi_0\bigl(\colimit_{j\in\Jj}\prod_{i\in I}X_{i,j}\bigr)$. Since $\pi_0$ commutes with colimits, we must have $x\in \pi_0\bigl(\prod_{i\in I}X_{i,j_0}\bigr)$ for some $j_0\in \Jj$. By \cref{lem:FilteredCofinal}, we may replace $\Jj$ by $\Jj_{j_0/}$ to assume that $j_0$ is initial in $\Jj$. In this case, the composition $\{x\}\rightarrow \prod_{i\in I}X_{i,j_0}\rightarrow X_{i,j_0}\rightarrow X_{i,j}$ for all $(i,j)\in I\times\Jj$ turns $X_{(-,-)}$ into a functor $X_{(-,-)}\colon I\times\Jj\rightarrow\cat{An}_{*/}$. Then $\pi_n\bigl(\colimit_{j\in\Jj}\prod_{i\in I}X_{i,j},x\bigr)\cong\pi_n\bigl(\prod_{i\in I}\colimit_{j\in\Jj}X_{i,j},x\bigr)$ follows from \cref{lem:HomotopyGroupsFilteredColimits} and the fact that $\kappa$-filtered colimits in $\cat{Grp}$ or $\cat{Ab}$ commute with $\kappa$-small products.%; the latter is straightforward to verify, observing that $\operatorname{ho}(\Jj)$ is $\kappa$-filtered in the ordinary sense.
	
	The case of fibre sequences is similar. Let $F_{(-)}\Rightarrow X_{(-)}\Rightarrow Y_{(-)}$ be a fibre sequence in $\Fun(\Jj,\cat{An})$; by \cref{lem:ColimitsInFunctorCategories}, this is equivalent to $F_j\rightarrow X_j\rightarrow Y_j$ being a fibre sequences for every $j\in\Jj$. We must show that
	\begin{equation*}
		\colimit_{j\in \Jj}F_j\longrightarrow\fib\Bigl(\colimit_{j\in\Jj}X_j\rightarrow \colimit_{j\in\Jj}Y_j\Bigr)
	\end{equation*}
	is an equivalence. This follows from a comparison of long exact sequences, using \cref{lem:LongExactFibrationSequence} and the five lemma together with the fact that filtered colimits preserve exact sequences. This finishes the proof that $\colimit\colon \Fun(\Jj,\cat{An})\rightarrow\cat{An}$ preserves $\kappa$-small limits.
	
	Conversely, assume this is the case; we must show that $\Jj$ is $\kappa$-filtered. Let $\alpha\colon \Ii\rightarrow \Jj$ be a functor from a essentially $\kappa$-small $\infty$-category. Consider the composition
	\begin{equation*}
		\Ii^\op\xrightarrow{\alpha^\op} \Jj^\op\xrightarrow{\Yo_{\Jj^{\op}}} \Fun(\Jj,\cat{An})\,.
	\end{equation*}
	Since $\Fun(\Jj,\cat{An})$ has all limits by \cref{lem:ColimitsInFunctorCategories}, we can put $E\coloneqq \lim(\Ii^\op\rightarrow \Fun(\Jj,\cat{An}))$ and extend the functor above to a limit cone $(\alpha^\op)^\triangleleft\colon (\Ii^\op)^\triangleleft\rightarrow\Fun(\Jj,\cat{An})$. Suppose there is an object $j\in\Jj^\op$ together with a natural transformation $\eta\colon \Hom_\Jj(j,-)\Rightarrow E$ in $\Fun(\Jj,\cat{An})$. We may view $(\alpha^\op)^\triangleleft$ as a natural transformation $\const E\Rightarrow \Yo_{\Jj^\op}\circ\alpha^\op$. Composing with $\const\eta\colon \const \Hom_\Jj(j,-)\Rightarrow \const E$ yields another natural transformation, which we may again view as a functor $(\beta^\op)^\triangleleft\colon (\Ii^\op)^\triangleleft\rightarrow\Fun(\Jj,\cat{An})$. Then $(\beta^\op)^\triangleleft$ lands in the essential image of $\Yo_{\Jj^\op}$. Since the Yoneda embedding is fully faithful by \cref{cor:YonedaEmbeddingFullyFaithful}, we obtain a functor $\beta^\triangleright\colon \Ii^\triangleright\rightarrow\Jj$, as desired.
	
	So assume on the contrary that there exists no $\eta\colon \Hom_\Jj(j,-)\Rightarrow E$ as above. By Yoneda's lemma, \cref{thm:Yoneda}, this implies $E(j)\simeq \emptyset$ for all $j\in \Jj$. Since initial objects are preserved under arbitrary colimits, $\colimit_{j\in\Jj}E(j)\simeq \emptyset$. On the other hand, \cref{lem:ColimitsInAnima} implies $\colimit_{j\in\Jj}\Hom_\Jj(j_0,j)\simeq \abs{\Jj_{j_0/}}\simeq*$ for every $j_0\in\Jj$. Since $\colimit\colon \Fun(\Jj,\cat{An})\rightarrow\cat{An}$ preserves $\kappa$-small limits by assumption, it follows that $\colimit_{j\in\Jj}E(j)\simeq \limit_{i\in\Ii^\op}*\simeq *$, as terminal objects are preserved under arbitrary limits. Since $\emptyset\not\simeq *$, we get a contradiction.
\end{proof}

%This finishes our lengthy discussion of filtered colimits in $\infty$-categories. Next, we'll introduce $\cat{Ind}$-categories in the world of $\infty$-categories and we'll finally define what a presentable $\infty$-category is supposed to be.

\subsection{Accessible and presentable \texorpdfstring{$\infty$}{Infinity}-categories}\label{subsec:Presentable}

We can now introduce a class of large $\infty$-categories that are generated by a small sub-$\infty$-category.

\begin{con}\label{con:Ind}
	Let $\kappa$ be a regular cardinal and let $\Cc$ be an essentially small $\infty$-category. We let $\cat{Ind}_\kappa(\Cc)\subseteq \PSh(\Cc)$ be the full sub-$\infty$-category spanned by those presheaves $E\colon \Cc^\op\rightarrow\cat{An}$ for which the unstraightening $\operatorname{Un}^{(\mathrm{right})}(E)$ is $\kappa$-filtered. In the case $\kappa=\aleph_0$, we often write $\cat{Ind}(\Cc)\coloneqq\cat{Ind}_{\aleph_0}(\Cc)$.
	
	Note that the Yoneda embedding $\Yo_\Cc\colon \Cc\rightarrow\PSh(\Cc)$ factors through $\cat{Ind}_\kappa(\Cc)$. Indeed, for every $x\in\Cc$, the unstraightening of $\Hom_\Cc(-,x)\colon \Cc^\op\rightarrow\Cc$ is the right fibration $\Cc_{/x}\rightarrow\Cc$ by the dual of \cref{exm:Straightening}\cref{enum:SliceLeftFibration}. Now $\Cc_{/x}$ has a terminal object $\id_x\colon x\rightarrow x$, hence it is $\kappa$-filtered for any $\kappa$. Indeed, composing any functor $\Ii\rightarrow\Cc_{/x}$ with $\id_{\Cc_{/x}}\Rightarrow \const \id_x$ yields an extension $\Ii^\triangleright\rightarrow\Cc_{/x}$, as desired. Alternatively, we could have used \cref{lem:FilteredColimitsPreserveFiniteLimits}: Since $\Cc_{/x}$ has a terminal object, every colimit over $\Cc_{/x}$ is just given by evaluating at that object. Therefore, it follows from \cref{lem:ColimitsInFunctorCategories} that $\colimit\colon \Fun(\Cc_{/x},\cat{An})\rightarrow \cat{An}$ preserves arbitrary limits. We'll denote the factorisation of $\Yo_\Cc$ by
	\begin{equation*}
		\Yo_\Cc^\kappa\colon \Cc\longrightarrow\cat{Ind}_\kappa(\Cc)\,.
	\end{equation*}
	If no confusion can occur, we'll usually drop the superscript and just write $\Yo_\Cc$.
	
	More generally, we have $\operatorname{Un}^{(\mathrm{right})}(E)\simeq \Cc_{/E}$ for all $E\in \PSh(\Cc)$, so $E$ is contained in $\cat{Ind}_\kappa(\Cc)$ if and only if $\Cc_{/E}$ is filtered. Indeed, the right fibration $\PSh(\Cc)_{/E}\rightarrow\PSh(\Cc)$ is the unstraightening of $\Hom_{\PSh(\Cc)}(-,E)\colon \PSh(\Cc)^\op\rightarrow\cat{An}$. By Yoneda's lemma (combined with \cref{par:YonedaFunctorial}), we have an equivalence $E\simeq \Hom_{\PSh(\Cc)}(\Yo_\Cc(-),E)$ of presheaves. Hence the unstraightening of $E$ is the pullback of $\PSh(\Cc)_{/E}\rightarrow\PSh(\Cc)$ along $\Yo_\Cc\colon \Cc\rightarrow\PSh(\Cc)$, which is $\Cc_{/E}$.
\end{con}
\begin{defi}\label{def:Presentable}
	Let $\kappa$ be a regular cardinal. A \embrace{not necessarily essentially small} $\infty$-category $\Cc$ is \emph{$\kappa$-accessible} if $\Cc\simeq\cat{Ind}_\kappa(\Cc_0)$ for some small $\infty$-category $\Cc_0$. We call $\Cc$ \emph{accessible} if it is $\kappa$-accessible for some regular cardinal $\kappa$. We call $\Cc$ \emph{presentable} if it is accessible and has all colimits.
\end{defi}
%We'll now prove several useful properties and characterisations of accessible $\infty$-categories, including an analogue of \cref{thm:PShFreeCocompletion}.%After that, we'll spend a few pages studying presentability before we finally get to the adjoint functor theorem.
\begin{lem}\label{lem:Ind}
	Let $\kappa$ be a regular cardinal and let $\Cc$ be a small $\infty$-category.
	\begin{alphanumerate}
		\item A presheaf $E\in\PSh(\Cc)$ belongs to $\cat{Ind}_\kappa(\Cc)$ if and only if $E$ can be written as a $\kappa$-filtered colimit of representable presheaves. Furthermore, $\cat{Ind}_\kappa(\Cc)\subseteq \PSh(\Cc)$ is closed under $\kappa$-filtered colimits.\label{enum:IndGeneratedUnderFilteredColimits}
		\item If $\Cc$ has $\kappa$-small colimits, then a presheaf $E\in\PSh(\Cc)$ belongs to $\cat{Ind}_\kappa(\Cc)$ if and only if $E\colon \Cc^\op\rightarrow\cat{An}$ preserves $\kappa$-small limits. \label{enum:IndLimits}
		\item If $\Dd$ is an $\infty$-category which has all $\kappa$-filtered colimits, then restriction along the Yoneda embedding induces an equivalence\label{enum:IndFreelyGenerated}
		\begin{equation*}
			\Yo_\Cc^*\colon \Fun^{\kappa\mhyph\mathrm{filt}}\bigl(\cat{Ind}_\kappa(\Cc),\Dd\bigr)\overset{\simeq}{\longrightarrow}\Fun(\Cc,\Dd)\,.
		\end{equation*}
		Here $\Fun^{\kappa\mhyph\mathrm{filt}}(\cat{Ind}_\kappa(\Cc),\Dd)\subseteq\Fun(\cat{Ind}_\kappa(\Cc),\Dd)$ is spanned by those functors that preserve $\kappa$-filtered colimits.
	\end{alphanumerate}
\end{lem}
\begin{proof}
	We begin with \cref{enum:IndGeneratedUnderFilteredColimits}. By \cref{lem:PresheafColimitOfRepresentables}, every presheaf $E$ can be written as a colimit of representables, with $\Cc_{/E}$ as indexing $\infty$-category. If $E\in\cat{Ind}_\kappa(\Cc)$, then $\Cc_{/E}$ is $\kappa$-filtered by \cref{con:Ind}, hence $E$ is a $\kappa$-filtered colimit of representables. Conversely, assume $E$ can be written as such a $\kappa$-filtered colimit, say, $E\simeq\colimit_{j\in\Jj}\Hom_\Cc(-,x_j)$. Since the unstraightening $\operatorname{Un}^{(\mathrm{right})}\colon \PSh(\Cc)\rightarrow\cat{Right}(\Cc)$ is an equivalence of $\infty$-categories, it preserves colimits. Recall from \cref{lem:KanExtensionForRight}\cref{enum:RightCofinalLeftAdjoint} that the inclusion $\cat{Right}(\Cc)\subseteq\Cat_{\infty/\Cc}$ has a left adjoint $c\colon \Cat_{\infty/\Cc}\rightarrow\cat{Right}(\Cc)$. Furthermore, by the dual of \cref{lem:ColimitsInSliceCategory}, $\cat{Cat}_{\infty/\Cc}\rightarrow\operatorname{Cat}_\infty$ preserves colimits. Hence a colimit in $\cat{Right}(\Cc)$ is computed by taking the colimit in $\cat{Cat}_\infty$ and then applying $c$. Therefore
	\begin{equation*}
		\operatorname{Un}^{(\mathrm{right})}(E)\simeq c\Bigl(\colimit_{j\in\Jj}\Cc_{/x_j}\Bigr)\,.
	\end{equation*}
	Now a $\kappa$-filtered colimit of $\kappa$-filtered $\infty$-categories is $\kappa$-filtered again, which follows by combining \cref{lem:ColimitManipulations}\cref{claim:AssembleColimits} with the characterisation of $\kappa$-filteredness from \cref{lem:FilteredColimitsPreserveFiniteLimits}. So $\colimit_{j\in\Jj}\Cc_{/x_j}$ is $\kappa$-filtered. By \cref{lem:FilteredColimitsPreserveFiniteLimits} again it's clear that being $\kappa$-filtered is preserved under coinitial functors. Since $\colimit_{j\in\Jj}\Cc_{/x_j}\rightarrow c(\colimit_{j\in\Jj}\Cc_{/x_j})$ is coinitial by \cref{lem:KanExtensionForRight}\cref{enum:RightCofinalLeftAdjoint}, we've shown that $\operatorname{Un}^{(\mathrm{right})}(E)$ is $\kappa$-filtered, as desired. The same argument shows that $\cat{Ind}_\kappa(\Cc)\subseteq \PSh(\Cc)$ is closed under $\kappa$-filtered colimits.
	
	For \cref{enum:IndLimits}, let's temporarily denote $\PSh^\kappa(\Cc)\subseteq\PSh(\Cc)$ the full sub-$\infty$-category of presheaves $E\colon \Cc^\op\rightarrow\cat{An}$ that preserve $\kappa$-small limits. Every representable presheaf preserves all limits by \cref{cor:HomPreservesLimits}, in particular, $\kappa$-small ones. By \cref{lem:FilteredColimitsPreserveFiniteLimits}, $\PSh^\kappa(\Cc)\subseteq \PSh(\Cc)$ is stable under $\kappa$-filtered colimits. By \cref{enum:IndGeneratedUnderFilteredColimits}, every $E\in\cat{Ind}_\kappa(\Cc)$ is a $\kappa$-filtered colimit of representables, hence $E\in \PSh^\kappa(\Cc)$. Conversely, assume $E\in\PSh^\kappa(\Cc)$. To show that $\Cc_{/E}$ is $\kappa$-filtered, we claim:
	\begin{alphanumerate}\itshape
		\item[\boxtimes] The restricted Yoneda embedding $\Yo_\Cc\colon \Cc\rightarrow\PSh^\kappa(\Cc)$ preserves $\kappa$-small colimits.\footnote{Note that this is completely false for the unrestricted Yoneda embedding: $\Yo_\Cc\colon \Cc\rightarrow \PSh(\Cc)$ preserves limits (by \cref{cor:HomPreservesLimits}), but not colimits. That is why, whenever we want to choose a fully faithful colimits-preserving functor $i\colon \Cc\rightarrow \Cc'$ into an $\infty$-category with all colimits, we have to take the awkward construction $(\Yo_{\Cc^\op})^\op\colon (\Cc^\op)^\op\rightarrow \PSh(\Cc^\op)^\op$.}\label{claim:YonedaPreservesColimits}
	\end{alphanumerate}
	If $\alpha\colon\Ii\rightarrow \Cc_{/E}$ is any functor from an essentially $\kappa$-small $\infty$-category, and $\ov\alpha\colon \Ii\rightarrow\Cc_{/E}\rightarrow\Cc$ denotes the composition of $\alpha$ with the projection to $\Cc$, then $x\coloneqq \colimit(\ov\alpha\colon\Ii\rightarrow\Cc)$ exists by assumption on $\Cc$. Now $\alpha$ corresponds to a natural transformation $\eta\colon\Yo_\Cc\circ\ov\alpha\Rightarrow\const E$ in $\Fun(\Ii,\PSh(\Cc))$. If \cref{claim:YonedaPreservesColimits} holds, then we can use the universal property of colimits to show that $\eta$ factors uniquely through $\Yo_\Cc\circ\ov\alpha\Rightarrow\const\Yo_\Cc(x)$. Since $\Yo_\Cc$ is fully faithful, this yields an extension $\alpha^\triangleright\colon \Ii\rightarrow\Cc_{/E}$, as desired.
	
	To prove \cref{claim:YonedaPreservesColimits}, let $x_{(-)}\colon \Ii\rightarrow\Cc$ be a functor from an essentially $\kappa$-small $\infty$-category $\Cc$ and let $E\in\PSh^\kappa(\Cc)$. Then $\Hom_{\PSh(\Cc)}(\Yo_\Cc(\colimit_{i\in\Ii}x_i),E)\simeq E(\colimit_{i\in\Ii}x_i)$ by Yoneda's lemma. Since $E$ preserves $\kappa$-small limits (and limits in $\Cc^\op$ correspond to colimits in $\Cc$), we can use Yoneda's lemma again to see
	\begin{equation*}
		E\Bigl(\colimit_{i\in\Ii}x_i\Bigr)\simeq \limit_{i\in\Ii}E(x_i)\simeq \limit_{i\in\Ii}\Hom_{\PSh(\Cc)}\bigl(\Yo_\Cc(x_i),E\bigr)\,.
	\end{equation*}
	Using \cref{cor:HomPreservesColimits}, this proves \cref{claim:YonedaPreservesColimits}.
	
	To prove \cref{enum:IndFreelyGenerated}, we can more or less copy the proof of \cref{thm:PShFreeCocompletion}: Let $F\colon \Cc\rightarrow\Dd$ be any functor. Since $\Dd$ has filtered colimits and $\Cc_{/E}$ is filtered for every $E\in\cat{Ind}_\kappa(\Cc)$, the Kan extension $\Lan_{\Yo_\Cc^\kappa}F\colon \cat{Ind}_\kappa(\Cc)\rightarrow\Dd$ exists by \cref{lem:KanExtensionFormula}. We must show that the Kan extension $\Lan_{\Yo_\Cc^\kappa}F$ preserves $\kappa$-filtered colimits. Let's first assume that $\Dd$ has all colimits. Consider the Kan extension $\Lan_{\Yo_\Cc}F\colon \PSh(\Cc)\rightarrow\Dd$. For formal reasons, $\Lan_{\Yo_\Cc}F$ is the left Kan extension of $\Lan_{\Yo_\Cc^\kappa}F$ along $\cat{Ind}_\kappa(\Cc)\subseteq\PSh(\Cc)$. Since the latter is fully faithful, \cref{cor:KanExtensionAlongFullyFaithful} shows $\Lan_{\Yo_\Cc^\kappa}F\simeq(\Lan_{\Yo_\Cc}F)|_{\cat{Ind}_\kappa(\Cc)}$. Now $\Lan_{\Yo_\Cc}F\colon\PSh(\Cc)\rightarrow\Dd$ preserves colimits by \cref{lem:LanAlongYonedaHasRightAdjoint} and $\cat{Ind}_\kappa(\Cc)\subseteq \PSh(\Cc)$ preserves $\kappa$-filtered colimits by \cref{enum:IndGeneratedUnderFilteredColimits}, so $\Lan_{\Yo_\Cc^\kappa}F$ preserves $\kappa$-filtered colimits, as desired. This concludes the case where $\Dd$ has all colimits. The general case can be reduced to this as follows: As in the proof of \cref{lem:ColimitsInSliceCategory}, we can choose a fully faithful colimits-preserving functor $i\colon \Dd\rightarrow\Dd'$ into an $\infty$-category with all colimits. The formula from \cref{lem:KanExtensionFormula} combined with \cref{thm:EquivalencePointwise} show that the canonical natural transformation $\Lan_{\Yo_\Cc^\kappa}(i\circ F)\Rightarrow i\circ \Lan_{\Yo_\Cc^\kappa}F$ is an equivalence, and so it suffices that $\Lan_{\Yo_\Cc^\kappa}(i\circ F)$ preserves $\kappa$-filtered colimits, which we did above.
	
	So the Kan extension functor $\Lan_{\Yo_\Cc^\kappa}\colon \Fun(\Cc,\Dd)\rightarrow\Fun(\cat{Ind}_\kappa(\Cc),\Dd)$ lands in the full sub-$\infty$-category $\Fun^\kappa(\cat{Ind}_\kappa(\Cc),\Dd)$. Therefore, we obtain an adjunction
	\begin{equation*}
		\Lan_{\Yo_\Cc}\colon \Fun(\Cc,\Dd)\doublelrmorphism \Fun^{\kappa\mhyph\mathrm{filt}}\bigl(\cat{Ind}_\kappa(\Cc),\Dd\bigr)\noloc \Yo_\Cc^*
	\end{equation*}
	By the same arguments as in the proof of \cref{thm:PShFreeCocompletion}, the unit $u\colon \id_{\Fun(\Cc,\Dd)}\Rightarrow\Yo_\Cc^*\circ \Lan_{\Yo_\Cc}$ is an equivalence and $\Yo_\Cc^*$ is conservative, hence the adjunction above is a pair of inverse equivalences by \cref{lem:FullyFaithfulConservativeAdjunction}.
\end{proof}
It's surprisingly common for an $\infty$-category to be accessible.% in the sense of \cref{def:Presentable}.
\begin{lem}\label{lem:KappaCompactlyGenerated}
	Let $\kappa$ be a regular cardinal and let $\Dd$ be a locally small $\infty$-category. Then the following are equivalent:
	\begin{alphanumerate}
		\item $\Dd$ is of the form $\Dd\simeq \cat{Ind}_\kappa(\Cc)$ for some essentially small $\infty$-category $\Cc$.\label{enum:DIsIndC}
		\item $\Dd$ admits $\kappa$-filtered colimits and there exists a set $S$ of $\kappa$-compact objects such that every object from $\Dd$ can be written as a $\kappa$-filtered colimit of objects from $S$.\label{enum:DGeneratedUnderFilteredColimits}
	\end{alphanumerate}
	In this case automatically $\Dd\simeq \cat{Ind}_\kappa(\Dd^\kappa)$, where $\Dd^\kappa\subseteq \Dd$ is the full sub-$\infty$-category spanned by the $\kappa$-compact objects. Furthermore, if $\Dd$ has $\kappa$-small colimits, then there is another equivalent condition:
	\begin{alphanumerate}[resume]
		\item $\Dd$ admits $\kappa$-filtered colimits and has a set of $\kappa$-compact generators; that is, a set $S\subseteq \Dd$ of $\kappa$-compact objects such that $\Hom_\Dd(s,-)\colon \Dd\rightarrow\cat{An}$, $s\in S$, are jointly conservative.\label{enum:CompactGenerators}
	\end{alphanumerate}%
	In the case where $\Dd$ has $\kappa$-small colimits and \cref{enum:CompactGenerators} holds, $\Dd$ is automatically presentable.
\end{lem}
\begin{proof}
	Assume \cref{enum:DIsIndC} is true. Then $\Dd$ has $\kappa$-filtered colimits by \cref{lem:Ind}\cref{enum:IndGeneratedUnderFilteredColimits}. We claim that $S\coloneqq\{\Yo_\Cc(x)\ \vert\ x\in\Cc\}$ generates $\Dd$ under $\kappa$-filtered colimits and forms a set of compact generators in the sense of \cref{enum:CompactGenerators}. The first assertion is \cref{lem:Ind}\cref{enum:IndGeneratedUnderFilteredColimits}. For the second assertion, Yoneda's lemma says $\Hom_{\PSh(\Cc)}(\Yo_\Cc(x),E)\simeq E(x)$ for every $E\in\cat{Ind}_\kappa(\Cc)$. Hence $\Hom_{\PSh(\Cc)}(\Yo_\Cc(x),-)$ for $x\in\Cc$ are jointly conservative by \cref{thm:EquivalencePointwise}. Furthermore, $\Yo_\Cc(x)$ is $\kappa$-compact since colimits in presheaf $\infty$-categories are computed pointwise by \cref{lem:ColimitsInFunctorCategories} and $\cat{Ind}_\kappa(\Cc)\subseteq\PSh(\Cc)$ preserves $\kappa$-filtered colimits by \cref{lem:Ind}\cref{enum:IndGeneratedUnderFilteredColimits}. This proves \cref{enum:DIsIndC} $\Rightarrow$ \cref{enum:DGeneratedUnderFilteredColimits} and \cref{enum:DIsIndC} $\Rightarrow$ \cref{enum:CompactGenerators} (even without the assumption that $\Dd$ has $\kappa$-small colimits).
	
	Now assume \cref{enum:DGeneratedUnderFilteredColimits}. Let's first sketch why $\Dd^\kappa$ is essentially small. We'll show that every $x\in\Dd^\kappa$ is a retract of some $s\in S$ and then leave it to you to verify that $S$ can't have \enquote{too many} retracts in the locally small $\infty$-category $\Dd$. Write $x\simeq \colimit_{j\in\Jj}s_j$ for some $s_j\in S$ and some $\kappa$-filtered $\infty$-category $\Jj$. Since $x$ is $\kappa$-compact and $\pi_0$ commutes with colimits by \cref{lem:HomotopyGroupsFilteredColimits}, we get $\colimit_{j\in\Jj}\pi_0\Hom_\Cc(x,s_j)\cong \pi_0\Hom_\Dd(x,x)$. Choosing a preimage of $\id_x$ yields a morphism $x\rightarrow s_j$ for some $j\in\Jj$, which exhibits $x$ as a retract of $s_j$, as desired.
	
	By \cref{lem:Ind}\cref{enum:IndFreelyGenerated}, the inclusion $\Dd^\kappa\subseteq \Dd$ extends uniquely to a functor $L^\kappa\colon \cat{Ind}_\kappa(\Dd^\kappa)\rightarrow\Dd$ that preserves $\kappa$-filtered colimits. Let's first construct a right adjoint $R^\kappa$. To this end, choose a fully faithful colimits-preserving functor $i\colon \Dd\rightarrow\Dd'$ into an $\infty$-category $\Dd'$ with all colimits; this can be done as in the proof of \cref{lem:Ind}\cref{enum:IndFreelyGenerated}. Furthermore, $i\circ L^\kappa$ extends uniquely to a colimits-preserving functor $L\colon \PSh(\Dd^\kappa)\rightarrow \Dd'$, which has a right adjoint $R$ by \cref{thm:PShFreeCocompletion}. We claim that $R\circ i\colon \Dd\rightarrow\PSh(\Dd)$ lands in $\cat{Ind}_\kappa(\Dd^\kappa)$. Indeed, let $y\in\Dd$ and write $y\simeq \colimit_{j\in\Jj}x_j$ where $\Jj$ is $\kappa$-filtered and $x_j\in\Dd^\kappa$; we could even choose $x_j\in S$. By the formula from \cref{lem:LanAlongYonedaHasRightAdjoint}, $R(i(y))$ is the presheaf $\Hom_\Dd(-,\colimit_{j\in\Jj}x_j)\colon (\Dd^\kappa)^\op\rightarrow\cat{An}$. By definition of $\Dd^\kappa$, this presheaf agrees with $\colimit_{j\in\Jj}\Hom_\Dd(-,x_j)$. Hence $R(i(y))$ is a $\kappa$-filtered colimit of representable presheaves and thus contained in $\cat{Ind}_\kappa(\Dd^\kappa)$ by \cref{lem:Ind}\cref{enum:IndGeneratedUnderFilteredColimits}. Thus, putting $R^\kappa\coloneqq R\circ i$, we obtain the desired adjunction $L^\kappa\colon \cat{Ind}_\kappa(\Dd^\kappa)\shortdoublelrmorphism\Dd\noloc R^\kappa$. Moreover, our argument shows that $R^\kappa$ commutes with $\kappa$-filtered colimits of objects from $\Dd^\kappa$. By inspection, the counit $c\colon L^\kappa\circ R^\kappa\Rightarrow \id_\Dd$ is an equivalence for objects from $\Dd^\kappa$. Since both sides commute with $\kappa$-filtered colimits of objects from $\Dd^\kappa$ and every object of $\Dd$ can be written as such a colimit, we see that $c$ is an equivalence. An analogous argument shows that $u\colon \id_{\cat{Ind}_\kappa(\Dd^\kappa)}\Rightarrow R^\kappa\circ L^\kappa$ is an equivalence. This proves \cref{enum:DGeneratedUnderFilteredColimits} $\Rightarrow$ \cref{enum:DIsIndC}.
	
	It remains to show \cref{enum:CompactGenerators} $\Rightarrow$ \cref{enum:DIsIndC}. First observe that if $\Dd$ has $\kappa$-small and $\kappa$-filtered colimits, then $\Dd$ has all colimits. Indeed, according to \cref{lem:ColimitsIffCoproductsAndPushouts}, we only need to check that $\Dd$ has arbitrary coproducts. This follows from the following claim:
	\begin{alphanumerate}\itshape
		\item[\boxtimes] Let $T$ be a discrete set and let $\Pp^{\kappa}(T)\subseteq \Pp(T)$ be the partially ordered set of all subsets $S\subseteq T$ of cardinality $\abs*{S}<\kappa$. Then $\Pp^{\kappa}(T)$ is $\kappa$-filtered and for every collection $(x_t)_{t\in T}$ of objects of $\Dd$ we have\label{claim:FilteredCoproduct}
		\begin{equation*}
			\colimit_{S\in\Pp^{\kappa}(T)}\coprod_{s\in S}x_s\overset{\simeq}{\longrightarrow}\coprod_{t\in T}x_t\,.
		\end{equation*}
	\end{alphanumerate}
	Using \cref{lem:SimplicialHoNerveAdjunction}, $\kappa$-filteredness of $\Pp^\kappa(T)$ reduces to a question about ordinary categories, which is easy. Now consider the tautological functor $U\colon \Pp^\kappa(T)\rightarrow\cat{Set}$ sending $S\mapsto S$ and let $\Uu$ be its unstraightening, which is an ordinary category and easy to describe.\footnote{Here we use that for functors into $\cat{Set}$ or $\cat{Grpd}^{(2)}$, Lurie's unstraightening recovers the \emph{Grothendieck construction} from classical category theory. We've seen something similar in \cref{par:Stacks}; in particular, compare the description of the unstraightening of $U\colon \Pp^\kappa(T)\rightarrow \cat{Set}$ to the unstraightening of $[S/G]\colon (\cat{Sch}_{/S})^\op\rightarrow \cat{Grpd}^{(2)}$.} Namely, $\Uu$ is the category of pairs $(S,s)$, where $S\subseteq T$ is a subset and $s\in S$ is an element. Morphisms can be described as follows: Fix $(S,s)$ and $(S',s')$. If $S\subseteq S'$ and $s=s'$, there exists a unique morphism $(S,s)\rightarrow(S',s')$ in $\Uu$; otherwise $\Hom_\Uu((S,s),(S',s'))=\emptyset$. There is a tautological natural transformation $U\Rightarrow\const T$, which induces a functor $\Uu\rightarrow \Pp^\kappa(T)\times T$ on unstraightenings. Note that $\Uu\rightarrow \Pp^\kappa(T)\times T\rightarrow T$ is coinitial. Indeed, for every $t\in T$, the slice $\Uu\times_TT_{t/}$ can be identified with the sub-partially ordered set $\Pp_t^\kappa(T)\subseteq \Pp^\kappa(T)$ of those $S$ such that $t\in S$. Then $\Pp_t^\kappa(T)$ has an initial object, namely $\{t\}$, and so $\mathopen|\Pp_t^\kappa(T)\mathclose|\simeq *$, whence \cref{thm:JoyalsQuillenA}\cref{enum:WeaklyContractible} is satisfied. Thus, if $T\rightarrow \Dd$ corresponds to the collection $(x_t)_{t\in T}$, then $\colimit(T\rightarrow\Dd)\simeq \colimit(\Uu\rightarrow T\rightarrow\Dd)$. Using \cref{lem:ColimitManipulations}\cref{claim:SliceColimits}, the right-hand side can be identified with $\colimit_{S\in\Pp^\kappa(T)}\coprod_{s\in S}x_s$. This finishes the proof of \cref{claim:FilteredCoproduct}
	
	Now let $S\subseteq \Dd$ be a set of $\kappa$-compact generators and let $\Cc\subseteq \Dd$ be the full sub-$\infty$-category generated by $S$ under $\kappa$-small colimits. Since $\Dd$ is locally small, one can verify that $\Cc$ is essentially small (there are \enquote{not too many} $\kappa$-small diagrams); we leave this to you. Since $\Dd$ has all colimits, we can apply \cref{thm:PShFreeCocompletion} to see that $\Cc\subseteq\Dd$ extends uniquely to a colimits-preserving functor $L\colon \PSh(\Cc)\rightarrow\Dd$, which has a right adjoint $R$. Observe that $R$ factors through $\cat{Ind}_\kappa(\Cc)$. Indeed, according to \cref{lem:LanAlongYonedaHasRightAdjoint}, for every $y\in\Dd$, the presheaf $R(y)$ is given by $\Hom_\Dd(-,y)\colon\Cc^\op\rightarrow\cat{An}$. This functor preserves arbitrary limits by \cref{cor:HomPreservesLimits}, in particular, $\kappa$-small ones, and so \cref{lem:Ind}\cref{enum:IndLimits} implies $R(y)\in\cat{Ind}_\kappa(\Cc)$. Restricting $L$, we thus obtain an adjunction $L\colon\cat{Ind}_\kappa(\Cc)\shortdoublelrmorphism\Dd\noloc R$. Observe that $R$ preserves $\kappa$-filtered colimits. Indeed, let $y_{(-)}\colon \Jj\rightarrow\Dd$ be a functor from a $\kappa$-filtered $\infty$-category. By \cref{thm:EquivalencePointwise} and \cref{lem:ColimitsInFunctorCategories}, it suffices to show that $\colimit_{j\in\Jj}\Hom_\Dd(x,y_j)\rightarrow \Hom_\Dd(x,\colimit_{j\in\Jj}y_j)$ is an equivalence for all $x\in\Cc$. But \cref{lem:FilteredColimitsPreserveFiniteLimits} easily implies that $\kappa$-compact objects are closed under $\kappa$-small colimits and so $x$ must be $\kappa$-compact, whence we get an equivalence as desired. Now we can apply the same argument as in the proof of \cref{enum:DGeneratedUnderFilteredColimits} $\Rightarrow$ \cref{enum:DIsIndC} to show that the unit $u\colon \id_{\cat{Ind}_\kappa(\Cc)}\Rightarrow R\circ L$ is an equivalence. Furthermore, $R$ is conservative. Indeed, if $\alpha\colon y\rightarrow z$ in $\Dd$ induces an equivalence $\alpha_*\colon \Hom_\Dd(-,y)\Rightarrow \Hom_\Dd(-,z)$ of presheaves, then, in particular, $\alpha_*\colon\Hom_\Dd(s,y)\rightarrow\Hom_\Dd(s,z)$ must be an equivalence for all $s\in S$. But $\Hom_\Dd(s,-)\colon \Dd\rightarrow\cat{An}$ for $s\in S$ are jointly conservative by assumption. Now \cref{lem:FullyFaithfulConservativeAdjunction}\cref{enum:Conservative} finishes the proof of the implication \cref{enum:CompactGenerators} $\Rightarrow$ \cref{enum:DIsIndC}.
\end{proof}
This finishes our discussion of accessibility. Next, we'll characterise presentable $\infty$-categories.

\begin{lem}\label{lem:Presentable}
	For a locally small $\infty$-category $\Dd$, the following are equivalent:
	\begin{alphanumerate}
		\item $\Dd$ is presentable.\label{enum:DIsPresentable}
		\item $\Dd$ is $\kappa$-accessible and has $\kappa$-small colimits for some regular cardinal $\kappa$.\label{enum:DHasKappaSmallColimits}
		\item \!There exists an essentially small $\infty$-category $\Cc$ and an adjunction $L\colon \PSh(\Cc)\shortdoublelrmorphism \Dd\noloc R$ such that $R$ is fully faithful and preserves $\kappa$-filtered colimits for some regular cardinal $\kappa$.\label{enum:AccessibleLocalisation}
		\item $\Dd$ is of the form $\Dd\simeq\cat{Ind}_\kappa(\Cc)$ for some essentially small $\infty$-category $\Cc$ which has $\kappa$-small colimits.\label{enum:DCHasKappaSmallColimits}
	\end{alphanumerate}
	In this case, $\Dd^\kappa$ automatically has all $\kappa$-small colimits \embrace{so that we may choose $\Cc\simeq \Dd^\kappa$ in \cref{enum:DCHasKappaSmallColimits} by \cref{lem:KappaCompactlyGenerated}}.
\end{lem}
\begin{proof}
	The implication \cref{enum:DIsPresentable} $\Rightarrow$ \cref{enum:DHasKappaSmallColimits} is trivial and \cref{enum:DIsPresentable} $\Rightarrow$ \cref{enum:AccessibleLocalisation} follows from the proof of \cref{lem:KappaCompactlyGenerated}. For \cref{enum:DHasKappaSmallColimits} $\Rightarrow$ \cref{enum:DCHasKappaSmallColimits}, we use \cref{lem:KappaCompactlyGenerated} to see that $\Dd\simeq \cat{Ind}_\kappa(\Dd^\kappa)$. It follows easily from \cref{lem:FilteredColimitsPreserveFiniteLimits} and \cref{cor:HomPreservesLimits} that $\kappa$-compact objects are closed under $\kappa$-small colimits. Therefore, if $\Dd$ has all $\kappa$-small colimits, then so has $\Dd^\kappa$. This proves \cref{enum:DHasKappaSmallColimits} $\Rightarrow$ \cref{enum:DCHasKappaSmallColimits}.
	
	For \cref{enum:AccessibleLocalisation} $\Rightarrow$ \cref{enum:DIsPresentable}, first note that $\Dd$ has all colimits. Indeed, given a diagram $\alpha\colon \Ii\rightarrow\Dd$, we can form the colimit $c\simeq\colimit_{i\in\Ii}R(\alpha(i))$ in $\PSh(\Cc)$. Since $L$ preserves colimits by \cref{lem:AdjointsPreserveColimits} and $L\circ R\simeq \id_\Dd$ by the dual of \cref{lem:FullyFaithfulConservativeAdjunction}\cref{enum:FullyFaithfulIffUnitEquivalence}, we see that $L(c)\simeq \colimit_{i\in\Ii}\alpha(i)$, as desired. Now consider the objects $L(\Yo_\Cc(x))$, where $x$ runs through a set of representatives for every equivalence class in $\Cc$. Then $\Hom_\Dd(L(\Yo_\Cc(x)),-)\simeq \Hom_{\PSh(\Cc)}(\Yo_\Cc(x),R(-))$. By Yoneda's lemma, $\Hom_{\PSh(\Cc)}(\Yo_\Cc(x),-)\colon \PSh(\Cc)\rightarrow\cat{An}$ are jointly conservative, and they preserve all colimits by \cref{lem:ColimitsInFunctorCategories}. Since $R$ is fully faithful and preserves $\kappa$-filtered colimits, it follows that $\Hom_\Dd(L(\Yo_\Cc(x)),-)\colon \Dd\rightarrow\cat{An}$ are jointly conservative and preserve $\kappa$-filtered colimits. So the set $\{L(\Yo_\Cc(x))\}$ satisfies the conditions from \cref{lem:KappaCompactlyGenerated}\cref{enum:CompactGenerators} and it follows that $\Dd$ is presentable. This proves \cref{enum:AccessibleLocalisation} $\Rightarrow$ \cref{enum:DIsPresentable}
	
	It remains to show \cref{enum:DCHasKappaSmallColimits} $\Rightarrow$ \cref{enum:DIsPresentable}. We need to show that $\cat{Ind}_\kappa(\Cc)$ has all colimits. We know from \cref{lem:Ind}\cref{enum:IndGeneratedUnderFilteredColimits} that $\cat{Ind}_\kappa(\Cc)$ has $\kappa$-filtered colimits, so by claim~\cref{claim:FilteredCoproduct} in the proof of \cref{lem:KappaCompactlyGenerated}, it's enough to show that $\cat{Ind}_\kappa(\Cc)$ has $\kappa$-small colimits. By \cref{lem:KappaSmallColimits}, it's enough to construct pushouts and $\kappa$-small coproducts. Also, we've seen in the proof of \cref{lem:Ind} that $\Yo_\Cc\colon \Cc\rightarrow\cat{Ind}_\kappa(\Cc)$ preserves $\kappa$-small colimits. Since $\Cc$ itself has all $\kappa$-small colimits by assumption, we see that $\cat{Ind}_\kappa(\Cc)$ has $\kappa$-small colimits of representable presheaves.
	
	Let's first construct the coproduct $\coprod_{s\in S}y_s$ for a discrete set $S$ of cardinality $\abs*{S}<\kappa$ and $y_s\in\cat{Ind}_\kappa(\Cc)$. Write $y_s\simeq \colimit_{j\in \Jj_s}x_{j,s}$ for some filtered $\infty$-category $\Jj_s$ and representable presheaves $x_{j,s}$. Observe that arbitrary products of $\kappa$-filtered $\infty$-categories is $\kappa$-filtered again. Furthermore, if $\Ii$ is any $\kappa$-filtered $\infty$-category, then $\colimit_{i\in\Ii}y_s\simeq y_s$ by \cref{lem:ContractibleColimit} and \cref{lem:FilteredCofinal}. Hence  $\colimit_{(i,j)\in \Ii\times\Jj_s}x_{j,s}\simeq y_s$ by \cref{lem:ColimitManipulations}. So we may replace $\Jj_s$ by $\Ii\times\Jj_s$ for any $\kappa$-filtered $\Ii$. In particular, we may replace $\Jj_s$ by $\Jj\coloneqq \prod_{s\in S}\Jj_s$ and thus we may assume that the diagrams $\Jj_s$ coincide for all $s\in S$. Then \cref{lem:ColimitManipulations} shows
	\begin{equation*}
		\colimit_{j\in\Jj}\coprod_{s\in S}x_{j,s}\simeq \coprod_{s\in S}\colimit_{j\in\Jj}x_{j,s}\simeq \coprod_{s\in S}y_s\,,
	\end{equation*}
	provided any of these colimits exists. But $\coprod_{s\in S}x_{j,s}$ exists for all $j\in \Jj$ because $\cat{Ind}_\kappa(\Cc)$ has $\kappa$-small coproducts of representable presheaves, and then $\colimit_{j\in\Jj}\coprod_{s\in S}x_{j,s}$ exists because $\cat{Ind}_\kappa(\Cc)$ has all $\kappa$-filtered colimits. This shows that $\cat{Ind}_\kappa(\Cc)$ has $\kappa$-small coproducts.
	
	It remains to construct pushouts. Fix a span $y\leftarrow x\rightarrow z$. Let's first construct the pushout in the case where $x$ is representable. Write $y\simeq \colimit_{j\in\Jj} y_j$ and $z\simeq \colimit_{k\in\Kk}z_k$, where $\Jj$ and $\Kk$ are $\kappa$-filtered and $y_j$, $z_k$ are representable presheaves. Since $x$ is representable and thus $\kappa$-compact, \cref{lem:HomotopyGroupsFilteredColimits} implies  $\pi_0\Hom_{\cat{Ind}_\kappa(\Cc)}(x,y)\simeq \colimit_{j\in\Jj}\pi_0\Hom_{\cat{Ind}_\kappa(\Cc)}(x,y_j)$. Hence $x\rightarrow y$ factors through $x\rightarrow y_{j_0}$ for some $j_0\in \Jj$. By \cref{lem:FilteredCofinal}, we can replace $\Jj$ by $\Jj_{j_0/}$ and thus assume that $\Jj$ contains an initial element $j_0$ such that $x\rightarrow y$ is induced by a map $x\rightarrow y_{j_0}$. The same argument applies to $x\rightarrow z$. Furthermore, as above, we can replace $\Jj$ and $\Kk$ by $\Jj\times \Kk$ and thus assume $\Jj=\Kk$. Finally, we have $\colimit_{j\in\Jj}x\simeq x$ by \cref{lem:ContractibleColimit} and \cref{lem:FilteredCofinal}. Hence, using \cref{lem:ColimitManipulations}, we can construct the desired pushout as
	\begin{equation*}
		\colimit_{j\in\Jj}(y_j\sqcup_xz_j)\simeq \colimit_{j\in \Jj}y_j\sqcup_{\colimit_{j\in\Jj}x}\colimit_{j\in\Jj}z_j\simeq y\sqcup_xz\,.
	\end{equation*}
	Here $y_j\sqcup_xz_j$ exists since $\cat{Ind}_\kappa(\Cc)$ has pushouts of representable presheaves, as we've noted above, and then $\colimit_{j\in\Jj}(y_j\sqcup_xz_j)$ exists because $\cat{Ind}_\kappa(\Cc)$ has $\kappa$-filtered colimits. This finishes the case where $x$ is representable. In the general case, write $x\simeq\colimit_{j\in\Jj}x_j$, where $\Jj$ is $\kappa$-filtered and $x_j$ are representable presheaves. By an argument we've seen several times, $y\simeq \colimit_{j\in\Jj}y$ and $z\simeq \colimit_{j\in\Jj}z$. Then
	\begin{equation*}
		\colimit_{j\in\Jj}(y\sqcup_{x_j}z)\simeq \colimit_{j\in \Jj}y\sqcup_{\colimit_{j\in\Jj}x_j}\colimit_{j\in\Jj}z\simeq y\sqcup_xz\,.
	\end{equation*}
	Here the pushouts $y\sqcup_{x_j}z$ exist by the representable case and then $\colimit_{j\in\Jj}(y\sqcup_{x_j}z)$ exists because $\cat{Ind}_\kappa(\Cc)$ has $\kappa$-filtered colimits. This finishes the proof that $\cat{Ind}_\kappa(\Cc)$ has pushouts.
\end{proof}


\begin{cor}\label{cor:AnPresentable}
	The $\infty$-categories $\cat{An}$ and $\cat{Cat}_\infty$ are presentable.
\end{cor}
\begin{proof}[Proof sketch]
	It's clear that $\cat{An}$ and $\cat{Cat}_\infty$ are locally small and they have all colimits by \cref{lem:ColimitsInAnima}. So it suffices to check that both are accessible. In fact, we'll show that both are $\aleph_0$-accessible, by verifying the condition from \cref{lem:KappaCompactlyGenerated}\cref{enum:CompactGenerators}. For $\cat{An}$, it's clear that $*$ is a compact generator as $\Hom_{\cat{An}}(*,X)\simeq X$ for all $X\in\cat{An}$. For $\cat{Cat}_\infty$, we claim that the $\infty$-categories $*$ and $\Delta^1$ are compact generators. 
	
	Let's first argue that $\Hom_{\cat{Cat}_\infty}(*,-)$ and $\Hom_{\cat{Cat}_\infty}(\Delta^1,-)$ are jointly conservative. To this end, recall from \cref{thm:CordierPorter} that $\Hom_{\cat{Cat}_\infty}(*,\Cc)\simeq \core(\Cc)$ and $\Hom_{\cat{Cat}_\infty}(\Delta^1,\Cc)\simeq \core\Ar(\Cc)$ for every $\infty$-category $\Cc$. Now if $F\colon \Cc\rightarrow\Dd$ is a functor such that $\core(F)\colon \core(\Cc)\rightarrow\core(\Dd)$ is an equivalence, then $F$ is essentially surjective. If furthermore $\core \Ar(\Cc)\rightarrow\core\Ar(\Dd)$ is an equivalence, then $F$ is fully faithful. Indeed, for all $x,y\in\Cc$ we can write $\Hom_\Cc(x,y)$ as a pullback of $\core\Ar(\Cc)\rightarrow\core(\Cc)\times\core(\Cc)$ by \cref{par:HomInQuasiCategories} plus the fact that $\core\colon \cat{Cat}_\infty\rightarrow\cat{An}$ preserves pullbacks, since it is a right adjoint by \cref{exm:Adjunctions}\cref{enum:AnToCatInfty}.\footnote{We're also implicitly using that the pullback diagram from \cref{par:HomInQuasiCategories}, which lived in simplicial sets, is also a pullback of $\infty$-categories. See model category fact~\cref{par:HomotopyPushout}.} This proves that $\Hom_{\cat{Cat}_\infty}(*,-)$ and $\Hom_{\cat{Cat}_\infty}(\Delta^1,-)$ are jointly conservative.
	
	We'll only sketch the argument why $*$ and $\Delta^1$ are compact in $\cat{Cat}_\infty$. The crucial observation is that equivalences of quasi-categories are preserved under filtered colimits in the ordinary category $\cat{QCat}$. Indeed, $\cat{QCat}\subseteq \cat{sSet}$ is closed under filtered colimits, because $\Lambda_i^n$ and $\Delta^n$ are finite simplicial sets and so every horn filling problem in a filtered colimit can be solved at some finite stage. So filtered colimits in $\cat{QCat}$ can be computed in $\cat{sSet}$ instead. Then it's straightforward to check that a filtered colimit of fully faithful and essentially surjective maps of quasi-categories is again fully faithful and essentially surjective. Now we can use the same arguments as in the proof of \cref{lem:HomotopyGroupsFilteredColimits} (including the black box \cref{blackbox:Cofinal} and an analogue of \cref{blackbox:Localisation}) to see that filtered colimits in $\cat{Cat}_\infty$ can be computed as ordinary filtered colimits in $\cat{QCat}$. So it remains to show that $ \colimit_{j\in J}\core \F(*,\Cc_j)\cong \core \F(*,\colimit_{j\in\Jj}\Cc_j)$ and $\colimit_{j\in J}\F(\Delta^1,\Cc_j)\cong \core\F(\Delta^1,\colimit_{j\in J}\Cc_j)$ holds for every filtered category $J$ and every diagram $\Cc_{(-)}\colon J\rightarrow\cat{QCat}$. This is straightforward.
\end{proof}
\begin{cor}\label{cor:FunctorCategoriesPresentable}
	If $\Dd$ is a presentable $\infty$-category, then for every $y\in\Dd$ the slice $\infty$-categories $\Dd_{y/}$ and $\Dd_{/y}$ are presentable again. Furthermore, if $\Cc$ is an essentially small $\infty$-category, then $\Fun(\Cc,\Dd)$ is presentable. In particular, $\PSh(\Cc)$ and $\Fun(\Cc,\cat{Cat}_\infty)$ are presentable.
\end{cor}
The same results are true for accessible $\infty$-categories, but this requires significantly more effort. In practice, the results about presentable $\infty$-categories are usually sufficient and so we refer to \cite[\S\href{https://people.math.harvard.edu/~lurie/papers/HTT.pdf\#section.5.4}{5.4}]{HTT} for the accessible case.
\begin{proof}[Proof of \cref{cor:FunctorCategoriesPresentable}]
	It follows from \cref{lem:ColimitsInSliceCategory} and its dual that if $\Dd$ has all colimits, then $\Dd_{y/}$ and $\Dd_{/y}$ have all colimits again. \cref{lem:ColimitsInFunctorCategories} shows the same for $\Fun(\Cc,\Dd)$. So it's enough to check accessibility in each case.
	
	By \cref{lem:KappaCompactlyGenerated}\cref{enum:CompactGenerators}, we can choose a set $S$ of $\kappa$-compact generators for $\Dd$. Since $\Dd$ has coproducts, one easily verifies via \cref{lem:Adjunction} that $\Dd_{y/}\rightarrow\Dd$ has a left adjoint, sending $z\in\Dd$ to $(y\rightarrow y\sqcup z)\in\Dd_{y/}$. Then $\Hom_{\Dd_{y/}}(y\rightarrow y\sqcup s,y\rightarrow z)\simeq \Hom_\Dd(s,z)$ and so $\Hom_{\Dd_{y/}}(y\rightarrow y\sqcup s,-)\colon \Dd_{y/}\rightarrow\cat{An}$ for $s\in S$ are jointly conservative. Using the adjunction property plus the fact that $\Dd_{y/}\rightarrow \Dd$ preserves $\kappa$-filtered colimits by \cref{lem:ColimitsInSliceCategory}\cref{enum:ColimitsInSlice}, we see that every $(y\rightarrow y\sqcup s)$ is $\kappa$-compact again. So $\Dd_{y/}$ satisfies the condition from \cref{lem:KappaCompactlyGenerated}\cref{enum:CompactGenerators} and is therefore accessible.
	
	For $\Fun(\Cc,\Dd)$, consider the functors $F_{x,s}\coloneqq\Lan_{\{x\}\rightarrow \Cc}(\const s)\colon \Cc\rightarrow\Dd$, where $s\in S$ and $x$ runs through a set of representatives of the equivalence classes of objects in $\Cc$. These Kan extensions exist by \cref{lem:KanExtensionFormula} since $\Dd$ has all colimits. The universal property of Kan extensions shows $\Hom_{\Fun(\Cc,\Dd)}(F_{x,s},G)\simeq \Hom_{\Fun(\{x\},\,\Cc)}(\const s,G|_{\{x\}})\simeq \Hom_\Dd(s,G(x))$ for every functor $G\in\Fun(\Cc,\Dd)$. Since colimits in functor categories are computed pointwise by \cref{lem:ColimitsInFunctorCategories} and $s$ is $\kappa$-compact by assumption, it follows that $F_{x,s}$ is $\kappa$-compact. Since equivalences of functors can be detected pointwise by \cref{thm:EquivalencePointwise}, it follows that $\Hom_{\Fun(\Cc,\Dd)}(F_{x,s},-)\colon \Fun(\Cc,\Dd)\rightarrow\cat{An}$ for $s\in S$ and $x$ running through all equivalence classes in $\Cc$ are jointly conservative. So $\Fun(\Cc,\Dd)$ satisfies the condition from \cref{lem:KappaCompactlyGenerated}\cref{enum:CompactGenerators} and is therefore accessible.
	
	For $\Dd_{/y}$, we will instead verify the condition from \cref{lem:KappaCompactlyGenerated}\cref{enum:DGeneratedUnderFilteredColimits}. First observe that $\Dd^\kappa_{/y}$ is essentially small. Indeed, $\Dd^\kappa_{/y}\simeq \Dd^\kappa\times_\Dd\Dd_{/y}\rightarrow \Dd_{/y}$ is fully faithful, hence $\Dd^\kappa_{/y}$ is locally small, because $\Dd_{/y}$ is locally small by the assumption on $\Dd$ and \cref{cor:HomInSliceCategories}. So its enough to show that $\pi_0\core(\Dd^\kappa_{/y})$ is a set. This follows from $\Dd^\kappa$ being essentially small (as we've seen in the proof of \cref{lem:KappaCompactlyGenerated}) and $\Dd$ being locally small, so that there can't be \enquote{too many} equivalence classes of morphisms $z\rightarrow y$ where $z\in\Dd^\kappa$. Since $\Dd_{/y}\rightarrow\Dd$ preserves arbitrary colimits by the dual of \cref{lem:ColimitsInSliceCategory}\cref{enum:LimitsInSlice} and $\kappa$-filtered colimits are preserved under pullbacks by \cref{lem:FilteredColimitsPreserveFiniteLimits}, we can use \cref{cor:HomInSliceCategories} to show that the objects in $\Dd^\kappa_{/y}$ are $\kappa$-compact in $\Dd_{/y}$. It remains to show that they generate $\Dd_{/y}$ under $\kappa$-filtered colimits. Pick some $(z\rightarrow y)\in\Dd_{/y}$ and write $z\simeq\colimit_{j\in\Jj}z_j$ for some $\kappa$-filtered $\infty$-category $\Jj$ and some diagram $z_{(-)}\colon\Jj\rightarrow\Dd^\kappa$. Composing the colimit transformation $u\colon z_{(-)}\Rightarrow \const z$ with $\const z\Rightarrow \const y$ yields a transformation $z_{(-)}\Rightarrow \const y$, which in turn defines a functor $(z_{(-)}\rightarrow y)\colon \Jj\rightarrow \Dd_{/y}$. Then $(z\rightarrow y)\simeq \colimit_{j\in\Jj}(z_j\rightarrow y)$ in $\Dd_{/y}$, as desired.
\end{proof}

\subsection{The adjoint functor theorem}\label{subsec:AdjointFunctorTheorem}
Finally, we can state and prove the adjoint functor theorem. The original version is of course Lurie's \cite[Corollary~\HTTthm{5.5.2.9}]{HTT}. Our version is slightly more general and is taken from Markus Land's book \cite[Theorems~5.2.2 and~5.2.14]{Land}, who in turn took them from \cite{AdjointFunctorTheorems}.
\begin{satanicthm}[Adjoint functor theorem]\label{thm:AdjointFunctorTheorem}
	Let $F\colon \Cc\rightarrow\Dd$ be a functor between locally small $\infty$-categories.
	\begin{alphanumerate}
		\item Assume that $\Cc$ and $\Dd$ have all colimits and $\Cc$ is generated under colimits by an essentially small sub-$\infty$-category $\Cc_0\subseteq\Cc$. Then $F$ admits a right adjoint $G\colon \Dd\rightarrow\Cc$ if and only if $F$ preserves colimits.\label{enum:AdjointFunctorTheoremLeft}
		\item Assume that $\Cc$ and $\Dd$ have all limits, that $\Cc$ is accessible, and that for every object $y\in\Dd$ there exists a regular cardinal $\kappa_y$ such that $y$ is $\kappa_y$-compact. If there exists a regular cardinal $\kappa$ such that $F$ preserves limits as well as $\kappa$-filtered colimits, then $F$ admits a left adjoint $G\colon \Dd\rightarrow\Cc$. The converse is true as well provided that $\Dd$ is accessible too.\label{enum:AdjointFunctorTheoremRight}
		%\item Assume that $\Dd$ is presentable and that $F$ is fully faithful. Then $F$ admits a left adjoint $G\colon \Dd\rightarrow \Cc$ if and only if $F$ preserves limits and there exists a regular cardinal $\kappa$ such $F$ preserves $\kappa$-filtered colimits.\label{enum:AdjointFunctorTheoremFullSubcategory}
	\end{alphanumerate}
	Furthermore, in both \cref{enum:AdjointFunctorTheoremLeft} and \cref{enum:AdjointFunctorTheoremRight}, the conditions on $\Cc$ and $\Dd$ are automatically satisfied if $\Cc$ and $\Dd$ are presentable.\hfill\smash{\GrothendieckRightDevil}
\end{satanicthm}


Before we embark on the proof of \cref{thm:AdjointFunctorTheorem}, we'll draw a somewhat surprising corollary and discuss a useful supplement.
\begin{cor}\label{cor:PresentableComplete}
	Let $\Cc$ be a locally small $\infty$-category.
	\begin{alphanumerate}
		\item If $\Cc$ has all colimits and is generated under colimits by an essentially small sub-$\infty$-category $\Cc_0\subseteq \Cc$, then $\Cc$ has all limits too. In particular, presentable $\infty$-categories have all limits.\label{enum:PresentableComplete}
		\item If $\Cc$ is accessible and has all limits, then $\Cc$ has all colimits too. In particular, $\Cc$ is presentable.\label{enum:AccessibleCocomplete}
	\end{alphanumerate}
\end{cor}
\begin{proof}
	Let $\Ii$ be an essentially small $\infty$-category. Then $\Fun(\Ii,\Cc)$ is locally small by \cref{rem:FunLocallySmall}. It follows from \cref{lem:ColimitsInFunctorCategories} that $\const\colon \Cc\rightarrow\Fun(\Ii,\Cc)$ preserves all limits and colimits. In the situation of \cref{enum:PresentableComplete}, we may apply \cref{thm:AdjointFunctorTheorem}\cref{enum:AdjointFunctorTheoremLeft} to see that $\const$ has a right adjoint $\limit_\Ii\colon \Fun(\Ii,\Cc)\rightarrow\Cc$, as desired.
	
	In the situation of \cref{enum:AccessibleCocomplete}, we only need to check that every element of $\Fun(\Ii,\Cc)$ is $\tau$-compact for some sufficiently large regular cardinal $\tau$, for then $\const$ will have a left adjoint $\colimit_\Ii\colon \Fun(\Ii,\Cc)\rightarrow\Cc$ by \cref{thm:AdjointFunctorTheorem}\cref{enum:AdjointFunctorTheoremRight}. Say $\Cc$ is $\kappa$-compact. Then every $x\in\Cc$ can be written as a colimit of $\kappa$-compact objects by \cref{lem:KappaCompactlyGenerated}\cref{enum:DGeneratedUnderFilteredColimits}. If $\kappa_x$ is larger than $\kappa$ and the cardinality of the indexing diagram, then $x$ will be $\kappa_x$-compact, because $\kappa_x$-compact objects are closed under $\kappa_x$-small colimits (we've seen this argument several times in the proofs of \cref{lem:KappaCompactlyGenerated,lem:Presentable}). Now let $\alpha\colon \Ii\rightarrow\Cc$ be a functor. Let $\tau_\alpha$ be a regular cardinal such that $\TwAr(\Cc)$ is $\tau_\alpha$-small and $\tau_\alpha\geqslant \kappa_{\alpha(i)}$ for every $i\in\Ii$. Using that $\tau_\alpha$-small limits commute with $\tau_\alpha$-filtered colimits by \cref{lem:FilteredColimitsPreserveFiniteLimits} and that colimits in functor categories are computed pointwise by \cref{lem:ColimitsInFunctorCategories}, the formula from \cref{cor:HomInFunctorCats} shows that $\alpha$ is $\tau_\alpha$-compact.
\end{proof}
A useful supplement to the adjoint functor theorem is the \emph{reflection theorem:}
\begin{thm}[Reflection theorem]\label{thm:ReflectionTheorem}
	Let $\Dd$ be a presentable $\infty$-category and let $\Cc\subseteq\Dd$ be a full sub-$\infty$-category such that $\Cc$ is closed under limits in $\Dd$ and there exists a regular cardinal $\kappa$ such that $\Cc$ is closed under $\kappa$-filtered colimits in $\Dd$. Then the inclusion $\Cc\subseteq \Dd$ has a left adjoint and $\Cc$ is presentable too.\hfill$\blacksquare$
\end{thm}
We won't prove the reflection theorem. A proof for ordinary categories can be found in Adamek and Rosicky's book \cite[Reflection Theorem~2.48]{AdamekRosicky}; the $\infty$-categorical version was only recently proven in \cite{ReflectionTheorem}.%
\footnote{\label{footnote:ReflectionTheorem}Interestingly, the proof for ordinary categories can not entirely be carried over. The step that fails is related to the following two important caveats:
\begin{alphanumerate}
	\item Let $\Cc$ be an $\infty$-category and suppose there are morphisms $\alpha\colon x\rightarrow y$ and $\beta\colon y\rightarrow x$ satisfying $\beta\circ\alpha\simeq \id_x$, so that $x$ is a retract of $y$. If $\Cc$ is an ordinary category, then $x$ can be expressed as the equaliser (and also as the coequaliser) of $\id_y$ and $\alpha\circ\beta$. However, this doesn't work in general $\infty$-categories---it already fails in $\cat{An}$, and even more spectacularly in the $\infty$-category of spectra. We can still express $x$ in terms of $y$, for example, as\label{enum:Retracts}
	\begin{equation*}
		x\simeq \colimit\left(y\xrightarrow{\alpha\circ\beta}y\xrightarrow{\alpha\circ\beta}y\xrightarrow{\alpha\circ\beta}\dotsb\right)\simeq \limit\left(\dotsb\xrightarrow{\alpha\circ\beta}y\xrightarrow{\alpha\circ\beta}y\xrightarrow{\alpha\circ\beta}y\right)
	\end{equation*}
	(to see this, just observe that these diagrams can be (co)initially replaced by constant $x$-valued diagrams). But a finite limit or colimit will never suffice.
	\item There is a notion of monomorphism in $\infty$-categories (see \cite[\S\href{https://people.math.harvard.edu/~lurie/papers/HTT.pdf\#subsection.5.5.6}{5.5.6}]{HTT}) and these allow for manipulation of equalisers as in ordinary categories. But the inclusion of a retract is usually not a monomorphism! Again, this already fails in $\cat{An}$---monomorphisms are inclusions of path components, but there are many more retracts---and even more spectacularly in $\cat{Sp}$, where there are no monomorphisms at all.\label{enum:Monomorphisms}
\end{alphanumerate}
If you'd like to read up on the proof of the $\infty$-categorical reflection theorem, I'd suggest you first read the proof for ordinary categories and identify where the above issues occur. Then check that all other arguments can be carried over. Finally, check out \cite{ReflectionTheorem} to see how the issues can be circumvented.
}
The only thing one has to show is that in the given situation $\Cc$ is automatically accessible. Indeed, if that's true, then $\Cc$ is presentable by \cref{cor:PresentableComplete}\cref{enum:AccessibleCocomplete}. Furthermore, \cref{cor:PresentableComplete}\cref{enum:PresentableComplete} shows that \cref{thm:AdjointFunctorTheorem}\cref{enum:AdjointFunctorTheoremRight} is applicable, producing the desired left adjoint. But proving that $\Cc$ is automatically accessible is surprisingly hard (and rather surprising altogether).


We start off the proof of \cref{thm:AdjointFunctorTheorem} with two preparatory lemmas.
\begin{lem}\label{lem:TerminalInSlice}
	Let $F\colon \Cc\rightarrow\Dd$ be a functor between $\infty$-categories and let $y\in\Dd$ be an object. Then $y$ admits a right adjoint object $x\in\Cc$ under $F$ if and only if the slice $\infty$-category $\Cc_{/y}\simeq \Cc\times_{\Dd}\Dd_{/y}$ has a terminal object.
\end{lem}
\begin{proof}[Proof sketch]
	Let $x\in \Cc$ be an object and $c\colon F(x)\rightarrow y$ a morphism in $\Cc$. Then $x$ is a right adjoint object to $y$ under $F$, with counit $c$, if and only if the composition
	\begin{equation*}
		\Hom_\Cc(-,x)\overset{F}{\Longrightarrow} \Hom_\Dd\bigl(F(-),F(x)\bigr)\overset{c_*}{\Longrightarrow}\Hom_\Dd\bigl(F(-),y\bigr)
	\end{equation*}
	is an equivalence of functors. By \cref{thm:EquivalencePointwise}, this can be checked on objects. So choose $x'\in\Cc$. To check that $\Hom_\Cc(x,x')\rightarrow \Hom_\Dd(F(x'),y)$, it's enough by \cref{thm:Whitehead} to check that the fibres over every $\alpha\in\Hom_\Dd(F(x'),y)$ are contractible. So fix some $\alpha\colon F(x')\rightarrow y$. Using the fact that $\Hom$ animae in pullbacks of $\infty$-categories are the pullbacks of the respective $\Hom$ animae (which we'll prove in more generality in \cref{lem:HomInLimits}\cref{enum:HomInLimits}), one easily computes that the fibre $\Hom_\Cc(x',x)\times_{\Hom_\Dd(F(x'),y)}\{\alpha\}$ is equivalent to $\Hom_{\Cc_{/y}}((x',\alpha\colon F(x')\rightarrow y),(x,c\colon F(x)\rightarrow y))$. So the fibres are all contractible if and only if $(x,c\colon F(x)\rightarrow y)$ is a terminal object of $\Cc_{/y}$.
\end{proof}
\begin{lem}\label{lem:TerminalObjectColimit}
	Let $\Cc$ be any \embrace{possibly large} $\infty$-category. Then $\Cc$ has a terminal object if and only if $\id_\Cc\colon \Cc\rightarrow\Cc$ has a colimit, in which case the terminal object is that colimit.
\end{lem}
\begin{proof}[Proof sketch]
	If $y\in\Cc$ is terminal, then $\{y\}\rightarrow\Cc$ is a right adjoint, hence coinitial by \cref{exm:Cofinal}\cref{enum:RightAdjointCofinal}. Hence $\colimit(\id_\Cc\colon \Cc\rightarrow\Cc)\simeq y$; in particular, the colimit exists.
	
	Conversely, assume the colimit exists, and let $u\colon \id_\Cc\Rightarrow \const y$ be the natural transformation exhibiting $y$ has the colimit. We wish to prove that $\Cc\shortdoublelrmorphism \{y\}$ is an adjunction. To this end, by \cref{lem:TriangleIdentities}, it suffices to construct the unit and the counit as well as to verify the triangle identities. We take $u$ to be our unit. The counit as well as the first triangle identity come for free since $\Fun(\Cc,\{y\})\simeq *$ and $\Fun(\{y\},\{y\})\simeq *$. By a quick unravelling, the second triangle identity comes down to proving that $u_y\colon y\rightarrow y$ is the identity on $y$. To this end, consider $u$ as a functor $u\colon \Delta^1\times\Cc\rightarrow\Cc$ and consider the composition $\sigma\coloneqq u\circ(\id_{\Delta^1}\times\Cc)\colon \Delta^1\times(\Delta^1\times\Cc)\rightarrow \Delta^1\times\Cc\rightarrow \Cc$. By \enquote{currying}, $\sigma$ corresponds to a functor $\Delta^1\times\Delta^1\rightarrow\Fun(\Cc,\Cc)$, or in other words, to a commutative square in $\Fun(\Cc,\Cc)$. By a somewhat confusing unravelling, that commutative square is
	\begin{equation*}
		\begin{tikzcd}[column sep=large]
			\id_\Cc\doublear["u"{black,above=0.1em}]{r}\doublear["u"'{black,left=0.1em}]{d}\drar[commutes] & \const y\doublear["\const u_y"{black,right=0.1em}]{d}\\
			\const y\doublear["\id_{\const y}"{black,above=0.1em}]{r} & \const y
		\end{tikzcd}
	\end{equation*}
	Thus, in the equivalence $\Hom_\Cc(y,y)\simeq \Hom_\Cc(\colimit_\Cc\id_\Cc,y)\simeq \Hom_{\Fun(\Cc,\Cc)}(\id_\Cc,y)$, both $\id_y$ and $u_y$ are mapped to $u\in\Hom_{\Fun(\Cc,\Cc)}(\id_\Cc,y)$. This proves $\id_y\simeq u_y$, as desired.
\end{proof}

\begin{proof}[Proof sketch of \cref{thm:AdjointFunctorTheorem}\cref{enum:AdjointFunctorTheoremLeft}]
	If $F$ admits a right adjoint, then $F$ preserves colimits by \cref{lem:AdjointsPreserveColimits}. Conversely, assume $F$ preserves colimits. Adjoints can be constructed pointwise by \cref{lem:Adjunction}, and thus by \cref{lem:TerminalInSlice}, it's enough to show that the slice $\infty$-category $\Cc_{/y}\simeq \Cc\times_\Dd\Dd_{/y}$ has a terminal object for every $y\in\Dd$. A straightforward generalisation of the arguments in the proof of \cref{cor:FunctorCategoriesPresentable} shows that $\Cc_{/y}$ has again all (small) colimits and is generated under colimits by its full sub-$\infty$-category $(\Cc_0)_{/y}\simeq \Cc_0\times_\Cc\Cc_{/y}$; this is the only time we use that $F$ preserves colimits. So we can replace $\Cc$ by $\Cc_{/y}$ and are thus reduced to showing that $\Cc$ has a terminal object.		
	
	%$F$ preserves colimits, we can use a similar (but dual) argument as in the proof of \cref{lem:ColimitsInSliceCategory}\cref{enum:LimitsInSlice} to show that $\Cc_{/y}$ has all (small) colimits. Furthermore, as we'll now explain a variation of the argument shows that $\Cc_{/y}$ is generated under colimits by its full sub-$\infty$-category $(\Cc_0)_{/y}\simeq \Cc_0\times_\Cc\Cc_{/y}$. Indeed, pick some $(x,\alpha\colon F(x)\rightarrow y)\in\Cc_{/y}$ and write $x\simeq\colimit_{i\in\Ii}x_i$ for some diagram $x_{(-)}\colon\Ii\rightarrow\Cc_0$. Applying $F$ to the natural transformation $u\colon x_{(-)}\Rightarrow \const x$ and composing with $\const\alpha\colon \const F(x)\Rightarrow \const y$ yields a transformation $\const\alpha\circ F(u)\colon F(x_{(-)})\Rightarrow \const y$, which in turn defines functors $(F(x_{(-)})\rightarrow y)\colon \Ii\rightarrow \Dd_{y/}$ and $(x_{(-)},F(x_{(-)}\rightarrow y))\colon \Ii\rightarrow \Cc_0\times_\Dd\Dd_{y/}$. Then a straightforward argument shows $(x,\alpha\colon F(x)\rightarrow y)\simeq \colimit_{i\in\Ii}(x_i,F(x_i)\rightarrow y)$ in $\Cc_{/y}$, as desired. Finally, it's easy to show that $(\Cc_0)_{/y}$ is essentially small. Clearly, $(\Cc_0)_{/y}$ is locally small, because \cref{cor:HomInSliceCategories} and the upcoming \cref{lem:HomInLimits} show that $\Hom_{(\Cc_0)_{/y}}$ can be written as a pullback involving $\Hom_\Cc$ and $\Hom_\Dd$, both of which are essentially small by the assumption that $\Cc$ and $\Dd$ are locally small. So its enough to show that $\pi_0\core((\Cc _0)_{/y})$ is a set. This easily follows from $\Cc$ and $\Dd$ being locally small, so that there can't be \enquote{too many} equivalence classes of pairs $(x_0,\alpha_0\colon F(x_0)\rightarrow y)$ where $x_0\in\Cc_0$. We leave the argument to you.
	
	By \cref{lem:TerminalObjectColimit}, we must show that $\id_\Cc\colon \Cc\rightarrow\Cc$ admits a colimit. Since $\Cc$ has all small colimits, it will be enough to show that $\Cc$ admits a coinitial functor from a small $\infty$-category. Note that this step requires some set-theoretic care, since it's not so clear why \cref{thm:JoyalsQuillenA} would be applicable to colimits with potentially large indexing $\infty$-categories. This problem can be solved by considering universes, and with some more effort even in ZFC; we'll ignore it in the following.
	
	Since $\Cc$ has all small colimits, the colimit $t\coloneqq\colimit(\Cc_0\rightarrow\Cc)$ exists. For every $x\in\Cc$ there exists a morphism $x\rightarrow t$. Indeed, since we assume $\Cc$ to be generated under colimits by $\Cc_0$, we can write $x$ as a colimit $x\simeq \colimit(\Ii\rightarrow\Cc_0\rightarrow\Cc)$ and then we can consider the morphism $x\simeq \colimit(\Ii\rightarrow\Cc_0\rightarrow\Cc)\rightarrow\colimit(\Cc_0\rightarrow\Cc)\simeq t$ using functoriality of colimits, see \cref{lem:ColimitsFunctorial}. Now let $\Tt\subseteq \Cc$ be the full sub-$\infty$-category spanned by $t$ (note that $\Tt$ is not just $\{t\}$, since we include all non-identity endomorphisms of $t$ as well). Since $\Cc$ is locally small, $\Tt$ must be essentially small. We claim that $\Tt\rightarrow\Cc$ is coinitial. To this end, we'll show that $\Tt\times_\Cc\Cc_{x/}$ is filtered; then \cref{lem:FilteredCofinal} will show that the condition from \cref{thm:JoyalsQuillenA}\cref{enum:WeaklyContractible} is satisfied. Let $\alpha\colon \Ii\rightarrow\Tt\times_\Cc\Cc_{x/}$ be a functor from any small $\infty$-category. If $\ov\alpha\colon \Ii\rightarrow\Tt\times_\Cc\Cc_{x/}\rightarrow\Cc$ denotes the underlying functor, then $\alpha$ is equivalently given by a natural transformation $\const x\Rightarrow \ov\alpha$ such that $\ov\alpha$ takes values in the full sub-$\infty$-category $\Tt\subseteq\Cc$. Since $\Cc$ has small colimits, $x\simeq\colimit_{i\in\Ii}\ov\alpha(i)$ exists in $\Cc$. As observed above, there exists a morphism $x\rightarrow t$. Composing the colimit transformation $\ov\alpha\Rightarrow \const x$ with $\const x\Rightarrow t$ yields a natural transformation $\ov\alpha\Rightarrow \const t$, or equivalently, a functor $\ov\alpha^\triangleright\colon \Ii^\triangleright \rightarrow \Cc$. By construction, $\ov\alpha^\triangleright$ takes values in the full sub-$\infty$-category $\Tt\subseteq\Cc$. Composing with $\const x\Rightarrow \ov\alpha$ provides a functor $\alpha^\triangleright\colon \Ii^\triangleright\rightarrow\Tt\times_\Cc\Cc_{x/}$, as desired. This finishes the proof that $\Tt\times_\Cc\Cc_{x/}$ is filtered.%So we're done!
\end{proof}
Our proof of \cref{thm:AdjointFunctorTheorem}\cref{enum:AdjointFunctorTheoremRight} will again be preceded by two preparatory lemmas.
\begin{lem}[\enquote{Right adjoints preserve sufficiently filtered colimits}]\label{lem:RightAdjointsAccessible}
	Let $G\colon \Dd\rightarrow\Cc$ be a functor between accessible $\infty$-categories. If $G$ admits a left adjoint $F$, then $G$ preserves $\tau$-filtered colimits for sufficiently large regular cardinals $\tau$.
\end{lem}
\begin{proof}
	Choose regular cardinals $\kappa$ and $\lambda$ such that $\Cc$ is $\kappa$-accessible and $\Dd$ is $\lambda$-accessible. By \cref{lem:KappaCompactlyGenerated}, we may identify $\Cc$ and $\Dd$ with $\cat{Ind}_\kappa(\Cc^\kappa)$ and $\cat{Ind}_\lambda(\Dd^\lambda)$, respectively. First note that for every $y\in\Dd$ there exists a regular cardinal $\lambda_y$ such that $y$ is $\lambda_y$-compact. Indeed, we may write $y$ has a colimit of $\lambda$-compact objects, and then it suffices to choose $\lambda_y\geqslant \lambda$ sufficiently large so that the indexing diagram of the colimit is $\lambda_y$-small. Since $\Cc^\kappa$ is essentially small, as we've seen in the proof of \cref{lem:KappaCompactlyGenerated}, we may choose a regular cardinal $\tau\geqslant \kappa$ such that $F(x)$ is $\tau$-compact for all $x\in\Cc^\kappa$. We claim that $G$ preserves $\tau$-filtered colimits. Since $\Cc\simeq\cat{Ind}_\kappa(\Cc^\kappa)\subseteq \PSh(\Cc^\kappa)$, the functors $\Hom_\Cc(x,-)\colon \Cc\rightarrow \cat{An}$ for $x\in\Cc^\kappa$ are jointly conservative and preserve $\kappa$-filtered and thus also $\tau$-filtered colimits. So it's enough to show that $\Hom_\Cc(x,G(-))$ preserves $\tau$-filtered colimits. But $\Hom_\Cc(x,G(-))\simeq \Hom_\Dd(F(x),-)$ and $F(x)$ is $\tau$-compact by construction.
\end{proof}
\begin{lem}\label{lem:Accessible}
	Let $\Cc$ be a $\kappa$-accessible $\infty$-category. Then $\Cc$ is also $\tau$-accessible for every sufficiently large regular cardinal $\tau$.
\end{lem}
\begin{rem}\label{rem:Accessible}
	It's usually \emph{not} true that a $\kappa$-accessible $\infty$-category $\Cc$ is $\tau$-accessible for all $\tau >\kappa$. However, this works if $\Cc$ is presentable. Indeed, it's immediately clear from \cref{lem:KappaCompactlyGenerated}\cref{enum:CompactGenerators} that any set of $\kappa$-compact generators is also a set of $\tau$-compact generators 
\end{rem}
\begin{proof}[Proof sketch of \cref{lem:Accessible}]
	By \cref{lem:KappaCompactlyGenerated}\cref{enum:DGeneratedUnderFilteredColimits} will be enough to show that $\Cc$ is generated under $\tau$-filtered colimits by $\Cc^\tau$, where $\tau$ is a sufficiently large regular cardinal that will be chosen at the end of the proof. Every $x\in\Cc$ can be written as $x\simeq \colimit_{j\in\Jj}x_j$, where $\Jj$ is $\kappa$-filtered and $x_j\in\Cc^\kappa$. We'll rewrite this as a $\tau$-filtered colimit of $\tau$-compact objects. First, by \cref{blackbox:Cofinal} in the proof of \cref{lem:HomotopyGroupsFilteredColimits}, we find a coinitial functor $J\rightarrow\Jj$ from a directed partially ordered set $J$. Note that $J$ is automatically a $\kappa$-filtered $\infty$-category by the criterion from \cref{lem:FilteredColimitsPreserveFiniteLimits}. We'll show that $J$ can be written as a colimit $J\simeq \colimit_{i\in I}J_i$ in $\cat{Cat}_\infty$, where $I$ is a $\tau$-filtered directed partially ordered set and $J_i\subseteq J$ are essentially $\tau$-small $\kappa$-filtered partially ordered subsets. If we can do this, we're done. Indeed, by \cref{lem:ColimitManipulations}\cref{claim:AssembleColimits}, we may then write $x\simeq \colimit_{i\in I}\colimit_{j\in J_i}x_j$. Each $\colimit_{j\in J_i}x_j$ exists, as $\Cc$ admits $\kappa$-filtered colimits by \cref{lem:Ind}\cref{enum:IndGeneratedUnderFilteredColimits}. Furthermore, $\colimit_{j\in J_i}x_j$ is $\tau$-compact because each $x_j$ is $\kappa$-compact, hence $\tau$-compact, and $\tau$-compact objects are stable under $\tau$-small colimits by an easy application of \cref{lem:FilteredColimitsPreserveFiniteLimits}.
	
	To write $J$ as such a colimit, let $\Pp^\tau(J)$ be the partially ordered set of subsets $S\subseteq J$ of cardinality $\abs*{S}<\tau$. Note that $\Pp^\tau(J)$ is $\tau$-filtered as an $\infty$-category. Indeed, using \cref{lem:SimplicialHoNerveAdjunction}, it's enough to show that $\Pp^\tau(J)$ is $\tau$-filtered as an ordinary category, which is true since we can just take unions of $<\tau$ subsets of cardinality $<\tau$. Each $S\in\Pp^\tau(J)$ can be identified with the full subcategory $J[S]\subseteq J$ spanned by $S$ and we have $J\simeq \colimit_{S\in\Pp^\tau(J)}J[S]$ in $\cat{Cat}_\infty$. One way to prove this would be to use that filtered colimits in $\cat{Cat}_\infty$ can be computed on the level of simplicial sets (see the proof of \cref{cor:AnPresentable}); then the desired equivalence is straightforward. For an alternative, model-independent argument, let $\Uu$ be the unstraightening of the functor $\Pp^\tau(J)\rightarrow \cat{Cat}_\infty$ sending $S\mapsto J[S]$. Then $\Uu$ is an ordinary category and can be easily described explicitly. The same argument as in the proof of claim~\cref{claim:FilteredCoproduct} in the proof of~\cref{lem:KappaCompactlyGenerated} shows that $\Uu\rightarrow J$ is coinitial. By \cref{lem:ColimitsInAnima}, $\colimit_{S\in\Pp^\tau(J)}J[S]$ is a localisation of $\Uu$. Since localisations are coinitial by  \cref{exm:Cofinal}\cref{enum:LocalisationsCofinal}, we conclude that $\colimit_{S\in\Pp^\tau(J)}J[S]\rightarrow J$ is coinitial too. This is not quite what we wanted, but it's enough for our purposes. Now we claim:
	\begin{alphanumerate}\itshape
		\item[\boxtimes] \!There exists a partially ordered subset $I\subseteq \Pp^\tau(J)$ such that $J[S]$ is $\kappa$-filtered for every $S\in I$ and such that the inclusion $I\rightarrow\Pp^\tau(J)$ has a left adjoint $L\colon \Pp^\tau(J)\rightarrow I$.\label{claim:Filterification}
	\end{alphanumerate}
	Since right adjoints are coinitial by \cref{exm:Cofinal}\cref{enum:RightAdjointCofinal}, we also get $J\simeq \colimit_{S\in I}J[S]$. Furthermore, this coinitiality implies that $I$ is $\tau$-filtered again, because it satisfies the criterion from \cref{lem:FilteredColimitsPreserveFiniteLimits}. So once we know \cref{claim:Filterification}, we're done.
	
	For every equivalence class of functors $\alpha\colon \Ii\rightarrow J$ from an essentially $\kappa$-small $\infty$-category $\Ii$, choose an extension $\alpha^\triangleright\colon \Ii^\triangleright\rightarrow J$. Let $S_0\in \Pp^\tau(J)$. Let $S_1\subseteq J$ be obtained from $S_0$ by adjoining the \enquote{tip of the cone} for every $\alpha^\triangleright \colon \Ii^\triangleright\rightarrow J$ such that $\alpha\colon \Ii \rightarrow J$ factors through $J[S_0]\rightarrow J$. If $\tau$ is larger than the set of equivalence classes of essentially $\kappa$-small $\infty$-categories, then $S_1$ will have cardinality $\abs*{S_1}<\tau$ again. By transfinite induction, we can repeat this construction $\kappa$ many times. The result is a subset $S_\kappa\subseteq J$ such that $\abs*{S_\kappa}<\tau$ and $J[S_\kappa]$ is $\kappa$-filtered. If we put $L(S_0)\coloneqq S_\kappa$, then $L\colon \Pp^\tau(J)\rightarrow \Pp^\tau(J)$ is a functor satisfying $L\circ L=L$ (we really get an equality, not just an equivalence). Thus, if $I\subseteq \Pp^\tau(J)$ is the image of $L$, then an easy argument shows that $L\colon \Pp^\tau(J)\rightarrow I$ is indeed left adjoint to the inclusion (bear in mind that we're working with ordinary categories here, so constructing functors and adjunctions can be done by hand). Therefore, the conditions from \cref{claim:Filterification} are satisfied.
\end{proof}
\begin{proof}[Proof sketch of \cref{thm:AdjointFunctorTheorem}\cref{enum:AdjointFunctorTheoremRight}]
	By the dual of \cref{lem:TerminalInSlice}, it's enough to show that the slice $\infty$-category $\Cc_{y/}\simeq \Cc\times_\Dd\Dd_{y/}$ has an initial object for all $y\in\Dd$. By the dual of \cref{lem:TerminalObjectColimit}, this is equivalent to showing that $\id_{\Cc_{y/}}\colon \Cc_{y/}\rightarrow\Cc_{y/}$ admits a limit. A straightforward generalisation of \cref{lem:ColimitsInSliceCategory}\cref{enum:LimitsInSlice} shows that $\Cc_{y/}$ has all (small) limits; this argument crucially uses that $F$ preserves limits. So it will be enough to construct an initial functor from an essentially small $\infty$-category into $\Cc_{y/}$.
	
	By assumption and \cref{lem:Accessible} we may choose a sufficiently large regular cardinal $\kappa$ such that $\Cc$ is $\kappa$-accessible, $F$ preserves $\kappa$-filtered colimits, and $y$ is $\kappa$-compact. Let $\Tt\subseteq\Cc_{y/}$ be the full sub-$\infty$-category spanned by those $(x, \alpha\colon y\rightarrow F(x))$ where $x$ is $\kappa$-compact. By an easy argument, the likes of which we've seen several times by now, $\Tt$ is essentially small. We claim that for every $z\in \Cc_{y/}$ there is an element $t\in \Tt$ and a morphism $t\rightarrow z$ in $\Cc_{y/}$. If we can show this, then a similar argument as in the proof of \cref{enum:AdjointFunctorTheoremLeft} shows that $\Tt\rightarrow\Cc_{y/}$ is initial. Indeed, we'll show that $\Tt\times_{\Cc_{y/}}(\Cc_{y/})_{/w}$ is cofiltered for every $w\in \Cc_{y/}$, which will imply initiality by the dual of \cref{thm:JoyalsQuillenA}\cref{enum:WeaklyContractible} and the dual of \cref{lem:FilteredCofinal}. So let $\alpha\colon \Ii\rightarrow \Tt\times_{\Cc_{y/}}(\Cc_{y/})_{/w}$ be a functor from a small $\infty$-category $\Ii$. Since $\Cc_{y/}$ admits small limits, the underlying functor $\ov\alpha\colon \Ii\rightarrow \Cc_{y/}$ admits a limit $z\simeq \limit_{i\in\Ii}\ov\alpha(i)$. Choosing a morphism $t\rightarrow z$ for some $t\in \Tt$, we get natural transformations $\const t\Rightarrow\const z\Rightarrow\ov\alpha$. The composition $\const t\Rightarrow \ov\alpha$ induces an extension $\alpha^\triangleleft\colon \Ii^\triangleleft\rightarrow \Tt\times_{\Cc_{y/}}(\Cc_{y/})_{/w}$ of $\alpha$, as desired.
	
	It remains to show our claim that for every $z\in\Cc_{y/}$ there exists a moprhism $t\rightarrow z$ for some $t\in\Tt$. Write $z$ as a pair $(x,\beta\colon y\rightarrow F(x))$ for some $x\in \Cc$. Since $\Cc$ is $\kappa$-accessible, we can write $x$ as a $\kappa$-filtered colimit $x\simeq \colimit_{j\in\Jj}x_j$ for some $x_j\in \Cc^\kappa$. Since $F$ preserves $\kappa$-filtered colimits, $F(x)\simeq \colimit_{j\in\Jj}F(x_j)$. Since $y$ is $\kappa$-compact by assumption and $\pi_0$ commutes with colimits by \cref{lem:HomotopyGroupsFilteredColimits}, the canonical map
	\begin{equation*}
		\colimit_{j\in\Jj}\pi_0\Hom_\Dd\bigl(y,F(x_j)\bigr)\overset{\cong}{\longrightarrow}\pi_0\Hom_\Dd\bigl(y,F(x)\bigr)
	\end{equation*}
	is a bijection. Hence $\beta\colon y\rightarrow F(x)$ factors over a map $\beta_j\colon y\rightarrow F(x_j)$ for some $j\in\Jj$. Then $x_j\rightarrow\colimit_{j\in\Jj}x_j\simeq x$ induces a morphism $(x_j,\beta_j\colon y\rightarrow F( x_j))\rightarrow (x,\beta\colon y\rightarrow F(x))$ in $\Cc_{y/}$. Since $x_j$ is $\kappa$-compact, we see that $(x_j,\beta_j\colon y\rightarrow F( x_j))\in\Tt$. This proves that there exists a morphism $t\rightarrow z$ for some $t\in\Tt$ and thus we've proved that $F\colon \Cc\rightarrow \Dd$ has a left adjoint.
	
	To prove the converse in the case where $\Cc$ and $\Dd$ are both accessible, just observe that if $F$ admits a left adjoint, then $F$ preserves all limits by \cref{lem:AdjointsPreserveColimits} and also all sufficiently filtered colimits by \cref{lem:RightAdjointsAccessible}. Finally, to show that the conditions in \cref{enum:AdjointFunctorTheoremLeft} and \cref{enum:AdjointFunctorTheoremRight} are satisfied in the case where $\Cc$ and $\Dd$ are presentable, the only non-obvious assertion is that for every $y\in\Dd$ there exists a regular cardinal $\kappa_y$ such that $y$ is $\kappa_y$-compact. But we've seen this in the proof of \cref{lem:RightAdjointsAccessible} already.
\end{proof}

\subsection{Lurie's magic \texorpdfstring{$\infty$}{Infinity}-category \texorpdfstring{$\cat{Pr}^\L$}{PrL}}\label{subsec:PrL}

To finish this appendix to \cref{sec:InftyCategoryTheory}, we'll introduce Lurie's $\infty$-category $\cat{Pr}^\L$. At first, it'll probably not be obvious to you why this construction is so useful, but hopefully you'll come to appreciate it more and more. Without further ado, here's the \enquote{definition}.
\begin{numpar}[\enquote{Definition}.]\label{par:PrL}
	The \emph{$\infty$-category of presentable $\infty\text{-}$categories and left adjoint functors} $\cat{Pr}^\L$ is the non-full sub-$\infty$-category of $\cat{Cat}_\infty$ spanned by the presentable $\infty$-categories and the left adjoint, or equivalently (by \cref{thm:AdjointFunctorTheorem}\cref{enum:AdjointFunctorTheoremLeft}), colimits-preserving functors.
\end{numpar}
As stated, \enquote{Definition}~\cref{par:PrL} doesn't make sense: $\cat{Cat}_\infty$ only contains small $\infty$-categories, but presentable $\infty$-categories usually aren't small. So to make \enquote{Definition}~\cref{par:PrL} work, we would need to assume two nested universes (of \enquote{small} and \enquote{large} sets) and define $\cat{Pr}^\L$ as a  non-full sub-$\infty$-category of the $\infty$-category $\widehat{\cat{Cat}}_\infty$ of all large $\infty$-categories (which is neither small nor large). But there's an alternative construction of $\Pr^\L$ that stays within ZFC; see \cref{par:PrLInZFC} below. For the moment, let's work with universes and assume $\widehat{\cat{Cat}}_\infty$ exists. Also note that $\cat{Pr}^\L$ is not even locally small: We have $\Hom_{\cat{Pr}^\L}(\Cc,\Dd)\simeq \core \Fun^\L(\Cc,\Dd)$; this is usually not an essential small anima.\footnote{However, $\Fun^\L(\Cc,\Dd)$ is at least locally small. To see this, write $\Cc\simeq \cat{Ind}_\kappa(\Cc^\kappa)$ for some regular cardinal $\kappa$. Using a similar argument as in \cref{lem:PrLKappa}\cref{enum:FunLKappa}, $\Fun^\L(\Cc,\Dd)$ can be identified with the full sub-$\infty$-category of $\Fun(\Cc^\kappa,\Dd)$ spanned by those functors that preserve $\kappa$-small colimits. This is clearly a locally small $\infty$-category.} %we'll ignore these set-theoretic issues and pretend that $\cat{Pr}^\L$ is a non-full sub-$\infty$-category of $\cat{Cat}_\infty$.

Let's begin by studying limits and colimits in $\cat{Pr}^\L$. To this end, we also consider the $\infty$-category $\cat{Pr}^\R$ of all presentable $\infty$-categories and right adjoint functors.

%The reason why people care so much about $\cat{Pr}^\L$ is that one can do algebra in it! In \cref{subsec:SymmetricMonoidal}, we'll define what a \emph{symmetric monoidal $\infty$-category} is. Lurie constructs a symmetric monoidal structure on $\cat{Pr}^\L$, called the \emph{Lurie tensor product}, which has a bunch of amazing properties. In particular, the Lurie tensor product allows us to do algebra not only on the level of rings~$R$, but also on the level of their derived $\infty$-categories $\Dd(R)$. We'll say a lot more about the Lurie tensor product in [TODO]. For now, as a preparation, let's study limits and colimits in $\cat{Pr}^\L$. Let $\cat{Pr}^\R$ be the $\infty$-category of presentable $\infty$-categories and right adjoint functors. Then \cref{cor:ExtractingAdjoints}\cref{enum:CatLCatR} restricts to an equivalence $\cat{Pr}^\L\simeq (\cat{Pr}^\R)^\op$. In particular, colimits in $\cat{Pr}^\L$ are just limits in $\cat{Pr}^\R$ and so it suffices to study limits.
\begin{lem}\label{lem:PrLColimits}
	The $\infty$-categories $\cat{Pr}^\L$ and $\cat{Pr}^\R$ have all small limits and colimits. The forgetful functors $\cat{Pr}^\L\rightarrow \widehat{\cat{Cat}}_\infty$ and $\cat{Pr}^\R\rightarrow \widehat{\cat{Cat}}_\infty$ preserve all small limits.
\end{lem}
To prove \cref{lem:PrLColimits}, we need two more technical lemmas. The first one has already been referenced several times before.
\begin{lem}\label{lem:HomInLimits}
	Let $\Cc_{(-)}\colon\Ii\rightarrow\cat{Cat}_\infty$ \embrace{or $\Cc_{(-)}\colon\Ii\rightarrow\widehat{\cat{Cat}}_\infty$} be a diagram of $\infty$-categories.
	\begin{alphanumerate}
		\item For every pair of objects $x,y\in\limit_{i\in\Ii}\Cc_i$ and their images $x_i,y_i\in \Cc_i$ under the projections $\pr_i\colon \limit_{i\in\Ii}\Cc_i\rightarrow \Cc_i$ there is a canonical equivalence\label{enum:HomInLimits}
		\begin{equation*}
			\Hom_{\limit_{i\in\Ii}\Cc_i}\left(x,y\right)\overset{\simeq}{\longrightarrow}\limit_{i\in\Ii}\Hom_{\Cc_i}\left(x_i,y_i\right)\,.
		\end{equation*}
		\item Let $F\colon \Jj\rightarrow \limit_{i\in\Ii}\Cc_i$ be a functor. Assume that all compositions ${\pr_i}\circ F\colon \Jj\rightarrow \Cc_i$ admit a colimit and that these colimits are preserved under $\Cc_i\rightarrow\Cc_j$ for all morphisms $i\rightarrow j$ in $\Ii$. Then $F$ admits a colimit and that colimit is preserved under $\pr_i\colon\limit_{i\in \Ii}\Cc_i\rightarrow \Cc_i$ for all $i\in \Ii$. A similar assertion is true for limits.\label{enum:ColimitsInLimits}
	\end{alphanumerate}
\end{lem}
\begin{proof}
	Let $\Cc$ be an $\infty$-category and $x,y\in\Cc$. Using $\Hom_{\cat{Cat}_\infty}(-,\Cc)\simeq\core\Fun(-,\Cc)$, we get a pullback diagram
	\begin{equation*}
		\begin{tikzcd}
			\Hom_\Cc(x,y)\rar\dar\drar[pullback] & \Hom_{\cat{Cat}_\infty}\bigl(\Delta^1,\Cc\bigr)\dar\\
			\{x\}\times\{y\}\rar & \Hom_{\cat{Cat}_\infty}(*\ \,*,\Cc)
		\end{tikzcd}
	\end{equation*}
	in $\cat{An}$. Now for every $\infty$-category $\Dd$, the functor $\Hom_{\cat{Cat}_\infty}(\Dd,-)\colon \cat{Cat}_\infty\rightarrow\cat{An}$ preserves arbitrary limits by \cref{cor:HomPreservesLimits}. Applying this for $\Dd\simeq \Delta^1$ and $\Dd\simeq *\ \,*$ and using that pullbacks commute with limits by the dual of \cref{lem:ColimitManipulations}, we deduce~\cref{enum:HomInLimits}.
	
	For~\cref{enum:ColimitsInLimits}, we only have to show that the colimit cocones $\Jj^\triangleright \rightarrow \Cc_i$ for all $i\in \Ii$ assemble into a functor $F^\triangleright\colon \Jj^\triangleright\rightarrow \limit_{i\in \Ii}\Cc_i$. If we can do this, then~\cref{enum:HomInLimits} combined with \cref{cor:HomPreservesColimits} and the fact that limits commute with limits (by the dual of \cref{lem:ColimitManipulations}) will show that $F^\triangleright$ is a colimit cocone. To construct $F^\triangleright$, we'll show a slightly stronger assertion: Consider the slice-$\infty$-category $(\cat{Cat}_\infty)_{\Jj/}$ and let $(\cat{Cat}_\infty)_{\Jj/}^{\colimit}$ be the non-full sub-$\infty$-category spanned by those objects $(\Jj\rightarrow \Cc)$ that admit a colimit and those morphisms that preserve this colimit. We wish to show that $(\cat{Cat}_\infty)_{\Jj/}^{\colimit}$ has all limits and that $(\cat{Cat}_\infty)_{\Jj/}^{\colimit}\rightarrow(\cat{Cat}_\infty)_{\Jj/}$ preserves all limits. By \cref{lem:ColimitsIffCoproductsAndPushouts}, it's enough to check this for products and pullbacks. So we can reduce the construction of $F^\triangleright\colon \Jj^\triangleright\rightarrow \limit_{i\in\Ii}\Cc_i$ to the case where $\limit_{i\in\Ii}\Cc_i$ is a product or a pullback.%
	\footnote{\label{footnote:LimitNonFullSubcategory}After constructing $F^\triangleright\colon \Jj^\triangleright\rightarrow \limit_{i\in\Ii}\Cc_i$ there's still something to show before we can conclude that $\limit_{i\in\Ii}\Cc_i$ (taken in $\cat{Cat}_\infty$ or $(\cat{Cat}_\infty)_{\Jj/}$, this doesn't matter by the dual of \cref{lem:ColimitsInSliceCategory}\cref{enum:LimitsInSlice}) is also the limit in $(\cat{Cat}_\infty)_{\Jj/}^{\colimit}$. The problem is that $(\cat{Cat}_\infty)_{\Jj/}^{\colimit}$ is only a \emph{non-full} sub-$\infty$-category of $(\cat{Cat}_\infty)_{\Jj/}$. But this is easily fixed. Using \cref{lem:NonFullSubcategory} and the universal property of $\limit_{i\in\Ii}\Cc_i$ in $(\cat{Cat}_\infty)_{\Jj/}$, verifying the corresponding universal property in $(\cat{Cat}_\infty)_{\Jj/}^{\colimit}$ reduces to a matching of path components. For example, in the case of a product, we have to show that the projections $\pr_i\colon \prod_{i\in I}\Cc_i\rightarrow \Cc_i$ preserve the colimit over $\Jj$; and furthermore, that a functor $\Dd\rightarrow \prod_{i\in I}\Cc_i$ in $(\cat{Cat}_\infty)_{\Jj/}$ preserves the colimit over $\Jj$ if and only if the same is true for each $\Dd\rightarrow \Cc_i$. A similar assertion would be to show for pullbacks. Both are straightforward.}
	
	In the product case it's clear how to construct $F^\triangleright\colon \Jj^\triangleright \rightarrow \prod_{i\in I}\Cc_i$. So let's consider a pullback $\Cc_0\times_{\Cc_2}\Cc_1$ of $\alpha_{0}\colon \Cc_0\rightarrow \Cc_2$ and $\alpha_{1}\colon \Cc_1\rightarrow\Cc_2$. Choose colimit cocones $F_0^\triangleright\colon \Jj^\triangleright \rightarrow \Cc_0$ and $F_1^\triangleright\colon \Jj^\triangleright \rightarrow \Cc_1$. Choose a composition $F_2^\triangleright\coloneqq \alpha_0\circ F_0^\triangleright$ and a composition $\alpha_{1}\circ F_1^\triangleright$. Then $F_2^\triangleright$ and $\alpha_{1}\circ F_1^\triangleright$ are both colimit cocones of the given functor ${\pr_2}\circ F\colon \Jj\rightarrow \Cc_2$. Since colimit cocones are unique up to equivalence, there must be a natural equivalence $\eta\colon F_2^\triangleright\overset{\simeq}{\Longrightarrow}\alpha_{1}\circ F_1^\triangleright$. Thus, we obtain a commutative diagram
	\begin{equation*}
		\begin{tikzcd}
			\Jj^\triangleright\eqar[r]\drar[commutes]\dar["F_0^\triangleright"'] & \Jj^\triangleright\eqar[r]\drar[commutes]\dar["F_2^\triangleright"'] & \Jj^\triangleright\dar["F_1^\triangleright"]\\
			\Cc_0\rar["\alpha_0"] & \Cc_2 & \Cc_1\lar["\alpha_1"']
		\end{tikzcd}
	\end{equation*}
	in $\cat{Cat}_\infty$ ($\eta$ is precisely what makes the right square commute). This diagram constitutes a natural transformation $\const \Jj^\triangleright \Rightarrow \Cc_{(-)}$ in $\Fun(\Lambda_2^2,\Cc)$, which induces the desired functor $\Jj^\triangleright\rightarrow \Cc_0\times_{\Cc_2}\Cc_2$. As argued above, this is automatically a colimit cone.
\end{proof}
\begin{lem}\label{lem:AccessibilityOfFibreProducts}
	Let $\cat{Acc}\subseteq \widehat{\cat{Cat}}_\infty$ be the non-full sub-$\infty$-category spanned by accessible $\infty$-categories and functors that preserve sufficiently filtered colimits. Then $\cat{Acc}$ has all limits and $\cat{Acc}\rightarrow \widehat{\cat{Cat}}_\infty$ preserves all limits.
\end{lem}
\begin{proof}[Proof sketch]
	By \cref{lem:ColimitsIffCoproductsAndPushouts} we can reduce to products and pullbacks. We start with products. Let $(\Cc_i)_{i\in I}$ be an collection of accessible $\infty$-categories. We must to show that $\Cc\coloneqq \prod_{i\in I}\Cc_i$ is accessible again and that the projections $\pr_i\colon \Cc\rightarrow \Cc_i$ preserve sufficiently filtered colimits. The latter is immediate from \cref{lem:HomInLimits}\cref{enum:ColimitsInLimits}. To prove that $\Cc$ is accessible, we use \cref{lem:Accessible} to choose a sufficiently large regular cardinal $\kappa$ such that $\kappa >\abs*{I}$ and all $\Cc_i$ are $\kappa$-accessible. If $x=(x_i)_{i\in I}$ is an object of $\Cc$ such that each $x_i\in \Cc_i$ is $\kappa$-compact, then $x$ is $\kappa$-compact too: Indeed, this follows from \cref{lem:HomInLimits}\cref{enum:HomInLimits} and the fact that $\kappa$-filtered colimits commute with $I$-indexed products in $\cat{An}$ by \cref{lem:FilteredColimitsPreserveFiniteLimits}. Now let $y=(y_i)_{i\in I}$ be another object of $\Cc$. For all $i\in I$, we can write $y_i\simeq \colimit_{j\in\Jj_i}x_{i}(j)$ as a $\kappa$-filtered colimit of $\kappa$-compact objects. By the same trick as in the proof of \cref{lem:Presentable}, we can replace each $\Jj_i$ by $\Jj\coloneqq \prod_{i\in I}\Jj_i$. Putting $x(j)\coloneqq (x_i(j))_{i\in I}\in\Cc$ we deduce that $y\simeq \colimit_{j\in \Jj}x(j)$ is expressible as a $\kappa$-filtered colimit of $\kappa$-compact objects by \cref{lem:HomInLimits}\cref{enum:ColimitsInLimits}. This shows that $\Cc$ is accessible.
	
	Now consider a pullback $\Cc\coloneqq \Cc_0\times_{\Cc_2}\Cc_1$ of accessible $\infty$-categories along functors $\alpha_0\colon \Cc_0\rightarrow \Cc_2$ and $\alpha_1\colon \Cc_1\rightarrow \Cc_2$ that preserve sufficiently filtered colimits. Again, \cref{lem:HomInLimits}\cref{enum:ColimitsInLimits} ensures that $\pr_i\colon \Cc\rightarrow \Cc_i$ preserve sufficiently filtered colimits, so it's enough to show that $\Cc$ is accessible. Choose a sufficiently large regular cardinal $\kappa$ such that $\Cc_0$, $\Cc_1$, $\Cc_2$ are $\kappa$-accessible and $\alpha_0$, $\alpha_1$ preserve $\kappa$-filtered colimits. Then $\Cc_0^\kappa$ and $\Cc_1^\kappa$ are small $\infty$-categories. Choose $\tau$ large enough such that the images of $\alpha_0|_{\Cc_0^\kappa}\colon \Cc_0^\kappa\rightarrow \Cc_2$ and $\alpha_1|_{\Cc_1^\kappa}\colon \Cc_1^\kappa\rightarrow \Cc_2$ land inside $\tau$-compact objects of $\Cc_2$. Let $\ov\Cc_0\subseteq \Cc_0$ be the full sub-$\infty$-category spanned by all retracts of objects that can be written as a $\tau$-small $\kappa$-filtered colimit of objects in $\Cc_0^\kappa$. Enlarging $\tau$ if necessary, we can ensure $\ov\Cc_0\simeq \Cc_0^\tau$. Indeed, it's clear that all objects of $\ov\Cc_0$ are $\tau$-compact in $\Cc_0$. Conversely, the proof of \cref{lem:Accessible} shows that, for sufficiently large $\tau$, every object of $\Cc_0$ can be written as a $\tau$-filtered colimit of objects in $\ov\Cc_0$. By the argument in the proof of \cref{lem:KappaCompactlyGenerated}, all $\tau$-compact objects of $\Cc_0$ must then be retracts of objects in $\ov\Cc_0$, as desired.
	
	By assumption, $\alpha_0\colon \Cc_0\rightarrow \Cc_1$ preserves $\kappa$-filtered colimits and retracts. Furthermore, $\Cc_2^\tau$ is closed under $\tau$-small colimits and under retracts. We deduce that for sufficiently large $\tau$, the functor $\alpha_0$ restricts to a functor $\alpha_0\colon \Cc_0^\tau\rightarrow \Cc_2^\tau$ and likewise $\alpha_1$ restricts to a functor $\alpha_1\colon \Cc_1^\tau\rightarrow \Cc_2^\tau$. Now consider the pullback $\ov\Cc\coloneqq \Cc_0^\tau\times_{\Cc_2^\tau}\Cc_1^\tau$. We wish to show that $\Cc\simeq \cat{Ind}_\tau(\ov\Cc)$. By construction, $\ov\Cc$ is a full sub-$\infty$-category of $\Cc$, hence according to \cref{lem:KappaCompactlyGenerated}\cref{enum:DGeneratedUnderFilteredColimits} it will be enough to show that every object in $\Cc$ can be written as a $\tau$-filtered colimit of objects in $\ov\Cc$. So let $x\in \Cc$ and let $x_i\coloneqq \pr_i(x)$ be its projections to $\Cc_i$ for $i=0,1,2$. We claim:
	\begin{alphanumerate}\itshape
		\item[\boxtimes_1] $\ov\Cc_{/x}$ is $\tau$-filtered.\label{claim:tauFiltered}
		\item[\boxtimes_2] The projections $\pr_i\colon \ov\Cc_{/x}\rightarrow \Cc_{i/x_i}^\tau$ for $i=0,1$ are coinitial functors.\label{claim:ProjectionsCoinitial}
	\end{alphanumerate}
	Together, these claims imply that the canonical morphism $\colimit(s\colon \ov\Cc_{/x}\rightarrow \Cc)\rightarrow x$ is an equivalence, so that $x$ can be written as a $\tau$-filtered colimits of objects in $\ov\Cc$, as desired. Indeed, by definition of $\Cc$ as a pullback, the projections $\pr_i$ for $i=0,1$ are jointly conservative and they preserve $\kappa$-filtered colimits by \cref{lem:HomInLimits}\cref{enum:ColimitsInLimits}. In particular, if \cref{claim:tauFiltered} holds, then $\pr_i$ preserves $\ov\Cc$-indexed colimits. Using \cref{claim:ProjectionsCoinitial}, we conclude
	\begin{equation*}
		\pr_i\Bigl(\colimit\left(\ov\Cc_{/x}\rightarrow \Cc\right)\Bigr)\simeq  \colimit\bigl(\Cc_{i/x_i}^\tau\rightarrow \Cc_i\bigr)\simeq x_i
	\end{equation*}
	for $i=0,1$, as desired.
	
	It remains to prove the two claims. It's straightforward to check that taking slice-$\infty$-categories commutes with pullbacks. Therefore we have a pullback diagram
	\begin{equation*}
		\begin{tikzcd}
			\ov\Cc_{/x}\rar["\pr_1"]\dar["\pr_0"']\drar[pullback] & \Cc_{1/x_1}^\tau\dar["\alpha_1"]\\
			\Cc_{0/x_0}^\tau\rar["\alpha_0"] & \Cc_{2/x_2}^\tau
		\end{tikzcd}
	\end{equation*}
	Since $\Cc_i\simeq \cat{Ind}_\tau(\Cc_i^\tau)$ for $i=0,1,2$, the argument from \cref{con:Ind} shows that each $\Cc_{i/x_i}^\tau$ is $\tau$-filtered. Furthermore, the legs of the pullback $\alpha_i\colon \Cc_{i/x_i}^\tau\rightarrow \Cc_{2/x_2}^\tau$ for $i=0,1$ are coinitial functors. Indeed, in general, if $F\colon \Uu\rightarrow \Vv$ is a functor of small $\infty$-categories, $u\in\cat{Ind}_\tau(\Uu)$ and $v$ is the image of $u$ under $\cat{Ind}_\tau(\Uu)\rightarrow \cat{Ind}_\tau(\Vv)$, then $\Uu_{/u}\rightarrow \Vv_{/v}$ is coinitial. To see this, observe that $\cat{Ind}_\tau(\Uu)\rightarrow \cat{Ind}_\tau(\Vv)$ fits into a diagram
	\begin{equation*}
		\begin{tikzcd}
			\Uu\dar["F"']\rar & \cat{Ind}_\tau(\Uu)\dar\rar & \PSh(\Uu)\dar["F_!"]\\
			\Vv\rar & \cat{Ind}_\tau(\Vv)\rar & \PSh(\Vv)
		\end{tikzcd}
	\end{equation*}
	(this follows from \cref{lem:Ind}\cref{enum:IndFreelyGenerated}, \cref{thm:PShFreeCocompletion}, and the fact that $\cat{Ind}_\tau(\Vv)\rightarrow \PSh(\Vv)$ preserves $\tau$-filtered colimits by \cref{lem:Ind}\cref{enum:IndGeneratedUnderFilteredColimits}). By \cref{con:Ind}, $\Uu_{/u}\rightarrow \Uu$ is the right fibration associated to the presheaf $u\colon \Uu^\op\rightarrow \cat{An}$. Hence coinitiality of $\Uu_{/u}\rightarrow \Vv_{/v}$ follows from \cref{lem:KanExtensionForRight}\cref{enum:RightPullbackLeftAdjoint}.
	
	To prove \cref{claim:tauFiltered}, let $\varphi\colon \Ii\rightarrow \ov\Cc_{/x}$ be any functor from a $\tau$-small $\infty$-category. Since $\Cc_{i/x_i}^\tau$ are $\tau$-filtered, we may extend ${\pr_i}\circ\varphi$ to functors $\varphi_i\colon \Ii^\triangleright \rightarrow \Cc_{i/x_i}^\tau$ for $i=0,1$. The problem is that we don't necessarily have an equivalence $\alpha_1\circ\varphi_0\simeq \alpha_0^\tau\circ \varphi_1$, so $\varphi_0$ and $\varphi_1$ can't necessarily be combined into an extension $\varphi^\triangleright\colon \Ii^\triangleright \rightarrow \ov\Cc_{/x}$ of $\varphi$. Instead, $\alpha_1\circ\varphi_0$ and $\alpha_0\circ \varphi_1$ define a functor $\psi\colon \Ii^\triangleright\sqcup_\Ii\Ii^\triangleright\rightarrow \Cc_{2/x_2}^\tau$. Since the target is $\tau$-filtered, we can extend $\psi$ to a functor $\psi^\triangleright\colon (\Ii^\triangleright\sqcup_\Ii\Ii^\triangleright)^\triangleright\rightarrow \Cc_{2/x_2}^\tau$. Restricting along $\Ii\rightarrow \Ii^\triangleright\sqcup_\Ii\Ii^\triangleright$ yields a functor $\varphi_2\colon \Ii^\triangleright \rightarrow \Cc_{2/x_2}^\tau$ which extends ${\pr_2}\circ\varphi$. The problem with this extension is that it doesn't necessarily factor through $\Cc_{0/x_0}^\tau$ and $\Cc_{1/x_1}^\tau$. 
	
	But at least $\psi^\triangleright$ defines natural transformations $\eta_0\colon\alpha_1\circ\varphi_0\Rightarrow \eta_2$ and $\eta_1\colon\alpha_0\circ\varphi_1\Rightarrow \varphi_2$. For $i=0,1,2$ let $y_i$ be the image of the tip $*\in\Ii^\triangleright$ under $\varphi_i$. Then $\psi^\triangleright$ defines morphisms $\alpha_i(y_i)\rightarrow y_2$ for $i=0,1$ and the natural transformations $\eta_i$ are necessarily given by postcomposition with these morphisms. Since $\alpha_i\colon \Cc_{i/x_i}^\tau\rightarrow \Cc_{2/x_2}^\tau$ is coinitial, we find $y_i'\in \Cc_{i/x_i}^\tau$ and a morphism $y_2\rightarrow \alpha_i(y_i')$ in $\Cc_{2/x_2}^\tau$ for $i=0,1$. Applying the same argument to $\alpha_i\colon(\Cc_{i/x_i}^\tau)_{y_i/}\rightarrow (\Cc_{2/x_2}^\tau)_{\alpha_i(y_i)/}$, which is still coinitial by \cref{lem:FilteredCofinal}, we can even arrange that a morphism $y_i\rightarrow y_i'$ exists in such a way that its image under $\alpha_i$ is the composition $\alpha_i(y_i)\rightarrow y_2\rightarrow \alpha_i(y_i')$. Postcomposition with $y_i\rightarrow y_i'$ yields another extension $\varphi_i'\colon \Ii^\triangleright\rightarrow \Cc_{i/x_i}^\tau$ of ${\pr_i}\circ\varphi$ together with a natural transformation $\vartheta_i\colon \varphi_i\Rightarrow \varphi_i'$, and we are in the same situation as before.
	
	Iterating the construction $\varphi_i\mapsto \varphi_i'$ transfinitely%
	%
	\footnote{The above construction only takes care of the transfinite induction step for successor ordinals. But the case of limit ordinals is entirely analogous.}
	many times, we obtain $\varphi_i^{(\gamma)}\colon \Ii^\triangleright\rightarrow \Cc_{i/x_i}^\tau$ for all ordinals $\gamma<\kappa$. Now define
	\begin{equation*}
		\varphi_i^\triangleright\coloneqq \colimit_{\gamma<\kappa}\varphi_i^{(\gamma)}\colon \Ii^\triangleright \rightarrow \Cc_{i/x_i}^\tau\,.
	\end{equation*}
	This colimit exists in $\Fun(\Ii^\triangleright,\Cc_{i/x_i}^\tau)$ since $\{\gamma<\kappa\}$ is $\kappa$-filtered and $\tau$-small. Furthermore, using that $\alpha_i\colon \Cc_{i/x_i}^\tau\rightarrow \Cc_{2/x_2}^\tau$ preserve $\kappa$-filtered $\tau$-small colimits, it's straightforward to check that now indeed $\alpha_1\circ\varphi_0^\triangleright\simeq \alpha_0\circ\varphi_1^\triangleright$ holds, so we obtain our desired extension $\varphi^\triangleright\colon \Ii^\triangleright\rightarrow \ov\Cc_{/x}$. This finishes the proof of \cref{claim:tauFiltered}.
	
	To show that $\pr_0\colon \ov\Cc_{/x}\rightarrow \Cc_{0/x_0}^\tau$ is coinitial, we use \cref{thm:JoyalsQuillenA}\cref{enum:WeaklyContractible}: For all $z_0\in\Cc_{0/x_0}^\tau$, we wish to show that $\ov\Cc_{/x}\times_{\Cc_{0/x_0}^\tau}(\Cc_{0/x_0}^\tau)_{z_0/}$ is weakly contractible. We'll even show that it is $\tau$-filtered, which is enough by \cref{lem:FilteredCofinal}. To this end, consider any functor
	\begin{equation*}
		\varphi\colon \Ii\longrightarrow \ov\Cc_{/x}\times_{\Cc_{0/x_0}^\tau}\bigl(\Cc_{0/x_0}^\tau\bigr)_{z_0/}
	\end{equation*}
	from a $\tau$-small $\infty$-category. Equivalently, $\varphi$ is a functor $\ov\varphi\colon \Ii\rightarrow \ov\Cc_{/x}$ together with an extension of ${\pr_0}\circ \ov\varphi$ to a functor $\varphi_0^\triangleleft\colon \Ii^\triangleleft\rightarrow \Cc_{0/x_0}^\tau$ that sends the tip $*\in \Ii^\triangleleft$ to $z_0$. By \cref{claim:tauFiltered}, we find an extension $\ov\varphi^\triangleright \colon \Ii^\triangleright\rightarrow \ov\Cc_{/x}$. The pushout of ${\pr_0}\circ\ov\varphi^\triangleright$ and $\ov\varphi_0^\triangleleft$ defines a functor $(\varphi_0^\triangleright)^\triangleleft\colon (\Ii^\triangleright)^\triangleleft \rightarrow \Cc_{0/x_0}^\tau$. Together, $\ov\varphi^\triangleright$ and $(\varphi_0^\triangleright)^\triangleleft$ define the desired extension $\varphi^\triangleright$ of $\varphi$. This shows that $\pr_0\colon \ov\Cc_{/x}\rightarrow \Cc_{0/x_0}^\tau$ is coinitial. The argument for $\pr_1\colon \ov\Cc_{/x}\rightarrow \Cc_{1/x_1}^\tau$ is analogous and thus we've shown \cref{claim:ProjectionsCoinitial}.
\end{proof}
\begin{proof}[Proof sketch of \cref{lem:PrLColimits}]
	The equivalence from \cref{cor:ExtractingAdjoints}\cref{enum:CatLCatR} (applied to $\widehat{\cat{Cat}}_\infty$ rather than $\cat{Cat}_\infty$) restricts to an equivalence $\cat{Pr}^\L\simeq (\cat{Pr}^\R)^\op$. In particular, colimits in $\cat{Pr}^\L$ are just limits in $\cat{Pr}^\R$ and vice versa. So it suffices to study limits in either case. By \cref{lem:ColimitsIffCoproductsAndPushouts} we can reduce to products and pullbacks.
	
	We start with products. Let $(\Cc_i)_{i\in I}$ be an collection of presentable $\infty$-categories and let $\Cc\coloneqq \prod_{i\in I}\Cc_i$. From \cref{lem:AccessibilityOfFibreProducts} we know that $\Cc$ is accessible and from \cref{lem:HomInLimits}\cref{enum:ColimitsInLimits} (and its dual) we know that $\Cc$ has all colimits and that the projections $\pr_i\colon \Cc\rightarrow \Cc_i$ preserve all limits and colimits. It's then straightforward to verify that $\Cc$ satisfies the universal property of the product in both $\cat{Pr}^\L$ and $\cat{Pr}^\R$ (compare this to footnote~\cref{footnote:LimitNonFullSubcategory} on page~\cpageref{footnote:LimitNonFullSubcategory}).
	
	It remains to do pullbacks. If $\Cc_0\rightarrow \Cc_2\leftarrow \Cc_1$ are functors in $\cat{Pr}^\L$ or $\cat{Pr}^\R$, then they preserve sufficiently filtered colimits (for $\cat{Pr}^\R$ this needs \cref{lem:RightAdjointsAccessible}) and so the pullback $\Dd\coloneqq\Cc_0\times_{\Cc_2}\Cc_1$ is accessible by \cref{lem:AccessibilityOfFibreProducts}. In the case of $\cat{Pr}^\L$, \cref{lem:HomInLimits}\cref{enum:ColimitsInLimits} shows that $\Cc$ has all colimits, so it is presentable, and that the projections $\pr_i\colon \Cc\rightarrow \Cc_i$ preserve all colimits, so they are functors in $\cat{Pr}^\L$. The required universal property in $\cat{Pr}^\L$ is then straightforward to verify. In the case of $\cat{Pr}^\R$, \cref{lem:HomInLimits}\cref{enum:ColimitsInLimits} shows that $\Cc$ has all limits, hence it is presentable by \cref{cor:PresentableComplete}\cref{enum:AccessibleCocomplete}, and that the projections $\pr_i\colon \Cc\rightarrow \Cc_i$ preserve all limits and sufficiently filtered colimits, so they are functors in $\cat{Pr}^\R$ by \cref{thm:AdjointFunctorTheorem}\cref{enum:AdjointFunctorTheoremRight}. Again, the required universal property in $\cat{Pr}^\R$ is then straightforward to verify.
\end{proof}
Let us now explain how to construct $\cat{Pr}^\L$ in ZFC. To this end, we need to introduce a variant of $\cat{Pr}^\L$. As we'll see in \cref{lem:Mindblow}, this variant has a truly mindblowing property, which makes it quite interesting on its own.
\begin{defi}\label{def:PrKappa}
	Let $\kappa$ be a regular cardinal.
	\begin{alphanumerate}
		\item A presentable $\infty$-category is called \emph{$\kappa$-compactly generated} if it is $\kappa$-accessible, that is, of the form $\Cc\simeq \cat{Ind}_\kappa(\Cc^\kappa)$, where $\Cc^\kappa\subseteq \Cc$ is the full sub-$\infty$-category spanned by the $\kappa$-compact objects.
		\item We let $\cat{Pr}_\kappa^\L$ be the non-full sub-$\infty$-category of $\cat{Pr}^\L$ spanned by the $\kappa$-compactly generated $\infty$-categories and those left adjoint functors that also preserve $\kappa$-compact objects.
	\end{alphanumerate} 
\end{defi}
\begin{lem}\label{lem:LeftAdjointPreservesCompactsRightAdjointPreservesFiltered}
	A left adjoint functor $F\colon \Cc\rightarrow \Dd$ between presentable $\infty$-categories preserves $\kappa$-compact objects if the right adjoint $G\colon \Dd\rightarrow \Cc$ preserves $\kappa$-filtered colimits. The converse is true as well, provided $\Cc$ is $\kappa$-compactly generated.
\end{lem}
\begin{proof}
	Let $x\in \Cc$ be $\kappa$-compact and let $y_{(-)}\colon \Ii\rightarrow \Dd$ be a $\kappa$-filtered diagram. If $G$ preserves $\kappa$-filtered colimits, then
	\begin{align*}
		\Hom_\Dd\Bigl(F(x),\colimit_{i\in\Ii}y_i\Bigr)&\simeq  \Hom_\Cc\Bigl(x,\colimit_{i\in\Ii}G(y_i)\Bigr)\\
		&\simeq \colimit_{i\in\Ii}\Hom_\Cc\bigl(x,G(y_i)\bigr)\\
		&\simeq \colimit_{i\in\Ii}\Hom_\Dd\bigl(F(x),y_i\bigr)\,,
	\end{align*}
	proving that $F(x)$ is $\kappa$-compact. Conversely, if $F$ preserves $\kappa$-compact objects, then the same calculation run backwards shows that the canonical morphism $\colimit_{i\in\Ii}G(y_i)\rightarrow G(\colimit_{i\in \Ii}y_i)$ becomes an equivalence after applying the functors $\Hom_\Cc(x,-)\colon \Cc\rightarrow \cat{An}$ for every $\kappa$-compact object $x\in \Cc^\kappa$. These functors are jointly conservative if $\Cc$ is $\kappa$-compactly generated. 
\end{proof}
\begin{numpar}[Constructing $\cat{Pr}^\L$ in ZFC.]\label{par:PrLInZFC}
	By \cref{rem:Accessible}, we have an inclusion $\cat{Pr}_\kappa^\L\subseteq \cat{Pr}_\lambda^\L$ of non-full sub-$\infty$-categories of $\cat{Pr}^\L$ for all regular cardinals $\lambda>\kappa$. By \cref{lem:Accessible}, every presentable $\infty$-category is $\kappa$-compactly generated for all sufficiently large regular cardinals $\kappa$. Furthermore, by \cref{lem:LeftAdjointPreservesCompactsRightAdjointPreservesFiltered} and \cref{lem:RightAdjointsAccessible}, every left adjoint functor $F\colon \Cc\rightarrow \Dd$ between presentable $\infty$-categories preserves $\kappa$-compact objects for sufficiently large $\kappa$. Therefore we can write
	\begin{equation*}
		\cat{Pr}^\L\simeq \bigcup_{\kappa}\cat{Pr}^\L_\kappa\,,
	\end{equation*}
	where the union is taken over all cardinals which are small with respect to our two nested universes. The union can be made precise as a colimit in $\widehat{\cat{Cat}}_\infty$, the $\infty$-category of all large $\infty$-categories, but it can also be viewed as simply a union of simplices in every degree.  In any case, we see that every functor $T\colon \Ii\rightarrow \cat{Pr}^\L$ from a small $\infty$-category $\Ii$ factors through $\cat{Pr}_\kappa^\L$ for all sufficiently large cardinals $\kappa$.
	
	So it suffices to explain how $\cat{Pr}_\kappa^\L$ can be constructed in ZFC. In \cref{lem:PrLKappa}\cref{enum:PrLKappaInZFC} below, we see that, for uncountable $\kappa$, $\cat{Pr}_\kappa^\L$ can be identified with a non-full sub-$\infty$-category of $\cat{Cat}_\infty$ in a non-trivial way. This can be viewed as an alternative construction of $\cat{Pr}_\kappa^\L$, thus providing a construction within the confines of ZFC.
\end{numpar}
\begin{lem}\label{lem:PrLKappa}
	Let $\kappa$ be an uncountable regular cardinal and let $\cat{Cat}_\infty^{\kappa\mhyph\!\colimit}\subseteq \cat{Cat}_\infty$ be the non-full sub-$\infty$-category spanned by those small $\infty$-categories that have all $\kappa$-small colimits and those functors that preserve all $\kappa$-small colimits. Let $(-)^\kappa$ be the functor that sends a $\kappa$-compactly generated presentable $\infty$-category $\Dd$ to its full sub-$\infty$-category $\Dd^\kappa$ spanned by the $\kappa$-compact objects.%
	%
	\footnote{We should explain how to construct this functor. In general, given a functor $F\colon \Cc\rightarrow \cat{Cat}_\infty$, it's easy to construct \emph{subfunctors} of $F$. Indeed, suppose for every $x\in \Cc$ we're given a full sub-$\infty$-category $F_0(x)\subseteq F(x)$ such that for every morphism $\alpha\colon x\rightarrow y$ in $F$, the functor $F(\alpha)\colon F(x)\rightarrow F(y)$ restricts to a functor $F_0(x)\rightarrow F_0(y)$. In this case, we automatically get a functor $F_0\colon \Cc\rightarrow\cat{Cat}_\infty$ together with a natural transformation $F_0\Rightarrow F$. Indeed, let $p\colon \Uu\rightarrow \Cc$ be the unstraightening of $F$ and let $\Uu_0\subseteq \Uu$ be the full sub-$\infty$-category spanned fibrewise by $F_0(x)\subseteq F(x)$ for all $x\in \Cc$. Then $p_0\colon \Uu_0\rightarrow\Cc$ given as the restriction of $p$ is still a cocartesian fibration, since our conditions on $F_0$ precisely ensure that $\Uu_0$ is closed under $p$-cocartesian lifts in $\Uu$. Thus, we can define $F_0$ as the straightening of $p_0$. In the case at hand, the identity on $\cat{Pr}_\kappa^\L$ can be viewed as a functor $\cat{Pr}_\kappa^\L\rightarrow \widehat{\cat{Cat}}_\infty$ and we can construct $(-)^\kappa$ as a subfunctor of it.}%
	%
	\begin{alphanumerate}
		\item If $\Cc$ is a small $\infty$-category with all $\kappa$-small colimits and $\Dd$ is a presentable $\infty$-category, then restriction along $\Yo_\Cc\colon \Cc\rightarrow \cat{Ind}_\kappa(\Cc)$ induces an equivalence of $\infty$-categories\label{enum:FunLKappa}
		\begin{equation*}
			\Yo_\Cc^*\colon \Fun_\kappa^\L\bigl(\cat{Ind}_\kappa(\Cc),\Dd\bigr)\overset{\simeq}{\longrightarrow} \Fun^{\kappa\mhyph\!\colimit}(\Cc,\Dd^\kappa)\,.
		\end{equation*}
		Here $\Fun^\L_\kappa\subseteq \Fun$ is the full sub-$\infty$-category spanned by left adjoint functors that preserve $\kappa$-compact objects and $\Fun^{\kappa\mhyph\!\colimit}\subseteq \Fun$ is spanned by $\kappa$-small colimits-preserving functors.
		\item If $\Cc$ is a small $\infty$-category with all $\kappa$-small colimits, then $\Cc\simeq \cat{Ind}_\kappa(\Cc)^\kappa$.\label{enum:KappaCompactObjects}
		\item The functor $(-)^\kappa$ induces an equivalence of $\infty$-categories\label{enum:PrLKappaInZFC}
		\begin{equation*}
			(-)^\kappa\colon \cat{Pr}_\kappa^\L\overset{\simeq}{\longrightarrow}\cat{Cat}_\infty^{\kappa\mhyph\!\colimit}\,.
		\end{equation*} 
	\end{alphanumerate}
\end{lem}
\begin{proof}
	We begin with~\cref{enum:KappaCompactObjects}. It's clear that the objects $\{\Yo_\Cc(x)\ \vert\ x\in\Cc\}$ form a set of $\kappa$-compact generators of $\cat{Ind}_\kappa(\Cc)$. As we've seen in the proof of \cref{lem:KappaCompactlyGenerated}, this means that every $\kappa$-compact object of $\cat{Ind}_\kappa(\Cc)$ is a retract of an object in $\{\Yo_\Cc(x)\ \vert\ x\in\Cc\}$. As we've seen in footnote~\cref{footnote:ReflectionTheorem} on \cpageref{footnote:ReflectionTheorem}, retracts can be written as countable colimits. Furthermore, we've seen in the proof of \cref{lem:Ind} that $\Yo_\Cc\colon \Cc\rightarrow\cat{Ind}_\kappa(\Cc)$ preserves $\kappa$-small colimits; in particular, $\Yo_\Cc$ preserves countable colimits, as $\kappa$ is assumed uncountable. Since we assume that $\Cc$ has all $\kappa$-small colimits, it follows that $\{\Yo_\Cc(x)\ \vert\ x\in\Cc\}$ is closed under retracts and thus comprises all $\kappa$-compact objects. The claim follows.
	
	To prove \cref{enum:FunLKappa}, our starting point is the equivalence $\Fun^{\kappa\mhyph\mathrm{filt}}(\cat{Ind}_\kappa(\Cc),\Dd)\simeq \Fun(\Cc,\Dd)$ from \cref{lem:Ind}\cref{enum:IndFreelyGenerated}. So we only have to match full sub-$\infty$-categories on either side. Suppose a functor $F\colon \cat{Ind}_\kappa(\Cc)\rightarrow \Dd$ preserves $\kappa$-compact objects. Then the associated functor $\Cc\rightarrow \Dd$ factors through a functor $G\colon \Cc\rightarrow \Dd^\kappa$. Furthermore, $\Dd^\kappa\subseteq \Dd$ is closed under $\kappa$-small colimits and so is $\Cc\simeq \cat{Ind}_\kappa(\Cc)^\kappa\subseteq \cat{Ind}_\kappa(\Cc)$ by \cref{enum:KappaCompactObjects}. Thus, if $F$ preserves all colimits, then $G$ preserves $\kappa$-small colimits. Conversely, suppose we're given a functor $G\colon \Cc\rightarrow \Dd^\kappa$ that preserves $\kappa$-small colimits. Let $F\colon \cat{Ind}_\kappa(\Cc)\rightarrow \Dd$ be the associated functor. By construction, $F$ preserves $\kappa$-compact objects and $\kappa$-filtered colimits. Furthermore, we've seen in the proof of \cref{lem:Presentable} that arbitrary colimits in $\cat{Ind}_\kappa(\Cc)$ can be built from $\kappa$-filtered colimits as well as $\kappa$-small colimits of objects in the image of $\Yo_\Cc\colon \Cc\rightarrow \cat{Ind}_\kappa(\Cc)$. Thus $F$ preserves all colimits. This finishes the proof of \cref{enum:FunLKappa}.
	
	Finally, \cref{enum:PrLKappaInZFC} is a formal consequence:~\cref{enum:KappaCompactObjects} shows that $(-)^\kappa$ is essentially surjective, and~\cref{enum:FunLKappa}, together with \cref{lem:Presentable}\cref{enum:DCHasKappaSmallColimits}, shows that $(-)^\kappa$ is fully faithful.
\end{proof}
\begin{cor}\label{cor:PrLKappaColimits}
	Let $\lambda >\kappa$ be regular cardinals.
	\begin{alphanumerate}
		\item The inclusion $\cat{Pr}_\kappa^\L\subseteq \cat{Pr}_\lambda^\L$ admits a right adjoint. On objects, it sends $\Dd\in\cat{Pr}_\lambda^\L$ to $\cat{Ind}_\kappa(\Dd^\lambda)$.\label{enum:PrLKappaLambda}
		\item The $\infty$-category $\cat{Pr}_\kappa^\L$ has all small limits and colimits. The forgetful functor $\cat{Pr}_\kappa^\L\rightarrow \cat{Pr}^\L$ preserves all small colimits.\label{enum:PrLKappaColimits}%
		%
		\footnote{\label{footnote:ProductsInPrLKappa}It's true that any product of $\kappa$-compactly generated $\infty$-categories in $\cat{Pr}^\L$ is $\kappa$-compactly generated again; we'll see this in the proof of \cref{cor:PrLKappaColimits}\cref{enum:PrLKappaColimits}. But, confusingly, the product in $\cat{Pr}^\L$ is usually not the product in $\cat{Pr}_\kappa^\L$. The reason is that preservation of limits or colimits can be detected factor-wise, but not preservation of $\kappa$-compact objects.}
	\end{alphanumerate}
\end{cor}
\begin{proof}
	We start with \cref{enum:PrLKappaLambda}. Let $\Dd\in \cat{Pr}_\lambda^\L$. By \cref{lem:Ind}\cref{enum:IndFreelyGenerated}, the identity functor $\id_{\Dd^\lambda}\colon \Dd^\lambda\rightarrow \Dd^\lambda$ induces a $\kappa$-filtered colimits-preserving functor $c_\Dd\colon \cat{Ind}_\kappa(\Dd^\lambda)\rightarrow \Dd$. Let's first argue why $c_\Dd$ is a functor in $\cat{Pr}_\lambda^\L$. Since $\Dd^\lambda$ has all $\kappa$-small colimits and $\id_{\Dd^\lambda}$ preserves them, the same argument as in the proof of \cref{lem:PrLKappa}\cref{enum:FunLKappa} shows that $c_\Dd$ preserves all colimits. Furthermore, the $\lambda$-compact objects of $\cat{Ind}_\kappa(\Dd^\lambda)$ are precisely those generated under $\lambda$-small colimits by objects in the image of $\Yo_{\Dd^\lambda}\colon \Dd^\lambda\rightarrow\cat{Ind}_\kappa(\Dd^\lambda)$. Indeed, if $\overline{\Dd}$ denotes the full sub-$\infty$-category spanned by these objects, then the proof of \cref{lem:KappaCompactlyGenerated}\cref{enum:CompactGenerators} yields an equivalence $\cat{Ind}_\lambda(\overline{\Dd})\simeq \cat{Ind}_\kappa(\Dd^\lambda)$; now apply \cref{lem:PrLKappa}\cref{enum:KappaCompactObjects}. Since $c_\Dd$ preserves all colimits and $\Dd^\lambda\subseteq \Dd$ is closed under $\lambda$-small colimits, it follows that all $\lambda$-compact objects of $\cat{Ind}_\kappa(\Dd^\lambda)$ land in $\Dd^\lambda$. This proves that $c_\Dd$ is indeed a functor in $\cat{Pr}_\lambda^\L$.
	
	To construct the desired right adjoint, it's now enough by \cref{lem:Adjunction} to show that the functor $(c_\Dd)_*\colon \Fun^\L_\kappa(\Cc,\cat{Ind}_\kappa(\Dd^\lambda))\longrightarrow\Fun^\L_\lambda(\Cc,\Dd)$ for all $\Cc\in \cat{Pr}_\kappa^\L$, given by postcomposition with $c_\Dd$, is an equivalence of $\infty$-categories. This functor fits into the following diagram:
	\begin{equation*}
		\begin{tikzcd}
			\Fun^\L_\kappa\bigl(\Cc,\cat{Ind}_\kappa(\Dd^\lambda)\bigr)\rar["(c_\Dd)_*"]\dar\drar[commutes] & \Fun^\L_\lambda(\Cc,\Dd)\dar\\
			\Fun\bigl(\Cc^\kappa,\cat{Ind}_\kappa(\Dd^\lambda)^\kappa\bigr)\rar["\simeq"] & \Fun(\Cc^\kappa,\Dd^\lambda)
		\end{tikzcd}
	\end{equation*}
	The vertical arrows are given by restriction along $\Cc^\kappa\subseteq \Cc$. By \cref{lem:PrLKappa}, the left vertical arrow is fully faithful and the bottom arrow is an equivalence. The right vertical arrow is fully faithful by \cref{lem:Ind}\cref{enum:IndFreelyGenerated}, using $\Cc\simeq \cat{Ind}_\kappa(\Cc^\kappa)$. It follows that $(c_\Dd)_*$ must be fully faithful too. Furthermore, by \cref{lem:PrLKappa}, the essential image of $\Fun_\kappa^\L(\Cc,\cat{Ind}_\kappa(\Dd^\lambda))\rightarrow \Fun(\Cc^\kappa,\Dd^\lambda)$ is spanned by those functors that preserve $\kappa$-small colimits. Since $\Cc^\kappa\subseteq \Cc$ and $\Dd^\lambda\subseteq \Dd$ are closed under $\kappa$-small colimits, the essential image of $\Fun_\lambda^\L(\Cc,\Dd)\rightarrow \Fun(\Cc^\kappa,\Dd^\lambda)$ must also be contained in the $\kappa$-small colimits-preserving functors. This shows that $(c_\Dd)_*$ is essentially surjective and we've finished the proof of \cref{enum:PrLKappaLambda}.
	
	The existence of limits in $\cat{Pr}_\kappa^\L$ follows from \cref{lem:Mindblow} below combined with \cref{cor:PresentableComplete}\cref{enum:PresentableComplete}. For colimits, let $\cat{Pr}_\kappa^\R$ be the $\infty$-category of all $\kappa$-compactly generated presentable $\infty$-categories and right adjoint functors that preserve $\kappa$-filtered colimits. Then \cref{cor:ExtractingAdjoints}\cref{enum:CatLCatR} and \cref{lem:LeftAdjointPreservesCompactsRightAdjointPreservesFiltered} show $\cat{Pr}_\kappa^\L\simeq (\cat{Pr}_\kappa^\R)^\op$. Thus it's enough to check that $\cat{Pr}_\kappa^\R$ is closed under limits in $\cat{Pr}^\R$.\footnote{As in the proof of \cref{lem:PrLColimits}, a straightforward extra-argument is needed since $\cat{Pr}_\kappa^\L$ is not a full sub-$\infty$-category of $\cat{Pr}^\R$. As preservation of limits and $\kappa$-filtered colimits can be checked factor-wise, we don't run into the same issue as in footnote~\cref{footnote:ProductsInPrLKappa}} As usual, it's enough to do products and pullbacks. In either case, we know from \cref{lem:PrLColimits} that the limit in $\widehat{\cat{Cat}}_\infty$ is also the limit in $\cat{Pr}^\R$, so by \cref{lem:KappaCompactlyGenerated}\cref{enum:CompactGenerators} we only need to construct a set of $\kappa$-compact generators. For a product $\prod_{i\in I}\Cc_i$, choose a set $S_i$ of $\kappa$-compact generators of $\Cc_i$ for all $i\in I$. Furthermore, choose an initial object $\emptyset_i\in \Cc_i$. For all $s_i\in S_i$ let $e_i(s_i)\in \prod_{i\in I}\Cc_i$ be the object given by $e_i(s_i)_i=s_i$ and $e_i(s_i)_j=\emptyset_j$ for $j\neq i$. Then $\left\{e_i(s_i)\ \middle|\ s_i\in S_i\right\}$ are $\kappa$-compact and jointly detect equivalences in the $i$\textsuperscript{th} component. Hence the union $\bigcup_{i\in I}\left\{e_i(s_i)\ \middle|\ s_i\in S_i\right\}$ is a set of $\kappa$-compact generators of $\prod_{i\in I}\Cc_i$.
	
	The argument for pullbacks is similar. Let $\Cc\coloneqq\Cc_0\times_{\Cc_2}\Cc_1$ be a pullback in $\cat{Pr}^\R$, where the underlying diagram is already contained in $\cat{Pr}_\kappa^\R$. By definition of $\cat{Pr}^\R$, the pullback projections $\pr_0\colon \Cc\rightarrow \Cc_0$ and $\pr_1\colon \Cc\rightarrow \Cc_1$ admit left adjoints $L_0\colon \Cc_0\rightarrow\Cc$ and $L_1\colon \Cc_1\rightarrow \Cc$. By \cref{lem:HomInLimits}\cref{enum:ColimitsInLimits}, the projections $\pr_0$ and $\pr_1$ preserve $\kappa$-filtered colimits, hence \cref{lem:LeftAdjointPreservesCompactsRightAdjointPreservesFiltered} shows that $L_0$ and $L_1$ preserve $\kappa$-compact objects. Now choose sets $S_0$ and $S_1$ of $\kappa$-compact generators of $\Cc_0$ and $\Cc_1$. Then $\left\{L_0(s_0)\ \middle|\ s_0\in S_0\right\}$ are $\kappa$-compact objects of $\Cc$ and jointly detect equivalences in the first factor. Similarly, $\left\{L_1(s_1)\ \middle|\ s_1\in S_1\right\}$ jointly detect equivalences in the second factor. Hence the union $\left\{L_0(s_0)\ \middle|\ s_0\in S_0\right\}\cup \left\{L_1(s_1)\ \middle|\ s_1\in S_1\right\}$ is a set of $\kappa$-compact generators of $\Cc$. This finishes the proof of \cref{enum:PrLKappaColimits}.
\end{proof}

To finish this rather lengthy subsection, we'll show the aforementioned mindblowing property of $\cat{Pr}_\kappa^\L$. If you're in a situation where you can fix an uncountable regular cardinal $\kappa$ and only work with $\kappa$-compactly generated $\infty$-categories (for most practical applications, $\kappa=\aleph_1$ is enough), \cref{lem:Mindblow} allows you to bypass all set-theoretic problems.
\begin{thm}[\enquote{$\text{Russel's paradox}=\text{skill issue}$}]\label{lem:Mindblow}
	Let $\kappa$ be an uncountable regular cardinal. Then $\cat{Pr}_\kappa^\L$ is an object of $\cat{Pr}_\kappa^\L$.
\end{thm}
\begin{proof}[Proof sketch]
	We already know from \cref{cor:PrLKappaColimits}\cref{enum:PrLKappaColimits} that $\cat{Pr}_\kappa^\L$ has all colimits. Thus, by \cref{lem:KappaCompactlyGenerated}\cref{enum:CompactGenerators} it's enough to find a set of $\kappa$-compact generators. We'll show that $\{\cat{An},\PSh(\Delta^1)\}$ does it. Let's first show that the functors $\Hom_{\cat{Pr}_\kappa^\L}(\cat{An},-)$ and $\Hom_{\cat{Pr}_\kappa^\L}(\PSh(\Delta^1),-)$ are jointly conservative. To this end, we claim more generally:
	\begin{alphanumerate}\itshape
		\item[\boxtimes_1] If $\Cc$ is a small $\infty$-category and $\Dd$ is a $\kappa$-compactly generated presentable $\infty$-category, then restriction along $\Yo_\Cc\colon \Cc\rightarrow \PSh(\Cc)$ induces an equivalence of $\infty$-categories\label{claim:FunLKappaPSh}
		\begin{equation*}
			\Yo_\Cc^*\colon \Fun_\kappa^\L\bigl(\PSh(\Cc),\Dd\bigr)\overset{\simeq}{\longrightarrow}\Fun(\Cc,\Dd^\kappa)\,.
		\end{equation*}
	\end{alphanumerate}
	Believing \cref{claim:FunLKappaPSh}, we find $\Hom_{\cat{Pr}_\kappa^\L}(\cat{An},\Dd)\simeq \core\Dd^\kappa$ and $\Hom_{\cat{Pr}_\kappa^\L}(\PSh(\Delta^1),\Dd)\simeq \core\Ar(\Dd^\kappa)$. Now $(-)^\kappa\colon \cat{Pr}_\kappa^\L\rightarrow \cat{Cat}_\infty$ is conservative by \cref{lem:PrLKappa}\cref{enum:PrLKappaInZFC}. Furthermore, $\core(-)\colon \cat{Cat}_\infty\rightarrow \cat{An}$ and $\core\Ar(-)\colon \cat{Cat}_\infty\rightarrow \cat{An}$ are jointly conservative, as we've seen in the proof of \cref{cor:AnPresentable}. It follows that $\cat{An}$ and $\PSh(\Delta^1)$ are generators of $\cat{Pr}_\kappa^\L$, as desired.
	
	To prove \cref{claim:FunLKappaPSh}, recall $\Fun^\L(\PSh(\Cc),\Dd)\simeq \Fun(\Cc,\Dd)$ from \cref{thm:PShFreeCocompletion}, so we only have to find out to which full sub-$\infty$-category $\Fun_\kappa^\L\subseteq \Fun^\L$ corresponds in $\Fun(\Cc,\Dd)$. Since $\{\Yo_\Cc(x)\ \vert\ x\in\Cc\}$ is a set of generators for $\PSh(\Cc)$, the same argument as in the proof of \cref{lem:PrLKappa}\cref{enum:PrLKappaLambda} shows that the $\kappa$-compact objects in $\PSh(\Cc)$ are precisely those generated under $\kappa$-small colimits from representable presheaves. Thus, a colimits-preserving functor $\PSh(\Cc)\rightarrow \Dd$ also preserves $\kappa$-compact objects if and only if it restricts to a functor $\Cc\rightarrow \Dd^\kappa$. This proves \cref{claim:FunLKappaPSh}.
	
	It remains to show that $\cat{An}$ and $\PSh(\Delta^1)$ are $\kappa$-compact in $\cat{Pr}_\kappa^\L$. As explained above, \cref{claim:FunLKappaPSh} shows $\Hom_{\cat{Pr}_\kappa^\L}(\cat{An},-)\simeq \core((-)^\kappa)$ and $\Hom_{\cat{Pr}_\kappa^\L}(\PSh(\Delta^1),-)\simeq \core\Ar((-)^\kappa)$. We know from \cref{lem:PrLKappa}\cref{enum:PrLKappaInZFC} that $(-)^\kappa\colon \cat{Pr}_\kappa^\L\rightarrow \cat{Cat}_\infty^{\kappa\mhyph\!\colimit}$ is an equivalence of $\infty$-categories, so it preserves all $\kappa$-filtered colimits, and we've sketched in the proof of \cref{cor:AnPresentable} that $\core(-)\colon \cat{Cat}_\infty\rightarrow \cat{An}$ and $\core\Ar(-)\colon \cat{Cat}_\infty\rightarrow \cat{An}$ preserve filtered colimits. Therefore, to finish the proof it's enough to show the following claim about $\kappa$-filtered colimits in $\cat{Cat}_\infty^{\kappa\mhyph\!\colimit}$.
	\begin{alphanumerate}\itshape
		\item[\boxtimes_2] The forgetful functor $\cat{Cat}_\infty^{\kappa\mhyph\!\colimit}\rightarrow\cat{Cat}_\infty$ preserves $\kappa$-filtered colimits.\label{claim:CatInftyKappaColimits}
	\end{alphanumerate}
	To prove~\cref{claim:CatInftyKappaColimits}, let $\Cc_{(-)}\colon \Jj\rightarrow \cat{Cat}_\infty^{\kappa\mhyph\!\colimit}$ be a $\kappa$-filtered diagram. We have to show that the colimit $\colimit_{j\in\Jj}\Cc_j$ in $\cat{Cat}_\infty$ also has all $\kappa$-small colimits and constitutes a colimit in $\cat{Cat}_\infty^{\kappa\mhyph\!\colimit}$. So let $\Ii$ be a $\kappa$-small $\infty$-category and let $T\colon \Ii\rightarrow \colimit_{j\in\Jj}\Cc_j$ be an $\Ii$-shaped diagram in $\colimit_{j\in\Jj}\Cc_j$. It's not hard to check that for uncountable regular cardinals $\kappa$ the $\kappa$-compact objects in $\cat{Cat}_\infty$ are precisely the $\kappa$-small $\infty$-categories.%
	%
	\footnote{Since $\cat{Cat}_\infty$ is generated by the compact objects $*$ and $\Delta^1$, hence the $\kappa$-compact objects are precisely those $\infty$-categories generated under $\kappa$-small colimits by $*$ and $\Delta^1$. Everyone of them is $\kappa$-small by \cref{rem:KappaSmallClosedUnderPushouts}. Conversely, for all $n\geqslant 0$, the $\infty$-category $\Delta^n$ can be written as a finite colimit in $*$ and $\Delta^1$ (namely, as the iterated pushout $\Delta^{\{0,1\}}\sqcup_{\{1\}}\dotsb \sqcup_{\{n-1\}}\Delta^{\{n-1,n\}}$). Using \cref{lem:KappaSmall}\cref{enum:KappaSmallC}, every other $\kappa$-small $\infty$-category is contained in the full sub-$\infty$-category of $\cat{Cat}_\infty$ generated under $\kappa$-small colimits by $\{\Delta^n\ \vert\ n\geqslant 0\}$.}
	%
	Hence $\Ii$ is $\kappa$-compact and so $T$ factors through a functor $T_0\colon \Ii\rightarrow \Cc_{j_0}$ for some $j_0\in\Jj$. By assumption, $\Cc_{j_0}$ has all $\kappa$-small colimits and so $T_0$ extends to a colimit cone $T_0^\triangleright\colon \Ii^\triangleright\rightarrow \Cc_{j_0}$. Furthermore, for every morphism $j_0\rightarrow j$ in $\Jj$ the functor $\Cc_{j_0}\rightarrow \Cc_j$ preserves $\kappa$-small colimits, hence $\Ii^\triangleright\rightarrow \Cc_{j_0}\rightarrow \Cc_j$ is still a colimit cone. We claim that then also $\Ii^\triangleright\rightarrow \Cc_{j_0}\rightarrow \colimit_{j\in \Jj}\Cc_j$ is a colimit cone. To show this, we need an analogue of \cref{lem:HomInLimits}\cref{enum:HomInLimits} for filtered colimits; this can be obtained in the exact same way, using that $\Hom_{\cat{Cat}_\infty}(\Delta^1,-)$ and $\Hom_{\cat{Cat}_\infty}(*\ \,*,-)$ also preserve filtered colimits and that pullbacks in $\cat{An}$ commute with filtered colimits by \cref{lem:FilteredColimitsPreserveFiniteLimits}. Then \cref{cor:HomPreservesColimits} shows that $\Ii^\triangleright\rightarrow \Cc_{j_0}\rightarrow \colimit_{j\in \Jj}\Cc_j$ is indeed a colimit cone, as claimed. This proves that $\colimit_{j\in \Jj}\Cc_j$ again has all $\kappa$-small colimits.
	
	To finish the proof, it remains to argue why $\colimit_{j\in\Jj}\Cc_j$ is also a colimit in the non-full sub-$\infty$-category $\cat{Cat}_\infty^{\kappa\mhyph\!\colimit}\subseteq \cat{Cat}_\infty$. Unravelling the definitions (and using \cref{lem:NonFullSubcategory}), we must check that for every natural transformation $\eta\colon \Cc_{(-)}\Rightarrow \const \Dd$ in $\cat{Cat}_\infty^{\kappa\mhyph\!\colimit}$ the induced functor $\colimit_{j\in\Jj}\Cc_j\rightarrow \Dd$ in $\cat{Cat}_\infty$ preserves $\kappa$-small colimits. But we've seen above that every $\kappa$-small colimit in $\colimit_{j\in \Jj}\Cc_j$ is inherited from $\Cc_{j_0}$ for some $j_0\in \Jj$ and the functor $\eta_{j_0}\colon \Cc_{j_0}\rightarrow \Dd$ preserves $\kappa$-small colimits by assumption on $\eta$.
\end{proof}

\postsectionappendix
