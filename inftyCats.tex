\documentclass[DIV=12,numbers=enddot,leqno,bibliography=totoc]{scrartcl}
\usepackage{./inftyStyle}
\addbibresource{./inftyLiterature.bib}

\title{$\infty$-Categories in Topology}
\author{Ferdinand Wagner}

%\includeonly{nothingtoseehere}
%\includeonly{./Sections/inftyCats1-5}
%\includeonly{./Sections/inftyCats1-5,./Sections/inftyCats6}
%\includeonly{./Sections/inftyCats6}
%\includeonly{./Sections/inftyCats7}
%\includeonly{./Sections/inftyCats6,./Sections/inftyCats7}

%\renewcommand{\sectionmark}[1]{\gdef\leftmark{#1}}

\begin{document}
	\maketitle

	\tableofcontents
	\addtocounter{section}{-1}
	
	\addtokomafont{section}{\S}
	\addtokomafont{subsection}{\S}
	\newpage
	
	
	\renewcommand{\ParagraphOrNot}{\S}
	\section{Introduction}
	At the QED Academy 2023 in Sonthofen, I gave a lecture course about \emph{$\infty$-categories in topology}. The goal of this course was ambitious, but we managed to have the Adams spectral sequence on the board in the end. After the academy was over, I decided, for future use, to translate my handwritten German notes into English and into \LaTeX. The result is this document. What began as a simple transscription quickly led to many additions and presented me with the opportunity to provide many missing details. While I fell for these temptations many times, I hope that at their heart, this notes still display a faithful (albeit not essentially surjective) representation of my original course.
	
	\begin{numpar}[$\infty$-Categories in topology.]
		When I first learned $\infty$-category theory, I felt like I'd just been given a cheat code for homotopy theory. Before, I'd felt lost in all the technicalities and struggled to develop intuition. With $\infty$-categories, everything suddenly made sense and I finally started to see the elegance and the clarity that I'd been looking for so long.
		
		So what are $\infty$-categories and what makes them so useful? Very roughly, an $\infty$-category not only contains \emph{$0$-morphisms} (objects) and \emph{$1$-morphism}, as an ordinary category, but also higher \emph{$n$-morphisms} for all $n\geqslant 2$. It turns out that all the usual results and constructions from category theory can be carried over to $\infty$-categories---however, getting $\infty$-category theory off the ground is \emph{much} more difficult than ordinary category theory: More than 30 years lie between the first definition of $\infty$-categories \cite{BoardmanVogt} and the first proof of the Yoneda lemma \cite{HTT}! But it's worth the effort! Among many other applications, which we'll not attempt to survey here, $\infty$-categories provide an incredibly powerful framework to do homotopy theory in. Already the vague idea explained above has a topological flavour: $n$-morphisms in an $\infty$-category, which run between morphisms of lower order, are reminiscent of $n$-cells in a CW-complexes, whose boundary is made up of cells of lower dimension. And indeed, every CW-complex (and then by CW-approximation every topological space) \emph{is} an example of an $\infty$-category. This is a key advantage of $\infty$-category theory over ordinary category theory:
		\begin{alphanumerate}[label={}]\itshape
			\item Ordinary category theory can be used to do homotopy theory---but homotopy theory is $\infty\text{-}$category theory.
		\end{alphanumerate}
		In particular, many formal $\infty$-categorical constructions, like presheaves or colimits, have a concrete topological meaning. It's surprising how many classical topological results can be reproved in a completely formal way just from abstract $\infty$-category theory! That number only increases through the introduction of the $\infty$-category of \emph{spectra}. Spectra combine the topological flavour of, well, topological spaces with the algebraic flavour of abelian groups; in particular, they admit a \emph{tensor product} and so it makes sense to talk about rings and modules in spectra. This allows us to bring algebra into the game---and again, algebra will not just be a \emph{tool} (like homology or homotopy groups) to prove classical theorems, but instead we'll be able to \emph{reinterpret} classical theorems as algebraic statements in the $\infty$-category of spectra.
		
		For someone like me, how came from an algebra background originally, it's amazing to be able to do topology with just the tools I feel confident with: category theory and algebra. But $\infty$-category theory does wonderful things in algebra. For example, the theory of \emph{derived categories}---another notoriously technical subject---becomes \emph{so much clearer} once you know spectra and the associated notion of a \emph{stable $\infty$-category}. In general, the future of algebra and algebraic geometry is derived, and all things derived become much clearer if you approach them using $\infty$-category theory.
		
		All of this is to say: You should learn $\infty$-category theory! It will be painful at first, but once you're there, you'll see mathematics with fresh eyes.
		
		
	\end{numpar}
	\begin{numpar}[Aim and scope of these notes.]
		The goal of these notes is to introduce $\infty$-categories and to explain many of their applications to topology. As prerequisites, you should feel comfortable with ordinary category theory; in addition, it would be beneficial to have a solid background in commutative algebra and topology (at least, you should have heard of simplicial sets, homology, and homotopy groups). None of this is strictly necessary---we'll recall the necessary ordinary category theory in \cref{sec:CategoryTheory} and we'll reintroduce many classical topological constructions in a way that's convenient for us---but it would certainly help you not to get overwhelmed by the material.
		
		These notes roughly consist of four parts: In \crefrange{sec:SimplicialSets}{sec:Straightening} we'll introduce $\infty$-categories as well as the technical ingredients that go into Lurie's proof of Yoneda's lemma. Unfortunately, this part contains several minor black boxes and a major one: I won't be able to prove Lurie's \emph{straightening/unstraightening} equivalence. The second part is \cref{sec:InftyCategoryTheory}, in which we'll redevelop classical category theory in the setting of $\infty$-categories. In the third part, spanning \crefrange{sec:TowardsSpectra}{sec:TensorProduct}, we'll introduce spectra and their tensor product. Finally, the last part is \cref{sec:CoolTopologyApplications}, in which we'll apply our theory to topology (altough many more applications are scattered throughout the text up to this point). One highlight will be the construction of the Adams spectral sequence.
	\end{numpar}
	
	\begin{numpar}[Model independence and notation.]\label{par:ModelIndependenceIntro}
		The model for $\infty$-categories we use in these notes will be \emph{quasi-categories}. But there are many other approaches to $\infty$-categories, like \emph{topologically} or \emph{Kan-enriched categories}, \emph{complete Segal spaces}, \emph{$1$-complicial sets}, \ldots\ Of course, all these approaches should be equivalent, but it's usually a non-trivial task to tranfer a result proven in one model into another model. A general theory of model independence that allows for such non-trivial transfers has been developed by Emily Riehl and Dominic Verity \cite{RiehlVerity}.
		
		In these notes, we take a somewhat different approach towards a model-independent theory. We will, or at least we would, in an ideal version of these notes, proceed in the following steps:
		\begin{alphanumerate}
			\item First, we'll set up the framework of quasi-categories, by any means necessary.\label{enum:SetupQuasicategories}
			\item After that, we'll prove (or black box) a few key statements in the model of quasi-categories. The statements themselves are model-independent, even though their proofs are not.\label{enum:KeyStatements}
			\item Finally, all further proofs will be done in a model-independent fashion.\label{enum:AllProofsModelIndependent}
		\end{alphanumerate}
		If you prefer a different model of quasi-categories, you'll probably know how to do steps~\cref{enum:SetupQuasicategories} and~\cref{enum:KeyStatements} in your model, and then, at least in an ideal world, everything from step~\cref{enum:AllProofsModelIndependent} will work in your model too.
		
		In reality, these notes don't quite live up to that ideal, but I dare say we come somewhat close. In \crefrange{sec:SimplicialSets}{sec:Straightening}, we'll sketch how to get the theory of quasi-categories off the ground. This corresponds to steps~\cref{enum:SetupQuasicategories} and~\cref{enum:KeyStatements} above, with most of \cref{enum:KeyStatements} happening in \cref{sec:JoyalLifting} and \cref{sec:Straightening}. Everything from \cref{sec:InftyCategoryTheory} onward mostly falls within step~\cref{enum:AllProofsModelIndependent}. I write \enquote{mostly} because, unfortunately, there are still a few non-model-independent arguments scattered throughout the text, the worst offender probably being our treatment of cardinality bounds and filteredness in \cref{subsec:Presentable}. I could, of course, tautologically claim that every non-model-independent proof still belongs to step~\cref{enum:KeyStatements}, but that would be a poor excuse for my inability to come up with better arguments.
		
		The transition to (mostly) model-independent arguments will be reflected in a change of terminology: Throughout \crefrange{sec:SimplicialSets}{sec:Straightening}, we'll use the term  \emph{quasi-category}, we'll write $\F(\Cc,\Dd)$ for the mapping object in simplicial sets and we'll always write $\N(\Ee)$ for the quasi-category obtained as the nerve of an ordinary category $\Ee$. Starting from \cref{sec:InftyCategoryTheory}, we'll simply say \emph{$\infty$-category}, write $\Fun(\Cc,\Dd)$, and consider every ordinary category $\Ee$ implicitly as an $\infty$-category, suppressing $\N$ everywhere. Only when we're using non-model-independent arguments, we'll switch back to the old terminology, to emphasise that what we're doing is morally questionable.
		
		Also, since I'm doing my PhD in Bonn, I'm legally required to use the term \emph{anima} for what other people would call \emph{space} or \emph{$\infty$-groupoid} or (in non-model-independent language) \emph{Kan complex}.
	\end{numpar}
	\begin{numpar}[Acknowledgments.]
		First I'd like to thank Fabian Hebestreit for his amazing cycle of lectures on $\infty$-categories and $\K$-theory \cite{HigherCatsI}, \cite{HigherCatsII}, \cite{KTheory}. I learnt all of this stuff in Fabian's lectures and these notes loosely follow his course. I'd also like to thank the participants of my QED academy course, Andrea Lachmann, Peter Langer, Malena Wasmeier, and Melvin Weiß, for their interest and for creating a thoroughly enjoyable teaching experience. Last but not least, I'd like to thank Dave Bowman and Yordan Toshev for their valuable comments on earlier versions of these notes.
	\end{numpar}
	
	\newpage
	\section{Category theory}\label{sec:CategoryTheory}
We assume you are familiar with categories, functors, natural transformations and the Yoneda lemma. In fact, you will probably be familiar with most of the stuff in this section, so we'll leave out many proofs (but give them later in the $\infty$-categorical context).

\subsection{Adjunctions}
\begin{defi}\label{def:1Adjunction}
	Let $L\colon \Cc\rightarrow \Dd$ be a functor.
	\begin{alphanumerate}
		\item Let $y\in \Dd$. An object $x\in \Cc$ is a \emph{right adjoint object to $y$ under $L$} if there exists an equivalence
		\begin{equation*}
			\Hom_\Cc(-,x)\simeq \Hom_\Dd\bigl(L(-),y\bigr)
		\end{equation*}
		in the functor category $\Fun(\Cc^\op,\cat{Set})$.
		\item A functor $R\colon \Dd\rightarrow \Cc$ is a \emph{right adjoint of $L$} if there exists an equivalence
		\begin{equation*}
			\Hom_\Cc\bigl(-,R(-)\bigr)\simeq \Hom_\Dd \bigl(L(-),-\bigr)
		\end{equation*}
		in the functor category $\Fun(\Cc^\op\times \Dd,\cat{Set})$. In this case we write $L\dashv R$.
	\end{alphanumerate}
\end{defi}
\begin{lem}[\enquote{Adjoints can be constructed pointwise}]\label{lem:1Adjunction}
	A functor $L\colon \Cc\rightarrow \Dd$ has a right adjoint if and only if every $y\in \Dd$ has a right adjoint object $x\in \Cc$.
\end{lem}
\begin{proof}
	One implication is trivial: If $R\colon \Dd\rightarrow \Cc$ is a right adjoint of $L$, then $R(y)$ is a right adjoint object of $y$ for every $y\in \Dd$. The other implication is left as an exercise. We'll prove an $\infty$-categorical variant in \cref{lem:Adjunction}.
\end{proof}
\begin{con}\label{con:1Unit}
	Let $L\colon \Cc \shortdoublelrmorphism \Dd\noloc R$ be an adjunction. For every $x\in \Cc$, the identity $\id_{L(x)}\colon L(x)\rightarrow L(x)$ is adjoint to a morphism $u_x\colon x\rightarrow RL(x)$. One can show that these morphisms assemble into a natural transformation $u\colon \id_\Cc\Rightarrow RL$, called the \emph{unit} of the adjunction. Dually, there is a \emph{counit} $c\colon LR\Rightarrow \id_\Dd$.
\end{con}
\begin{lem}[Triangle identities]\label{lem:1TriangleIdentities}
	Let $L\colon \Cc\shortdoublelrmorphism \Dd\noloc R$ be an adjunction. Then the diagrams 
	\begin{equation*}
		\begin{tikzcd}
			L \doublear["Lu"{black,above=0.1em}]{r}\doublear["\id_L"'{black}]{dr} & LRL\doublear["cL"{black,right=0.1em}]{d}\dar[phantom,""{name=A}]\arrow[from=1-1,to=A,commutes,pos=0.7]\\
			& L
		\end{tikzcd}\quad\text{and}\quad
		\begin{tikzcd}
			R \doublear["uR"{black,above=0.1em}]{r}\doublear["\id_R"'{black}]{dr} & RLR\doublear["Rc"{black,right=0.1em}]{d}\dar[phantom,""{name=A}]\arrow[from=1-1,to=A,commutes,pos=0.7]\\
			& R
		\end{tikzcd}
	\end{equation*}
	commute. Conversely, if $L$, $R$ are functors and $u:\id_\Cc\Rightarrow RL$, $c\colon LR\Rightarrow \id_\Dd$ are natural transformations such that the diagrams above commute, then $L$ and $R$ determine an adjunction. 
\end{lem}
\begin{proof}
	Exercise. We'll prove an $\infty$-categorical variant in \cref{lem:TriangleIdentities}.
\end{proof}
\begin{cor}\label{cor:1FunctorCategoryAdjunctions}
	Let $L\colon \Cc \shortdoublelrmorphism \Dd\noloc R$ be an adjunction and let $\Ii$ be another category. Then the pre- and postcomposition functors determine adjunctions
	\begin{align*}
		L\circ-\colon \Fun(\Ii,\Cc)&\doublelrmorphism \Fun(\Ii,\Dd)\noloc R\circ -\,,\\
		{-}\circ {R}\colon \Fun(\Cc,\Ii)&\doublelrmorphism \Fun(\Dd,\Ii)\noloc {-}\circ {L}\,.
	\end{align*}
\end{cor}
\begin{proof}
	By \cref{lem:1TriangleIdentities}, we only need to construct unit and a counit transformations satisfying the triangle identities. These are immediately inherited from the adjunction $L\dashv R$.
\end{proof}
\subsection{Limits and colimits}
\begin{defi}\label{def:1Colimits}
	Let $F\colon \Ii\rightarrow \Cc$ be a functor. A \emph{limit of $F$}, denoted $\limit F$ (or sometimes $\limit_{i\in\Ii}F(i)$), is a right adjoint object of $F$ under the functor $\operatorname{const}\colon\Cc\rightarrow \Fun(\Ii,\Cc)$ that sends $i\in\Ii$ to the constant functor with value $i$. Dually, a \emph{colimit of $F$}, denoted $\colimit F$ (or sometimes $\colimit_{i\in\Ii}F(i)$), is a left adjoint object of $F$ under $\operatorname{const}$.
\end{defi}
Concretely, \cref{def:1Colimits} means that we have the following natural bijections for all $x,y\in \Cc$:
\begin{align*}
	\Hom_\Cc(x,\limit F)&\cong \Hom_{\Fun(\Ii,\Cc)}(\operatorname{const}x,F)\,,\\ \Hom_\Cc(\colimit F,y)&\cong \Hom_{\Fun(\Ii,\Cc)}(F,\operatorname{const}y)\,.
\end{align*}
\begin{lem}\label{lem:1AdjointsPreserveColimits}
	Left adjoint functors preserve colimits and right adjoint functors preserve limits.
\end{lem}
\begin{proof}
	Let $L\colon \Cc \shortdoublelrmorphism \Dd\noloc R$ be an adjunction and let $\Ii$ be another category. By \cref{cor:1FunctorCategoryAdjunctions}, the postcomposition functors $L_*\coloneqq L\circ -$ and $R_*\coloneqq R\circ -$ determine an adjunction
	\begin{equation*}
		L_*\colon \Fun(\Ii,\Cc)\doublelrmorphism \Fun(\Ii,\Dd)\noloc R_*
	\end{equation*}
	Now let $F\colon \Ii\rightarrow \Cc$ be a functor admitting a colimit $\colimit F$. Since left adjoint functors clearly preserve left adjoint objects, we see that $L(\colimit F)$ is a left adjoint object of $F$ under $\const R(-)$. But $\const R(-)\simeq R_*\const\colon \Dd\rightarrow \Fun(\Ii,\Cc)$. A left adjoint object of $F$ under $R_*\const$ is also a left adjoint object of $L_*F$ under $\const\colon \Dd\rightarrow \Fun(\Ii,\Dd)$ by the adjunction above. In summary, this proves that $L(\colimit F)$ is a left adjoint object of $L_*F$ under $\const$, which is precisely what we want. The case of limits is analogous.
\end{proof}
\begin{lem}[\enquote{Colimits in functor categories are computed pointwise.}]\label{lem:1ColimitsInFunctorCategories}
	Let $\Cc$, $\Dd$, and $\Ii$ be categories such that $\Dd$ has all $\Ii$-shaped colimits; that is, all functors $\Ii\rightarrow \Dd$ admit colimits. Then $\Fun(\Cc,\Dd)$ has again all $\Ii$-shaped colimits and the evaluation functor 
	\begin{equation*}
		\ev_x\colon \Fun(\Cc,\Dd)\longrightarrow \Fun\bigl(\{x\},\Dd\bigr)\simeq \Dd
	\end{equation*}
	preserves $\Ii$-shaped colimits for all $x\in \Cc$. A dual assertion holds for limits.
\end{lem}
\begin{proof}
	By \cref{lem:1Adjunction}, the condition that $\Dd$ has all $\Ii$-shaped colimits implies that the functor $\const\colon \Dd\rightarrow \Fun(\Ii,\Dd)$ has a left adjoint $\colimit\colon \Fun(\Ii,\Dd)\rightarrow \Dd$. Under the \enquote{currying} equivalence
	\begin{equation*}
		\Fun\bigl(\Ii,\Fun(\Cc,\Dd)\bigr)\simeq \Fun\bigl(\Cc,\Fun(\Ii,\Dd)\bigr)\,,
	\end{equation*}
	the functor $\const\colon \Fun(\Cc,\Dd)\rightarrow \Fun(\Ii,\Fun(\Cc,\Dd))$ corresponds to the postcomposition functor $\const_*\colon \Fun(\Cc,\Dd)\rightarrow \Fun(\Cc,\Fun(\Ii,\Dd))$. By \cref{cor:1FunctorCategoryAdjunctions}, we have an adjunction
	\begin{equation*}
		{\colimit}_*\colon \Fun\bigl(\Cc,\Fun(\Ii,\Dd)\bigr)\doublelrmorphism\Fun(\Cc,\Dd)\noloc \const_*\,.
	\end{equation*}
	Hence $\const\colon \Fun(\Cc,\Dd)\rightarrow \Fun(\Ii,\Fun(\Cc,\Dd))$ has a left adjoint too, which proves that $\Fun(\Cc,\Dd)$ has $\Ii$-shaped colimits. The additional assertion that $\ev_x$ preserves $\Ii$-shaped colimits follows by unravelling how $\colimit$ is constructed from $\colimit_*$.
\end{proof}
\subsection{Kan extensions}
\begin{numpar}[Setup]\label{par:1KanExtensionSetup}
	Suppose we are given functors $f$ and $F$ as follows:
	\begin{equation*}
		\begin{tikzcd}
			\Cc\rar["F"]\dar["f"'] & \Dd\\
			\Cc'\urar[dashed] &
		\end{tikzcd}
	\end{equation*}
	Often one would like to extend $F$ to a functor $F'\colon \Cc'\rightarrow\Dd$. Of course, in general there's no such functor making the diagram above commute, and if there is, it might not be unique.
\end{numpar}
\begin{defi}\label{def:1KanExtensions}
	In the situation of Setup~\cref{par:1KanExtensionSetup}, a \emph{left Kan extension of $F$ along $f$}, denoted $\Lan_fF\colon \Cc'\rightarrow \Dd$, is a left adjoint object of $F$ under $f^*=-\circ f\colon \Fun(\Cc',\Dd)\rightarrow \Fun(\Cc,\Dd)$. Dually, a \emph{right Kan extension of $F$ along $f$}, denoted $\Ran_fF\colon \Cc'\rightarrow\Dd$, is a right adjoint object of $F$ under $f^*$.
\end{defi}
\begin{warn}\label{warn:KanDoesntExtend}
	In general, even if the respective Kan extensions exist, the diagrams
	\begin{equation*}
		\begin{tikzcd}
			\Cc\dar["f"']\rar["F",""{name=B,sloped}]& \Dd\arrow[from=B,to=2-1,draw=none,"\Longleftarrow"{sloped,marking,pos=0.5}]\\
			\Cc'\urar["\Lan_fF"']
		\end{tikzcd}
		\quad\text{and}\quad
		\begin{tikzcd}
			\Cc\dar["f"']\rar["F",""{name=B,sloped}]& \Dd\arrow[from=B,to=2-1,draw=none,"\Longrightarrow"{sloped,marking,pos=0.5}]\\
			\Cc'\urar["\Ran_fF"']
		\end{tikzcd}
	\end{equation*}
	only commute up to the indicated natural transformations. Indeed, the defining property from \cref{def:1KanExtensions} says that there are natural bijections
	\begin{align*}
		\Hom_{\Fun(\Cc',\Dd)}(\Lan_fF,F')&\cong \Hom_{\Fun(\Cc,\Dd)}(F,F'\circ f)\,,\\
		\Hom_{\Fun(\Cc',\Dd)}(F',\Ran_fF)&\cong \Hom_{\Fun(\Cc,\Dd)}(F'\circ f,F)\,,
	\end{align*}
	for all $F'\colon \Cc'\rightarrow \Dd$. Plugging in $F'=\Lan_fF$ , then taking the image of $\id_{\Lan_fF}$ produces the indicated natural transformation; and likewise for $\Ran_fF$ (so in other words, we're considering the unit of the adjunction $\Lan_f\dashv f^*$ and the counit of $f^*\dashv\Ran_f$, respectively).
	%		\begin{equation*}
		%			\begin{tikzcd}
			%				\Cc\drar["f"',""{name=A,sloped}]\rar["r"] & f/\Ee\dar["t"',""{name=B,sloped}]\rar["s"]& \Cc\dlar["f"]\arrow[from=1-3,to=B,draw=none,"\Longleftarrow"{sloped,marking,pos=0.7},"\eta"{swap,pos=0.65}]\arrow[from=A,to=1-2,phantom,"\scriptscriptstyle/\!/\!/"]\\
			%				& \Ee &
			%			\end{tikzcd}
		%		\end{equation*}
\end{warn}
\begin{exm}\label{exm:1ColimitAsKanExtension}
	Let $\Cc'=*$, then $\Fun(\Cc',\Dd)\simeq \Dd$ and we see immediately that $\Lan_fF$ corresponds to $\colimit F$ (if either exists). Likewise, $\Ran_fF$ corresponds to $\limit F$.
\end{exm}
Next we set out to answer the question when Kan extensions exist. To this end, we need to introduce two constructions that will feature prominently throughout the text.
\begin{con}\label{con:1ArrowCategory}
	Let $\Cc$ be a category and let $[1]\coloneqq \{\InlineDelta\}$ be the category with two objects and one non-identity morphism. The \emph{arrow category of $\Cc$} is the category
	\begin{equation*}
		\Ar(\Cc)\coloneqq \Fun\bigl([1],\Cc\bigr)\,.%
		%				\simeq \begin{cases*}
			%				\text{\emph{Objects:} Morphisms $\alpha\colon x\rightarrow y$ in $\Cc$.}\\
			%				\text{\emph{Morphisms:} Commutative diagrams $\begin{tikzcd}[cramped,column sep=small,row sep=small,ampersand replacement=\&,baseline=(ALPHA.base)]
					%						x\dar["\alpha"',"\textstyle \vphantom{x}"{name=ALPHA}]\rar \& x'\dar["\smash{\alpha'}\vphantom{\alpha}"]\\
					%						y\rar \& \smash{y'}\vphantom{y}
					%				\end{tikzcd}$ in $\Cc$.}
			%			\end{cases*}
	\end{equation*}
	Concretely, objects in $\Ar(\Cc)$ are morphisms $\alpha\colon x\rightarrow y$ in $\Cc$, and morphisms in $\Ar(\Cc)$ are commutative diagrams
	\begin{equation*}
		\begin{tikzcd}
			x\dar["\alpha"']\rar\drar[commutes] & x'\dar["\alpha'"]\\
			y\rar & y'
		\end{tikzcd}
	\end{equation*}
	in $\Cc$. There are functors $s,t\colon \Ar(\Cc)\rightarrow \Cc$ (\enquote{source} and \enquote{target} projection) sending an arrow $(\alpha\colon x\rightarrow y)$ to $x$ and $y$, respectively.
\end{con}
\begin{con}\label{con:1SliceCategory}
	Let $f\colon \Cc\rightarrow\Cc'$ be a functor and $x'\in \Cc'$. The \emph{slice category of $\Cc$ over $x'$} is the pullback\footnote{The pullback ist taken in the category of \emph{small categories}, that is, those categories whose class of objects is a set. But the explicit description works with the weaker assumption that $\Cc$ and $\Cc'$ are \emph{locally small}, meaning that $\Hom_\Cc(x,y)$ and $\Hom_{\Cc'}(x',y')$ are sets for all $x,y\in\Cc$, $x',y'\in \Cc'$.}
	\begin{equation*}
		\begin{tikzcd}
			\Cc_{/x'}\rar\dar\drar[pullback] & \Ar(\Cc')\dar["{(s,t)}"]\\
			\Cc\times\{x'\}\rar["f\times x'"] & \Cc'\times \Cc'
		\end{tikzcd}
	\end{equation*}
	Concretely, objects in the slice category $\Cc_{/x'}$ are pairs $(x,f(x)\rightarrow x')$, where $x\in \Cc$ and $f(x)\rightarrow x'$ is a morphism in $\Cc'$. Morphisms in $\Cc_{/x'}$ are given by morphisms $\alpha\colon x\rightarrow y$ such that
	\begin{equation*}
		\begin{tikzcd}
			f(x)\dar\rar["f(\alpha)"]\drar[commutes] & f(y)\dar\\
			x'\eqar[r] & x'
		\end{tikzcd}
	\end{equation*}
	commutes. Dually, there's also $\Cc_{x'/}$, the \emph{slice category of $\Cc$ under $x'$}.
\end{con}
\begin{lem}[Kan extension formula]\label{lem:1KanExtensionFormula}
	In the situation of Setup~\cref{par:1KanExtensionSetup}, assume that for all $x'\in \Cc'$ the following colimits exist in $\Dd$:
	\begin{equation*}
		\colimit_{(x,f(x)\rightarrow x')\in \Cc_{/x'}}F(x)\coloneqq \colimit\left(\Cc_{/x'}\longrightarrow \Cc\overset{F}{\longrightarrow}\Dd\right)\,.
	\end{equation*}
	Then $\Lan_fF$ exists and $\Lan_fF(x')$ is given by that colimit.
\end{lem}
\begin{proof}
	Exercise. We'll prove an $\infty$-categorical variant in \cref{lem:KanExtensionFormula}.
\end{proof}
\begin{cor}%[\enquote{Kan extensions along fully faithful functors behave nicely.}]
	\label{cor:1KanExtensionAlongFullyFaithful}
	In the situation of Setup~\cref{par:1KanExtensionSetup}, assume that $f\colon \Cc\rightarrow \Cc'$ is fully faithful and that the colimits from \cref{lem:1KanExtensionFormula} exist. Then the natural transformation $u_F\colon F\Rightarrow \Lan_fF\circ f$ from \cref{warn:KanDoesntExtend} is an equivalence.
\end{cor}
\begin{proof}
	If $x'=f(y)$ for some $y\in \Cc$, then $f$ being fully faithful implies that the slice category $\Cc_{/f(y)}$ is equivalent to $\Cc_{/y}$ (that is, the slice category formed with respect to $\id_\Cc\colon \Cc\rightarrow\Cc$). The latter has a terminal object, namely $\{\id_y\}$. Hence
	\begin{equation*}
		\Lan_f\bigl(Ff(y)\bigr)\cong \colimit_{(x,f(x)\rightarrow y)\in \Cc_{/f(y)}}F(x)\cong \colimit_{(x\rightarrow y)\in \Cc_{/y}}F(x)\cong F(y).\qedhere
	\end{equation*}
\end{proof}
To finish this subsection, we prove a result about the \emph{category $\PSh(\Cc)\coloneqq \Fun(\Cc^\op,\cat{Set})$ of presheaves on $\Cc$}. This will seem rather technical at first, but, together with its $\infty$-categorical version, it will be invaluable throughout the text.
\begin{thm}[\enquote{$\PSh(\Cc)$ arises by freely adding colimits to $\Cc$.}]\label{thm:1PShFreeCocompletion}
	Let $\Cc$ and $\Dd$ be categories, where $\Dd$ has all colimits. Let $\Yo_\Cc\colon \Cc\rightarrow\PSh(\Cc)$ denote the Yoneda embedding, sending $x\in \Cc$ to $\Hom_\Cc(-,x)\colon \Cc^\op\rightarrow\cat{Set}$. Then restriction along $\Yo_\Cc$ induces an equivalence
	\begin{equation*}
		\Yo_\Cc^*\colon \Fun^{\colimit}\bigl(\PSh(\Cc),\Dd\bigr)\overset{\simeq}{\longrightarrow}\Fun(\Cc,\Dd)\,.
	\end{equation*}
	Here $\Fun^{\colimit}(\PSh(\Cc),\Dd)\subseteq \Fun(\PSh(\Cc),\Dd)$ is the full subcategory spanned by the colimit-preserving functors. Furthermore, every colimit-preserving functor $\PSh(\Cc)\rightarrow \Dd$ admits a right adjoint.
\end{thm}
To prove \cref{thm:1PShFreeCocompletion}, we send two lemmas in advance.
\begin{lem}[\enquote{Every presheaf is a colimit of representables.}]\label{lem:1PresheafColimitOfRepresentables}
	Let $\Cc$ be a category. For every $E\in \PSh(\Cc)$, the natural morphism
	\begin{equation*}
		\colimit_{(y,\Hom_\Cc(-,y)\rightarrow E)\in \Cc_{/E}}\Hom_\Cc(-,y)\overset{\cong}{\longrightarrow}E
	\end{equation*}
	is an isomorphism.
\end{lem}
\begin{proof}
	Exercise (use Yoneda's lemma). We'll prove an $\infty$-categorical version in \cref{lem:PresheafColimitOfRepresentables}.
\end{proof}
\begin{lem}\label{lem:1LanAlongYonedaHasRightAdjoint}
	Let $\Cc$ and $\Dd$ be categories, where $\Dd$ has all colimits. For every $F\colon \Cc\rightarrow \Dd$, the left Kan extension $\Lan_{\Yo_\Cc}F\colon \PSh(\Cc)\rightarrow\Dd$ \embrace{which exists due to \cref{lem:1KanExtensionFormula}} admits a right adjoint. The right adjoint sends $y\in \Dd$ to $\Hom_\Dd(F(-),y)\colon \Cc^\op\rightarrow\cat{Set}$.
\end{lem}
\begin{proof}
	Exercise. We'll prove an $\infty$-categorical version in \cref{lem:LanAlongYonedaHasRightAdjoint}.
\end{proof}
Furthermore, we need the following general lemma (which will occasionally be useful in the future too).
\begin{lem}\label{lem:1FullyFaithfulConservativeAdjunction}
	Let $\Cc$ and $\Dd$ be categories and let $L\colon \Cc\shortdoublelrmorphism \Dd\noloc R$ be an adjunction.
	\begin{alphanumerate}
		\item The left adjoint $L$ is fully faithful if and only if the unit transformation $u\colon \id_\Cc\Rightarrow RL$ is an equivalence.\label{enum:1FullyFaithfulIffUnitEquivalence}
		\item Suppose $L$ ist fully faithful and $R$ is conservative \embrace{that is, if $\alpha\colon x\rightarrow y$ is a morphism in $\Dd$ such that $R(\alpha)$ is an isomorphism, then $\alpha$ is an isomorphism too}. Then $L$ and $R$ are inverse equivalences of categories.\label{enum:1Conservative}
	\end{alphanumerate}
\end{lem}
\begin{proof}
	To prove \cref{enum:FullyFaithfulIffUnitEquivalence}, first observe that for all elements $x,y\in\Cc$, the postcomposition map $(u_y)_*\colon \Hom_\Cc(x,y)\rightarrow\Hom_\Cc(x,RL(y))$ is given by
	\begin{equation*}
		(u_y)_*\colon\Hom_\Cc(x,y)\overset{L}{\longrightarrow}\Hom_\Dd\bigl(L(x),L(y)\bigr)\overset{\cong}{\longrightarrow}\Hom_\Cc\bigl(x,RL(y)\bigr)\,,
	\end{equation*}
	where the second map is the adjunction bijection. By Yoneda's lemma, 
	$u_y\colon y\rightarrow RL(y)$ is an equivalence if and only if $(u_y)_*\colon \Hom_\Cc(x,y)\rightarrow\Hom_\Cc(x,RL(y))$ is a bijection for all $x$. By the above, this happens if and only if $L\colon \Hom_\Cc(x,y)\rightarrow\Hom_\Dd(L(x),L(y))$ is a bijection for all $x\in\Cc$. This proves \cref{enum:FullyFaithfulIffUnitEquivalence}.
	
	For \cref{enum:Conservative}, the second of the triangle identities from \cref{lem:1TriangleIdentities} shows that $Rc\colon RLR\Rightarrow R$ is a natural equivalence. Since $R$ is conservative, $c\colon LR\Rightarrow \id_\Dd$ must be an equivalence too. Since $u\colon \id_\Cc\Rightarrow RL$ is an equivalence by assumption, we are done.
\end{proof}
\begin{proof}[Proof of \cref{thm:1PShFreeCocompletion}]
	By \cref{lem:1LanAlongYonedaHasRightAdjoint} and \cref{lem:1AdjointsPreserveColimits}, the adjunction $\Lan_{\Yo_\Cc}\dashv\Yo_\Cc^*$ restricts to an adjunction
	\begin{equation*}
		\Lan_{\Yo_\Cc}\colon \Fun(\Cc,\Dd)\doublelrmorphism\Fun^{\colimit}\bigl(\PSh(\Cc),\Dd\bigr)\noloc \Yo_\Cc^*\,.
	\end{equation*}
	Since $\Yo_\Cc$ is fully faithful, \cref{cor:1KanExtensionAlongFullyFaithful} implies that the unit $u\colon \id_{\Fun(\Cc,\Dd)}\Rightarrow\Yo_\Cc^*\circ \Lan_{\Yo_\Cc}$ is an equivalence. Furthermore, it's clear that $\Yo_\Cc^*$ is conservative: If a natural transformation $\eta\colon F\Rightarrow G$ between colimit-preserving functors $F,G\colon \PSh(\Cc)\rightarrow\Dd$ is an equivalence when restricted to representable presheaves, then it is an equivalence everywhere, because every presheaf can be written as a colimit of representables (\cref{lem:1PresheafColimitOfRepresentables}). Then \cref{lem:1FullyFaithfulConservativeAdjunction}\cref{enum:1Conservative} finishes the proof.
\end{proof}

\newpage
\section{The simplicial model}\label{sec:SimplicialSets}
In this section, we'll introduce our model for $\infty$-categories and the main object of interest in \cref{sec:SimplicialSets}, \cref{sec:JoyalLifting}, and \cref{sec:Straightening}: quasi-categories! We'll see some first signs that quasi-categories behave a lot like ordinary categories and we'll define the quasi-category of quasi-categories $\cat{Cat}_\infty$.
\subsection{Recollections on simplicial sets}
\begin{defi}\label{def:SimplicialSet}
	\begin{alphanumerate}
		\item The \emph{simplex category $\IDelta$} is the category whose objects are finite non-empty totally ordered sets $[n]=\{0<1<\dotsb<n\}$ for all $n\geqslant 0$ and whose morphisms are order-preserving maps, that is, maps $\alpha\colon [m]\rightarrow [n]$ such that $\alpha(0)\leqslant \alpha(1)\leqslant\dotsb\leqslant \alpha(m)$.\label{enum:SimplexCategory}
		\item A \emph{simplicial set} is a presheaf on $\IDelta$, that is, a functor $X\colon \IDelta^\op\rightarrow\cat{Set}$. The \emph{category of simplicial sets} is the category $\cat{sSet}\coloneqq \PSh(\IDelta)\simeq\Fun(\IDelta^\op,\cat{Set})$ of presheaves on $\IDelta$.\label{enum:SimplicialSet}
	\end{alphanumerate}
\end{defi}
\begin{con}\label{con:FaceDegeneracyMaps}
	For all $i=0,\dotsc,n$ and all $j=0,\dotsc,n-1$ let $d_i\colon [n-1]\rightarrow [n]$ be the unique injective morphism in $\IDelta$ that doesn't hit $i$ and let $s_j\colon [n]\rightarrow[n-1]$ be the unique surjective morphism in $\IDelta$ that hits $j$ twice. It's straightforward to see that every morphism $\alpha\colon [m]\rightarrow[n]$ in $\IDelta$ can be written as a composition of some $s_j$ and some $d_i$. Therefore, a simplicial set can be described by the following data:
	\begin{alphanumerate}
		\item Sets $X_n\coloneqq X([n])$ for all $n\geqslant 0$.
		\item \emph{Face maps} $d_i^*\colon X_n\rightarrow X_{n-1}$ for all $i=0,\dotsc,n$.
		\item \emph{Degeneracy maps} $s_j^*\colon X_{n-1}\rightarrow X_{n}$ for all $j=0,\dotsc,n-1$.
	\end{alphanumerate}
	The face and degeneracy maps satisfy $d_j^*\circ d_i^*=d_{i-1}^*\circ d_j^*$ and $s_j^*\circ s_i^*=s_{i-1}^*\circ s_j^*$ for all $i>j$ as well as
	\begin{equation*}
		d_j^*\circ s_i^*=\ScaledBracesCases{\!\begin{plaincases*}
			s_i^*\circ d_{j-1}^* & if $i<j-1$\\
			\id_{X_{n-1}} & if $i=j-1$ or $i=j$\\
			s_{i-1}^*\circ d_j^* & if $i>j$
		\end{plaincases*}}\,.
	\end{equation*}
	It's customary to call elements of $X_n$ \emph{$n$-simplices of $X$}. An $n$-simplex is called \emph{degenerate} if it is in the image of $s_j\colon X_{n-1}\rightarrow X_n$ for some $j$.
\end{con}
Let's give some first examples of simplicial sets and explain some basic constructions. %with them. Here we'll also see \cref{thm:1PShFreeCocompletion} in action for the first time!

\begin{numpar}[Boundaries and horns.]\label{par:Horns}
	For all $n\geqslant 0$, the functor $\Delta^n\coloneqq\Hom_{\IDelta}(-,[n])\colon \IDelta^\op\rightarrow\cat{Set}$ is a simplicial set, called the \emph{$n$-simplex}. Yoneda's lemma implies that $\Hom_{\cat{sSet}}(\Delta^n,X)\cong X_n$ for all simplicial sets $X$. The maps $d_i\colon [n-1]\rightarrow [n]$ and $s_j\colon [n]\rightarrow[n-1]$ from \cref{con:FaceDegeneracyMaps} induce maps $d_i\colon \Delta^{n-1}\rightarrow \Delta^n$ and $s_j\colon \Delta^n\rightarrow \Delta^{n-1}$ in $\cat{sSet}$.\footnote{This may be confusing at first but the maps $d_i\colon \Delta^{n-1}\rightarrow \Delta^n$ and $s_j\colon \Delta^n\rightarrow \Delta^{n-1}$ really run in the indicated directions. The point is that while $\Hom_{\IDelta}(-,[n])\colon \IDelta^\op\rightarrow\cat{Set}$ is contravariant, the functor that assigns $[n]\mapsto \Hom_{\IDelta}(-,[n])$, that is, the Yoneda embedding $\Yo_{\IDelta}\colon \IDelta\rightarrow \PSh(\IDelta)\simeq \cat{sSet}$, is covariant.} Using these maps, we can define the following sub-simplicial sets of $\Delta^n$:
	\begin{align*}
		\partial \Delta^n&\coloneqq \bigcup_{i=0}^n\im\left(d_i\colon \Delta^{n-1}\rightarrow\Delta^n\right)\subseteq \Delta^n\,\text{, the \emph{boundary of $\Delta^n$},}\\
		\Lambda_j^n&\coloneqq \bigcup_{\substack{i=0\\i\neq j}}^n\im\left(d_i\colon \Delta^{n-1}\rightarrow\Delta^n\right)\subseteq \Delta^n\,\text{, the \emph{$j$-horn in $\Delta^n$}.}
	\end{align*}
	Here the unions are taken degree-wise.\footnote{Therefore, they're colimits in $\cat{sSet}$, as limits and colimits in functor categories are computed pointwise by \cref{lem:1ColimitsInFunctorCategories}.} It's customary to call horns $\Lambda_j^n$ \emph{inner horns} if $0<j<n$ and \emph{outer horns} if $j=0$ or $j=n$. Concretely, for all $m\geqslant 0$, the $m$-simplices of the boundary $\partial \Delta^n$ and the $j$-horn $\Lambda_j^n$ are given by the following formulae:
	\begin{align*}
		(\partial \Delta^n)_m&=\bigl\{\alpha\colon [m]\rightarrow [n]\ \big|\ [n]\neq \im(\alpha)\bigr\}\,,\\
		(\Lambda_j^n)_m&=\bigl\{\alpha\colon [m]\rightarrow [n]\ \big|\ [n]\neq \im(\alpha)\cup\{j\}\bigr\}\,.
	\end{align*}
	Here are some pictures in the case $n=2$ (in the bottom line, the dotted lines mark the faces that are missing in the respective horns):
	\begin{gather*}
		\Delta^2=\begin{tikzpicture}[commutative diagrams/every diagram,baseline=(mid.base), decoration={markings,mark=at position 0.5 with {\arrow{to}}}]
			\path node[outer sep=0.25ex] (0) at (0,0) {$0$} ++(0:3.8em) node[text depth=0pt,outer sep=0.25ex] (1) {$1$} ++ (120:3.8em) node[outer sep=0.25ex] (2) {$2$};
			\path (0) to node[pos=0.5] (mid) {} (2);
			\path (0) to node[pos=0.5] (05) {} (1);
			\path (05) to node[pos=0.333] {$\scriptscriptstyle/\!/\!/$} (2);
			\path[commutative diagrams/.cd, every arrow, every label]
			(0) edge[postaction={decorate},-] (1)
			(1) edge[postaction={decorate},-] (2)
			(0) edge[postaction={decorate},-] (2);
		\end{tikzpicture}\,,\quad
		\partial\Delta^2=\begin{tikzpicture}[commutative diagrams/every diagram,baseline=(mid.base), decoration={markings,mark=at position 0.5 with {\arrow{to}}}]
			\path node[outer sep=0.25ex] (0) at (0,0) {$0$} ++(0:3.8em) node[text depth=0pt,outer sep=0.25ex] (1) {$1$} ++ (120:3.8em) node[outer sep=0.25ex] (2) {$2$};
			\path (0) to node[pos=0.5] (mid) {} (2);
			\path[commutative diagrams/.cd, every arrow, every label]
			(0) edge[postaction={decorate},-] (1)
			(1) edge[postaction={decorate},-] (2)
			(0) edge[postaction={decorate},-] (2);
		\end{tikzpicture}\\
		\Lambda_0^2=\begin{tikzpicture}[commutative diagrams/every diagram,baseline=(mid.base), decoration={markings,mark=at position 0.5 with {\arrow{to}}}]
			\path node[outer sep=0.25ex] (0) at (0,0) {$0$} ++(0:3.8em) node[text depth=0pt,outer sep=0.25ex] (1) {$1$} ++ (120:3.8em) node[outer sep=0.25ex] (2) {$2$};
			\path (0) to node[pos=0.5] (mid) {} (2);
			\path[commutative diagrams/.cd, every arrow, every label]
			(0) edge[postaction={decorate},-] (1)
			(1) edge[dotted,postaction={decorate},-] (2)
			(0) edge[postaction={decorate},-] (2);
		\end{tikzpicture}\,,\quad
		\Lambda_1^2=\begin{tikzpicture}[commutative diagrams/every diagram,baseline=(mid.base), decoration={markings,mark=at position 0.5 with {\arrow{to}}}]
			\path node[outer sep=0.25ex] (0) at (0,0) {$0$} ++(0:3.8em) node[text depth=0pt,outer sep=0.25ex] (1) {$1$} ++ (120:3.8em) node[outer sep=0.25ex] (2) {$2$};
			\path (0) to node[pos=0.5] (mid) {} (2);
			\path[commutative diagrams/.cd, every arrow, every label]
			(0) edge[postaction={decorate},-] (1)
			(1) edge[postaction={decorate},-] (2)
			(0) edge[dotted,postaction={decorate},-] (2);
		\end{tikzpicture}\,,\quad
		\Lambda_2^2=\begin{tikzpicture}[commutative diagrams/every diagram,baseline=(mid.base), decoration={markings,mark=at position 0.5 with {\arrow{to}}}]
			\path node[outer sep=0.25ex] (0) at (0,0) {$0$} ++(0:3.8em) node[text depth=0pt,outer sep=0.25ex] (1) {$1$} ++ (120:3.8em) node[outer sep=0.25ex] (2) {$2$};
			\path (0) to node[pos=0.5] (mid) {} (2);
			\path[commutative diagrams/.cd, every arrow, every label]
			(0) edge[dotted,postaction={decorate},-] (1)
			(1) edge[postaction={decorate},-] (2)
			(0) edge[postaction={decorate},-] (2);
		\end{tikzpicture}
	\end{gather*}
\end{numpar}
\begin{numpar}[Geometric Realisation.]\label{par:GeometricRealisation}
	These pictures suggest a geometric way to think about simplices, boundaries of simplices, and horns. In fact, we can associate to every simplicial set $X$ a topological space (in fact, a CW-complex) $\abs*{X}$, called the \emph{geometric realisation of $X$}. To describe this construction, we first define $\abs*{\Delta^n}$ to be the \emph{topological $n$-simplex}, that is, the space $\{(t_0,\dotsc,t_n)\in\IR^n\ |\ 0\leqslant t_i\leqslant 1, \sum_{i=1}^nt_i=1\}\subseteq \mathbb R^n$. For all $i=0,\dotsc,n$ and all $j=0,\dotsc,n-1$ we define maps $\abs*{d_i}\colon\abs*{\Delta^{n-1}}\rightarrow\abs*{\Delta^n}$ and $\abs*{s_j}\colon\abs*{\Delta^{n}}\rightarrow\abs*{\Delta^{n-1}}$ via
	\begin{align*}
		\abs*{d_i}(t_0,\dotsc,t_{n-1})&\coloneqq (t_0,\dotsc,t_{i-1},0,t_i,\dotsc,t_n)\\
		\abs*{s_j}(t_0,\dotsc,t_n)&\coloneqq (t_0,\dotsc,t_{i-1},t_i+t_{i+1},t_{i+2},\dotsc,t_n)
	\end{align*}
	For general simplicial sets $X$, we can now construct $\abs*{X}$ by taking a topological $n$-simplex $\abs*{\Delta^n}$ for every $\sigma\in X_n$ and gluing them together according to the face and degeneracy maps above. More precisely, we take
	\begin{equation*}
		\abs*{X}\coloneqq\colimit_{(n,\Delta^n\rightarrow X)}\abs*{\Delta^n}\in\cat{Top}\,.
	\end{equation*}
	This agrees with the Kan extension formula from \cref{lem:1KanExtensionFormula}! So $\abs*{\,\cdot\,}\colon \cat{sSet}\rightarrow \cat{Top}$ must be the unique colimit-preserving extension, guaranteed by \cref{thm:1PShFreeCocompletion}, of the functor $\IDelta\rightarrow\cat{Top}$ that sends $[n]\mapsto \abs*{\Delta^n}$.
	
	Furthermore, \cref{thm:1PShFreeCocompletion} guarantees that $\abs*{\,\cdot\,}\colon \cat{sSet}\rightarrow \cat{Top}$ admits a right adjoint, which we denote $\Sing\colon \cat{Top}\rightarrow\cat{sSet}$. By \cref{lem:1LanAlongYonedaHasRightAdjoint}, it is given by $(\Sing Y)_n\cong \Hom_{\cat{Top}}(\abs*{\Delta^n},Y)$. So $\Sing Y$ is indeed the construction you know from the definition of singular homology.
\end{numpar}
\begin{numpar}[Nerve and homotopy category.]\label{par:Nerve}
	Every partially ordered set defines a category. In particular, we can regard the totally ordered sets $[n]$ as categories. Accordingly, we obtain a functor $U\colon \IDelta\rightarrow \cat{Cat}$ into the category of small categories; $U$ simply sends $[n]\mapsto [n]$. For every small category $\Cc$, this allows us to define a simplicial set $\N(\Cc)$, called the \emph{nerve of $\Cc$}, as the composition
	\begin{equation*}
		\N(\Cc)\colon \IDelta^\op\xrightarrow{U^\op}\cat{Cat}^\op\xrightarrow{\Hom_{\cat{Cat}}(-,\,\Cc)}\cat{Set}\,.
	\end{equation*}
	Concretely, $\N(\Cc)_n= \Hom_{\cat{Cat}}([n],\Cc)\cong \{x_0\rightarrow \dotsb\rightarrow x_n\text{ in $\Cc$}\}$ is the set of all chains of $n$ morphisms in $\Cc$. The face maps $d_i^*\colon \N(\Cc)_n\rightarrow \N(\Cc)_{n-1}$ compose the $(i-1)$\textsuperscript{st} and $i$\textsuperscript{th} morphism in the chain (in the cases $i=0$ or $i=n$, the face map $d_0^*$ just discards $x_0$ and $d_n^*$ just discards $x_n$). The degeneracy maps $s_j^*\colon \N(\Cc)_{n-1}\rightarrow \N(\Cc)_n$ insert an identity at the $j$\textsuperscript{th} position.
	
	Observe that the formula $\N(\Cc)_n\cong\Hom_{\cat{Cat}}([n],\Cc)$ is exactly of the form of a right-adjoint as in \cref{lem:1LanAlongYonedaHasRightAdjoint}! So what's the corresponding left adjoint? According to \cref{thm:1PShFreeCocompletion}, it has to be the unique colimit-preserving extension of the functor $U\colon\IDelta\rightarrow \cat{Cat}$ above.\footnote{Here we use implicitly that $\cat{Cat}$ has all colimits (which is easy to check, but not completely trivial, due to the same composition issues as in the description of $\operatorname{ho}(\Lambda_1^2)$).} We'll denote this extension by $\operatorname{ho}\colon \cat{sSet}\rightarrow\cat{Cat}$ and for a simplicial set $X$, we call $\operatorname{ho}(X)$ the \emph{homotopy category of $X$}. The objects of $\operatorname{ho}(X)$ are the set of $0$-simplices $X_0$. However, the morphisms of $\operatorname{ho}(X)$ are a little more difficult to describe. For example, $\operatorname{ho}(\Lambda_1^2)$ contains a morphism $\alpha\colon 0\rightarrow 1$ and a morphism $\beta\colon 1\rightarrow 2$; $\alpha$ and $\beta$ are induced by the functors $[1]\cong \operatorname{ho}(\Delta^{\{0,1\}})\rightarrow \operatorname{ho}(\Lambda_1^2)$ and $[1]\cong \operatorname{ho}(\Delta^{\{1,2\}})\rightarrow \operatorname{ho}(\Lambda_1^2)$. Hence $\operatorname{ho}(\Lambda_1^2)$ must also contain a morphism $\beta\circ\alpha\colon 0\rightarrow 2$, even though there's no $1$-simplex from $0$ to $2$ in $\Lambda_1^2$. So in general, not all morphisms in $\operatorname{ho}(X)$ come from $1$-simplices of $X$. Instead, we have to take \emph{chains} of $1$-simplices and quotient out a suitable equivalence relation. This is not too hard to make precise, but quite technical and we won't pursue it here. We'll see an explicit description in the case of quasi-categories in \cref{par:HomotopyCategory} below; the general description can be found in \cite[Construction/Proposition~II.24]{HigherCatsI}.
\end{numpar}
\begin{numpar}[Mapping objects in simplicial sets.]\label{par:FInternalHom}
	For every simplicial set $X$ the functor $-\times X\colon\cat{sSet}\rightarrow\cat{sSet}$ commutes with colimits.\footnote{Using that limits and colimits in $\cat{sSet}$ are computed degree-wise by \cref{lem:1ColimitsInFunctorCategories}, this can be reduced to the fact that products in $\cat{Set}$ commute with colimits, which is straightforward to check.} Hence, by \cref{thm:1PShFreeCocompletion}, it must be the unique colimit-preserving extension of the functor $\IDelta\rightarrow\cat{sSet}$ sending $[n]\mapsto \Delta^n\times X$. But more importantly, $-\times X$ must have a right adjoint, which we denote $\F(X,-)\colon \cat{sSet}\rightarrow\cat{sSet}$. By the formula in \cref{lem:1LanAlongYonedaHasRightAdjoint}, the right adjoint is given by $\F(X,Y)_n\cong \F(\Delta^n\times X,Y)$.
\end{numpar}

At this point, let's take a moment to appreciate the power of \cref{thm:1PShFreeCocompletion}: It gave us adjunctions
\begin{equation*}
	\abs*{\,\cdot\,}\colon \cat{sSet}\doublelrmorphism\cat{Top}\noloc {\Sing}\,,\quad \operatorname{ho}\colon \cat{sSet}\doublelrmorphism\cat{Cat}\noloc {\N}\,,\quad\text{and}\quad -\times X\colon \cat{sSet}\doublelrmorphism \cat{sSet}\noloc {\F(X,-)}
\end{equation*}
 essentially for free!

\subsection{Quasi-categories and Kan complexes}
In this subsection, we'll introduce \emph{quasi-categories}, a class of simplicial sets that behaves very similarly to ordinary categories. To motivate the definition, we start with a lemma.
\begin{lem}\label{lem:LiftingConditions}
	\begin{alphanumerate}
		\item Let $Y$ be a topological space and let $\Sing Y$ be the singular simplicial set of $Y$ as in \cref{par:GeometricRealisation}. For all $n\geqslant 1$ and all $0\leqslant i\leqslant n$, every horn filling problem\label{enum:LiftingSing}
		\begin{equation*}
			\begin{tikzcd}
				\Lambda_i^n\rar\dar & \Sing Y\\
				\Delta^n\urar[dashed]
			\end{tikzcd}
		\end{equation*}
		has a solution.
		\item Let $\Cc$ be a small category and let $\N(\Cc)$ be the nerve of $\Cc$ as in \cref{par:Nerve}. For all $n\geqslant 2$ and all $0<i<n$, every inner horn filling problem\label{enum:LiftingN}
		\begin{equation*}
			\begin{tikzcd}
				\Lambda_i^n\rar\dar & \N(\Cc)\\
				\Delta^n\urar[dashed]
			\end{tikzcd}
		\end{equation*}
		has a unique solution. Furthermore, if $X$ is a simplicial set with this horn filling property, then $X\cong \N(\Cc)$ for some category $\Cc$ \embrace{which is necessarily the homotopy category $\operatorname{ho}(X)$}.
	\end{alphanumerate}
\end{lem}
\begin{proof}[Proof sketch]
	By the adjunction $\abs*{\,\cdot\,}\colon \cat{sSet}\shortdoublelrmorphism\cat{Top}\noloc {\Sing}$ from \cref{par:GeometricRealisation}, a horn filling problem as in \cref{enum:LiftingSing} is equivalent to
	\begin{equation*}
		\begin{tikzcd}
			\abs*{\Lambda_i^n}\rar\dar & Y\\
			\abs*{\Delta^n}\urar[dashed]
		\end{tikzcd}
	\end{equation*}
	This one always has a solution since the topological space $\abs*{\Lambda_i^n}$ is a retract of $\abs*{\Delta^n}$. This proves \cref{enum:LiftingSing}. For \cref{enum:LiftingN}, recall from \cref{par:Nerve} that a morphism $\Delta^n\rightarrow \N(\Cc)$ corresponds to a chain $x_0\rightarrow \dotsb\rightarrow x_n$ in $\Cc$. But the morphisms $x_j\rightarrow x_{j+1}$ are already given by $\Lambda_i^n\rightarrow \N(\Cc)$. This shows the unique horn filling assertion from \cref{enum:LiftingN}. The additional assertion is more or less straightforward if you use the description of the homotopy category from \cref{par:HomotopyCategory} below. For a complete proof, see \cite[Theorem~II.25]{HigherCatsI}.
\end{proof}
Recall that by Grothendieck's \emph{homotopy hypothesis}, topological spaces should be the same as $\infty$-groupoids, so in particular, they should provide examples of $\infty$-categories. Furthermore, every ordinary category should give rise to an $\infty$-category too. So if we try to model $\infty$-categories by a specific class of simplicial sets, that class should contain $\Sing Y$ for every topological space $Y$ and $\N(\Cc)$ for every small category $\Cc$. It then feels reasonable to look for a common generalisation of the horn filling conditions from \cref{lem:LiftingConditions}\cref{enum:LiftingSing} and~\cref{enum:LiftingN}, which is precisely what the definition of quasi-categories does:
\begin{defi}[Boardman--Vogt, \cite{BoardmanVogt}]\label{def:QuasiCategory}
	A \emph{quasi-category} (or \emph{$\infty$-category}) is a simplicial set $\Cc$ such that for all $n\geqslant 2$ and all $0<i<n$, every inner horn filling problem
	\begin{equation*}
		\begin{tikzcd}
			\Lambda_i^n\rar\dar & \Cc\\
			\Delta^n\urar[dashed]
		\end{tikzcd}
	\end{equation*}
	has a solution. If, moreover, all horn filling problems for $n\geqslant 1$ and $0\leqslant i\leqslant n$ have solutions, then $\Cc$ is called a \emph{Kan complex}. We let $\cat{Kan}\subseteq\cat{QCat}\subseteq\cat{sSet}$ denote the full subcategories spanned by Kan complexes and quasi-categories.
\end{defi}
The rest of \cref{sec:SimplicialSets} as well as the entirety of \cref{sec:JoyalLifting} and \cref{sec:Straightening} will be spent convincing you that \cref{def:QuasiCategory} is a sensible definition and that quasi-categories really behave like ordinary categories. Let's begin by giving a dictionary of the most basic categorical notions and their counterparts in the world of quasi-categories.
\begin{numpar}[Objects and Morphisms.]
	Let $\Cc$ be a quasi-category. We'll use the following suggestive terminology. If $x$ is a $0$-simplex in $\Cc$, we'll say that \emph{$x$ is an object in $\Cc$} and write $x\in\Cc$ instead of $x\in\Cc_0$. We also write $\{x\}\rightarrow \Cc$ for the map $\Delta^0\rightarrow \Cc$ induced by $x$. If $\alpha$ is a $1$-simplex in $\Cc$ and $x=d_1^*(\alpha)$, $y=d_0^*(\alpha)$, we'll say that \emph{$\alpha\colon x\rightarrow y$ is a morphism in $\Cc$}. For an object $x\in\Cc$, we'll call the degenerate $1$-simplex $s_0^*(x)\in\Cc_1$ the \emph{identity on $x$} and we'll write $\id_x\colon x\rightarrow x$.
\end{numpar}
\begin{numpar}[Functors and natural transformations.]
	A \emph{functor of quasi-categories} is simply a map of simplicial sets. If $\Cc$ and $\Dd$ are quasi-categories, then the construction $\F(\Cc,\Dd)$ from \cref{par:FInternalHom} plays the role of the category of functors from $\Cc$ to $\Dd$. We'll show in \cref{cor:FIsKanComplex} that $\F(\Cc,\Dd)$ is indeed a quasi-category again. Furthermore, if $F,G\colon \Cc\rightarrow \Dd$ are functors of quasi-categories, then a \emph{natural transformation $\eta\colon F\Rightarrow G$} is a functor $\eta\colon \Delta^1\times\Cc\rightarrow \Dd$ such that the following diagram commutes:
	\begin{equation*}
		\begin{tikzcd}
			\{0\}\times \Cc\dar\dar[phantom,""{name=A}]\arrow[from=2-2,to=1-1,commutes,xshift=-1ex]\drar[bend left,"F"]& \\
			\Delta^1\times \Cc\rar["\eta"]& \Dd\\
			\{1\}\times \Cc\uar\uar[phantom,""{name=A}]\arrow[from=2-2,to=3-1,commutes,xshift=-1ex]\urar[bend right, "G"'] & 
		\end{tikzcd}
	\end{equation*}
	By \cref{par:FInternalHom}, we may equivalently view $\eta$ as a $1$-simplex $\Delta^1\rightarrow \F(\Cc,\Dd)$ from $F$ to $G$. That is, natural transformations are morphisms in the functor quasi-category, as they should be (except that we don't know yet that $\F(\Cc,\Dd)$ is a quasi-category again). Further evidence that $\F(\Cc,\Dd)$ is the right construction will be given in \cref{lem:SimplicialHoNerveAdjunction} below.
\end{numpar}
\begin{numpar}[Arrows, slices, and $\Hom$.]\label{par:HomInQuasiCategories}
	We let $\Ar(\Cc)\coloneqq \F(\Delta^1,\Cc)$ denote the \emph{arrow quasi-category} of $\Cc$. The inclusions $\{0\}\rightarrow \Delta^1$ and $\{1\}\rightarrow\Delta^1$ induce a source and a target projection $s,t\colon \Ar(\Cc)\rightarrow \Cc$. Furthermore, for $x,y\in \Cc$, we define the \emph{Hom anima $\Hom_\Cc(x,y)$} and the \emph{slice quasi-category $\Cc_{x/}$} via the pullbacks\label{enum:HomInQuasiCategories}
	\begin{equation*}
		\begin{tikzcd}
			\Hom_\Cc(x,y)\dar\rar\drar[pullback] & \Cc_{x/}\dar\rar\drar[pullback] & \Ar(\Cc)\dar["{(s,t)}"]\\
			\{x\}\times\{y\}\rar & \{x\}\times\Cc\rar & \Cc\times\Cc
		\end{tikzcd}
	\end{equation*}
	We'll prove in \cref{cor:HomAnima} that $\Hom_\Cc(x,y)$ is always an anima in the sense of \cref{def:Anima} below, and we'll prove in \cref{cor:FIsKanComplex} that $\Ar(\Cc)$ and $\Cc_{x/}$ are quasi-categories. So these constructions live up to their names. Furthermore, it follows from \cref{lem:SimplicialHoNerveAdjunction} below and $\Delta^1\cong\N([1])$ that we have an isomorphism of simplicial sets $\Ar(\N(\Dd))\cong \N(\Ar(\Dd))$ for every ordinary category $\Dd$, so it makes sense to use the same notation as in \cref{con:1ArrowCategory}. Furthermore, since $\N\colon \cat{Cat}\rightarrow \cat{sSet}$ preserves pullbacks (being a right adjoint), it follows that $\N(\Dd)_{y/}\cong \N(\Dd_{y/})$ for all $y\in\Dd$. Finally, it follows that $\Hom_{\N(\Dd)}(x,y)$ is a \emph{discrete simplicial set}, that is, a disjoint union of copies of $\Delta^0$, with the indexing set being $\Hom_\Dd(x,y)$. So our construction of $\Hom$ recovers the usual notion for ordinary categories.
	
	Be aware that $\Cc_{x/}$ is \emph{not} the slice construction from \cite[Proposition~\HTTthm{1.2.9.2}]{HTT} or \cite[Definition~1.4.13]{Land}; instead, it corresponds to their \emph{fat slice} $\Cc^{x/}$. I like our definition better because it is in line with \cref{con:1SliceCategory} and we'll avoid using the other slice construction (or rather hide its unavoidable usages in black boxes). It can be shown that while the two slice constructions are not isomorphic, they are equivalent as quasi-categories  (see \cite[Proposition~\HTTthm{4.2.1.5}]{HTT} or \cite[Proposition~2.5.27]{Land}), so once we're out of the simplicial swamp (that is, starting from \cref{sec:InftyCategoryTheory}), the distinction won't matter.
\end{numpar}
\begin{numpar}[Compositions.]\label{par:Composition}
	Morphisms in a quasi-category can be composed, albeit not uniquely. To explain how this works, let's first describe an equivalence relation on morphisms. For morphisms $\alpha,\alpha'\colon x\rightarrow y$ we say \emph{$\alpha$ and $\alpha'$ are equivalent}, $\alpha\simeq \alpha'$, if the map $\sigma\colon \partial\Delta^2\rightarrow \Cc$ represented by the hollow triangle
	\begin{equation*}
		\sigma=\begin{tikzpicture}[commutative diagrams/every diagram,baseline=(mid.base)]
			\path node[outer sep=0.25ex] (0) at (0,0) {$x$} ++(0:3.8em) node[text depth=0pt,outer sep=0.25ex] (1) {$y$} ++ (120:3.8em) node[outer sep=0.25ex] (2) {$y$};
			\path (0) to node[pos=0.5] (mid) {\phantom{x}} (2);
			\path[commutative diagrams/.cd, every arrow, every label]
			(0) edge node[swap] {$\alpha$} (1)
			(1) edge node[swap,text depth=0pt] {$\id_z$} (2)
			(0) edge node {$\alpha'$} (2);
		\end{tikzpicture}
	\end{equation*}
	can be extended to a map $\ov\sigma\colon\Delta^2\rightarrow \Cc$ satisfying $\ov\sigma|_{\partial\Delta^2}=\sigma$ (thus \enquote{filling} the triangle above). Even though the definition is asymmetric in $\alpha$ and $\alpha'$, it turns out that \enquote{$\simeq$} is an equivalence relation on $\Cc_1$. For reflexivity, we can fill the triangle by taking $\ov\sigma_\mathrm{ref}\coloneqq s_1^*(\alpha)$ to be a degenerate simplex. For symmetry and transitivity, consider the maps $\vartheta_\mathrm{sym}\colon \Lambda_1^3\rightarrow \Cc$ and $\vartheta_\mathrm{trans}\colon \Lambda_2^3\rightarrow \Cc$ represented as the following hollow tetrahedra (each tetrahedron is missing its interior as well as one face; the missing faces have been highlighted):
	\begin{equation*}
		\vartheta_\mathrm{sym}=\begin{tikzpicture}[commutative diagrams/every diagram, baseline=(mid.base)]
			\path node[outer sep=0.25ex] (0) at (0,0) {$x$} ++(0:3.8em) node[text depth=0pt,outer sep=0.25ex] (1) {$y$} ++ (120:3.8em) node[outer sep=0.25ex] (2) {$y$} ++(180:3.8em) node[outer sep=0.25ex] (3) {$y$};
			\path (0) to coordinate[pos=0] (020) coordinate[pos=1] (021) (2);
			\path (0) to coordinate[pos=0] (030) coordinate[pos=1] (031) (3);
			\path (2) to coordinate[pos=0] (230) coordinate[pos=1] (231) (3);
			\fill[white!93!black]%[pattern={Lines[xshift=-0.5em,angle=45, line width=0.2em, distance=0.4em]},pattern color=white!93!black]
			(020) to (021) to[out=150,in=270] (230) to (231) to[out=270,in=30] (031) to (030) to[out=30,in=150] cycle;
			\path[commutative diagrams/.cd, every arrow, every label]
			(0) edge node[swap] {$\alpha$} (1)
			(1) edge node[swap,text depth=0pt] {$\id_y$} (2)
			(0) edge node[text depth=0pt,swap,xshift=-1.3em] {$\smash{\alpha'}\vphantom{\alpha}$} (2)
			(0) edge node[text depth=0pt] {$\alpha$} (3)
			(2) edge node[swap] {$\id_y$} (3)
			(1) edge[shorten <=2.75em] (3)
			(1) edge[shorten >=2.75em,-] node[text depth=0.15em,pos=0.3,sloped] {$\id_y$} (3);
			\path (0) to node[pos=0.5] (mid) {\phantom{x}} (2);
		\end{tikzpicture}\quad\text{and}\quad
		\vartheta_\mathrm{trans}=\begin{tikzpicture}[commutative diagrams/every diagram, baseline=(mid.base)]
			\path node[outer sep=0.25ex] (0) at (0,0) {$x$} ++(0:3.8em) node[text depth=0pt,outer sep=0.25ex] (1) {$y$} ++ (120:3.8em) node[outer sep=0.25ex] (2) {$y$} ++(180:3.8em) node[outer sep=0.25ex] (3) {$y$};
			\path (0) to coordinate[pos=0] (010) coordinate[pos=1] (011) (1);
			\path (0) to coordinate[pos=0] (030) coordinate[pos=1] (031) (3);
			\path (1) to coordinate[pos=0] (130) coordinate[pos=1] (131) (3);
			\fill[white!93!black]%[pattern={Lines[xshift=-0.75em,angle=25, line width=0.2em, distance=0.4em]}, pattern color=white!93!black]
			(010) to (011) to[out=90,in=240] (130) to (131) to[out=240,in=30] (031) to (030) to[out=30,in=90] cycle;
			\path[commutative diagrams/.cd, every arrow, every label]
			(0) edge node[swap] {$\alpha$} (1)
			(1) edge node[swap,text depth=0pt] {$\id_y$} (2)
			(0) edge  node[text depth=0pt,swap,xshift=-1.3em] {$\smash{\alpha'}\vphantom{\alpha}$} (2)
			(0) edge node[text depth=0pt] {$\smash{\alpha''}\vphantom{\alpha}$} (3)
			(2) edge node[swap] {$\id_y$} (3)
			(1) edge[shorten <=2.75em] (3)
			(1) edge[shorten >=2.75em,-] node[text depth=0.15em,pos=0.3,sloped] {$\id_y$} (3);
			\path (0) to node[pos=0.5] (mid) {\phantom{x}} (2);
		\end{tikzpicture}
	\end{equation*}
	More precisely, the face $\vartheta_\mathrm{sym}|_{\Delta^{\{0,1,2\}}}$ is a $2$-simplex witnessing $\alpha\simeq \alpha'$, whereas the faces $\vartheta_\mathrm{sym}|_{\Delta^{\{0,1,3\}}}=s_1^*(\alpha)$ and $\vartheta_\mathrm{sym}|_{\Delta^{\{1,2,3\}}}=s_1^*(\id_y)=s_1^*s_0^*(y)$ are degenerate simplices. Likewise, the faces $\vartheta_\mathrm{trans}|_{\Delta^{\{0,1,2\}}}$ and $\vartheta_\mathrm{trans}|_{\Delta^{\{0,2,3\}}}$ are $2$-simplices witnessing $\alpha\simeq \alpha'$ and $\alpha'\simeq \alpha''$, respectively, whereas the face $\vartheta_\mathrm{trans}|_{\Delta^{\{1,2,3\}}}=s_1^*(\id_y)=s_1^*s_0^*(y)$ is a degenerate simplex. By \cref{def:QuasiCategory}, the horns $\vartheta_\mathrm{sym}$ and $\vartheta_\mathrm{trans}$ can be extended to $3$-simplices $\ov\vartheta_\mathrm{sym},\ov\vartheta_\mathrm{trans}\colon \Delta^3\rightarrow \Cc$ satisfying $\ov\vartheta_\mathrm{sym}|_{\Lambda_1^3}=\vartheta_\mathrm{sym}$ and $\ov\vartheta_\mathrm{trans}|_{\Lambda_2^3}=\vartheta_\mathrm{trans}$ (in other words, the hollow tetrahedra can be \enquote{filled}). Now the face $\ov\sigma_\mathrm{sym}\coloneqq \ov\vartheta_\mathrm{sym}|_{\Delta^{\{0,2,3\}}}$ is a $2$-simplex witnessing $\alpha'\simeq \alpha$ and $\ov\sigma_\mathrm{trans}\coloneqq \ov\vartheta_\mathrm{trans}|_{\Delta^{\{0,1,3\}}}$ is a $2$-simplex witnessing $\alpha\simeq \alpha''$, which proves symmetry and transitivity.	
	
	Now let's define compositions. For morphisms $\alpha\colon x\rightarrow y$ and $\beta\colon y\rightarrow z$ in $\Cc$, consider the map $\sigma\colon\Lambda_1^2\rightarrow \Cc$ represented by
	\begin{equation*}
		\sigma=\begin{tikzpicture}[commutative diagrams/every diagram,baseline=(mid.base)]
			\path node[outer sep=0.25ex] (0) at (0,0) {$x$} ++(0:3.8em) node[text depth=0pt,outer sep=0.25ex] (1) {$y$} ++ (120:3.8em) node[outer sep=0.25ex] (2) {$z$};
			\path (0) to node[pos=0.5] (mid) {\phantom{x}} (2);
			\path[commutative diagrams/.cd, every arrow, every label]
			(0) edge node[swap] {$\alpha$} (1)
			(1) edge node[swap,text depth=0pt] {$\beta$} (2)
			(0) edge[dotted] (2);
		\end{tikzpicture}
	\end{equation*}
	By \cref{def:QuasiCategory}, this horn admits a filler, that is, a morphism $\ov\sigma\colon \Delta^2\rightarrow\Cc$ such that $\ov\sigma|_{\Lambda_1^2}=\sigma$. If $\gamma\colon x\rightarrow z$ is the morphism in $\Cc$ represented by $\ov\sigma|_{\Delta^{\{0,2\}}}\colon \Delta^{\{0,2\}}\rightarrow \Cc$, then $\gamma$ is called \emph{a composition of $\alpha$ and $\beta$} and we write $\gamma\simeq\beta\circ\alpha$. In particular, composition of morphisms is not unique in a general quasi-category, as filling inner horns is not unique.\footnote{Conversely, uniqueness of composition in ordinary categories accounts for uniqueness of filling inner horns in nerves of an ordinary categories, see \cref{lem:LiftingConditions}\cref{enum:LiftingN}.} However, if $\gamma,\gamma'\colon x\rightarrow z$ are any two compositions, then $\gamma$ and $\gamma'$ are equivalent in the sense of \cref{par:Composition} above. Indeed, then we can consider the morphism $\vartheta\colon\Lambda_1^3\rightarrow \Cc$ represented as follows (the missing face has been highlighted again):
	\begin{equation*}
		\vartheta=\begin{tikzpicture}[commutative diagrams/every diagram, baseline=(mid.base)]
			\path node[outer sep=0.25ex] (0) at (0,0) {$x$} ++(0:3.8em) node[text depth=0pt,outer sep=0.25ex] (1) {$y$} ++ (120:3.8em) node[outer sep=0.25ex] (2) {$z$} ++(180:3.8em) node[outer sep=0.25ex] (3) {$z$};
			\path (0) to coordinate[pos=0] (020) coordinate[pos=1] (021) (2);
			\path (0) to coordinate[pos=0] (030) coordinate[pos=1] (031) (3);
			\path (2) to coordinate[pos=0] (230) coordinate[pos=1] (231) (3);
			\fill[white!93!black]%[pattern={Lines[xshift=-3em,angle=45, line width=0.2em, distance=0.4em]},pattern color=white!93!black]
			(020) to (021) to[out=150,in=270] (230) to (231) to[out=270,in=30] (031) to (030) to[out=30,in=150] cycle;
			\path[commutative diagrams/.cd, every arrow, every label]
			(0) edge node[swap] {$\alpha$} (1)
			(1) edge node[swap,text depth=0pt] {$\beta$} (2)
			(0) edge node[text depth=0pt,swap,xshift=-1.2em] {$\gamma$} (2)
			(0) edge node[text depth=0pt] {$\smash{\gamma'}\vphantom{\gamma}$} (3)
			(2) edge node[swap] {$\id_z$} (3)
			(1) edge[shorten <=2.75em] (3)
			(1) edge[shorten >=2.75em,-] node[text depth=0pt,pos=0.35,swap] {$\beta$} (3);
			%\fill[black,opacity=0.2] (020) to (021) to[out=60,in=0] (230) to (231) to[out=180,in=120] (031) to (030) to[out=300,in=240] cycle;
			%\fill[black,opacity=0.2] (0.center) to (2.center) to (3.center) to cycle;
			\path (0) to node[pos=0.5] (mid) {\phantom{x}} (2);
		\end{tikzpicture}
	\end{equation*}
	Concretely, $\vartheta|_{\Delta^{\{0,1,2\}}}$ and $\vartheta|_{\Delta^{\{0,1,3\}}}$ are $2$-simplices that witness $\gamma$ and $\gamma'$ being compositions of $\alpha$ and $\beta$, whereas $\vartheta|_{\Delta^{\{1,2,3\}}}=s_1^*(\beta)\colon \Delta^{\{1,2,3\}}\rightarrow\Cc$ is a degenerate simplex. By \cref{def:QuasiCategory}, we can extend $\vartheta$ to a map $\ov\vartheta\colon \Delta^3\rightarrow \Cc$ and then the face $\ov\sigma\coloneqq\ov\vartheta|_{\Delta^{\{0,2,3\}}}$ is a $2$-simplex witnessing an equivalence $\gamma\simeq\gamma'$.
\end{numpar}
\begin{numpar}[The homotopy category.]\label{par:HomotopyCategory}
	We can now describe the homotopy category $\operatorname{ho}(\Cc)$ from \cref{par:Nerve} in more explicit terms. As already explained there, the objects of $\operatorname{ho}(\Cc)$ are the $0$-simplices $\Cc_0$, that is, the objects of $\Cc$. We've seen in \cref{par:Nerve} that the morphism may cause some problems since we might need to add compositions. However, by \cref{par:Composition} above, compositions already exist in $\Cc$, they just might not be unique. So we find that the set of morphisms of $\operatorname{ho}(\Cc)$ is given by $\Cc_1/\!\simeq$, the set of $1$-simplices modulo the equivalence condition from \cref{par:Composition}.
	
	To make this argument precise, one would have to check that $\operatorname{ho}(\Cc)$ as described above satisfies the universal property of the colimit $\colimit_{(n,\Delta^n\rightarrow\Cc)}[n]$ in $\cat{Cat}$. This is technical, but straightforward, and we leave the details to you.
\end{numpar}
As a consequence, we can prove that the construction $\F(-,-)$ from \cref{par:FInternalHom} is compatible with the functor category construction for ordinary categories.
\begin{lem}\label{lem:SimplicialHoNerveAdjunction}
	If $\Cc$ is a quasi-category and $\Dd$ is an ordinary category, then there is an isomorphism of simplicial sets
	\begin{equation*}
		\F\bigl(\Cc,\N(\Dd)\bigr)\cong \N\bigl(\Fun(\operatorname{ho}(\Cc),\Dd)\bigr)
	\end{equation*}
	In particular, if $\Cc\cong \N(\Cc')$ is the nerve of an ordinary category $\Cc'$, we get an isomorphism $\F(\N(\Cc'),\N(\Dd))\cong \N(\Fun(\Cc',\Dd))$.
\end{lem}
\begin{proof}[Proof sketch]
	For all $n\geqslant 0$, we obtain the following chain of bijections, all of which are compatible with the simplicial structure maps:
	\begin{align*}
		\F\bigl(\Cc,\N(\Dd)\bigr)_n\cong \Hom_{\cat{sSet}}\bigl(\Delta^n\times \Cc,\N(\Dd)\bigr)&\cong \Hom_{\cat{Cat}}\bigl(\operatorname{ho}(\Delta^n\times \Cc),\Dd\bigr)\\
		&\cong \Hom_{\cat{Cat}}\bigl(\operatorname{ho}(\Cc)\times[n],\Dd\bigr)\\
		&\cong \Hom_{\cat{Cat}}\bigl([n],\Fun(\operatorname{ho}(\Cc),\Dd)\bigr)\\
		&\cong \N\bigl(\Fun(\operatorname{ho}(\Cc),\Dd)\bigr)_n\,.
	\end{align*} 
	In the first step, we use the definition of $\F(-,-)$ from \cref{par:FInternalHom}. In the second step, we use the adjunction $\operatorname{ho}\colon\cat{sSet}\shortdoublelrmorphism \cat{Cat}\noloc\N$ from \cref{par:Nerve}. In the third step, we use that $\operatorname{ho}$ commutes with products of quasi-categories, which follows from the description in \cref{par:HomotopyCategory}. In the fourth step, we use \enquote{currying} for ordinary categories. Finally, in the fifth step we plug in the definition of $\N(\Fun(\operatorname{ho}(\Cc),\Dd))$.
	
	To prove the \enquote{in particular}, it suffices to see that the unit $u_{\Cc'}\colon\Cc'\rightarrow \operatorname{ho}\N(\Cc')$ of the adjunction $\operatorname{ho}\dashv \N$ from \cref{par:Nerve} is an isomorphism of categories (and we really need an isomorphism, not just an equivalence of categories). This is easy to check using \cref{lem:LiftingConditions}\cref{enum:LiftingN} and the explicit description of $\operatorname{ho}\N(\Cc)$ from \cref{par:HomotopyCategory}.%(also note that, by \cref{lem:1FullyFaithfulConservativeAdjunction}\cref{enum:1FullyFaithfulIffUnitEquivalence}, this implies that $\N\colon \cat{Cat}\rightarrow\cat{sSet}$ is fully faithful).
\end{proof}
\begin{numpar}[Equivalences in quasi-categories.]\label{par:Equivalence}
	We say that a morphism $\alpha\colon x\rightarrow y$ in $\Cc$ is an \emph{equivalence} if it becomes an isomorphism in the homotopy category $\operatorname{ho}(\Cc)$. Equivalently, $\alpha$ is an equivalence if and only if the horns $\sigma_\mathrm{left}\colon\Lambda_0^2\rightarrow\Cc$ and $\sigma_\mathrm{right}\colon\Lambda_2^2\rightarrow \Cc$ represented by\label{enum:Equivalence}
	\begin{equation*}
		\sigma_\mathrm{left}=\begin{tikzpicture}[commutative diagrams/every diagram,baseline=(mid.base)]
			\path node[outer sep=0.25ex] (0) at (0,0) {$x$} ++(0:3.8em) node[text depth=0pt,outer sep=0.25ex] (1) {$y$} ++ (120:3.8em) node[outer sep=0.25ex] (2) {$x$};
			\path (0) to node[pos=0.5] (mid) {} (2);
			\path[commutative diagrams/.cd, every arrow, every label]
			(0) edge node[swap] {$\alpha$} (1)
			(1) edge[dotted] (2)
			(0) edge node {$\id_y$} (2);
		\end{tikzpicture}
		\quad\text{and}\quad
		\sigma_\mathrm{right}=\begin{tikzpicture}[commutative diagrams/every diagram,baseline=(mid.base)]
			\path node[outer sep=0.25ex] (0) at (0,0) {$y$} ++(0:3.8em) node[text depth=0pt,outer sep=0.25ex] (1) {$x$} ++ (120:3.8em) node[outer sep=0.25ex] (2) {$y$};
			\path (0) to node[pos=0.5] (mid) {} (2);
			\path[commutative diagrams/.cd, every arrow, every label]
			(0) edge[dotted] (1)
			(1) edge node[swap] {$\alpha$} (2)
			(0) edge node {$\id_y$} (2);
		\end{tikzpicture}
	\end{equation*}
	can be filled, that is, if and only if there are $2$-simplices $\ov\sigma_\mathrm{left},\ov\sigma_\mathrm{right}\colon \Delta^2\rightarrow\Cc$ such that $\ov\sigma_\mathrm{left}|_{\Lambda_0^2}=\sigma_\mathrm{left}$ and $\ov\sigma_\mathrm{right}|_{\Lambda_2^2}=\sigma_\mathrm{right}$. Indeed, by \cref{par:Composition} above, $\ov\sigma_\mathrm{left}$ corresponds to a left inverse of $\alpha$ and $\ov\sigma_\mathrm{right}$ corresponds to a right inverse. We say \emph{$x$ and $y$ are equivalent} and write $x\simeq y$ if there exists an equivalence $\alpha\colon x\rightarrow y$.
\end{numpar}
\begin{numpar}[Sub-quasi-categories.]\label{par:SubQuasiCategories}
	If $\Cc$ is a quasi-category and $S_0\subseteq \Cc_0$ is a set of $0$-simplices, we can define a sub-simplicial set $\Cc[S_0]\subseteq \Cc$ by declaring that a simplex $\Delta^n\rightarrow \Cc$ belongs to $\Cc[S_0]$ if and only if all its vertices $\{i\}\rightarrow\Delta^n\rightarrow\Cc$ for $0\leqslant i\leqslant n$ belong to $S_0$. It's straightforward to check that $\Cc[S_0]$ is a quasi-category again: If $\Lambda_i^n\rightarrow \Cc[S_0]$ is an inner horn, any filler $\Delta^n\rightarrow\Cc$ will automatically belong to $\Cc[S_0]$, because $\Lambda_i^n\rightarrow\Delta^n$ is a bijection on vertices whenever $n\geqslant 2$. We call $\Cc[S_0]$ the \emph{full sub-quasi-category spanned by $S_0$}.\label{enum:SubQuasiCategories}
	
	Similarly, assume $S_1\subseteq \Cc_1$ is a set of $1$-simplices which contains all identities and is closed under the equivalence relation from \cref{par:Composition} as well as under compositions. We can define a sub-simplicial set $\Cc[S_1]\subseteq \Cc$ by declaring that a simplex $\Delta^n\rightarrow \Cc$ belongs to $\Cc[S_1]$ if and only if all its edges $\Delta^{\{i,j\}}\rightarrow\Delta^n\rightarrow\Cc$ for $0\leqslant i,j\leqslant n$ belong to $S_1$. Once again, if $\Lambda_i^n\rightarrow \Cc[S_1]$ is an inner horn, any filler $\Delta^n\rightarrow\Cc$ will automatically belong to $\Cc[S_1]$, because any \enquote{missing} edge in $\Delta^n\smallsetminus \Lambda_i^n$ is a composition of edges in $\Lambda_i^n$. Hence $\Cc[S_1]$ is a quasi-category again, and we call it the \emph{sub-quasi-category spanned by $S_1$} (and usually we'll emphasise that $\Cc[S_1]$ is not full).
\end{numpar}
\begin{numpar}[The opposite quasi-category.]\label{par:Opposite}
	Every quasi-category $\Cc$ admits an \emph{opposite quasi-category} $\Cc^\op$. In fact, this construction works for arbitrary simplicial sets.  Let $(-)^\op\colon \cat{Cat}\rightarrow\cat{Cat}$ be the functor that sends a category to its opposite. Consider the composition
	\begin{equation*}
		\nabla\colon\IDelta\overset{U}{\longrightarrow}\cat{Cat}\xrightarrow{(-)^\op}\cat{Cat}\overset{\N}{\longrightarrow}\cat{sSet}\,,
	\end{equation*}
	where $U$ and $\N$ are the functors from \cref{par:Nerve}. This composition sends $[n]\mapsto\N([n]^\op)\cong \Delta^n$, since there is an isomorphism of categories $[n]^\op\cong [n]$ given by sending $i\mapsto n-i$. Nevertheless, $\nabla$ does \emph{not} coincide with the Yoneda embedding $\Yo_{\IDelta}\colon\IDelta\rightarrow\cat{sSet}$, which also sends $[n]\mapsto\Delta^n$, since the effect on morphisms is different ($\nabla$ \enquote{reverses the order} of face and degeneracy maps). According to \cref{thm:1PShFreeCocompletion}, $\nabla$ admits a unique colimit-preserving extension, which we denote $(-)^\op\colon \cat{sSet}\rightarrow\cat{sSet}$. Intuitively, if $X$ is a simplicial set, then $X^\op$ is given by inverting the direction of every $1$-simplex and by reversing the order of all face and degeneracy maps. It's straightforward to check that $(-)^\op\circ (-)^\op\simeq \id_{\cat{sSet}}$ (so the right adjoint from \cref{thm:1PShFreeCocompletion} is just $(-)^\op$ again) and that $\N(\Dd)^\op\cong \N(\Dd^\op)$ holds for every ordinary category $\Dd$. Furthermore, if $\Cc$ is a quasi-category, then so is $\Cc^\op$, because $(-)^\op$ transforms an inner horn inclusion $\Lambda_i^n\rightarrow \Delta^n$, where $n\geqslant 2$ and $0<i<n$, into $\Lambda_{n-i}^n\rightarrow\Delta^n$, which is again an inner horn inclusion.
\end{numpar}
This finishes our preliminary ordinary-to-quasi-categories dictionary. Next, we'll introduce another notion that will play a central role in these notes.
\begin{defi}\label{def:Anima}
	A quasi-category $\Cc$ is called an \emph{anima} (plural \emph{animae}) if all its morphisms are equivalences in the sense of \cref{par:Equivalence}. For an arbitrary quasi-category, we let $\core(\Cc)\subseteq \Cc$ be the (non-full) sub-quasi-category spanned by the equivalences, as defined in \cref{par:SubQuasiCategories}.
\end{defi}
It follows immediately that $\core(\Cc)$ is the largest anima contained in $\Cc$.
\begin{thm}[Joyal, {\cite[Corollary~\href{https://people.math.rochester.edu/faculty/doug/otherpapers/Joyal-QCKC.pdf\#page=3}{1.4}]{JoyalLifting}}]\label{thm:AnimaeAreKanComplexes}
	A quasi-category $\Cc$ is a Kan complex if and only if it is an anima.
\end{thm}
\begin{proof}
	If $\Cc$ is a Kan complex, then the horns from \cref{par:Equivalence} can be filled, so $\Cc$ is an anima. The converse is much harder to prove and we'll postpone it to \cref{cor:AnimaKanComplexes}.
\end{proof}
So on one hand, by \cref{def:Anima}, animae are the analogues of groupoids in quasi-category theory. In fact, people used (and continue to use) the term \emph{$\infty$-groupoid}, before Beilinson, Clausen, and Scholze decided to invent a new term. On the other hand, \cref{thm:AnimaeAreKanComplexes} says that animae are the same as \emph{Kan complexes}. We'll see in \cref{sec:SimplicialHomotopyTheory} that for the purposes of homotopy theory, Kan complexes and topological spaces can be used interchangeably. This fits perfectly with Grothendieck's \emph{homotopy hypothesis}, which predicts that the theory of $\infty$-groupoids/animae should essentially be the homotopy theory of topological spaces.

We'll keep the terms \emph{anima} and \emph{Kan complex} distinct until we've finished the proof that they coincide (\cref{cor:AnimaKanComplexes}). After that, we'll use the terms interchangeably. Starting from \cref{sec:InftyCategoryTheory}, we try to keep our arguments as model-independent as possible. Accordingly, we'll settle on \emph{anima}, only using \emph{Kan complex} to emphasise that a certain (non-model-independent) argument takes place in thequasi-categorical model.


\subsection{Simplicially enriched categories}
Until now, we know a good supply of Kan complexes, given by $\Sing Y$ for every topological space $Y$ (see \cref{lem:LiftingConditions}\cref{enum:LiftingSing}). We'll see in \cref{thm:SimplicialApproximation} that these exhaust essentially all Kan complexes. Besides that, our only other examples of quasi-categories are nerves of ordinary categories (see \cref{lem:LiftingConditions}\cref{enum:LiftingN}). These can't possibly be all! The goal of this subsection is to provide a rich source of non-trivial examples of quasi-categories, using a fancier version of the nerve construction.
\begin{numpar}[\enquote{Definition}.]
	A \emph{simplicially enriched category} $\Cc$ is the same as a category, except that the morphisms sets $\Hom_\Cc(x,y)$ for $x,y\in\Cc$ are replaced by simplicial sets $\F_\Cc(x,y)$. Composition of morphisms is now a map of simplicial sets $\circ\colon \F_\Cc(x,y)\times\F_\Cc(y,z)\rightarrow\F_\Cc(x,z)$ and the identity on any object $x\in\Cc$ is a $0$-simplex $\id_x\in\F_\Cc(x,x)_0$. Composition and identities are supposed to satisfy some straightforward compatibilities that we won't spell out. Furthermore, if $\Cc$ and $\Dd$ are simplicially enriched categories, there is an obvious notion of a \emph{simplicially enriched functor $F\colon \Cc\rightarrow\Dd$}. We let $\cat{Cat}_\Delta$ denote the category of (small) simplicially enriched categories and simplicially enriched functors between them. 
	
	If you would like to see a formal definition of these notions, have a look at \cite[Definitions~1.2.34 and~1.2.35]{Land}.
\end{numpar}
\begin{con}\label{con:SimplicialNerve}
	We'll construct a \enquote{simplicially thickened} versions of the ordinary categories $[n]$ and use them to define a simplicial nerve functor $\N^\Delta\colon\cat{Cat}_\Delta\rightarrow\cat{sSet}$. This is originally due to Cordier and Porter \cite{CordierPorter}.
	
	To start with, the simplicially enriched category $\CC[\Delta^n]$ is given as follows: It's objects are $0,1,\dotsc,n$ and it's morphisms are given by\label{enum:CDeltan}
	\begin{equation*}
		\F_{\CC[\Delta^n]}(i,j)\coloneqq\ScaledBracesCases{\!\begin{plaincases*}
			\emptyset & if $i>j$\\
			\Delta^0 & if $i=j$\\
			\square^{j-i-1} & if $i<j$
		\end{plaincases*}}\,.
	\end{equation*}
	Here $\square^n\coloneqq (\Delta^1)^n$ is the \emph{$n$-cube}. Note the shift by $-1$ in the definition! In particular, $\F_{\CC[\Delta^n]}(i,i)=\Delta^0$ (and that $0$-simplex is necessarily $\id_i$), but also $\F_{\CC[\Delta^n]}(i,i+1)=\square^0\cong \Delta^0$. The composition map $\circ\colon\F_{\CC[\Delta^n]}(i,j)\times \F_{\CC[\Delta^n]}(j,k)\rightarrow \F_{\CC[\Delta^n]}(i,k)$ is given by
	\begin{equation*}
		\square^{j-i-1}\times\square^{k-j-1}\overset{\cong}{\longrightarrow}\square^{j-i-1}\times\{1\}\times\square^{k-j-1}\subseteq \square^{k-i-1}
	\end{equation*}
	if $i<j<k$; in the other cases, there's only one possible composition map. The simplicially enriched categories $\CC[\Delta^n]$ can be assembled into a functor $\CC[-]\colon\IDelta\rightarrow\cat{Cat}_\Delta$. A conceptual construction of this functor is given in \cite[Definition~\HTTthm{1.1.5.3}]{HTT} or \cite[Lemma~1.2.62]{Land}. Since it's quite annoying to unravel said conceptual construction, let us describe the simplicially enriched functors $\CC[d_i]\colon \CC[\Delta^{n-1}]\rightarrow\CC[\Delta^n]$ and $\CC[s_j]\colon \CC[\Delta^n]\rightarrow\CC[\Delta^{n-1}]$ explicitly: On objects, $\CC[d_i]$ and $\CC[s_j]$ are just given by $d_i$ and $s_j$, repectively. For the effect on morphisms, let's first describe $\CC[d_i]\colon \F_{\CC[\Delta^{n-1}]}(k,\ell)\rightarrow \F_{\CC[\Delta^n]}(d_i(k),d_i(\ell))$ in the case $k<i\leqslant\ell$ (in all other cases, we simply get the identity). Then $d_i(k)=k$ and $d_i(\ell)=\ell+1$ and the desired morphism is\label{enum:CDeltanFunctorial}
	\begin{equation*}
		\square^{\ell-k-1}\overset{\cong}{\longrightarrow}\square^{i-k-1}\times\{0\}\times\square^{(\ell+1)-i-1}\subseteq \square^{(\ell+1)-k-1}\,.
	\end{equation*}
	Similarly, $\CC[s_j]\colon \F_{\CC[\Delta^n]}(k,\ell)\rightarrow\F_{\CC[\Delta^{n-1}]}(s_j(k),s_j(\ell))$ is only interesting for $k\leqslant j<\ell$. If $k=j$ or $j+1=\ell$, then the desired morphism $\square^{\ell-k-1}\cong (\Delta^1)^{\ell-k-1}\rightarrow (\Delta^1)^{\ell-k-2}\cong \square^{\ell-k-2}$ is given by forgetting the first or the last factor, respectively. If $k<j$ and $j+1<\ell$, then the desired morphism is
	\begin{equation*}
		\square^{\ell-k-1}\cong \square^{j-k-1}\times\square^2\times\square^{\ell-(j+1)-1}\longrightarrow\square^{j-k-1}\times\Delta^1\times\square^{\ell-(j+1)-1}\cong \square^{\ell-k-2}\,,
	\end{equation*}
	induced by the map $\square^2\rightarrow\Delta^1$ that sends $(0,0)\in\square^2$ to $0\in\Delta^1$ and the other three $0$-simplices of $\square^2$ to $1\in\Delta^1$.
	
	It can be shown that the category $\cat{Cat}_\Delta$ has all colimits (see \cite[Corollary~1.2.45]{Land}). Consequently, by \cref{thm:1PShFreeCocompletion}, the functor above admits a unique colimit-preserving extension $\CC[-]\colon\cat{sSet}\rightarrow \cat{Cat}_\Delta$, which in turn has a right-adjoint $\N^\Delta\colon \cat{Cat}_\Delta\rightarrow\cat{sSet}$, called the \emph{simplicial nerve} or \emph{coherent nerve}. By the formula from \cref{lem:1LanAlongYonedaHasRightAdjoint}, the simplicial nerve is given by
	\begin{equation*}
		\N^\Delta(\Cc)_n\cong \Hom_{\cat{Cat}_\Delta}\bigl(\CC[\Delta^n],\Cc\bigr)\,.
	\end{equation*}
\end{con}
\begin{lem}[Cordier--Porter, {\cite[Theorem~2.1]{CordierPorter}}]\label{lem:SimplicialNerveYieldsQuasiCategories}
	Let $\Cc$ be a small simplicially enriched category. If $\Cc$ is even Kan-enriched, that is, if $\F_\Cc(x,y)$ is a Kan complex for all $x,y\in\Cc$, then $\N^\Delta(\Cc)$ is a quasi-category.
\end{lem}
\begin{proof}[Proof sketch]
	By the adjunction $\CC[-]\colon\cat{sSet}\shortdoublelrmorphism\cat{Cat}_\Delta\noloc\N^\Delta$ from \cref{con:SimplicialNerve}, an inner horn filling problem for $\N^\Delta(\Cc)$ as in \cref{def:QuasiCategory} is equivalent to an extension problem
	\begin{equation*}
		\begin{tikzcd}
			\CC[\Lambda_i^n]\rar["f"]\dar & \Cc\\
			\CC[\Delta^n]\urar[dashed]
		\end{tikzcd}
	\end{equation*}
	of simplicially enriched categories. The functor $\CC[\Lambda_i^n]\rightarrow\CC[\Delta^n]$ is a bijection on objects and an isomorphism on all but one simplicial sets of morphisms. The only difference between these two simplicially enriched categories is that $\F_{\CC[\Lambda_i^n]}(0,n)\rightarrow\F_{\CC[\Delta^n]}(0,n)\cong \square^{n-1}$ is not an isomorphism. Instead, $\F_{\CC[\Lambda_i^n]}(0,n)$ is given by deleting the interior and the bottom $i$-face of the $(n-1)$-cube $\square^{n-1}$. More precisely, if $\partial \square^{n-1}\coloneqq \bigcup_{j=1}^{n-1}(\square^{j-1}\times(\{0\}\sqcup\{1\})\times\square^{n-j-1})$ denotes the \emph{boundary of the $(n-1)$-cube}, then $\F_{\CC[\Lambda_i^n]}(0,n)\rightarrow\F_{\CC[\Delta^n]}(0,n)$ can be identified with the inclusion of simplicial sets
	\begin{equation*}
		\partial \square^{n-1}\smallsetminus \bigl(\square^{i-1}\times\{0\}\times\square^{n-i-1}\bigr)\subseteq \square^{n-1}\,.
	\end{equation*}
	To make this precise, one has to show that the description of $\CC[\Lambda_i^n]$ given above satisfies the universal property of $\colimit_{(m,\Delta^m\rightarrow\Lambda_i^n)}\CC[\Delta^m]$ in $\cat{Cat}_\Delta$. This is not hard, but technical. A full argument is in \cite[Lemma~1.2.69]{Land}.
	
	So to solve the extension problem of simplicially enriched categories above, it's enough to solve the extension problem
	\begin{equation*}
		\begin{tikzcd}
			\partial \square^{n-1}\smallsetminus \bigl(\square^{i-1}\times\{0\}\times\square^{n-i-1}\bigr)\rar["f"]\dar & \F_\Cc\bigl(f(0),f(n)\bigr)\\
			\square^{n-1}\urar[dashed]
		\end{tikzcd}
	\end{equation*}
	of simplicial sets. This can be done by successive horn filling (or by applying the upcoming \cref{lem:AnodynePushout}, which is also proved by successive horn filling), using the fact that $\F_\Cc(f(0),f(n))$ is a Kan complex, as $\Cc$ is supposed to be Kan-enriched. A complete argument is in \cite[Lemma~1.2.70]{Land}.
\end{proof}
\begin{exm}\label{exm:SimplicialNerve}
	The category of simplicial sets can be turned into a simplicially enriched category $\cat{sSet}^\Delta$ by putting $\F_{\cat{sSet}^\Delta}(X,Y)\coloneqq \F(X,Y)$. This can be used to construct some interesting quasi-categories as follows:
	\begin{alphanumerate}
		\item Restricting to the full subcategory $\cat{Kan}\subseteq \cat{sSet}$ yields a simplicial enrichment $\cat{Kan}^\Delta$. Note that $\cat{Kan}^\Delta$ is actually a Kan-enriched category, since $\F(X,Y)$ is a Kan complex whenever $Y$ is a Kan complex, as we'll see in \cref{cor:FIsKanComplex}. Up to set-theoretic difficulties that we'll not address here, \cref{lem:SimplicialNerveYieldsQuasiCategories} shows that\label{enum:An}
		\begin{equation*}
			\cat{An}\coloneqq \N^\Delta(\cat{Kan}^\Delta)
		\end{equation*}
		is a quasi-category; we call it the \emph{quasi-category of animae}.
		\item For quasi-categories $\Cc$ and $\Dd$, the simplicial set $\F(\Cc,\Dd)$ is a quasi-category; once again, this will be shown in \cref{cor:FIsKanComplex}. Then $\core \F(\Cc,\Dd)$ from \cref{def:Anima} is an anima, hence a Kan complex by \cref{thm:AnimaeAreKanComplexes}. So we can turn the category of quasi-categories $\cat{QCat}$ into a Kan-enriched category $\cat{QCat}^\Delta$ by putting $\F_{\cat{QCat}^\Delta}(\Cc,\Dd)\coloneqq \core\F(\Cc,\Dd)$. By \cref{lem:SimplicialNerveYieldsQuasiCategories} (and up to set-theoretic difficulties),\label{enum:CatInfty}
		\begin{equation*}
			\cat{Cat}_\infty\coloneqq \N^\Delta(\cat{QCat}^\Delta)
		\end{equation*}
		is a quasi-category; we call it the \emph{quasi-category of \embrace{small} quasi-categories}.
	\end{alphanumerate}
	Let's unravel how the notions from \cref{par:Composition}, \cref{par:HomotopyCategory}, and \cref{par:Equivalence} look like in the cases of $\cat{An}$ and $\cat{Cat}_\infty$. A $1$-simplex $\alpha\colon \Delta^1\rightarrow\cat{Cat}_\infty$ is equivalently a simplicially enriched functor $\ov\alpha\colon\CC[\Delta^1]\rightarrow\cat{QCat}^\Delta$. Let $\Cc\coloneqq \ov\alpha(0)$ and $\Dd\coloneqq \ov\alpha(1)$. As we've seen in \cref{con:SimplicialNerve},  $\F_{\CC[\Delta^1]}(0,1)\cong \Delta^0$. Hence $\alpha$ is given by a morphism $\Delta^0\rightarrow\core \F(\Cc,\Dd)$ of simplicial sets. In other words, a morphism in $\cat{Cat}_\infty$ is given by a functor $\Cc\rightarrow\Dd$ of quasi-categories, as we would expect.
	
	Next, let's consider a $2$-simplex $\sigma\colon\Delta^2\rightarrow\cat{Cat}_\infty$, or equivalently, a simplicially enriched functor $\ov\sigma\colon \CC[\Delta^2]\rightarrow\cat{QCat}^\Delta$. Let $\Cc\coloneqq \ov\sigma(0)$, $\Dd\coloneqq \ov\sigma(1)$, and $\Ee\coloneqq \ov\sigma(2)$. Furthermore, let $F\colon \Cc\rightarrow\Dd$ and $G\colon \Dd\rightarrow \Ee$ be the functors of quasi-categories corresponding to the $1$-simplices $\sigma|_{\Delta^{\{0,1\}}}=d_2^*(\sigma)$ and $\sigma|_{\Delta^{\{1,2\}}}=d_0^*(\sigma)$. Now $\F_{\CC[\Delta^2]}(0,2)\cong\Delta^1$ by \cref{con:SimplicialNerve}, so $\ov\sigma$ induces a map $\Delta^1\rightarrow\core\F(\Cc,\Ee)$. By definition of the composition in $\CC[\Delta^2]$, we find that $\{0\}\rightarrow \Delta^1\rightarrow \core\F(\Cc,\Ee)$ is $G\circ F\colon \Cc\rightarrow \Ee$, whereas $\{1\}\rightarrow \Delta^1\rightarrow \core\F(\Cc,\Ee)$ is another functor $H\colon \Cc\rightarrow\Ee$. The morphism $\Delta^1\rightarrow\core\F(\Cc,\Ee)$ is an equivalence $G\circ F\simeq H$ in $\F(\Cc,\Ee)$.
	
	Therefore, if $F\colon \Cc\rightarrow\Dd$ and $G\colon \Dd\rightarrow\Ee$ are functors of quasi-categories, hence morphisms in $\cat{Cat}_\infty$, then a composition of $F$ and $G$ in the quasi-category $\cat{Cat}_\infty$, as defined in \cref{par:Composition}, is a functor $H\colon \Cc\rightarrow\Ee$ of quasi-categories together with an equivalence $G\circ F\simeq H$ in $\F(\Cc,\Ee)$. The same analysis can be done for $\cat{An}$. So if $f\colon X\rightarrow Y$ and $g\colon Y\rightarrow Z$ are maps of Kan complexes, corresponding to morphisms in the quasi-category $\cat{An}$, then a composition of $f$ and $g$ in the quasi-category $\cat{An}$ is a morphism $h\colon X\rightarrow Z$ together with a $1$-simplex $\Delta^1\rightarrow \F(X,Z)$ from $g\circ f$ to $h$. By \cref{par:FInternalHom}, such a $1$-simplex $\Delta^1\rightarrow \F(X,Z)$ is equivalently a map $\eta\colon \Delta^1\times X\rightarrow Z$ such that
	\begin{equation*}
		\begin{tikzcd}
			\{0\}\times X\dar\dar[phantom,""{name=A}]\arrow[from=2-2,to=1-1,commutes,xshift=-1ex]\drar[bend left,"g\circ f"]& \\
			\Delta^1\times X\rar["\eta"]& Z\\
			\{1\}\times X\uar\uar[phantom,""{name=A}]\arrow[from=2-2,to=3-1,commutes,xshift=-1ex]\urar[bend right, "h"'] & 
		\end{tikzcd}
	\end{equation*}
	commutes. In other words, $\eta$ is a \emph{homotopy from $g\circ f$ to $h$}. In summary, we obtain the following slogans:
	\begin{alphanumerate}[label={}]\itshape
		\item\enquote{Compositions in $\cat{Cat}_\infty$ are compositions in $\cat{sSet}$ up to equivalence of functors.}
		
		\item\enquote{Compositions in $\cat{An}$ are compositions in $\cat{sSet}$ up to homotopy.}
	\end{alphanumerate}
	Furthermore, this analysis shows that two functors of quasi-categories $F,G\colon \Cc\rightarrow \Dd$ are equivalent as morphisms in $\cat{Cat}_\infty$ in the sense of \cref{par:Composition} if and only if they are equivalent as objects in $\F(\Cc,\Dd)$. Similarly, two morphisms of animae $f,g\colon X\rightarrow Y$ are equivalent as morphisms in $\cat{An}$ in the sense of \cref{par:Composition} if and only they are homotopic. This somewhat explains the term \emph{homotopy category}.
	
	Finally, we see that an equivalence $\Cc\simeq \Dd$ in the quasi-category $\cat{Cat}_\infty$, as defined in \cref{par:Equivalence}, is given by functors of quasi-categories $F\colon \Cc\rightarrow \Dd$ and $G\colon \Dd\rightarrow \Cc$ together with equivalences $G\circ F\simeq \id_\Cc$ and $F\circ G\simeq \id_\Dd$, exactly as an equivalence of ordinary categories. Analogously, an equivalence $X\simeq Y$ in $\cat{An}$ is given by maps of Kan complexes $f\colon X\rightarrow Y$ and $g\colon Y\rightarrow X$, together with homotopies $g\circ f\simeq\id_X$ and $f\circ g\simeq \id_Y$. In other words, equivalences in $\cat{An}$ are simply \emph{homotopy equivalences}. We'll explore this in much more detail in \cref{sec:SimplicialHomotopyTheory}.
\end{exm}
If $X$, $Y$ are Kan complexes, then $\F(X,Y)$ is a Kan complex too, as we'll see in \cref{cor:FIsKanComplex}. Hence $\F(X,Y)=\core\F(X,Y)$ and therefore the Kan-enriched category $\cat{Kan}^\Delta$ is a full sub-simplicially enriched category of $\cat{QCat}^\Delta$. Using the explicit formula for the simplicial nerve from \cref{con:SimplicialNerve}, it's straightforward to see that $\N^\Delta(-)$ sends full sub-simplicially enriched categories to full sub-quasi-categories in the sense of \cref{par:SubQuasiCategories}. Thus $\cat{An}\subseteq \cat{Cat}_\infty$ is a full sub-quasi-category of $\cat{Cat}_\infty$. In particular, if $X$ and $Y$ are Kan complexes, then
\begin{equation*}
	\Hom_{\cat{An}}(X,Y)\overset{\cong}{\longrightarrow}\Hom_{\cat{Cat}_\infty}(X,Y)
\end{equation*}
is an isomorphism of simplicial sets. In general, \cref{thm:CordierPorter} below describes the $\Hom$ anima from \cref{par:HomInQuasiCategories} in a simplicial nerve. A relatively short proof of that theorem was given by Achim Krause and Fabian in \cite{AchimFabian}.
\begin{thm}\label{thm:CordierPorter}
	Let $\Cc$ be a Kan-enriched category. Then there is a homotopy equivalence of Kan complexes
	\begin{equation*}
		\Hom_{\N^\Delta(\Cc)}(x,y)\simeq \F_\Cc(x,y)\,.
	\end{equation*}
	In particular, $\Hom_{\cat{An}}(X,Y)\simeq \F(X,Y)$ for all $X,Y\in\cat{An}$ and $\Hom_{\cat{Cat}_\infty}(\Cc,\Dd)\simeq \core\F(\Cc,\Dd)$ for all $\Cc,\Dd\in\cat{Cat}_\infty$.\hfill$\blacksquare$
\end{thm}
\begin{exm}\label{exm:CatAs2Category}
	We can also turn the category of ordinary categories $\cat{Cat}$ into a Kan enriched category\footnote{Don't confuse $\cat{Cat}^\Delta$, the simplicially enriched category of categories, with $\cat{Cat}_\Delta$, the category of simplicially enriched categories.} $\cat{Cat}^\Delta$ via $\F_{\cat{Cat}^\Delta}(\Cc,\Dd)\coloneqq \core\N(\Fun(\Cc,\Dd))$. We let
	\begin{equation*}
		\cat{Cat}^{(2)}\coloneqq \N^\Delta(\cat{Cat}^\Delta)
	\end{equation*}
	denote its simplicial nerve. According to \cref{thm:CordierPorter}, $\Hom_{\cat{Cat}^{(2)}}(\Cc,\Dd)\simeq \core\N(\Fun(\Cc,\Dd))$. In particular, we see that $\cat{Cat}^{(2)}$ is different from $\N(\cat{Cat})$, the nerve of the ordinary category of categories. Indeed, we've seen in \cref{par:HomInQuasiCategories} that $\Hom_{\N(\cat{Cat})}(\Cc,\Dd)$ would be a discrete: a disjoint union of copies of $\Delta^0$, where the indexing set is precisely the set of functors from $\Cc$ to $\Dd$. In contrast to that, $\core\N(\Fun(\Cc,\Dd))\simeq \N(\core \Fun(\Cc,\Dd))$, where $\core \Fun(\Cc,\Dd)\subseteq \Fun(\Cc,\Dd)$ denotes the maximal groupoid contained in $\Fun(\Cc,\Dd)$. So $\Hom_{\cat{Cat}^{(2)}}(\Cc,\Dd)$ is the nerve of a groupoid and usually not a discrete simplicial set.\footnote{One says that $\cat{Cat}^{(2)}$ is the \emph{$2$-category of categories}, and we've just seen why: $\cat{Cat}^{(2)}$ not only knows about categories and functors, but through $\Hom_{\cat{Cat}^{(2)}}(\Cc,\Dd)\simeq \N(\core\Fun(\Cc,\Dd))$ it also contains information about natural equivalences between functors. In general, a quasi-category $\Ee$ is said to be an \emph{$n$-category} if for all objects $x,y\in \Ee$ and all morphisms $f\in\Hom_\Ee(x,y)$ one has $\pi_i(\Hom_\Ee(x,y),f)\cong 0$ whenever $i\geqslant n$. Here $\pi_i$ refers to the homotopy groups introduced in \cref{con:HomotopyGroups} and we've used implicitly that $\Hom_\Ee(x,y)$ is a Kan complex, as will be shown in \cref{cor:AnimaKanComplexes,cor:HomAnima}. It is not hard to check that $\cat{Cat}^{(2)}$ is indeed a $2$-category. Indeed, we've seen that $\Hom_{\cat{Cat}^{(2)}}(\Cc,\Dd)\simeq \N(\core\Fun(\Cc,\Dd))$. By the observation in the proof of \cref{lem:FullyFaithfulAnimae}, we get
	\begin{equation*}
		\pi_i\bigl(\N(\core\Fun(\Cc,\Dd)),F\bigr)\cong \pi_{i-1}\bigl(\Hom_{\N(\core\Fun(\Cc,\Dd))}(F,F),\id_F\bigr)
	\end{equation*}
	for all $F\in \N(\core\Fun(\Cc,\Dd))$. But now $\Hom_{\N(\core\Fun(\Cc,\Dd))}(F,F)$ is a discrete simplicial set, because it is the $\Hom$ anima in the nerve of an ordinary category. So the right-hand side vanishes for $i-1\geqslant 1$, as desired.
	
	You might have expected the $2$-category of categories to encompass all natural transformations, not only the natural equivalences. The reason for this confusion is an unfortunate oversimplification of language on our part: What we call \emph{$\infty$-categories} (or \emph{$n$-categories}) should more accurately be called $(\infty,1)$-categories (or \emph{$(n,1)$-categories}). The first entry of the pair \enquote{$(\infty,1)$} signifies that such an object contains \enquote{$d$-morphisms} for every dimension $0\leqslant d<\infty$, whereas the second entry refers to the fact that all $d$-morphisms for $d>1$ are invertible. This is evidenced by the fact that $\Hom_\Ee(x,y)$ is a Kan complex for any quasi-category $\Ee$ and all $x,y\in \Ee$. Thanks to the effort of many mathematicians, we now have well-studied notions of \emph{$(\infty,k)$-categories} (with $(\infty,0)$-categories corresponding to animae and $(\infty,1)$-categories corresponding to what we call \emph{$\infty$-categories} in these notes), in which only $d$-morphisms for $d>k$ need to be invertible. These have become important tools in modern mathematics---for example, it's sometimes necessary to use the fact that $\cat{Cat}_\infty$ can be enhanced to an $(\infty,2)$-category---but this goes beyond the scope of these notes.}
	
	Thanks to \cref{lem:SimplicialHoNerveAdjunction}, the nerve functor $\N\colon \cat{Cat}\rightarrow\cat{QCat}$ defines a fully faithful functor of simplicially enriched categories $\N\colon \cat{Cat}^\Delta\rightarrow \cat{QCat}^\Delta$. Accordingly, we can regard $\cat{Cat}^{(2)}$ as the full sub-quasi-category of $\cat{Cat}_\infty$ spanned by those quasi-categories that are nerves of ordinary categories.
	
	In a similar way, one can define equip the category of groupoids $\cat{Grpd}$ with a Kan enrichment $\cat{Grpd}^\Delta$ (simply given by restriction from $\cat{Cat}^\Delta$) and we let
	\begin{equation*}
		\cat{Grpd}^{(2)}\coloneqq \N^\Delta(\cat{Grpd}^\Delta)
	\end{equation*}
	denotes its simplicial nerve. As above, $\cat{Grpd}^{(2)}$ is the full sub-quasi-category of $\cat{Cat}_\infty$ spanned by the nerves of groupoids. Since every nerve of a groupoid is a Kan complex (which follows from \cref{cor:AnimaKanComplexes}, but can also be checked by hand), we see that $\cat{Grpd}^{(2)}$ is also a full sub-quasi-category of $\cat{An}$.
\end{exm}

\newpage
\section{Simplicial homotopy theory}\label{sec:SimplicialHomotopyTheory}
The goal of this section is to describe how to do homotopy theory with simplicial sets instead of topological spaces. This doesn't quite work on the nose, since simplicial sets are much more rigid than topological spaces. For example, consider the naive definition of homotopies: Two maps $f,g\colon X\rightarrow Y$ are said to be \emph{homotopic}, $f\simeq g$, if there exists a map $\eta\colon \Delta^1\times X\rightarrow Y$ such that the diagram
\begin{equation*}
	\begin{tikzcd}
		\{0\}\times X\dar\dar[phantom,""{name=A}]\arrow[from=2-2,to=1-1,commutes,xshift=-1ex]\drar[bend left,"f"]& \\
		\Delta^1\times X\rar["\eta"]& Z\\
		\{1\}\times X\uar\uar[phantom,""{name=A}]\arrow[from=2-2,to=3-1,commutes,xshift=-1ex]\urar[bend right, "g"'] & 
	\end{tikzcd}
\end{equation*}
commutes. This relation is \emph{not} an equivalence relation! For example, if $d_1\colon\Delta^0\simeq\{0\}\rightarrow\Delta^1$ and $d_0\colon \Delta^0\simeq\{1\}\rightarrow \Delta^1$ are the two maps from the $0$-simplex to the $1$-simplex, then $d_1\simeq d_0$, but $d_0\not\simeq d_1$. So the relation is not symmetric (nor transitive). However, as we will see, everything works fine as long as we work with Kan complexes!

So the upshot of this section will be that instead of replacing topological spaces by arbitrary simplicial sets as a habitat for homotopy theory, we should replace them with Kan complexes. In view of \cref{thm:AnimaeAreKanComplexes}, this fits perfectly with Grothendieck's \emph{homotopy hypothesis} that $\infty$-groupoids should essentially be topological spaces.

\subsection{Fibrations and lifting properties}\label{subsec:Fibrations}
We start with several definitions that generalise the horn filling properties from \cref{def:QuasiCategory}.
\begin{defi}\label{def:Lifting}
	We say that a map $f\colon X\rightarrow Y$ of simplicial sets \emph{has lifting against $i\colon A\rightarrow B$} if every lifting problem
	\begin{equation*}
		\begin{tikzcd}
			A\rar\dar["i"']& X\dar["f"]\\
			B\rar\urar[dashed] & Y
		\end{tikzcd}
	\end{equation*}
	has a solution.
\end{defi}
\begin{defi}\label{def:Fibration}
	Let $f\colon X\rightarrow Y$ be a map of simplicial sets.
	\begin{alphanumerate}
		\item We call $f$ a \emph{Kan fibration} if it has lifting agains all horn inclusions $\Lambda_i^n\rightarrow \Delta^n$ for $n\geqslant 1$ and $0\leqslant i\leqslant n$. We call $f$ a \emph{left}, \emph{right}, or \emph{inner fibration}, if it has lifting against all horn inclusions for $0\leqslant i<n$, all $0<i\leqslant n$, or $0<i<n$, respectively.\label{enum:KanFibration}
		\item We call $f$ a \emph{trivial fibration} if it has lifting against all boundary inclusions $\partial\Delta^n\rightarrow\Delta^n$ for all $n\geqslant 0$.\label{enum:TrivialFibration}
	\end{alphanumerate}
\end{defi}
\begin{exm}\label{exm:KanFibration}
	A simplicial set $X$ is a Kan complex if and only if $X\rightarrow *$ is a Kan fibration, and a quasi-category if and only if $X\rightarrow *$ is an inner fibration. Here and in the following we put $*\coloneqq\Delta^0$ for convenience. Furthermore, if $f\colon X\rightarrow Y$ is a Kan fibration and $Y$ is a Kan complex, then $X$ is a Kan complex too. Similarly, if $f$ is an inner fibration and $Y$ is a quasi-category, then $X$ is a quasi-category too.
\end{exm}
To analyse lifting properties, we need to introduce yet another technical notion.
\begin{defi}\label{def:Saturated}
	A class $\Sigma$ of morphisms of simplicial sets is called \emph{saturated} if the following conditions are satisfied:
	\begin{alphanumerate}
		\item $\Sigma$ is \emph{closed under pushouts}: If $(A\rightarrow B)\in\Sigma$ and $A\rightarrow C$ is an arbitrary map of simplicial sets, then $(C\rightarrow B\sqcup_AC)\in\Sigma$.
		\item $\Sigma$ is \emph{closed under retracts}: If we're given a commutative diagram
		\begin{equation*}
			\begin{tikzcd}
				A'\rar\dar["i'"']\drar[commutes]\ar[rr,bend left=42,"\id_{A'}"{name=A}]\arrow[from=1-2,to=A,commutes,pos=0.46] & A\dar["i"] \rar\drar[commutes] &A'\dar["i'"]\\
				B'\rar\ar[rr,bend right=42,"\id_{B'}"{swap,name=B}]\arrow[from=2-2,to=B,commutes,pos=0.46] & B\rar & B'
			\end{tikzcd}
		\end{equation*}
		such that $(i\colon A\rightarrow B)\in\Sigma$, then also $(i'\colon A'\rightarrow B')\in\Sigma$.
		\item $\Sigma$ is \emph{closed under coproducts}: If $(A_i\rightarrow B_i)\in\Sigma$, then also $\left(\coprod A_i\rightarrow\coprod B_i\right)\in\Sigma$.
		\item $\Sigma$ is \emph{closed under \embrace{countable} infinite compositions}: If $A_0\rightarrow A_1\rightarrow A_2\rightarrow \dotsb$ are all in $\Sigma$, then also $(A_0\rightarrow\colimit_{n\geqslant 0}A_n)\in\Sigma$.
	\end{alphanumerate}
	For an arbitrary class $\Sigma$ of morphisms in $\cat{sSet}$, the \emph{saturation of $\Sigma$}, $\operatorname{sat}(\Sigma)$, is the smallest saturated class containing $\Sigma$.
\end{defi}
\begin{lem}\label{lem:LiftingSaturated}
	A morphism $f\colon X\rightarrow Y$ of simplicial sets has lifting against all $(A\rightarrow B)\in\Sigma$ if and only $f$ has lifting against all $(A\rightarrow B)\in\operatorname{sat}(\Sigma)$.
\end{lem}
\begin{proof}[Proof sketch]
	It's straightforward to check that the class of morphisms that $f$ has lifting against is saturated as in \cref{def:Saturated}.
\end{proof}
\begin{defi}\label{def:Anodyne}
	\begin{alphanumerate}
		\item A morphism of simplicial sets is called \emph{anodyne} if it is contained in $\operatorname{sat}\left\{\Lambda_i^n\rightarrow\Delta^n\ \middle|\ n\geqslant 1,\,0\leqslant i\leqslant n\right\}$, the saturation of all horn inclusions. Similarly, a morphism is called \emph{left}, \emph{right}, or \emph{inner anodyne} if it is contained in the saturation of those horn inclusions where $0\leqslant i<n$, $0<i\leqslant n$, or $0<i<n$, respectively.
		\item A morphism of simplicial set is a \emph{cofibration} if it is contained in $\operatorname{sat}\left\{\partial\Delta^n\rightarrow\Delta^n\ \middle|\ n\geqslant 0\right\}$, the saturation of all boundary inclusions.
	\end{alphanumerate}
\end{defi}
\begin{exm}
	Using \cref{lem:LiftingSaturated}, we see that Kan fibrations have lifting against all anodyne morphisms and left/right/inner fibrations have lifting against all left/right/inner anodyne morphisms. Furthermore, trivial fibrations have lifting against all cofibrations.
\end{exm}	
\begin{lem}\label{lem:Cofibration}
	A map $i\colon A\rightarrow B$ is simplicial sets is a cofibration if and only if $i$ is injective in every degree.
\end{lem}
\begin{proof}[Proof sketch]
	It's straightforward to check that degree-wise injectivity is closed under pushouts, retracts, coproducts, and infinite compositions, whence all cofibrations are degree-wise injective. Conversely, a degree-wise injective map can be built from boundary inclusions by successively adding simplices. This successive procedures needs pushouts (to add new simplices), coproducts (to add arbitrarily many simplices at once), and infinite compositions.
\end{proof}
\begin{lem}\label{lem:AnodynePushout}
	If $A\rightarrow B$ is anodyne and $A'\rightarrow B'$ is a cofibration, then
	\begin{equation*}
		A\times B'\sqcup_{A\times A'}B\times A'\longrightarrow B\times B'
	\end{equation*}
	is anodyne again. Analogous assertions are true for left/right/inner anodyne maps.
\end{lem}
\begin{proof}[Proof sketch]
	Fix $A'\rightarrow B'$ and consider the class $\Sigma$ of all morphisms $A\rightarrow B$ for which $A\times B'\sqcup_{A\times A'}B\times A'\rightarrow B\times B'$ is anodyne. Then $\Sigma$ is easily checked to be saturated. Hence it suffices to consider the case where $A\rightarrow B$ is a horn inclusion $\Lambda_i^n\rightarrow \Delta^n$. By the same argument, we can reduce to the case where $A'\rightarrow B'$ is a boundary inclusion $\partial \Delta^m\rightarrow \Delta^m$. So it suffices to check that $\Lambda_i^n\times \Delta^m\sqcup_{\Lambda_i^n\times \partial\Delta^m}\Delta^n\times \partial\Delta^m\rightarrow \Delta^n\times \Delta^m$ is anodyne. This can be done by hand, explicitly writing said map as a sequence of horn inclusions. For a complete proof in all its gory details, see \cite[Lemma~1.3.31]{Land}.
\end{proof}
\begin{cor}\label{cor:FKanFibration}
	If $i\colon A\rightarrow B$ is a cofibration and $f\colon X\rightarrow Y$ is a Kan fibration, then
	\begin{equation*}
		\F(B,X)\longrightarrow\F(B,Y)\times_{\F(A,Y)}\F(A,X)
	\end{equation*}
	is a Kan fibration. If $i\colon A\rightarrow B$ is anodyne, then the map above is even a trivial fibration. Analogous conclusions are true for left/right/inner fibrations and left/right/inner anodyne cofibrations.
\end{cor}
\begin{proof}[Proof sketch]
	By playing around with the universal properties of pushouts and pullbacks as well as the adjunction from \cref{par:FInternalHom}, we find that the following lifting problems are equivalent:
	\begin{equation*}
		\begin{tikzcd}
			\Lambda_i^n\rar\dar & \F(B,X)\dar\\
			\Delta^n\rar\urar[dashed] & \F(B,Y)\times_{\F(A,Y)}\F(A,X)
		\end{tikzcd}\quad\text{and}\quad
		\begin{tikzcd}
			\Lambda_i^n\times B\sqcup_{\Lambda_i^n\times A}\Delta^n\times A\rar\dar & X\dar["f"]\\
			\Delta^n\times B\urar[dashed]\rar & Y
		\end{tikzcd}
	\end{equation*}
	Since $\Lambda_i^n\times B\sqcup_{\Lambda_i^n\times A}\Delta^n\times  A\rightarrow \Delta^n\times B$ is anodyne by \cref{lem:AnodynePushout} and $f\colon X\rightarrow Y$ has lifting against all anodyne maps by \cref{lem:LiftingSaturated}, the lifting problem on the right can be solved, proving that $\F(B,X)\rightarrow \F(B,Y)\times_{\F(A,Y)}\F(A,X)$ indeed has lifting against all horn inclusions. If $A\rightarrow B$ is anodyne, then the same argument shows that we even get lifting against all boundary inclusions. The other assertions are entirely analogous.
\end{proof}
\begin{cor}\label{cor:FIsKanComplex}
	Let $X$ be a Kan complex, $\Cc$ a quasi-category, and $B$ an arbitrary simplicial set. Then $\F(B,X)$ is a Kan complex and $\F(B,\Cc)$ is a quasi-category. In particular, $\Ar(\Cc)$ is a quasi-category again, and if $x\in \Cc$ is an object, then the slice $\Cc_{x/}$ from \cref{par:HomInQuasiCategories} is a quasi-category too.
\end{cor}
\begin{proof}
	For the first two assertions, apply \cref{cor:FKanFibration} to the cofibration $\emptyset\rightarrow B$ and the Kan fibration $X\rightarrow *$ or the inner fibration $\Cc\rightarrow *$, respectively. The assertion about $\Ar(\Cc)$ is just the case $B=\Delta^1$. Finally, for $\Cc_{x/}$ we use that $(s,t)\colon \Ar(\Cc)\rightarrow \Cc\times\Cc$ is an inner fibration by \cref{cor:FKanFibration} applied to the cofibration $\partial\Delta^1\rightarrow\Delta^1$. Hence its pullback $\Cc_{x/}\rightarrow \{x\}\times \Cc$ must be an inner fibration too and so $\Cc_{x/}$ is a quasi-category by \cref{exm:KanFibration}.
\end{proof}
We conclude this subsection with an immensely useful lemma.
\begin{lem}[\enquote{Quillen's small object argument}]\label{lem:SmallObjectArgument}
	Every morphism of simplicial sets $f\colon X\rightarrow Y$ can be factored as
	\begin{equation*}
		f\colon X\overset{i}{\longrightarrow}\ov X\overset{\ov f}{\longrightarrow}Y\,,
	\end{equation*}
	where $i$ is anodyne and $\ov f$ is a Kan fibration. Similarly, every morphism of simplicial sets can be factored into a left/right/inner anodyne map followed by a left/right/inner fibration, and also into a cofibration followed by a trivial fibration.
\end{lem}
\begin{proof}
	We only prove the first assertion; the others are completely analogous. Let
	\begin{equation*}
		\Sigma(f)\coloneqq \ScaledBraces{\!\left.\sigma=\begin{tikzcd}[baseline=(A.base),ampersand replacement=\&]
			\Lambda_i^n\rar\dar\drar[commutes]\drar[phantom,""{name=A}] \& X\dar["f"]\\
			\Delta^n\rar \& Y 
		\end{tikzcd}\ \middle|\ n\geqslant 1 ,\,0\leqslant i\leqslant n\right.\!}
	\end{equation*}
	and consider the simplicial set $S(f)$ defined as the pushout
	\begin{equation*}
		\begin{tikzcd}
			\coprod_{\sigma\in\Sigma(f)}\Lambda_i^n\dar\rar\drar[pushout] & X\dar\\
			\coprod_{\sigma\in\Sigma(f)}\Delta^n\rar & S(f)
		\end{tikzcd}
	\end{equation*}
	Then $X\rightarrow S(f)$ is anodyne, because it is a pushout of a coproduct of horn inclusions, and $f$ factors as $f\colon X\rightarrow S(f)\rightarrow Y$. Let $X_0\coloneqq X$ and $f_0\coloneqq f$. Inductively putting $X_{n+1}\coloneqq S(f_n)$, we get factorisations
	\begin{equation*}
		f\colon X\longrightarrow X_n\overset{f_n}{\longrightarrow}Y
	\end{equation*}
	for all $n\geqslant 0$, where $X\rightarrow X_n$ is anodyne. Now let $\ov X\coloneqq \colimit_{n\geqslant 0}X_n$ and let $\ov f\colon \ov X\rightarrow Y$ be the induced map. Since anodyne maps are closed under infinite compositions, $X\rightarrow \ov X$ is anodyne. So it suffices to show that $\ov f$ is a Kan fibration. Note that $\Lambda_i^n$ is built from finitely many simplices, which are in turn glued along finitely many subsimplices. Hence, for every map $\sigma\colon \Lambda_i^n\rightarrow\ov X$, each of these finitely many simplices must occur at some finite stage of the colimit $\ov X\coloneqq \colimit_{n\geqslant 0}X_n$, and each gluing condition must be satisfied at some finite stage. Consequently, every $\sigma\colon \Lambda_i^n\rightarrow\ov X$ must factor through $X_m\rightarrow\ov X$ for $m\gge 0$. Consequently, by construction of $X_{m+1}$, every lifting problem involving $\sigma$ can be solved as
	\begin{equation*}
		\begin{tikzcd}
			\Lambda_i^n\rar\dar& X_m\rar & X_{m+1}\rar &\ov X\dar["\ov f"]\\
			\Delta^n\ar[rrr]\ar[urr,dashed] & & & Y
		\end{tikzcd}
	\end{equation*}
	which proves that $\ov f\colon \ov X\rightarrow Y$ is a Kan fibration, as desired.
\end{proof}

\subsection{Homotopy groups}
The goal of this subsection is to introduce homotopy groups of Kan complexes (\cref{con:HomotopyGroups}) and to prove an analogue of Whitehead's theorem (\cref{thm:Whitehead}). We start noting that the naive definition of homotopies from the beginning of \cref{sec:SimplicialHomotopyTheory} works fine if $X$ is Kan.
\begin{defi}\label{def:Homotopy}
	Let $X$ be a Kan complex.
	\begin{alphanumerate}
		\item We say that $x,y\in X$ \emph{belong to the same connected component} and write $x\simeq y$ if there is a $1$-simplex $\Delta^1\rightarrow X$ from $x$ to $y$. By \cref{thm:AnimaeAreKanComplexes}, this is an equivalence relation and the notation is compatible with \cref{par:Equivalence}. We let $\pi_0(X)\coloneqq X_0/\!\simeq$ denote the set of \emph{connected components of $X$}.\label{enum:Pi0}
		\item Let $A$ be an arbitrary simplicial set. We say that $f,g\colon A\rightarrow X$ are \emph{homotopic} and write $f\simeq g$ if and only if they belong to the same connected component of $\F(A,X)$, which is a Kan complex by \cref{cor:FIsKanComplex}. A \emph{homotopy} $\eta\colon f\Rightarrow g$ is a $1$-simplex $\Delta^1\rightarrow \F(A,X)$ from $f$ to $g$.\label{enum:Homotopy}
	\end{alphanumerate}
\end{defi}
\begin{con}\label{con:FOfPairs}
	Let $A\subseteq B$ be an inclusion of arbitrary simplicial sets and $X\subseteq Y$ be an inclusion of Kan complexes. Consider the following pullback (taken in $\cat{sSet}$):
	\begin{equation*}
		\begin{tikzcd}
			\F\bigl((B,A),(Y,X)\bigr)\dar\rar\drar[pullback] & \F(B,Y)\dar\\
			\F(A,X)\rar & \F(A,Y)
		\end{tikzcd}
	\end{equation*}
	Note that $\F(B,Y)\rightarrow\F(A,Y)$ is a Kan fibration by \cref{cor:FKanFibration} and $\F(A,X)$ is a Kan complex by \cref{cor:FIsKanComplex}. Therefore $\F((B,A),(Y,X))$ is a Kan complex too.
\end{con}
\begin{con}\label{con:HomotopyGroups}
	Let $X$ be a Kan complex, $x\in X$ a point, and $n\geqslant 1$. Furthermore, recall from \cref{con:SimplicialNerve} that we use $\square^n$ and $\partial\square^n$ to denote the $n$-cube $(\Delta^1)^n$ and its boundary $\bigcup_{i=1}^n\square^{i-1}\times(\{0\}\sqcup\{1\})\times\square^{n-i}$. We define the \emph{$n$\textsuperscript{th} homotopy group of $X$ with basepoint $x$} as
	\begin{equation*}
		\pi_n(X,x)\coloneqq\pi_0\F\bigl((\square^n,\partial\square^n),(X,x)\bigr)\,.
	\end{equation*}
	As the name suggests, $\pi_n(X,x)$ should be a group, so let's construct a group operation! Given elements $[\alpha],[\beta]\in\pi_n(X,x)$, represented by maps of pairs $\alpha,\beta\colon (\square^n,\partial\square^n)\rightarrow (X,x)$, we can define a map $(\alpha,\beta)\colon \Lambda_1^2\times\square^{n-1}\rightarrow X$ by $(\alpha,\beta)|_{\Delta^{\{0,1\}}\times\square^{n-1}}\coloneqq \alpha$ and $(\alpha,\beta)|_{\Delta^{\{1,2\}}\times\square^{n-1}}\coloneqq \beta$; this is possible since $\alpha$ and $\beta$ agree on the \enquote{overlap} $\{1\}\times\square^{n-1}$, as they're both equal to $\const x$ there. Now consider the extension problem
	\begin{equation*}
		\begin{tikzcd}[column sep=5.2em]
			\Lambda_1^2\times\square^{n-1}\sqcup_{\Lambda_1^2\times\partial\square^{n-1}}\Delta^2\times\partial\square^{n-1}\dar\rar["{(\alpha,\,\beta)\,\cup\, \const x}"] & X\\
			\Delta^2\times\square^{n-1}\urar[dashed,"\vartheta"',end anchor=202]
		\end{tikzcd}
	\end{equation*}
	Since the vertical arrow is anodyne by \cref{lem:AnodynePushout}, this extension problem has a solution $\vartheta$. By construction, $\vartheta|_{\partial(\Delta^{\{0,2\}}\times\square^{n-1})}=\const x$. We then define $[\alpha]\mathbin{\boldsymbol{\cdot}}[\beta]\coloneqq [\vartheta|_{\Delta^{\{0,2\}}\times\square^{n-1}}]$.
\end{con}
\begin{rem}\label{rem:HomotopyGroupsTopology}
	Let us explain how \cref{con:HomotopyGroups} is related to the usual construction of the group structure on $\pi_n(X,x)$ from topology, as this nicely illustrates the \enquote{rigidity} of simplicial sets and how said rigidity is overcome by the Kan condition. First, observe that $\Lambda_1^2\times\square^{n-1}\cong \Delta^{\{0,1\}}\times\square^{n-1}\sqcup_{\{1\}\times \square^{n-1}}\Delta^{\{1,2\}}\times\square^{n-1}$ is simply given by \enquote{stacking one cube on top of another}. In topological spaces, we can identify two stacked cubes with another cube, which immediately yields the group operation. In simplicial sets, this identification no longer works; in fact, there isn't even a suitable map $\Delta^1\times\square^{n-1}\rightarrow \Lambda_1^2\times \square^{n-1}$. But instead we can use the zigzag $\Delta^{\{0,2\}}\times\square^{n-1}\rightarrow \Delta^2\times\square^{n-1}\leftarrow \Lambda_1^2\times\square^{n-1}$, thanks to the Kan condition.
\end{rem}
\begin{lem}\label{lem:HomotopyGroups}
	Let $X$ be a Kan complex, let $x\in X$ be a point, and let $n\geqslant 1$.
	\begin{alphanumerate}
		\item The operation $\cdot$ from \cref{con:HomotopyGroups} is well-defined \embrace{that is, independent of the choices of $\alpha$, $\beta$, and $\vartheta$} and defines a group structure on $\pi_n(X,x)$.\label{enum:HomotopyGroupsWellDefined}
		\item Let $n\geqslant 2$. By \enquote{permuting the coordinates of the cube $\square^n$} we obtain operations $\boldsymbol{\cdot}_1,\boldsymbol{\cdot}_2,\dotsc,\boldsymbol{\cdot}_n$, where $\boldsymbol{\cdot}_1=\boldsymbol{\cdot}$. Then these operations all coincide and are commutative.\label{enum:EckmannHilton}
	\end{alphanumerate}
\end{lem}
\begin{proof}[Proof sketch]
	All assertions in \cref{enum:HomotopyGroupsWellDefined} can be proved by solving extension problems of the form
	\begin{equation}\label{eq:HomotopyGroups}\tag{$*$}
		\begin{tikzcd}
			A\times\square^{n-1}\sqcup_{A\times\partial\square^{n-1}}B\times\partial\square^{n-1}\dar\rar & X\\
			B\times\square^{n-1}\urar[dashed,end anchor=202]
		\end{tikzcd}
	\end{equation}
	where $A\rightarrow B$ is anodyne (so that a solution always exists by \cref{lem:AnodynePushout}).\footnote{Also note that the diagram from \cref{con:HomotopyGroups} is of this form too, with $(A\rightarrow B)=(\Lambda_1^2\rightarrow\Delta^2)$}
	
	Let's start with independence of the choice of $\vartheta$. So let $\vartheta'$ be another choice. Using the same idea as in \cref{par:Composition}, we can pose an extension problem \cref{eq:HomotopyGroups}, with $(A\rightarrow B)=(\Lambda_1^3\rightarrow \Delta^3)$. Restricting any solution to $\Delta^{\{0,2,3\}}\times \square^{n-1}$ yields a homotopy of pairs $\vartheta|_{\Delta^{\{0,2\}}\times\square^{n-1}}\simeq\vartheta'|_{\Delta^{\{0,2\}}\times\square^{n-1}}$. To show that the choices of $\alpha$ and $\beta$ don't matter, suppose we're given homotopies of pairs $\alpha\simeq\alpha'$ and $\beta\simeq\beta'$ and let $[\alpha']\mathbin{\boldsymbol{\cdot}}[\beta']=[\vartheta'|_{\Delta^{\{0,2\}}\times\square^{n-1}}]$. Using the given homotopies, we can write down an extension problem \cref{eq:HomotopyGroups}, with $(A\rightarrow B)=(\Delta^1\times\Lambda_1^2\rightarrow \Delta^1\times\Delta^2)$. Restricting to $\Delta^1\times\Delta^{\{0,2\}}\times\square^{n-1}$ yields a homotopy  of pairs $\vartheta|_{\Delta^{\{0,2\}}\times\square^{n-1}}\simeq\vartheta'|_{\Delta^{\{0,2\}}\times\square^{n-1}}$. This shows well-definedness. To show associativity, choose $(A\rightarrow B)=(\Delta^{\{0,1\}}\cup\Delta^{\{1,2\}}\cup\Delta^{\{2,3\}}\rightarrow\Delta^3)$. A neutral element is $\const x\colon (\square^n,\partial\square^n)\rightarrow (X,x)$; to show $[\alpha]\mathbin{\boldsymbol{\cdot}}[\const x]=[\alpha]=[\const x]\mathbin{\boldsymbol{\cdot}}[\alpha]$ for all $\alpha$, simply solve the corresponding lifting problem via degenerate simplices. Finally, to construct inverses, we take inspiration from \cref{par:Equivalence} and write down extension problems \cref{eq:HomotopyGroups} with $(A\rightarrow B)=(\Lambda_0^2\rightarrow\Delta^2)$ and $(A\rightarrow B)=(\Lambda_2^2\rightarrow\Delta^2)$ to construct a left and a right inverse. This finishes the proof sketch of \cref{enum:HomotopyGroupsWellDefined}.
	
	For \cref{enum:EckmannHilton}, we use the \emph{Eckmann--Hilton trick}: We can show simultaneously that $\boldsymbol{\cdot}_1=\boldsymbol{\cdot}_2$ and that both operations are commutative by verifying the single identity
	\begin{equation*}
		\bigl([\alpha]\mathbin{\boldsymbol{\cdot}_1}[\beta]\bigr)\mathbin{\boldsymbol{\cdot}_2}\bigl([\alpha']\mathbin{\boldsymbol{\cdot}_1}[\beta']\bigr)=\bigl([\alpha]\mathbin{\boldsymbol{\cdot}_2}[\alpha']\bigr)\mathbin{\boldsymbol{\cdot}_1}\bigl([\beta]\mathbin{\boldsymbol{\cdot}_2}[\beta']\bigr)
	\end{equation*}
	for all $\alpha$, $\alpha'$, $\beta$, and $\beta'$!\footnote{If you haven't seen this trick before, it will probably blow your mind.} To show the Eckmann--Hilton identity, consider the extension problem
	\begin{equation*}
		\begin{tikzcd}[column sep=8.5em]
			\Lambda_1^2\times\Lambda_1^2\times\square^{n-2}\sqcup_{\Lambda_1^2\times\Lambda_1^2\times\partial\square^{n-2}}\Delta^2\times\Delta^2\times\partial\square^{n-2}\dar\rar["{((\alpha,\,\beta),\,(\alpha',\,\beta'))}\,\cup\,{\const x}"] & X\\
			\Delta^2\times\Delta^2\times\square^{n-1}\urar[dashed,"\rho"',end anchor=202]
		\end{tikzcd}
	\end{equation*}
	which has a solution by \cref{lem:AnodynePushout}. Then observe that for any solution $\rho$, both sides of the Eckmann--Hilton identity are given by $[\rho|_{\Delta^{\{0,2\}}\times \Delta^{\{0,2\}}\times\square^{n-2}}]$.
\end{proof}
%The rest of this subsection will be devoted to prove an analogue of Whitehead's theorem for Kan complexes.
\begin{thm}[\enquote{Whitehead's theorem for Kan complexes}]\label{thm:Whitehead}
	Let $f\colon X\rightarrow Y$ be a morphism of Kan complexes. Then $f$ is a homotopy equivalence if and only if it induces a bijection $\pi_0(X)\cong \pi_0(Y)$ and isomorphisms $\pi_n(X,x)\cong \pi_n(Y,f(x))$ for all $x\in X$ and all $n\geqslant 1$.
\end{thm}
The proof of \cref{thm:Whitehead} will occupy the rest of this subsection. The first step is an analogue of the long exact sequence of a Serre fibration.
\begin{lem}[\enquote{Long exact sequence of a fibration}]\label{lem:FibrationSequence}\label{lem:LongExactFibrationSequence}
	Let $f\colon X\rightarrow Y$ be a Kan fibration between Kan complexes. Let $x\in X$, let $y\coloneqq f(x)$ be its image, and let $F\coloneqq f^{-1}\{y\}=\{y\}\times_YX$ be the fibre over $y$. Then there exists a long exact sequence of groups/pointed sets
	\begin{multline*}
		\dotsb\longrightarrow\pi_n(F,x)\longrightarrow \pi_n(X,x)\longrightarrow\pi_n(Y,y)\overset{\partial}{\longrightarrow}\pi_{n-1}(F,x)\longrightarrow\dotsb\\
		\dotsb\longrightarrow\pi_1(X,x)\longrightarrow\pi_1(Y,y)\overset{\partial}{\longrightarrow}\pi_0(F)\longrightarrow\pi_0(X)\longrightarrow\pi_0(Y)\,.
	\end{multline*}
	In low degrees, exactness means the following:
	\begin{alphanumerate}
		\item There is an action $\pi_1(Y,y)\times\pi_0(F)\rightarrow\pi_0(F)$ in such a way that the boundary map $\partial\colon \pi_1(Y,y)\rightarrow \pi_0(F)$ is given by acting on $[x]\in \pi_0(F)$, the stabiliser of $[x]$ is precisely the image of $\pi_1(X,x)\rightarrow\pi_1(Y,y)$, and two elements of $\pi_0(F)$ map to the same element in $\pi_0(X)$ if and only if they lie in the same orbit of the $\pi_1(Y,y)$-action.\label{enum:ActionOfPi1}
		\item An element in $\pi_0(X)$ maps to the class $[y]\in\pi_0(Y)$ if and only if it lies in the image of $\pi_0(F)\rightarrow \pi_0(X)$.
	\end{alphanumerate}
\end{lem}
\begin{proof}[Proof sketch]
	You can take any proof of the long exact sequence of a Serre fibration, like \cite[Theorem~\href{https://pi.math.cornell.edu/~hatcher/AT/AT.pdf\#page=385}{4.41}]{Hatcher} and adapt the arguments to the simplicial setting. To illustrate how this can be done, we'll explain how to construct the boundary map $\partial$. So let $[\alpha]\in\pi_{n+1}(Y,y)$, where $\alpha\colon(\square^{n+1},\partial\square^{n+1})\rightarrow(Y,y)$ is a map of pairs as usual. Consider the lifting problem
	\begin{equation*}
		\begin{tikzcd}
			\{0\}\times\square^n\sqcup_{\{0\}\times\square^n}\Delta^1\times\partial\square^n\rar["\const x"]\dar & X\dar["f"]\\
			\Delta^1\times\square^n\rar["\alpha"]\urar[dashed,"\vartheta"'] & Y
		\end{tikzcd}
	\end{equation*}
	which has a solution $\vartheta$ by \cref{lem:AnodynePushout}. Then $\vartheta|_{\{1\}\times\square^n}\colon \{1\}\times\square^n\rightarrow X$ factors through $F\rightarrow X$ and it maps $\{1\}\times \partial\square^n$ to $x$. Thus we can define $\partial[\alpha]\coloneqq [\vartheta|_{\{1\}\times\square^n}]\in\pi_n(F,x)$.
\end{proof}
\begin{rem}\label{rem:ExactnessInLowDegrees}
	We will often use \cref{lem:FibrationSequence} in conjunction with the five lemma to deduce that a map of Kan complexes induces a bijection on $\pi_0$ and isomorphisms on $\pi_n$ for all basepoints and all $n\geqslant 1$ (and is thus a homotopy equivalence by \cref{thm:Whitehead}). But the five lemma only applies for exact sequences of groups, not pointed sets. However, these arguments can be saved using the group action from \cref{lem:FibrationSequence}\cref{enum:ActionOfPi1}. We will usually skip the verification in low degrees and just cite the five lemma.
\end{rem}
We also need an alternative description of homotopy groups. This is how Goerss and Jardine define them in \cite[\S \href{http://dodo.pdmi.ras.ru/~topology/books/goerss-jardine.pdf\#page=37}{I.7}]{GoerssJardine}; we chose the cubical approach since it makes the group multiplication easier to visualise.
\begin{lem}\label{lem:HomotopyGroupsSimplex}
	Let $X$ be a Kan complex, let $x\in X$, and let $n\geqslant 0$. Then there is a bijection
	\begin{equation*}
		\pi_n(X,x)\cong \pi_0\F\bigl((\Delta^n,\partial\Delta^n),(X,x)\bigr)\,.
	\end{equation*}
\end{lem}
\begin{proof}[Proof sketch]
	By cutting out a single $n$-simplex from $\square^n$, we can obtain a sub-simplicial set $C^n\subseteq\square^n$ such that $\partial\square^n\subseteq C^n$ is anodyne (in fact, it can be obtained by successively filling horns) and $\square^n/C^n\cong \Delta^n/\partial\Delta^n$. For example, in the case $n=2$, we can choose $C^n$ as in the following picture:
	\begin{equation*}
		\begin{tikzpicture}[commutative diagrams/every diagram,baseline=(mid.base), decoration={markings,mark=at position 0.5 with {\arrow{to}}}]
			\path node[outer sep=0.25ex] (00) at (0,0) {$\bullet$} ++(3.8em,0) node (10) {$\bullet$} ++ (0,3.8em) node (11) {$\bullet$} ++(-3.8em,0) node (01) {$\bullet$};
			\path[commutative diagrams/.cd, every arrow, every label]
			(00) edge[-,postaction={decorate}] node[below=1ex] {$\displaystyle\partial\square^2$} (10)
			(10) edge[-,postaction={decorate}] (11)
			(00) edge[-,postaction={decorate}] (01)
			(01) edge[-,postaction={decorate}] (11);
		\end{tikzpicture}\subseteq
		\begin{tikzpicture}[commutative diagrams/every diagram,baseline=(mid.base), decoration={markings,mark=at position 0.5 with {\arrow{to}}}]
			\path node[outer sep=0.25ex] (00) at (0,0) {$\bullet$} ++(3.8em,0) node (10) {$\bullet$} ++ (0,3.8em) node (11) {$\bullet$} ++(-3.8em,0) node (01) {$\bullet$};
			\path[commutative diagrams/.cd, every arrow, every label]
			(00) edge[-,postaction={decorate}] node[below=1ex] {$\displaystyle C^2$} (10)
			(10) edge[-,postaction={decorate}] (11)
			(00) edge[-,postaction={decorate}] (01)
			(01) edge[-,postaction={decorate}] (11)
			(00) edge[-,postaction={decorate}] (11);
			\path
			(01) to node[pos=0.25] {$\scriptscriptstyle/\!/\!/$} (10);
		\end{tikzpicture}\subseteq
		\begin{tikzpicture}[commutative diagrams/every diagram,baseline=(mid.base), decoration={markings,mark=at position 0.5 with {\arrow{to}}}]
			\path node[outer sep=0.25ex] (00) at (0,0) {$\bullet$} ++(3.8em,0) node (10) {$\bullet$} ++ (0,3.8em) node (11) {$\bullet$} ++(-3.8em,0) node (01) {$\bullet$};
			\path[commutative diagrams/.cd, every arrow, every label]
			(00) edge[-,postaction={decorate}] node[below=1ex] {$\displaystyle \square^2$} (10)
			(10) edge[-,postaction={decorate}] (11)
			(00) edge[-,postaction={decorate}] (01)
			(01) edge[-,postaction={decorate}] (11)
			(00) edge[-,postaction={decorate}] (11);
			\path
			(01) to node[pos=0.25] {$\scriptscriptstyle/\!/\!/$} node[pos=0.75] {$\scriptscriptstyle/\!/\!/$} (10);
		\end{tikzpicture}
	\end{equation*}
	In general, the $n$-simplex $\Delta^n\rightarrow \square^n$ that we cut out to obtain $C^n$ sends the vertex $\{i\}$ to the vertex $\{1\}^i\times\{0\}^{n-i}$.
	
	Now observe that since $x$ is just a point, we have $\F((B,A),(X,x))\cong \F((B/A,*),(X,x))$ for every inclusion $A\subseteq B$ of simplicial sets. Since $\square^n/C^n\cong \Delta^n/\partial\Delta^n$, we only need to prove $\pi_0\F((\square^n,\partial\square^n),(X,x))\cong\pi_0\F((\square^n,C^n),(X,x))$. This follows from a more general claim:
	\begin{alphanumerate}\itshape
		\item[\boxtimes] Let $A'\subseteq A$ be anodyne and let $A\subseteq B$ be any inclusion of simplicial sets. Then we have a bijection $\pi_0\F((B,A),(X,x))\cong \pi_0\F((B,A'),(X,x))$.\label{claim:Annoying}
	\end{alphanumerate}
	To prove \cref{claim:Annoying}, put $F\coloneqq \F((B,A),(X,x))$ and $F'\coloneqq \F((B,A'),(X,x))$ for short. Consider the pullback $P\coloneqq \F(A',\{x\})\times_{\F(A',X)}\F(A,X)$. Then $F'\rightarrow P$ is a Kan fibration, since it is a pullback of $\F(B,X)\rightarrow\F(A,X)$, which is Kan by \cref{cor:FKanFibration}. Manipulating pullbacks, we find $F\cong \F(A,\{x\})\times_PF'$. Now $\F(A,\{x\})\cong *$ is just a point, so $F$ is a fibre of the Kan fibration $F'\rightarrow P$. Furthermore, $P\rightarrow \F(A',\{x\})\cong *$ is a trivial fibration, since it is a pullback of $\F(A,X)\rightarrow\F(A',X)$, which is a trivial fibration by \cref{cor:FKanFibration}. By \cref{lem:TrivialFibrationHomotopyEquivalence} below, this means that $P\rightarrow *$ is a homotopy equivalence and so all homotopy groups of $P$ vanish. Using the long exact sequence from \cref{lem:LongExactFibrationSequence} (for every basepoint in $F'$; a single basepoint won't suffice), we can conclude $\pi_0(F)\cong \pi_0(F')$, as claimed. Note that this works even though we only have an exact sequence of pointed sets on $\pi_0$. This finishes the proof. Another proof of (a more general version of) \cref{claim:Annoying} is in \cite[Lemma~V.3.13]{HigherCatsI}.
\end{proof}
\begin{lem}\label{lem:TrivialFibrationHomotopyEquivalence}
	If $f\colon X\rightarrow Y$ is a trivial fibration between Kan complexes\footnote{The only reason why we restrict ourselves to Kan complexes (or quasi-categories) is that we haven't defined what a homotopy equivalence of arbitrary simplicial sets would be. The lifting problems in the proof can be solved for arbitrary trivial fibrations $f\colon X\rightarrow Y$, with no assumptions on $X$ or $Y$.}, then $f$ is a homotopy equivalence. Similarly, if $F\colon \Cc\rightarrow\Dd$ is a trivial fibration between quasi-categories, then $F$ is an equivalence as in \cref{exm:SimplicialNerve}.
\end{lem}
\begin{proof}%[Proof of \cref{lem:TrivialFibrationHomotopyEquivalence}]
	Since trivial fibrations have lifting against all cofibrations, we can use the lifting problems
	\begin{equation*}
		\begin{tikzcd}
			\emptyset\rar\dar & X\dar["f"]\\
			Y\eqar[r]\urar[dashed,"g"'] & Y
		\end{tikzcd}\quad\text{and}\quad
		\begin{tikzcd}
			\{0\}\times X\sqcup \{1\}\times X\ar[rr,"{(g\circ f,\,\id_X)}"]\dar & & X\dar["f"]\\
			\Delta^1\times X\rar\ar[urr,dashed,"\eta"',end anchor=202] & X\rar["f"] &Y
		\end{tikzcd}
	\end{equation*}
	to first construct a map $g\colon Y\rightarrow X$ such that $f\circ g=\id_Y$ and then to construct a homotopy $\eta\colon g\circ f\Rightarrow \id_X$. Similarly, if $F\colon \Cc\rightarrow \Dd$ is a trivial fibration between quasi-categories, we get a functor $G\colon \Dd\rightarrow\Cc$ such that $F\circ G=\id_\Dd$ and a natural transformation $\eta\colon G\circ F\Rightarrow \id_\Cc$. To show that $\eta$ is an equivalence in $\F(\Cc,\Cc)$ (and thus prove that $F$ and $G$ are mutually inverse equivalences of quasi-categories), we lift furthermore against $\Delta^1\times\Cc\rightarrow\N(J)\times \Cc$, where $J\coloneqq \left\{\InlineJ\right\}$ is the \enquote{free-living isomorphism}, the category with two objects and a pair of mutually inverse isomorphisms between them. 
\end{proof}
The crucial step in the proof of \cref{thm:Whitehead} is to show a \enquote{compression lemma}, as in the proof of Whitehead's theorem in topology (compare to \cite[Lemma~\href{https://pi.math.cornell.edu/~hatcher/AT/AT.pdf\#page=356}{4.6}]{Hatcher}). 
\begin{lem}[\enquote{Compression lemma}]\label{lem:CompressionLemma}
	Let $X$ be a connected\footnote{That is, $\pi_0(X)=*$.} Kan complex such that $\pi_n(X,x)\cong0$ for all $x\in X$ and all $n\geqslant 1$. Let $A\subseteq B$ be an inclusion of simplicial sets, $f\colon B\rightarrow X$ a map and $\eta\colon \Delta^1\times A\rightarrow X$ a homotopy from $f|_A$ to $\const x$. Then $\eta$ can be extended to a homotopy $\ov\eta\colon \Delta^1\times B\rightarrow X$ from $f$ to $\const x$.
\end{lem}
\begin{proof}
	We can construct $\ov\eta$ simplex by simplex, so it suffices to treat the case $A=\partial\Delta^n$ and $B=\Delta^n$ for some $n\geqslant 0$. Consider the extension problem
	\begin{equation*}
		\begin{tikzcd}
			\{0\}\times\Delta^n\sqcup_{\{0\}\times\partial\Delta^n}\Delta^1\times\partial\Delta^n\rar["{(f,\,\eta)}"]\dar & X\\
			\Delta^1\times\Delta^n\urar[dashed,"\vartheta"'] & 
		\end{tikzcd}
	\end{equation*}
	which has a solution $\vartheta$ by \cref{lem:AnodynePushout}. Then $\vartheta|_{\{1\}\times\partial\Delta^n}=\const x$, hence $\vartheta|_{\{1\}\times\Delta^n}$ defines an element in $\pi_0\F((\Delta^n,\partial\Delta^n),(X,x))\cong \pi_n(X,x)\cong0$ using \cref{lem:HomotopyGroupsSimplex} and our assumption on $X$. Hence there is a homotopy $\vartheta'\colon \Delta^1\times\Delta^n\rightarrow X$ such that $\vartheta'|_{\Delta^1\times\partial\Delta^n}=\const x$ as well as $\vartheta'|_{\{0\}\times\Delta^n}=\vartheta|_{\{1\}\times\Delta^n}$ and $\vartheta'|_{\{1\}\times\Delta^n}=\const x$ (in other words, $\vartheta'$ is a homotopy $\vartheta|_{\{1\}\times\Delta^n}\Rightarrow \const x$ relative to the boundary $\partial \Delta^n$). To construct $\ov\eta$, we can now simply compose the homotopies $\vartheta$ and $\vartheta'$ (which we do similarly to \cref{con:HomotopyGroups} and \cref{rem:HomotopyGroupsTopology}, by solving an extension problem along the anodyne map $\Lambda_1^2\times\Delta^n\rightarrow\Delta^2\times\Delta^n$ and then restricting to $\Delta^{\{0,2\}}\times \Delta^n$).
\end{proof}
\begin{lem}\label{lem:ContractibleKanComplex}
	If $X$ is a Kan complex as in \cref{lem:CompressionLemma}, then $X\rightarrow *$ is a trivial fibration.
\end{lem}
\begin{proof}
	We have to show that every extension problem of the following form is solvable:
	\begin{equation*}
		\begin{tikzcd}
			\partial\Delta^n\rar["\sigma"]\dar& X\\
			\Delta^n\urar[dashed] &
		\end{tikzcd}
	\end{equation*}
	Using \cref{lem:CompressionLemma} (applied to $A=\emptyset$ and $B=\partial\Delta^n$), there is a homotopy $\eta\colon \Delta^1\times\partial\Delta^n\rightarrow X$ from $\sigma$ to $\const x$. Now consider the extension problem
	\begin{equation*}
		\begin{tikzcd}[column sep=4em]
			\Delta^1\times\partial\Delta^n\sqcup_{\{1\}\times\partial\Delta^n}\{1\}\times\Delta^n\dar\rar["{(\eta,\,\const x)}"] & X\\
			\Delta^1\times\Delta^n\urar[dashed,"\vartheta"',end anchor=202] &
		\end{tikzcd}
	\end{equation*}
	which has a solution $\vartheta$ by \cref{lem:AnodynePushout} as usual. Then $\vartheta|_{\{0\}\times\Delta^n}$ provides a solution of the original extension problem.
\end{proof}
\begin{lem}\label{lem:TrivialFibration}
	Let $f\colon X\rightarrow Y$ be a Kan fibration between Kan complexes and assume $f$ satisfies the condition from \cref{thm:Whitehead}. Then $f$ is a trivial fibration.
\end{lem}
\begin{proof}
	We have to show that every extension problem of the following form is solvable:
	\begin{equation*}
		\begin{tikzcd}
			\partial\Delta^n\rar["\ov\sigma"]\dar& X\dar["f"]\\
			\Delta^n\urar[dashed]\rar["\sigma"] & Y
		\end{tikzcd}
	\end{equation*}
	Let $\eta\colon\Delta^1\times\Delta^n\rightarrow\Delta^n$ be a homotopy from $\id_{\Delta^n}$ to the constant map $\const n$ (such a homotopy can easily be constructed by hand). Then $\sigma\circ \eta\colon\Delta^1\times\Delta^n\rightarrow Y$ is a homotopy from $\sigma$ to $\const y$, where $y=\sigma(n)$. Now let $F\coloneqq f^{-1}\{y\}=\{y\}\times_YX$ be the fibre over $y$ and consider the lifting problem
	\begin{equation*}
		\begin{tikzcd}[column sep=huge]
			\{0\}\times\partial\Delta^n\rar["\ov\sigma"]\dar & X\dar["f"]\\
			\Delta^1\times\partial\Delta^n\urar[dashed,"\vartheta"']\rar["\sigma\circ\eta|_{\Delta^1\times\partial\Delta^n}"] & Y
		\end{tikzcd}
	\end{equation*}
	which can be solved by \cref{lem:AnodynePushout}. Then $\vartheta|_{\{1\}\times\partial\Delta^n}\colon \{1\}\times\partial\Delta^n\rightarrow X$ factors through $F\rightarrow X$. Using the long exact sequence from \cref{lem:FibrationSequence} and the assumption on $f$, we see $\pi_n(F,x)\cong0$ for all $x\in F$ and all $n\geqslant 0$. Hence $F\rightarrow *$ is a trivial fibration by \cref{lem:ContractibleKanComplex} and so $\vartheta|_{\{1\}\times\partial\Delta^n}\colon \{1\}\times\partial\Delta^n\rightarrow F$ can be extended to a map $\ov\vartheta\colon \{1\}\times\Delta^n\rightarrow F$. Finally, consider the lifting problem
	\begin{equation*}
		\begin{tikzcd}
			\Delta^1\times\partial\Delta^n\sqcup_{\{1\}\times\partial\Delta^n}\{1\}\times\Delta^n\rar["{(\vartheta,\,\ov\vartheta)}"]\dar & X\dar["f"]\\
			\Delta^1\times\Delta^n\urar[dashed,"\rho"']\rar & Y
		\end{tikzcd}
	\end{equation*}
	which can be solved by \cref{lem:AnodynePushout}. Then $\rho|_{\{0\}\times\Delta^n}$ provides a solution for the original lifting problem and we're done.
\end{proof}
\begin{proof}[Proof of \cref{thm:Whitehead}]
	Let's first assume that $f\colon X\rightarrow Y$ is a homotopy equivalence. Then $f$ clearly induces a bijection $\pi_0(X)\cong \pi_0(Y)$. But to get isomorphisms $\pi_n(X,x)\cong \pi_n(Y,y)$ for all $x\in X$, $y=f(x)$, and all $n\geqslant 1$, we have to show that $f\colon (X,x)\rightarrow(Y,y)$ is also a pointed homotopy equivalence, which is not entirely trivial. 
	
	It suffices to show that $\pi_0\F((Y,y),(Z,z))\rightarrow \pi_0\F((X,x),(Z,z))$ is surjective for every pointed Kan complex $(Z,z)$. Indeed, if this is true, then plugging in $(Z,z)=(X,x)$ yields a pointed map $g\colon (Y,y)\rightarrow (X,x)$ together with a pointed homotopy $g\circ f\simeq\id_{(X,x)}$. In particular, $g$ is a homotopy equivalence too. Repeating the argument with $g$, we obtain $h\colon (X,x)\rightarrow (Y,y)$ together with $h\circ g\simeq\id_{(Y,y)}$. Then $g$ is a pointed homotopy equivalence and thus $f$ must be a pointed homotopy equivalence too. To show that $\pi_0\F((Y,y),(Z,z))\rightarrow \pi_0\F((X,x),(Z,z))$ is surjective, first note that we have Kan fibrations
	\begin{equation*}
		\ev_x\colon \F(X,Z)\longrightarrow\F\bigl(\{x\},Z\bigr)\cong Z\quad\text{and}\quad\ev_y\colon \F(Y,Z)\longrightarrow \F\bigl(\{y\},Z\bigr)\cong Z
	\end{equation*}
	by \cref{cor:FKanFibration}. Using \cref{con:FOfPairs}, we see that the fibres of these fibrations are given by $\ev_x^{-1}\{z\}\cong \F((X,x),(Z,z))$ and $\ev_y^{-1}\{z\}\cong \F((Y,y),(Z,z))$. Using \cref{lem:FibrationSequence}, we obtain a diagram of exact sequences
	\begin{equation*}
		\begin{tikzcd}
			\dotsb\rar & \pi_1(Z,z)\eqar[d]\rar & \pi_0\F\bigl((Y,y),(Z,z)\bigr)\dar\rar & \pi_0\F(Y,Z)\dar["\cong"] \rar & \pi_0(Z)\rar\eqar[d] & \dotsb\\
			\dotsb\rar & \pi_1(Z,z)\rar & \pi_0\F\bigl((X,x),(Z,z)\bigr)\rar & \pi_0\F(X,Z)\rar & \pi_0(Z)\rar& \dotsb
		\end{tikzcd}
	\end{equation*}
	Since we assume $f\colon X\rightarrow Y$ to be an unpointed homotopy equivalence, it induces a bijection $\pi_0\F(Y,Z)\cong\pi_0\F(X,Z)$, as indicated above. Then a diagram chase involving \cref{lem:FibrationSequence}\cref{enum:ActionOfPi1} shows that $\pi_0\F((Y,y),(Z,z))\rightarrow \pi_0\F((X,x),(Z,z))$ is surjective. As argued above, this is what we need.
	
	Conversely, assume that $f\colon X\rightarrow Y$ induces a bijection $\pi_0(X)\cong \pi_0(Y)$ and isomorphisms $\pi_n(X,x)\cong \pi_n(Y,y)$ for all $x\in X$, $y=f(x)$, and all $n\geqslant 1$. \cref{lem:SmallObjectArgument} allows us to choose a factorisation
	\begin{equation*}
		f\colon X\overset{i}{\longrightarrow} \ov X\overset{\ov f}{\longrightarrow}Y
	\end{equation*}
	where $i$ is anodyne and $\ov f$ is a Kan fibration. Then $i$ and $\ov f$ are homotopy equivalences. Indeed, \cref{cor:FKanFibration} shows that $\F(\ov X,Z)\rightarrow\F(X,Z)$ is a trivial fibration for every Kan complex $Z$, hence a bijection on $\pi_0$ by \cref{lem:TrivialFibrationHomotopyEquivalence}. Plugging in $Z=X$ and $Z=\ov X$ shows that $i$ is a homotopy equivalence, as claimed.  The Kan fibration $\ov f\colon \ov X\rightarrow Y$ is a trivial fibration by \cref{lem:TrivialFibration}, hence a homotopy equivalence by \cref{lem:TrivialFibrationHomotopyEquivalence}. We are done!
\end{proof}

\subsection{Simplicial approximation and model categories}

\begin{thm}[Simplicial approximation]\label{thm:SimplicialApproximation}
	For every Kan complex $X$ we have a bijection $\pi_0(X)\cong \pi_0(\abs*{X})$ and isomorphisms $\pi_n(X,x)\cong \pi_n(\abs*{X},x)$ for all $x\in X$ and all $n\geqslant 1$. Similarly, for every topological space $Y$ we have a bijection $\pi_0(Y)\cong \pi_0(\Sing Y)$ and isomorphisms $\pi_n(Y,y)\cong \pi_n(\Sing Y,y)$ for all $y\in Y$ and all $n\geqslant 1$. In particular, the adjunction
	\begin{equation*}
		\abs*{\,\cdot\,}\colon \cat{Kan}\doublelrmorphism \cat{Top}\noloc {\Sing}
	\end{equation*}
	from \cref{par:GeometricRealisation} induces homotopy equivalences $u_X\colon X\overset{\simeq}{\longrightarrow} \Sing{\abs*{X}}$ for all Kan complexes $X$ and weak equivalences $c_Y\colon \abs*{\Sing Y}\rightarrow Y$ for every topological space $Y$.\hfill$\blacksquare$
\end{thm}
The proof of \cref{thm:SimplicialApproximation} is a technical headache. For a full proof, have a look at Fabian's and Christoph Winges' lecture notes \cite[\S V.5]{HigherCatsI}; several versions of this theorem can also be found in \cite[\S \href{https://pi.math.cornell.edu/~hatcher/AT/AT.pdf\#page=186}{2.C}]{Hatcher}.

\cref{thm:SimplicialApproximation} is an incarnation of Grothendieck's \emph{homotopy hypothesis}. It tells us, essentially, that as long as we're only interested in topological spaces up to weak equivalence, or CW-complexes up to homotopy equivalence, we can safely pass to the category of Kan complexes, or better yet, to the quasi-category $\cat{An}$ from \cref{exm:SimplicialNerve}\cref{enum:An}. In particular, everything we would ever like to know about homotopy groups (or homology groups etc.) will be captured by $\cat{An}$! We'll see through many examples how this point of view leads to clean, abstract, and conceptually satisfying proofs of many classical topological results and ultimately to a deeper understanding of homotopy theory.

At this point it seems natural to leave a few words about \emph{model categories}. Historically, these have played a dominating role in the development of $\infty$-category theory and to this day they are an indispensible tool in the foundations of the topic (especially in the proof of Lurie's straightening/unstraightening equivalence, \cref{thm:Straightening}) as well as in many other areas of topology. So dismissing them as a tool of the past would be blatantly ignorant and outright disrespectful. Still, model categories run contrary to the modern point of view that I'm trying to get across in these notes, and so I'll try to avoid them entirely---which is, of course, only achievable by conveniently hiding their unavoidable uses in black boxes. But at the very least, I should tell you the definition.	
\begin{defi}\label{def:ModelCategory}
	Let $\Aa$ be a category with finite limits and colimits. A \emph{model structure on $\Aa$} consists of 3 classes of morphisms $C$, $F$, and $W$ (\emph{cofibrations}, \emph{fibrations}, and \emph{weak equivalences}) satisfying the following properties:
	\begin{alphanumerate}
		\item All isomorphisms of $\Aa$ are contained in each of the classes $C$, $F$, $W$, and these classes are all closed under retracts.\label{enum:ModelCategoryCFW}
		\item $W$ is closed under $2$-out-of-$3$. That is, if two of $f$, $g$, and $g\circ f$ are weak equivalences, then so is the third.\label{enum:ModelCategory2OutOf3}
		\item A lifting problem\label{enum:ModelCategoryLifting}
		\begin{equation*}
			\begin{tikzcd}
				a\rar\dar["i"'] & x\dar["f"]\\
				b\urar[dashed]\rar & y
			\end{tikzcd}
		\end{equation*}
		with $i\in C$ a cofibration and $f\in F$ a fibration always has a solution provided that $i$ is a \emph{trivial cofibration} (a cofibration that is also a weak equivalence) or $f$ is a \emph{trivial fibration} (a fibration that is also a weak equivalence).
		\item Every morphism in $\Aa$ can be factored into a cofibration followed by a trivial fibration and into a trivial cofibration followed by a fibration. That is, if $a\rightarrow y$ is a morphism in $\Aa$, then there exist factorisations\label{enum:ModelCategoryFactorisations}
		\begin{equation*}
			a\longrightarrow x\longrightarrow y\quad\text{and}\quad a\longrightarrow b\longrightarrow y\,,
		\end{equation*}
		where $(a\rightarrow x)\in C$ and $(x\rightarrow y)\in F\cap W$ as well as $(a\rightarrow b)\in C\cap W$ and $(b\rightarrow y)\in F$. Sometimes these factorisations are required to be functorial (which is satisfied in virtually all examples).
	\end{alphanumerate}
	A category $\Aa$ equipped with a model structure is called a \emph{model category}. If $\Aa$ is a model category, then $x\in \Aa$ is called \emph{cofibrant} if the map from the initial object to $x$ is a cofibration, and \emph{fibrant} if the map from $x$ to the terminal object is a fibration. We call $x$ \emph{bifibrant} if it is both fibrant and cofibrant.
\end{defi}
\begin{exm}\label{exm:KanQuillenModelStructure}
	Basically, the entirety of \cref{subsec:Fibrations} can be summarised by saying that $\cat{sSet}$ carries a model structure in which cofibrations are exactly that, fibrations are Kan fibrations, and weak equivalences are morphisms that can be factored into an anodyne map followed by a trivial fibration. This model structure is called the \emph{Kan--Quillen model structure}.
\end{exm}
\begin{exm}\label{exm:JoyalModelStructure}
	As the Kan--Quillen model structure \enquote{models} the quasi-category $\cat{An}$ from  \cref{exm:SimplicialNerve}, it seems natural to ask whether there is another model structure on $\cat{sSet}$ that \enquote{models} $\cat{Cat}_\infty$. The naive attempt would be to ask that fibrations be inner fibrations and that trivial cofibrations be inner anodyne maps. But there are examples of cofibrations between quasi-categories that are equivalences in $\Cat_\infty$, but not inner anodyne; for example, the functor $\{0\}\rightarrow \N(J)$, where $J=\left\{\InlineJ\right\}$ is the \enquote{free-living isomorphism}. It was an insight of Joyal how this can be fixed: There is a model structure on $\cat{sSet}$, called the \emph{Joyal model structure} such that cofibrations are just that and weak equivalences are those maps $A\rightarrow B$ such that $\pi_0\core\F(B,\Cc)\rightarrow \pi_0\core\F(A,\Cc)$ is bijective for every quasi-category $\Cc$. Here $\pi_0\core$ means the set of equivalence classes of objects; once we've proved the hard part of \cref{thm:AnimaeAreKanComplexes}, this notation will be consistent with \cref{def:Homotopy}\cref{enum:Pi0}. Weak equivalences in the Joyal model structure are called \emph{Joyal equivalences}. We'll give another characterisation in \cref{lem:JoyalEquivalence} below.
	
	Fibrant objects in the Joyal model structure are precisely quasi-categories. General fibrations in the Joyal model structure are harder to pin down. However, fibrations between quasi-categories are characterised by a lifting property: They are those inner fibrations that also have lifing against $\{0\}\rightarrow\N(J)$. We'll call these \emph{isofibrations} and we'll meet them again in model category fact~\cref{par:HomotopyPullback}\cref{enum:HomotopyPullbackOfQuasicategories}. For proofs see \cite[Theorem~\href{https://mat.uab.cat/~kock/crm/hocat/advanced-course/Quadern45-2.pdf\#page=153}{6.12}]{JoyalQuasiCategoriesAndApplications} or \cite[Theorem~VIII.23]{HigherCatsII}.
\end{exm}
\begin{lem}\label{lem:JoyalEquivalence}
	A map $A\rightarrow B$ of simplicial sets is a Joyal equivalence if and only if $\F(B,\Cc)\rightarrow\F(A,\Cc)$ is an equivalence in $\cat{Cat}_\infty$ for every quasi-category $\Cc$.
\end{lem}
\begin{proof}
	The \enquote{if} part is trivial, so assume $A\rightarrow B$ is a Joyal equivalence as in \cref{exm:JoyalModelStructure}.  By \cref{lem:SmallObjectArgument}, we may choose an inner anodyne map $B\rightarrow\Bb$ into a quasi-category and a factorisation $A\rightarrow\Aa\rightarrow\Bb$ into an inner anodyne map followed by an inner fibration. Then $\Aa$ is a quasi-category too. Note that $\F(\Aa,\Cc)\rightarrow \F(A,\Cc)$ is a trivial fibration by \cref{cor:FKanFibration} and thus an equivalence of quasi-categories by \cref{lem:TrivialFibrationHomotopyEquivalence}; the same is true for $\F(\Bb,\Cc)\rightarrow \F(B,\Cc)$. So it's enough to show that $\F(\Bb,\Cc)\rightarrow\F(\Aa,\Cc)$ is an equivalence of quasi-categories, and for this, it's enough to show that our functor $F\colon \Aa\rightarrow\Bb$ is an equivalence of quasi-categories.
	
	We know that $F^*\colon \pi_0\core\F(\Bb,\Cc)\rightarrow\pi_0\core\F(\Aa,\Cc)$ is bijective for every quasi-category $\Cc$. Plugging in $\Cc=\Aa$ and choosing a preimage of $\id_\Aa$ yields a functor $G\colon \Bb\rightarrow\Aa$ together with an equivalence $G\circ F\simeq \id_\Aa$. Since $F^*$ and $(G\circ F)^*$ are bijective, it follows that $G^*\colon \pi_0\core\F(\Aa,\Cc)\rightarrow\pi_0\core\F(\Bb,\Cc)$ must too be bijective for every quasi-category $\Cc$. By the same argument, we obtain $H\colon \Aa\rightarrow \Bb$ together with an equivalence $H\circ G\simeq \id_\Bb$. Then $G$ must be an isomorphism in $\operatorname{ho}(\cat{Cat}_\infty)$ and so $F$ must be too. This shows that $F$ is an equivalence in $\cat{Cat}_\infty$, as desired.
\end{proof}
\begin{exm}\label{exm:QuillenAdjunction}
	There are also several model structures on $\cat{Top}$. For example, there is the \emph{Serre--Quillen model structure}, in which cofibrations are retracts of relative CW-inclusions, weak equivalences are just that, and fibrations are Serre fibrations.
	
	An adjunction $L\colon \Aa\shortdoublelrmorphism \Bb\noloc R$ between model categories is called a \emph{Quillen adjunction} if the left adjoint $L$ preserves cofibrations and trivial cofibrations, or equivalently, if the right adjoint $R$ preserves fibrations and trivial fibrations. It is called a \emph{Quillen equivalence} if, additionally, the following conditions hold:
	\begin{alphanumerate}
		\item For every cofibrant object $x\in\Aa$ and every trivial cofibration $i\colon L(x)\rightarrow y$ into a fibrant object in $\Bb$, the composition $R(i)\circ u_x\colon x\rightarrow RL(x)\rightarrow R(y)$ is a weak equivalence in $\Aa$.
		\item For every fibrant object $y\in\Bb$ and every trivial fibration $f\colon x\rightarrow R(y)$ from a cofibrant object in $\Aa$, the composition $c_y\circ L(f)\colon L(x)\rightarrow LR(y)\rightarrow y$ is a weak equivalence in $\Bb$.
	\end{alphanumerate}
	Then one way to understand \cref{thm:SimplicialApproximation} is that the adjunction $\abs*{\,\cdot\,}\colon\cat{sSet}\shortdoublelrmorphism \cat{Top}\noloc \Sing$ from \cref{par:GeometricRealisation} is a Quillen equivalence.
\end{exm}

\newpage
\section{Joyal's lifting theorem}\label{sec:JoyalLifting}
Let's begin by stating the theorem that this section owes its name to. We won't give a proof; the proof is not too difficult, at least compared to our later black box \cref{thm:Straightening}, but it uses some constructions (joins and thin slices) that we've avoided so far and will continue to avoid. If you're interested, Joyal's original proof \cite[Theorem~\href{https://people.math.rochester.edu/faculty/doug/otherpapers/Joyal-QCKC.pdf\#page=5}{2.2}]{JoyalLifting} as well as the accounts in \cite[Theorem~2.1.8]{Land} or \cite[Tag~\href{https://kerodon.net/tag/01H0}{01H0}]{Kerodon} are all very readable.
\begin{thm}[Joyal's lifting theorem]\label{thm:JoyalLifting}
	Let $p\colon \Cc\rightarrow \Dd$ be an inner fibration of quasi-categories. Then for all $n\geqslant 2$, every lifting problem of the form
	\begin{equation*}
		\begin{tikzcd}
			\Lambda_0^n\rar\dar & \Cc\dar["p"]\\
			\Delta^n\rar\urar[dashed] & \Dd
		\end{tikzcd}\quad\text{or}\quad
		\begin{tikzcd}
			\Lambda_n^n\rar\dar & \Cc\dar["p"]\\
			\Delta^n\rar\urar[dashed] & \Dd
		\end{tikzcd}
	\end{equation*}
	in which the $1$-simplex $\Delta^{\{0,1\}}\subseteq \Lambda_0^n\rightarrow\Cc$ or $\Delta^{\{n-1,n\}}\subseteq \Lambda_n^n\rightarrow\Cc$ is sent to an equivalence in $\Cc$, admits a solution.\hfill$\blacksquare$
\end{thm}

\subsection{Consequences of Joyal's lifting theorem}
This subsection is devoted to convincing you what a ridiculously strong result \cref{thm:JoyalLifting} actually is. We begin with some simple corollaries and work our way up to two highly non-trivial theorems.
\begin{cor}[\enquote{Animae and Kan complexes are the same}]\label{cor:AnimaKanComplexes}
	A quasi-category $\Cc$ is a Kan complex if and only if it is an anima, that is, if and only if all its morphisms are equivalences.
\end{cor}
\begin{proof}
	We've seen in \cref{thm:AnimaeAreKanComplexes} that Kan complexes are animae. So let's assume $\Cc$ is an anima. Since $\Cc$ is a quasi-category, it suffices to show that all outer horns $\Lambda_0^n\rightarrow \Cc$ and $\Lambda_n^n\rightarrow \Cc$ have fillers. For $n=1$, this is clear, since we can extend $\{0\}\rightarrow \Cc$ or $\{1\}\rightarrow\Cc$ to a degenerate simplex $\Delta^1\rightarrow\Cc$. For $n\geqslant 2$, we can apply \cref{thm:JoyalLifting} to the inner fibration $p\colon \Cc\rightarrow *$.
\end{proof}
\begin{cor}[\enquote{Left fibrations over animae are Kan fibrations}]\label{cor:LeftFibrationsOverAnima}
	Let $Y$ be a Kan complex and let $f\colon X\rightarrow Y$ be a left fibration. Then $f$ is a Kan fibration and thus $X$ is a Kan complex. A dual assertion holds for right fibrations.
\end{cor}
\begin{proof}[Proof sketch]
	We must show that for all $n\geqslant1$ every lifting problem
	\begin{equation*}
		\begin{tikzcd}
			\Lambda_n^n\rar\dar & X\dar["f"]\\
			\Delta^n\rar\urar[dashed] & Y
		\end{tikzcd}
	\end{equation*}
	has a solution. Let's first consider the case $n=1$. Since every morphism of $Y$ is an equivalence by \cref{cor:AnimaKanComplexes}, the map $\Delta^1\rightarrow Y$ extends to a map $\N(J)\rightarrow Y$, where $J\coloneqq \{\InlineJ\}$ is the \enquote{free-living isomorphism}, the category with two objects and a pair of mutually inverse isomorphisms between them. It can be shown via explicit horn filling that $\{1\}\rightarrow \N(J)$ is both left and right anodyne. Since $f$ is a left fibration, we get a lift $\N(J)\rightarrow X$, which upon restriction along $\Delta^1\rightarrow \N(J)$ yields a solution of our original lifting problem.
	
	Now let $n\geqslant 2$. It suffices to show that every morphism in $X$ is an equivalence, because then \cref{thm:JoyalLifting} will solve our lifting problem. So let $\alpha\colon x\rightarrow y$ be a morphism in $X$ and consider the map $\sigma\colon\Lambda_0^2\rightarrow X$ represented by
	\begin{equation*}
		\sigma=\begin{tikzpicture}[commutative diagrams/every diagram,baseline=(mid.base)]
			\path node[outer sep=0.25ex] (0) at (0,0) {$x$} ++(0:3.8em) node[text depth=0pt,outer sep=0.25ex] (1) {$y$} ++ (120:3.8em) node[outer sep=0.25ex] (2) {$x$};
			\path (0) to node[pos=0.5] (mid) {} (2);
			\path[commutative diagrams/.cd, every arrow, every label]
			(0) edge node[swap] {$\alpha$} (1)
			(1) edge[dotted] (2)
			(0) edge node {$\id_x$} (2);
		\end{tikzpicture}
	\end{equation*}
	Since $Y$ is an anima, $f(\alpha)$ is an equivalence and so $\vartheta\coloneqq f\circ\sigma \colon \Lambda_0^2\rightarrow Y$ can be extended to a map $\ov\vartheta\colon \Delta^2\rightarrow Y$. Since $f\colon X\rightarrow Y$ is a left fibration, we can lift $\ov\vartheta$ to a map $\ov\sigma\colon \Delta^2\rightarrow X$ such that $\ov\sigma|_{\Lambda_0^2}=\sigma$ and $f\circ \ov\sigma=\ov\vartheta$. The $2$-simplex $\ov\sigma$ shows that $\alpha$ has a left inverse $\beta$. Repeating the argument with $\beta$, we see that $\beta$ itself has a left inverse. Then $\beta$ must be an equivalence. Hence its right inverse $\alpha$ must be an equivalence too. 
\end{proof}
\begin{cor}[\enquote{$\Hom_\Cc$ takes values in animae}]\label{cor:HomAnima}
	Let $\Cc$ be a quasi-category. Then for all $x,y\in\Cc$, the slice category projection $t\colon \Cc_{x/}\rightarrow \Cc$ from \cref{par:HomInQuasiCategories} is a left fibration and $\Hom_\Cc(x,y)$ is an anima.
\end{cor}
\begin{proof}[Proof sketch]
	By \cref{par:HomInQuasiCategories}, $t\colon \Cc_{x/}\rightarrow \Cc$ is a pullback of $(s,t)\colon \Ar(\Cc)\rightarrow \Cc\times\Cc$, which is an inner fibration by \cref{cor:FKanFibration}. Hence $t\colon \Cc_{x/}\rightarrow\Cc$ is an inner fibration too and we only need to solve outer horn lifting problems
	\begin{equation*}
		\begin{tikzcd}
			\Lambda_0^n\rar\dar & \Cc_{x/}\dar["t"]\\
			\Delta^n\urar[dashed]\rar & \Cc
		\end{tikzcd}
	\end{equation*}
	for all $n\geqslant 1$. Write $\Cc_{x/}\cong \{x\}\times_{\Cc,s}\Ar(\Cc)$ as in \cref{par:HomInQuasiCategories}. By the usual adjunction tricks, a horn lifting problem as above is equivalent to an extension problem
	\begin{equation*}
		\begin{tikzcd}
			\Lambda_0^n\times\Delta^1\sqcup_{\Lambda_0^n\times\{1\}}\Delta^n\times\{1\}\rar["f"]\dar & \Cc\\
			\Delta^n\times\Delta^1\urar[dashed,"\ov f"'] & 
		\end{tikzcd}
	\end{equation*}
	with the additional condition that $f$ satisfies $f|_{\Lambda_0^n\times\{0\}}=\const x$ and the extension must satisfy $\ov f|_{\Delta^n\times\{0\}}=\const x$. Such an extension problem can be written as a sequence of horn filling problems. Each horn is either an inner horn, which can be filled by \cref{def:QuasiCategory}, or a horn whose first edge is sent to $\const x$, which can be filled by \cref{thm:JoyalLifting}, or a horn that can be filled with a degenerate simplex. Up to the horn filling combinatorics, which we skip as usual, this proves that $t\colon \Cc_{x/}\rightarrow\Cc$ is a left fibration. 
	
	To prove that $\Hom_\Cc(x,y)$ is an anima, recall the pullback diagram
	\begin{equation*}
		\begin{tikzcd}
			\Hom_\Cc(x,y)\rar\dar\drar[pullback] & \Cc_{x/}\dar["t"]\\
			\{y\}\rar & \Cc
		\end{tikzcd}
	\end{equation*}
	from \cref{par:HomInQuasiCategories}. It follows that $\Hom_\Cc(x,y)\rightarrow\{y\}$ is a left fibration. Hence $\Hom_\Cc(x,y)$ is a Kan complex by \cref{cor:LeftFibrationsOverAnima}.
\end{proof}
%Next, we'll use \cref{thm:JoyalLifting} to prove some very non-trivial and very useful theorems.
\begin{thm}[\enquote{Equivalences of functors can be checked pointwise}]\label{thm:EquivalencePointwise}
	Let $F,G\colon \Cc\rightarrow\Dd$ be functors of quasi-categories and let $\eta\colon F\Rightarrow G$ be a natural transformation \embrace{that is, a $1$-simplex $\Delta^1\rightarrow\F(\Cc,\Dd)$ from $F$ to $G$}. Then $\eta$ is an equivalence of functors if and only if $\eta_x\colon F(x)\rightarrow G(x)$ is an equivalence in $\Dd$ for all $x\in\Cc$.
\end{thm}
If all $\eta_x\colon F(x)\rightarrow G(x)$ are equivalences, we can choose inverses $\vartheta_x\colon G(x)\rightarrow F(x)$, but already that step is non-canonical, since inverses are no longer unique in quasi-categories. To assemble the $\vartheta_x$ into a natural transformation $\vartheta\colon G\Rightarrow F$ involves infinitely more non-canonical choices, and we have to make them all in a coherent way. This is an impossible task to do by hand, but incredibly, \cref{thm:EquivalencePointwise} does it for us!
\begin{proof}[Proof sketch of \cref{thm:EquivalencePointwise}]
	The \enquote{only if} part is clear. To prove the \enquote{if} part, we start with some general observations. Let $i\colon A\rightarrow B$ be a cofibration of simplicial sets and consider lifting problems of the form
	\begin{equation}\tag*{}\label{toast}
		\tabcolsep0pt
		\noindent\begin{tabularx}{\textwidth}{l X c X c X l X c X}
			$(*)$ & & $\begin{tikzcd}[ampersand replacement=\&]
				\Lambda_0^2\dar\rar \& \F(B,\Dd)\dar\\
				\Delta^2\urar[dashed]\rar \& \F(A,\Dd)
			\end{tikzcd}$ & & or equivalently & & $(**)$ & & $\begin{tikzcd}[ampersand replacement=\&]
				\Lambda_0^2\times B\sqcup_{\Lambda_0^2\times A}\Delta^2\times A\dar\rar \& \Dd\\
				\Delta^2\times B\urar[dashed]\&
			\end{tikzcd}$ &
		\end{tabularx}
	\end{equation}
	Consider those (\hyperref[toast]{$**$}) for which the $1$-simplex $\Delta^{\{0,1\}}\times\{b\}\rightarrow\Lambda_0^2\times B\rightarrow \Dd$ is an equivalence in $\Dd$ for all $b\in B_0$. We claim:
	\begin{alphanumerate}\itshape
		\item[\boxtimes] Let $\Sigma$ be the class of all cofibrations $i\colon A\rightarrow B$ such that every extension problem \embrace{\hyperref[toast]{$**$}}, for which $\Delta^{\{0,1\}}\times\{b\}\rightarrow\Lambda_0^2\times B\rightarrow \Dd$ is an equivalence in $\Dd$ for all $b\in B_0$, can be solved. Then $\Sigma$ is saturated and contains $\partial \Delta^n\rightarrow\Delta^n$ for all $n\geqslant 0$.\label{claim:Saturated}
	\end{alphanumerate}
	Saturatedness of $\Sigma$ is straightforward to check. To see that $\Sigma$ contains $\partial \Delta^n\rightarrow\Delta^n$, one uses \cref{thm:JoyalLifting}; as usual, we skip the horn filling combinatorics. A full argument is in Fabian's notes \cite[Lemma~VII.2]{HigherCatsI}.
	
	By \cref{claim:Saturated} and \cref{lem:Cofibration}, $\Sigma$ contains all cofibrations of simplicial sets. In particular, $\Sigma$ contains $i\colon \coprod_{x\in\Cc}\{x\}\rightarrow\Cc$. Now let $\eta\colon F\Rightarrow G$ be a natural transformation such that $\eta_x\colon F(x)\rightarrow G(x)$ is an equivalence for all $x\in \Cc$. To construct a left inverse of $\eta$, consider the map $\sigma\colon\Lambda_0^2\rightarrow\F(\Cc,\Dd)$ represented by
	\begin{equation*}
		\sigma=\begin{tikzpicture}[commutative diagrams/every diagram,baseline=(mid.base)]
			\path node[outer sep=0.25ex] (0) at (0,0) {$F$} ++(0:3.8em) node[text depth=0pt,outer sep=0.25ex] (1) {$G$} ++ (120:3.8em) node[outer sep=0.25ex] (2) {$F$};
			\path (0) to node[pos=0.5] (mid) {} (2);
			\path[commutative diagrams/.cd, every arrow, every label]
			(0) edge[white,double, double equal sign distance,-{implies[black]}] (1)
			(1) edge[white,double, double equal sign distance,-{implies[black]}] (2)
			(0) edge[white,double, double equal sign distance,-{implies[black]}] (2);	\path[commutative diagrams/.cd, every arrow, every label,shift right=0.1em]
			(0) edge[shorten >=0.22em,-] node[swap] {$\eta$} (1)
			(1) edge[dotted,shorten >=0.22em,-] (2)
			(0) edge[shorten >=0.22em,-] (2);
			\path[commutative diagrams/.cd, every arrow, every label,shift left=0.1em]
			(0) edge[shorten >=0.22em,-] (1)
			(1) edge[dotted,shorten >=0.22em,-] (2)
			(0) edge[shorten >=0.22em,-] node {$\id_F$} (2);
		\end{tikzpicture}
	\end{equation*}
	Our assumption on $\eta$ means that its image under $i^*\colon \F(\Cc,\Dd)\rightarrow\F\left(\coprod_{x\in\Cc}\{x\},\Dd\right)\cong \prod_{x\in\Cc}\Dd$ is an equivalence. Hence $i^*\circ \sigma$ can be extended to a $2$-simplex $\Delta^2\rightarrow \F\left(\coprod_{x\in\Cc}\{x\},\Dd\right)$ and we obtain a lifting diagram
	\begin{equation*}
		\begin{tikzcd}
			\Lambda_0^2\rar\dar & \F(\Cc,\Dd)\dar\\
			\Delta^2\rar\urar[dashed] & \F\left(\coprod_{x\in\Cc}\{x\},\Dd\right)
		\end{tikzcd}
	\end{equation*}
	which has a solution by what the above arguments. Hence $\eta$ has a left inverse $\vartheta\colon G\Rightarrow F$. Again, $\vartheta_x\colon G(x)\rightarrow F(x)$ must be equivalences for all $x\in\Cc$. Repeating the argument with $\vartheta$ shows that $\vartheta$ must have a left inverse too. Then $\vartheta$ must be an equivalence and so its right inverse $\eta$ must be an equivalence too.
\end{proof}
A similar miracle as \cref{thm:EquivalencePointwise} is the following theorem.

\begin{thm}[\enquote{Fully faithful \&\ essentially surjective implies equivalence}]\label{thm:EquivalenceFullyFaithfulEssentiallySurjective}
	A functor $F\colon \Cc\rightarrow \Dd$ of $\infty$-categories is an equivalence if and only if the following conditions are satisfied:
	\begin{alphanumerate}
		\item $F$ is fully faithful. That is, $F$ induces homotopy equivalences of animae
		\begin{equation*}
			\Hom_\Cc(x,y)\overset{\simeq}{\longrightarrow}\Hom_\Dd\bigl(F(x),F(y)\bigr)
		\end{equation*}
		for all $x,y\in\Cc$.
		\item $F$ is essentially surjective. That is, $F$ induces a surjection $\pi_0\core (\Cc)\rightarrow\pi_0\core(\Dd)$.
	\end{alphanumerate}
\end{thm}
\begin{rem}\label{rem:FullyFaithfulImpliesInjectiveOnPi0Core}
	The \enquote{only if} part of \cref{thm:EquivalenceFullyFaithfulEssentiallySurjective} is easy. For later use, we remark that $F\colon \Cc\rightarrow\Dd$ being fully faithful implies that $\pi_0\core(\Cc)\rightarrow\pi_0\core(\Dd)$ is injective. %Indeed, we have to show that $x\simeq y$ in $\Cc$ if and only if $F(x)\simeq F(y)$ in $\Dd$, which follows from the homotopy equivalences $\Hom_\Cc(x,x)\simeq \Hom_\Dd(F(x),F(x))$, $\Hom_\Cc(x,y)\simeq \Hom_\Dd(F(x),F(y))$, $\Hom_\Cc(y,y)\simeq \Hom_\Dd(F(y),F(y))$, and $\Hom_\Cc(y,x)\simeq \Hom_\Dd(F(y),F(x))$.
	Indeed, this is purely an assertion about the homotopy categories of $\Cc$ and $\Dd$ and it follows from the fact that if $F\colon \Cc\rightarrow\Dd$ is a fully faithful functor of quasi-categories, then $\operatorname{ho}(F)\colon \operatorname{ho}(\Cc)\rightarrow \operatorname{ho}(\Dd)$ is a fully faithful functor of ordinary categories. This in turn follows from the fact that $\Hom_{\operatorname{ho}(\Cc)}(x,y)\cong \pi_0\Hom_\Cc(x,y)$, which is straightforward to check from \cref{par:HomotopyCategory}.
\end{rem}
To prove the \enquote{if} part of \cref{thm:EquivalenceFullyFaithfulEssentiallySurjective}, let's first consider the case where $\Cc$ and $\Dd$ are animae.
\begin{lem}\label{lem:FullyFaithfulAnimae}
	Let $F\colon \Cc\rightarrow\Dd$ be a fully faithful and essentially surjective functor of animae. Then $F$ is a homotopy equivalence.
\end{lem}
\begin{proof}
	Since $\Cc$ and $\Dd$ are animae, we have $\Cc=\core(\Cc)$ and $\Dd=\core(\Dd)$. By \cref{rem:FullyFaithfulImpliesInjectiveOnPi0Core} and the fact that $F$ is essentially surjective, we see that $\Cc\rightarrow\Dd$ is a bijection on path components. Hence we may assume without loss of generality that $\Cc$ and $\Dd$ are connected. Now choose $x\in\Cc$ and observe that
	\begin{equation*}
		\pi_{n+1}(\Cc,x)\cong \pi_n\bigl(\Hom_\Cc(x,x),\id_x\bigr)
	\end{equation*}
	for all $n\geqslant 0$. Indeed, by the pullback square from \cref{par:HomInQuasiCategories}, a map $(\square^n,\partial\square^n)\rightarrow (\Hom_\Cc(x,x),\id_x)$ is equivalently a morphism $\square^n\times\Delta^1\rightarrow \Cc$ such that $\partial \square^n\times\Delta^1\cup\square^n\times\{0,1\}\rightarrow \Cc$ is constant on $x$. But that's just a map $(\square^{n+1},\partial\square^{n+1})\rightarrow(\Cc,x)$, as claimed.
	
	Hence $F$ being fully faithful implies that $\pi_{n+1}(\Cc,x)\cong\pi_{n+1}(\Dd,F(x))$ is an isomorphism for all $n\geqslant 0$. But then $F$ is a homotopy equivalence by \cref{thm:Whitehead}.
\end{proof}
Furthermore we need:
\begin{lem}\label{lem:NonFullSubcategory}
	Let $\Cc$ be a quasi-category and let $\Cc[S_1]\subseteq \Cc$ be a \embrace{not necessarily full} sub-quasi-category spanned by a collection $S_1\subseteq \Cc_1$ of morphisms as in \cref{par:SubQuasiCategories}. Then for all $x,y\in\Cc$
	\begin{equation*}
		\Hom_{\Cc[S_1]}(x,y)\longrightarrow \Hom_\Cc(x,y)
	\end{equation*}
	is an equivalence onto the set of path components of morphisms from $S_1$.
\end{lem}
\begin{proof}[Proof sketch]
	By unravelling \cref{par:HomInQuasiCategories}, an $n$-simplex $\Delta^n\rightarrow \Hom_\Cc(x,y)$ is the same as a map $\sigma\colon \Delta^1\times\Delta^n\rightarrow \Cc$ such that $\sigma|_{\{0\}\times \Delta^n}=\sigma|_{\{1\}\times\Delta^n}=\const x$. Then $\sigma$ defines an $n$-simplex $\Delta^n\rightarrow \Hom_{\Cc[S_1]}(x,y)$ if and only if $\sigma$ maps all morphisms in $\Delta^1\times\Delta^n$ to $S_1$. Note that all morphisms in $\{0\}\times\Delta^n$ and $\{1\}\times\Delta^n$ are mapped to $\id_x$, which is contained $S_1$ because we assume that $S_1$ contains all identities. So it suffices to check that $\sigma|_{\Delta^1\times\{i\}}\colon \Delta^1\times\{i\}\rightarrow \Cc$ is contained in $S_1$ for all $i=0,\dotsc,n$, because all other morphisms in $\Delta^1\times\Delta^n$ are generated under compositions by these as well as the morphisms in $\{0\}\times \Delta^n$ and $\{1\}\times\Delta^n$. This means that an $n$-simplex $\Delta^n\rightarrow \Hom_\Cc(x,y)$ belongs to $\Hom_{\Cc[S_1]}(x,y)$ if and only if all its vertices correspond to morphisms in $S_1$. In other words, $\Hom_{\Cc[S_1]}(x,y)\subseteq\Hom_\Cc(x,y)$ is the collection of path components of morphisms from $S_1$, as desired.
\end{proof}
\begin{proof}[Proof sketch of \cref{thm:EquivalenceFullyFaithfulEssentiallySurjective}]
	Assume $F$ is fully faithful and essentially surjective. We'll show that $\core \F(K,\Cc)\rightarrow\core\F(K,\Dd)$ is a homotopy equivalence of animae for all simplicial sets $K$ (note that both $\F(K,\Cc)$ and $\F(K,\Dd)$ are indeed quasi-categories by \cref{cor:FIsKanComplex}). Once we have this, plugging in $K=\Dd$ yields a functor $G\colon \Dd\rightarrow\Cc$ with an equivalence $G\circ F\simeq\id_\Cc$. It's straightforward to see that $G$ is again fully faithful and essentially surjective, so repeating the argument with $G$ shows that $G$ has a left inverse too. Then $G$ must be an equivalence and so its right inverse $F$ must be an equivalence too.
	
	\emph{Case $K=*$.} Since $\core \F(*,\Cc)\cong \core(\Cc)$, we must show that $\core(\Cc)\rightarrow\core(\Dd)$ is a homotopy equivalence of animae. By \cref{lem:NonFullSubcategory}, $\Hom_{\core (\Cc)}(x,y)\rightarrow \Hom_\Cc(x,y)$ is an equivalence onto those path components that correspond to equivalences from $x$ to $y$. The same is true for $\Dd$, whence $\core (\Cc)\rightarrow\core(\Dd)$ is fully faithful again. Clearly, it is essentially surjective too, so \cref{lem:FullyFaithfulAnimae} shows that $\core (\Cc)\rightarrow\core(\Dd)$ must be a homotopy equivalence.
	
	\emph{Case $K=\Delta^n$, $n\geqslant1$.} Let $I^n\coloneqq\bigcup_{i=0}^{n-1}\Delta^{\{i-1,i\}}\subseteq \Delta^n$. It's straightforward to check that $I^n\rightarrow\Delta^n$ is inner anodyne, so $\F(\Delta^n,\Cc)\rightarrow\F(I^n,\Cc)$ is a trivial fibration by \cref{cor:FKanFibration}. The same is true for $\Dd$. We may thus replace $K=\Delta^n$ by $K=I^n$. Now we claim:
	\begin{alphanumerate}\itshape
		\item[\boxtimes] If $i\colon A\rightarrow B$ is a cofibration of simplicial sets, then $i^*\colon\core \F(B,\Cc)\rightarrow\core\F(A,\Cc)$ is a Kan fibration. Furthermore, for all $x_0,x_1,\dotsc,x_n\in\Cc$, the following diagram is a pullback diagram of Kan complexes and its vertical arrows are Kan fibrations:\label{claim:Pullback}
		\begin{equation*}
			\begin{tikzcd}
				\Hom_\Cc(x_0,x_1)\times\dotsb \times\Hom_\Cc(x_{n-1},x_n)\dar\rar\drar[pullback] & \core \F(I^n,\Cc)\dar\\
				\{x_0\}\times\dotsb\times\{x_n\}\rar & \core (\Cc)\times\dotsb\times\core (\Cc)
			\end{tikzcd}
		\end{equation*}
	\end{alphanumerate}
	We know that $F$ induces a homotopy equivalence $(\core (\Cc))^{n+1}\simeq(\core(\Dd))^{n+1}$ by the case $K=*$. Furthermore, since we assume $F$ to be fully faithful, we know that $F$ induces homotopy equivalences $\prod_{j=1}^n\Hom_\Cc(x_{j-1},x_j)\simeq \prod_{j=1}^n\Hom_\Dd(F(x_{j-1}),F(x_j))$. So if we believe \cref{claim:Pullback} (and its analogue for $\Dd$), then \cref{lem:FibrationSequence} plus the five lemma (plus \cref{rem:ExactnessInLowDegrees}) show that $\core \F(I^n,\Cc)\rightarrow\core \F(I^n,\Dd)$ induces a bijection on $\pi_0$ and isomorphisms on $\pi_n$ for all basepoints and all $n\geqslant1$. Hence $\core \F(I^n,\Cc)\rightarrow\core \F(I^n,\Dd)$ must be a homotopy equivalence by \cref{thm:Whitehead}.
	
	To prove \cref{claim:Pullback}, first note that $\F(B,\Cc)\rightarrow \F(A,\Cc)$ is an inner fibration by \cref{cor:FKanFibration}. Furthermore, if $m\geqslant 2$ and $\sigma\colon\Lambda_j^m\rightarrow \core \F(B,\Cc)$ is any $m$-dimensional horn (we allow $j=0$ or $j=m$), then any $m$-simplex $\ov\sigma\colon \Delta^m\rightarrow \F(B,\Cc)$ with $\ov\sigma|_{\Lambda_j^m}=\sigma$ is already contained in $\core \F(B,\Cc)$. Indeed, equivalences in $\Cc$ are closed under $2$-out-of-$3$, hence the edges of $\Delta^m\smallsetminus\Lambda_j^m$ will automatically be mapped to equivalences too. This observation immediately shows that $i^*\colon \core\F(B,\Cc)\rightarrow \core\F(A,\Cc)$ is an inner fibration again. Furthermore, \cref{thm:JoyalLifting} shows that $i^*$ has lifting against $\Lambda_0^m\rightarrow\Delta^m$ and $\Lambda_m^m\rightarrow \Delta^m$ for all $m\geqslant 2$. It remains to deal with the case $m=1$, that is, to show lifting agains $\{0\}\rightarrow\Delta^1$ and $\{1\}\rightarrow\Delta^1$. Let's sketch how to prove the former; the latter is analogous. Building $B$ from $A$ by successively attaching simplices, we can reduce to the case where $i\colon \partial \Delta^k\rightarrow\Delta^k$ is a simplex boundary inclusion. The case $k=0$ is trivial. For $k\geqslant 1$, the map $\partial\Delta^k\rightarrow\Delta^k$ is bijective on $0$-simplices and so it suffices to show that any extension problem
	\begin{equation*}
		\begin{tikzcd}
			\partial\Delta^k\times\Delta^1\sqcup_{\partial\Delta^k\times\{0\}}\Delta^k\times\{0\}\rar["\sigma"]\dar & \Cc\\
			\Delta^k\times\Delta^1\urar[dashed,"\ov\sigma"'] & 
		\end{tikzcd}
	\end{equation*}
	in which $\sigma|_{\{j\}\times\Delta^1}\colon \{j\}\times\Delta^1\rightarrow\Cc$ is an equivalence in $\Cc$ for every $0$-simplex $j\in(\partial\Delta^k)_0$, admits a solution. Indeed, it follows from \cref{thm:EquivalencePointwise} that any extension $\ov\sigma\colon \Delta^k\times\Delta^1\rightarrow\Cc$ will automatically define a map $\Delta^1\rightarrow\core\F(\Delta^k,\Cc)$. To construct the desired extension, write it as a sequence of horn filling problems; each inner horn can be filled by \cref{def:QuasiCategory} and each outer horn by \cref{thm:JoyalLifting}. As usual, we skip the horn filling combinatorics. This finishes the proof that $i^*\colon \core\F(B,\Cc)\rightarrow\F(A,\Cc)$ is a Kan fibration.
	
	Choosing $i$ to be the cofibration $\{0\}\sqcup\dotsb\sqcup\{n\}\rightarrow I^n$, we see that $\core\F(I^n,\Cc)\rightarrow (\core(\Cc))^{n+1}$ is indeed a Kan fibration. It remains to show that we get a pullback diagram. We can write $I^n$ as an iterated pushout $I^n\cong \Delta^{\{0,1\}}\sqcup_{\{1\}}\dotsb\sqcup_{\{n-1\}}\Delta^{\{n-1,n\}}$ and thus $\F(I^n,\Cc)$ as an iterated pullback $\F(I^n,\Cc)\cong \Ar(\Cc)\times_{t,\Cc,s}\dotsb\times_{t,\Cc,s}\Ar(\Cc)$. Plugging in the definition of $\Hom_\Cc(x_{i-1},x_i)$ from \cref{par:HomInQuasiCategories} yields the desired pullback diagram---except for one problem: The left vertical arrow reads $\F(I^n,\Cc)\rightarrow\Cc^{n+1}$ instead of $\core \F(I^n,\Cc)\rightarrow (\core(\Cc))^{n+1}$. To get $\core$ into the picture, observe that as a consequence of \cref{cor:AnimaKanComplexes}, $\core\colon \cat{QCat}\rightarrow\cat{Kan}$ is a right adjoint to the inclusion $\cat{Kan}\subseteq\cat{QCat}$. Hence $\core$ turns pullbacks in $\cat{QCat}$ into pullbacks in $\cat{Kan}$. However, the pullbacks at hand are supposed to be taken in $\cat{sSet}$, and in general it's not true that pullbacks in $\cat{QCat}$ or $\cat{Kan}$ coincide with those in $\cat{sSet}$.\footnote{It's not even true that pullbacks always exist in $\cat{QCat}$ and $\cat{Kan}$.} But if a pullback of quasi-categories, taken in $\cat{sSet}$, happens to be a quasi-category again, then it's automatically a pullback in $\cat{QCat}$ too, and likewise for a pullback of Kan complexes that happens to be Kan again. Since we've seen that $\F(I^n,\Cc)\rightarrow\Cc^{n+1}$ is an inner fibration and $\core \F(I^n,\Cc)\rightarrow(\core (\Cc))^{n+1}$ is a Kan fibration, this is is true in our situation. So $\core$ preserves the pullback at hand and we conclude that
	\begin{equation*}
		\begin{tikzcd}
			\core\bigl(\Hom_\Cc(x_0,x_1)\times\dotsb \times\Hom_\Cc(x_{n-1},x_n)\bigr)\dar\rar\drar[pullback] & \core \F(I^n,\Cc)\dar\\
			\core\bigl(\{x_0\}\times\dotsb\times\{x_n\}\bigr)\rar & \core (\Cc)\times\dotsb\times\core (\Cc)
		\end{tikzcd}
	\end{equation*}
	is a pullback of simplicial sets. But $\{x_0\}\times\dotsb\times\{x_n\}$ and $\Hom_\Cc(x_0,x_1)\times\dotsb\times\Hom_\Cc(x_{n-1},x_n)$ are Kan complexes (the latter by \cref{cor:HomAnima}), hence coincide with their cores. This shows that we get a pullback as desired, thus finishing the proof of \cref{claim:Pullback} and the case $K=\Delta^n$.
	
	\emph{Case $K$ is finite-dimensional.} A simplicial set $K$ is called \emph{finite-dimensional} if it has non-degenerate simplices in only finitely many degrees. We use induction on maximal dimension $d$ of a non-degenerate simplex.  The case $d=0$ follows from the case $K=\Delta^0$ above. For the inductive step, we can write a $(d+1)$-dimensional simplicial set $K$ as a pushout of some $d$-dimensional simplicial set along a disjoint union $\coprod\partial\Delta^{d+1}\rightarrow\coprod\Delta^{d+1}$ of simplex boundary inclusions. Accordingly, $\F(K,\Cc)$ and $\F(K,\Dd)$ can be written as pullbacks. By arguments as in \cref{claim:Pullback}, we still get pullbacks after applying $\core$ and the legs $\core \F\left(\coprod\Delta^{d+1},\Cc\right)\rightarrow\F\left(\coprod\partial\Delta^{d+1},\Cc\right)$ and $\core \F\left(\coprod\Delta^{d+1},\Dd\right)\rightarrow\F\left(\coprod\partial\Delta^{d+1},\Dd\right)$ are Kan fibrations. Using the inductive hypothesis and the case $K=\Delta^{d+1}$ together with \cref{lem:FibrationSequence} and the five lemma (plus \cref{rem:ExactnessInLowDegrees}), we see that $\core\F(K,\Cc)\rightarrow\core\F(K,\Dd)$ induces a bijection on $\pi_0$ and isomorphisms on $\pi_n$ for all basepoints and all $n\geqslant 1$. Hence $\core\F(K,\Cc)\rightarrow\core\F(K,\Dd)$ must be a homotopy equivalence by \cref{thm:Whitehead}.
	
	\emph{General case.} Write $K\cong \colimit_{d\geqslant 0}\operatorname{sk}_dK$, where $\operatorname{sk}_dK$ is the \emph{$d$-skeleton of $K$}. It is defined as the left Kan extension 
	\begin{equation*}
		\operatorname{sk}_dK\coloneqq \Lan_{\IDelta_{\leqslant d}^\op\rightarrow\IDelta^\op}\left(K|_{\IDelta_{\leqslant d}^\op}\right)\,,
	\end{equation*}
	where $\IDelta_{\leqslant d}^\op\subseteq\IDelta^\op$ is the full subcategory spanned by $[0],\dotsc,[d]$. It's straightforward to see, using the Kan extension formula from \cref{lem:1KanExtensionFormula}, that $\operatorname{sk}_dK$ is $d$-dimensional and the transition maps $\operatorname{sk}_dK\rightarrow\operatorname{sk}_{d+1}K$ are cofibrations. By the finite-dimensional case, $F$ induces equivalences $\core \F(\operatorname{sk}_dK,\Cc)\overset{\simeq}{\longrightarrow}\core\F(\operatorname{sk}_dK,\Dd)$ for all $d\geqslant 0$.
	
	By the colimit above, $\F(K,\Cc)\cong\limit_{d\geqslant 0}\F(\operatorname{sk}_dK,\Cc)$. This limit is preserved by $\core$. Indeed, \cref{claim:Pullback} shows that $\core \F(\operatorname{sk}_{d+1}K,\Cc)\rightarrow \core \F(\operatorname{sk}_dK,\Cc)$ is a Kan fibration and we can apply an argument as above. The same applies to $\Dd$ instead of $\Cc$. So it remains to see that equivalences of Kan complexes are preserved under limits along Kan fibrations. This can be shown using a Milnor sequence for homotopy groups, for example, or by hand, using a straightforward, but technical argument. See \cite[Lemma~VII.12]{HigherCatsI} for example.
\end{proof}

\subsection{Localisations of \texorpdfstring{$\infty$}{Infinity}-categories}
\begin{con}\label{con:Localisation}
	Let $\Cc$ be a quasi-category and $W\subseteq \Cc_1$ a subset of morphisms. We wish to construct the localisation $\Cc\rightarrow\Cc[W^{-1}]$, that is, the universal functor of quasi-categories that sends the morphisms from $W$ to equivalences. To do so, consider the the pushout
	\begin{equation*}
		\begin{tikzcd}
			\coprod_W\Delta^1\rar\dar\drar[pushout] & \Cc\dar\\
			\coprod_W\N(J)\rar & \ov\Cc
		\end{tikzcd}
	\end{equation*}
	in simplicial sets, where $J\coloneqq \{\InlineJ\}$ is the \enquote{free-living isomorphism}, the category with two objects and a pair of mutually inverse isomorphisms between them. By \cref{lem:SmallObjectArgument}, we can choose an inner anodyne map $\ov\Cc\rightarrow\Cc[W^{-1}]$ into a quasi-category. We call the composition $p\colon \Cc\rightarrow\Cc[W^{-1}]$ the \emph{localisation of $\Cc$ at $W$}. We'll check in a moment that $p$ is independent of the choices (up to equivalence), so the definite article is justified.
\end{con}
\begin{lem}\label{lem:Localisation}
	For every quasi-category $\Dd$, the functor $p\colon \Cc\rightarrow\Cc[W^{-1}]$ from \cref{con:Localisation} above induces an equivalence
	\begin{equation*}
		p^*\colon \Hom_{\cat{Cat}_\infty}\bigl(\Cc[W^{-1}],\Dd\bigr)\overset{\simeq}{\longrightarrow}\Hom_{\cat{Cat}_\infty}^W(\Cc,\Dd)\subseteq \Hom_{\cat{Cat}_\infty}(\Cc,\Dd)\,,
	\end{equation*}
	where $\Hom_{\cat{Cat}_\infty}^W(\Cc,\Dd)\subseteq \Hom_{\cat{Cat}_\infty}(\Cc,\Dd)$ is the collection of path components of those functors $F\colon \Cc\rightarrow\Dd$ that send $W$ to equivalences in $\Dd$.
\end{lem}
\begin{proof}
	Let $\F^W(\Cc,\Dd)\subseteq\F(\Cc,\Dd)$ be the full sub-quasi-category (as in \cref{par:SubQuasiCategories}) spanned by those functors $F\colon \Cc\rightarrow\Dd$ that send $W$ to equivalences in $\Dd$. We know from \cref{thm:CordierPorter} that $\Hom_{\cat{Cat}_\infty}(\Cc[W^{-1}],\Dd)\simeq \core\F(\Cc[W^{-1}],\Dd)$ and $\Hom_{\cat{Cat}_\infty}(\Cc,\Dd)\simeq \core\F(\Cc,\Dd)$; it's then straightforward to check that
	\begin{equation*}
		\Hom_{\cat{Cat}_\infty}^W(\Cc,\Dd)\simeq \core\F^W(\Cc,\Dd)\,.
	\end{equation*}
	Since $\ov\Cc\rightarrow\Cc[W^{-1}]$ is inner anodyne by \cref{con:Localisation}, $\F(\Cc[W^{-1}],\Dd)\rightarrow\F(\ov\Cc,\Dd)$ is a trivial fibration by \cref{cor:FKanFibration}.
	
	We'll show that $\core\F(\ov\Cc,\Dd)\rightarrow\core\F^W(\Cc,\Dd)$ is a trivial fibration too to finish the proof. This is straightforward, but a little annoying thanks to technicalities. The pushout from \cref{con:Localisation} shows that
	\begin{equation*}
		\begin{tikzcd}
			\F(\ov\Cc,\Dd)\rar\dar\drar[pullback] & \F(\Cc,\Dd)\dar\\
			\prod_W\F\bigl(\N(J),\Dd\bigr)\rar & \prod_W\F\bigl(\Delta^1,\Dd\bigr)
		\end{tikzcd}
	\end{equation*}
	is a pullback of simplicial sets. Note that $\F(\ov\Cc,\Dd)\rightarrow \F(\Cc,\Dd)$ factors through the full sub-quasi-category $\F^W(\Cc,\Dd)\subseteq \F(\Cc,\Dd)$ and $\F(\N(J),\Dd)\rightarrow\F(\Delta^1,\Dd)$ factors through the full sub-quasi-category $\F^{\{0\rightarrow 1\}}(\Delta^1,\Dd)\subseteq \F(\Delta^1,\Dd)$. Since pullbacks behave well under passing to sub-simplicial sets, the following diagram is a pullback too:
	\begin{equation*}
		\begin{tikzcd}
			\F(\ov\Cc,\Dd)\rar\dar\drar[pullback] & \F^W(\Cc,\Dd)\dar\\
			\prod_W\F\bigl(\N(J),\Dd\bigr)\rar & \prod_W\F^{\{0\rightarrow 1\}}\bigl(\Delta^1,\Dd\bigr)
		\end{tikzcd}
	\end{equation*}
	To finish the proof, it's enough to show the following two claims:
	\begin{alphanumerate}\itshape
		\item[\boxtimes_1] The pullback above stays a pullback after applying $\core$ everywhere.\label{claim:PullbackAfterCore}
		\item[\boxtimes_2] The map $\core \F(\N(J),\Dd)\rightarrow \core\F^{\{0\rightarrow 1\}}(\Delta^1,\Dd)$ is a trivial fibration.\label{claim:TrivialFibration}
	\end{alphanumerate}
	To prove \cref{claim:PullbackAfterCore}, observe that $\F(\N(J),\Dd)\rightarrow \F(\Delta^1,\Dd)$ is an inner fibration by \cref{cor:FKanFibration} and $\core \F(\N(J),\Dd)\rightarrow \core \F(\Delta^1,\Dd)$ is a Kan fibration by claim~\cref{claim:Pullback} in the proof of \cref{thm:EquivalenceFullyFaithfulEssentiallySurjective}. By an argument similar to \cref{lem:NonFullSubcategory}, the fact that $\F^{\{0\rightarrow 1\}}(\Delta^1,\Dd)\subseteq \F(\Delta^1,\Dd)$ is a full sub-quasi-category implies that $\core\F^{\{0\rightarrow 1\}}(\Delta^1,\Dd)\subseteq \core\F(\Delta^1,\Dd)$ is a collection of path components. By inspection, this means that $\F(\N(J),\Dd)\rightarrow \F^{\{0\rightarrow 1\}}(\Delta^1,\Dd)$ must be an inner fibration too and $\core \F(\N(J),\Dd)\rightarrow \core \F^{\{0\rightarrow 1\}}(\Delta^1,\Dd)$ must be a Kan fibration too. By the same argument as in the proof of \cref{thm:EquivalenceFullyFaithfulEssentiallySurjective} it follows that $\core$ preserves the pullback, as required.
	
	To prove \cref{claim:TrivialFibration}, we claim $\core \F(\N(J),\Dd)\cong \core\F(\N(J),\core(\Dd))$. Indeed, an $n$-simplex $\Delta^n\rightarrow\core \F(\N(J),\Dd)$ is the same as an $n$-simplex $\Delta^n\rightarrow \F(\N(J),\Dd)$ all of whose edges are mapped to equivalences; by \cref{par:FInternalHom} and \cref{thm:EquivalencePointwise}, that's the same as a map $\sigma\colon \N(J)\times \Delta^n\rightarrow \Dd$ such that $\sigma|_{\{x\}\times \Delta^{\{i,j\}}}\colon \{x\}\times \Delta^{\{i,j\}}\rightarrow \Dd$ maps to an equivalence for all $x\in \N(J)$ and all edges $\Delta^{\{i,j\}}\subseteq \Delta^n$. But then all morphisms in $\N(J)\times \Delta^n$ must be mapped to equivalences, because every morphism in $\N(J)$ is already an equivalence. So $\sigma$ necessarily factors through $\core(\Dd)\subseteq \Dd$, as desired.
	
	Since $\core(\Dd)$ is a Kan complex by \cref{cor:AnimaKanComplexes}, $\F(\N(J),\core(\Dd))$ is a Kan complex by \cref{cor:FIsKanComplex} and so $\core\F(\N(J),\core(\Dd))\cong \F(\N(J),\core(\Dd))$. By analogous arguments, we get isomorphisms $\core\F^{\{0\rightarrow 1\}}(\Delta^1,\Dd)\cong \core\F^{\{0\rightarrow 1\}}(\Delta^1,\core (\Dd))\cong \F(\Delta^1,\core(\Dd))$. Now $\Delta^1\rightarrow \N(J)$ is anodyne (in fact, both left and right anodyne) by an explicit horn filling argument. Hence $\F(\N(J),\core(\Dd))\rightarrow\F(\Delta^1,\core(\Dd))$ is a trivial fibration by \cref{cor:FKanFibration}. This finishes the proof of \cref{claim:TrivialFibration} and we are done.
\end{proof}
\begin{numpar}[Corollary/Warning.]\label{cor:Localisation}\itshape
	If $\Cc$ is a \embrace{small} ordinary category and $W$ a collection of morphisms in $\Cc$, then the localisation $\N(\Cc)[W^{-1}]$ from \cref{con:Localisation} is not necessarily the nerve of an ordinary category. But the homotopy category $\operatorname{ho}(\N(\Cc)[W^{-1}])$ is equivalent to the localisation of $\Cc$ at $W$ in the world of ordinary categories.
\end{numpar}
\begin{proof}[Proof sketch]
	For counterexamples see \cref{thm:AnAsALocalisation} or the discussion in \cref{con:DerivedCategoryI} below. The assertion about $\operatorname{ho}(\N(\Cc)[W^{-1}])$ follows easily from a combination of \cref{lem:Localisation} and \cref{lem:SimplicialHoNerveAdjunction} as well as the universal property of localisations in ordinary category theory.
\end{proof}

So localisations provide another way to construct non-trivial examples of quasi-categories. In fact, both $\cat{An}$ and $\cat{Cat}_\infty$ can be constructed in this way:
\begin{thm}\label{thm:AnAsALocalisation}
	If $\cat{Kan}^\Delta$ and $\cat{QCat}^\Delta$ are the Kan-enriched categories from \cref{exm:SimplicialNerve}, then there are canonical equivalences of quasi-categories
	\begin{align*}
		\N(\cat{Kan})\left[\{\text{homotopy equivalences}\}^{-1}\right]&\overset{\simeq }{\longrightarrow}\N^\Delta(\cat{Kan}^\Delta)=\cat{An}\,,\\
		\N(\cat{QCat})\left[\{\text{equivalences of quasi-categories}\}^{-1}\right]&\overset{\simeq }{\longrightarrow}\N^\Delta(\cat{QCat}^\Delta)=\cat{Cat}_\infty\,.\equationblackbox
	\end{align*}
\end{thm}
\begin{rem}\label{rem:SimplicialModelCategory}
	It's not hard to construct the functors in \cref{thm:AnAsALocalisation}: By a direct inspection of their constructions, we can build a map $\N(\cat{Kan})\rightarrow\N^\Delta(\cat{Kan}^\Delta)$ of simplicial sets (or rather simplicial \emph{classes}, but we'll ignore the set-theoretic difficulties). Using \cref{lem:Localisation}, we only need to check that this map sends homotopy equivalences in $\cat{Kan}$ to equivalences in $\N^\Delta(\cat{Kan}^\Delta)$, which is clear from the unravelling in \cref{exm:SimplicialNerve}. An analogous argument works of course for $\N^\Delta(\cat{QCat}^\Delta)$.
	
	However, proving that these functors are equivalences is not easy. There is a general notion of \emph{simplicial model categories}: These are model categories $\Aa$ (\cref{def:ModelCategory}) together with a simplicial enrichment $\Aa^\Delta$ that interacts with the model structure in a certain way. One can show that the model structures on $\cat{sSet}$ from \cref{exm:KanQuillenModelStructure,exm:JoyalModelStructure} can be made into simplicial model structures. In general, if $\Aa$ is a simplicial model category and $\Aa^\mathrm{cf}\subseteq \Aa$, $(\cat{\Aa}^\Delta)^\mathrm{cf}\subseteq \Aa^\Delta$ is the full subcategory respectively the full sub-simplicially enriched category spanned by the bifibrant objects, there is an equivalence of quasi-categories
	\begin{equation*}
		\N(\Aa^\mathrm{cf})\left[\{\text{weak equivalences}\}^{-1}\right]\overset{\simeq}{\longrightarrow}\N^\Delta\bigl((\Aa^\Delta)^\mathrm{cf}\bigr)\,.
	\end{equation*}
	A proof can be found in \cite[Theorem~\HAthm{1.3.4.20}]{HA}.
\end{rem}
\begin{rem}\label{rem:ModelCategoryUnderlyingInftyCategory}
	For a general model category $\Aa$, it's customary to call 
	\begin{equation*}
		\Aa_\infty\coloneqq \N(\Aa^\mathrm{cf})\left[\{\text{weak equivalences}\}^{-1}\right]
	\end{equation*}
	the \emph{underlying quasi-category of $\Aa$}. Its homotopy category $\operatorname{ho}(\Aa_\infty)$ is called the \emph{homotopy category of $\Aa$}. By Corollary/Warning~\cref{cor:Localisation}, this agrees with the ordinary localisation of $\Aa^\mathrm{cf}$ at the weak equivalences. Furthermore, if the factorisations from \cref{def:ModelCategory}\cref{enum:ModelCategoryFactorisations} can be chosen functorially (which is always the case in practice), then $\Aa_\infty$ could be obtained equally well by inverting the weak equivalences in either of $\Aa^\mathrm{c}$, $\Aa^\mathrm{f}$, or $\Aa$ itself, where $\Aa^\mathrm{c},\Aa^\mathrm{f}\subseteq \Aa$ denote the full subcategories spanned by the cofibrant or fibrant objects, respectively. So in practice, all possible alternative definitions of $\Aa_\infty$ agree. 
	
	We'll sketch the argument why $\N(\Aa^\mathrm{c})\rightarrow \N(\Aa)$ becomes an equivalences after localisation at all weak equivalences; the other cases are entirely analogous. Let's assume that $\Aa$ has functorial cofibrant replacements. That is, for $x\in \Aa$ the map from the initial object to $x$ factors functorially through a trivial fibration $\eta_x\colon c(x)\rightarrow x$, where $c(x)$ is cofibrant. Then we get a natural transformation $\eta\colon c(-)\Rightarrow \id_{\N(\Aa)}$ of endofunctors of $\N(\Aa)$. Using \cref{lem:Localisation}, we can show that this natural transformation passes to the localisation at all weak equivalences.\footnote{Here's the full argument: Put $W\coloneqq\{\text{weak equivalences}\}$ for short; we wish to construct a natural transformation $\N(\Aa)[W^{-1}]\times \Delta^1\rightarrow \N(\Aa)[W^{-1}]$. It's clear from the construction that $\N(\Aa)[W^{-1}]\times\Delta^1$ can also be described as the localisation of $\N(\Aa)\times \Delta^1$ at $W\times\{\id_0\}\cup W\times\{\id_1\}$. Thus, by \cref{lem:Localisation}, it's enough to provide a natural transformation $\N(\Aa)\times\Delta^1\rightarrow \N(\Aa)[W^{-1}]$ that sends these morphisms to equivalences. Now the composition of $\eta\colon \N(\Aa)\times\Delta^1\rightarrow \N(\Aa)$ with the localisation functor $\N(\Aa)\rightarrow \N(\Aa)[W^{-1}]$ does just that.} After the localisation, $\eta$ becomes an equivalence of endofunctors by \cref{thm:EquivalencePointwise}. Then $\N(\Aa^\mathrm{c})\rightarrow \N(\Aa)$ and $c\colon \N(\Aa)\rightarrow \N(\Aa^\mathrm{c})$ become equivalences of quasi-categories after localisation at all weak equivalences, as desired. 
	
	
	
	
	In general, it's hard to describe morphisms in any localisation. However, if $\Aa$ is a simplicial model category, then \cref{rem:SimplicialModelCategory} and \cref{thm:CordierPorter} provide convenient access to the $\Hom$ animae in $\Aa_\infty$. 
\end{rem}



\newpage
\section{Lurie's straightening equivalence}\label{sec:Straightening}
We've seen in~\cref{par:HomInQuasiCategories} how to construct the $\Hom$ animae $\Hom_\Cc(x,y)$ in a quasi-category $\Cc$. But we never explained how to assemble these values into a functor $\Hom_\Cc\colon \Cc^\op\times\Cc\rightarrow\cat{An}$ (where $\Cc^\op$ is as in \cref{par:Opposite}). In this section, we give such a construction and prove the Yoneda lemma. To do this, we'll use Lurie's \emph{straightening/unstraightening equivalence}, which deals with the problem of constructing functors $F\colon \Cc\rightarrow\cat{An}$ and $F\colon \Cc\rightarrow\cat{Cat}_\infty$. In the end, straightening/unstraightening will not only allow us to prove Yoneda's lemma, but the statement itself will be indispensible for developing quasi-category theory as a higher analogue of ordinary category theory.

\subsection{Cocartesian fibrations and the straightening equivalence}\label{subsec:Straightening}
%\setcounter{theorem}{-1}
\begin{numpar}[Some informal motivation.]\label{par:StraighteningMotivation}
	Let's think about what a functor $F\colon \Cc\rightarrow\cat{Cat}_\infty$ looks like. Suppose $x,y\in \Cc$ are objects and $\alpha\colon x\rightarrow y$ is a morphism. Then $F(x)$, $F(y)$ will be quasi-categories and $F(\alpha)\colon F(x)\rightarrow F(y)$ will be a functor between them. So for every $u\in F(x)$, there will be an associated object $v\simeq F(\alpha)(u)\in F(y)$. In a picture, this could look as follows:
	\begin{center}
		%\vspace{-1ex}
		\begin{tikzpicture}[line cap=round, line join=round, line width=rule_thickness, decoration={markings,mark=at position 0.5 with {\arrow{to}}},scale=0.95]
			\begin{scope}[yshift=-3.5cm]
				\begin{scope}[xscale=0.64,xshift=-0.15cm]
					\draw[dashed,shift={(0,-0.25)},fill=white!93!black]%,preaction={pattern={Lines[xshift=-0.4em,angle=150, line width=0.2em, distance=0.4em]}}, pattern color=white!93!black]
					 (-0.8,0) to[out=35,in=270] (-0.6,0.2) to[out=90,in=305] (-0.8,0.6) to[out=125,in=180]  (-0.2,1) to[out=0,in=140] (0.4,0.65) to[out=320,in=100] (0.8,0.45) to[out=280,in=45] (0.6,0) to[out=225,in=90] (0.45,-0.3) to[out=270,in=135] (0.5,-0.6) to[out=315,in=90] (0.6,-0.85) to[out=270,in=70] (0.5,-1) to[out=250,in=0] (0.1,-1.2) to[out=180,in=315] (-0.6,-1) to[out=135,in=340] (-0.8,-0.7) to[out=160,in=270] (-0.9,-0.5) to[out=90,in=270] (-0.8,-0.25) to[out=90,in=205] cycle;
				\end{scope}
				\begin{scope}[xscale=0.64,xshift=2.1cm]
					\draw[dashed,shift={(3.5,-0.25)}, dash phase=3,fill=white!93!black]%preaction={pattern={Lines[xshift=-0.4em,angle=150, line width=0.2em, distance=0.4em]}}, pattern color=white!93!black]
					(-0.9,0) to[out=35,in=270] (-0.8,0.2) to[out=90,in=305] (-0.9,0.6) to[out=125,in=180]  (-0.2,1.1) to[out=0,in=140] (0.4,0.8) to[out=320,in=100] (0.85,0.6) to[out=280,in=45] (0.7,0) to[out=225,in=90] (0.6,-0.3) to[out=270,in=90] (0.65,-0.6) to[out=270,in=90] (0.75,-0.85) to[out=270,in=70] (0.65,-1.1) to[out=250,in=0] (0.1,-1.2) to[out=180,in=0] (-0.4,-1.3) to[out=180,in=315] (-0.8,-1) to[out=135,in=340] (-0.9,-0.7) to[out=160,in=270] (-1.1,-0.5) to[out=90,in=270] (-1,-0.25) to[out=90,in=205] cycle;
				\end{scope}
				\fill (0,0) circle (0.45ex) node[outer sep=0.5ex] (u) {} node[above left] {$u$};
				\fill (3.5,0) circle (0.45ex) node[outer sep=0.5ex] (v) {} node[above right] {$v$};
				\fill (3.5,-1) circle (0.45ex) node[outer sep=0.5ex] (w) {} node[below right] {$w$};
				\draw[postaction={decorate}] (v) to node[pos=0.5] (mid) {} (w);
				\draw[|-to] (u) to (v);
				\draw[-to] (0.4,-0.5) to node[pos=0.5,below,outer sep=0.25ex] {$\scriptstyle F(\alpha)$} (2.75,-0.5);
				\node at (-1.15,-0.5) {$F(x)$};
				\node at (4.55,-0.5) {$F(y)$};
			\end{scope}
			\draw[dotted] (-4,-5) to[out=90,in=270] (-3.5,-4) to[out=90,in=195] (-3,-3.6) to[out=15,in=290] (-2,-3) to[out=110,in=180] (0,-2.2) to[out=0,in=180] (1.75,-2) to[out=0,in=180] (3,-2.1) to[out=0,in=135] (4.75,-2.5) to[out=315,in=100] (5.4,-2.8) to[out=280,in=85] (5.2,-3.1) to[out=265,in=160] (5.4,-3.3) to[out=340,in=90] (6,-3.5) to[out=270,in=80] (5.7,-3.8) to[out=260,in=90] (6.6,-4.3) to[out=270,in=90] (7.5,-5);
			\node at (-2.9,-3.1) {$\cat{Cat}_\infty$};
			\begin{scope}[yshift=3.65cm]
				\begin{scope}[yscale=0.3,xscale=0.7,shift={(1.2,-12.7)}]
					\draw[dotted] (-3,0) to[out=90,in=270] (-2.4,0.5) to[out=90,in=300] (-2,1) to[out=120,in=270] (-2.2,1.6) to[out=90,in=180] (-0.8,2) to[out=0,in=180] (1,2.5) to[out=0,in=180] (1.95,2.2) to[out=0,in=135] (4.5,1.4) to[out=315,in=160] (4.7,0.8) to[out=340,in=90] (5.7,0) to[out=270,in=60] (4.9,-0.8) to[out=240,in=135] (5.2,-1.2) to[out=315,in=90] (4.7,-1.7) to[out=270,in=90] (4.8,-1.9) to[out=270,in=0] (3.6,-2.2) to[out=180,in=0] (2.75,-2.5) to[out=180,in=305] (0,-2.5) to[out=125,in=275] (-2,-1.75) to [out=95,in=270] (-2.5,-0.75) to[out=90,in=270] cycle;
				\end{scope}
				\node (Cc) at (-1.15,-3.4) {$\Cc$};
				\fill (0,-3.8) circle (0.45ex) node[outer sep=0.5ex] (x) {} node[left=0.5ex] (xlabel) {$x$};
				\fill (3.5,-3.8) circle (0.45ex) node[outer sep=0.5ex] (y) {} node[right=0.5ex] {$\smash{y}\vphantom{x}$};
				\draw[postaction={decorate}] (x) to node[pos=0.5,above,outer sep=0.25ex] (alpha) {$\scriptstyle\alpha$} (y);
				\draw[|-to,shorten >=4.5ex] (x) to (u);
				\draw[|-to,shorten >=4.5ex] (y) to (v);
			\end{scope}
			%\draw[very thick,-to] (alpha) ++ (0,-1) to ++(0,-2);
			%\draw[dotted] (-0.95,0.5) to[out=10,in=180] (-0.25,0.85) to[out=0,in=170] (0.25,0.7) to[out=350,in=190] (2,0.7) to[out=10,in=200] (3.15,0.85) to[out=20,in=180] (3.45,0.95) to[out=0,in=170] (4.2,0.75) to[out=350,in=120] (4.9,0.3) to[out=300,in=90] (5.5,-0.1) to[out=270,in=60] (5,-0.7) to[out=240,in=135] (5.2,-1) to[out=315,in=90] (4.9,-1.3) to[out=270,in=120] (4.95,-1.4) to[out=300,in=350] (3.74,-1.57) to[out=170,in=0] (3.33,-1.65) to[out=180,in=350] (3,-1.5) to[out=170,in=35] (2.2,-1.7) to[out=215,in=0] (0,-1.57) to[out=180,in=285] (-0.78,-1.25) to [out=105,in=270] (-1.2,-0.75) to[out=90,in=270] (-2,-0.2) to[out=90,in=270] (-1.5,0.2) to[out=90,in=300] (-1.3,0.4) to[out=120,in=190] cycle;
			%\begin{scope}[scale=0.9,shift={(0.4,-0.1)}]
			%\draw[dotted] (-3,0) to[out=90,in=270] (-2.4,0.5) to[out=90,in=300] (-2,1) to[out=120,in=270] (-2.2,1.6) to[out=90,in=180] (-0.8,2) to[out=0,in=180] (0,2.2) to[out=0,in=180] (1.75,2) to[out=0,in=135] (4.5,1.4) to[out=315,in=160] (4.7,0.8) to[out=340,in=90] (5.7,0) to[out=270,in=60] (4.9,-0.8) to[out=240,in=135] (5.2,-1.2) to[out=315,in=90] (4.7,-1.7) to[out=270,in=90] (4.8,-1.9) to[out=270,in=0] (3.6,-2.2) to[out=180,in=0] (2.75,-2.5) to[out=180,in=305] (0,-2.5) to[out=125,in=275] (-2,-1.25) to [out=95,in=270] (-2.5,-0.75) to[out=90,in=270] cycle;
			%\end{scope}
			%\draw[|-to] (0,-1.65) to (x);
			%\node (Uu) at (-2,0.75) {$\Uu$};
			%\draw[very thick,-to] (1.75,-2.1) to (alpha);
			%\draw[dashed, shorten <=1.25ex] (-0.9,-0.8) to (x);
			%\draw[dashed, shorten <=0.75ex,dash phase=1.5] (0.6,-1.1) to (x);
			%\draw[dashed,shorten <=1.5ex, dash phase=0] (2.595,-0.8) to (y);
			%\draw[dashed, shorten >=0.75ex,dash phase=0] (y) to (4.245,-1.15);
			%\draw[|-to] (3.5,-1.65) to (y);
		\end{tikzpicture}
	\end{center}
	Now let's turn this picture upside down! Take $F(x)$ and $F(y)$ and place them above $x$ and $y$, respectively. We would like to think of them as the fibres over $x$ and $y$ in some kind of fibration $p\colon \Uu\rightarrow \Cc$:
	\begin{center}
		\vspace{-1ex}
		\begin{tikzpicture}[line cap=round, line join=round, line width=rule_thickness, decoration={markings,mark=at position 0.5 with {\arrow{to}}},scale=0.95]
			\begin{scope}[xscale=0.64,xshift=-0.15cm]
				\draw[dashed,shift={(0,-0.25)},fill=white!93!black]%preaction={pattern={Lines[xshift=-0.4em,angle=150, line width=0.2em, distance=0.4em]}}, pattern color=white!93!black]
				(-0.8,0) to[out=35,in=270] (-0.6,0.2) to[out=90,in=305] (-0.8,0.6) to[out=125,in=180]  (-0.2,1) to[out=0,in=140] (0.4,0.65) to[out=320,in=100] (0.8,0.45) to[out=280,in=45] (0.6,0) to[out=225,in=90] (0.45,-0.3) to[out=270,in=135] (0.5,-0.6) to[out=315,in=90] (0.6,-0.85) to[out=270,in=70] (0.5,-1) to[out=250,in=0] (0.1,-1.2) to[out=180,in=315] (-0.6,-1) to[out=135,in=340] (-0.8,-0.7) to[out=160,in=270] (-0.9,-0.5) to[out=90,in=270] (-0.8,-0.25) to[out=90,in=205] cycle;
			\end{scope}
			\begin{scope}[xscale=0.64,xshift=2.1cm]
				\draw[dashed,shift={(3.5,-0.25)}, dash phase=3,fill=white!93!black]%,preaction={pattern={Lines[xshift=-0.4em,angle=150, line width=0.2em, distance=0.4em]}}, pattern color=white!93!black]
				(-0.9,0) to[out=35,in=270] (-0.8,0.2) to[out=90,in=305] (-0.9,0.6) to[out=125,in=180]  (-0.2,1.1) to[out=0,in=140] (0.4,0.8) to[out=320,in=100] (0.85,0.6) to[out=280,in=45] (0.7,0) to[out=225,in=90] (0.6,-0.3) to[out=270,in=90] (0.65,-0.6) to[out=270,in=90] (0.75,-0.85) to[out=270,in=70] (0.65,-1.1) to[out=250,in=0] (0.1,-1.2) to[out=180,in=0] (-0.4,-1.3) to[out=180,in=315] (-0.8,-1) to[out=135,in=340] (-0.9,-0.7) to[out=160,in=270] (-1.1,-0.5) to[out=90,in=270] (-1,-0.25) to[out=90,in=205] cycle;
			\end{scope}
			\fill (-0.1,0) circle (0.45ex) node[outer sep=0.5ex] (u) {} node[above left,xshift=0.4ex] {$u$};
			\fill (3.5,0) circle (0.45ex) node[outer sep=0.5ex] (v) {} node[above right] {$v$};
			\fill (3.5,-1) circle (0.45ex) node[outer sep=0.5ex] (w) {} node[below right] {$w$};
			\draw[postaction={decorate}] (u) to node[pos=0.5,above,outer sep=0.25ex] {$\scriptstyle\varphi$} (v);
			\draw[postaction={decorate}] (v) to node[pos=0.5] (mid) {} (w);
			\draw[postaction={decorate}] (u) to (w);
			\begin{scope}[yscale=0.3,xscale=0.7,shift={(1.2,-12.7)}]
				\draw[dotted] (-3,0) to[out=90,in=270] (-2.4,0.5) to[out=90,in=300] (-2,1) to[out=120,in=270] (-2.2,1.6) to[out=90,in=180] (-0.8,2) to[out=0,in=180] (1,2.5) to[out=0,in=180] (1.95,2.2) to[out=0,in=135] (4.5,1.4) to[out=315,in=160] (4.7,0.8) to[out=340,in=90] (5.7,0) to[out=270,in=60] (4.9,-0.8) to[out=240,in=135] (5.2,-1.2) to[out=315,in=90] (4.7,-1.7) to[out=270,in=90] (4.8,-1.9) to[out=270,in=0] (3.6,-2.2) to[out=180,in=0] (2.75,-2.5) to[out=180,in=305] (0,-2.5) to[out=125,in=275] (-2,-1.75) to [out=95,in=270] (-2.5,-0.75) to[out=90,in=270] cycle;
			\end{scope}
			\node at (0,1.25) {$F(x)$};
			\node at (3.5,1.25) {$F(y)$};
			\fill (-0.1,-3.8) circle (0.45ex) node[outer sep=0.5ex] (x) {} node[below=0.5ex] {$x$};
			\fill (3.5,-3.8) circle (0.45ex) node[outer sep=0.5ex] (y) {} node[below=0.5ex] {$y$};
			\draw[postaction={decorate}] (x) to node[pos=0.5,above,outer sep=0.25ex] (alpha) {$\scriptstyle\alpha$} (y);
			%\draw[dotted] (-0.95,0.5) to[out=10,in=180] (-0.25,0.85) to[out=0,in=170] (0.25,0.7) to[out=350,in=190] (2,0.7) to[out=10,in=200] (3.15,0.85) to[out=20,in=180] (3.45,0.95) to[out=0,in=175] (4.5,0.75) (4.6,-1.7) to[out=185,in=350] (3.74,-1.57) to[out=170,in=0] (3.33,-1.65) to[out=180,in=350] (3,-1.5) to[out=170,in=35] (2.2,-1.7) to[out=215,in=0] (0,-1.57) to[out=180,in=285] (-0.7,-1.25) to [out=105,in=340] (-1.35,-1);
			\draw[dotted] (-0.95,0.5) to[out=10,in=180] (-0.25,0.85) to[out=0,in=170] (0.25,0.7) to[out=350,in=190] (2,0.7) to[out=10,in=200] (3.15,0.85) to[out=20,in=180] (3.45,0.95) to[out=0,in=170] (4.2,0.75) to[out=350,in=120] (4.9,0.3) to[out=300,in=90] (5.5,-0.1) to[out=270,in=60] (5,-0.7) to[out=240,in=135] (5.2,-1) to[out=315,in=90] (4.9,-1.3) to[out=270,in=120] (4.95,-1.4) to[out=300,in=350] (3.74,-1.57) to[out=170,in=0] (3.33,-1.65) to[out=180,in=350] (3,-1.5) to[out=170,in=35] (2.2,-1.7) to[out=215,in=0] (0,-1.57) to[out=180,in=285] (-0.7,-1.25) to [out=105,in=270] (-1.2,-0.75) to[out=90,in=270] (-2,-0.2) to[out=90,in=270] (-1.5,0.2) to[out=90,in=300] (-1.3,0.4) to[out=120,in=190] cycle;
			%\begin{scope}[scale=0.9,shift={(0.4,-0.1)}]
			%\draw[dotted] (-3,0) to[out=90,in=270] (-2.4,0.5) to[out=90,in=300] (-2,1) to[out=120,in=270] (-2.2,1.6) to[out=90,in=180] (-0.8,2) to[out=0,in=180] (0,2.2) to[out=0,in=180] (1.75,2) to[out=0,in=135] (4.5,1.4) to[out=315,in=160] (4.7,0.8) to[out=340,in=90] (5.7,0) to[out=270,in=60] (4.9,-0.8) to[out=240,in=135] (5.2,-1.2) to[out=315,in=90] (4.7,-1.7) to[out=270,in=90] (4.8,-1.9) to[out=270,in=0] (3.6,-2.2) to[out=180,in=0] (2.75,-2.5) to[out=180,in=305] (0,-2.5) to[out=125,in=275] (-2,-1.25) to [out=95,in=270] (-2.5,-0.75) to[out=90,in=270] cycle;
			%\end{scope}
			%\draw[|-to] (0,-1.65) to (x);
			%\node (Uu) at (-2,0.75) {$\Uu$};
			\node (Uu) at (-1.7,0.55) {$\Uu$};
			\node (Cc) at (-1.15,-3.4) {$\Cc$};
			%\draw[very thick,-to] (1.75,-2.1) to (alpha);
			\draw[dashed, shorten >=0.25ex] (x) to (-0.675,-0.8);
			\draw[dashed] (x) to (0.29,-1.1);
			\draw[dashed,shorten >=0.25ex] (y) to (2.88,-0.8);
			\draw[dashed,shorten >=1.25ex] (y) to (4.06,-1.15);
			\path (u) to node[pos=0.75] {$\scriptscriptstyle/\!/\!/$} (mid);
			%\draw[dashed, shorten <=1.25ex] (-0.9,-0.8) to (x);
			%\draw[dashed, shorten <=0.75ex,dash phase=1.5] (0.6,-1.1) to (x);
			%\draw[dashed,shorten <=1.5ex, dash phase=0] (2.595,-0.8) to (y);
			%\draw[dashed, shorten >=0.75ex,dash phase=0] (y) to (4.245,-1.15);
			%\draw[|-to] (3.5,-1.65) to (y);
		\end{tikzpicture}
	\end{center}
	But, of course, $\Uu$ shouldn't just be a disjoint union of some fibres. Instead, we need to capture somehow that the values $F(x)$ of the functor $F$ \enquote{vary functorially in $x$}. To do this, we connect every $u\in F(x)$ to its image $v\simeq F(\alpha)(u)$ by a new $1$-simplex $\varphi\colon u\rightarrow v$. Then we keep adding further simplices to make sure that the object we end up with is a quasi-category. For example, in our picture we have a morphism $v\rightarrow w$ in $F(y)$, so we have to add a $2$-simplex $\sigma\colon \Delta^2\rightarrow \Uu$ in such a way that $\sigma|_{\Delta^{\{0,2\}}}\colon u\rightarrow w$ is a composition of $\varphi$ and $v\rightarrow w$.
	
	To summarise, we've given some vague motivation why functors $F\colon \Cc\rightarrow\cat{Cat}_\infty$ should correspond to certain fibrations $p\colon \Uu\rightarrow \Cc$, in such a way that the values $F(x)$ correspond to the fibres $p^{-1}\{x\}$. Furthermore, we've motivated that for every $\alpha\colon x\rightarrow y$ in $\Cc$ and every $u\in p^{-1}\{x\}\simeq F(x)$, there should be a special $1$-simplex $\varphi\colon u\rightarrow v$ that connects $u$ to its image under $F(\alpha)$. Every other $1$-simplex from $u$ to an object in $p^{-1}\{y\}$ should arise as a composition with some morphism $v\rightarrow w$ in $p^{-1}\{y\}\simeq F(y)$. These vague ideas are captured in a precise sense by the following definition:
\end{numpar}
\begin{defi}\label{def:Cocartesian}
	Let $p\colon \Uu\rightarrow\Cc$ be an inner fibration of quasi-categories.
	\begin{alphanumerate}
		\item A morphism $\varphi\colon u\rightarrow v$ in $\Uu$ is called \emph{$p$-cocartesian} if, for every $n\geqslant 2$, every lifting problem\label{enum:CocartesianMorphism}
		\begin{equation*}
			\begin{tikzcd}
				\Lambda_0^n\rar\dar & \Uu\dar["p"]\\
				\Delta^n\urar[dashed]\rar & \Cc
			\end{tikzcd}
		\end{equation*}
		in which $\Delta^{\{0,1\}}\subseteq \Lambda_0^n$ is sent to $\varphi$, has a solution.
		\item We call $p$ a \emph{cocartesian fibration} if every lifting problem\label{enum:CocartesianFibration}
		\begin{equation*}
			\begin{tikzcd}
				\{0\}\rar\dar & \Uu\dar["p"]\\
				\Delta^1\urar[dashed]\rar & \Cc
			\end{tikzcd}
		\end{equation*}
		has a solution in which $\Delta^1$ is sent to a $p$-cocartesian morphism.
	\end{alphanumerate}
	There are dual notions of \emph{$p$-cartesian morphisms} and \emph{cartesian fibrations}, in which we use $\Delta^{\{n-1,n\}}\subseteq \Lambda_n^n\rightarrow \Delta^n$ and $\{1\}\rightarrow \Delta^1$ instead.
\end{defi}
It's easy to identify those cocartesian fibrations that correspond to functors $F\colon \Cc\rightarrow \cat{An}$.
\begin{lem}[\enquote{Left fibrations are cocartesian fibrations whose fibres are animae}]\label{lem:CocartesianLeft}
	For a cocartesian fibration $p\colon \Uu\rightarrow\Cc$, the following conditions are equivalent:
	\begin{alphanumerate}
		\item Every morphism in $\Uu$ is $p$-cocartesian.\label{enum:EveryMorphismCocartesian}
		\item $p\colon \Uu\rightarrow \Cc$ is a left fibration.\label{enum:CocartesianLeft}
		\item All fibres $p^{-1}\{x\}$ for $x\in\Cc$ are animae.\label{enum:FibresAreAnimae}
	\end{alphanumerate}
\end{lem}
\begin{proof}[Proof sketch]
	The equivalence \cref{enum:EveryMorphismCocartesian} $\Leftrightarrow$ \cref{enum:CocartesianLeft} is clear and the implication \cref{enum:CocartesianLeft} $\Rightarrow$ \cref{enum:FibresAreAnimae} follows from \cref{cor:LeftFibrationsOverAnima}. To prove \cref{enum:FibresAreAnimae} $\Rightarrow$ \cref{enum:EveryMorphismCocartesian}, let $\varphi\colon u\rightarrow v$ be a morphism in $\Uu$. We wish to show that $\varphi$ is $p$-cocartesian. Suppose we're given a lifting problem of the sort we're interested in: a diagram
	\begin{equation*}
		\begin{tikzcd}
			\Lambda_0^n\rar["\sigma"]\dar & \Uu\dar["p"]\\
			\Delta^n\urar[dashed]\rar & \Cc
		\end{tikzcd}
	\end{equation*}
	such that $\sigma|_{\Delta^{\{0,1\}}}$ is $\varphi\colon u\rightarrow v$. We will construct a new lifting problem
	\begin{equation*}
		\begin{tikzcd}
			\Lambda_0^n\dar\drar[commutes]\rar["d_1"] & \Lambda_1^{n+1}\dar\rar["\ov\sigma"] & \Uu\dar["p"]\\
			\Delta^n\rar["d_1"] & \Delta^{n+1}\urar[dashed]\rar &\Cc
		\end{tikzcd}
	\end{equation*}
	such that $\sigma=\ov\sigma\circ d_1$ (note that $d_1$ maps $\Delta^{\{0,1\}}$ to $\Delta^{\{0,2\}}$, so $\varphi$ is now the image of $\Delta^{\{0,2\}}$ under $\ov\sigma$). Since $p$ is an inner fibration, we'll be able to solve the new lifting problem and get a solution for our original one.
	
	
	Let's construct $\ov\sigma$! By \cref{def:Cocartesian}\cref{enum:CocartesianFibration}, applied to $p(\varphi)\colon \Delta^1\rightarrow \Cc$, we can choose a $p$-cocartesian morphism $\varphi'\colon u\rightarrow v'$ such that $v'\in p^{-1}\{p(v)\}$. By \cref{def:Cocartesian}\cref{enum:CocartesianMorphism}, applied to a suitable $\Lambda_0^2\rightarrow \Uu$, we can find a morphism $\psi\colon v'\rightarrow v$ such that $\varphi\simeq \psi\circ \varphi'$. Note that $\psi$ is an equivalence since $p^{-1}\{p(v)\}$ is an anima by assumption. Now we construct $\ov\sigma$ piece by piece. We put $\ov\sigma|_{d_1(\Lambda_0^n)}\coloneqq \sigma$ and we send $\{1\}$ to $v'$ as well as $\Delta^{\{0,1\}}$ to $\varphi'$. Furthermore, we send $\Delta^{\{0,1,2\}}$ to the $2$-simplex witnessing $\varphi\simeq \psi\circ \varphi'$. The rest of $\Lambda_1^{n+1}\smallsetminus (d_1(\Lambda_0^n)\cup \Delta^{\{0,1,2\}})$ can be filled by a sequence of horn filling problems in which either the first edge is $\varphi'$, so a filler exists by \cref{def:Cocartesian}\cref{enum:CocartesianMorphism}, or the first edge is $\psi\colon v\rightarrow v'$, so a filler exists by Joyal's lifting theorem (\cref{thm:JoyalLifting}) since $\psi$ is an equivalence. As usual, we skip the horn filling combinatorics.
\end{proof}
We can now state the straightening/unstraightening equivalence. As was, unfortunately, clear from the beginning, we won't give a proof here. The most readable proof available is probably due to Gijs Heuts \cite{HeutsStraightening} with contributions by Fabian Hebestreit and Jaco Ruit, building on previous work by Cisinski and Nguyen. Lurie's original proof can be found in \cite[\S\href{https://people.math.harvard.edu/~lurie/papers/HTT.pdf\#section.3.2}{3.2}]{HTT}. There's also another approach by Cisinski \cite{Cisinski} in which straightening/unstraightening is much easier to obtain, but much more work is needed to identify $\cat{An}\subseteq \cat{Cat}_\infty$ with the full sub-quasi-category spanned by the Kan complexes.
\begin{thm}[Straightening/unstraightening]\label{thm:Straightening}
	Let $\Cc$ be a quasi-category.
	\begin{alphanumerate}
		\item Let $\cat{Cocart}(\Cc)\subseteq\cat{Cat}_{\infty/\Cc}$ be the \embrace{non-full!} sub-quasi-category spanned by cocartesian fibrations over $\Cc$ and those maps that preserve cocartesian morphisms \embrace{see \cref{par:SubQuasiCategories}}. Then there are inverse equivalences of quasi-categories\label{enum:CocartesianStraightening}
		\begin{equation*}
			\operatorname{St}^{(\mathrm{cocart})}\colon \cat{Cocart}(\Cc) \underset{\simeq}{\overset{\simeq}{\doublelrmorphism}}\F(\Cc,\cat{Cat}_\infty)\noloc \operatorname{Un}^{(\mathrm{cocart})}\vphantom{\overset{\simeq}{\overline{x}}}
		\end{equation*}
		called \enquote{straightening} and \enquote{unstraightening}. For a cocartesian fibration $p\colon \Uu\rightarrow \Cc$, the value of $\operatorname{St}^{(\mathrm{cocart})}(p)\colon \Cc\rightarrow\cat{Cat}_\infty$ at $x\in \Cc$ is given by the fibre $p^{-1}\{x\}$. Furthermore, $F\colon \Cc\rightarrow \Dd$ is a functor, then the unstraightening equivalence sends the precomposition functor $-\circ F\colon \F(\Dd,\cat{Cat}_\infty)\rightarrow \F(\Cc,\cat{Cat}_\infty)$ to the pullback $F^*\colon \cat{Cocart}(\Dd)\rightarrow\cat{Cocart}(\Cc)$.
		\item Let $\cat{Left}(\Cc)\subseteq \cat{Cat}_{\infty/\Cc}$ be the full sub-quasi-category spanned by the left fibrations over $\Cc$ \embrace{see \cref{par:SubQuasiCategories}}. Then the equivalences from \cref{enum:CocartesianStraightening} restrict to equivalences\label{enum:LeftStraightening}
		\begin{equation*}
			\operatorname{St}^{(\mathrm{left})}\colon \cat{Left}(\Cc) \underset{\simeq}{\overset{\simeq}{\doublelrmorphism}}\F(\Cc,\cat{An})\noloc \operatorname{Un}^{(\mathrm{left})}\vphantom{\overset{\simeq}{\overline{x}}}\,.
		\end{equation*}
	\end{alphanumerate}
	Dually, there are equivalences $\cat{Cart}(\Cc)\simeq \F(\Cc^\op,\cat{Cat}_\infty)$ and $\cat{Right}(\Cc)\simeq \F(\Cc^\op,\cat{An})$. Here $\Cc^\op$ is the opposite quasi-category from \cref{par:Opposite}.\hfill$\blacksquare$
\end{thm}
Most of the time, one can treat \cref{thm:Straightening} as a black box and work purely with the statement, without knowing how exactly and we don't need to know how exactly the functors $\operatorname{St}^{(\mathrm{cocart})}$ and $\operatorname{Un}^{(\mathrm{cocart})}$ are constructed! 

Still, we'll spend the rest of \cref{subsec:Straightening} to give you some idea how the construction works. In \cref{par:StraighteningOnMorphisms}, we'll explain the effect of straightening on morphisms. After that, we'll discuss two (hopefully enlightening) classical examples in the language of straightening/unstraightening in~\cref{par:Stacks} and~\cref{par:Coverings}. But let's begin by giving some examples.
\begin{exm}\label{exm:Straightening}
	Let $\Cc$ be a quasi-category. The following are examples of cocartesian fibrations and their unstraightenings.
	\begin{alphanumerate}
		\item For every quasi-category $\Dd$, the unique functor $\Dd\rightarrow *$ is a cocartesian fibrations, with the cocartesian morphisms given by the equivalences in $\Dd$. This is an easy application of Joyal's lifting theorem (\cref{thm:JoyalLifting}). Furthermore, since cocartesian fibrations are clearly preserved under pullbacks, we see that $\pr_2\colon \Dd\times \Cc\rightarrow\Cc$ is a cocartesian fibration for every quasi-category $\Cc$. The $\pr_2$-cocartesian morphisms in $\Dd\times\Cc$ are precisely those that are equivalences in the $\Dd$-component. The straightening of $\pr_2\colon \Dd\times \Cc\rightarrow\Cc$ is the constant functor $\const \Dd\colon \Cc\rightarrow \cat{Cat}_\infty$; this follows from the pullback statement in \cref{thm:Straightening}\cref{enum:CocartesianStraightening}, but it's probably also pretty clear intuitively.\label{enum:ProjectionsStraightenToConstantFunctors}
		\item For every $x\in \Cc$, we've seen in \cref{cor:HomAnima} that $t\colon \Cc_{x/}\rightarrow \Cc$ is even a left fibration. The fibre of $t$ over $y\in\Cc$ is $\Hom_\Cc(x,y)$ by \cref{par:HomInQuasiCategories}, so we can use the straightening of $t$ as our definition of the Hom functor $\Hom_\Cc(x,-)\colon \Cc\rightarrow \cat{An}$. Analogously, the dual construction $s\colon \Cc_{/y}\rightarrow \Cc$ is a right fibration and its cartesian straightening is, by definition, the contravariant Hom functor $\Hom_\Cc(-,y)\colon \Cc^\op\rightarrow \cat{An}$. This still leaves the question how to construct the two-variable Hom functor $\Hom_\Cc\colon \Cc^\op\times\Cc\rightarrow\cat{An}$, which we'll discuss in \cref{con:HomInTwoVariables,con:HomTwAr} below.\label{enum:SliceLeftFibration}
		\item The target projection $t\colon \Ar(\Cc)\rightarrow \Cc$ from \cref{par:HomInQuasiCategories} is a cocartesian fibration, and its is our definition of the functor $\Cc_{/-}\colon \Cc\rightarrow\cat{Cat}_\infty$ that sends $x\in\Cc$ to the slice quasi-category $\Cc_{/x}$. A morphism $\varphi\colon (\alpha\colon u\rightarrow u')\rightarrow (\beta\colon v\rightarrow v')$ in $\Ar(\Cc)$, that is, a commutative diagram\label{enum:ArCocartesianFibration}
		\begin{equation*}
			\begin{tikzcd}
				u\rar\dar["\alpha"']\drar[commutes] & v\dar["\beta"]\\
				u'\rar & v'
			\end{tikzcd}
		\end{equation*}
		in $\Cc$, is $t$-cocartesian if and only if $u\rightarrow v$ is an equivalence in $\Cc$. Proving this is a somewhat subtle and will lead us on a detour in \cref{subsec:HomotopyPullbacks}. One way to see the \enquote{if}-part (which is the difficult part) would be to reformulate a lifting problem for $t$ against $\Lambda_0^n\rightarrow \Delta^n$ into a lifting problem for $\Cc\rightarrow *$ against $\Lambda_0^n\times \Delta^1\sqcup_{\Lambda_0^n\times\{1\}}\Delta^n\times\{1\}\rightarrow \Delta^n\times\Delta^1$, as in the proof of \cref{cor:FKanFibration}. Then one proves, using Joyal's lifting problem (\cref{thm:JoyalLifting}), that a lifting problem of the latter kind is always solvable if the original lifting problem maps $\Delta^{\{0,1\}}\subseteq\Lambda_0^n$ to a morphism $\varphi$ as above.
		
		However, a much nicer proof of the \enquote{if}-part  is provided by \cref{lem:HomInArrowCategories} and \cref{lem:CocartesianMorphisms} below (except that there are some black boxes involved \ldots). For the \enquote{only if}-part, it's enough to write down the correct lifting diagrams; we leave this to you.
	\end{alphanumerate}
\end{exm}
\begin{numpar}[Straightening/unstraightening on morphisms]\label{par:StraighteningOnMorphisms}
	Suppose we're given a cocartesian fibration $p\colon \Uu\rightarrow \Cc$ and let $F\colon \Cc\rightarrow \cat{Cat}_\infty$ be its straightening. We know what $F$ does on objects: It sends $x\in \Cc$ to the fibre $p^{-1}\{x\}$. We'll now explain what $F$ does on morphisms. To this end, let $\Ar^{(\mathrm{cocart})}(\Uu)\subseteq \Ar(\Uu)$ denote the full sub-quasi-category spanned by the $p$-cocartertesian morphisms. Then 
	\begin{equation*}
		\Ar^{(\mathrm{cocart})}(\Uu)\longrightarrow \Ar(\Cc)\times_{s,\Cc}\Uu
	\end{equation*}
	is a trivial fibration.\footnote{Intuitively, this says that given a morphism $\alpha\colon x\rightarrow y$ in $\Cc$ and an object $u\in p^{-1}\{x\}$, then lifting $\alpha$ to a $p$-cocartesian morphism $\varphi\colon u\rightarrow v$ can not only be done, but even in a unique way (up to contractible ambiguity). This fits perfectly into the picture from \cref{par:StraighteningMotivation}: Such a $p$-cocartesian lift $\varphi$ connects $u\in p^{-1}\{x\}\simeq F(x)$ to its image under $F(\alpha)\colon F(x)\rightarrow F(y)$. So $\varphi$ should be unique.} To see this, first consider the case where $p$ is a left fibration. Then $\Ar^{(\mathrm{cocart})}(\Uu)= \Ar(\Uu)$ by \cref{lem:CocartesianLeft}\cref{enum:EveryMorphismCocartesian}. Furthermore, $\{0\}\rightarrow \Delta^1$ is left anodyne. Hence the map above is a trivial fibration by \cref{cor:FKanFibration}. In general, we can adapt the proof of \cref{cor:FKanFibration} to show that $\Ar^{(\mathrm{cocart})}(\Uu)\rightarrow \Ar(\Cc)\times_{s,\Cc}\Uu$ has lifting against $\partial\Delta^n\rightarrow\Delta^n$: Rewrite such a lifting problem as a lifting problem for $p\colon \Uu\rightarrow \Cc$ against $\partial\Delta^n\times\Delta^1\sqcup_{\partial\Delta^n\times\{0\}}\Delta^n\times\{0\}\rightarrow \Delta^n\times\Delta^1$. The latter is a sequence of horn lifting problems, each of which can be solved either because $p$ is an inner fibration or by employing \cref{def:Cocartesian}\cref{enum:CocartesianMorphism}.%As usual, we skip the simplicial combinatorics.
	
	Given a morphism $\alpha\colon x\rightarrow y$ in $\Cc$, we can now give the desired description of the functor $F(\alpha)\colon p^{-1}\{x\}\rightarrow p^{-1}\{y\}$ as follows: Pull back $\Ar^{(\mathrm{cocart})}(\Uu)\rightarrow \Ar(\Cc)\times_{s,\Cc}\Uu$ along $\{\alpha\}\rightarrow \Ar(\Cc)$ to obtain a trivial fibration $\{\alpha\}\times_{\Ar(\Cc)}\Ar(\Uu)\rightarrow \{\alpha\}\times_{s,\Cc}\Uu\cong \{x\}\times_\Cc\Uu\cong p^{-1}\{x\}$. Every trivial fibration admits a section. By choosing such a section and composing with the target projections $t\colon \Ar(\Uu)\rightarrow \Uu$ and $t\colon \Ar(\Cc)\rightarrow \Cc$, we obtain (up to natural equivalence) the desired functor
	\begin{equation*}
		F(\alpha)\colon p^{-1}\{x\}\longrightarrow \{\alpha\}\times_{\Ar(\Cc)}\Ar(\Uu)\overset{t}{\longrightarrow} \{y\}\times_{\Cc}\Uu\cong p^{-1}\{y\}\,.
	\end{equation*}	
	With (a lot) more care, one can continue these considerations to give a complete description of the functor $F\colon \Cc\rightarrow \cat{Cat}_\infty$. This was first done by Haugseng and is described in \cite[\S3.3]{Land}. The proofs of straightening/unstraightening in \cite{HeutsStraightening} or \cite[\S\href{https://people.math.harvard.edu/~lurie/papers/HTT.pdf\#section.3.2}{3.2}]{HTT} proceed instead by constructing a simplicially enriched functor $\CC[\Cc]\rightarrow \cat{sSet}^\Delta$, as they deduce \cref{thm:Straightening} from a suitable Quillen equivalence of model categories.
\end{numpar}
\begin{numpar}[Straightening/unstraightening and stacks]\label{par:Stacks}
	We'll briefly explain the relation between \cref{thm:Straightening} and the language of stacks from algebraic geometry. If you already know stacks, this will hopefully make \cref{thm:Straightening} less mysterious. If you'd like to learn about stacks, this remark will hopefully make the literature on stacks less mysterious. If you don't care about stacks at all, you can safely skip this remark.
	
	In algebraic geometry, one is naturally lead to functors whose values should be groupoids. For example, given a scheme $S$ and a group scheme $G$ acting on $S$, one would like to study the functor $[S/G]\colon (\cat{Sch}_{/S})^\op\rightarrow \cat{Grpd}$ that sends any scheme $X$ over $S$ to the groupoid of $G_X$-torsors, where $G_X\coloneqq G\times_SX$ is the base change of $G$ to $X$. A morphism $f\colon X\rightarrow Y$ in $\cat{Sch}_{/S}$ should be sent to the pullback functor $f^*\colon \{G_Y\text{-torsors}\}\rightarrow \{G_X\text{-torsors}\}$. Here one quickly runs into a problem: If $g\colon Y\rightarrow Z$ is another morphism in $\cat{Sch}_{/S}$, then the associated pullback functors come with a natural \emph{equivalence} $f^*\circ g^*\simeq (g\circ f)^*$, but that equivalence is not an \emph{equality}. So $[S/G]$ can't exist as a functor $[S/G]\colon (\cat{Sch}_{/S})^\op\rightarrow \cat{Grpd}$ into the category of groupoids; instead, it's a functor
	\begin{equation*}
		[S/G]\colon \N\left(\cat{Sch}_{/S}\right)^\op\longrightarrow \cat{Grpd}^{(2)}
	\end{equation*}
	into the \emph{$2$-category of groupoids} as introduced in \cref{exm:CatAs2Category}. The ancient algebraic geometers didn't have the language to deal with functors into a $2$-category, but they made do with the tools of their time: They instead constructed a functor $p\colon \Uu\rightarrow \cat{Sch}_{/S}$ in such a way that $\N(p)\colon \N(\Uu)\rightarrow \N(\cat{Sch}_{/S})$ is a right fibration whose straightening $\operatorname{St}^{(\mathrm{right})}(\N(p))\simeq [S/G]$ is the functor above.%
	\footnote{\label{footnote:UnstraighteningOrdinaryCategory}It's not a coincidence that the unstraightening of $[S/G]$ is the nerve of an ordinary category. Let $\Cc$ be any ordinary category, let $F\colon \N(\Cc)^\op\rightarrow \cat{Cat}_\infty$ be any functor that lands in the full sub-quasi-category $\cat{Cat}^{(2)}\subseteq \cat{Cat}_\infty$ from \cref{exm:CatAs2Category}, and let $p\colon \Uu\rightarrow\N(\Cc)$ be the cartesian unstraightening of $F$. Then $\Uu$ is equivalent to the nerve of an ordinary category. We'll give a sketch of the proof, which uses the notion of \emph{homotopy pullbacks} from \cref{subsec:HomotopyPullbacks}. First, it's enough to show that $\Hom_\Uu(u,v)$ is a discrete anima for all $u,v\in \Uu$, because then the essentialy surjective functor $u_\Uu\colon \Uu\rightarrow \N(\operatorname{ho}(\Uu))$ (given by the unit of the adjunction $\operatorname{ho}\dashv \N$) is also fully faithful, hence an equivalence by \cref{thm:EquivalenceFullyFaithfulEssentiallySurjective}. To prove that $\Hom_\Uu(u,v)$ is discrete, observe that we have a map $\Hom_\Uu(u,v)\rightarrow \Hom_{\N(\Cc)}(p(u),p(v))$. Since the target is a discrete anima, the source $\Hom_\Uu(u,v)$ is a disjoint union of the fibres of this map. So it's enough to show that each individual fibre is a discrete anima. Restricting to the fibre over $\alpha\in\Hom_{\N(\Cc)}(p(u),p(v))$ amounts to base changing along the map $\alpha\colon \Delta^1\rightarrow \N(\Cc)$. So we may assume $\N(\Cc)\cong \Delta^1$.
	
	Now $\{0\}\rightarrow \Delta^1$ is fully faithful. Hence, if $\Uu_0\coloneqq \{0\}\times_{\Delta^1}\Uu$ denotes the fibre over $0$, then $\Uu_0\rightarrow \Uu$ is fully faithful too. But $\Uu_0$ is also equivalent to the nerve of an ordinary category, because we assume our original functor $F$ takes values in $\cat{Cat}^{(2)}\subseteq \cat{Cat}_\infty$. So if $u,v\in \Uu_0$, then $\Hom_\Uu(u,v)\simeq \Hom_{\Uu_0}(u,v)$ is a discrete anima. The same reasoning applies if $u$ and $v$ both belong to the fibre over $1$. If $p(u)=1$ and $p(v)=0$, then the map $\Hom_\Uu(u,v)\rightarrow \Hom_{\Delta^1}(1,0)\simeq\emptyset$ forces $\Hom_\Uu(u,v)\simeq \emptyset$, which is discrete too. It remains to deal with the case $p(u)=0$ and $p(v)=1$. After choosing a $p$-cartesian lift $\varphi\colon u'\rightarrow v$ of $0\rightarrow 1$, the dual of \cref{lem:CocartesianMorphisms} below provides a homotopy pullback
	\begin{equation*}
		\begin{tikzcd}[ampersand replacement=\&]
			\Hom_\Uu(u,u')\rar["\varphi_*"] \dar["p"']\drar[dash, phantom, "\scriptstyle\rlap{\phantom{$\pullbacksign$}}\smash{\pullbacksign_{\scriptstyle h}}", start anchor=center, end anchor=center] \& \Hom_\Uu(u,v)\dar["p"]\\
			\Hom_{\Delta^1}(0,0)\rar["{p(\varphi)_*}"] \& \Hom_{\Delta^1}(0,1)
		\end{tikzcd}
	\end{equation*}
	The bottom horizontal arrow is clearly a homotopy equivalence, so the top arrow must be one as well, whence $\Hom_\Uu(u,u')\simeq \Hom_\Uu(u,v)$. As we already know $\Hom_\Uu(u,u')$ to be discrete, we're done.}
	Explicitly, $\Uu$ is the category of pairs $(X,\Pp)$, where $X\in\cat{Sch}_{/S}$ and $\Pp$ is a $G_X$-torsor. Morphisms $(X,\Pp)\rightarrow (Y,\Qq)$ in $\Uu$ are pairs $(f,\alpha)$ where $f\colon X\rightarrow Y$ is a morphism in $\cat{Sch}_{/S}$ and $\alpha\colon f^*\Qq\overset{\simeq}{\longrightarrow}\Pp$ is an isomorphism of $G_X$-torsors.
	See \cite[Example~8.1.10]{OlssonStacks} or \cite[\stackstag{036Z}]{Stacks}.
	
	More generally, a \emph{fibred category} is a functor $p\colon \Uu\rightarrow \Cc$ of ordinary categories such that $\N(p)\colon \N(\Uu)\rightarrow\N(\Cc)$ is a cartesian fibration.\footnote{It follows from the uniqueness statement in \cref{lem:LiftingConditions}\cref{enum:LiftingN} that any map between nerves of ordinary categories is automatically an inner fibration. So to check whether such a map is a cartesian fibration, it's enough to show the existence of cartesian lifts.} The classical definition of cartesian morphisms, see \cite[Definition~3.1.1]{OlssonStacks} or \cite[\stackstag{02XK}]{Stacks}, differs from \cref{def:Cocartesian}\cref{enum:CocartesianMorphism}, but it's still equivalent, as we'll see in \cref{lem:CocartesianMorphisms}. By \cref{thm:Straightening}, the data of a fibred category defines a functor $\operatorname{St}^{(\mathrm{cart})}\colon \N(\Cc)^\op\rightarrow \cat{Cat}_\infty$, which necessarily factors through the full sub-quasi-category $\cat{Cat}^{(2)}\subseteq \cat{Cat}_\infty$ from \cref{exm:CatAs2Category}, because the fibres of $\N(p)\colon \N(\Uu)\rightarrow\N(\Cc)$ must be nerves of ordinary categories again, Conversely, we show in footnote~\cref{footnote:UnstraighteningOrdinaryCategory} below that the unstraightening of such a functor is necessarily the nerve of an ordinary category.
	
	A \emph{category fibred in groupoids} is a fibred category $p\colon \Uu\rightarrow \Cc$ such that all fibres are groupoids; equivalently, the associated functor $\operatorname{St}^{(\mathrm{cart})}\colon \N(\Cc)^\op\rightarrow \cat{Cat}^{(2)}$ factors through $\cat{Grpd}^{(2)}\subseteq \cat{Cat}^{(2)}$. Finally, if $\Cc$ is equipped with a Grothendieck topology, we call $p\colon \Uu\rightarrow \Cc$ a \emph{stack} if the functor $\operatorname{St}^{(\mathrm{right})}(\N(p))\colon \Cc^\op\rightarrow \cat{Grpd}^{(2)}$ is a sheaf. To formulate the sheaf condition, one needs an appropriate notion of limits in $\cat{Grpd}^{(2)}$ (or in $\cat{An}$), which we'll see in \cref{def:Colimits}. Fortunately, these limits can be pinned down in explicit terms; for example, it's not too hard to unravel \cref{lem:ColimitsInAnima} to arrive at the description from \cite[\S4.2]{OlssonStacks} or \cite[\stackstag{026B}]{Stacks}. In particular, the result is (the nerve of) a groupoid again.
	
	Thus, by exclusively working on the fibration side of the cartesian straightening equivalence, the theory of stacks can be and has been developed within the framework of ordinary category theory. But with today's tools, its actually possible to talk about stacks in the intended way: as functors into $\cat{Grpd}^{(2)}$. I find the latter much easier.
\end{numpar}
We've seen in \cref{par:Stacks} that \cref{thm:Straightening} is already interesting for functors into $\cat{Grpd}^{(2)}\subseteq \cat{An}$. But there's an even simpler case: functors into \emph{sets!} As it turns out, even this simplest possible special case is interesting and recovers classical theory.

\begin{numpar}[Straightening/unstraightening and covering theory.]\label{par:Coverings}
	An anima is called \emph{discrete} if it is homotopy equivalent to a disjoint union of copies of the point $*$. Equivalently, all path components are contractible. Considering sets as discrete animae, it's easy to construct a functor $\N(\cat{Set})\rightarrow \cat{An}$; this functor is fully faithful and an equivalence onto the full sub-quasi-category spanned by the discrete animae.\footnote{Let's sketch how to do this: One can equip $\cat{Set}$ with a trivial Kan enrichment $\cat{Set}^\Delta$ in which $\F_{\cat{Set}^\Delta}(S,T)$ is just a disjoint union of $\Hom_{\cat{Set}}(S,T)$ many points. Sending $S\mapsto \coprod_{s\in S}*$ then defines a fully faithful simplicially enriched functor $\cat{Set}^\Delta\rightarrow \cat{Kan}^\Delta$. Applying $\N^\Delta(-)$, we obtain a fully faithful functor of quasi-categories $\N(\cat{Set})\cong \N^\Delta(\cat{Set}^\Delta)\rightarrow \N^\Delta(\cat{Kan}^\Delta)$, whose essential image are precisely the discrete animae.}
	
	Let $X$ be an anima. A \emph{covering of $X$} is a Kan fibration (or equivalently a left fibration, see \cref{cor:LeftFibrationsOverAnima}) $p\colon X'\rightarrow X$ such that for all $x\in X$, the fibres $p^{-1}\{x\}$ are discrete animae. This recovers the usual notion of coverings from topology. More precisely, let's call a covering $p\colon X'\rightarrow X$ \emph{strict} if the fibres $p^{-1}\{x\}$ are not only equivalent to but \emph{isomorphic} to disjoint unions of copies of $*$. Then every covering is equivalent to a strict covering and the adjunction $\abs*{\,\cdot\,}\colon \cat{Kan}\shortdoublelrmorphism \cat{Top}\noloc \Sing$ from \cref{par:GeometricRealisation} transform strict coverings of animae into usual coverings of topological spaces and vice versa.\footnote{To see that every covering $p\colon X'\rightarrow X$ is equivalent to a strict one, we need to use some details of the construction of the equivalence from \cref{thm:Straightening}\cref{enum:LeftStraightening}: Since the functor $F\colon X\rightarrow \cat{An}$ associated to $X$ lands in discrete animae, we can factor it, up to equivalence, through a functor $F_0\colon X\rightarrow \N(\cat{Set})$. By \cref{lem:SimplicialHoNerveAdjunction}, $F_0$ is induced by a functor of ordinary categories $\overline{F}_0\colon \operatorname{ho}(X)\rightarrow\cat{Set}$. If $\overline{p}\colon \Uu\rightarrow \operatorname{ho}(X)$ is the \emph{Grothendieck construction} of $\overline{F}_0$, then $\N(\overline{p})\colon \N(\Uu)\rightarrow \N(\operatorname{ho}(X))$ can be shown to be a strict covering. Hence the pullback $\N(\Uu)\times_{\N(\operatorname{ho}(X))}X\rightarrow X$ is a strict covering too and equivalent to our original covering $p$.
		
	Now suppose $p\colon X'\rightarrow X$ is a strict covering. Using that the fibres of $p$ are disjoint unions of copies of $*$ together with the lifting properties of Kan fibrations, it's easy to see that for every $\Delta^n\rightarrow X$, the pullback $\Delta^n\times_XX'$ consists of a disjoint union of copies of $\Delta^n$. This means that the preimage of any cell in the CW-complex $\abs*{X}$ under $\abs*{p}\colon \abs*{X'}\rightarrow \abs*{X}$ is a disjoint union of copies of that cell. Via some technical arguments that we omit, this shows that $\abs*{p}$ is a covering in the usual sense. Conversely, if $q\colon Y'\rightarrow Y$ is a covering of topological spaces, $q$ is Serre fibration. $\Sing$ turns Serre fibrations into Kan fibrations because $\abs*{\,\cdot\,}\colon \cat{Kan}\shortdoublelrmorphism \cat{Top}\noloc \Sing$ is a Quillen adjunction (even a Quillen equivalence) by \cref{exm:QuillenAdjunction}. Hence $\Sing q\colon \Sing Y'\rightarrow\Sing Y$ is a Kan fibration. This could also be shown by an easy direct argument (observe that the pair $(\abs*{\Delta^n},\abs*{\Lambda_i^n})$ is homeomorphic to $([0,1]^{n-1}\times[0,1],[0,1]^{n-1}\times\{0\})$ and use the homotopy lifting property of covering spaces, see \cite[Proposition~\href{https://pi.math.cornell.edu/~hatcher/AT/AT.pdf\#page=69}{1.30}]{Hatcher} for example). Since $\Sing$ preserves pullbacks and sends discrete topological spaces to disjoint unions of copies of $*$, we see that $\Sing q$ is indeed a strict covering.}
	
	We let $\cat{Cov}(X)\subseteq\cat{Left}(X)$ denote the full sub-quasi-category spanned by the coverings of $X$. Under the straightening equivalence from \cref{thm:Straightening}\cref{enum:LeftStraightening}, coverings $p\colon X'\rightarrow X$ correspond to those functors $F\colon X\rightarrow\cat{An}$ that land in discrete animae. Thus, we get an equivalence of quasi-categories
	\begin{equation*}
		\cat{Cov}(X)\overset{\simeq}{\longrightarrow}\F\bigl(X,\N(\cat{Set})\bigr)\,.
	\end{equation*}
	Now recall $\F(X,\N(\cat{Set}))\cong \N(\Fun(\operatorname{ho}(X),\cat{Set}))$ from \cref{lem:SimplicialHoNerveAdjunction}. But what is $\operatorname{ho}(X)$? We know $X\simeq \Sing {\abs*{X}}$ from the simplicial approximation theorem, hence $\operatorname{ho}(X)\simeq \operatorname{ho}(\Sing{\abs*{X}})$. By the description in \cref{par:HomotopyCategory}, the objects of $\operatorname{ho}(\Sing{\abs*{X}})$ are given by $\Sing_0{\abs*{X}}$, the points of $\abs*{X}$. The morphisms of $\operatorname{ho}(\Sing{\abs*{X}})$ are equivalence classes of $\Sing_1{\abs*{X}}$, that is, equivalence classes of paths in $X$. A quick unravelling of definitions shows that the equivalence relation is precisely for two paths to be homotopic. Hence $\operatorname{ho}(\Sing{\abs*{X}})$ is precisely the \emph{fundamental groupoid} $\Pi_1\abs{X}$ of the topological space $\abs*{X}$, and therefore $\operatorname{ho}(X)\simeq \Pi_1\abs{X}$. We have thus proved a classical classification result from topology:
\end{numpar}
\begin{thm}[Classification of covering animae]\label{thm:CoveringTheory}
	Let $X$ be an anima. Then $\cat{Cov}(X)$ is equivalent to the nerve of the ordinary category $\Fun(\Pi_1\abs{X},\cat{Set})$. In particular, there's an equivalence of ordinary categories
	\begin{equation*}
		\operatorname{ho}\bigl(\cat{Cov}(X)\bigr)\simeq \Fun\bigl(\Pi_1\abs{X},\cat{Set}\bigr)\,.
	\end{equation*}
\end{thm}
It might not be immediately obvious, but \cref{thm:CoveringTheory} comprises all you would ever want to know about covering theory. Since it fits the theme of these notes, let us spell this out in detail:

\begin{cor}\label{cor:CoveringTheory}
	Suppose $X$ is connected and let a basepoint $x\in X$ be chosen.
	\begin{alphanumerate}
		\item There's a Galois correspondence \embrace{that is, a bijection} between connected coverings of $X$ and subgroups $H\subseteq \pi_1(X,x)$.\label{enum:GaloisCorrespondence}
		\item $X$ is simply connected \embrace{that is, $\pi_1(X,x)\cong 0$} if and only if every covering $p\colon X'\rightarrow X$ admits a section $s\colon X\rightarrow X'$ if and only if every covering of $X$ splits into a disjoint union of copies of $X$.\label{enum:SplitCovering}
		\item $\cat{Cov}(X)$ contains a unique object $\widetilde{p}\colon \widetilde{X}\rightarrow X$ \embrace{up to equivalence} with the property that $\widetilde{X}$ is simply connected. This covering $\widetilde{p}\colon \widetilde{X}\rightarrow X$ is called the universal covering of $X$. If $\Aut_X(\widetilde{X})$ denotes the group of deck transformations of $\widetilde{X}$, that is, the automorphism group of $p\colon \widetilde{X}\rightarrow X$ in $\operatorname{ho}(\cat{Cov}(X))$, then $\Aut_X(\widetilde{X})\cong \pi_1(X,x)$.\label{enum:UniversalCovering}
		\item Suppose $p'\colon X'\rightarrow X$ is a covering and $x'\in X'$ is a point such that $p(x')=x$. Let $f\colon (Z,z)\rightarrow (X,x)$ be a morphism of pointed animae, where $Z$ is connected too. Then the pointed lifting problem\label{enum:LiftingPropertyOfCoverings}
		\begin{equation*}
			\begin{tikzcd}
				& (X',x')\dar["p"]\\
				(Z,z)\urar[dashed,"f'"]\rar["f"] & (X,x)
			\end{tikzcd}
		\end{equation*}
		has a solution if and only if the image of $f_*\colon \pi_1(Z,z)\rightarrow \pi_1(X,x)$ is contained in the image of $p_*\colon \pi_1(X',x')\rightarrow \pi_1(X,x)$. In this case, the lift $f'$ is necessarily unique.
	\end{alphanumerate}
\end{cor}
\begin{proof}[Proof sketch]
	Let's denote $G\coloneqq \pi_1(X,x)$ for short. The crucial observation is that $\Pi_1\abs*{X}$ is equivalent to its full sub-groupoid spanned by $\{x\}$. This full sub-groupoid consists of one element $x$ with $\Hom_{\Pi_1\abs*{X}}(x,x)\cong G$ many automorphisms. Therefore, the functor category $\Fun(\Pi_1\abs*{X},\cat{Set})$ is equivalent to the category $G\mhyph\cat{Set}$ of sets together with a left action of $G$ and we obtain an equivalence of quasi-categories
	\begin{equation*}
		\cat{Cov}(X)\simeq \N\left(G\mhyph\cat{Set}\right)\,.
	\end{equation*}
	With this observation, \cref{enum:GaloisCorrespondence} is immediate: We just have to note that a $G$-set $S$ is \emph{connected}---that is, $S$ can't be written as a disjoint union of two non-empty $G$-sets---if and only if $S$ consists of a single $G$-orbit. This in turn happens if and only if $S\cong G/H$ is the set of left cosets for some subgroup $H\subseteq G$. Part~\cref{enum:SplitCovering} is just as trivial: We have $G\cong 0$ if and only if every $G$-set $S$ has a fixed point (or in other words, the map $S\rightarrow *$ admits a $G$-equivariant section). Furthermore, $G\cong 0$ if and only if every $G$-set is a disjoint union of fixed points.
	
	For \cref{enum:UniversalCovering}, consider $G$ with the natural action of itself as a $G$-set. We let $\widetilde{p}\colon \widetilde{X}\rightarrow X$ be the associated covering. Then $\Aut_X(\widetilde{X})\cong \Aut_{G\mhyph\cat{Set}}(G,G)\cong G$. Furthermore, it's easy to see that $\widetilde{X}$ is simply connected. Indeed, if $q\colon Y\rightarrow \widetilde{X}$ is a covering of $\widetilde{X}$, then $\widetilde{p}\circ q\colon Y\rightarrow X$ is a covering of $X$ and so $q$ determines a morphism in $\cat{Cov}(X)$. This morphism corresponds to a morphism $S\rightarrow G$ of $G$-sets. Every such morphism has a section, which shows that every covering of $\widetilde{X}$ has a section and so $\widetilde{X}$ is simply connected by \cref{enum:SplitCovering}. Conversely, suppose $p\colon X'\rightarrow X$ is a covering of $X$ such that $X'$ is simply connected. Let $S$ be the associated $G$-set; by \cref{enum:GaloisCorrespondence}, we must have $S\cong G/H$ for some subgroup $H\subseteq G$. Let $\pi\colon G\rightarrow G/H$ be the canonical projection. By abuse of notation, $\pi\colon \widetilde{X}\rightarrow X'$ also denotes the associated morphism in $\cat{Cov}(X)$. It's easy to see that $\pi$ is a covering of $X'$.\footnote{We only need to show that $\pi$ is a Kan fibration, because the fact that the fibres of $\pi$ are discrete follows easily from the fact that the fibres of $p\colon X'\rightarrow X$ and $p\circ \pi=\widetilde{p}\colon \widetilde{X}\rightarrow X$ are both discrete. To show that $\pi$ is a Kan fibration, consider any simplex $\sigma\colon \Delta^n\rightarrow X'$ and its image $p\circ\sigma\colon \Delta^n\rightarrow X$ in $X$. To solve any horn lifting problem involving $\sigma$, we may as well base change $\widetilde{X}$ and $X$ along $p\circ\sigma$. But the pullbacks $\Delta^n\times_{p\circ\sigma,X,p}X'$ and $\Delta^n\times_{p\circ\sigma,X,\widetilde{p}}\widetilde{X}$ are both disjoint unions of copies of $\Delta^n$ because we assume $p\colon X'\rightarrow X$ and $\widetilde{p}\colon \widetilde{X}\rightarrow X$ to be coverings. So the new horn lifting problem has a solution for trivial reasons.} Since $X'$ is simply connected, $\pi$ must admit a section $s\colon X'\rightarrow \widetilde{X}$. But then $\pi\colon G\rightarrow G/H$ also admits a section, which is only possible if $H$ is the trivial subgroup.
	
	For \cref{enum:LiftingPropertyOfCoverings}, the \enquote{only if}-part is trivial. For the \enquote{if}-part, we may assume that $X'$ is connected; otherwise just replace $X'$ by the connected component of $x'$. Then $X'$ corresponds to a $G$-set of the form $G/H$. Furthermore, we must have $\pi_1(X',x')\cong H$. To see this, construct a map $\pi\colon \widetilde{X}\rightarrow X'$ as in the proof of \cref{enum:UniversalCovering}. This is necessarily the universal covering of $X'$. Then $\Aut_{X'}(\widetilde{X})\subseteq \Aut_X(\widetilde{X})\cong G$ is the subgroup of those automorphisms $\tau\colon G\rightarrow G$ that satisfy $\pi\circ\tau=\pi$. Hence indeed $\Aut_{X'}(\widetilde{X})$. To solve our lifting problem, note that a lift of $f$ is equivalent to a section of the pullback covering $p_Z\colon Z\times_XX'\rightarrow Z$. Since straightening/unstraightening transforms pullbacks into precompositions (see \cref{thm:Straightening}\cref{enum:CocartesianStraightening}), $p_Z$ corresponds to the set $G/H$ with $\pi_1(Z,z)$-action induced by $f_*\colon \pi_1(Z,z)\rightarrow \pi_1(X,x)=G$. By assumption, the image of $f_*$ is contained in $H$, so the action is trivial. Hence $G/H$ is a disjoint union of fixed points; each fixed point determines a section of $p_Z$. Together with the requirement $f'(z)=x'$, we then get a unique solution.
\end{proof}

\subsection{Digression: Homotopy pullbacks}\label{subsec:HomotopyPullbacks}
After getting acquainted with straightening/unstraightening, our next goal is to prove Yoneda's lemma. But before we can do that, we need to go on a brief detour about \emph{homotopy pullbacks}. These guys will allow us to compute $\Hom_{\Ar(\Cc)}$ and $\Hom_{\Cc_{/y}}$ in terms of $\Hom_\Cc$ for any quasi-category $\Cc$, which will be used countless times throughout the rest of this text. 
\begin{numpar}[\enquote{Definition}.]\label{def:HomotopyPullback}
	Suppose we're given a diagram of Kan complexes or quasi-categories
	\begin{equation*}
		\begin{tikzcd}
			X\rar\dar & X'\dlar[phantom,"\Longleftarrow"{sloped}]\dar\\
			Y\rar & Y'
		\end{tikzcd}\quad\text{or}\quad
		\begin{tikzcd}
			\Cc\rar\dar & \Cc'\dlar[phantom,"\Longleftarrow"{sloped}]\dar\\
			\Dd\rar & \Dd'
		\end{tikzcd}
	\end{equation*}
	that commutes up to homotopy or up to natural equivalence, respectively (so that the corresponding diagram in $\cat{An}$ or $\cat{Cat}_\infty$-commutes; see the discussion in \cref{exm:SimplicialNerve}). We say that the diagram is a \emph{homotopy pullback} if its image in $\cat{An}$ or $\cat{Cat}_\infty$ is a pullback in the $\infty$-categorical sense (which we will only define in \cref{def:Colimits}\cref{enum:Colimit} below).
	
	As stated, this \enquote{definition} doesn't lead to vicious circles, but once you try to prove anything with it, it surely does. So let's just say there is a way to define homotopy pullbacks properly, in any model category. This is done in any sensible treatment of model categories; see \cite[Definition~{\href{https://cisinski.app.uni-regensburg.de/CatLR.pdf\#thm.2.3.22}{2.3.22}}]{Cisinski} or \cite[Definition~VIII.49(vi)]{HigherCatsII}. For the Kan--Quillen model structure and the Joyal model structure on $\cat{sSet}$ (see \cref{exm:KanQuillenModelStructure,exm:JoyalModelStructure}) the above \enquote{definition} is recovered, albeit not obviously so.\hfill$\blacksquare$
\end{numpar}
\begin{numpar}[Model category fact.]\label{par:HomotopyPullback}
	A pullback diagram in a model category is automatically a homotopy pullback diagram if all objects are fibrant and at least one of the legs is a fibration. See \cite[Proposition~{\href{https://cisinski.app.uni-regensburg.de/CatLR.pdf\#thm.2.3.27}{2.3.27}}]{Cisinski} for a proof. In the examples at hand, we deduce:
	\begin{alphanumerate}
		\item A pullback of Kan complexes is automatically a homotopy pullback if at least one if its legs is a Kan fibration.\label{enum:HomotopyPullbackOfKanComplexes}
		\item A pullback of quasi-categories is automatically a homotopy pullback if at least one of its legs is an \emph{isofibration} (or \emph{categorical fibration} in Lurie's terminology). That is, it is an inner fibration and has the lifting property against $\{0\}\rightarrow \N(J)$. Here $J\coloneqq \{\InlineJ\}$ is the category of two objects and a pair of mutually inverse isomorphisms between them, so lifting against $\{0\}\rightarrow \N(J)$ means that we can lift equivalences.\label{enum:HomotopyPullbackOfQuasicategories}
	\end{alphanumerate}
	So homotopy pullbacks, or equivalently, pullbacks in $\cat{An}$ or $\cat{Cat}_\infty$ can be computed as follows: First write down the diagram as a diagram of simplicial sets. Then replace one of its legs by an equivalence followed by a Kan fibration or an isofibration; this can be done by \cref{lem:SmallObjectArgument}.\footnote{To replace a functor of quasi-categories by an equivalence followed by an isofibration, a small variation of the argument from \cref{lem:SmallObjectArgument} is needed. The problem is that $\N(J)$ has countably many non-degenerate simplices, whereas $\Lambda_i^n$ had only finitely many. This has the effect that it's no longer sufficient to iterate the construction of $S(f)$ countably many times. To fix this, we simply do $\aleph_1$ many iterations instead of $\aleph_0$ many.} Finally, take the usual pullback along that Kan or isofibration.
	
	As a consequence, with some care, homotopy pullbacks can usually be manipulated in the same way as ordinary pullbacks. We'll use this freely throughout the rest of this section.\hfill$\blacksquare$
\end{numpar}
\begin{lem}\label{lem:HomInArrowCategories}
	Let $\Cc$ be a quasi-category and let $\alpha\colon x\rightarrow y$, $\alpha'\colon x'\rightarrow y'$ be morphisms in $\Cc$. Then there exists a homotopy pullback diagram of animae
	\begin{equation*}
		\begin{tikzcd}[baseline=(H.base)]
			\Hom_{\Ar(\Cc)}\bigl((\alpha\colon x\rightarrow y),(\alpha'\colon x'\rightarrow y')\bigr)\drar[hpullback]\dar\rar & \Hom_\Cc(y,y')\dar["\alpha^*"]\\
			\Hom_\Cc(x,x')\rar["\alpha'_*"] & |[alias=H]| \Hom_\Cc(x,y') 
		\end{tikzcd}
	\end{equation*}
	Here the pre- and postcomposition maps $\alpha^*$ and $\alpha_*'$ are defined by means of the functors $\Hom_\Cc(x,-)\colon \Cc\rightarrow\cat{An}$ and $\Hom_\Cc(-,y')\colon \Cc^\op\rightarrow \cat{An}$ from \cref{exm:Straightening}\cref{enum:SliceLeftFibration}.\hfill$\blacksquare$
\end{lem}
\begin{rem}
	The only proof I know is in Fabian's handwritten notes \cite[Proposition~VIII.5]{HigherCatsII}. It's not particularly difficult: You work directly with the definition of $\Hom_{\Ar(\Cc)}$ to write it as an honest pullback in which both legs are Kan fibrations. Then you check that the corners of the pullback are homotopy equivalent to $\Hom_\Cc(x,x')$, $\Hom_\Cc(y,y')$, and $\Hom_\Cc(x,y')$, respectively. Along the way, you should also check (but this will be quite apparent from the description in~\cref{par:StraighteningOnMorphisms}) that the maps you obtain are really the pre- and postcomposition maps $\alpha^*$ and $\alpha'_*$ as defined above. 
	
	Also note that in the case where $\Cc$ is an ordinary category we recover our original description of morphisms in an arrow category from \cref{con:1ArrowCategory}. Indeed, in this case all Hom animae are discrete (that is, sets), hence the ambiguity of $\alpha^*$ and $\alpha'_*$ up to homotopy goes away. Furthermore, any map of discrete Kan complexes is automatically a Kan fibration, so the homotopy pullback is a pullback on the nose by model category fact~\cref{par:HomotopyPullback}\cref{enum:HomotopyPullbackOfKanComplexes}. If you think about this pullback briefly, that's exactly how morphisms in $\Ar(\Cc)$ are described.
\end{rem}
\begin{cor}\label{cor:HomInSliceCategories}
	Let $\Cc$ be a quasi-category, let $y\in \Cc$ be an object, and let $\alpha\colon x\rightarrow y$ and $\alpha'\colon x'\rightarrow y$ be morphisms in $\Cc$. Then there exists a homotopy pullback diagram of animae
	\begin{equation*}
		\begin{tikzcd}
			\Hom_{\Cc_{/y}}\bigl((\alpha\colon x\rightarrow y),(\alpha'\colon x'\rightarrow y)\bigr)\drar[hpullback]\dar\rar & \{\alpha\}\dar\\
			\Hom_\Cc(x,x')\rar["\alpha'_*"] & \Hom_\Cc(x,y)
		\end{tikzcd}
	\end{equation*}
	Here the postcomposition map $\alpha'_*$ is again defined as in \cref{lem:HomInArrowCategories}.
\end{cor}
\begin{proof}[Proof sketch]
	We use the following pullback square from \cref{par:HomInQuasiCategories}:
	\begin{equation*}
		\begin{tikzcd}
			\Cc_{/y}\rar\dar\drar[pullback] & \Ar(\Cc)\dar\\
			\{y\}\rar & \Cc
		\end{tikzcd}
	\end{equation*}
	In general, its straightforward to check that Hom in a pullback of quasi-categories is the pullback of Hom in each component. Then we plug in \cref{lem:HomInArrowCategories} and check that everything works out with homotopy pullbacks too. For a complete proof, see \cite[Corollary~VIII.6]{HigherCatsII}, where Fabian deduces the result from \cref{lem:HomInArrowCategories} as we do here, or \cite[Lemma~\HTTthm{5.5.5.12}]{HTT}, in which Lurie gives a direct argument.
\end{proof}
Homotopy pullbacks can be used to give an equivalent characterisation of cocartesian edges. In fact, the terminology \emph{\embrace{co}cartesian morphism} was originally introduced in the classical theory of stacks (see \cref{par:Stacks}), where it was defined using the criterion from \cref{lem:CocartesianMorphisms} below.\footnote{Of course, in the classical theory the homotopy pullback of animae was replaced by an ordinary pullback of sets. But observe that a homotopy pullback of simplicial sets, in which all participating objects are disjoint unions of copies of $*$, must automatically be an ordinary pullback. The reason is that any map between two such simplicial sets is automatically a Kan fibration and any homotopy equivalence is automatically an isomorphism.} The equivalence with \cref{def:Cocartesian}\cref{enum:CocartesianMorphism} is due to Lurie; see \cite[Proposition~\HTTthm{2.4.4.3}]{HTT} or \cite[Corollary~3.1.16]{Land}.
\begin{lem}\label{lem:CocartesianMorphisms}
	Let $p\colon \Uu\rightarrow\Cc$ be an inner fibration of quasi-categories. Then a morphism $\varphi\colon u\rightarrow v$ is $p$-cocartesian if and only if the following diagram is a homotopy pullback of animae for every $w\in\Uu$:
	\begin{equation*}
		\begin{tikzcd}
			\Hom_\Uu(v,w)\dar["p"']\rar["\varphi^*"]\drar[hpullback] & \Hom_\Uu(u,w) \dar["p"]\\
			\Hom_\Cc\bigl(p(v),p(w)\bigr)\rar["p(\varphi)^*"] & \Hom_\Cc\bigl(p(u),p(w)\bigr)
		\end{tikzcd}
	\end{equation*}
	Here the precomposition maps $\varphi^*$ and $p(\varphi)^*$ are once again defined as in \cref{lem:HomInArrowCategories}.\hfill$\blacksquare$
\end{lem}
To finish our excursion into homotopy pullbacks, we introduce a variant of cocartesian fibrations that is occasionally quite useful, but will only play a very minor role in these notes.
\begin{defi}\label{def:LocallyCocartesian}
	Let $p\colon \Uu\rightarrow \Cc$ be an inner fibration of quasi-categories.
	\begin{alphanumerate}
		\item Let $\varphi\colon u\rightarrow v$ be a morphism in $\Uu$, corresponding to a map $\varphi\colon \Delta^1\rightarrow \Uu$. We call $\varphi$ a \emph{locally $p$-cocartesian morphism} if it is $p_{p\circ \varphi}$-cocartesian, where $p_{p\circ \varphi}\colon \Delta^1\times_{p\circ \varphi,\Cc}\Uu\rightarrow \Delta^1$ denotes the pullback of $p$ along $p\circ \varphi\colon \Delta^1\rightarrow \Cc$.
		\item We call $p$ a \emph{locally cocartesian fibration} if the pullback $p_\alpha\colon \Delta^1\times_{\alpha,\Cc}\Uu\rightarrow \Delta^1$ is a cocartesian fibration for every $\alpha\colon \Delta^1\rightarrow \Cc$.
	\end{alphanumerate}
	There are dual notions of \emph{locally $p$-cartesian morphisms} and \emph{locally cartesian fibrations}.
\end{defi}
\begin{cor}\label{cor:LocallyCocartesianComposition}
	Let $p\colon \Uu\rightarrow \Cc$ be a locally cocartesian fibration. Then $p$ is a cocartesian fibration if and only if the set of locally $p$-cartesian morphisms is closed under composition.
\end{cor}
\begin{proof}[Proof sketch]
	First assume that $p$ is a cocartesian fibration. Then cocartesian lifts are unique up to equivalence, as we've seen in \cref{par:StraighteningOnMorphisms}. A morphism being cocartesian is preserved under pullbacks. Hence every $p$-cocartesian morphism $\varphi$ is also $p_{p\circ \varphi}$-cocartesian. The above-mentioned uniqueness then implies that every locally $p$-cocartesian morphism must also be $p$-cocartesian. So locally $p$-cocartesian morphisms being closed under composition reduces to the same assertion about $p$-cocartesian morphisms, which is easy to check (for example, using \cref{lem:CocartesianMorphisms}).\footnote{Closedness under composition is also clear intuitively: If $\beta\colon y\rightarrow z$ is another morphism in  $\Cc$ and we compose a $p$-cocartesian lift of $\alpha$ with a $p$-cocartesian lift of $\beta$, then we have connected an element of $F(x)$ with its image under $F(\beta)\circ F(\alpha)\simeq F(\beta\circ\alpha)\colon F(x)\rightarrow F(z)$. And that's a $p$-cocartesian lift of $\beta\circ \alpha$.}
	
	Conversely, assume that locally $p$-cocartesian morphisms are closed under compositions. Let $\varphi\colon u\rightarrow v$ be locally $p$-cocartesian. We wish to show that $\varphi$ is also $p$-cocartesian. To this end, we'll verify that the diagram from \cref{lem:CocartesianMorphisms} is a homotopy pullback for all $w\in\Uu$. Using \cref{thm:Whitehead}, \cref{lem:LongExactFibrationSequence}, and the five lemma (plus \cref{rem:ExactnessInLowDegrees}), it's enough to show that for every $\alpha\in \Hom_\Cc(p(v),p(w))$, the induced map on homotopy fibres over $\alpha$ is a homotopy equivalence. So fix $\alpha\colon p(v)\rightarrow p(w)$ in $\Cc$. Let $\Uu_\alpha\coloneqq \Delta^1\times_{\alpha,\Cc}\Uu$ be the fibre over $\alpha$. Furthermore, let $\psi\colon v\rightarrow v'$ be a locally $p$-cocartesian lift of $\alpha$ (so that $p(v')=p(w)$). We claim that the homotopy-commutative diagram
	\begin{equation}\label{eq:HomotopyFibre}\tag{$*$}
		\begin{tikzcd}
			\Hom_{\Uu_\alpha}(v',w)\dar\rar["\psi^*"]\drar[hcommutes] & \Hom_\Uu(v,w)\rar["\varphi^*"]\dar["p"]\drar[hcommutes] & \Hom_\Uu(u,w)\dar["p"]\\
			\bigl\{\id_{p(w)}\bigr\}\rar["\alpha^*"] & \Hom_\Cc\bigl(p(v),p(w)\bigr)\rar["p(\varphi)^*"] & \Hom_\Cc\bigl(p(u),p(w)\bigr)
		\end{tikzcd}
	\end{equation}
	exhibits $\Hom_{\Uu_{p(w)}}(v',w)$ both as the homotopy fibre of $\Hom_\Uu(v,w)\rightarrow \Hom_\Cc(p(v),p(w))$ over $\{\alpha\}$ and the homotopy fibre of $\Hom_\Uu(u,w)\rightarrow\Hom_\Cc(p(u),p(w))$ over $\{\alpha\circ p(\varphi)\}$. As explained above, if we could show this, we would be done.
	
	To see this, observe that $\Hom$ animae in pullbacks are given as pullbacks of $\Hom$ animae in the respective factors (which is straightforward to see from \cref{par:HomInQuasiCategories} and we'll see a more general assertion in \cref{lem:HomInLimits}\cref{enum:HomInLimits}). Combining this with the assumption that $\psi$ is locally $p$-cocartesian and \cref{lem:CocartesianMorphisms}, we see that the following diagram consists of a homotopy pullback square and a pullback square on the nose:
	\begin{equation*}
		\begin{tikzcd}
			\Hom_{\Uu_\alpha}(v',w)\dar\rar["\psi^*"]\drar[hpullback] & \Hom_{\Uu_\alpha}(v,w)\rar\dar["p"]\drar[pullback] & \Hom_\Uu(v,w)\dar["p"]\\
			\Hom_{\Delta^1}(0,0)\rar & \Hom_{\Delta^1}(0,1)\rar["\alpha"] & \Hom_\Cc\bigl(p(v),p(w)\bigr)
		\end{tikzcd}
	\end{equation*}
	Hence the outer rectangle must be a homotopy pullback too. Since $\Hom_{\Delta^1}(0,0)\simeq *\simeq \{\id_{p(w)}\}$, it follows that the left square in \cref{eq:HomotopyFibre} is a homotopy pullback. Since, by assumption, any choice of composition $\psi\circ \varphi$ is locally $p$-cocartesian, the same argument can be used to show that the outer rectangle in \cref{eq:HomotopyFibre} is a homotopy pullback too. This proves that  $\Hom_{\Uu_{p(w)}}(v',w)$ agrees with both homotopy fibres in question and we're done.
\end{proof}

\subsection{Yoneda's lemma}\label{subsec:Yoneda}
\begin{thm}[Quasi-categorical Yoneda lemma]\label{thm:Yoneda}
	Let $\Cc$ be a quasi-category, $x\in \Cc$ an object, and $E\colon \Cc\rightarrow \cat{An}$ a functor. Then evaluation at $\id_x$ induces an equivalence of animae
	\begin{equation*}
		\ev_{\id_x}\colon \Hom_{\F(\Cc,\cat{An})}\bigl(\Hom_\Cc(x,-),E\bigr)\overset{\simeq}{\longrightarrow} E(x)\,.
	\end{equation*}
	Here $\Hom_\Cc(x,-)\colon \Cc\rightarrow \cat{An}$ is the functor from \cref{exm:Straightening}\cref{enum:SliceLeftFibration}. A dual statement holds for the contravariant Hom functor $\Hom_\Cc(-,x)$ and $\F(\Cc^\op,\cat{An})$.
\end{thm}
For the proof we need, more or less, the fact that $*\simeq \{\id_x\}\rightarrow \Cc_{x/}$ is a left anodyne map. This is proved in \cite[Lemma~4.1.4]{Land} or \cite[Corollary~D.7]{HigherCatsII}. Their proofs use some constructions we haven't mentioned yet, but we can circumvent these at the cost of showing a slightly weaker statement, which will still be sufficient for our purposes.
\begin{lem}\label{lem:WeaklyLeftAnodyne}
	Let $\Cc$ be a quasi-category and $x\in\Cc$ an object. For every left fibration $X\rightarrow \Cc$, the natural map
	\begin{equation*}
		\F\bigl(\Cc_{x/},X\bigr)\overset{\simeq}{\longrightarrow}\F(*,X)\times_{\F(*,\Cc)}\F\bigl(\Cc_{x/},\Cc\bigr)
	\end{equation*}
	is an equivalence of quasi-categories.
\end{lem}
\begin{proof}[Proof sketch]
	We call a cofibration $A\rightarrow B$ of simplicial sets \emph{weakly left anodyne} if the natural map $\F(B,X)\rightarrow \F(A,X)\times_{\F(A,\Cc)}\F(B,\Cc)$ is an equivalence of quasi-categories for all left fibrations $X\rightarrow \Cc$. Every left anodyne map is weakly left anodyne by \cref{cor:FKanFibration}. Our goal is to show that $*\rightarrow \Cc_{x/}$ is weakly left anodyne.
	
	The idea to show this is as follows: Intuitively, it's clear that $\id_x\in \Cc_{x/}$ is an initial object. Therefore, there should be a natural transformation $\eta\colon \const\{\id_x\}\Rightarrow \id_{\Cc_{x/}}$. So $\eta$ witnesses the fact that $*\rightarrow \Cc_{x/}$ is a homotopy equivalence---except that $\Cc_{x/}$ is not an anima. Still, as we'll see, $\eta$ can then be leveraged to show the desired statement.
	
	To construct $\eta$, we can proceed as follows: The identity on $\Ar(\Cc)$ is adjoint to a map $\Ar(\Cc)\times\Delta^1\rightarrow \Cc$. Combining this with the map $\Cc\simeq \F(*,\Cc)\rightarrow \F(\Delta^1,\Cc)\simeq \Ar(\Cc)$ induced by $\Delta^1\rightarrow *$ provides a map $\Ar(\Cc)\times\Delta^1\rightarrow \Ar(\Cc)$; restricting this to $\Cc_{x/}$ yields the desired map $\eta\colon \Cc_{x/}\times\Delta^1\rightarrow \Cc_{x/}$. Putting $\Cc_{x/}^\triangleleft\coloneqq (\Cc_{x/}\times\Delta^1)/(\Cc_{x/}\times\{0\})$, it's easy to check that $\eta$ factors through a map $\ov\eta\colon \Cc_{x/}^\triangleleft\rightarrow \Cc_{x/}$. The map $*\rightarrow \Cc_{x/}$ induces a map $\Delta^1\simeq *^\triangleleft\rightarrow \Cc_{x/}^\triangleleft$. This fits into a diagram
	\begin{equation*}
		\begin{tikzcd}
			*\rar\dar\drar[commutes] & \Delta^1\rar\dar\drar[commutes] & *\dar\\
			\Cc_{x/}\rar & \Cc_{x/}^\triangleleft\rar["\ov\eta"] & \Cc_{x/}
		\end{tikzcd}
	\end{equation*}
	which exhibits $*\rightarrow \Cc_{x/}$ as a retract of $\Delta^1\rightarrow \Cc_{x/}^\triangleleft$. Hence to show that $*\rightarrow \Cc_{x/}$ is weakly left anodyne, it's enough to show the same for $\Delta^1\rightarrow \Cc_{x/}^\triangleleft$. For this, note that $\{0\}\rightarrow \Delta^1$ is left anodyne, and so is the composition $\{0\}\rightarrow \Delta^1\rightarrow \Cc_{x/}^\triangleleft$ since it is a pushout of $\Cc_{x/}\times\{0\}\rightarrow \Cc_{x/}\times \Delta^1$. It's easy to check that being weakly left anodyne is closed under 2-out-of-3, and so $\Delta^1\rightarrow \Cc_{x/}^\triangleleft$ must be weakly left anodyne too.
\end{proof}
\begin{proof}[Proof sketch of \cref{thm:Yoneda}]
	Let $p\colon \Uu\rightarrow\Cc$ be the unstraightening of $E\colon \Cc\rightarrow \cat{An}$. Then
	\begin{align*}
		\Hom_{\F(\Cc,\cat{An})}\bigl(\Hom_\Cc(x,-),E\bigr)&\simeq \Hom_{\cat{Left}(\Cc)}\bigl(t\colon \Cc_{x/}\rightarrow \Cc,p\colon \Uu\rightarrow\Cc\bigr)\\
		&\simeq \Hom_{\cat{Cat}_{\infty/\Cc}}\bigl(t\colon \Cc_{x/}\rightarrow \Cc,p\colon \Uu\rightarrow\Cc\bigr)
	\end{align*}
	using the straightening equivalence (\cref{thm:Straightening}\cref{enum:LeftStraightening}) and the fact that $\cat{Left}(\Cc)\rightarrow{\cat{Cat}_\infty}_{/\Cc}$ is fully faithful. By \cref{cor:HomInSliceCategories}, the diagram
	\begin{equation*}
		\begin{tikzcd}
			\Hom_{\cat{Cat}_{\infty/\Cc}}\bigl((t\colon \Cc_{x/}\rightarrow \Cc),(p\colon \Uu\rightarrow\Cc)\bigr)\rar\dar\drar[hpullback] & *\dar\\
			\Hom_{\cat{Cat}_\infty}\bigl(\Cc_{x/},\Uu\bigr)\rar & \Hom_{\cat{Cat}_\infty}\bigl(\Cc_{x/},\Cc\bigr)
		\end{tikzcd}
	\end{equation*}
	is a homotopy pullback, where $*$ is sent to $t\colon \Cc_{x/}\rightarrow \Cc$. Now recall from \cref{thm:CordierPorter} that $\Hom_{\cat{Cat}_\infty}(-,-)\simeq \core \F(-,-)$. Furthermore, we claim that the following diagrams are homotopy pullbacks:
	\begin{equation*}
		\begin{tikzcd}
			\F\bigl(\Cc_{x/},\Uu\bigr)\rar\dar\drar[hpullback] & \F\bigl(\Cc_{x/},\Cc\bigr)\dar\\
			\Uu\rar & \Cc
		\end{tikzcd}\quad\text{and}\quad \begin{tikzcd}
			\core\F\bigl(\Cc_{x/},\Uu\bigr)\rar\dar\drar[hpullback] & \core\F\bigl(\Cc_{x/},\Cc\bigr)\dar\\
			\core(\Uu)\rar & \core (\Cc)
		\end{tikzcd}
	\end{equation*}
	For the left one, we use \cref{lem:WeaklyLeftAnodyne} and model category fact~\cref{par:HomotopyPullback}\cref{enum:HomotopyPullbackOfQuasicategories}: We only need to check that $p\colon \Uu\rightarrow\Cc$ is an isofibration. But any left fibration has lifting against $\{0\}\rightarrow \N(J)$, as this map is left anodyne (by an explicit horn filling argument). To see that the right square is a homotopy pullback too, we need to check that $\core\colon \cat{QCat}\rightarrow \cat{Kan}$ preserves homotopy pullbacks. The deeper reason for this is of course that $\core\colon \cat{Cat}_\infty\rightarrow\cat{An}$ is right adjoint to the inclusion $\cat{An}\subseteq\cat{Cat}_\infty$ (see \cref{exm:Adjunctions}\cref{enum:AnToCatInfty}). For a direct argument, we can use \cref{par:HomotopyPullback}: By an easy application of Joyal's lifting theorem (\cref{thm:JoyalLifting}), $\core$ transforms isofibrations into Kan fibrations and then by arguments as in the proof of \cref{thm:EquivalenceFullyFaithfulEssentiallySurjective} we can show that $\core$ preserves pullbacks of quasi-categories in which at least one leg is an isofibration.
	
	Combining the homotopy pullbacks so far (this kind of manipulation is fine by~\cref{par:HomotopyPullback}), we find that
	\begin{equation*}
		\begin{tikzcd}
			\Hom_{\cat{Cat}_{\infty/\Cc}}\bigl((t\colon \Cc_{x/}\rightarrow \Cc),(p\colon \Uu\rightarrow\Cc)\bigr)\rar\dar\drar[hpullback] & *\dar\\
			\core(\Uu)\rar & \core(\Cc)
		\end{tikzcd}
	\end{equation*}
	is a homotopy pullback, where $*$ is sent to $x\in\core (\Cc)$. As observed above, $\Uu\rightarrow\Cc$ is an isofibration and so $\core (\Uu)\rightarrow (\Cc)$ is a Kan fibration. Thus, the homotopy pullback agrees with the ordinary pullback. Now $\core(\Uu)\times_{\core(\Cc)}\{x\}\cong \core(\Uu\times_\Cc\{x\})\cong \Uu\times_\Cc\{x\}\simeq F(x)$ using that the fibres $\Uu\times_\Cc\{x\}\eqqcolon p^{-1}\{x\}$ of $p$ are animae and compute the values of $F$. This is what we wanted to prove.
\end{proof}
Finally, we would like to construct the functor $\Hom_\Cc\colon \Cc^\op\times\Cc\rightarrow\cat{An}$. There are several ways to do this and we'll outline two possibilities in \cref{con:HomInTwoVariables,con:HomTwAr} below. We won't prove that they are equivalent (they are), but we won't ever need that either. So you can just choose whichever is your favourite.
\begin{con}\label{con:HomInTwoVariables}
	Consider the functor $\Cc_{/-}\colon \Cc\rightarrow \cat{Cat}_\infty$ from \cref{exm:Straightening}\cref{enum:ArCocartesianFibration}. For every $x\in \Cc$ there is a natural functor $\Cc_{/x}\rightarrow \Cc$, so we would expect that $\Cc_{/-}$ lifts to a functor $\Cc_{/-}\colon \Cc\rightarrow \cat{Cat}_{\infty/\Cc}$. To construct such a lift, first note that for all quasi-categories $\Dd$ and all $y\in \Dd$, we have an equivalence  $\F(\Cc,\Dd_{/y})\simeq\F(\Cc,\Dd)_{/\const y}$ (in fact, even an isomorphism of simplicial sets), as can be checked by a simple calculation. Furthermore, \cref{thm:Straightening}\cref{enum:CocartesianStraightening} and \cref{exm:Straightening}\cref{enum:ProjectionsStraightenToConstantFunctors} imply that $\F(\Cc,\cat{Cat}_\infty)_{/\const \Cc}\simeq \cat{Cocart}(\Cc)_{/(\pr_2\colon\Cc\times \Cc\rightarrow \Cc)}$ holds. So to lift our functor $\Cc_{/-}$ to $\cat{Cat}_{\infty/\Cc}$, it suffices to observe that the following diagram is a morphism of cocartesian fibrations over $\Cc$:
	\begin{equation*}
		\begin{tikzcd}
			\Ar(\Cc)\ar[r,"{(s,t)}"]\dar["\smash{t}\vphantom{\pr_2}"']\dar[phantom,""{name=A}]\arrow[from=1-2,to=A,commutes,pos=0.7]& \Cc\times\Cc\dlar["\pr_2"]\\
			\Cc &
		\end{tikzcd}
	\end{equation*}
	By the dual of \cref{cor:HomAnima}, $\Cc_{/x}\rightarrow \Cc$ is a right fibration for all $x\in \Cc$, hence $\Cc_{/-}\colon \Cc\rightarrow \cat{Cat}_{\infty/\Cc}$ takes values in the full sub-quasi-category $\cat{Right}(\Cc)\subseteq \cat{Cat}_{\infty/\Cc}$. Since $\cat{Right}(\Cc)\simeq \F(\Cc^\op,\cat{An})$ by the dual of \cref{thm:Straightening}\cref{enum:LeftStraightening}, we obtain a functor
	\begin{equation*}
		\Yo_\Cc\colon \Cc\xrightarrow{\Cc_{/-}} \cat{Right}(\Cc)\xrightarrow{\operatorname{St}^{(\mathrm{right})}}\F(\Cc^\op,\cat{An})\,,
	\end{equation*}
	which we take as our definition of the Yoneda embedding (we'll see in \cref{cor:YonedaEmbeddingFullyFaithful} below that it is indeed fully faithful). Finally, we let $\Hom_\Cc\colon \Cc^\op\times\Cc\rightarrow \cat{An}$ be the image of $\Yo_\Cc$ under the \enquote{currying} equivalence $\F(\Cc,\F(\Cc^\op,\cat{An}))\simeq \F(\Cc^\op\times \Cc,\cat{An})$. We define the \emph{twisted arrow quasi-category} $(s,t)\colon \TwAr(\Cc)\rightarrow\Cc^\op\times\Cc$ to be the unstraightening of $\Hom_\Cc\colon \Cc^\op\times\Cc\rightarrow\cat{An}$ via \cref{thm:Straightening}\cref{enum:LeftStraightening}.
\end{con}
\cref{con:HomInTwoVariables} has the advantage that it allows for a straightforward proof of \cref{cor:YonedaEmbeddingFullyFaithful} below. On the downside, however, the unstraightening $\TwAr(\Cc)$ is very inexplicit in this description. So alternatively, one can write down an explicit simplicial model for $(s,t)\colon\TwAr(\Cc)\rightarrow \Cc^\op\times\Cc$ and define $\Hom_\Cc$ to be its straightening.
\begin{con}\label{con:HomTwAr}
	For an ordinary category $\Cc$, we define $\TwAr(\Cc)$ to be the ordinary category whose objects are arrows $\alpha\colon x\rightarrow y$ in $\Cc$ and whose morphisms $(\alpha\colon x\rightarrow y)\rightarrow (\alpha'\colon x'\rightarrow y')$ are \enquote{twisted} commutative squares
	\begin{equation*}
		\begin{tikzcd}
			x\dar["\alpha"']\drar[commutes] & x'\lar\dar["\alpha'"]\\
			y\rar & y'
		\end{tikzcd}
	\end{equation*}
	There are functors $s\colon \TwAr(\Cc)\rightarrow \Cc^\op$ and $t\colon \TwAr(\Cc)\rightarrow \Cc$ that send $\alpha\colon x\rightarrow y$ to $x$ and $y$, respectively. For a quasi-category $\Cc$, we re-define $\TwAr(\Cc)$ to be the simplicial set given by
	\begin{equation*}
		\TwAr(\Cc)_n\coloneqq \Hom_{\cat{sSet}}
		\bigl(\N\bigl([n]^\op\star [n]\bigr),\Cc\bigr)\,.
	\end{equation*}
	Here $[n]^\op\star[n]$ is the \emph{join} of the totally ordered sets $[n]^\op$ and $[n]$. In general, if $\Ii$ and $\Jj$ are ordinary categories, we let $\Ii\star\Jj$ be the category obtained from the disjoint union $\Ii\sqcup\Jj$ by adding precisely one morphism $i\rightarrow j$ for all $i\in\Ii$, $j\in\Jj$.
	
	The natural maps $(\Delta^n)^\op\cong \N([n]^\op)\rightarrow \N([n]^\op\star [n])$ and $\Delta^n\cong \N([n])\rightarrow \N([n]^\op\star[n])$ induce maps of simplicial sets $s\colon \TwAr(\Cc)\rightarrow \Cc^\op$ and $t\colon\TwAr(\Cc)\rightarrow \Cc$. It turns out that $(s,t)\colon \TwAr(\Cc)\rightarrow \Cc^\op\times \Cc$ is always a left fibration; in particular, $\TwAr(\Cc)$ is a quasi-category. See \cite[Proposition~\HAthm{5.2.1.3}]{HA} or \cite[Proposition~4.2.4]{Land} for proofs. We can then define $\Hom_\Cc\colon \Cc^\op\times \Cc\rightarrow \cat{An}$ to be the straightening of $(s,t)\colon \TwAr(\Cc)\rightarrow \Cc^\op\times \Cc$.
\end{con}
\begin{rem}\label{rem:TwAr}
	If $\Cc$ is an ordinary category, then $\N(\TwAr(\Cc))\simeq \TwAr(\N(\Cc))$, no matter how you define the right-hand side. So the notational overload checks out. If you use \cref{con:HomTwAr}, this equivalence is even an isomorphism of simplicial sets and straightforward to verify. If you use \cref{con:HomInTwoVariables} instead, the proof is still not too hard, but it requires you to know how straightening/unstraightening works under the hood, at least for ordinary categories (in which case straightening/unstraightening is known as the \emph{Grothendieck construction}). 
\end{rem}
Let's do three quick reality checks for our newly constructed functor $\Hom_\Cc$:
\begin{lem}\label{lem:HomRealityCheck}
	Let $\Hom_\Cc\colon \Cc^\op\times\Cc\rightarrow\cat{An}$ be the functor from \cref{con:HomInTwoVariables} or from \cref{con:HomTwAr}. For all $x,y\in\Cc$, the restrictions
	\begin{equation*}
		\Hom_\Cc|_{\{x\}\times\Cc}\colon \Cc\longrightarrow\cat{An}\quad\text{and}\quad \Hom_\Cc|_{\Cc^\op\times\{y\}}\colon \Cc^\op\longrightarrow \cat{An}
	\end{equation*}
	agree with the functors $\Hom_\Cc(x,-)$ and $\Hom_\Cc(-,y)$ constructed in \cref{exm:Straightening}\cref{enum:SliceLeftFibration}.
\end{lem}
\begin{proof}[Proof sketch, assuming \cref{con:HomInTwoVariables}]
	It's straightforward to see from the construction that $\Hom_\Cc|_{\Cc^\op\times\{y\}}\colon \Cc^\op\rightarrow\cat{An}$ is the straightening of the right fibration $\Cc_{/y}\rightarrow \Cc$, which is also the definition of $\Hom_\Cc(-,y)$ in \cref{exm:Straightening}\cref{enum:SliceLeftFibration}. Now let $\Uu\rightarrow\Cc$ be the unstraightening of $\Hom_\Cc|_{\{x\}\times\Cc}$. Note that evaluating a functor $T\colon\Cc^\op\rightarrow\cat{An}$ at $x\in \Cc^\op$ is the same as restriction along $\F(\Cc^\op,\cat{An})\rightarrow\F(\{x\},\cat{An})\simeq \cat{An}$. By the dual of \cref{thm:Straightening}\cref{enum:LeftStraightening}, this corresponds to the pullback functor $x^*\colon \cat{Right}(\Cc)\rightarrow\cat{Right}(\{x\})\simeq \cat{An}$ under the right straightening equivalence. So $\Hom_\Cc|_{\{x\}\times\Cc}$ can be described as the composition
	\begin{equation*}
		\Hom_\Cc|_{\{x\}\times\Cc}\colon\Cc\xrightarrow{\Cc_{/-}}\cat{Right}(\Cc)\overset{x^*}{\longrightarrow}\cat{Right}\bigl(\{x\}\bigr)\simeq \cat{An}\,.
	\end{equation*}
	By \enquote{inspection}\footnote{Unfortunately, verifying that the above diagram is a homotopy pullback requires us to know a little more about how the straightening/unstraightening equivalence is constructed. The idea is to rewrite the equivalence $\Hom_\Cc|_{\{x\}\times\Cc}\simeq x^*\circ (\Cc_{/-})$ as a pullback $\Hom_\Cc|_{\{x\}\times\Cc}\simeq (\Cc_{/-})\times_{\const \Cc}\const \{x\}$ in the functor quasi-category $\F(\Cc,\cat{Cat}_\infty)$ and then to transform this into a pullback in the quasi-category $\cat{Cocart}(\Cc)$ via \cref{thm:Straightening}\cref{enum:CocartesianStraightening}. This immediately yields that the diagram is a pullback in $\cat{Cocart}(\Cc)$, hence a homotopy pullback of quasi-categories. However, to make this argument work as stated, we would need \cref{lem:ColimitsInFunctorCategories} below, which would lead to circular reasoning. So instead, one has to show that $x^*\circ (\Cc_{/-})$ can be written as a homotopy pullback in a suitable simplicial model category whose underlying quasi-category (in the sense of \cref{rem:SimplicialModelCategory,rem:ModelCategoryUnderlyingInftyCategory}) is $\F(\Cc,\cat{Cat}_\infty)$. In fact, the proof of \cref{thm:Straightening} works by deducing it from a Quillen equivalence between simplicial model categories \ldots
	}, this means that the following diagram is a homotopy pullback:
	\begin{equation*}
		\begin{tikzcd}
			\Uu\dar\rar\drar[hpullback] & \{x\}\times\Cc\dar\\
			\Ar(\Cc)\rar["{(s,t)}"] & \Cc\times\Cc
		\end{tikzcd}
	\end{equation*}
	To compute this homotopy pullback, we use model category fact~\cref{par:HomotopyPullback}\cref{enum:HomotopyPullbackOfQuasicategories}: We claim that $(s,t)\colon \Ar(\Cc)\rightarrow \Cc\times\Cc$ is already an isofibration. Indeed, it's an inner fibration by \cref{cor:FKanFibration} and lifting of equivalences follows easily from \cref{thm:EquivalencePointwise}. So we can just take the pullback on the nose, which is $\Cc_{x/}$ by \cref{par:HomInQuasiCategories}. But $\Hom_\Cc(x,-)\colon \Cc\rightarrow \cat{An}$ was defined to be the straightening of $t\colon \Cc_{x/}\rightarrow \Cc$. This shows $\Hom_\Cc|_{\{x\}\times\Cc}\simeq \Hom_\Cc(x,-)$.	
\end{proof}
\begin{proof}[Proof sketch, assuming \cref{con:HomTwAr}]
	In this case, we need to show that the pullbacks of $(s,t)\colon \TwAr(\Cc)\rightarrow \Cc^\op\times\Cc$ along $\{x\}\times\Cc\rightarrow\Cc^\op\times\Cc$ and $\Cc^\op\times\{y\}\rightarrow\Cc^\op\times\Cc$ are equivalent to $\Cc_{x/}\rightarrow\Cc$ and $(\Cc_{/y})^\op\rightarrow\Cc^\op$, respectively. This is not quite trivial; see \cite[Proposition~\HAthm{5.2.1.10}]{HA} or \cite[Lemma~4.2.7]{Land}.
\end{proof}
\begin{lem}\label{lem:HomFunctorial}
	Let $F\colon \Cc\rightarrow\Dd$ be a functor of quasi-categories. Then the natural maps $\Hom_\Cc(x,y)\rightarrow \Hom_\Dd(F(x),F(y))$ assemble into a natural transformation
	\begin{equation*}
		\Hom_\Cc(-,-)\Longrightarrow \Hom_\Dd\bigl(F(-),F(-)\bigr)
	\end{equation*}
	in $\F(\Cc^\op\times\Cc,\cat{An})$. Here $\Hom_\Cc$ and $\Hom_\Dd$ are the functors from Constructions~\labelcref{con:HomInTwoVariables} or~\labelcref{con:HomTwAr}.
\end{lem}
\begin{proof}[Proof sketch, assuming \cref{con:HomInTwoVariables}]
	Consider the morphism $t\colon \Cc\times_{F,\Dd,s}\Ar(\Dd)\rightarrow \Dd$. It is a cocartesian fibration, which can be shown using \cref{lem:HomInArrowCategories,lem:CocartesianMorphisms} in the same way as \cref{exm:Straightening}\cref{enum:ArCocartesianFibration}. Hence the following diagram is a diagram of cocartesian fibrations over $\Cc$:
	\begin{equation*}
		\begin{tikzcd}
			\Ar(\Cc)\drar["t"']\rar & \Cc\times_{F,\Dd,s}\Ar(\Dd)\times_{t,\Dd,F}\Cc\dar["t"]\dar[phantom,""{name=A}]\rar["{(s,t)}"]\arrow[from=1-1,to=A,commutes,pos=0.7]\arrow[from=1-3,to=A,commutes,pos=0.7]& \Cc\times\Cc \dlar["\pr_2"]\\
			& \Cc & 
		\end{tikzcd}
	\end{equation*}
	After unravelling \cref{con:HomInTwoVariables} and using the fact that precompositions correspond to pullbacks under straightening/unstraightening (see \cref{thm:Straightening}\cref{enum:CocartesianStraightening}), the diagram above will induce the desired natural transformation $\Hom_\Cc(-,-)\Rightarrow \Hom_\Dd(F(-),F(-))$, provided we can show the following claim:
	\begin{alphanumerate}\itshape
		\item[\boxtimes] \!The cocartesian straightening of the middle vertical arrow is the composite functor\label{enum:HomNaturalTransformation}
		\begin{equation*}
			\Cc\overset{F}{\longrightarrow}\Dd\xrightarrow{\Dd_{/-}}\cat{Cat}_{\infty/\Dd}\overset{F^*}{\longrightarrow}\cat{Cat}_{\infty/\Cc}\,.
		\end{equation*}
	\end{alphanumerate}
	To prove \cref{enum:HomNaturalTransformation} it's enough to show that the straightening of $t\colon \Cc\times_{F,\Dd,s}\Ar(\Dd)\rightarrow \Dd$ is $F^*\circ \Dd_{/-}$, since once again, precompositions correspond to pullbacks. Now the following diagram is a homotopy pullback:
	\begin{equation*}
		\begin{tikzcd}
			\Cc\times_{F,\Dd,s}\Ar(\Dd)\dar\rar\drar[hpullback] & \Cc\times\Dd\dar["F\times {\id_\Dd}"]\\
			\Ar(\Dd)\rar["{(s,t)}"] & \Dd\times\Dd
		\end{tikzcd}
	\end{equation*}
	(in fact, it's a pullback on the nose, and $(s,t)\colon \Ar(\Dd)\rightarrow \Dd\times\Dd$ is an isofibration; see the argument in the proof of \cref{lem:HomRealityCheck}). By a similar \enquote{inspection} as in the proof of \cref{lem:HomRealityCheck}, this observation shows that the straightening of $t\colon \Cc\times_{F,\Dd,s}\Ar(\Dd)\rightarrow \Dd$ is indeed $F^*\circ \Dd_{/-}$, thus proving \cref{enum:HomNaturalTransformation}.
\end{proof}
\begin{proof}[Proof sketch, assuming \cref{con:HomTwAr}]
	It's clear from the construction that $F$ induces a map $\TwAr(\Cc)\rightarrow\TwAr(\Dd)$. By the universal property of pullbacks, this factors over a map
	\begin{equation*}
		\TwAr(\Cc)\longrightarrow (\Cc^\op\times\Cc)\times_{F^\op\times F,\Dd^\op\times\Dd,(s,t)}\TwAr(\Dd)\,.
	\end{equation*}
	The latter is a morphism of left fibrations over $\Cc^\op\times\Cc$. Since precompositions correspond to pullbacks under straightening/unstraightening (see \cref{thm:Straightening}\cref{enum:CocartesianStraightening}), we get a natural transformation $\Hom_\Cc(-,-)\Rightarrow \Hom_\Dd(-,-)\circ (F^\op\times F)\simeq \Hom_\Dd(F(-),F(-))$, as desired.
\end{proof}
\begin{lem}\label{lem:HomNaturalTransformation}
	Let $F,G\colon \Cc\rightarrow \Dd$ be functors of quasi-categories and let $\eta\colon F\Rightarrow G$ be a natural transformation. Then the natural transformation from \cref{lem:HomFunctorial} as well as $\eta_*$, postcomposition with $\eta$, and $\eta^*$, precomposition with $\eta$, fit into a commutative diagram
	\begin{equation*}
		\begin{tikzcd}
			\Hom_\Cc(-,-)\doublear["F"{black,swap,left=0.1em}]{d}\doublear["G"{black,above=0.1em}]{r}\drar[commutes] & \Hom_\Dd\bigl(G(-),G(-)\bigr)\doublear["\eta^*"{black,right=0.1em}]{d}\\
			\Hom_\Dd\bigl(F(-),F(-)\bigr)\doublear["\eta_*"{black,above=0.1em}]{r} & \Hom_\Dd\bigl(F(-),G(-)\bigr)
		\end{tikzcd}
	\end{equation*}
	in the quasi-category $\F(\Cc^\op\times\Cc,\cat{An})$.
\end{lem}
\begin{proof}[Proof sketch]
	First observe that if $\Cc'$ is another quasi-category, then $\Hom_{\Cc\times\Cc'}\simeq \Hom_\Cc\times\Hom_{\Cc'}$ holds in $\F((\Cc\times\Cc')^\op\times (\Cc\times\Cc'),\cat{An})$. Depending on whether you use \cref{con:HomInTwoVariables} or \cref{con:HomTwAr}, this basically reduces to the observations that $\Ar(\Cc\times\Cc')\cong \Ar(\Cc)\times\Ar(\Cc')$ and $\TwAr(\Cc\times\Cc')\cong \TwAr(\Cc)\times\TwAr(\Cc')$, respectively, but in each case you need some model-category arguments to do the reduction, similar to the \enquote{inspection} in the proof of \cref{lem:HomRealityCheck}. We'll skip these arguments.
	
	Now we regard $\eta$ as a functor $\eta\colon \Delta^1\times\Cc\rightarrow \Dd$. Then \cref{lem:HomFunctorial} can be applied to $\eta$ to obtain a natural transformation
	\begin{equation*}
		\Hom_{\Delta^1\times\Cc}(-,-)\Longrightarrow \Hom_\Dd\bigl(\eta(-),\eta(-)\bigr)
	\end{equation*}
	in the functor quasi-category $\F((\Delta^1\times\Cc)^\op\times(\Delta^1\times\Cc),\cat{An})$. Applying the usual \enquote{currying} isomorphism, we obtain $\F((\Delta^1\times\Cc)^\op\times(\Delta^1\times\Cc),\cat{An})\cong \F((\Delta^1)^\op\times\Delta^1,\F(\Cc^\op\times\Cc,\cat{An}))$. Since $(\Delta^1)^\op\times\Delta^1\cong \Delta^1\times\Delta^1\cong \square^2$, an object in $\F((\Delta^1)^\op\times\Delta^1,\F(\Cc^\op\times\Cc,\cat{An}))$ corresponds to a commutative square in $\F(\Cc^\op\times\Cc,\cat{An})$. By unravelling the constructions, $\Hom_{\Delta^1\times\Cc}(-,-)$ corresponds to the following square:
	\begin{equation*}
		\begin{tikzcd}
			\Hom_\Cc(-,-)\times\Hom_{\Delta^1}(1,0)\doublear{r}\doublear{d}\drar[commutes] & \Hom_\Cc(-,-)\times \Hom_{\Delta^1}(1,1)\doublear{d}\\
			\Hom_\Cc(-,-)\times \Hom_{\Delta^1}(0,0)\doublear{r} & \Hom_\Cc(-,-)\times \Hom_{\Delta^1}(0,1)
		\end{tikzcd}
	\end{equation*}
	Indeed, this follows from the fact that $\Hom_{\Delta^1\times\Cc}\simeq \Hom_{\Delta^1}\times\Hom_\Cc$, as we've checked above; also note that $\Hom_{\Delta^1}(1,0)$ sits in the top left corner rather than $\Hom_{\Delta^1}(0,0)$ because of the way in which we identified $(\Delta^1)^\op\times\Delta^1$ with $\square^2$. Now observe that $\Hom_{\Delta^1}(1,0)\cong \emptyset$, whereas $\Hom_{\Delta^1}(0,0)\cong \Hom_{\Delta^1}(0,1)\cong \Hom_{\Delta^1}(1,1)\cong *$. This implies that the top left corner of the diagram above is $\const \emptyset$, whereas the other corners are given by $\Hom_\Cc(-,-)$. Similarly, $\Hom_\Dd(\eta(-),\eta(-))$ corresponds to the following commutative square:
	\begin{equation*}
		\begin{tikzcd}
			\Hom_\Dd\bigl(G(-),F(-)\bigr)\doublear["\eta_*"{black,above=0.1em}]{r}\doublear["\eta^*"{black,left=0.1em}]{d}\drar[commutes] & \Hom_\Dd\bigl(G(-),G(-)\bigr)\doublear["\eta^*"{black,right=0.1em}]{d}\\
			\Hom_\Dd\bigl(F(-),F(-)\bigr)\doublear["\eta_*"{black,above=0.1em}]{r} & \Hom_\Dd\bigl(F(-),G(-)\bigr)
		\end{tikzcd}
	\end{equation*}
	The natural transformation $\Hom_{\Delta^1\times\Cc}(-,-)\Rightarrow \Hom_\Dd(\eta(-),\eta(-))$ from \cref{lem:HomFunctorial} corresponds to a morphism between these commutative squares. By inspection, this yields the desired commutative square.
\end{proof}
%The embedding version of Yoneda's lemma is now a simple consequence.
\begin{cor}\label{cor:YonedaEmbeddingFullyFaithful}
	For every quasi-category $\Cc$, the Yoneda embedding $\Yo_\Cc\colon \Cc\rightarrow\F(\Cc^\op,\cat{An})$ is fully faithful.
\end{cor}
\begin{proof}[Proof sketch]
	We must show that $\Yo_\Cc$ induces equivalences
	\begin{equation*}
		\Hom_\Cc(x,y)\overset{\simeq}{\longrightarrow} \Hom_{\F(\Cc^\op,\cat{An})}\bigl(\Hom_\Cc(-,x),\Hom_\Cc(-,y)\bigr)
	\end{equation*}
	for all $x,y\in\Cc$. It's clear from \cref{thm:Yoneda} that both sides are equivalent via evaluation at $\id_x$. If you go with \cref{con:HomInTwoVariables}, it's straightforward to see that the morphism induced by $\Yo_\Cc$ is an inverse to evaluation at $\id_x$, so it is an equivalence too. If you prefer \cref{con:HomTwAr}, this needs a little more work, which we omit.
\end{proof}
We finish this section with two final remarks.
\begin{numpar}[Model independence.]\label{par:ModelIndependence}
	Recall from \cref{par:ModelIndependenceIntro} that, at least in an ideal version of these notes, we planned to proceed via the following steps:
	\begin{alphanumerate}
		\item First, throughout \crefrange{sec:SimplicialSets}{sec:Straightening}, we would set up the framework of quasi-categories.\label{enum:SetupQuasicategoriesNotIntro}
		\item After that, we would identify a few key model-independent statements and prove (or black box) them in the model of quasi-categories.\label{enum:KeyStatementsNotIntro}
		\item Finally, starting from \cref{sec:InftyCategoryTheory}, all further proofs would be done in a model-independent fashion.\label{enum:AllProofsModelIndependentNotIntro}
	\end{alphanumerate}
	Step~\cref{enum:SetupQuasicategoriesNotIntro} is done by now, whereas step~\cref{enum:AllProofsModelIndependentNotIntro} lays ahead. So let's talk about what the key statements from step~\cref{enum:KeyStatementsNotIntro} are supposed to be. Of course, \cref{thm:EquivalencePointwise,thm:EquivalenceFullyFaithfulEssentiallySurjective} are among them, as are \cref{thm:Yoneda} as well as Lemmas~\labelcref{lem:HomRealityCheck}, \labelcref{lem:HomFunctorial}, and \labelcref{lem:HomNaturalTransformation}. In each of these cases, it's clear that the statement is really a model-independent one, even though we formulated them in the model of quasi-categories. Somewhat surprisingly though, \cref{thm:Straightening} can also be reformulated in a model-independent way. In particular, there are model-independent definitions of cocartesian and left fibrations! Indeed, if $p\colon \Uu\rightarrow \Cc$ is a functor of $\infty$-categories, \cref{lem:CocartesianMorphisms} provides a model-independent definition of a morphism $\varphi\colon u\rightarrow v$ in $\Uu$ being $p$-cocartesian.%\footnote{Of course, instead of \enquote{homotopy pullback}, one has to say \enquote{pullback square in the $\infty$-category $\cat{An}$}, which is a model-independent notion as will hopefully become clear in \cref{def:Colimits}.}
	The condition that cocartesian lifts exist (\cref{def:Cocartesian}\cref{enum:CocartesianFibration}) can be replaced by the condition that
	\begin{equation*}
		\begin{tikzcd}
			\Ar^{(\mathrm{cocart})}(\Uu)\rar\dar\drar[pullback] & \Ar(\Cc)\dar["s"]\\
			\Uu\rar["p"] & \Cc
		\end{tikzcd}
	\end{equation*}
	is a pullback in $\cat{Cat}_\infty$, where $\Ar^{(\mathrm{cocart})}(\Uu)\subseteq \Ar(\Uu)$ is the full sub-$\infty$-category spanned by the $p$-cocartesian morphisms. This model-independent definition of cocartesian fibrations recovers \cref{def:Cocartesian}, as we'll see in \cref{lem:CocartesianModelIndependent} below. Finally, in light of \cref{lem:CocartesianLeft}, we can redefine $p$ to be a \emph{left fibration} if it is a cocartesian fibration and $\Ar^{(\mathrm{cocart})}(\Uu)\subseteq \Ar(\Uu)$ is an equivalence of $\infty$-categories.
	
	
	In particular, \cref{thm:Straightening} can be reinterpreted as a model-independent statement about an equivalence of $\infty$-categories $\cat{Cocart}(\Cc)\simeq \Fun(\Cc,\cat{Cat}_\infty)$ for any $\infty$-category $\Cc$. This statement absolutely belongs to step~\cref{enum:KeyStatementsNotIntro} and it will play a much more prominent role in our treatment of $\infty$-categories than it does in ordinary category theory.
	
	The example of cocartesian and left fibrations is only the first of many instances where our constructions with quasi-categories can be retconned into model-independent constructions. We'll see more of that in model category fact~\cref{par:HomotopyPushout}.
\end{numpar}
\begin{lem}\label{lem:CocartesianModelIndependent}
	The functor $\Ar^{(\mathrm{cocart})}(\Uu)\rightarrow \Uu\times_{\Cc,s}\Ar(\Cc)$ is always fully faithful \embrace{where the pullback is taken in $\cat{Cat}_\infty$}. Furthermore, the following conditions are equivalent:
	\begin{alphanumerate}
		\item $\Ar^{(\mathrm{cocart})}(\Uu)\rightarrow \Uu\times_{\Cc,s}\Ar(\Cc)$ is also essentially surjective. In particular, it must be an equivalence \embrace{by \cref{thm:EquivalenceFullyFaithfulEssentiallySurjective}}, so that $p\colon \Uu\rightarrow \Cc$ is a cocartesian fibration in the model-independent sense.\label{enum:CocartesianModelIndependent}
		\item For every factorisation $p\colon \Uu\rightarrow \Uu'\rightarrow \Cc$ into an equivalence of quasi-categories followed by an isofibration, $\Uu'\rightarrow \Cc$ is a cocartesian fibration in the old sense.\label{enum:CocartesianAllFactorisations}
		\item For some factorisation $p\colon \Uu\rightarrow \Uu'\rightarrow \Cc$ into an equivalence of quasi-categories followed by an isofibration, $\Uu'\rightarrow \Cc$ is a cocartesian fibration in the old sense.\label{enum:CocartesianSomeFactorisations}
	\end{alphanumerate}
\end{lem}
\begin{proof}[Proof sketch]
	To show that $\Ar^{(\mathrm{cocart})}(\Uu)\rightarrow \Uu\times_{\Cc,s}\Ar(\Cc)$ is fully faithful, use \cref{lem:HomInArrowCategories} to compute $\Hom$ on either side and \cref{lem:CocartesianMorphisms} to show that they coincide. This is fun to figure out yourself so we'll leave it to you.
	
	The implication \cref{enum:CocartesianAllFactorisations} $\Rightarrow$ \cref{enum:CocartesianSomeFactorisations} is trivial. For \cref{enum:CocartesianSomeFactorisations} $\Rightarrow$ \cref{enum:CocartesianModelIndependent}, it's enough to show that the functor $\Ar^{(\mathrm{cocart})}(\Uu')\rightarrow \Uu'\times_{\Cc,s}\Ar(\Cc)$ is essentially surjective, because $\Uu\rightarrow \Uu'$ is an equivalence of quasi-categories. Since $\Uu'\rightarrow \Cc$ is a cocartesian fibration, it's an isofibration too, hence the pullback $\Uu'\times_{\Cc,s}\Ar(\Cc)$ in $\cat{Cat}_\infty$ can also be taken in simplicial sets (see~\cref{par:HomotopyPullback}\cref{enum:HomotopyPullbackOfQuasicategories}). Furthermore, the condition from \cref{def:Cocartesian}\cref{enum:CocartesianFibration} translates into $\Ar^{(\mathrm{cocart})}(\Uu')\rightarrow \Uu'\times_{\Cc,s}\Ar(\Cc)$ being surjective on $0$-simplices. Hence it must be an essentially surjective functor of quasi-categories.
	
	By the same reasoning, to show \cref{enum:CocartesianModelIndependent} $\Rightarrow$ \cref{enum:CocartesianAllFactorisations}, we must show that for all factorisations $p\colon \Uu\rightarrow \Uu'\rightarrow \Cc$ into an equivalence followed by an isofibration, the map of simplicial sets $\Ar^{(\mathrm{cocart})}(\Uu')\rightarrow \Uu'\times_{\Cc,s}\Ar(\Cc)$ is surjective on $0$-simplices. The condition from~\cref{enum:CocartesianModelIndependent} tells us this map is an equivalence of quasi-categories. In particular, it hits every equivalence class of $0$-simplices. To show that it really hits every $0$-simplex, it then suffices to show that $\Ar^{(\mathrm{cocart})}(\Uu')\rightarrow \Uu'\times_{\Cc,s}\Ar(\Cc)$ is an isofibration. Observe that $\Ar(\Uu')\rightarrow\Uu'\times_{\Cc,s}\Ar(\Cc)$ is an isofibration---it's an inner fibration by \cref{cor:FKanFibration} and lifting of equivalences can be shown by a straightforward argument, using that the isofibration $\Uu'\rightarrow \Cc$ admits lifting of equivalences too. Since $\Ar^{(\mathrm{cocart})}(\Uu')\subseteq \Ar(\Uu')$ is a full sub-quasi-category in the sense of \cref{par:SubQuasiCategories} and furthermore closed under equivalences, the map $\Ar^{(\mathrm{cocart})}(\Uu')\rightarrow \Ar(\Uu')$ must be an isofibration. It follows that $\Ar^{(\mathrm{cocart})}(\Uu')\rightarrow \Uu'\times_{\Cc,s}\Ar(\Cc)$ is an isofibration too, as desired.
\end{proof}

%A functor $p\colon \Uu\rightarrow \Cc$ of quasi-categories is a cocartesian fibration in the new sense if and only if $\Uu'\rightarrow \Cc$ is a cocartesian fibration in the old sense, where $p\colon \Uu\rightarrow \Uu'\rightarrow \Cc$ is any factorisation into an equivalence followed by an isofibration. A proof of this uses \cref{thm:EquivalenceFullyFaithfulEssentiallySurjective} and \cref{lem:HomInArrowCategories} and is not particularly difficult. You can find it in my notes on Fabian's $\K$-Theory lecture, see \cite[Lemma*~\href{https://florianadler.github.io/AlgebraBonn/KTheory.pdf\#smallerdummy.1.24.2}{I.24$b$}]{KTheory}.

%Furthermore, once we know how to define pullbacks in $\infty$-categories (\cref{def:Colimits}), we see that the pullback diagram from \cref{par:HomInQuasiCategories} defining $\Cc_{x/}$ and $\Hom_\Cc(x,y)$ is also a pullback in $\cat{Cat}_\infty$; this follows from model category fact~\cref{par:HomotopyPullback} since $(s,t)\colon\Ar(\Cc)\rightarrow \Cc\times\Cc$ is an isofibration (see the argument in the proof of \cref{lem:HomRealityCheck}). So our definitions of $\Cc_{x/}$ and $\Hom_\Cc(x,y)$ immediately generalise to any other model.
\begin{numpar}[Functoriality of the Yoneda lemma.]\label{par:YonedaFunctorial}
	Let $\Cc$ be a quasi-category. Let's write $\Fun(\Cc^\op,\cat{An})\eqqcolon \PSh(\Cc)$ for the \emph{$\infty$-category of presheaves on $\Cc$}.\footnote{In \cref{sec:CategoryTheory} we had defined $\PSh(\Cc)\coloneqq \Fun(\Cc,\cat{Set})$ for ordinary categories. This is, of course, \emph{not} compatible with the new definition. In the following, all presheaves will be presheaves of animae and we'll never talk about presheaves of sets again.} For an object $x\in\Cc$ and a presheaf $E\colon \Cc^\op\rightarrow \cat{An}$, the dual of \cref{thm:Yoneda} tells us that
	\begin{equation*}
		\ev_{\id_x}\colon \Hom_{\PSh(\Cc)}\bigl(\Yo_\Cc(x),E\bigr)\overset{\simeq}{\longrightarrow}E(x)
	\end{equation*}
	is an equivalence. It turns out that this equivalence is functorial both in $x$ and in $E$. Of course, thanks to \cref{thm:EquivalencePointwise}, the only difficulty lies in making the map $\ev_{\id_x}$ functorial. It's straightforward to make it functorial in $E$, but I couldn't find any easy argument for functoriality in $x$. So if you do, please tell me!
	
	One way of producing this natural transformation is to construct it on the level of simplicially enriched functors. Implicitly, this requires that the construction of $\Hom_\Cc\colon \Cc^\op\times\Cc\rightarrow \cat{An}$ via simplicially enriched functors agrees with our construction; at least in the case of \cref{con:HomTwAr}, this was done by Lurie \cite[Proposition~\HAthm{5.2.1.11}]{HA}. A somewhat nicer argument, which is at least model-independent (but makes heavy use of \cref{sec:InftyCategoryTheory}), goes as follows: Let's instead construct an inverse of $\ev_{\id_x}$ in a functorial way. That is, we are looking for a natural transformation
	\begin{equation*}
		\eta\colon\id_{\PSh(\Cc)}\Longrightarrow\Hom_{\PSh(\Cc)}\bigl(\Yo_\Cc(-),-\bigr)\,.
	\end{equation*}
	We'll see in \cref{lem:PresheafColimitOfRepresentables} that $\id_{\PSh(\Cc)}$ is the $\infty$-categorical left Kan extension of $\Yo_\Cc$ along itself. So it's enough to construct a natural transformation $\Yo_\Cc\Rightarrow \Yo_\Cc^*\Hom_{\PSh(\Cc)}(\Yo_\Cc(-),-)$; this can be taken to be the image of the natural transformation $\Hom_\Cc(-,-)\Rightarrow \Hom_{\PSh(\Cc)}(\Yo_\Cc(-),\Yo_\Cc(-))$ from \cref{lem:HomFunctorial} under the \enquote{currying} equivalence $\Fun(\Cc^\op\times\Cc,\cat{An})\simeq \Fun(\Cc,\PSh(\Cc))$.
	
	Even though this argument looks super fishy, I think it doesn't run into vicious circles. As far as I can see, we only used material up to \cref{lem:PresheafColimitOfRepresentables}, whereas the first time we need functoriality of the Yoneda lemma will be, conveniently, in \cref{lem:LanAlongYonedaHasRightAdjoint}.
\end{numpar}




	\section{\texorpdfstring{$\infty$}{Infinity}-Category theory}\label{sec:InftyCategoryTheory}

Armed with Lurie's straightening equivalence and the quasi-categorical Yoneda lemma, we will spend \crefrange{subsec:Adjunctions}{subsec:KanExtensions} redeveloping the theory from \cref{sec:CategoryTheory} (and more) in the setting of quasi-categories. In \cref{subsec:EilenbergMacLane} we will see a first major application to topology. After that, there will be a lengthy appendix (\crefrange{subsec:EssentiallySmall}{subsec:PrL}) in which we discuss presentable $\infty$-categories and prove Lurie's adjoint functor theorem.

Even though, implicitly, we work with quasi-categories, our arguments in \crefrange{subsec:Adjunctions}{subsec:KanExtensions} will be almost entirely model-independent; the same is true, at least in large parts, for \crefrange{subsec:EilenbergMacLane}{subsec:PrL}. So from now on, instead of \emph{quasi-categories}, we'll simply write \emph{$\infty$-categories}. We'll consider ordinary categories as $\infty$-categories via the nerve construction, but we'll always suppress $\N$ in our notation.\footnote{In particular, the partially ordered set $[n]$ is now identified with its nerve, the quasi-category $\Delta^n$. However, we'll continue to write $\Delta^n$. There are at least to reasons for this. First, I believe $\Delta^n$ is easier to parse for a human brain. Second, once we reach \cref{sec:TowardsSpectra}, we'll consider simplicial objects in $\infty$-categories, and it will become necessary to keep the $\infty$-category $\Delta^n$ and the object $[n]\in\IDelta$ notationally separate.} Furthermore, we'll write $\Fun(\Cc,\Dd)$ instead of $\F(\Cc,\Dd)$ for $\infty$-categories $\Cc$ and $\Dd$. We'll only switch back to the old terminology in the few instances where non-model-independent arguments are used. I believe these few exceptions could easily be treated in any other model of $\infty$-categories as well.%Also recall from \cref{par:ModelIndependence} that many constructions and results so far (like cocartesian/left fibrations) can be reformulated in a model-independent fashion, and this is how we're going to use them.
%
%, which 
%
%the notions of cocartesian and left fibrations as well as Lurie's straightening/unstraightening equivalence can be reformulated in a model-independent way, and this is the way in which we're going to use them.

\subsection{Adjunctions}\label{subsec:Adjunctions}
\begin{defi}\label{def:Adjunction}
	Let $L\colon \Cc\rightarrow \Dd$ be a functor of $\infty$-categories.
	\begin{alphanumerate}
		\item Let $y\in \Dd$. An object $x\in \Cc$ is a \emph{right adjoint object to $y$ under $L$} if there exists an equivalence
		\begin{equation*}
			\Hom_\Cc(-,x)\simeq \Hom_\Dd\bigl(L(-),y\bigr)
		\end{equation*}
		in the functor category $\Fun(\Cc^\op,\cat{An})$.
		\item A functor $R\colon \Dd\rightarrow \Cc$ is a \emph{right adjoint of $L$} if there exists an equivalence
		\begin{equation*}
			\Hom_\Cc\bigl(-,R(-)\bigr)\simeq \Hom_\Dd\bigl(L(-),-\bigr)
		\end{equation*}
		in the functor category $\Fun(\Cc^\op\times \Dd,\cat{Set})$. In this case we write $L\dashv R$.
	\end{alphanumerate}
	There are obvious dual notions of \emph{left adjoint objects/functors}.
\end{defi}
\begin{lem}[\enquote{Adjoints can be constructed pointwise}]\label{lem:Adjunction}
	A functor $L\colon \Cc\rightarrow \Dd$ has a right adjoint if and only if every $y\in \Dd$ has a right adjoint object $x\in \Cc$.
\end{lem}
\begin{proof}
	One implication is trivial: If $R\colon \Dd\rightarrow \Cc$ is a right adjoint of $L$, then $R(y)$ is a right adjoint object of $y$ for every $y\in \Dd$. For the other implication, consider $\Hom_\Dd(L(-),-)\colon \Cc^\op\times\Dd\rightarrow\cat{An}$ as a functor $\overline{R}\colon \Dd\rightarrow \Fun(\Cc^\op,\cat{An})$. Our assumption implies that $\overline{R}$ takes values in the image of the Yoneda embedding $\Yo_\Cc\colon \Cc\rightarrow\Fun(\Cc^\op,\cat{An})$; namely, $\overline{R}(y)\simeq \Hom_\Cc(-,x)$ if $x\in\Cc$ is a right adjoint object of $y\in\Dd$ under $L$. Since $\Yo_\Cc$ is an equivalence onto its image, we obtain a functor $R\colon \Dd\rightarrow \Cc$ with the required properties.
\end{proof}
\begin{exm}\label{exm:Adjunctions}
	It's clear that any adjunction of ordinary categories is also an adjunction of $\infty$-categories. Furthermore, we already know some non-trivial examples of adjunctions of $\infty$-categories:
	\begin{alphanumerate}
		\item The inclusion $\cat{An}\subseteq \cat{Cat}_\infty$ is fully faithful (by \cref{thm:CordierPorter} and \cref{cor:FIsKanComplex}) and has both adjoints: A right adjoint $\core\colon \cat{Cat}_\infty\rightarrow \cat{An}$ and a left adjoint $\abs*{\,\cdot\,}\colon \cat{Cat}_\infty\rightarrow \cat{An}$ sending $\Cc$ to $\abs*{\Cc}$, the localisation of $\Cc$ at all its morphisms.\label{enum:AnToCatInfty}
		\item For every $\infty$-category $\Cc$, the functor $-\times \Cc\colon \cat{Cat}_\infty\rightarrow \cat{Cat}_\infty$ has a right adjoint, which sends an $\infty$-category $\Dd$ to $\Fun(\Cc,\Dd)$.\label{enum:Currying}
	\end{alphanumerate}
	Both \cref{enum:AnToCatInfty} and \cref{enum:Currying} can easily be seen using \cref{lem:Adjunction}: For \cref{enum:AnToCatInfty}, it's enough to check that the functors $i\colon \core(\Dd)\rightarrow \Dd$ and $p\colon \Cc\rightarrow \abs*{\Cc}$ induce functorial equivalences
	\begin{equation*}
		i_*\colon \Hom_{\cat{An}}\bigl(-,\core(\Dd)\bigr)\overset{\simeq}{\Longrightarrow}\Hom_{\cat{Cat}_\infty}(-,\Dd)\,,\ \,p^*\colon \Hom_{\cat{An}}\bigl(\abs{\Cc},-\bigr)\overset{\simeq}{\Longrightarrow}\Hom_{\cat{Cat}_\infty}(\Cc,-)
	\end{equation*}
	via post- and precomposition, respectively. Indeed, equivalences can be checked pointwise by \cref{thm:EquivalencePointwise} and then we can apply \cref{lem:Localisation} (and a similar assertion for $\core$). For \cref{enum:Currying}, observe that we have an \emph{evaluation functor} $\ev\colon \Fun(\Cc,\Dd)\times\Cc\rightarrow \Dd$ for all $\infty$-categories $\Cc$ and $\Dd$. If we work with quasi-categories, this functor is simply given by the counit of the adjunction $-\times\Cc\colon\cat{sSet}\shortdoublelrmorphism \cat{sSet}\noloc \F(\Cc,-)$. Using \cref{lem:Adjunction}, it's enough to show that the composition
	\begin{equation*}
		\Hom_{\cat{Cat}_\infty}\bigl(-,\Fun(\Cc,\Dd)\bigr)\xRightarrow{-\times\Cc\vphantom{_y}}\Hom_{\cat{Cat}_\infty}\bigl(-\times\Cc,\Fun(\Cc,\Dd)\times\Cc\bigr)\overset{\ev_*\vphantom{_y}}{\Longrightarrow}\Hom_{\cat{Cat}_\infty}(-\times\Cc,\Dd)
	\end{equation*}
	is an equivalence. Again, \cref{thm:EquivalencePointwise} allows us to check this pointwise, and then \cref{thm:CordierPorter} reduces everything to the  adjunction $-\times\Cc\colon\cat{sSet}\shortdoublelrmorphism \cat{sSet}\noloc \F(\Cc,-)$ of ordinary categories.
	
	In particular, \cref{enum:Currying} allows us to define a functor $\Fun(\Cc,-)\colon \cat{Cat}_\infty\rightarrow \cat{Cat}_\infty$. With a little more work%
	\footnote{\label{footnote:Fun}Here's the argument: First, let $*\ \,*$ be the discrete category on two objects. In \cref{rem:ColimitFunctor} below, we'll construct a functor $\lim\colon \Fun(*\ \,*,\cat{Cat}_\infty)\rightarrow \cat{Cat}_\infty$. Under the identification $\Fun(*\ \,*,\cat{Cat}_\infty)\simeq \cat{Cat}_\infty\times\cat{Cat}_\infty$, this functor sends a pair $(\Cc,\Dd)$ of $\infty$-categories to the product $\Cc\times\Dd$. By \enquote{currying}, this functor corresponds to a functor $P\colon \cat{Cat}_\infty\rightarrow \Fun(\cat{Cat}_\infty,\cat{Cat}_\infty)$ sending $\Cc$ to $P(\Cc)\simeq -\times\Cc\colon \cat{Cat}_\infty\rightarrow \cat{Cat}_\infty$. By the way, this is also how you construct the functor $-\times\Cc$ in \cref{enum:Currying}. As we've seen above, $P(\Cc)$ is a left adjoint, and so $P$ factors through the full sub-$\infty$-category $\Fun^\L(\cat{Cat}_\infty,\cat{Cat}_\infty)\subseteq \Fun(\cat{Cat}_\infty,\cat{Cat}_\infty)$ spanned by the left adjoint functors. By \cref{cor:ExtractingAdjoints} below, extracting adjoints induces an equivalence of $\infty$-categories $\Fun^\L(\cat{Cat}_\infty,\cat{Cat}_\infty)\simeq \Fun^\R(\cat{Cat}_\infty,\cat{Cat}_\infty)^\op$. Thus, $P^\op$ can be regarded as a functor $P^\op\colon \cat{Cat}_\infty^\op\rightarrow \Fun^\R(\cat{Cat}_\infty,\cat{Cat}_\infty)$, sending $\Cc$ to $P^\op(\Cc)\simeq \Fun(\Cc,-)$. \enquote{Currying} back, we obtain the desired functor $\Fun(-,-)\colon \cat{Cat}_\infty^\op\times\cat{Cat}_\infty\rightarrow \cat{Cat}_\infty$.
	
	
	It's true that the functor $\core\Fun(-,-)$ agrees with $\Hom_{\cat{Cat}_\infty}(-,-)$, but this is not so easy to see (and we won't need it). One way would be to turn $\F(-,-)\colon \cat{QCat}\times\cat{QCat}\rightarrow \cat{QCat}$ into a Kan-enriched functor and show that $\N^\Delta(\F(-,-))$ agrees with $\Fun(-,-)$. This is easy since $\N^\Delta$ turns Kan-enriched adjunctions into adjunctions of $\infty$-categories. Then one has to check that $\Hom_{\cat{Cat}_\infty}(-,-)$ agrees with $\N^\Delta$ applied to $\core\F(-,-)\colon\cat{QCat}\times\cat{QCat}\rightarrow \cat{Kan}$. At least for \cref{con:HomTwAr}, this is done in \cite[Proposition~\HAthm{5.2.1.11}]{HA}.}, these functors can be assembled into a two-argument functor
	\begin{equation*}
		\Fun(-,-)\colon \cat{Cat}_\infty^\op\times\cat{Cat}_\infty\longrightarrow \cat{Cat}_\infty\,.
	\end{equation*}
\end{exm}
%Next, we'll characterise adjunctions in terms of unit and counit.
\begin{numpar}[Unit and counit.]\label{par:Unit}
	Let $L\colon \Cc \shortdoublelrmorphism \Dd\noloc R$ be an adjunction. We wish to construct a \emph{unit} transformation $u\colon \id_\Cc\Rightarrow RL$ and a \emph{counit} transformation $c\colon LR\Rightarrow \id_\Dd$ in such a way that the adjunction equivalence $\eta\colon\Hom_\Dd(L(-),-)\overset{\simeq}{\Longrightarrow} \Hom_\Cc(-,R(-))$ and its inverse are given by
	\begin{align*}
		\Hom_\Dd\bigl(L(-),-\bigr)\overset{R}{\Longrightarrow}&\Hom_\Cc\bigl(RL(-),R(-)\bigr)\overset{u^*}{\Longrightarrow}\Hom_\Cc\bigl(-,R(-)\bigr)\,,\\
		\Hom_\Cc\bigl(-,R(-)\bigr)\overset{L}{\Longrightarrow}&\Hom_\Dd\bigl(L(-),LR(-)\bigr)\overset{c_*}{\Longrightarrow}\Hom_\Dd\bigl(L(-),-\bigr)\,.
	\end{align*}
	To construct $u$, we start with $\Hom_\Cc(-,-)\Rightarrow \Hom_\Dd(L(-),L(-))\simeq \Hom_\Cc(-,RL(-))$, where the first one is induced by functoriality of $L$ and the second by the given adjunction. We can consider this composition as a natural transformation $\Yo_\Cc\Rightarrow \Yo_\Cc\circ RL$ in $\Fun(\Cc,\Fun(\Cc^\op,\cat{An}))$. Since $\Yo_\Cc$ is fully faithful (\cref{cor:YonedaEmbeddingFullyFaithful}), we obtain a natural transformation $u\colon \id_\Cc\Rightarrow RL$.
	
	With this definition, the natural transformations $u^*\circ R$ and $\eta$ agree pointwise; more precisely, for every fixed $x\in \Cc$ they induce the same transformation $\Hom_\Dd(L(x),-)\Rightarrow \Hom_\Cc(x,R(-))$. Indeed, Yoneda's lemma (\cref{thm:Yoneda}) shows that any such transformation is uniquely determined by the image of $\id_{L(x)}$ under $\Hom_\Dd(L(x),L(x))\rightarrow \Hom_\Cc(x,RL(x))$. Now by construction, both $u^*\circ R$ and $\eta$ map $\id_{L(x)}$ to $u_x\colon x\rightarrow RL(x)$.
	
	In ordinary categories two natural transformations that agree pointwise must be equal, because natural transformations are just pointwise data subject to certain conditions. But in $\infty$-land the situation is more subtle and we need a slightly convoluted argument to conclude $u^*\circ R\simeq\eta$.
	
	First observe that since $u^*\circ R$ is still an equivalence by \cref{thm:EquivalencePointwise}, since it agrees pointwise with $\eta$. Consider 
	\begin{equation*}
		\eta\circ (u^*\circ R)^{-1}\colon \Hom_\Cc\bigl(-,R(-)\bigr)\overset{\simeq}{\Longrightarrow} \Hom_\Cc\bigl(-,R(-)\bigr)\,.
	\end{equation*}
	Write this as a natural equivalence $\Yo_\Cc\circ R\overset{\simeq}{\Longrightarrow} \Yo_\Cc\circ R$. Since $\Yo_\Cc$ is fully faithful, $\eta\circ (u^*\circ R)^{-1}$ must be induced by a natural equivalence $\epsilon\colon R\overset{\simeq}{\Longrightarrow}R$. Put $\ov u\coloneqq \epsilon L\circ \ov u\colon \id_\Cc\Rightarrow RL$. Then $\eta\simeq \ov u^*\circ R$. Indeed, this follows from the diagram
	\begin{equation*}
		\begin{tikzcd}
			\Hom_\Dd\bigl(L(-),-\bigr)\doublear["R"{black,above=0.1em}]{r}\doublear["R"{black,left=0.1em}]{d}\drar[commutes] & \Hom_\Cc\bigl(RL(-),R(-)\bigr)\doublear["\epsilon_*"{black,right=0.1em}]{d}\doublear["\ov u^*"{black,above=0.1em}]{r}\drar[commutes] & \Hom_\Cc\bigl(-,R(-)\bigr)\doublear["\epsilon_*"{black,right=0.1em}]{d}\\
			\Hom_\Cc\bigl(RL(-),R(-)\bigr) \doublear["\epsilon^*"{black,above=0.1em}]{r} & \Hom_\Cc\bigl(RL(-),R(-)\bigr) \doublear["\ov u^*"{black,above=0.1em}]{r} & \Hom_\Cc\bigl(-,R(-)\bigr)
		\end{tikzcd}
	\end{equation*}
	in which the  left square commutes by \cref{lem:HomNaturalTransformation} and the right square by \cref{lem:PrecompositionCommutesWithPostcomposition}. To conclude $u\simeq \ov u$, we plug $\eta\simeq \ov u\circ R$ into the construction of $u$ to obtain the following commutative diagram:
	\begin{equation*}
		\begin{tikzcd}
			\Hom_\Cc(-,-)\eqar[d]\doublear["L"{black,above=0.1em}]{r}\ar[commutes]{drr} & \Hom_\Dd\bigl(L(-),L(-)\bigr)\doublear["R"{black,above=0.1em}]{r} & \Hom_\Dd\bigl(RL(-),RL(-)\bigr)\doublear["\ov u^*"{black,right=0.1em}]{d}\\
			\Hom_\Cc(-,-)\doublear["u_*"{black,above=0.1em}]{rr} &  & \Hom_\Dd\bigl(-,RL(-)\bigr)
		\end{tikzcd}
	\end{equation*}
	Using \cref{lem:HomNaturalTransformation} we conclude $u_*\simeq \ov u_*$ as natural transformations $\Yo_\Cc\Rightarrow \Yo_\Cc\circ RL$, and thus indeed $u\simeq \ov u$, as desired.
\end{numpar}
\begin{lem}[Triangle identities]\label{lem:TriangleIdentities}
	Let $L\colon \Cc\shortdoublelrmorphism \Dd\noloc R$ be an adjunction of $\infty$-categories. Then there are commutative diagrams 
	\begin{equation*}
		\begin{tikzcd}
			L \doublear["Lu"{black,above=0.1em}]{r}\doublear["\id_L"'{black}]{dr} & LRL\doublear["cL"{black,right=0.1em}]{d}\dar[phantom,""{name=A}]\arrow[from=1-1,to=A,commutes,pos=0.7]\\
			& L
		\end{tikzcd}\quad\text{and}\quad
		\begin{tikzcd}
			R \doublear["uR"{black,above=0.1em}]{r}\doublear["\id_R"'{black}]{dr} & RLR\doublear["Rc"{black,right=0.1em}]{d}\dar[phantom,""{name=A}]\arrow[from=1-1,to=A,commutes,pos=0.7]\\
			& R
		\end{tikzcd}
	\end{equation*}
	Conversely, let $L\colon \Cc\rightarrow \Dd$ and $R\colon \Dd\rightarrow \Cc$ be functors. Suppose we're given a natural transformation $u\colon \id_\Cc\Rightarrow RL$ as well as pointwise morphisms $c_y\colon LR(y)\rightarrow y$ for all $y\in\Dd$ such that $c_{L(x)}\circ L(u_x)$ and $R(c_y)\circ u_{R(y)}$ are equivalences \embrace{but not necessarily the identity on $L(x)$ and $R(y)$, respectively} for all $x\in\Cc$, $y\in \Dd$. Then $L$ and $R$ determine an adjunction.
\end{lem}
\begin{proof}
	First suppose $L\colon \Cc\shortdoublelrmorphism\Dd\noloc R$ is an adjunction. Consider the diagram
	\begin{equation*}
		\begin{tikzcd}
			\Hom_\Dd\bigl(L(-),-\bigr)\doublear["R"{black,above=0.1em}]{r}\ar["(cL)^*"{black,swap},white,double, double equal sign distance,-{implies[black]},ddr,bend right]\ar[ddr,dash,shift right=0.1em,bend right,shorten >=0.35ex]\ar[ddr,dash,shift left=0.1em,bend right,shorten >=0.59ex,shorten <=0.35ex]\ar[ddr,commutes,pos=0.4]& \Hom_\Cc\bigl(RL(-),R(-)\bigr)\doublear["u^*"{black,above=0.1em}]{r}\doublear["L"{black,left=0.1em}]{d}\drar[commutes] & \Hom_\Cc\bigl(-,R(-)\bigr)\doublear["L"{black,right=0.1em}]{d}\\
			& \Hom_\Dd\bigl(LRL(-),LR(-)\bigr)\doublear["(Lu)^*"{black,above=0.1em}]{r}\doublear["c_*"{black,left=0.1em}]{d}\drar[commutes] & \Hom_\Dd\bigl(L(-),LR(-)\bigr)\doublear["c_*"{black,right=0.1em}]{d}\\
			& \Hom_\Dd\bigl(LRL(-),-\bigr)\doublear["(Lu)^*"{black,above=0.1em}]{r} & \Hom_\Dd\bigl(L(-),-\bigr)
		\end{tikzcd}
	\end{equation*}
	The top right square commutes by functoriality of $L$, the bottom right square commutes since pre- and postcomposition commute by \cref{lem:PrecompositionCommutesWithPostcomposition}, and the left cell commutes by \cref{lem:HomNaturalTransformation}. Now walking around the perimeter shows that $(cL)^*\circ (Lu)^*\colon \Hom_\Dd(L(-),-)\Rightarrow \Hom_\Dd(L(-),-)$ agrees with $c_*\circ L\circ u^*\circ R$. This is the identity, since we've arranged in \cref{par:Unit} that $c_*\circ L$ and $u^*\circ R$ are inverses. This establishes the first triangle identity; the second one is analogous.
	
	Conversely, suppose $L\colon\Cc\rightarrow \Dd$, $R\colon \Dd\rightarrow \Cc$ are functors and $u\colon \id_\Cc\Rightarrow RL$, $c_y\colon LR(y)\rightarrow y$ satisfy the conditions above. For $L$ and $R$ to be adjoints, it suffices that the composition
	\begin{equation*}
		\Hom_\Dd\bigl(L(-),-\bigr)\overset{R}{\Longrightarrow}\Hom_\Cc\bigl(RL(-),R(-)\bigr)\overset{u^*}{\Longrightarrow}\Hom_\Cc\bigl(-,R(-)\bigr)
	\end{equation*}
	is a natural equivalence. By \cref{thm:EquivalencePointwise}, it's enough to check this pointwise. In the case where $c_{L(x)}\circ L(u_x)\simeq \id_{L(x)}$ and $R(c_y)\circ u_{R(y)}\simeq \id_{R(y)}$ are the identity on $L(x)$ and $R(y)$, respectively, plugging $x$ and $y$ into the diagram above (as well as its analogue for the second triangle identity) shows that
	\begin{equation*}
		\Hom_\Cc\bigl(x,R(y)\bigr)\overset{L}{\longrightarrow} \Hom_\Dd\bigl(L(x),LR(y)\bigr)\xrightarrow{c_{y,*}}\Hom_\Dd\bigl(L(x),y\bigr)
	\end{equation*}
	is a pointwise inverse. If $c_{L(x)}\circ L(u_x)$ and $R(c_y)\circ u_{R(y)}$ are merely equivalences, but not necessarily the identity on $L(x)$ and $R(y)$, this argument is still enough to show that the composition $u^*\circ R\colon \Hom_\Dd(L(-),-)\Rightarrow \Hom_\Cc(-,R(-))$ is a natural equivalence.
\end{proof}
\begin{cor}\label{cor:FunctorCategoryAdjunctions}
	Let $L\colon \Cc \shortdoublelrmorphism \Dd\noloc R$ be an adjunction and let $\Ii$ be another category. Then the pre- and postcomposition functors determine adjunctions
	\begin{align*}
		L\circ-\colon \Fun(\Ii,\Cc)&\doublelrmorphism \Fun(\Ii,\Dd)\noloc R\circ -\,,\\
		{-}\circ {R}\colon \Fun(\Cc,\Ii)&\doublelrmorphism \Fun(\Dd,\Ii)\noloc {-}\circ {L}\,.
	\end{align*}
\end{cor}
\begin{proof}
	The proof of \cref{cor:1FunctorCategoryAdjunctions} can be copied verbatim.
\end{proof}
To finish this subsection about adjunctions, we connect adjunctions to the theory of straightening/unstraightening. This won't be needed in the main text, but it shows up in the appendices and is a standard fact in other treatments of $\infty$-categories.
\begin{lem}\label{lem:AdjunctionBicartesian}
	Let $F\colon \Cc\rightarrow \Dd$ be a functor of $\infty$-categories, corresponding to a functor $\Delta^1\rightarrow \cat{Cat}_\infty$ \embrace{see \cref{exm:SimplicialNerve}}, which in turn corresponds to a cocartesian fibration $p\colon \Uu\rightarrow \Delta^1$ by \cref{thm:Straightening}\cref{enum:CocartesianStraightening}. Then the following are equivalent:
	\begin{alphanumerate}
		\item $F$ admits a right adjoint $G\colon \Dd\rightarrow \Cc$.\label{enum:BicartesianAdjoint}
		\item \!The cocartesian fibration $p\colon \Uu\rightarrow \Delta^1$ is also a cartesian fibration.\label{enum:Bicartesian}
	\end{alphanumerate}
	Furthermore, in this case $G$ agrees with the functor classified by the cartesian straightening $\operatorname{St}^{\mathrm{cart}}(p)\colon (\Delta^1)^\op\rightarrow \cat{Cat}_\infty$.
\end{lem}
\begin{proof}
	The crucial observation is the following claim:
	\begin{alphanumerate}\itshape
		\item[\boxtimes] The functor $\Hom_\Dd(F(-),-)\colon \Cc^\op\times\Dd\rightarrow \cat{An}$ is equivalent to the composition\label{claim:HomInUnstraightening}
		\begin{equation*}
			\Cc^\op\times\Dd\xrightarrow{i_0^\op\times i_1} \Uu^\op\times\Uu\xrightarrow{\Hom_\Uu}\cat{An}\,.
		\end{equation*}
		Here the first arrow is given by $i_0\colon \Cc\simeq \{0\}\times_{\Delta^1}\Uu\rightarrow \Uu$ and $i_1\colon \Dd\simeq \{1\}\times_{\Delta^1}\Uu\rightarrow \Uu$.
	\end{alphanumerate}
	To prove \cref{claim:HomInUnstraightening}, first observe that $i_0\colon \Cc\rightarrow \Uu$ and $i_1\colon \Dd\rightarrow\Uu$ are fully faithful. Indeed, $\Hom$ animae in pullbacks are given as pullbacks of $\Hom$ animae in the respective factors (which is straightforward to see from \cref{par:HomInQuasiCategories} and we'll see a more general assertion in \cref{lem:HomInLimits}\cref{enum:HomInLimits}). So pullbacks of fully faithful functors are still fully faithful and it remains to observe that $\{0\}\rightarrow \Delta^1$ and $\{1\}\rightarrow \Delta^1$ are both fully faithful, which is obvious. Now consider the following commutative square in $\cat{Cat}_\infty$:
	\begin{equation*}
		\begin{tikzcd}
			\Cc\eqar[r]\eqar[d]\drar[commutes] & \Cc\dar["F"]\\
			\Cc\rar["F"] & \Dd
		\end{tikzcd}
	\end{equation*}
	It can be viewed as a natural transformation $\const \Cc\Rightarrow F$ in $\Fun(\Delta^1,\cat{Cat}_\infty)$. After cocartesian unstraightening, it thus induces a morphism $\varphi\colon \Delta^1\times\Cc\rightarrow \Uu$ in $\cat{Cocart}(\Delta^1)$. Consider the composite
	\begin{equation*}
		(\Delta^1\times\Cc)^\op\times\Dd\xrightarrow{\varphi^\op\times i_1}\Uu^\op\times\Uu\xrightarrow{\Hom_\Uu}\cat{An}\,.
	\end{equation*}
	By unravelling the definitions, this composite can be regarded as a natural transformation $\eta\colon \Hom_\Uu(i_1F(-),i_1(-))\Rightarrow \Hom_\Uu(i_0(-),i_1(-))$ in $\Fun(\Cc^\op\times\Dd,\cat{An})$. We wish to show that $\eta$ is an equivalence of functors. By \cref{thm:EquivalencePointwise} this can be checked pointwise. So fix $x\in \Cc$, $y\in\Dd$. By unravelling the constructions $\eta_{(x,y)}\colon \Hom_\Uu(i_1F(x),i_1(y))\rightarrow \Hom_\Uu(i_0(x),i_1(y))$ is given by precomposition with the morphism $\varphi_x\colon i_0(x)\rightarrow i_1F(x)$ given as the image of $\Delta^1\times\{x\}\rightarrow \Delta^1\times\Cc$ under $\varphi$. As $\varphi$ preserves cocartesian lifts, $\varphi_x$ must be a cocartesian morphism. Since $\Hom_{\Delta^1}(1,1)\simeq \Hom_{\Delta^1}(0,1)$, \cref{lem:CocartesianMorphisms} implies that precomposition with $\varphi_x$ must be an equivalence. Thus $\eta$ is an equivalence of functors, as desired. To finish the proof of \cref{claim:HomInUnstraightening}, it remains to observe $\Hom_\Uu(i_1F(-),i_1(-))\simeq \Hom_\Dd(F(-),-)$ as we've checked above that $i_1\colon \Dd\rightarrow \Uu$ is fully faithful.
	
	Now assume that $p\colon \Uu\rightarrow \Delta^1$ is a cartesian fibration too and let $G\colon \Dd\rightarrow \Cc$ correspond to $\operatorname{St}^{\mathrm{cart}}(p)\colon (\Delta^1)^\op\rightarrow \cat{Cat}_\infty$. Then \cref{claim:HomInUnstraightening} and its dual provide an equivalences of functors $\Hom_\Dd(F(-),-)\simeq \Hom_\Uu(i_0(-),i_1(-))\simeq \Hom_\Cc(-,G(-))$, so $F$ and $G$ are adjoints. This proves \cref{enum:Bicartesian} $\Rightarrow$ \cref{enum:BicartesianAdjoint}.
	
	Conversely, suppose $G\colon \Dd\rightarrow \Cc$ is a right adjoint of $F$. Fix $y\in \Dd$. Then \cref{claim:HomInUnstraightening} and the fact that $i_0$ is fully faithful shows
	\begin{equation*}
		\Hom_\Uu\bigl(i_0(-),i_0G(y)\bigr)\simeq \Hom_\Cc\bigl(-,G(y)\bigr)\simeq \Hom_\Dd\bigl(F(-),y\bigr)\simeq \Hom_\Uu\bigl(i_0(-),i_1(y)\bigr)\,.
	\end{equation*}
	The image of $\id_{i_0G(y)}$ defines a morphism $\psi_y\colon i_0G(y)\rightarrow i_1(y)$ in $\Uu$. Furthermore, Yoneda's lemma (or more precisely, the dual of \cref{thm:Yoneda}) shows that any natural transformation $\Hom_\Cc(-,G(y))\simeq \Hom_\Uu(i_0(-),i_0G(y))\Rightarrow \Hom_\Uu(i_0(-),i_1(y))$ is uniquely determined by the image of $\id_{G(y)}$. That uniqueness ensures that the chain of equivalences above is must be given by postcomposition with $\psi_y$. Hence the dual of \cref{lem:CocartesianMorphisms} shows that $\psi_y$ is a $p$-cartesian morphism and we have constructed a sufficient supply of $p$-cartesian lifts. This finishes the proof of \cref{enum:BicartesianAdjoint} $\Rightarrow$ \cref{enum:Bicartesian}.
\end{proof}
\begin{cor}[\enquote{Extracting adjoints is functorial}]\label{cor:ExtractingAdjoints}
	Let $\Fun^\L,\Fun^\R\subseteq \Fun$ denote the full sub-$\infty$-categories spanned by the left/right adjoint functors and let $\cat{Cat}_\infty^\L,\cat{Cat}_\infty^\R\subseteq \cat{Cat}_\infty$ be the non-full sub-$\infty$-categories \embrace{in the sense of \cref{par:SubQuasiCategories}} spanned by all objects but only the left/right adjoint functors.
	\begin{alphanumerate}
		\item For all $\infty$-categories $\Cc$ and $\Dd$, sending a left adjoint functor $L\colon \Cc\rightarrow \Dd$ to its right adjoint $R\colon \Dd\rightarrow \Cc$ can be turned into an equivalence of $\infty$-categories $\Fun^\L(\Cc,\Dd)^\op\simeq \Fun^\R(\Dd,\Cc)$.\label{enum:FunLFunR}
		\item \!There exists an equivalence of $\infty$-categories $\cat{Cat}_\infty^\L\simeq (\cat{Cat}_\infty^\R)^\op$ which is the identity on objects and sends morphisms in $\cat{Cat}_\infty^\L$, that is, left adjoint functors $L\colon \Cc\rightarrow \Dd$, to their right adjoints $R\colon \Dd\rightarrow \Cc$.\label{enum:CatLCatR}
	\end{alphanumerate}
\end{cor}
\begin{proof}[Proof sketch]
	For the equivalence in \cref{enum:FunLFunR}, it suffices show that the essential images of $\Fun^\L(\Cc,\Dd)$ and $\Fun^\R(\Dd,\Cc)^\op$ under the fully faithful Yoneda embeddings
	\begin{gather*}
		\Fun^\R(\Dd,\Cc)\xrightarrow{(\Yo_\Cc)_*}\Fun\bigl(\Dd,\Fun(\Cc^\op,\cat{An})\bigr)\simeq \Fun\left(\Cc^\op\times\Dd,\cat{An}\right)\,,\\
		\Fun^\L(\Cc,\Dd)^\op\simeq \Fun^\R\left(\Cc^\op,\Dd^\op\right)\xrightarrow{(\Yo_{\Dd^\op})_*}\Fun\bigl(\Cc^\op,\Fun(\Dd,\cat{An})\bigr)\simeq \Fun(\Cc^\op\times\Dd,\cat{An})
	\end{gather*}
	coincide. Using \cref{lem:Adjunction} and the definition of the Yoneda embedding, it's straightforward to check that both essential images consist of those functors $H\colon \Cc^\op\times\Dd\rightarrow \cat{An}$ such that for every $x\in \Cc$ there exists a $y\in \Dd$ such that $H(x,-)\simeq \Hom_\Dd(y,-)$ and for every $y'\in\Dd$ there exists an $x'\in \Cc$ such that $H(-,y')\simeq \Hom_\Cc(-,x')$. This proves that there exists an equivalence $\Fun^\L(\Cc,\Dd)^\op\simeq \Fun^\R(\Dd,\Cc)$ as desired. Furthermore if $L\colon \Cc\shortdoublelrmorphism \Dd\noloc R$ is an adjunction, then the Yoneda embeddings above send both $R$ and $L$ to the functor $\Hom_\Cc(-,R(-))\simeq \Hom_\Dd(L(-),-)\colon \Cc^\op\times\Dd\rightarrow \cat{An}$. So the equivalence we've constructed is really given by extracting adjoints.
	
	To prove \cref{enum:CatLCatR}, we grossly neglect set theory and regard both $\cat{Cat}_\infty^\L$ and $\cat{Cat}_\infty^\R$ as objects in $\cat{Cat}_\infty$. This can be repaired by considering universes or, with some care, by imposing cardinality bounds (similar to the argument in \cref{lem:StraighteningFunctorial} below, where we do this in detail). We'll show that there exists a functorial bijection $\pi_0\Hom_{\cat{Cat}_\infty}(\Cc,\cat{Cat}_\infty^\L)\cong \pi_0\Hom_{\cat{Cat}_\infty}(\Cc,(\cat{Cat}_\infty^\R)^\op)$ for all $\infty$-categories $\Cc$; if we can do this, then the Yoneda lemma in the ordinary category $\operatorname{ho}(\cat{Cat}_\infty)$ will show that $\cat{Cat}_\infty^\L$ and $\cat{Cat}_\infty^\R$ are isomorphic in the homotopy category, hence equivalent as $\infty$-categories. We know $\Hom_{\cat{Cat}_\infty}(\Cc,\cat{Cat}_\infty^\L)\simeq \core\Fun(\Cc,\cat{Cat}_\infty^\L)$ by \cref{thm:CordierPorter} and $\Fun(\Cc,\cat{Cat}_\infty^\L)\simeq \cat{Cocart}(\Cc)$ by \cref{thm:Straightening}\cref{enum:CocartesianStraightening}. Let $F\colon \Cc\rightarrow \cat{Cat}_\infty$ be a functor and $p\colon \Uu\rightarrow \Cc$ be its cocartesian unstraightening. By \cref{lem:AdjunctionBicartesian}, $F$ factors through $\cat{Cat}_\infty^\L\rightarrow \cat{Cat}_\infty$ if and only if for all $\alpha\colon\Delta^1\rightarrow \Cc$, the pullback $p_{\alpha}\colon \Delta^1\times_{\alpha,\Cc}\Uu\rightarrow \Delta^1$ is not only a cocartesian, but also a cartesian fibration. In other words, $p$ is a \emph{locally cartesian fibration} in the sense of \cref{def:LocallyCocartesian}. Since right adjoints compose, it's clear that locally $p$-cartesian morphisms are closed under composition, and so $p$ is automatically a cartesian fibration by \cref{cor:LocallyCocartesianComposition}. In summary, we obtain a bijection 
	\begin{equation*}
		\pi_0\Hom_{\cat{Cat}_\infty}(\Cc,\cat{Cat}_\infty^\L)\cong \pi_0\core\cat{Bicart}(\Cc)\,,
	\end{equation*}
	where we define $\cat{Bicart}(\Cc)\subseteq \cat{Cat}_{\infty/\Cc}$ as the non-full sub-$\infty$-category spanned by the \emph{bicartesian fibrations}. That is, objects of $\cat{Bicart}(\Cc)$ are those $p\colon \Uu\rightarrow \Cc$ that are both cocartesian and cartesian fibrations, and morphisms are those functors in $\cat{Cat}_{\infty/\Cc}$ that preserve both $p$-cocartesian and $p$-cartesian morphisms. In the same way, we find bijections 
	\begin{equation*}
		\pi_0\Hom_{\Cat_\infty}\bigl(\Cc,(\cat{Cat}_\infty^\R)^\op\bigr)\cong \pi_0\Hom_{\Cat_\infty}\bigl(\Cc^\op,\cat{Cat}_\infty^\R\bigr)\cong \pi_0\core\cat{Bicart}(\Cc)\,.
	\end{equation*}
	Hence $\pi_0\Hom_{\cat{Cat}_\infty}(\Cc,\cat{Cat}_\infty^\L)\cong \pi_0\Hom_{\cat{Cat}_\infty}(\Cc,(\cat{Cat}_\infty^\R)^\op)$ and so $\cat{Cat}_\infty^\L\simeq (\cat{Cat}_\infty^\R)^\op$, as argued above. By unravelling the cases $\Cc\simeq *$ and $\Cc\simeq \Delta^1$ (the latter using \cref{lem:AdjunctionBicartesian}), we find that this adjunction is really the identity on objects and given by extracting adjoints on morphisms.
\end{proof}

\subsection{Limits and colimits}
\begin{defi}\label{def:Colimits}
	Let $\Ii$ and $\Cc$ be $\infty$-categories.
	\begin{alphanumerate}
		\item Let $F\colon \Ii\rightarrow \Cc$ be a functor of $\infty$-categories. A \emph{colimit of $F$}, denoted $\colimit F$ (or sometimes $\colimit_{i\in\Ii}F(i)$), is a left adjoint object of $F$ under  $\operatorname{const}\colon\Cc\rightarrow \Fun(\Ii,\Cc)$ that sends $x\in\Cc$ to the constant functor with value $x$. Dually, a \emph{limit of $F$}, denoted $\limit F$ (or sometimes $\limit_{i\in\Ii}F(i)$), is a right adjoint object of $F$ under $\operatorname{const}$.\label{enum:Colimit}
		\item We say that \emph{$\Cc$ has all $\Ii$-shaped colimits} or \emph{all $\Ii$-shaped limits} if all functors $\Ii\rightarrow \Cc$ admit colimits or limits, respectively.\label{enum:ColimitFunctor}
	\end{alphanumerate}
\end{defi}
\begin{rem}\label{rem:ColimitFunctor}
	 If $\Cc$ has all $\Ii$-shaped colimits, then \cref{lem:Adjunction} implies that forming colimits assembles into a functor $\colimit\colon \Fun(\Ii,\Cc)\rightarrow \Cc$. The same is true for limits.
\end{rem}
\begin{lem}\label{lem:AdjointsPreserveColimits}
	Left adjoint functors between $\infty$-categories preserve colimits and right adjoint functors preserve limits.
\end{lem}
\begin{proof}
	The proof of \cref{lem:1AdjointsPreserveColimits} can be copied verbatim.
\end{proof}
\begin{lem}[\enquote{Colimits in functor $\infty$-categories are computed pointwise.}]\label{lem:ColimitsInFunctorCategories}
	Let $\Cc$, $\Dd$, and $\Ii$ be $\infty$-categories such that $\Dd$ has all $\Ii$-shaped colimits. Then $\Fun(\Cc,\Dd)$ has again all $\Ii$-shaped colimits and the evaluation functor 
	\begin{equation*}
		\ev_x\colon \Fun(\Cc,\Dd)\longrightarrow \Fun\bigl(\{x\},\Dd\bigr)\simeq \Dd
	\end{equation*}
	preserves $\Ii$-shaped colimits for all $x\in \Cc$. A dual assertion holds for limits.
\end{lem}
\begin{proof}
	The proof of \cref{lem:1ColimitsInFunctorCategories} can be copied verbatim.
\end{proof}
Our next goal is to analyse limits and colimits in the $\infty$-categories $\cat{An}$ and $\cat{Cat}_\infty$. We start with a procedure for computing pullbacks and pushouts which is very useful in practice.
\begin{numpar}[Pushouts and pullbacks in $\cat{An}$ and $\cat{Cat}_\infty$.]\label{par:HomotopyPushout}
	Pushouts and pullbacks in $\cat{An}$ or $\cat{Cat}_\infty$ can be computed using the following recipe:
	\begin{alphanumerate}
		\item Write down the diagram on the level of Kan complexes or quasi-categories.\label{enum:PushoutStepA}
		\item For pushouts, use \cref{lem:SmallObjectArgument} to replace one leg by a cofibration. For pullbacks, use \cref{lem:SmallObjectArgument} to replace one leg by a Kan fibration/isofibration (depending on whether you take the pullback in $\cat{An}$ or $\cat{Cat}_\infty$, respectively).\label{enum:PushoutStepB}
		\item Take the pushout or pullback in $\cat{sSet}$.\label{enum:PushoutStepC}
		\item For pushouts, the result of \cref{enum:PushoutStepC} will usually not be a Kan complex/quasi-category, so we need to use \cref{lem:SmallObjectArgument} once again to replace it by a Kan complex/quasi-category. For pullbacks, this step is unnecessary.\label{enum:PushoutStepD}
	\end{alphanumerate}
	We've already seen the case of pullbacks in \enquote{Definition}~\cref{def:HomotopyPullback}. The procedure above is a consequence of the general model category fact that a pushout of cofibrant objects in a model category is automatically a homotopy pushout too if at least one leg is a cofibration, and a pullback of fibrant objects is a homotopy pullback if at least one leg is a fibration. See \cite[Corollary~{\href{https://cisinski.app.uni-regensburg.de/CatLR.pdf\#thm.2.3.28}{2.3.28}}]{Cisinski} for a proof of the general fact and \cite[Theorem~\HTTthm{4.2.4.1}, Remark~\HTTthm{A.3.3.14}]{HTT} or \cite[Theorem~X.21]{HigherCatsII} for a proof that homotopy colimits/limits in a simplicial model category agree with colimits/limits in the underlying $\infty$-category.
	
	The procedure above implies that many pullback constructions we've seen so far with simplicial sets are also pullbacks in $\cat{An}$ or $\cat{Cat}_\infty$ and can thus be reinterpreted as model-independent constructions. For example, the diagram from \cref{par:HomInQuasiCategories} defining $\Cc_{x/}$ and $\Hom_\Cc(x,y)$ is also a pullback in $\cat{Cat}_\infty$, because $(s,t)\colon \Ar(\Cc)\rightarrow \Cc\times\Cc$ is an isofibration (see the proof of \cref{lem:HomRealityCheck}). As another example, if $p\colon\Uu\rightarrow \Cc$ is a cocartesian fibration, then the fibre $p^{-1}\{x\}$, which computes the value of the associated functor $\operatorname{St}^{\mathrm{cocart}}\colon \Cc\rightarrow\cat{Cat}_\infty$ at $x$, can also be identified with the $\infty$-categorical pullback $\{x\}\times_\Cc\Uu$, because any cocartesian fibration $p$ is automatically an isofibration. We'll often use these facts without mention. Let us also mention, and later use without mention, that $\cat{An}\subseteq\cat{Cat}_\infty$ preserves both pushouts and pullbacks; in fact, it preserves all limits and colimits by \cref{exm:Adjunctions}\cref{enum:AnToCatInfty} and \cref{lem:AdjointsPreserveColimits}.\hfill$\blacksquare$
\end{numpar}
%\begin{numpar}[Pushouts in $\cat{An}$ and $\cat{Cat}_\infty$.]
%	It turns out that pushouts can be computed analogously: Let 
%	\begin{equation*}
%		\begin{tikzcd}
%			X\rar\dar\drar[pushout] & X'\dar\\
%			Y\rar & \ov{Y}'
%		\end{tikzcd}\quad\text{and}\quad
%		\begin{tikzcd}
%			\Cc\rar\dar\drar[pushout] & \Cc'\dar\\
%			\Dd\rar & \ov{\Dd}'
%		\end{tikzcd}
%	\end{equation*}
%	be pushouts in $\cat{sSet}$ such that $X$, $X'$, $Y$ are Kan complexes and $\Cc$, $\Cc'$, $\Dd$ are quasi-categories. Assume furthermore that at least one leg is a cofibration in either case. Choose an anodyne map $\ov{Y}'\rightarrow Y'$ and an inner anodyne map $\ov{\Dd}'\rightarrow \Dd'$. Then $Y'$ and $\Dd'$ are the pushouts in the $\infty$-categories $\cat{An}$ and $\cat{Cat}_\infty$, respectively. For a proof of the general model category fact behind this see \cite[Definition~{\href{https://cisinski.app.uni-regensburg.de/CatLR.pdf\#thm.2.3.27}{2.3.27}}]{Cisinski} for example.
%	
%	So to compute a pushout in $\cat{An}$ or $\cat{Cat}_\infty$, write it down on the level of simplicial sets, replace at least one leg by a cofibration (via \cref{lem:SmallObjectArgument}), take the pushout in $\cat{sSet}$, and finally replace the result by a Kan complex or quasi-category, respectively. Note that we can skip the final replacement step for pullbacks, since then the result is already a Kan complex or a quasi-category.\hfill$\blacksquare$
%\end{numpar}
%
But there's also a description of limits and colimits that works in full generality and doesn't rely on the simplicial model.\footnote{I'd like to see a proof of model category fact~\cref{par:HomotopyPushout} using only \cref{lem:ColimitsInAnima} below; I'm not sure if this works, so I'll leave it to you to figure out.} To formulate this, we need to introduce some notation. Let $\Ii$ be an $\infty$-category and let $p\colon \Uu\rightarrow \Ii$ a cocartesian fibration. Furthermore, let $\Fun_\Ii(\Ii,\Uu)\coloneqq \Fun(\Ii,\Uu)\times_{\Fun(\Ii,\Ii)}\{\id_\Ii\}$, the pullback being taken in $\cat{Cat}_\infty$ (but we could take it in $\cat{sSet}$ as well by~\cref{par:HomotopyPushout}) and let
\begin{equation*}
	\Fun_\Ii^{\mathrm{cocart}}(\Ii,\Uu)\subseteq \Fun_\Ii(\Ii,\Uu)
\end{equation*}
be the full sub-$\infty$-category spanned by those $\Ii\rightarrow \Uu$ such that all morphisms in $\Ii$ are sent to $p$-cocartesian morphisms. Note that if $p$ is a left fibration, then
\begin{equation*}
	\Fun_\Ii^{\mathrm{cocart}}(\Ii,\Uu)\simeq \Fun_\II(\Ii,\Uu)\simeq \Hom_{\Cat_{\infty/\Ii}}(\Ii,\Uu)\,.
\end{equation*}
Indeed, the first equivalence is clear since in this case all morphisms in $\Uu$ are $p$-cocartesian by \cref{lem:CocartesianLeft}. The second equivalence follows from \cref{cor:HomInSliceCategories} combined with the facts that $\core\colon \cat{An}\rightarrow \cat{Cat}_\infty$ preserves pullbacks (because it is a right adjoint by \cref{exm:Adjunctions}\cref{enum:AnToCatInfty}) and that $\Fun_\Ii(\Ii,\Uu)$ is already an anima (by \cref{cor:FKanFibration} and \cref{cor:LeftFibrationsOverAnima}).
\begin{lem}\label{lem:ColimitsInAnima}
	Let $F\colon \Ii\rightarrow \Cat_\infty$ be a functor and let $p\colon \Uu\rightarrow \Ii$ be its cocartesian unstraightening. Then the colimit and the limit of $F$ in $\cat{Cat}_\infty$ are given by
	\begin{equation*}
		\colimit_{i\in\Ii}F(i)\simeq \Uu\left[\{\text{cocartesian morphisms}\}^{-1}\right]\quad\text{and}\quad
		\limit_{i\in\Ii}F(i)\simeq \Fun_\Ii^{\mathrm{cocart}}(\Ii,\Uu)\,.
	\end{equation*}
	In particular, if $F$ takes values in $\cat{An}$, then the colimit and the limit of $F$ in $\cat{An}$ are given by
	\begin{equation*}
		\colimit_{i\in\Ii}F(i)\simeq \abs*{\Uu}\quad\text{and}\quad \limit_{i\in\Ii}F(i)\simeq \Hom_{\Cat_{\infty/\Ii}}(\Ii,\Uu)\,.
	\end{equation*}
\end{lem}
For the proof, we need the following lemma. In \cref{cor:HomPreservesLimits}, a more general version of \cref{lem:HomPreservesPullbacks} is proved, but we need this special case as an input.
\begin{lem}\label{lem:HomPreservesPullbacks}
	For every $\infty$-category $\Cc$, the functor $\Hom_{\Cat_\infty}(\Cc,-)\colon \cat{Cat}_\infty\rightarrow \cat{An}$ preserves pullbacks.
\end{lem}
\begin{proof}[Proof sketch]
	As explained in footnote~\cref{footnote:Fun} in \cref{exm:Adjunctions}, Lurie constructs an equivalence of functors $\Hom_{\Cat_\infty}(-,-)\simeq \core\Fun(-,-)$ in \cite[Proposition~\HAthm{5.2.1.11}]{HA}, provided that you go with \cref{con:HomTwAr}. Thanks to \cref{lem:HomRealityCheck}, this proves $\Hom_{\Cat_\infty}(\Cc,-)\simeq \core\Fun(\Cc,-)$, no matter whether you use \cref{con:HomInTwoVariables} or~\labelcref{con:HomTwAr}. Now the claim is obvious, since both $\core\colon \cat{Cat}_\infty\rightarrow \cat{An}$ and $\Fun(\Cc,-)\colon \cat{Cat}_\infty\rightarrow\cat{Cat}_\infty$ are right adjoints by \cref{exm:Adjunctions}.
	
	In fact, to show preservation of pullbacks in this way, we can get away with a little less than Lurie's result: We only need that $\Hom_{\Cat_\infty}(\Cc,-)$ and  $\core\Fun(\Cc,-)$ agree on objects and morphisms. The former is clear by \cref{thm:CordierPorter}. Unfortunately, the latter still needs some care (and simplicial arguments): We know what $\core \Fun(-,-)$ does on morphisms, because it agrees with $\N^\Delta(\core\F(-,-))$ (the argument is in \cref{exm:Adjunctions}). For $\Hom_{\cat{Cat}_\infty}(\Cc,-)$, we need to unravel what straightening does on morphisms; this is quite nasty, but doable via \cref{par:StraighteningOnMorphisms}.
\end{proof}
\begin{proof}[Proof sketch of \cref{lem:ColimitsInAnima}]
	The idea in all of these statements is that the unstraightening of a constant functor $\const X$ is precisely the projection $\pr_2\colon X\times \Ii\rightarrow \Ii$. Let's first consider the case of colimits in $\cat{An}$ and see where this ideas takes us. To show that $\abs*{\Uu}$ is the desired colimit, we want an equivalence $\Hom_{\cat{An}}(\abs{\Uu},-)\simeq \Hom_{\Fun(\Ii,\cat{An})}(F,\const(-))$. Let's start manipulating the right-hand side. By \cref{thm:Straightening}\cref{enum:LeftStraightening}, $\Fun(\Ii,\cat{An})\simeq\cat{Left}(\Ii)$, hence
	\begin{equation*}
		\Hom_{\Fun(\Ii,\cat{An})}\bigl(F,\const(-)\bigr)\simeq \Hom_{\cat{Left}(\Ii)}\left(\Uu,\operatorname{Un}^{\mathrm{left}}\bigl(\const(-)\bigr)\right)\,.
	\end{equation*}
	The unstraightening of $\const X\colon \Ii\rightarrow\cat{An}$ is the projection $X\times \Ii\rightarrow\Ii$, functorially in $X\in\cat{An}$ (this is a consequence of the fact that precomposition corresponds to pullback in \cref{thm:Straightening}\cref{enum:CocartesianStraightening}). So we can continue our manipulations as follows:
	\begin{align*}
		\Hom_{\cat{Left}(\Ii)}(\Uu,-\times\Ii)&\simeq \Hom_{\cat{Cat}_{\infty/\Ii}}(\Uu,-\times\Ii)\\
		&\simeq \Hom_{\Cat_\infty}(\Uu,-\times\Ii)\times_{\Hom_{\Cat_\infty}(\Uu,\Ii)}\{p\}\\
		&\simeq \bigl(\Hom_{\cat{Cat}_\infty}(\Uu,-)\times\Hom_{\cat{Cat}_\infty}(\Uu,\Ii)\bigr)\times_{\Hom_{\Cat_\infty}(\Uu,\Ii)}\{p\}\\
		&\simeq \Hom_{\cat{Cat}_\infty}(\Uu,-)\,.
	\end{align*}
	In the first step we use that $\cat{Left}(\Ii)\rightarrow \cat{Cat}_{\infty/\Ii}$ is fully faithful. In the second step we use \cref{cor:HomInSliceCategories}; by \cref{lem:ColimitsInFunctorCategories}, the pullback is automatically functorial  provided the square from \cref{cor:HomInSliceCategories} is functorial, which it clearly is by construction. In the third step, we use $\Hom_{\Cat_\infty}(\Uu,-\times\Ii)\simeq \Hom_{\cat{Cat}_\infty}(\Uu,-)\times\Hom_{\cat{Cat}_\infty}(\Uu,\Ii)$ by \cref{lem:HomPreservesPullbacks}. Finally, in the fourth step we use that $\Hom_{\cat{Cat}_\infty}(\Uu,\Ii)\times_{\Hom_{\Cat_\infty}(\Uu,\Ii)}\{p\}\simeq \{p\}$ is just a point.
	
	It remains to observe $\Hom_{\cat{An}}(\abs{\Uu},-)\simeq \Hom_{\cat{Cat}_\infty}(\Uu,-)$ because $\abs{\,\cdot\,}\colon \cat{Cat}_\infty\rightarrow\cat{An}$ is left adjoint to the inclusion $\cat{An}\subseteq\cat{Cat}_\infty$. Thus, we have proved $\colimit_{i\in\Ii}F(i)\simeq \abs{\Uu}$ by verifying that $\abs*{\Uu}$ satisfies the desired universal property.
	
	Let us now indicate the necessary changes to prove the other cases. For limits in $\cat{An}$, we can use a similar calculation; the crucial step is $\Hom_{\cat{Cat}_\infty}(-,\Fun_\Ii(\Ii,\Uu))\simeq \Hom_{\cat{Cat}_{\infty/\Ii}}((-)\times\Ii,\Uu)$, which uses \cref{lem:HomPreservesPullbacks}, the adjunction from \cref{exm:Adjunctions}\cref{enum:Currying}, and \cref{cor:HomInSliceCategories}. We leave the details to you. When taking colimits or limits in $\cat{Cat}_\infty$, we can no longer argue that $\cat{Cocart}(\Ii)\rightarrow \cat{Cat}_{\infty/\Ii}$ is fully faithful. Instead, in the colimit case, \cref{lem:NonFullSubcategory} shows that $\Hom_{\cat{Cocart}(\Ii)}(\Uu,-\times\Ii)\subseteq \Hom_{\cat{Cat}_{\infty/\Ii}}(\Uu,\Cc\times\Ii)$ is a collection of path components; we have to check that it agrees with $\Hom_{\cat{Cat}_\infty}(\Uu[\{\text{cocartesian morphisms}\}^{-1}],-)\subseteq \Hom_{\Cat_\infty}(\Uu,-)$, which is also a collection of path components by \cref{lem:Localisation}. This reduces us to checking a condition on the level of $\pi_0$-sets, which is straightforward. A similar argument applies in the limit case.
\end{proof}
\begin{cor}\label{cor:HomPreservesColimits}
	Let $F\colon \Ii\rightarrow\Cc$ be a functor of $\infty$-categories. A natural transformation $c_F\colon \const y\Rightarrow F$ exhibits $y\in \Cc$ as a limit of $F$ if and only if the natural map
	\begin{equation*}
		c_F^*\colon \Hom_\Cc(x,y)\overset{\simeq}{\longrightarrow}\limit_{i\in\Ii}\Hom_\Cc\bigl(x,F(i)\bigr)
	\end{equation*}
	is an equivalence for all $x\in \Cc$. A dual assertion holds for colimits.
\end{cor}
\begin{proof}
	The unstraightening of $\Hom_\Cc(x,F(-))\colon \Ii\rightarrow\cat{An}$ is the left fibration $F^*(\Cc_{x/})\rightarrow \Ii$, the pullback of the slice-$\infty$-category projection $t\colon \Cc_{x/}\rightarrow \Cc$ along $F\colon \Ii\rightarrow\Cc$. Hence, according to \cref{lem:ColimitsInAnima}, $\limit_{i\in\Ii}\Hom_\Cc(x,F(i))\simeq\Hom_{\Cat_{\infty/\Ii}}(\Ii,F^*(\Cc_{x/}))$. Let us now manipulate the right-hand side as follows:
	\begin{align*}
		\Hom_{\Cat_{\infty/\Ii}}\bigl(\Ii,F^*(\Cc_{x/})\bigr)&\simeq \Hom_{\cat{Cat}_\infty}\bigl(\Ii,F^*(\Cc_{x/})\bigr)\times_{t,\Hom_{\cat{Cat}_\infty}(\Ii,\Ii)}\{\id_\Ii\}\\%\Hom_{\Cat_\infty{}_{/\Cc}}\bigl(\Ii,\Cc_{x/}\bigr)\\
		%&\simeq \core\F\bigl(\Ii,\Cc_{x/}\bigr)\times_{t,\core \F(\Ii,\Cc)}\{F\}\\
		&\simeq \Hom_{\cat{Cat}_\infty}\bigl(\Ii,\{x\}\times_{\Cc,s}\Ar(\Cc)\times_{t,\Cc,F}\Ii\bigr)\times_{t,\Hom_{\cat{Cat}_\infty}(\Ii,\Ii)}\{\id_\Ii\}\\
		&\simeq \{\const x\}\times_{\Hom_{\cat{Cat}_\infty}(\Ii,\Cc),s}\Hom_{\cat{Cat}_\infty}\bigl(\Ii,\Ar(\Cc)\bigr)\times_{t,\Hom_{\cat{Cat}_\infty}(\Ii,\Cc)}\{F\}\\
		&\simeq \Hom_{\Fun(\Ii,\Cc)}(\const x,F)\,.
	\end{align*}
	In the first step we plug in \cref{cor:HomInSliceCategories} to write $\Hom_{\Cat_{\infty/\Ii}}$ as a pullback. In the second step, we plug in $F^*(\Cc_{x/})\cong\{x\}\times_{\Cc,s}\Ar(\Cc)\times_{t,\Cc,F}\Ii$. In the third step we use \cref{lem:HomPreservesPullbacks} and simplify the pullback. Finally, in the fourth step we write $\Hom_{\cat{Cat}_\infty}(\Ii,\Cc)\simeq \core \Fun(\Ii,\Cc)$ and $\Hom_{\Cat_\infty}(\Ii,\Ar(\Cc))\simeq \Hom(\Delta^1,\Fun(\Ii,\Cc))\simeq \core\Ar(\Fun(\Ii,\Cc))$ and use the definition of $\Hom_{\Fun(\Ii,\Cc)}(\const x,F)$ from \cref{par:HomInQuasiCategories}; as we've seen in model category fact~\cref{par:HomotopyPushout}, the pullbacks in $\cat{sSet}$ from \cref{par:HomInQuasiCategories} can be taken in $\cat{Cat}_\infty$ as well, and since $\Hom_{\Fun(\Ii,\Cc)}(\const x,F)$ is an anima anyway, it doesn't matter that we apply $\core$ everywhere.
	
	Therefore, at least pointwise, $c_F^*$ takes the form $c_F^*\colon \Hom_\Cc(x,y)\rightarrow \Hom_{\Fun(\Ii,\Cc)}(\const x,F)$ for all $x\in\Cc$. Since a natural transformation is an equivalence if and only if it is a pointwise equivalence (\cref{thm:EquivalencePointwise}), we are done.
\end{proof}
\begin{cor}\label{cor:HomPreservesLimits}
	For every $\infty$-category $\Cc$, the functors $\Hom_\Cc(x,-)\colon \Cc\rightarrow\cat{An}$ and $\Hom_\Cc(-,y)\colon \Cc^\op\rightarrow\cat{An}$ preserve limits for all $x,y\in\Cc$ \embrace{note that limits in $\Cc^\op$ correspond to colimits in $\Cc$}. Likewise, the Yoneda embedding $\Yo_\Cc\colon \Cc\rightarrow\Fun(\Cc^\op,\cat{An})$ preserves limits.
\end{cor}
\begin{proof}
	The first two assertions follow immediately from \cref{cor:HomPreservesColimits}. The last one follows from the first plus the fact that limits and equivalences in functor categories are pointwise by \cref{lem:ColimitsInFunctorCategories} and \cref{thm:EquivalencePointwise}.
\end{proof}

\subsection{Coinitial and initial functors}
Our next goal is to develop a general theory of diagrams that have the same limit or colimit. This is summarised by the following theorem due to Joyal, with a first written proof appearing in \cite[Theorem~\HTTthm{4.1.3.1}]{HTT}.
\begin{thm}[Joyal's version of Quillen's theorem A]\label{thm:JoyalsQuillenA}
	For a functor $\alpha\colon\Ii\rightarrow \Jj$ of $\infty$-categories, the following are equivalent:
	\begin{alphanumerate}
		\item For every $\infty$-category $\Cc$ and every $F\colon \Jj\rightarrow \Cc$, the functor $F$ has a colimit if and only if $F\circ \alpha$ has a colimit. Furthermore, in this case the following natural map is an equivalence:\label{enum:Cofinal}
		\begin{equation*}
			\colimit_{i\in\Ii}F\bigl(\alpha(i)\bigr)\overset{\simeq}{\longrightarrow}\colimit_{j\in\Jj}F(j)\,.
		\end{equation*}
		\item For every right fibration $f\colon X\rightarrow\Jj$, the following natural map 
		is an equivalence:\label{enum:RightAnodyne}
		\begin{equation*}
			\Hom_{\Cat_{\infty/\Jj}}(\Ii,X)\overset{\simeq}{\longrightarrow} \Hom_{\Cat_{\infty/\Jj}}(\Jj,X)\,.
		\end{equation*}
		\item For every $j\in\Jj$, the slice-$\infty$-category $\Ii_{j/}\coloneqq \Ii\times_{\Jj}\Jj_{j/}$ is weakly contractible. That is, we have $\abs{\Ii_{j/}}\simeq *$.\label{enum:WeaklyContractible}
	\end{alphanumerate}
	A dual assertion holds for limits, left fibrations, and the slice-$\infty$-categories $\Ii_{/j}$.
\end{thm}
\begin{defi}
	If $\alpha\colon \Ii\rightarrow\Jj$ satisfies the equivalent conditions from \cref{thm:JoyalsQuillenA}, then $\alpha$ is called \emph{coinitial}. Dually, $\alpha$ is called \emph{initial} if it satisfies the dual equivalent conditions for limits.%
	\footnote{Calling these functors \emph{\embrace{co-}initial} is slightly nonstandard: People usually use the terms \emph{\embrace{co-}final}, but there seems to be no universally agreed convention on which of these terms refers to limits and which refers to colimits. In any case, our naming convention is objectively the correct one. It is uniquely determined by the following two desiderata:
	\begin{alphanumerate}
		\item The concept for colimits should have the presyllable \emph{co-}, whereas the concept for limits should have no presyllable at all.
		\item The inclusion of an initial object should be \emph{initial}, or maybe \emph{cofinal}, but definitely not \emph{final}. Likewise for terminal objects.
		\end{alphanumerate}}
\end{defi}
\begin{exm}\label{exm:Cofinal}
	The following are examples of coinitial functors:
	\begin{alphanumerate}
		\item Right anodyne maps are coinitial. It's clear from \cref{cor:FKanFibration} and \cref{cor:HomInSliceCategories} that the condition from \cref{thm:JoyalsQuillenA}\cref{enum:RightAnodyne} is satisfied.\label{enum:RightAnodyneCofinal}
		\item Right adjoint functors $\alpha\colon \Ii\rightarrow\Jj$ are coinitial. Indeed, if $\beta$ is a left adjoint, then $\alpha^*\colon \Fun(\Jj,\Cc)\shortdoublelrmorphism \Fun(\Ii,\Cc)\noloc \beta^*$ is an adjunction by \cref{cor:FunctorCategoryAdjunctions} and so to verify the condition ${\colimit_\Ii}\circ{\alpha^*}\simeq \colimit_\Jj$ from \cref{thm:JoyalsQuillenA}\cref{enum:Cofinal}, it's enough to check ${\beta^*}\circ{\const}\simeq\const$, which is clear.\label{enum:RightAdjointCofinal}
		\item Localisations $p\colon \Ii\rightarrow \Ii[W^{-1}]$ are coinitial. One way to see this is that localisations are right anodyne, since by construction, $p$ factors into $\Ii\rightarrow\ov{\Ii}\rightarrow \Ii[W^{-1}]$, where the second arrow is inner anodyne and the first arrow is right anodyne, because $\Delta^1\rightarrow J$ is right anodyne. Then \cref{enum:RightAnodyneCofinal} does it.\label{enum:LocalisationsCofinal}
		
		But of course there's also a synthetic way to see this. Since the precomposition functor $p^*\colon\Fun(\Ii[W^{-1}],\Cc)\rightarrow \Fun(\Ii,\Cc)$ is fully faithful by \cref{lem:Localisation}, we have
		\begin{equation*}
			\Hom_{\Fun(\Ii,\Cc)}(F\circ p,\const y)\simeq \Hom_{\Fun(\Ii[W^{-1}],\Cc)}(F,\const y)\,,
		\end{equation*}
		functorially in $F\colon \Ii[W^{-1}]\rightarrow \Cc$ and all $y\in \Cc$, which proves that the condition from \cref{thm:JoyalsQuillenA}\cref{enum:Cofinal} is satified.
	\end{alphanumerate}
\end{exm}
\begin{proof}[Proof of \cref{thm:JoyalsQuillenA}, \cref{enum:Cofinal} $\Leftrightarrow$ \cref{enum:RightAnodyne}]
	Let $F\colon \Jj\rightarrow \cat{An}$ be a functor with unstraightening $\Uu\rightarrow \Jj$. Then the pullback $\alpha^*(\Uu)\rightarrow \Ii$ is the unstraightening of $F\circ\alpha\colon \Ii\rightarrow\cat{An}$. \cref{lem:ColimitsInAnima} shows $\lim_{j\in\Jj}F(j)\simeq\Hom_{\Cat_{\infty/\Jj}}(\Jj,\Uu)$. Similarly,
	\begin{equation*}
		\lim_{i\in\Ii}F\bigl(\alpha(i)\bigr)\simeq\Hom_{\Cat_{\infty/\Ii}}\bigl(\Ii,\alpha^*(\Uu)\bigr)\simeq \Hom_{\Cat_{\infty/\Ii}}(\Ii,\Uu)\,;
	\end{equation*}
	here the second equivalence is a quick calculation using \cref{cor:HomInSliceCategories} and \cref{cor:HomPreservesColimits}. This shows that \cref{enum:RightAnodyne} holds if and only if \cref{enum:Cofinal} holds for functors $F\colon \Jj\rightarrow \cat{An}$. Now let $F\colon \Jj\rightarrow \Cc$ be an arbitrary functor. By \cref{cor:HomPreservesColimits}, $\Yo_\Cc\colon \Cc\rightarrow \Fun(\Cc^\op,\cat{An})$ preserves limits and it is fully faithful, so \cref{enum:Cofinal} holds for $F\colon \Jj\rightarrow \Cc$ if and only if it holds for $\Yo_\Cc\circ F\colon \Jj\rightarrow \Fun(\Cc^\op,\cat{An})$. Finally, limits in $\Fun(\Cc^\op,\cat{An})$ are computed pointwise by \cref{lem:ColimitsInFunctorCategories} and equivalences can be checked pointwise by \cref{thm:EquivalencePointwise}, so \cref{enum:Cofinal} holds for functors into $\Fun(\Cc^\op,\cat{An})$ if and only if it holds for functors into $\cat{An}$. This finishes the proof of \cref{enum:Cofinal} $\Leftrightarrow$ \cref{enum:RightAnodyne}.
\end{proof}
Before we can prove \cref{enum:Cofinal}  $\Rightarrow$ \cref{enum:WeaklyContractible} $\Rightarrow$ \cref{enum:RightAnodyne}, we need another lemma.
\begin{lem}\label{lem:CartesianCofinal}
	A cartesian fibration $p\colon \Uu\rightarrow \Jj$ satisfies the conclusion of \cref{thm:JoyalsQuillenA}\cref{enum:RightAnodyne} if and only if the fibres $p^{-1}\{j\}$ of $p$ are weakly contractible, that is, $\abs*{p^{-1}\{j\}}\simeq *$ for all $j\in \Jj$.
\end{lem}
\begin{proof}
	Let $E\colon \Jj^\op\rightarrow\cat{Cat}_\infty$ be the straightening of $p\colon \Uu\rightarrow \Jj$ and let $f\colon X\rightarrow \Jj$ be a right fibration with straightening $F\colon \Jj^\op\rightarrow \cat{An}$. Then the cartesian straightening equivalence (the dual of \cref{thm:Straightening}\cref{enum:CocartesianStraightening}) shows
	\begin{equation*}
		\Hom_{\Cat_{\infty/\Jj}}(\Uu,X)\simeq \Hom_{\cat{Cart}(\Jj)}(\Uu,X)\simeq \Hom_{\Fun(\Jj^\op,\cat{Cat}_\infty)}(E,F)\,.
	\end{equation*}
	Note that the first equivalence holds even though $\cat{Cart}(\Jj)\rightarrow\cat{Cat}_{\infty/\Jj}$ is not fully faithful, since we're mapping into a right fibration where every morphism is cartesian (by the dual of \cref{lem:CocartesianLeft}). Now $\abs*{\,\cdot\,}\colon \cat{Cat}_\infty\rightarrow \cat{An}$ is left adjoint to the inclusion $\cat{An}\subseteq \cat{Cat}_\infty$ by \cref{exm:Adjunctions}\cref{enum:AnToCatInfty} and that adjunction persists to functor-$\infty$-categories by \cref{cor:FunctorCategoryAdjunctions}. Thus
	\begin{equation*}
		\Hom_{\Fun(\Jj^\op,\cat{Cat}_\infty)}(E,F)\simeq \Hom_{\Fun(\Jj^\op,\cat{Cat}_\infty)}\bigl(\abs{E},F\bigr)\,.
	\end{equation*}
	The cartesian straightening of $\id_\Jj\colon \Jj\rightarrow \Jj$ is $\const *\colon \Jj^\op\rightarrow\cat{An}$. By the same arguments as above we then obtain
	\begin{equation*}
		\Hom_{\Cat_{\infty/\Jj}}(\Jj,X)\simeq \Hom_{\Fun(\Jj^\op,\cat{Cat}_\infty)}(\const *,F)\,.
	\end{equation*}
	Putting everything together, we see that the condition from \cref{thm:JoyalsQuillenA}\cref{enum:RightAnodyne} is satisfied if and only if $p\colon\Uu\rightarrow \Jj$ induces an equivalence $\abs*{E}\Rightarrow \const *$ in $\Fun(\Jj^\op,\cat{An})$. Since equivalences can be checked pointwise (\cref{thm:EquivalencePointwise}), this becomes precisely the condition that all fibres of $p$ are weakly contractible.
\end{proof}
Note that \cref{lem:CartesianCofinal} and \cref{thm:JoyalsQuillenA}\cref{enum:WeaklyContractible} look very much alike, but are a priori two different criteria for a cartesian fibration $p\colon \Uu\rightarrow\Jj$ to be coinitial. As a reality check, let's see that they are indeed equivalent. This isn't necessary to complete our proof of \cref{thm:JoyalsQuillenA}, but we'll need it later.
\begin{lem}\label{lem:CartesianFibres}
	Let $p\colon \Uu\rightarrow \Jj$ be a cartesian fibration. Then for every $j\in\Jj$, the natural functor $p^{-1}\{j\}\rightarrow \Uu\times_\Jj\Jj_{j/}$ admits a right adjoint. In particular, we obtain a homotopy equivalence of animae $\abs{p^{-1}\{j\}}\simeq \abs{\Uu\times_\Jj\Jj_{j/}}$.
\end{lem}
\begin{proof}[Proof sketch]
	By \cref{lem:Adjunction}, right adjoints can be constructed pointwise. This can be done as follows: Fix an object $(u,\ov\varphi)\in \Uu\times_\Jj\Jj_{j/}$, given by an element $u\in \Uu$ and a morphism $\ov\varphi\colon j\rightarrow p(u)$ in $\Jj$. Let $\varphi\colon u'\rightarrow u$ be a $p$-cartesian lift of $\ov\varphi$. Then $u'\in p^{-1}\{j\}$ is a right adjoint object to $(u,\ov\varphi)$ under $p^{-1}\{j\}\rightarrow \Uu\times_\Jj\Jj_{j/}$. To see this, note that, by construction, we have a morphism $c\colon (u',\id_j)\rightarrow (u,\ov\varphi)$ in $\Uu\times_\Jj\Jj_{j/}$ (which will play the role of the counit); using \cref{thm:EquivalencePointwise}, we have to show that the composition
	\begin{equation*}
		\Hom_{p^{-1}\{j\}}(u'',u')\longrightarrow \Hom_{\Uu\times_\Jj\Jj_{j/}}\bigl((u'',\id_j),(u',\id_j)\bigr)\overset{c_*}{\longrightarrow}\Hom_{\Uu\times_\Jj\Jj_{j/}}\bigl((u'',\id_j),(u,\ov\varphi)\bigr)
	\end{equation*}
	is an equivalence for all $u''\in p^{-1}\{j\}$. Now use the characterisation of cartesian morphisms from the dual of \cref{lem:CocartesianMorphisms} together with \cref{cor:HomInSliceCategories} and the fact that Hom animae in pullbacks of $\infty$-categories are pullbacks of the respective Hom animae (which is straightforward to see; we'll prove a more general statement in \cref{lem:HomInLimits}\cref{enum:HomInLimits}) to show that both sides are equivalent to $\Hom_\Uu(u'',u)\times_{\Hom_\Jj(j,p(u))}\{\ov\varphi\}$ and that the morphism between them is equivalent to the identity. We'll leave the details to you.
\end{proof}

\begin{proof}[Proof of \cref{thm:JoyalsQuillenA}, \cref{enum:Cofinal}  $\Rightarrow$ \cref{enum:WeaklyContractible} $\Rightarrow$ \cref{enum:RightAnodyne}]
	Assume \cref{enum:Cofinal} holds true and consider the functor $\Hom_\Jj(j_0,-)\colon \Jj\rightarrow \cat{An}$ for some $j_0\in \Jj$. Its unstraightening is the slice-$\infty$-category projection $t\colon \Jj_{j_0/}\rightarrow\Jj$, hence $\colimit_{j\in\Jj} \Hom_\Jj(j_0,j)\simeq \abs{\Jj_{j_0/}}$ by \cref{lem:ColimitsInAnima}. But $\Jj_{j_0/}$ has an initial element given by $\id_{j_0}$, and so $\{\id_{j_0}\}\shortdoublelrmorphism \Jj_{j_0/}$ is an adjunction. Since adjunctions induce homotopy equivalences after $\abs{\,\cdot\,}$, we conclude $\abs{\Jj_{j_0/}}\simeq *$.
	
	Now consider $\Hom_\Jj(j_0,\alpha(-))\colon \Ii\rightarrow \cat{An}$. Its unstraightening is $\Ii_{j_0/}\simeq \Ii\times_\Jj \Jj_{j_0/}$ (here we use that precomposition with $\alpha$ corresponds to pullback along $\alpha$ under the unstraightening equivalence, see \cref{thm:Straightening}). Combining \cref{lem:ColimitsInAnima} with condition \cref{enum:Cofinal}, we obtain
	\begin{equation*}
		\bigl\lvert\Ii_{j_0/}\bigr\rvert\simeq\colimit_{i\in\Ii}\Hom_\Jj\bigl(j_0,\alpha(i)\bigr)\simeq \colimit_{j\in\Jj}\Hom_\Jj(j_0,j)\simeq *\,,
	\end{equation*}
	as claimed. This finishes the proof of the implication \cref{enum:Cofinal}  $\Rightarrow$ \cref{enum:WeaklyContractible}.
	
	Now assume \cref{enum:WeaklyContractible}. We can factor $\alpha\colon \Ii\rightarrow \Jj$ into $\Ii\rightarrow \Ii\times_{\Jj,s}\Ar(\Jj)\rightarrow \Jj$, where the first functor sends $i\in \Ii$ to the pair $(i,\id_{\alpha(i)}\colon \alpha(i)\rightarrow \alpha(i))$ and the second functor is induced by the target projection $t\colon \Ar(\Jj)\rightarrow \Jj$. It's straightforward to verify that $\Ii\rightarrow \Ii\times_{\Jj,s}\Ar(\Jj)$ is right adjoint to the projection $s\colon \Ii\times_{\Jj,s}\Ar(\Jj)\rightarrow \Ii$.\footnote{For example, one could use \cref{lem:HomInArrowCategories}; alternatively, unit and counit as well as the triangle identities are easily constructed by hand.} By \cref{exm:Cofinal}\cref{enum:RightAdjointCofinal}, we see that the functor $\Ii\rightarrow \Ii\times_{\Jj,s}\Ar(\Jj)$ satisfies \cref{enum:Cofinal}, hence also \cref{enum:RightAnodyne}. Furthermore, a slight generalisation of \cref{exm:Straightening}\cref{enum:ArCocartesianFibration} (which can be proved by the same argument) shows that $t\colon\Ii\times_{\Jj,s}\Ar(\Jj)\rightarrow \Jj$ is a cocartesian fibration. Its fibres $t^{-1}\{j\}\simeq \Ii\times_\Jj \Jj_{j/}$ are weakly contractible. Hence $t\colon\Ii\times_{\Jj,s}\Ar(\Jj)\rightarrow \Jj$ satisfies \cref{enum:RightAnodyne} by \cref{lem:CartesianCofinal}. We conclude that $\alpha\colon \Ii\rightarrow \Jj$ must satisfy \cref{enum:RightAnodyne} as well. Indeed, if $f\colon X\rightarrow \Jj$ is a right fibration, then
	\begin{align*}
		\Hom_{\Cat_{\infty/\Jj}}(\Jj,X)&\simeq \Hom_{\Cat_{\infty/\Jj}}\bigl(\Ii\times_{\Jj,s}\Ar(\Jj),X\bigr)\\
		&\simeq  \Hom_{\Cat_{\infty/\Ii\times_{\Jj,s}\Ar(\Jj)}}\bigl(\Ii\times_{\Jj,s}\Ar(\Jj),t^*(X)\bigr)\,.
	\end{align*}
	In the first equivalence we use \cref{enum:RightAnodyne} for $t\colon\Ii\times_{\Jj,s}\Ar(\Jj)\rightarrow \Jj$. In the second equivalence we let $t^*(X)\rightarrow \Ii\times_{\Jj,s}\Ar(\Jj)$ be the pullback of $f$ along $t$ and use \cref{lem:KanExtensionForRight}\cref{enum:ForgetfulFunctor} below. Now a pullback of a right fibration is again a right fibration, whence
	\begin{align*}
		\Hom_{\Cat_{\infty/\Ii\times_{\Jj,s}\Ar(\Jj)}}\bigl(\Ii\times_{\Jj,s}\Ar(\Jj),t^*(X)\bigr)&\simeq \Hom_{\Cat_{\infty/\Ii\times_{\Jj,s}\Ar(\Jj)}}\bigl(\Ii,t^*(X)\bigr)\\
		&\simeq \Hom_{\Cat_{\infty/\Jj}}(\Ii,X)\,.
	\end{align*}
	In the first equivalence we use \cref{enum:RightAnodyne} for $\Ii\rightarrow \Ii\times_{\Jj,s}\Ar(\Jj)$ and in the second we use \cref{lem:KanExtensionForRight}\cref{enum:ForgetfulFunctor} below again. This finishes the proof of the implication \cref{enum:WeaklyContractible} $\Rightarrow$ \cref{enum:RightAnodyne}.
\end{proof}
\subsection{Kan extensions}\label{subsec:KanExtensions}
We're now working towards an $\infty$-categorical analogue of \cref{thm:1PShFreeCocompletion}. %For an $\infty$-category $\Cc$, we let $\PSh(\Cc)\coloneqq\Fun(\Cc^\op,\cat{An})$ denote the \emph{$\infty$-category of presheaves on $\Cc$} (note that this is \emph{not} compatible with the previous definition if $\Cc$ is an ordinary category).
Our first goal is to construct left Kan extensions for presheaf categories. As it turns out, this is most easily done in the fibration picture.
\begin{lem}\label{lem:KanExtensionForRight}
	Let $F\colon \Cc\rightarrow \Dd$ be a functor of $\infty$-categories.
	\begin{alphanumerate}
		\item \!The pullback functor $F^*\colon \cat{Cat}_{\infty/\Dd}\rightarrow\cat{Cat}_{\infty/\Cc}$ has a left adjoint, namely the forgetful functor $\cat{Cat}_{\infty/\Cc}\rightarrow\cat{Cat}_{\infty/\Dd}$ that sends $f\colon \Cc'\rightarrow \Cc$ to $F\circ f\colon \Cc'\rightarrow \Dd$.\label{enum:ForgetfulFunctor}
		\item \!The inclusion $\cat{Right}(\Dd)\subseteq\cat{Cat}_{\infty/\Dd}$ has a left adjoint that sends $g\colon \Dd'\rightarrow \Dd$ to $q\colon Y\rightarrow \Dd$, where
		\begin{equation*}
			\Dd'\longrightarrow Y\overset{q}{\longrightarrow}\Dd
		\end{equation*}
		is any factorisation of $g$ into a coinitial functor followed by a right fibration.\label{enum:RightCofinalLeftAdjoint}
		\item \!The functor $F^*\colon \cat{Right}(\Dd)\rightarrow \cat{Right}(\Cc)$ has a left adjoint $F_!\colon \cat{Right}(\Cc)\rightarrow \cat{Right}(\Dd)$. On objects, $F_!$ is given as follows: Let $p\colon X\rightarrow \Cc$ be a right fibration and let\label{enum:RightPullbackLeftAdjoint}
		\begin{equation*}
			X\longrightarrow Y\overset{q}{\longrightarrow} \Dd
		\end{equation*}
		be any factorisation of $F\circ p$ into a coinitial functor followed by a right fibration. Then we have $F_!(p\colon X\rightarrow \Cc)\simeq (q\colon Y\rightarrow \Dd)$. In particular, all such factorisations are equivalent.
	\end{alphanumerate}
\end{lem}
\begin{proof}
	For \cref{enum:ForgetfulFunctor}, note that left adjoints can be constructed pointwise by \cref{lem:Adjunction}, so its enough to show that $F\circ f\colon \Cc'\rightarrow \Cc$ is a left adjoint object to $f\colon \Cc'\rightarrow\Cc$ under $F^*$. To this end, let $g\colon \Dd'\rightarrow \Dd$ be an element in $\cat{Cat}_{\infty/\Dd}$. We have a diagram
	\begin{equation*}
		\begin{tikzcd}
			\Hom_{\Cat_{\infty/\Cc}}\bigl(\Cc',F^*(\Dd')\bigr)\dar\rar\drar[pullback]&  \Hom_{\Cat_\infty}\bigl(\Cc',F^*(\Dd')\bigr)\rar\dar\drar[pullback] & \Hom_{\Cat_\infty}(\Cc',\Dd')\dar\\
			\{f\}\rar & \Hom_{\Cat_\infty}(\Cc',\Cc)\rar["F_*"] & \Hom_{\Cat_\infty}(\Cc',\Dd)
		\end{tikzcd}
	\end{equation*}
	in which the left square is a pullback by \cref{cor:HomInSliceCategories} and the right square is a pullback by \cref{cor:HomPreservesColimits}. Hence the outer rectangle is a pullback too. Combining this with \cref{cor:HomInSliceCategories}, we obtain
	\begin{equation*}
		\Hom_{\Cat_{\infty/\Cc}}\bigl(\Cc',F^*(\Dd')\bigr)\simeq\Hom_{\Cat_{\infty/\Dd}}(\Cc',\Dd')\,.
	\end{equation*}
	Since every step in the argument can be made functorial in $g\colon \Dd'\rightarrow \Dd$, we have proved \cref{enum:ForgetfulFunctor}.
	
	For \cref{enum:RightCofinalLeftAdjoint}, note that \cref{enum:ForgetfulFunctor} combined with \cref{thm:JoyalsQuillenA}\cref{enum:RightAnodyne} immediately implies that $q\colon Y\rightarrow \Dd$ is a left adjoint object to $g\colon \Dd'\rightarrow \Dd$ under the inclusion $\cat{Right}(\Dd)\subseteq \cat{Cat}_{\infty/\Dd}$. Since left adjoints can be constructed pointwise by \cref{lem:Adjunction}, we only need to check that such a factorisation always exists. But that's easy! For example, we could choose $\Dd'\rightarrow Y$ to be right anodyne by \cref{lem:SmallObjectArgument} and \cref{exm:Cofinal}\cref{enum:RightAnodyneCofinal}. If you'd like to avoid simplicial sets, we could also argue as follows: Choose a factorisation $\Dd'\rightarrow Y'\rightarrow \Dd$ into a right adjoint functor followed by a cartesian fibration $g'\colon Y'\rightarrow \Dd$ as in the proof of \cref{thm:JoyalsQuillenA}. Then put
	\begin{equation*}
		(g\colon Y\rightarrow \Dd)\coloneqq \operatorname{Un}^{\mathrm{right}}\left(\Dd\xrightarrow{\operatorname{St}^{\mathrm{cart}}(g)}\cat{Cat}_\infty\overset{\abs{\,\cdot\,}}{\longrightarrow}\cat{An}\right)\,.
	\end{equation*}
	Finally, \cref{enum:RightPullbackLeftAdjoint} follows from the combined powers of \cref{enum:ForgetfulFunctor} and \cref{enum:RightCofinalLeftAdjoint}.
\end{proof}
%Then \cref{lem:KanExtensionForRight} implies the following:
In the following, we let $\PSh(\Cc)\coloneqq\Fun(\Cc^\op,\cat{An})$ denote the \emph{$\infty$-category of presheaves on $\Cc$}. Note that this is \emph{not} compatible with the previous definition of presheaves on ordinary categories, which would be $\Fun(\Cc^\op,\cat{Set})$.
\begin{cor}\label{cor:f_!:PC->PD}
	Let $F\colon \Cc\rightarrow \Dd$ be a functor of $\infty$-categories. Then the precomposition functor $F^*\colon \PSh(\Dd)\rightarrow\PSh(\Cc)$ has a left adjoint $F_!$ such that the diagram
	\begin{equation*}
		\begin{tikzcd}
			\Cc\rar["F"]\dar["\Yo_\Cc"']\drar[commutes]& \Dd\dar["\Yo_\Dd"]\\			\PSh(\Cc)\rar["F_!"]&\PSh(\Dd)
		\end{tikzcd}
	\end{equation*}
	commutes in the $\infty$-category $\Cat_\infty$.
\end{cor}
\begin{proof}
	It's clear from \cref{lem:KanExtensionForRight} and the right straightening equivalence (the dual of \cref{thm:Straightening}\cref{enum:LeftStraightening}) that $F_!$ exists, so we only have to show that the diagram commutes. To this end, first note that the natural transformation $\Hom_\Cc(-,-)\Rightarrow\Hom_\Dd(F(-),F(-))$ gets transformed into $\Yo_\Cc\Rightarrow F^*\circ \Yo_\Dd\circ F$ under the equivalence in $\Fun(\Cc^\op\times\Cc,\cat{An})\simeq \Fun(\Cc,\PSh(\Cc))$. Using the adjunction $F_!\dashv F^*$ as well as \cref{cor:FunctorCategoryAdjunctions}, this transformation is adjoint to a natural transformation $F_!\circ \Yo_\Cc\Rightarrow \Yo_\Dd\circ F$.
	
	So our diagram commutes up to natural transformation, and we have to show that said natural transformation is an equivalence. By \cref{thm:EquivalencePointwise}, this can be done pointwise. So choose $x\in\Cc$. Under the straightening equivalence, the functor $\Yo_\Cc(x)\simeq \Hom_\Cc(-,x)$ corresponds to the right fibration $\Cc_{/x}\rightarrow \Cc$. Likewise, $\Yo_\Dd(F(x))\simeq \Hom_\Dd(-,F(x))$ corresponds to $\Dd_{/F(x)}\rightarrow \Dd$. Using \cref{lem:KanExtensionForRight}\cref{enum:RightPullbackLeftAdjoint}, we only have to show that the top horizontal arrow in the diagram
	\begin{equation*}
		\begin{tikzcd}
			\Cc_{/x}\dar\rar\drar[commutes]& \Dd_{/F(x)}\dar\\
			\Cc\rar["F"]&\Dd
		\end{tikzcd}
	\end{equation*}
	is coinitial. But that's easy! Both $\Cc_{/x}$ and $\Dd_{/F(x)}$ have terminal objects, hence there are adjunctions $\Cc_{/x}\shortdoublelrmorphism \{\id_x\}$ and $\Dd_{/F(x)}\shortdoublelrmorphism\{\id_{F(x)}\}$. Hence $*\rightarrow \Cc_{/x}$ and $*\rightarrow \Dd_{/F(x)}$ are both coinitial by \cref{exm:Cofinal}\cref{enum:RightAdjointCofinal}. Since being coinitial is closed under $2$-out-of-$3$ (for example, by the condition from \cref{thm:JoyalsQuillenA}\cref{enum:Cofinal}), $\Cc_{/x}\rightarrow \Dd_{/F(x)}$ must be coinitial too.
\end{proof}
%Before we move on to define and construct general Kan extensions in the $\infty$-categorical world, let us record a very pleasant consequence of \cref{cor:f_!:PC->PD}: We can compute Hom animae in functor $\infty$-categories!
\cref{cor:f_!:PC->PD} allows us to compute Hom animae in functor $\infty$-categories!
\begin{cor}\label{cor:HomInFunctorCats}
	Given functors $F,G\colon \Cc\rightarrow\Dd$ of $\infty$-categories, the anima of natural transformations $\Hom_{\Fun(\Cc,\Dd)}(F,G)$ can be computed as the following limit:
	\begin{equation*}
		\limit_{(x\rightarrow y)\in\TwAr(\Cc)}\Hom_\Dd\bigl(F(x),G(y)\bigr)\coloneqq \limit\left(\TwAr(\Cc)\xrightarrow{\!(s,t)\!}\Cc^\op\times\Cc\xrightarrow{\!F^\op\times G\!}\Dd^\op\times\Dd\xrightarrow{\!\Hom_\Dd\!}\cat{An}\right)\,.
	\end{equation*}
\end{cor}
\begin{proof}
	By \cref{lem:ColimitsInAnima}, the right-hand side can be computed as $\Hom_{\Cat_{\infty/\TwAr(\Cc)}}(\TwAr(\Cc),\Uu)$, where $\Uu$ denotes the unstraightening of ${\Hom_\Dd}\circ(F^\op\times G)\circ (s,t)\colon \TwAr(\Cc)\rightarrow\cat{An}$. Since unstraightening transforms compositions into pullbacks and the unstraightening of $\Hom_\Dd$ is $\TwAr(\Dd)\rightarrow \Dd^\op\times\Dd$ by \cref{con:HomInTwoVariables} or~\labelcref{con:HomTwAr}, we have a pullback diagram
	\begin{equation*}
		\begin{tikzcd}
			\Uu\rar\dar\drar[pullback] & \Uu'\rar\dar\drar[pullback] &[1em] \TwAr(\Dd)\dar\\
			\TwAr(\Cc)\rar["{(s,t)}"] & \Cc^\op\times\Cc\rar["F^\op\times G"] & \Dd^\op\times\Dd
		\end{tikzcd}
	\end{equation*}
	Using \cref{lem:KanExtensionForRight}\cref{enum:RightPullbackLeftAdjoint}, we see $\Hom_{\Cat_{\infty/\TwAr(\Cc)}}(\TwAr(\Cc),\Uu)\simeq\Hom_{\Cat_{\infty/\Cc^\op\times\Cc}}(\TwAr(\Cc),\Uu')$. But these are both left fibrations over $\Cc^\op\times\Cc$, so the Hom anima on the right-hand side can be equivalently computed as $\Hom_{\Fun(\Cc^\op\times\Cc,\cat{An})}(\Hom_\Cc,{\Hom_\Dd}\circ(F^\op\times G))$. Now the \enquote{currying} equivalence  $\Fun(\Cc^\op\times\Cc,\cat{An})\simeq\Fun(\Cc,\PSh(\Cc))$ sends $\Hom_\Cc$ to $\Yo_\Cc$ and ${\Hom_\Dd}\circ(F^\op\times G)$ to $F^*\circ \Yo_\Dd\circ G$, hence the Hom anima under consideration is given by
	\begin{align*}
		\Hom_{\Fun(\Cc,\PSh(\Cc))}\left(\Yo_\Cc,F^*\circ \Yo_\Dd\circ G\right)&\simeq \Hom_{\Fun(\Cc,\PSh(\Dd))}\left(F_!\circ \Yo_\Cc,\Yo_\Dd\circ G\right)\\
		&\simeq \Hom_{\Fun(\Cc,\PSh(\Dd))}\left(\Yo_\Dd\circ F,\Yo_\Dd\circ G\right)\\
		&\simeq \Hom_{\Fun(\Cc,\Dd)}(F,G)\,,
	\end{align*}
	as claimed. For the first equivalence, we use that $F_!\circ -$ is an adjoint of $F^*\circ -$ by construction and \cref{cor:FunctorCategoryAdjunctions}, the second equivalence follows from \cref{cor:f_!:PC->PD}, and the third one since $\Yo_\Dd\colon \Dd\rightarrow\PSh(\Dd)$ is fully faithful by Yoneda's lemma (\cref{cor:YonedaEmbeddingFullyFaithful}).
\end{proof}
We'll now define and construct Kan extensions in the $\infty$-categorical world.
\begin{defi}\label{def:KanExtensions}
	Let $f\colon \Cc\rightarrow \Cc'$ and $F\colon \Cc\rightarrow \Dd$ be functors of $\infty$-categories. A~\emph{left Kan extension of $F$ along $f$}, denoted $\Lan_fF\colon \Cc'\rightarrow\Dd$, is a left adjoint object to $F$ under the precomposition functor $f^*\colon \Fun(\Cc',\Dd)\rightarrow\Fun(\Cc,\Dd)$. Dually, a \emph{right Kan extension of $F$ along $f$}, denoted $\Ran_fF\colon \Cc'\rightarrow\Dd$, is a right adjoint object to $F$ under $f^*$.
\end{defi}
Kan extensions in the $\infty$-categorical world can be computed by the same formula as in the ordinary case (\cref{lem:1KanExtensionFormula}):
\begin{lem}[Kan extension formula]\label{lem:KanExtensionFormula}
	In the situation of \cref{def:KanExtensions}, assume that for all $x'\in \Cc'$ the following colimits exist in $\Dd$:
	\begin{equation*}
		\colimit_{(x,f(x)\rightarrow x')\in \Cc_{/x'}}F(x)\coloneqq \colimit\left(\Cc_{/x'}\longrightarrow \Cc\overset{F}{\longrightarrow}\Dd\right)
	\end{equation*}
	Then $\Lan_fF$ exists and $\Lan_fF(x')$ is given by that colimit.
\end{lem}
To prove this, we first show that taking colimits is functorial in both the indexing $\infty$-category and the functor. As it will turn out during the proof, this is equivalent to constructing a partial left adjoint to the Yoneda embedding $\Yo_\Dd\colon \Dd\rightarrow\PSh(\Dd)$.
\begin{lem}[\enquote{Colimits are functorial}]\label{lem:ColimitsFunctorial}
	Let $\Dd$ be an $\infty$-category. Let $\Tt\subseteq\Cat_{\infty/\Dd}$ be spanned by those $\alpha\colon \Ii\rightarrow \Dd$ that admit a colimit. Consider the functor $\Dd_{/-}\colon \Dd\rightarrow \cat{Cat}_{\infty/\Dd}$ that sends $y\in \Dd$ to $\Dd_{/y}\rightarrow \Dd$. Then $\Dd_{/-}$ lands in $\Tt$ and admits a left adjoint $\colimit\colon \Tt\rightarrow \Dd$ that sends $\alpha\colon \Ii\rightarrow\Dd$ to $\colimit_{i\in\Ii}\alpha(i)\in\Dd$.
\end{lem}
\begin{proof}
	Formally, the functor $\Dd_{/-}\colon \Dd\rightarrow\cat{Cat}_{\infty/\Dd}$ is defined via
	\begin{equation*}
		\Dd\xrightarrow{\Yo_\Dd}\PSh(\Dd)\simeq\cat{Right}(\Dd)\longrightarrow \cat{Cat}_{\infty/\Dd}\,,
	\end{equation*}
	using the Yoneda embedding and the right straightening equivalence (the dual of \cref{thm:Straightening}\cref{enum:LeftStraightening}). It's clear that $\Dd_{/-}$ takes values in $\Tt$. Indeed, $\Dd_{/y}$ has a terminal object and so the colimit over $\Dd_{/y}\rightarrow \Dd$ is just $y$. To prove the second assertion, by \cref{lem:Adjunction}, it's enough to prove that for every $\alpha\colon \Ii\rightarrow \Dd$, the colimit $\colimit_{i\in\Ii}\alpha(i)\in\Dd$ is a left adjoint object to $\alpha$ under $\Dd_{/-}\colon \Dd\rightarrow\Tt$. This can be seen as follows: If $c\simeq \colimit_{i\in\Ii}\alpha(i)$, then the associated natural transformation $\alpha\Rightarrow\const c$ induces a functor $u_\alpha\colon \Ii\rightarrow \Dd_{/c}$ in $\cat{Cat}_{\infty/\Dd}$. We then get a natural transformation
	\begin{equation*}
		\Hom_\Dd(c,-)\xRightarrow{\Dd_{/-}}\Hom_{\Cat_{\infty/\Dd}}\bigl(\Dd_{/c},\Dd_{/-}\bigr)\xRightarrow{u_\alpha^*}\Hom_{\Cat_{\infty/\Dd}}\bigl(\Ii,\Dd_{/-}\bigr)\,.
	\end{equation*}
	Equivalences can be checked pointwise by \cref{thm:EquivalencePointwise}. So choose $y\in\Dd$. We compute
	\begin{align*}
		\Hom_{\Cat_{\infty/\Dd}}\bigl(\Ii,\Dd_{/y}\bigr)&\simeq \{\alpha\}\times_{\Hom_{\Cat_\infty}(\Ii,\Dd),s}\Hom_{\Cat_\infty}\bigl(\Ii,\Ar(\Dd)\times_{t,\Dd}\{y\}\bigr)\\
		&\simeq \{\alpha\}\times _{\Hom_{\Cat_\infty}(\Ii,\Dd),s}\Hom_{\Cat_\infty}\bigl(\Ii,\Ar(\Dd)\bigr)\times_{t,\Hom_{\Cat_\infty}(\Ii,\Dd)}\{\const y\}\\
		&\simeq \Hom_{\Fun(\Ii,\Dd)}(\alpha,\const y)\,,
	\end{align*}
	and this agrees with $\Hom_\Dd(c,y)$ by definition of $c$. In the first step we use \cref{cor:HomInSliceCategories} as well as $\Dd_{/y}\simeq \Ar(\Dd)\times_{t,\Dd}\{y\}$. In the second step we use \cref{cor:HomPreservesLimits}. In the third step, we use  \enquote{currying} in the form of $\Hom_{\Cat_\infty}(\Ii,\Ar(\Dd))\simeq \Hom_{\Cat_\infty}(\Delta^1,\Fun(\Ii,\Dd))$ and then plug in the definition of $\Hom_{\Fun(\Ii,\Dd)}$ as in \cref{par:HomInQuasiCategories}.
\end{proof}
\begin{proof}[Proof of \cref{lem:KanExtensionFormula}]
	Consider the diagram of functors
	\begin{equation*}
		\begin{tikzcd}
			\Fun(\Cc,\Dd)\rar["{(\Yo_\Dd)_*}"] &  \Fun\bigl(\Cc,\PSh(\Dd)\bigr)\dar["\simeq"']\rar[dashed] &[2em]  \Fun\bigl(\Cc',\cat{Right}(\Dd)\bigr)\dar["\simeq"]\\
			& \cat{Right}(\Dd\times\Cc^\op)\rar["(\id_\Dd\times f^\op)_!"] & \cat{Right}\bigl(\Dd\times(\Cc')^\op\bigr)
		\end{tikzcd}
	\end{equation*}
	(the vertical equivalences follow from the right straightening equivalence, see the dual of \cref{thm:Straightening}\cref{enum:LeftStraightening}). Let $F'\colon \Cc'\rightarrow\cat{Right}(\Dd)$ denote the image of $F$ under the top row functors and let $\ov{\Tt}\coloneqq \Tt\cap\cat{Right}(\Dd)$, where $\Tt$ is defined as in \cref{lem:ColimitsFunctorial}. If we can show that $F'$ is contained in the full sub-$\infty$-category $\Fun(\Cc',\ov\Tt)\subseteq \Fun(\Cc',\cat{Right}(\Dd))$, then we can define $\Lan_fF\coloneqq{\colimit}\circ F'\in \Fun(\Cc',\Dd)$. It's clear from the various equivalences and adjunctions involved (more precisely, from \cref{cor:YonedaEmbeddingFullyFaithful}, \cref{lem:KanExtensionForRight}\cref{enum:RightPullbackLeftAdjoint}, and \cref{lem:ColimitsFunctorial} combined with \cref{cor:FunctorCategoryAdjunctions}) that $\Lan_fF$ is indeed a left adjoint object of $F$ under the precomposition functor $f^*\colon \Fun(\Cc',\Dd)\rightarrow \Fun(\Cc,\Dd)$.
	
	So we have to check that $F'$ is indeed contained in $\Fun(\Cc',\ov\Tt)$. The image of $F$ under $(\Yo_\Dd)_*$ followed by the \enquote{currying} equivalence $\Fun(\Cc,\Fun(\Dd^\op,\cat{An}))\simeq \Fun(\Dd^\op\times\Cc,\cat{An})$ is $\Hom_\Dd(-,F(-))\colon \Dd^\op\times\Cc\rightarrow \cat{An}$. Its right unstraightening is
	\begin{equation*}
		\TwAr(\Dd)^\op\times_{t^\op,\Dd^\op,F^\op}\Cc^\op\longrightarrow \Dd\times\Cc^\op\,.
	\end{equation*}
	Indeed, the right unstraightening of $\Hom_\Dd\colon \Dd^\op\times\Dd\rightarrow\cat{An}$ is $(s^\op,t^\op)\colon \TwAr(\Dd)^\op\rightarrow \Dd\times\Dd^\op$ by definition (of either $\Hom_\Dd$ or $\TwAr(\Dd)$, see \cref{con:HomInTwoVariables,con:HomTwAr}), and precomposition with $F\colon \Cc\rightarrow\Dd$ corresponds to pullback along $F^\op$.
	
	By  \cref{lem:KanExtensionForRight}\cref{enum:RightPullbackLeftAdjoint}, the functor $(\id_\Dd\times f^\op)_!$ sends $\TwAr(\Dd)^\op\times_{t^\op,\Dd^\op,F^\op}\Cc^\op\rightarrow \Dd\times\Cc^\op$ to a coinitial replacement of $\TwAr(\Dd)^\op\times_{t^\op,\Dd^\op,F^\op}\Cc^\op\rightarrow \Dd\times\Cc^\op\rightarrow \Dd\times(\Cc')^\op$ by a right fibration. To figure out how such a coinitial replacement looks like, we claim the following:
	\begin{alphanumerate}\itshape
		\item[\boxtimes_1] In the diagram below, both vertical arrows are coinitial:\label{claim:TwArCofinal}
		\begin{equation*}
			\begin{tikzcd}[column sep=-7em]
				& \TwAr(\Dd)^\op\times_{t^\op,\Dd^\op,F\circ t^\op}\TwAr(\Cc)^\op\times_{f\circ s^\op,\Cc',s^\op}\TwAr(\Cc')^\op\dlar[start anchor=186]\drar[start anchor=-6]& \\
				\Cc\times_{f,\Cc',s^\op}\TwAr(\Cc')^\op & & \TwAr(\Dd)^\op\times_{t^\op,\Dd^\op,F^\op}\Cc^\op
			\end{tikzcd}
		\end{equation*} 
	\end{alphanumerate}
	To prove claim~\cref{claim:TwArCofinal}, we first observe that for every $\infty$-category $\Ii$, both the source projection $s^\op\colon \TwAr(\Ii)^\op\rightarrow \Ii$ and the target projection $t^\op\colon \TwAr(\Ii)^\op\rightarrow \Ii^\op$ are coinitial cartesian fibrations. Indeed, cartesianness is clear since $\TwAr(\Ii)^\op\rightarrow\Ii\times\Ii^\op$ is a right fibration and projection to either factor is cartesian. For coinitiality, we use \cref{lem:CartesianCofinal}: The fibre of $t^\op$ over $i\in \Ii^\op$ is $(t^\op)^{-1}\{i\}\simeq \Ii_{/i}$; this follows from \cref{lem:HomRealityCheck}, regardless of which construction of $\TwAr(\Ii)$ you use. Now $\Ii_{/i}$ is weakly contractible since it has a terminal object. The same argument applies to $s^\op$. To apply this observation, observe that in the diagram above, the left vertical arrow is a composition of a base change of $t^\op\colon \TwAr(\Cc)^\op\rightarrow \Cc^\op$ and a base change of $s^\op\colon \TwAr(\Cc')^\op\rightarrow \Cc'$. Since the conditions from \cref{lem:CartesianCofinal} are stable under base change, this proves that the left vertical arrow is indeed coinitial. Similarly, the right vertical arrow is a composition of a base change of $s^\op\colon \TwAr(\Cc)^\op\rightarrow \Cc$ and a base change of $t^\op\colon \TwAr(\Dd)^\op\rightarrow \Dd^\op$, whence the same argument applies.
	
	So we may equivalently look for a coinitial replacement of $\Cc\times_{f,\Cc',s^\op}\TwAr(\Cc')^\op\rightarrow \Dd\times(\Cc')^\op$ by a right fibration. Once again, we won't do this directly; instead, we claim another claim:
	\begin{alphanumerate}\itshape
		\item[\boxtimes_2] In the diagram below, the vertical arrows are cartesian fibrations over $(\Cc')^\op$ and the horizontal arrows preserve cartesian lifts:\label{claim:CartesianDiagram}
		\begin{equation*}
			\begin{tikzcd}[column sep=large]
				\Dd\times(\Cc')^\op\drar["\pr_2"']&[1em] \Cc\times(\Cc')^\op\dar["\pr_2"]\lar["F\times\id_{(\Cc')^\op}"']\dar[phantom,""{name=A}]\arrow[from=1-1,to=A,commutes,pos=0.7]\dar[phantom,""{name=A}]\arrow[from=A,to=1-3,commutes,pos=0.3] &\Cc\times_{\Cc',s^\op}\TwAr(\Cc')^\op \lar["{({s^\op},\,{t^\op})}"']\dlar["t^\op"] \\
				&(\Cc')^\op &
			\end{tikzcd}
		\end{equation*}
	\end{alphanumerate}
	Indeed, by definition of $\TwAr(\Cc')$, the arrow labelled $(s^\op,t^\op)$ is a right fibration, and it's clear that both arrows labelled $\pr_2$ are cartesian fibrations (see \cref{exm:Straightening}\cref{enum:ProjectionsStraightenToConstantFunctors}). Hence $t^\op$, being a composition of cartesian fibrations, is cartesian too. Furthermore, by a simple unravelling, we see that $t^\op$-cartesian lifts are precisely the $(s^\op,t^\op)$-cartesian lifts of $\pr_2$-cartesian lifts, which immediately proves that $(s^\op,t^\op)$ preserves cartesian lifts. Finally, it's clear that $F\times\id_{(\Cc')^\op}$ preserves cartesian lifts, since these are given by those morphisms in $\Cc\times(\Cc')^\op$ and $\Dd\times(\Cc')^\op$ that are equivalences in the first component. This proves claim~\cref{claim:CartesianDiagram}.
	
	The cartesian straightening $\operatorname{St}^{\mathrm{cart}}(t^\op)$ is a functor $\Cc'\rightarrow\cat{Cat}_\infty$. By the diagram above, it comes with a natural transformation $\operatorname{St}^{\mathrm{cart}}(t^\op)\Rightarrow \const \Dd$, so that $\operatorname{St}^{\mathrm{cart}}(t^\op)$ lifts to a functor $\Cc_{/-}\colon \Cc'\rightarrow \cat{Cat}_{\infty/\Dd}$. On objects, $\Cc_{/-}$ is given by sending $x'\in \Cc'$ to the slice category $\Cc_{/x'}$, which becomes an object in $\cat{Cat}_{\infty/\Dd}$ via
	\begin{equation*}
		\Cc_{/x'}\longrightarrow \Cc\overset{F}{\longrightarrow}\Dd\,.
	\end{equation*}
	Now that's something we've seen before! Our assumption that the functor above admits a colimit precisely tells us that $\Cc_{/-}$ restricts to a functor $\Cc_{/-}\colon \Cc'\rightarrow\Tt$. To finish the proof, let $c\colon \cat{Cat}_{\infty/\Dd} \rightarrow \cat{Right}(\Dd)$ denote the left adjoint to $\cat{Right}(\Dd)\subseteq \cat{Cat}_{\infty/\Dd}$, which exists due to \cref{lem:KanExtensionForRight}\cref{enum:RightCofinalLeftAdjoint}. It's clear from \cref{thm:JoyalsQuillenA}\cref{enum:Cofinal} that $c$ sends $\Tt$ to $\ov\Tt$, hence we obtain a functor $c\circ \Cc_{/-}\colon \Cc'\rightarrow\ov\Tt$. We claim that this finally allows us to compute the desired coinitial replacement:
	\begin{alphanumerate}\itshape
		\item[\boxtimes_3] If $p\colon X\rightarrow \Dd\times(\Cc')^\op$ is a coinitial replacement of $\Cc\times_{f,\Cc',s^\op}\TwAr(\Cc')^\op\rightarrow \Dd\times(\Cc')^\op$ by a right fibration, then the image of $p$ under $\cat{Right}(\Dd\times(\Cc')^\op)\simeq \Fun(\Cc',\cat{Right}(\Dd))$ will coincide with $c\circ \Cc_{/-}$.\label{claim:CofinalReplacement}
	\end{alphanumerate}
	To prove claim~\cref{claim:CofinalReplacement}, consider the following diagram, in which the dashed arrows are left adjoints (whose existence we're going to prove below):
	\begin{equation*}
		\begin{tikzcd}
			\bigl(\Cat_{\infty/(\Cc')^\op}\bigr)_{/\Dd\times(\Cc')^\op}\dar["\simeq"']\drar[commutes] & \cat{Cart}\bigl((\Cc')^\op\bigr)_{/\Dd\times(\Cc')^\op}\rar["\simeq"]\lar\dar[dashed,shift right=0.2em,"c"']\drar[commutes] & \Fun\bigl(\Cc',\Cat_{\infty/\Dd}\bigr)\dar[dashed,shift right=0.2em,"c_*"']\\
			\Cat_{\infty/\Dd\times(\Cc')^\op}\rar[dashed, shift left=0.2em, "c"] & \cat{Right}\bigl(\Dd\times(\Cc')^\op\bigr)\rar["\simeq"]\lar[shift left=0.2em]\uar[shift right=0.2em] & \Fun\bigl(\Cc',\cat{Right}(\Dd)\bigr)\uar[shift right=0.2em]
		\end{tikzcd}
	\end{equation*}
	The horizontal equivalences as well as commutativity of the square on the right follow from the cartesian straightening equivalence (the dual of \cref{thm:Straightening}). Furthermore, once we know that the left adjoints exist, they will also form a commutative square on the right, since taking left adjoints is always compatible with equivalences. The vertical equivalence on the left follows by inspection (\enquote{a slice of a slice is a slice}). The vertical left adjoint $c_*$ exists by \cref{cor:FunctorCategoryAdjunctions}. The horizontal left adjoint $c\colon \Cat_{\infty/\Dd\times(\Cc')^\op}\rightarrow \cat{Right}(\Dd\times(\Cc')^\op)$ exists by \cref{lem:KanExtensionForRight}\cref{enum:RightCofinalLeftAdjoint}, and it we claim that it induces a left adjoint
	\begin{equation*}
		c\colon  \cat{Cart}\bigl((\Cc')^\op\bigr)_{/\Dd\times(\Cc')^\op}\longrightarrow \cat{Right}\bigl(\Dd\times(\Cc')^\op\bigr)
	\end{equation*}
	to the forgetful functor $\cat{Right}(\Dd\times(\Cc')^\op)\rightarrow \cat{Cart}((\Cc')^\op)_{/\Dd\times(\Cc')^\op}$. Indeed, if $\Uu\rightarrow \Dd\times(\Cc')^\op$ and $\Uu'\rightarrow \Dd\times(\Cc')^\op$ are objects in $\cat{Cart}((\Cc')^\op)_{/\Dd\times(\Cc')^\op}$, then
	\begin{equation*}
		\Hom_{\cat{Cart}((\Cc')^\op)_{/\Dd\times(\Cc')^\op}}(\Uu,\Uu')\longrightarrow \Hom_{\cat{Cat}_{\infty/\Dd\times(\Cc')^\op}}(\Uu,\Uu')
	\end{equation*}
	is usually \emph{not} an equivalence, only an inclusion of path components, since on the left-hand side, cartesian lifts need to be preserved. However, if $\Uu'\rightarrow \Dd\times(\Cc')^\op$ happens to be a right fibration, then cartesian lifts are preserved automatically\footnote{It's easy to get confused here: $\Uu'\rightarrow (\Cc')^\op$ need not be a right fibration, so we can't appeal to (the dual of) \cref{lem:CocartesianLeft} directly. But the argument is still straightforward: If $\Uu'\rightarrow \Dd\times (\Cc')^\op$ is a right fibration, then \emph{any} lift of a cartesian morphism in $\Dd\times(\Cc')^\op$ will be cartesian again, thanks to (the dual of) \cref{lem:CocartesianLeft}. So a morphism $\Uu\rightarrow\Uu'$ in $\cat{Cat}_{\infty/\Dd\times(\Cc')^\op}$ preserves cocartesian lifts if and only if $\Uu\rightarrow \Dd\times(\Cc')^\op$ does. But the latter is true by definition, since $\Uu\rightarrow \Dd\times(\Cc')^\op$ is a morphism in $\cat{Cart}((\Cc')^\op)$ if $\Uu$ is an object of the slice $\infty$-category $\cat{Cart}((\Cc')^\op)_{/\Dd\times(\Cc')^\op}$.}, so in this case we \emph{do} get an equivalence, which proves that $c$ is still a left adjoint when restricted along the non-fully faithful functor $\cat{Cart}((\Cc')^\op)_{/\Dd\times(\Cc')^\op}\rightarrow \cat{Cat}_{\infty/\Dd\times(\Cc')^\op}$. So we've proved that the diagram above also commutes if we take the dashed left adjoints into account. This is precisely what we need to prove claim~\cref{claim:CofinalReplacement}.
	
	Using claim~\cref{claim:CofinalReplacement}, we've now succeeded in proving that $F\colon \Cc'\rightarrow\cat{Right}(\Dd)$ takes values in $\ov\Tt$, which proves that $\Lan_fF$ exists. Furthermore, for every $c'\in\Cc'$, the value $\Lan_fF(x')$ is given by a colimit over $c(\Cc_{/x'})$. Since the unit morphism $u_{\Cc_{/x'}}\colon \Cc_{/x'}\rightarrow c(\Cc_{/x'})$ is coinitial by \cref{lem:KanExtensionForRight}\cref{enum:RightPullbackLeftAdjoint}, we may as well take the colimit over $\Cc_{/x'}$. This proves that $\Lan_fF(x')$ is given by the desired formula and we're finally done!		
\end{proof}
\begin{cor}%[\enquote{Kan extensions along fully faithful functors behave nicely.}]
	\label{cor:KanExtensionAlongFullyFaithful}
	In the situation from \cref{def:KanExtensions}, assume that $f\colon \Cc\rightarrow \Cc'$ is fully faithful and that the colimits from \cref{lem:KanExtensionFormula} exist in $\Dd$. Then the natural transformation $u_F\colon F\Rightarrow \Lan_fF\circ f$ is an equivalence.
\end{cor}
\begin{proof}
	This follows from the same argument as in \cref{cor:1KanExtensionAlongFullyFaithful}, plus the fact that equivalences can be checked pointwise by \cref{thm:EquivalencePointwise}.
\end{proof}
We can now state the main result of this section: the $\infty$-categorical analogue of \cref{thm:1PShFreeCocompletion}!
\begin{thm}[\enquote{$\PSh(\Cc)$ arises by freely adding colimits to $\Cc$.}]\label{thm:PShFreeCocompletion}
	Let $\Cc$ and $\Dd$ be $\infty$-categories, where $\Dd$ has all colimits. Then restriction along the Yoneda embedding $\Yo_\Cc$ induces an equivalence
	\begin{equation*}
		\Yo_\Cc^*\colon \Fun^{\colimit}\bigl(\PSh(\Cc),\Dd\bigr)\overset{\simeq}{\longrightarrow}\Fun(\Cc,\Dd)\,.
	\end{equation*}
	Here $\Fun^{\colimit}(\PSh(\Cc),\Dd)\subseteq \Fun(\PSh(\Cc),\Dd)$ is the full sub-$\infty$-category spanned by the colimit-preserving functors. Furthermore, every colimit-preserving functor $\PSh(\Cc)\rightarrow \Dd$ admits a right adjoint.
\end{thm}
As it turns out, the proof will be exactly the same as for ordinary categories. Let's start with the two lemmas whose proofs where omitted in the ordinary case.%; at last, this will be rectified now!
\begin{lem}[\enquote{Every presheaf is a colimit of representables.}]\label{lem:PresheafColimitOfRepresentables}
	Let $\Cc$ be an $\infty$-category. For every $E\in \PSh(\Cc)$, the natural morphism
	\begin{equation*}
		\colimit_{(y,\Hom_\Cc(-,y)\rightarrow E)\in \Cc_{/E}}\Hom_\Cc(-,y)\overset{\simeq}{\longrightarrow}E
	\end{equation*}
	 is an equivalence.
\end{lem}
\begin{proof}
	Since we get the natural transformation for free, we can check pointwise whether it is an equivalence (\cref{thm:EquivalencePointwise}). So fix $x\in\Cc$. Since colimits in $\PSh(\Cc)$ are computed pointwise (\cref{lem:ColimitsInFunctorCategories}), what we need to show is
	\begin{equation*}
		\colimit\left(\Cc_{/E}\overset{s}{\longrightarrow}\Cc\xrightarrow{\Hom_\Cc(x,-)}\cat{An}\right)\simeq E(x)\,.
	\end{equation*}
	By \cref{lem:ColimitsInAnima}, the colimit on the left-hand side is given by $\abs*{\Uu}$, where $\Uu$ is the unstraightening of $\Hom_\Cc(x,-)\circ s$. Since precomposition transforms into pullbacks under unstraightening, we find that $\Uu$ sits inside a pullback
	\begin{equation*}
		\begin{tikzcd}
			\Uu\rar\dar\drar[pullback] &\Cc_{/E}\rar \dar["s"]\drar[pullback]& \PSh(\Cc)_{/E}\dar["s"]\\
			\Cc_{x/}\rar & \Cc\rar["\Yo_\Cc"] & \PSh(\Cc)
		\end{tikzcd}
	\end{equation*}
	Since $\PSh(\Cc)_{/E}\rightarrow\PSh(\Cc)$ is a right fibration, $\Uu\rightarrow \Cc_{x/}$ is one too. In particular, it is a cartesian fibration. Hence \cref{lem:CartesianFibres} shows $\abs{\Uu\times_{\Cc_{x/}}\{\id_x\}}\simeq \abs{\Uu\times_{\Cc_{x/}}(\Cc_{x/})_{{(\id_x\colon x\rightarrow x)}/}}\simeq \abs{\Uu}$; here we use $(\Cc_{x/})_{{(\id_x\colon x\rightarrow x)}/}\simeq \Cc_{x/}$ (\enquote{a slice of a slice is a slice}). Now
	\begin{align*}
		\bigl\lvert\Uu\times_{\Cc_{x/}}\{\id_x\}\bigr\rvert\simeq \Uu\times_{\Cc_{x/}}\{\id_x\}&\simeq \PSh(\Cc)_{/E}\times_{\PSh(\Cc)}\left\{\Yo_\Cc(x)\right\}\\
		&\simeq \Hom_{\PSh(\Cc)}\bigl(\Yo_\Cc(x),E\bigr)\\
		&\simeq E(x)\,.
	\end{align*}
	In the first step, we use that the fibre $\Uu\times_{\Cc_{x/}}\{\id_x\}$ is already an anima, since $\Uu\rightarrow \Cc_{x/}$ is a right fibration. The second equivalence follows from the pullback diagram above. In the third step, we use the definition of $\Hom_{\PSh(\Cc)}$, and in the fourth step, we use Yoneda's lemma (\cref{thm:Yoneda}). In total, we find $\abs*{\Uu}\simeq E(x)$, which is exactly what we wanted to prove.
\end{proof}
\begin{lem}\label{lem:LanAlongYonedaHasRightAdjoint}
	For every $F\colon \Cc\rightarrow \Dd$, the left Kan extension $\Lan_{\Yo_\Cc}F\colon \PSh(\Cc)\rightarrow\Dd$ \embrace{which exists due to \cref{lem:KanExtensionFormula}} admits a right adjoint. The right adjoint sends $y\in \Dd$ to $\Hom_\Dd(F(-),y)\colon \Cc^\op\rightarrow\cat{Set}$.
\end{lem}
\begin{proof}
	Fix $y\in\Dd$. Since adjoints can be constructed pointwise (\cref{lem:Adjunction}), we only need to construct an equivalence
	\begin{equation*}
		\Hom_\Dd\bigl(\Lan_{\Yo_\Cc}F(-),y\bigr)\simeq \Hom_{\PSh(\Cc)}\bigl(-,\Hom_\Dd(F(-),y)\bigr)
	\end{equation*}
	of functors $\PSh(\Cc)^\op\rightarrow\cat{An}$. Restricting along $\Yo_\Cc^\op\colon \Cc^\op\rightarrow\PSh(\Cc)^\op$, both sides become $\Hom_\Dd(F(-),y)$: The left-hand side by \cref{cor:KanExtensionAlongFullyFaithful}, the right-hand side by Yoneda's lemma (\cref{thm:Yoneda}; see also \cref{par:YonedaFunctorial}). By the universal property of right Kan extension, we thus obtain natural transformations
	\begin{equation*}
		\Hom_\Dd\bigl(\Lan_{\Yo_\Cc}F(-),y\bigr)\Longrightarrow \Ran_{\Yo_\Cc^\op} \Hom_\Dd\bigl(F(-),y\bigr)\Longleftarrow\Hom_{\PSh(\Cc)}\bigl(-,\Hom_\Dd\bigl(F(-),y\bigr)\bigr)\,.
	\end{equation*}
	We claim that they're both equivalences. In either case, this can be checked pointwise by \cref{thm:EquivalencePointwise}. So plug in some $E\in \PSh(\Cc)$. We obtain a diagram
	\begin{equation*}
		\begin{tikzcd}[column sep=-0.5em]
			\Hom_\Dd\bigl(\Lan_{\Yo_\Cc}F(E),y\bigr)\rar\drar[bend right=15,end anchor=180,"\simeq"']\drar[commutes,pos=0.55]& \Ran_{\Yo_\Cc^\op} \Hom_\Dd\bigl(F(E),y\bigr)\dar["\simeq"] &\Hom_{\PSh(\Cc)}\bigl(E,\Hom_\Dd\bigl(F(-),y\bigr)\bigr)\dlar[commutes,pos=0.55]\lar\dlar[bend left=15,end anchor=0,"\simeq"]\\
			& \limit_{(x,\Yo_\Cc(x)\rightarrow E)\in (\Cc_{/E})^\op}\Hom_\Dd\bigl(F(x),y\bigr)
		\end{tikzcd}
	\end{equation*}
	The vertical arrow in the middle is an equivalence by the dual of \cref{lem:KanExtensionFormula}. For the vertical arrow on the left, we plug in the left Kan extension formula from \cref{lem:KanExtensionFormula} and use \cref{cor:HomPreservesColimits} to see that $\Hom_\Dd(-,y)$ transforms the colimit into a limit. For the vertical arrow on the right, we plug in \cref{lem:PresheafColimitOfRepresentables}, use \cref{cor:HomPreservesColimits} again to see that $\Hom_{\PSh(\Cc)}(-,\Hom_\Dd(F(-),y))$ transforms the colimit into a limit, and then use Yoneda's lemma. This proves that we obtain equivalences as desired.
\end{proof}
%The final ingredient in the proof of \cref{thm:PShFreeCocompletion} is the $\infty$-categorical analogue of \cref{lem:1FullyFaithfulConservativeAdjunction}.
\begin{lem}\label{lem:FullyFaithfulConservativeAdjunction}
	Let $\Cc$ and $\Dd$ be categories and let $L\colon \Cc\shortdoublelrmorphism \Dd\noloc R$ be an adjunction.
	\begin{alphanumerate}
		\item \!The left adjoint $L$ is fully faithful if and only if the unit transformation $u\colon \id_\Cc\Rightarrow RL$ is an equivalence.\label{enum:FullyFaithfulIffUnitEquivalence}
		\item Suppose the condition from \cref{enum:FullyFaithfulIffUnitEquivalence} is true. Furthermore, suppose that $R$ is conservative \embrace{that is, if $\alpha\colon x\rightarrow y$ is a morphism in $\Dd$ such that $R(\alpha)$ is an equivalence, then $\alpha$ is an equivalence too}. Then $L$ and $R$ are inverse equivalences of categories.\label{enum:Conservative}
	\end{alphanumerate}
\end{lem}
\begin{proof}
	The proof of \cref{lem:1FullyFaithfulConservativeAdjunction} can be copied verbatim.
\end{proof}
\begin{proof}[Proof of \cref{thm:PShFreeCocompletion}]
	By \cref{lem:LanAlongYonedaHasRightAdjoint} and \cref{lem:AdjointsPreserveColimits}, the adjunction $\Lan_{\Yo_\Cc}\dashv\Yo_\Cc^*$ restricts to an adjunction
	\begin{equation*}
		\Lan_{\Yo_\Cc}\colon \Fun(\Cc,\Dd)\doublelrmorphism\Fun^{\colimit}\bigl(\PSh(\Cc),\Dd\bigr)\noloc \Yo_\Cc^*\,.
	\end{equation*}
	By \cref{lem:FullyFaithfulConservativeAdjunction}\cref{enum:Conservative}, to prove that $\Lan_{\Yo_\Cc}$ and $\Yo_\Cc^*$ are inverse equivalences, we need to show that the unit $u\colon \id_{\Fun(\Cc,\Dd)}\Rightarrow\Yo_\Cc^*\circ \Lan_{\Yo_\Cc}$ is an equivalence and that $\Yo_\Cc^*$ is conservative. That $u$ is an equivalence can be checked object-wise by \cref{thm:EquivalencePointwise}, where it follows from \cref{cor:KanExtensionAlongFullyFaithful}, since the Yoneda embedding $\Yo_\Cc$ is fully faithful (\cref{cor:YonedaEmbeddingFullyFaithful}). To see that $\Yo_\Cc^*$ is conservative, we must show that a natural transformation $\eta\colon F\Rightarrow G$ between colimit-preserving functors $F,G\colon \PSh(\Cc)\rightarrow\Dd$ is an equivalence already if it is an equivalence when restricted to representable presheaves. But this is clear since every presheaf can be written as a colimit of representables (\cref{lem:PresheafColimitOfRepresentables}).
\end{proof}
\subsection{Homology, cohomology, Eilenberg--MacLane animae}\label{subsec:EilenbergMacLane}
\cref{thm:PShFreeCocompletion} is surprisingly powerful even in the special case $\Cc\simeq *$. In this case we have $\PSh(*)\simeq \cat{An}$ and so  \cref{thm:PShFreeCocompletion} says that a colimit-preserving functor $\cat{An}\rightarrow\Dd$ is uniquely determined by what it does on $*\in\cat{An}$.\footnote{If you think about this fact for a bit, it becomes very natural: \cref{thm:SimplicialApproximation} says that animae are essentially CW complexes and every CW complex is glued together from topological disks $D^n$. But $D^n\simeq *$. So it makes sense that $*$ should generate all of $\cat{An}$ under colimits. Another way to see this is via \cref{lem:ColimitsInAnima}: It's immediately clear that $X\simeq \colimit(\const*\colon X\rightarrow\cat{An})$ for all $X\in\cat{An}$.} Using this observation, our goal in this subsection is to give a purely abstract proof of the Eilenberg--MacLane theorem (\cref{thm:EilenbergMacLane}).

The first step is to construct an interesting $\infty$-category $\Dd$ with all colimits: For a ring $R$ (not necessarily commutative), we'll give a brief introduction to the \emph{derived $\infty$-category} $\Dd(R)$ and its variant $\Dd_{\geqslant 0}(R)$.

\begin{numpar}[Crash course in derived $\infty$-categories I: Basic definitions.]\label{con:DerivedCategoryI}%Of course, \enquote{very brief} means, unfortunately, that we won't give any proofs, only references.
	Let $\Ch(R)$ be the category of chain complexes
	\begin{equation*}
		M_*=\left(\dotsb \overset{\partial}{\longrightarrow} M_{n+1}\overset{\partial}{\longrightarrow} M_n\overset{\partial}{\longrightarrow} M_{n-1}\overset{\partial}{\longrightarrow}\dotsb\right)
	\end{equation*}
	of left $R$-modules and let $\Ch_{\geqslant 0}(R)\subseteq \Ch(R)$ be the full subcategory of those chain complexes that satisfy $M_n\cong 0$ for $n<0$. We usually write $\mathrm Z_n(M_*)\coloneqq \ker(\partial\colon M_n\rightarrow M_{n-1})$ and $\mathrm B_n(M_*)\coloneqq \im(\partial\colon M_{n+1}\rightarrow M_n)$. The quotient $\H_n(M_*)\coloneqq \mathrm Z_n(M_*)/\mathrm B_n(M_*)$ is called the \emph{$n$\textsuperscript{th} homology of $M_*$}. A morphism $\alpha\colon M_*\rightarrow N_*$ in $\Ch(R)$ is called a \emph{quasi-isomorphism} if $\H_n(\alpha)\colon \H_n(M_*)\overset{\cong}{\longrightarrow} \H_n(N_*)$ is an isomorphism for all $n$. Then we put
	\begin{align*}
		\Dd(R)&\coloneqq \Ch(R)\left[\{\text{quasi-isomorphisms}\}^{-1}\right]\\
		\Dd_{\geqslant 0}(R)&\coloneqq \Ch_{\geqslant 0}(R)\left[\{\text{quasi-isomorphisms}\}^{-1}\right]\,,
	\end{align*}
	where the localisations are taken in the $\infty$-categorical sense (see \cref{con:Localisation}). If you've seen the ordinary derived categories $D(R)$ and $D_{\geqslant 0}(R)$ before, then Corollary/Warning~\cref{cor:Localisation} will convince you that these are simply the homotopy categories $\operatorname{ho}\Dd(R)$ and $\operatorname{ho}\Dd_{\geqslant 0}(R)$.
	
	This definition of $\Dd(R)$ is easy to state, but just from that it's nearly impossible to say anything about colimits in $\Dd(R)$, which is why, we will describe a more explicit construction of $\Dd(R)$ in crash course~\cref{con:DerivedCategoryIII} below. However, already with the abstract definition one can get quite far. For example, let's show that $\Dd_{\geqslant 0}(R)$ is indeed a full sub-$\infty$-category of $\Dd(R)$. To this end, observe that the inclusion $\Ch_{\geqslant 0}(R)\subseteq \Ch(R)$ has a right adjoint $\tau_{\geqslant 0}\colon \Ch(R)\rightarrow \Ch_{\geqslant 0}(R)$ given by \emph{smart truncation}: For a chain complex $M_*$ and an integer $i\in \IZ$, we let $\tau_{\geqslant i}M_*$ be the chain complex given by
	\begin{equation*}
		(\tau_{\geqslant i}M_*)_n\coloneqq \ScaledBracesCases{\!\begin{plaincases*}
				M_n & if $n>i$\\
				\mathrm Z_i(M_*) & if $n=i$\\
				0 & if $n<i$
		\end{plaincases*}}\,,
	\end{equation*}
	so that $\H_n(\tau_{\geqslant i}M_*)\cong \H_n(M_*)$ if $n\geqslant i$ and $\H_n(\tau_{\geqslant i}M_*)\cong 0$ for $n<i$. It's clear that $\tau_{\geqslant i}$ preserves quasi-isomorphisms, hence it descends to a functor $\tau_{\geqslant i}\colon \Dd(R)\rightarrow\Dd_{\geqslant 0}(R)$ by \cref{lem:Localisation}. We claim that $\tau_{\geqslant 0}$ is a right adjoint to $\Dd_{\geqslant 0}(R)\rightarrow \Dd(R)$. By \cref{lem:TriangleIdentities}, it's enough to provide a unit and a counit transformation and to verify the triangle identities. But \cref{lem:Localisation} allows us to inherit all this data from the adjunction $i\colon \Ch_{\geqslant 0}(R)\shortdoublelrmorphism \Ch(R)\noloc \tau_{\geqslant 0}$.\footnote{We've seen a similar argument in \cref{rem:ModelCategoryUnderlyingInftyCategory}. The crucial observation to construct natural transformations via \cref{lem:Localisation} is the following: For every $\infty$-category $\Cc$ and every collection of morphisms $W$ in $\Cc$, the functor
	\begin{equation*}
		\bigl(\Cc\times\Delta^1\bigr)\Bigl[\bigl(W\times\{\id_0\}\cup W\times\{\id_1\}\bigr)^{-1}\Bigr]\overset{\simeq}{\longrightarrow}\Cc[W^{-1}]\times\Delta^1
	\end{equation*}
	(which is itself constructed via \cref{lem:Localisation}) is an equivalence of $\infty$-categories. Back in \cref{rem:ModelCategoryUnderlyingInftyCategory}, we appealed to the explicit simplicial construction, but there's also a model-independent way to see this fact. By Yoneda's lemma, \cref{thm:EquivalencePointwise}, and \cref{lem:Localisation}, it's enough to check for every $\infty$-category $\Dd$ that the morphism of animae $\Hom_{\cat{Cat}_\infty}(\Cc[W^{-1}]\times\Delta^1,\Dd)\rightarrow \Hom_{\Cat_\infty}(\Cc\times\Delta^1,\Dd)$ exhibits the left-hand side as the collection of path components of functors that send $W\times\{\id_0\}\cup W\times\{\id_1\}$ to equivalences. By \cref{exm:Adjunctions}\cref{enum:Currying}, we can rewrite the morphism in question as $\Hom_{\Cat_\infty}(\Cc[W^{-1}],\Ar(\Dd))\rightarrow \Hom_{\Cat_\infty}(\Cc,\Ar(\Dd))$ and then \cref{lem:Localisation} shows that, indeed, we get the correct inclusion of path components.} Now to show that $\Dd_{\geqslant 0}(R)\rightarrow \Dd(R)$ is fully faithful, it's enough to check that the unit is an equivalence (see \cref{lem:FullyFaithfulConservativeAdjunction}\cref{enum:FullyFaithfulIffUnitEquivalence}), which is obvious.
	
	Apart from $\tau_{\geqslant i}\colon \Dd(R)\rightarrow \Dd(R)$, there are some more useful functors that can be constructed directly using our definition of $\Dd(R)$ and \cref{lem:Localisation}. For example, if $M_*$ is a chain complex, its \emph{shift by $i$} is the chain complex $M[i]_*$ given by $M[i]_n\cong M_{n-i}$; the differentials are those of $M_*$, but multiplied by $(-1)^i$ (for technical reasons). It's clear that $(-)[i]\colon \Ch(R)\rightarrow \Ch(R)$ preserves quasi-isomorphisms and so it defines a functor $(-)[i]\colon \Dd(R)\rightarrow \Dd(R)$. For an even more obvious example, consider $\H_n\colon \Ch(R)\rightarrow \cat{Mod}_R$ and $\H_n\colon \Ch_{\geqslant 0}(R)\rightarrow \cat{Mod}_R$. These functors send quasi-isomorphisms to isomorphisms (by definition), hence they define essentially unique functors
	\begin{equation*}
		\H_n\colon\Dd(R)\longrightarrow\cat{Mod}_R\quad\text{and}\quad \H_n\colon\Dd_{\geqslant 0}(R)\longrightarrow\cat{Mod}_R\,.
	\end{equation*}
	by \cref{lem:Localisation}. It's probably clear to you, but let us mention that neither $\mathrm{Z}_n\colon \Ch(R)\rightarrow \cat{Mod}_R$ nor $\mathrm{B}_n\colon \Ch(R)\rightarrow \cat{Mod}_R$ preserves quasi-isomorphisms, so they don't extend to $\Dd(R)$, even though their quotient $\H_n\cong \mathrm{Z}_n/\mathrm{B}_n$ does.
\end{numpar}
	For a chain complex $M_*$, we often write $M$ for its image in $\Dd(R)$ to emphasise that this is no longer a \enquote{complex up to isomorphism}, but a \enquote{complex up to quasi-isomorphism}, so that for $M\in\Dd(R)$ there is no longer a well-defined notion of \enquote{$M_n$, the degree-$n$ part of $M$}.\footnote{On a related note, the inclusion $\Ch_{\geqslant 0}(R)\subseteq \Ch(R)$ also has a left adjoint, which simply replaces everything in negative degrees by $0$. This is called the \emph{stupid truncation}. It doesn't preserve quasi-isomorphisms (hence the name) and so we couldn't have used it to show that $\Dd_{\geqslant 0}(R)\rightarrow \Dd(R)$ is fully faithful.}
\begin{numpar}[Crash course in derived $\infty$-categories II: Simplicialities.]\label{con:DerivedCategoryII}
	The famous \emph{Dold--Kan correspondence} (see \cite[Theorem~\HAthm{1.2.3.7}]{HA} or \cite[\S \href{http://dodo.pdmi.ras.ru/~topology/books/goerss-jardine.pdf\#page=169}{III.2}]{GoerssJardine} for example) states that there is an equivalence
	\begin{equation*}
		\Ch_{\geqslant 0}(\IZ)\overset{\simeq}{\longrightarrow}\cat{sAb}
	\end{equation*}
	between the category of chain complexes in non-negative degrees and the category of simplicial abelian groups. We'll need following two facts about the Dold--Kan correspondence:
	\begin{alphanumerate}\itshape
		\item Every simplicial abelian groups is automatically a Kan complex and every degree-wise surjective morphism in $\cat{sAb}$ maps to a Kan fibration in $\cat{sSet}$.\label{enum:SimplicialAbelianGroupKanComplex}
		\item If $A$ is a simplicial abelian group and $M_*$ is the associated chain complex, then there are isomorphisms $\pi_n(A,a)\cong \H_n(M_*)$ for all $a\in A$ and all $n\geqslant 0$.\label{enum:DoldKanHomologyToHomotopy}
	\end{alphanumerate}
	Fact~\cref{enum:SimplicialAbelianGroupKanComplex} not particularly difficult, but not completely obvious either; see \cite[Tags~\href{https://stacks.math.columbia.edu/tag/08NZ}{08NZ} and~\href{https://stacks.math.columbia.edu/tag/08P0}{08P0}]{Stacks}. Let us sketch how to prove \cref{enum:DoldKanHomologyToHomotopy}. First, we may assume $a=0$, since $(-)+a\colon A\rightarrow A$ is an automorphism of $A$ as a simplicial set and induces an isomorphism $\pi_n(A,0)\cong \pi_n(A,a)$. Now $\pi_n(A,0)\cong [(\Delta^n,\partial\Delta^n),(A,0)]$ by \cref{lem:HomotopyGroupsSimplex}, where $[-,-]$ denotes homotopy classes of maps of pairs. Since $(A,0)$ is a group object, even in the homotopy category of pairs, $[(\Delta^n,\partial\Delta^n),(A,0)]$ inherits a group structure. Using the Eckmann--Hilton trick (see the proof of \cref{lem:HomotopyGroups}\cref{enum:EckmannHilton}), we see that this group structure agrees with the one on $\pi_n(A,0)$.
	
	Using \cref{cor:1FunctorCategoryAdjunctions}, the free-forgetful adjunction $\IZ[-]\colon \cat{Set}\shortdoublelrmorphism \cat{Ab}\noloc {\operatorname{forget}}$ induces a similar adjunction $\IZ[-]\colon \cat{sSet}\shortdoublelrmorphism \cat{sAb}\noloc {\operatorname{forget}}$. Then a map of pairs $(\Delta^n,\partial\Delta^n)\rightarrow (A,0)$ is the same as a morphism $\IZ[\Delta^n]/\IZ[\partial\Delta^n]\rightarrow A$ in $\cat{sAb}$. We are, however, not interested in maps, but \emph{homotopy classes} of maps. Our analysis of the group structure on $\pi_n(A,0)$ shows: Instead of quotienting out the equivalence relation generated by homotopies, we may as well quotient out the subgroup generated by the nullhomotopic maps. Using the results from \cref{sec:SimplicialHomotopyTheory}, it's straightforward to show that, for any pointed Kan complex $(X,x)$, a map of pairs $\sigma\colon (\Delta^n,\partial\Delta^n)\rightarrow (X,x)$ is nullhomotopic if and only if it can be extended to a map of pairs $\overline{\sigma}\colon (\Delta^{n+1},\Lambda_{n+1}^{n+1})\rightarrow (X,x)$ in such a way that $d_{n+1}^*(\overline{\sigma})=\sigma$. By the same reasoning as above, such a map is the same as a morphism $\IZ[\Delta^{n+1}]/\IZ[\Lambda_{n+1}^{n+1}]\rightarrow A$. In total, this proves:
	\begin{equation*}
		\pi_n(A,0)\cong \Hom_{\cat{sAb}}\bigl(\IZ[\Delta^n]/\IZ[\partial\Delta^n],A\bigr)/\Hom_{\cat{sAb}}\bigl(\IZ[\Delta^{n+1}]/\IZ[\Lambda_{n+1}^{n+1}],A\bigr)
	\end{equation*}
	A simple unravelling of the Dold--Kan correspondence shows that $\IZ[\Delta^n]/\IZ[\partial\Delta^n]$ is sent to $\IZ[n]_*$, the chain complex consisting of a single $\IZ$ in degree $n$ and zeros everywhere else. Hence $\Hom_{\cat{sAb}}(\IZ[\Delta^n]/\IZ[\partial\Delta^n],A)\cong \Hom_{\Ch_{\geqslant 0}(\IZ)}(\IZ[n]_*,M_*)\cong \mathrm{Z}_n(M_*)$. A similar analysis shows that $\Hom_{\cat{sAb}}(\IZ[\Delta^{n+1}]/\IZ[\Lambda_{n+1}^{n+1}],A)\cong \mathrm{B}_n(M_*)$. Hence $\pi_n(A,0)\cong \H_n(M_*)$, as desired.
\end{numpar}
\begin{numpar}[Crash course in derived $\infty$-categories III: Projective resolutions.]\label{con:DerivedCategoryIII}
	Recall the simplicial nerve from \cref{con:SimplicialNerve}. It's also possible to construct $\Dd(R)$ and $\Dd_{\geqslant 0}(R)$ in this way; this alternative construction will allow us to study colimits. We'll first explain how to equip $\Dd(R)$ and $\Dd_{\geqslant 0}(R)$ with a Kan enrichment: Let $\Hhom_R(M_*,N_*)$ be the chain complex of abelian groups given by
	\begin{equation*}
		\Hhom_R(M_*,N_*)_n\coloneqq\prod_{i\in\IZ}\Hom_R(M_i,N_{i+n}) \,.
	\end{equation*}
	The differentials send a family of morphisms $f=(f_i)_{i\in\IZ}\in\prod_{i\in\IZ}\Hom_R(M_i,N_{i+n})$ to the family $\partial f\coloneqq(\partial_N\circ f_i-(-1)^nf_{i-1}\circ \partial_M)_{i\in\IZ}$; here $\partial_M$ and $\partial_N$ denote the differentials of $M_*$ and $N_*$, respectively. By unravelling the definitions, we see that the $n$-cycles and $n$-boundaries of $\Hhom_R(M_*,N_*)$ are given by
	\begin{align*}
		\mathrm Z_n\bigl(\Hhom_R(M_*,N_*)\bigr)&\cong \Hom_{\Ch(R)}\bigl(M_*,N[-n]_*\bigr)\\
		\mathrm B_n\bigl(\Hhom_R(M_*,N_*)\bigr)&\cong  \left\{f\in\Hom_{\Ch(R)}\bigl(M_*,N[-n]_*\bigr)\ \middle|\ f\text{ nullhomotopic}\right\}\,.
	\end{align*}
	Here $N[-n]_*$ denotes the shift from crash course~\cref{con:DerivedCategoryI}. Since $\mathrm{H}_n\cong \mathrm{Z}_n/\mathrm{B}_n$, we deduce that $\H_n(\Hhom_R(M_*,N_*))$ is in bijection with the set of homotopy classes of maps $M_*\rightarrow N[-n]_*$.
	
	The complexes $\Hhom_R(-,-)$ provide an enrichment of $\Ch(R)$ over $\Ch(\cat{Ab})$ (in fact, even an enrichment of $\Ch(R)$ over itself). To make this into a Kan enrichment, we let $\tau_{\geqslant 0}\Hhom_R(M_*,N_*)$ be the smart truncation from crash course~\cref{con:DerivedCategoryI} and let $\F_{\Ch(R)}(M_*,N_*)$ denote the simplicial abelian group corresponding to $\tau_{\geqslant0}\Hhom_R(M_*,N_*)$ under the Dold--Kan correspondence. The simplicial abelian groups $\F_{\Ch(R)}(-,-)$ provide an enrichment of $\Ch(R)$ in simplicial sets, which is automatically a Kan enrichment by crash course~\cref{con:DerivedCategoryII}\cref{enum:SimplicialAbelianGroupKanComplex}.
	
	A complex $P_*$ of $R$-modules is called \emph{$K$-projective} if $\Hhom_R(P_*,-)\colon \Ch(R)\rightarrow \Ch(R)$ preserves quasi-isomorphisms. It was shown by Spaltenstein \cite{KProjective} that every chain complex of $R$-modules $M_*$ admits a quasi-isomorphism $P_*\rightarrow M_*$ from a $K$-projective complex. If $P_*$ is $K$-projective, then it is \emph{degree-wise projective} in the sense that every $P_n$ is a projective $R$-module. Conversely, if $P_*$ is degree-wise projective and bounded below in the sense that $P_n\cong0$ for $n\lle 0$, then $P_*$ is $K$-projective. These statements can be found in \cite[Lemma~\href{https://people.math.rochester.edu/faculty/doug/otherpapers/hovey-model-cats.pdf\#page=52}{2.3.6}]{HoveyModelCategories}; the second statement also appears (in dual form) in \cite[\stackstag{070J}]{Stacks}.
	
	Let $K\mhyph\cat{Proj}(R)\subseteq\Ch(R)$ and $\cat{Proj}_{\geqslant 0}(R)\subseteq \Ch_{\geqslant 0}(R)$ be the full subcategories spanned by the $K$-projective complexes. Equip $K\mhyph\cat{Proj}(R)$ and $\cat{Proj}_{\geqslant 0}(R)$ with the Kan enrichment above. Then
	\begin{equation*}
		\Dd(R)\simeq \N^\Delta\bigl(K\mhyph\cat{Proj}(R)\bigr)\quad\text{and}\quad \Dd_{\geqslant 0}(R)\simeq \N^\Delta\bigl(\cat{Proj}_{\geqslant 0}(R)\bigr)\,.
	\end{equation*}
	The idea to prove this is, of course, similar to \cref{thm:AnAsALocalisation}: One can construct a simplicial model structure on $\Ch(R)$ (and, by restriction, on $\Ch_{\geqslant 0}(R)$) in such a way that $K\mhyph\cat{Proj}(R)\simeq \Ch(R)^\mathrm{cf}$ are precisely the bifibrant objects; see \cite[\S\href{https://people.math.rochester.edu/faculty/doug/otherpapers/hovey-model-cats.pdf\#page=50}{2.3}]{HoveyModelCategories}. Then the above equivalences follow from \cref{rem:SimplicialModelCategory,rem:ModelCategoryUnderlyingInftyCategory}.\footnote{It's worth pointing out that the process of choosing a cofibrant replacement in $\Ch_{\geqslant 0}(R)$ precisely recovers the method of \emph{projective resolutions} that you may be familiar with from homological algebra. Indeed, the cofibrant objects in $\Ch_{\geqslant 0}(R)$ are precisely the degree-wise projective complexes in non-negative degrees. Now if $M$ is a left $R$-module and we think of $M$ as a complex $M[0]_*$ concentrated in degree~$0$ (see \cref{con:Homology}, then a cofibrant replacement of $M[0]_*$, that is, a quasi-isomorphism $P_*\rightarrow M[0]_*$ from a degree-wise projective complex, is precisely a projective resolution of $M$. This begs the question how \emph{injective resolutions} fit into the picture. There is another simplicial model structure on $\Ch(R)$ in which the bifibrant objects are the \emph{$K$-injective} complexes, that is, those $I_*$ for which $\Hhom_R(-,I_*)$ preserves quasi-isomorphisms. A $K$-injective complex is degree-wise an injective $R$-module and conversely any degree-wise injective and bounded above complex is $K$-injective. One then has similar equivalences $\Dd(R)\simeq \N^\Delta(K\mhyph\cat{Inj}(R))$ and $\Dd_{\leqslant 0}(R)\simeq \N^\Delta(\cat{Inj}_{\leqslant 0}(R))$.}
	
	This alternative construction is useful to compute $\Hom_{\Dd(R)}$. Let $M_*$ and $N_*$ be complexes and let $P_*\rightarrow M_*$ and $Q_*\rightarrow N_*$ be quasi-isomorphisms from $K$-projective complexes. Using \cref{thm:CordierPorter}, we get $\Hom_{\Dd(R)}(M,N)\simeq \F_{\Ch(R)}(P_*,Q_*)$. In particular, since the Dold--Kan correspondence transforms homotopy groups of simplicial abelian groups into homology groups of the associated chain complexes by crash course~\cref{con:DerivedCategoryII}\cref{enum:DoldKanHomologyToHomotopy}, we find
	\begin{equation*}
		\pi_n\Hom_{\Dd(R)}(M,N)\cong \pi_n\F_{\Ch(R)}(P_*,Q_*)\cong \H_n\bigl(\Hhom_R(P_*,Q_*)\bigr)
	\end{equation*}
	for all $n\geqslant 0$ and all basepoints. Furthermore, $\H_n(\Hhom_R(P_*,Q_*))\cong \H_n(\Hhom_R(P_*,N_*))$ by definition of $P_*$ being $K$-projective, so we only need to resolve $M_*$ by a $K$-projective complex. If you've seen derived categories before, you'll probably have noticed that $\Hom_{\Dd(R)}(M,N)$ looks suspiciously like the derived Hom functor $\RHom_R(M,N)$: We'll see in \cref{cor:RHom} how exactly these two are related.\hfill$\blacksquare$
\end{numpar}

To be able to apply \cref{thm:PShFreeCocompletion} to $\Dd(R)$ or $\Dd_{\geqslant 0}(R)$, we need to show that these $\infty$-categories have all colimits. In order to to this, we'll show some general results about colimits in $\infty$-categories; these results will also be very useful later on.

\begin{lem}\label{lem:ColimitsIffCoproductsAndPushouts}
	An $\infty$-category $\Cc$ has all colimits if and only if $\Cc$ has pushouts and arbitrary coproducts. A functor $F\colon \Cc\rightarrow\Dd$ of $\infty$-categories preserves colimits if and only if it preserves pushouts and arbitrary coproducts. A dual assertion holds for limits.
\end{lem}
The crucial point, and the reason why we get away with \enquote{ordinary} colimits like pushouts and coproducts, is that $\{n\}\rightarrow \Delta^n$ is coinitial (in fact, right anodyne, so \cref{exm:Cofinal}\cref{enum:RightAnodyneCofinal} applies). Hence every functor $T\colon \Delta^n\rightarrow \Cc$ admits a colimit. For a general functor $T\colon\Ii\rightarrow \Cc$, we write $\Ii$ as a colimit of its skeleta to build $\colimit_{i\in\Ii}T(i)$ \enquote{simplex-by-simplex}: This needs pushouts (to attach $n$-simplices in the $n$\textsuperscript{th} step) and coproducts (to attach arbitrarily many $n$-simplices at the same time). 

To make this precise, we'll prove a lemma that will allow us to manipulate colimits: We can \enquote{slice a colimit into pieces} and \enquote{assemble colimits from subdiagrams}:
\begin{lem}\label{lem:ColimitManipulations}
	Let $\Ii$ and $\Cc$ be $\infty$-categories.
	\begin{alphanumerate}
		\item Suppose $p\colon \Uu\rightarrow \Ii$ is a cocartesian fibration and $T\colon \Uu\rightarrow \Cc$ is a functor such that $T|_{p^{-1}\{i\}}\colon p^{-1}\{i\}\rightarrow \Cc$ admits a colimit for all $i\in \Ii$. Then these colimits assemble into a functor $\ov T\colon \Ii\rightarrow \Cc$ satisfying $\ov T(i)\simeq \colimit_{u\in p^{-1}\{i\}}T(u)$. Furthermore,\label{claim:SliceColimits}
		\begin{equation*}
			\colimit_{u\in\Uu}T(u)\simeq \colimit_{i\in\Ii}\ov T(i)\,,
		\end{equation*}
		provided that at least one of these colimits exists in $\Cc$ \embrace{in which case the other exists as well}. Informally, we can rephrase this as $\colimit_{u\in\Uu}T(u)\simeq \colimit_{i\in\Ii}\colimit_{u\in p^{-1}\{i\}}T(u)$.
		\item Suppose $\Ii\simeq \colimit_{j\in\Jj}\Ii_j$ in $\Cat_\infty$. Let $T\colon \Ii\rightarrow \Cc$ be a functor such that the restrictions $T|_{\Ii_j}\colon \Ii_j\rightarrow \Cc$ admit colimits for all $j\in\Jj$. Then these colimits assemble into a functor $\ov T\colon \Jj\rightarrow \Cc$ satisfying $\ov T(j)\simeq \colimit_{i\in\Ii_j}T(i)$. Furthermore,
		\label{claim:AssembleColimits}
		\begin{equation*}
			\colimit_{i\in\Ii}T(i)\simeq \colimit_{j\in\Jj}\ov T(j)\,.
		\end{equation*}
		provided that at least one of these colimits exists in $\Cc$ \embrace{in which case the other exists as well}. Informally, we can rephrase this as $\colimit_{i\in\Ii}T(i)\simeq \colimit_{j\in\Jj}\colimit_{i\in\Ii_j}T(i)$.
	\end{alphanumerate}
	In particular, \enquote{colimits commute with colimits}: If $\Jj$ is an $\infty$-category and $T\colon \Ii\times\Jj\rightarrow \Cc$ is any functor, then
	\begin{equation*}
		\colimit_{i\in\Ii}\colimit_{j\in\Jj}T(i,j)\simeq\colimit_{(i,j)\in\Ii\times\Jj}T(i,j)\simeq \colimit_{j\in\Jj}\colimit_{i\in\Ii}T(i,j)\,.
	\end{equation*}
	%Dual assertions hold for limits \embrace{meaning that in \cref{claim:SliceColimits}, we consider a cartesian fibration, but in \cref{claim:AssembleColimits}, we still consider $\Ii\simeq\colimit_{j\in\Jj}\Ii_j$}.
\end{lem}
\begin{proof}
	To prove \cref{claim:SliceColimits}, first note that $\Lan_pT$ exists. Indeed, $p^{-1}\{i\}\rightarrow \Uu_{/i}$ is coinitial by the dual of \cref{lem:CartesianFibres} and \cref{exm:Cofinal}\cref{enum:RightAdjointCofinal}, so the existence of the colimits over $p^{-1}\{i\}$ implies that the condition from \cref{lem:KanExtensionFormula} is satisfied. So we can put $\ov T\coloneqq \Lan_pT$. Now $\colimit T$ corresponds to taking the left Kan extension of $T$ along $\Uu\rightarrow *$ (see \cref{exm:1ColimitAsKanExtension}). But we may as well first left Kan extend along $\Uu\rightarrow \Ii$ and then left Kan extend along $\Ii\rightarrow *$. This proves $\colimit T\simeq \colimit \Lan_pT$ and we've finished the proof of \cref{claim:SliceColimits}. In the special case where $p$ is the projection $\pr_1\colon \Ii\times\Jj\rightarrow \Jj$, we obtain the \enquote{in particular}.
	
	For \cref{claim:AssembleColimits}, let $p\colon \Uu\rightarrow \Jj$ be the unstraightening of the functor $\Jj\rightarrow \cat{Cat}_\infty$ of which $\Ii$ is the colimit. Then $\Ii$ is a localisation of $\Uu$ by \cref{lem:ColimitsInAnima}, so there's a natural functor $q\colon \Uu\rightarrow\Ii$. We have $p^{-1}\{j\}\simeq \Ii_j$, so we can apply \cref{claim:SliceColimits} to the functor $q\circ T\colon \Uu\rightarrow \Cc$. This allows us to construct $\ov T$ and we obtain $\colimit \ov T\simeq \colimit q\circ T$. But $q$, being a localisation, is coinitial by \cref{exm:Cofinal}\cref{enum:LocalisationsCofinal}, and so $\colimit q\circ T\simeq \colimit T$. This proves \cref{claim:AssembleColimits}.
\end{proof}
%\begin{rem}\label{rem:ColimitsCommute}
%	An important special case of \cref{lem:ColimitManipulations}\cref{claim:SliceColimits} is the case where $p$ is the projection $\pr_1\colon \Ii\times\Jj\rightarrow \Jj$: We obtain
%	\begin{equation*}
	%		\colimit_{i\in\Ii}\colimit_{j\in\Jj}T(i,j)\simeq\colimit_{(i,j)\in\Ii\times\Jj}T(i,j)\simeq \colimit_{j\in\Jj}\colimit_{i\in\Ii}T(i,j)
	%	\end{equation*}
%	for every functor $T(i,j)\rightarrow\Cc$. A dual assertion holds for limits. Informally, \enquote{colimits commute with colimits} and \enquote{limits commute with limits}.
%\end{rem}
\begin{proof}[Proof sketch of \cref{lem:ColimitsIffCoproductsAndPushouts}]
	In simplicial sets, we can write $\Ii\cong \colimit_{n\geqslant 0}\operatorname{sk}_n\Ii$, where $\operatorname{sk}_{n}\Ii$ is obtained from $\operatorname{sk}_{n-1}\Ii$ by attaching copies of $\Delta^n$; that is, we take a pushout along some coproduct of the form $\coprod \partial\Delta^n\rightarrow \coprod\Delta^n$. Up to replacing everything by quasi-categories (using \cref{lem:SmallObjectArgument}), we can thus write $\Ii\simeq \colimit_{n\geqslant 0}\Ii_n$ in $\cat{Cat}_\infty$ in such a way that $\Ii_n$ is obtained from $\Ii_{n-1}$ by a pushout along $\coprod \Bb^n\rightarrow \coprod \Delta^n$, where $\Bb^n$ is defined by choosing an inner anodyne map $\partial\Delta^n\rightarrow \Bb^n$ into a quasi-category. By an inductive argument (in which \cref{lem:ColimitManipulations}\cref{claim:AssembleColimits} powers the inductive step), we find that $\colimit_{i\in\Ii_n}T(i)$ exists in $\Cc$ for all $n\geqslant 0$. Using \cref{lem:ColimitManipulations}\cref{claim:AssembleColimits} once again, it remains to show that $\colimit_{n\geqslant 0}\colimit_{i\in\Ii_n}T(i)$ exists in $\Cc$. But this colimit can be easily written as a suitable pushout of the disjoint union $\coprod_{n\geqslant 0}\colimit_{i\in\Ii_n}T(i)$.
\end{proof}

To study colimits in derived $\infty$-categories, we introduce the following convenient terminology.
\begin{defi}\label{def:Cofibre}
	Let $\Cc$ be an $\infty$-category with a terminal object $*$ and let $\alpha\colon x\rightarrow y$ be a morphism in $\Cc$. The \emph{cofibre of $\alpha$} is defined as the pushout
	\begin{equation*}
		\begin{tikzcd}
			x\rar["\alpha"]\dar\drar[pushout] & y\dar\\
			*\rar & \cofib(\alpha)
		\end{tikzcd}
	\end{equation*}
	(provided this exists in $\Cc$). We say that $x\overset{\alpha}{\longrightarrow}y\rightarrow z$ is a \emph{cofibre sequence in $\Cc$} if the induced morphism $x\rightarrow z$ can be factored through $*$ in such a way that it exhibits $z$ as the cofibre of $\alpha$. There are dual notions of the \emph{fibre of $\alpha$} $\fib(\alpha)$  (given as the pullback against an initial object) and \emph{fibre sequences}.
\end{defi}
\begin{lem}\label{lem:ColimitsInDR}
	Let $R$ be any ring \embrace{not necessarily commutative}. The $\infty$-category $\Dd(R)$ has all colimits. The full sub-$\infty$-caegory $\Dd_{\geqslant 0}(R)\subseteq \Dd(R)$ is closed under colimits in $\Dd(R)$ and therefore has all colimits too. Coproducts and pushouts in $\Dd(R)$ can be described as follows:
	\begin{alphanumerate}
		\item If $I$ is any set and $M_{i,*}$ are chain complexes of left $R$-modules, then the chain complex $\bigoplus_{i\in I}M_{i,*}$ defines a coproduct of the objects $M_i\in \Dd(R)$.\label{enum:CoproductsInDR}
		\item Let $\alpha\colon M_*\rightarrow N_*$ be a morphism of chain complexes of $R$-modules for some ring $R$. Then the cofibre of $\alpha$ in $\Dd(R)$ can be computed as the mapping cone\footnote{The mapping cone $\cone(\alpha)_*$ is the chain complex given by $\cone(\alpha)_n\cong N_n\oplus M_{n-1}$, with differentials given by the matrix
		\begin{equation*}
			\begin{pmatrix}
				\partial_N & \alpha\\
				0 & -\partial_M
			\end{pmatrix}\colon N_n\oplus M_{n-1}\rightarrow N_{n-1}\oplus M_{n-2}\,;
		\end{equation*}
		here $\partial_M$ and $\partial_N$ denote the differentials of $M_*$ and $N_*$, respectively. The mapping cone comes equipped with obvious maps $N_*\rightarrow \cone(\alpha)_*$ and $\cone(\alpha)_*\rightarrow M[1]_*$. The induced maps $\H_n(N_*)\rightarrow \H_n(\cone(\alpha)_*)$ and $\H_n(\cone(\alpha)_*)\rightarrow \H_{n-1}(M_*)$ on homology fit into a long exact sequence
		\begin{equation*}
			\dotsb\longrightarrow\H_{n+1}\bigl(\cone(\alpha)_*\bigr)\overset{\partial}{\longrightarrow} \H_n(M_*)\longrightarrow \H_n(N_*)\longrightarrow \H_n\bigl(\cone(\alpha)_*\bigr)\overset{\partial}{\longrightarrow}\H_{n-1}(M_*)\longrightarrow \dotsb
		\end{equation*}
		called the \emph{cone sequence}. In fact, this is nothing but the long exact homology sequence associated to the short exact sequence of complexes $0\rightarrow N_*\rightarrow \cone(\alpha)_*\rightarrow M[1]_*\rightarrow 0$.}\label{enum:CofibresInDR}
		\begin{equation*}
			\cofib\left(\alpha\colon M\rightarrow N\right)\simeq \cone\left(\alpha\colon M_*\rightarrow N_*\right)\,.
		\end{equation*}
		More generally, if $\beta\colon M_*\rightarrow M'_*$ is another morphism of complexes, then the pushout of the span $N_*\leftarrow M_*\rightarrow M'_*$ in $\Dd(R)$ is given by $\cone((\alpha,-\beta)\colon M_*\rightarrow M'_*\oplus N_*)$.
	\end{alphanumerate}
\end{lem}
\begin{proof}[Proof sketch]
	By \cref{lem:ColimitsIffCoproductsAndPushouts}, to show that $\Dd(R)$ has all colimits, it's enough to check that coproducts and pushouts in $\Dd(R)$ can be described as in~\cref{enum:CoproductsInDR} and~\cref{enum:CofibresInDR}. Furthermore, it's clear from these descriptions that $\Dd_{\geqslant 0}(R)\subseteq \Dd(R)$ is closed under formation of coproducts and pushouts, hence under all colimits.
	
	To prove \cref{enum:CoproductsInDR}, choose quasi-isomorphisms $P_{i,*}\rightarrow M_{i,*}$ from $K$-projective complexes. Since quasi-isomorphisms are preserved under direct sums, it's enough to show that $\bigoplus_{i\in I}P_i$ is a coproduct of the $P_i$. Using \cref{cor:HomPreservesColimits}, we must show that 
	\begin{equation*}
		\Hom_{\Dd(R)}\biggl(\bigoplus_{i\in I}P_i,T\biggr)\overset{\simeq}{\longrightarrow}\prod_{i\in I}\Hom_{\Dd(R)}(P_i,T)
	\end{equation*}
	is an equivalence of animae for all $K\in\Dd(R)$. But if $T_*$ is any chain complex representing $T$, then $\Hhom_R\bigl(\bigoplus_{i\in I}M_{P,*},T_*\bigr)\cong \prod_{i\in I}\Hhom_R(P_{i,*},T_*)$ holds in $\Ch(R)$. Both the functor $\tau_{\geqslant 0}\colon \Ch(R)\rightarrow \Ch_{\geqslant 0}(R)$ and the Dold--Kan equivalence $\Ch_{\geqslant 0}(R)\simeq \cat{sAb}$ preserve products. Hence we get an isomorphism of Kan complexes (in fact, of simplicial abelian groups) $\F_{\Ch(R)}\bigl(\bigoplus_{i\in I} P_{i,*},T_*\bigr)\cong \prod_{i\in I}\F_{\Ch(R)}(P_{i,*},T_*)$. By crash course~\cref{con:DerivedCategoryIII}, this is (stronger than) what we need.
	
	The proof of \cref{enum:CofibresInDR} is similar, but needs a little more care. First, the assertion about pushouts is a formal consequence of the assertion about cofibres; we leave this to you (just verify the universal property). To show the assertion about cofibres, choose quasi-isomorphisms $P_*\rightarrow M_*$ and $Q_*\rightarrow N_*$ from $K$-projective complexes and replace $\alpha$ by a morphism $\alpha'\colon P_*\rightarrow Q_*$. Since the mapping cone construction preserves quasi-isomorphisms (by the long exact cone sequence and the five lemma), it suffices to show $\cofib(\alpha')\simeq \cone(\alpha')$. To this end, first note that the composition $P_*\rightarrow Q_*\rightarrow \cone(\alpha')_*$ is nullhomotopic as a map of complexes. Any choice of nullhomotopy defines a morphism $\cofib(\alpha')\rightarrow \cone(\alpha')$ in $\Dd(R)$. To see that this morphism is an equivalence, we can appeal again to \cref{cor:HomPreservesColimits} and the Yoneda lemma: It's enough to show that
	\begin{equation*}
		\Hom_{\Dd(R)}\bigl(\cone(\alpha'),T\bigr)\longrightarrow\Hom_{\Dd(R)}(Q,T)\longrightarrow\Hom_{\Dd(R)}(P,T)
	\end{equation*}
	is a fibre sequence of animae for all $T\in\Dd(R)$. Now we claim:
	\begin{alphanumerate}\itshape
		\item[\boxtimes] Let $\varphi\colon K_*\rightarrow L_*$ be any morphism of chain complexes and consider the canonical sequence\label{enum:DoldKanFibreSequence}
		\begin{equation*}
			\cone(\varphi)[-1]_*\longrightarrow K_*\overset{\varphi}{\longrightarrow} L_*
		\end{equation*}
		in $\Ch(R)$. Upon applying $\tau_{\geqslant 0}\colon \Ch(R)\rightarrow \Ch_{\geqslant 0}(R)$ and the Dold--Kan correspondence, this sequence is sent to a fibre sequence of animae.
	\end{alphanumerate}
	Once we know \cref{enum:DoldKanFibreSequence}, we're done. Indeed, it's straightforward to check that the sequence $\Hhom_{\Dd(R)}(\cone(\alpha')_*,T_*)\rightarrow\Hhom_{R}(Q_*,T_*)\rightarrow\Hhom_{R}(P_*,T_*)$ is of the desired form. Then \cref{enum:DoldKanFibreSequence} ensures that $\F_{\Ch(R)}(\cone(\alpha')_*,T_*)\rightarrow\F_{\Ch(R)}(Q_*,T_*)\rightarrow\F_{\Ch(R)}(P_*,T_*)$ is a fibre sequence; by crash course~\cref{con:DerivedCategoryIII}, this is what we need.
	
	To prove \cref{enum:DoldKanFibreSequence}, we can restrict ourselves to the case where $\varphi\colon K_*\rightarrow L_*$ is degree-wise surjective. Indeed, we can always replace $K_*$ by $K_*\oplus \cone(\id_{L_*}\colon L_*\rightarrow L_*)[-1]_*$; this doesn't change anything since $\cone(\id_{L_*})_*\rightarrow 0$ is a quasi-isomorphism, $\tau_{\geqslant 0}$ preserves quasi-isomorphisms, and Dold--Kan sends quasi-isomorphisms to homotopy equivalences of Kan complexes (because it sends $\H_n$ to $\pi_n$ by crash course~\cref{con:DerivedCategoryII}\cref{enum:DoldKanHomologyToHomotopy}). If $\varphi\colon K_*\rightarrow L_*$ is surjective, a well-known fact from homological algebra states that there is a quasi-isomorphism $\ker(\varphi)_*\simeq \cone(\varphi)[-1]_*$.\footnote{To see this, the first step is to construct a morphism $\ker(\varphi)_*\rightarrow \cone(\varphi)[-1]_*$: This is straightforward from the construction of $\cone(\varphi)[-1]_*$. Alternatively, we can invoke a universal property: It can be shown that maps $T_*\rightarrow \cone(\varphi)[-1]_*$ are in bijection with pairs $(\alpha,\eta)$, where $\alpha\colon T_*\rightarrow K_*$ is a morphism and $\eta$ is a nullhomotopy of the composition $\varphi\circ\alpha\colon T_*\rightarrow L_*$. Since $\ker(\varphi)_*\rightarrow L_*$ is zero on the nose, we can just choose the trivial nullhomotopy. To show that the map $\ker(\varphi)_*\rightarrow \cone(\varphi)[-1]_*$ is a quasi-isomorphism, it's straightforward to check that this map induces a morphism between the long exact homology sequence associated to $0\rightarrow \ker(\varphi)_*\rightarrow K_*\rightarrow L_*\rightarrow 0$ and the cone sequence for $\cone(\varphi)_*$. Then the five lemma does the rest.} So it's enough to show that $\ker(\varphi)_*\rightarrow K_*\rightarrow L_*$ is sent to a fibre sequence.
	
	Since $\tau_{\geqslant 0}\colon \Ch(R)\rightarrow\Ch_{\geqslant 0}(R)$ is a right adjoint, we see that $\tau_{\geqslant 0}\ker(\varphi)_*$ is still the kernel of $\tau_{\geqslant 0}\varphi\colon \tau_{\geqslant 0}K_*\rightarrow \tau_{\geqslant 0}L_*$, but that map might not be surjective anymore in degree~$0$. So consider $\im(\tau_{\geqslant 0}\varphi)_*\rightarrow \tau_{\geqslant 0}L_*$.  Under the Dold--Kan correspondence, this map is sent to the inclusion of a collection of path components. Indeed, $\im(\tau_{\geqslant 0}\varphi)_*\rightarrow \tau_{\geqslant 0}L_*$ is an isomorphism on $\H_n$ for all $n\geqslant 1$ and injective on $\H_0$, so after Dold--Kan, we obtain an isomorphism on $\pi_n$ for all $n\geqslant 1$ and an injection on $\pi_0$. Therefore, it's enough to show that $\tau_{\geqslant 0}\ker(\varphi)_*\rightarrow \tau_{\geqslant 0}K_*\rightarrow \im(\tau_{\geqslant 0}\varphi)_*$ is sent to a fibre sequence. But that's a short exact sequence in $\Ch_{\geqslant 0}(R)$, and so it's sent to a short exact sequence in $\cat{sAb}$. As mentioned in crash course~\cref{con:DerivedCategoryII}\cref{enum:SimplicialAbelianGroupKanComplex}, surjections of simplicial abelian groups are Kan fibrations. By model category fact~\cref{par:HomotopyPushout}, this means that the kernel, that is, the fibre over $0$ taken in simplicial sets, agrees with the homotopy fibre. So we're done.
\end{proof}
With all the preparatory stuff about $\Dd(R)$ out of the way, we can now finally get to the actual subject of \cref{subsec:EilenbergMacLane}: Homology, cohomology, and Eilenberg--MacLane animae.
\begin{con}\label{con:Homology}
	For all integers $n$ and all abelian groups $A$ define a chain complex
	\begin{equation*}
		A[n]_*\coloneqq \left(\dotsb\rightarrow 0\rightarrow 0\rightarrow A\rightarrow 0\rightarrow 0\rightarrow \dotsb\right)
	\end{equation*}
	with $A$ in degree $n$ and $0$ everywhere else. Consider the functor $\cat{Ab}\rightarrow\Ch_{\geqslant 0}(\IZ)\rightarrow\Dd_{\geqslant 0}(\IZ)$ that sends $A$ to $A[0]$. Since $\Dd_{\geqslant 0}(\IZ)$ has all colimits, so has $\Fun(\cat{Ab},\Dd_{\geqslant 0}(\IZ))$ by \cref{lem:ColimitsInFunctorCategories}. Hence, by \cref{thm:PShFreeCocompletion}, there exists a unique colimit-preserving functor $\cat{An}\rightarrow\Fun(\cat{Ab},\Dd_{\geqslant 0}(\IZ))$ that sends $*\in\cat{An}$ to the functor $\cat{Ab}\rightarrow\Ch_{\geqslant 0}(\IZ)\rightarrow\Dd_{\geqslant 0}(\IZ)$ discussed above. By \enquote{currying}, we obtain a functor
	\begin{equation*}
		\C(-,-)\colon \cat{An}\times\cat{Ab}\longrightarrow \Dd_{\geqslant 0}(\IZ)\,.
	\end{equation*} 
	For every abelian group $A$, $\C(-,A)\colon \cat{An}\rightarrow\Dd_{\geqslant 0}(\IZ)$ is the unique colimit-preserving functor that sends $*\in\cat{An}$ to $A[0]$ as above. For an anima $X$, we call $\C(X,A)$ the \emph{chains of $X$ with coefficients in $A$} and we call $\widetilde{\C}(X,A)\coloneqq \fib(\C(X,A)\rightarrow\C(*,A))$ the \emph{reduced chains of $X$ with coefficients in $A$} (using the fibre construction from \cref{def:Cofibre}). For all $n\geqslant 0$, we let
	\begin{equation*}
		\H_n(X,A)\coloneqq \H_n\bigl(\C(X,A)\bigr)\,,\quad \widetilde{\H}_n(X,A)\coloneqq\H_n\bigl(\widetilde{\C}(X,A)\bigr)
	\end{equation*}
	denote the \emph{$n$\textsuperscript{th} homology of $X$ with coefficients in $A$} and the \emph{$n$\textsuperscript{th} reduced homology of $X$ with coefficients in $A$}; here $\H_n\colon \Dd_{\geqslant 0}(\IZ)\rightarrow\cat{Ab}$ is the functor from crash course~\cref{con:DerivedCategoryI}. Finally,
	\begin{equation*}
		\H^n(X,A)\coloneqq \pi_0\Hom_{\Dd_{\geqslant 0}(\IZ)}\bigl(\C(X,\IZ),A[n]\bigr)\,,\quad\widetilde{\H}^n(X,A)\coloneqq \pi_0\Hom_{\Dd_{\geqslant 0}(\IZ)}\bigl(\widetilde{\C}(X,\IZ),A[n]\bigr)
	\end{equation*}
	denote the \emph{$n$\textsuperscript{th} cohomology of $X$ with coefficients in $A$} and the \emph{$n$\textsuperscript{th} reduced cohomology of $X$ with coefficients in $A$}. We'll verify in \cref{lem:Homology} below that this definition of homology and cohomology is compatible with the one you are familiar with.
	
	But before we do that, let's give yet another unfamiliar formulation of a familiar definition! By \cref{thm:PShFreeCocompletion}, $\C(-,\IZ)\colon \cat{An}\rightarrow \Dd_{\geqslant 0}(\IZ)$ automatically acquires a right adjoint, which we denote $\K\colon \Dd_{\geqslant 0}(\IZ)\rightarrow\cat{An}$. For $M\in\Dd_{\geqslant 0}(\IZ)$ we call $\K(M)$ the \emph{generalised Eilenberg--MacLane anima of $M$}. In the special case $M\simeq A[n]$ we say that $\K(A,n)\coloneqq \K(A[n])$ is the \emph{Eilenberg--MacLane anima of type $(A,n)$}. Again, we'll justify in \cref{thm:EilenbergMacLane} below that this recovers the definition you're familiar with.
\end{con}
\begin{lem}\label{lem:Homology}
	Let $A$ be an arbitrary abelian group.
	\begin{alphanumerate}
		\item The functors $\H_*(-,A)$ and $\widetilde{\H}_*(-,A)$ from \cref{con:Homology} satisfy the Eilenberg--Steenrod axioms. In particular, they are homotopy invariants; if $f\colon Y\rightarrow X$ is a morphism of animae with cofibre $X/Y\coloneqq\cofib(f)$, then there is a long exact sequence\label{enum:HomologyEilenbergSteenrod}
		\begin{equation*}
			\dotsb\longrightarrow\H_n(Y,A)\longrightarrow\H_n(X,A)\longrightarrow\widetilde{\H}_n(X/Y,A)\overset{\partial}{\longrightarrow}\H_{n-1}(Y,A)\longrightarrow\dotsb
		\end{equation*}
		\embrace{so $\widetilde{\H}_*(X/Y,A)$ plays the role of the relative homology $\H_*(X,Y,A)$}; and $\H_*(-,A)$ sends disjoint unions to direct sums. Similar assertions hold for $\widetilde{\H}_*(-,A)$. Furthermore, the suspension isomorphism is satisfied and pushouts of animae yield long exact Mayer--Vietoris sequences.
		\item Let $X$ be a Kan complex with geometric realisation $\abs*{X}\in\cat{Top}$. Let $\C_*^\mathrm{sing}(\abs*{X},A)$ denote the singular chain complex of $\abs*{X}$ with coefficients in $A$. Then there is a natural quasi-somorphism
		\begin{equation*}
			\C(X,A)\overset{\simeq}{\longrightarrow} \C^\mathrm{sing}_*\bigl(\abs{X},\IZ\bigr)\,.
		\end{equation*}
		In particular, we get $\H_*(X,A)\cong \H_*^\mathrm{sing}(\abs*{X},A)$, as well as similar isomorphisms for reduced homology and for cohomology \embrace{both unreduced and reduced}.\label{enum:SingularHomology}
	\end{alphanumerate}
\end{lem}
\begin{proof}[Proof sketch]
	We begin with \cref{enum:HomologyEilenbergSteenrod}. Homotopy invariance follows from the definition of $\cat{An}$. The other Eilenberg--Steenrod axioms all follow from the fact that $\C(-,A)\colon \cat{An}\rightarrow\Dd_{\geqslant 0}(\IZ)$ preserves colimits. To demonstrate these kinds of arguments, we'll show the long exact sequence for $\H_*(-,A)$; the disjoint union axiom as well as the Eilenberg--Steenrod axioms for $\widetilde{\H}_*(-,A)$ will be left to you. We start with the following diagram:
	\begin{equation*}
		\begin{tikzcd}
			\C(Y,A)\rar\dar\drar[pushout] & \C(*,A)\rar\dar\drar[pushout] & 0\dar\\
			\C(X,A)\rar & \C(X/Y,A)\rar & \widetilde{\C}(X/Y,A)
		\end{tikzcd}
	\end{equation*}
	The left square is a pushout since $\C(-,A)$ preserves pushouts. To obtain the pushout square on the right, observe that $X/Y$ is canonically a pointed anima via $*\simeq Y/Y\rightarrow X/Y$. In general, for any pointed anima $(Z,z)\in\cat{An}_{*/}$, the canonical morphism $Z\rightarrow*$ has a preferred section given by $\{z\}\rightarrow Z$. Thus $\C(Z,\IZ)\rightarrow\C(*,\IZ)$ has a section and we obtain
	\begin{equation*}
		\C(Z,\IZ)\simeq \widetilde{\C}(Z,\IZ)\oplus \C\bigl(\{z\},\IZ\bigr)\,.
	\end{equation*}
	Hence the reduced chains $\widetilde{\C}(Z,\IZ)$ from \cref{con:Homology} can also be written as the cofibre $\widetilde{\C}(Z,\IZ)\simeq\cofib(\C(\{z\},\IZ)\rightarrow \C(Z,\IZ))$, functorially in $(Z,z)$. This explains the right square in the diagram above. Hence $\C(Y,A)\rightarrow\C(X,A)\rightarrow\widetilde{\C}(X/Y,A)$ is a cofibre sequence in $\Dd_{\geqslant 0}(\IZ)$ and thus in $\Dd(\IZ)$. By \cref{lem:ColimitsInDR}\cref{enum:CofibresInDR}, cofibre sequences in $\Dd(\IZ)$ can be represented by cone sequences, so the desired long exact sequence is simply the cone sequence.
	
	The suspension isomorphism and the Mayer--Vietoris sequence are formal consequences of the Eilenberg--Steenrod axioms. Alternatively, they can be checked by hand. For example, for Mayer--Vietoris, use that $\C(-,A)$ preserves pushouts and then apply the characterisation of pushouts in $\Dd(\IZ)$ from \cref{lem:ColimitsInDR}\cref{enum:CofibresInDR}.
	
	For~\cref{enum:SingularHomology}, let's first describe how to get a functor $\C^{\mathrm{sing}}(\abs*{\,-\,},A)\colon\cat{An}\rightarrow\Dd_{\geqslant 0}(\IZ)$. By \cref{thm:AnAsALocalisation} and \cref{lem:Localisation}, it's enough to check that $\C_*^\mathrm{sing}(\abs{\,-\,},A)\colon \cat{Kan}\rightarrow \Ch_{\geqslant 0}(\IZ)$ sends homotopy equivalences to quasi-isomorphisms, which is obviously true. So we get our desired functor. We know from \cref{lem:PresheafColimitOfRepresentables} and the Kan extension formula that $\C(-,A)\colon \cat{An}\rightarrow \Dd_{\geqslant 0}(\IZ)$ is the left Kan extension of its restriction to $\{*\}\subseteq \cat{An}$. By the universal property of Kan extensions, the equivalence $\C(*,A)\simeq A[0]\simeq \C^\mathrm{sing}(\abs*{*},A)$ extends to a natural transformation
	\begin{equation*}
		\C(-,A)\Longrightarrow \C^\mathrm{sing}\bigl(\abs*{\,-\,},A\bigr)\,.
	\end{equation*}
	Whether this is an equivalence can be checked pointwise and on homology groups. Now we can use the well-known fact that any unreduced homology theory $h_*$ with $h_0(*)\cong A$ and $h_n(*)\cong 0$ for $n\geqslant 1$ must necessarily coincide with singular homology with coefficients in $A$. See \cite[Theorem~\href{https://pi.math.cornell.edu/~hatcher/AT/AT.pdf\#page=408}{4.59}]{Hatcher}. Alternatively, we can directly show that $\C^{\mathrm{sing}}(\abs*{\,-\,},A)$ preserves colimits. Using \cref{lem:ColimitsInDR}, this is straightforward (preservation of coproducts is trivial and preservation of pushouts is, essentially, the Mayer--Vietoris sequence).
	
	The isomorphism $\H_*(X,A)\cong \H_*^\mathrm{sing}(\abs*{X},A)$ is an immediate consequence, as is its variant for reduced homology. For cohomology, we can argue as follows: We already know that $\C(X,\IZ)\simeq \C^\mathrm{sing}(\abs*{X},\IZ)$. The chain complex $\C_*^\mathrm{sing}(\abs{X},\IZ)$ is $K$-projective because it is degree-wise free over $\IZ$ and bounded below. Using the computation from crash course~\cref{con:DerivedCategoryIII}, we deduce
	\begin{equation*}
		\pi_0\Hom_{\Dd_{\geqslant 0}(\IZ)}\bigl(\C^\mathrm{sing}\bigl(\abs{X},\IZ\bigr),A[n]\bigr)\cong \H_0\Hhom_\IZ\bigl(\C_*^\mathrm{sing}\bigl(\abs{X},\IZ\bigr),A[n]\bigr)
	\end{equation*}
	Now $\Hhom_\IZ(\C_*^\mathrm{sing}(\abs{X},\IZ),A)\cong \C_\mathrm{sing}^{-*}(\abs{X},A)$ is the cochain complex that computes singular cohomology of $\abs{X}$, placed in negative degrees (so that it becomes a chain complex). Taking the shifts into account, we get $\H^n(X,A)\cong \H_\mathrm{sing}^n(\abs{X},A)$, as claimed. The same argument also shows the assertion about reduced cohomology.
\end{proof}
\begin{rem}
	Let's take a moment to appreciate the beauty of \cref{lem:Homology}\cref{enum:HomologyEilenbergSteenrod}. With our definition of homology, the Eilenberg--Steenrod axioms and all the usual properties of homology are completely formal. The only input we need is of algebraic nature: namely, the description of colimits in $\Dd(\IZ)$ from \cref{lem:ColimitsInDR}. Now compare that to the classical construction of singular homology: To prove the Mayer--Vietoris sequence (or equivalently excision), one has to take sufficiently fine barycentric subdivisions, apply Lebesgue's covering theorem, and construct a bunch of chain homotopies by hand (see \cite[Proposition~\href{https://pi.math.cornell.edu/~hatcher/AT/AT.pdf\#page=128}{2.21}]{Hatcher} for such an argument). Blargh! I find our approach much more enlightening and much less technical.\footnote{The careful traditionalist will---rightfully---object that our theory doesn't really avoid barycentric subdivision, it just moves it to the proof of \cref{thm:SimplicialApproximation}, conveniently hidden in a black box. Sure, but that doesn't undermine my point. Barycentric subdivision is a technical tool to compare animae to CW complexes---that's its natural place in the theory. Beyond that, it can be avoided.} In fact, I'd argue that \cref{con:Homology} is the better definition of homology (and an even better one is \cref{cor:Homology})!
	
	Although our definition is very abstract, for a given anima $X$, it's often easy to write down an explicit complex that computes $\C(X,A)$. For example, assume we're given a \enquote{CW decomposition} of $X$; that is, a way to write $X$ as a sequence of pushouts along $S^n\rightarrow *$ (the $n$-disk is contractible, so we may as well use $*$ instead). Then, $\C(X,A)$ can be written as a similar sequence of pushouts in $\Dd_{\geqslant 0}(\IZ)$. Since we understand $\C(S^n,A)\simeq A[0]\oplus A[n]$ as well as $\C(*,A)\simeq A[0]$ and since we know how to compute pushouts in $\Dd_{\geqslant 0}(\IZ)$ by \cref{lem:ColimitsInDR}\cref{enum:CofibresInDR}, we can compute $\C(X,A)$. If you think about this, the complex we end up with is precisely the \emph{cellular complex of $X$}, so we've just proved that homology agrees with cellular homology. Combining this with the classical fact that cellular and singular homology agree, we get an alternative proof of \cref{lem:Homology}\cref{enum:SingularHomology}.
\end{rem}
We finish this subsection by proving the classical Eilenberg--MacLane theorem. As we'll see, once again, the proof is entirely formal.
\begin{thm}[\enquote{Eilenberg--MacLane animae represent cohomology}]\label{thm:EilenbergMacLane}
	For every abelian group $A$ and all $n\geqslant 0$, the Eilenberg--MacLane anima $\K(A,n)$ from \cref{con:Homology} satisfies $\pi_n\K(A,n)\cong A$ and $\pi_i\K(A,n)\cong0$ for $i\neq n$. This condition determines $\K(A,n)$ uniquely up to homotopy equivalence. Furthermore, $\K(A,n)$ represents cohomology with coefficients in $A$ \embrace{both unreduced and reduced} in the sense that the functors $\H^n(-,A)\colon \cat{An}\rightarrow \cat{Ab}$ and $\widetilde{\H}^n(-,A)\colon \cat{An}_{*/}\rightarrow \cat{Ab}$ are given by
	\begin{equation*}
		\bigl[-,\K(A,n)\bigr]\coloneqq\pi_0\Hom_{\cat{An}}\bigl(-,\K(A,n)\bigr)\quad\text{and}\quad \bigl[-,\K(A,n)\bigr]_*\coloneqq\pi_0\Hom_{\cat{An}_{*/}}\bigl(-,\K(A,n)\bigr)\,,
	\end{equation*}
	respectively.
\end{thm}
\begin{proof}
	For every anima $X$ and every $M_*\in\Ch_{\geqslant 0}(\IZ)$, the adjunction $\C(-,\IZ)\colon \cat{An}\shortdoublelrmorphism\Dd_{\geqslant 0}(\IZ)\noloc \K$ from \cref{con:Homology} shows
	\begin{equation*}
		\Hom_{\cat{An}}\bigl(X,\K(M)\bigr)\cong \Hom_{\Dd_{\geqslant 0}(\IZ)}\bigl(\C(X,\IZ),M\bigr)\,.
	\end{equation*}
	In the case $M_*\simeq A[n]_*$, we immediately obtain $\pi_0\Hom_{\cat{An}}(X,\K(A,n))\simeq \H^n(X,A)$. This shows that $[-,\K(A,n)]\cong\H^n(-,A)$. The assertion about reduced cohomology follows analogously if we can show that the adjunction $\C(-,\IZ)\colon \cat{An}\shortdoublelrmorphism \Dd_{\geqslant 0}(\IZ)\noloc{\K}$ lifts to an adjunction
	\begin{equation*}
		\widetilde{\C}(-,\IZ)\colon \cat{An}_{*/}\doublelrmorphism \Dd_{\geqslant 0}(\IZ)\noloc {\K}\,.
	\end{equation*}
	In \cref{lem:SliceAdjunction} below, we'll show a general fact about passing adjunctions to slice $\infty$-categories. Let's explain how this applies in our situation: Since $\K$ is a right adjoint, it preserves terminal objects, whence $\K(0)\simeq *$. But $0$ is also an initial object in $\Dd_{\geqslant 0}(\IZ)$. Hence $\Dd_{\geqslant 0}(\IZ)\simeq \Dd_{\geqslant 0}(\IZ)_{0/}$ and so the induced functor $\K\colon \Dd_{\geqslant 0}(\IZ)\simeq \Dd_{\geqslant 0}(\IZ)_{0/}\rightarrow \cat{An}_{\K(0)/}\simeq \cat{An}_{*/}$ on slice $\infty$-categories is indeed of the form studied in \cref{lem:SliceAdjunction}. We've seen in the proof of \cref{lem:Homology}\cref{enum:HomologyEilenbergSteenrod} that for every pointed anima $(X,x)$ one has $\widetilde{\C}(X,\IZ)\simeq \cofib(\C(\{x\},\IZ)\rightarrow \C(X,\IZ))$, so $\widetilde{\C}(-,\IZ)$ agrees with the left adjoint constructed in \cref{lem:SliceAdjunction}.
	
	To compute the homotopy groups of $\K(A,n)$, let $S^i\in\cat{An}$ be the \emph{$i$-sphere}.\footnote{There are many possible constructions for $S^i$. The most conceptual way would be to define $S^i\coloneqq \Sigma^i(*\ \,*)$ as the $i$-fold suspension of two points, using the upcoming definition \cref{def:Loop}. But there are also many possible simplicial models. For example, if $\partial D^{i+1}\subseteq D^{i+1}$ is the boundary of the topological $(i+1)$-disk, we could take $\Sing \partial D^{i+1}$ as our model for $S^i$. Alternatively, we could choose anodyne maps from $\square^i/\partial\square^i$ or $\Delta^i/\partial\Delta^i$ or $\partial\Delta^{i+1}$ into Kan complexes. All constructions you could possibly come up with will be homotopy equivalent, so you can just choose your favourite option.} Plugging in $(S^i,*)$ for any choice of basepoint yields
	\begin{equation*}
		\pi_i\K(A,n)\cong \pi_0\Hom_{\cat{An}_{*/}}\bigl((S^i,*),\K(A,n)\bigr)\cong \widetilde{\H}^n(S^i,A)\,.
	\end{equation*}
	By \cref{lem:Homology}, one has $\widetilde{\H}^n(S^n,A)\cong A$ and $\widetilde{\H}^n(S^i,A)\cong 0$ for $i\neq n$, and the desired description of $\pi_*\K(A,n)$ follows. By the usual argument from topology, $\K(A,n)$ is uniquely determined by this property up to homotopy equivalence.
\end{proof}
The following lemma was used in the proof:
\begin{lem}\label{lem:SliceAdjunction}
	Let $L\colon \Cc\shortdoublelrmorphism \Dd\noloc R$ be an adjunction of $\infty$-categories and let $y\in \Dd$. If for every morphism $R(y)\rightarrow x$ in $\Cc$ the pushout
	\begin{equation*}
		\begin{tikzcd}
			LR(y)\rar\dar["c_y"']\drar[pushout] & L(x)\dar\\
			y\rar & y\sqcup_{LR(y)}L(x)
		\end{tikzcd}
	\end{equation*}
	exists in $\Dd$, then the functor $R\colon \Dd_{y/}\rightarrow \Cc_{R(y)/}$ on slice $\infty$-categories still has a left adjoint $L_y\colon \Cc_{R(y)/}\rightarrow \Dd_{y/}$. On objects, $L_y$ is given by $L_y(R(y)\rightarrow x)\simeq (y\rightarrow y\sqcup_{LR(y)}L(x))$, constructed via the pushout square above. Moreover, the pushout square can be made functorial in an obvious way and this recovers $L_y$ as a functor \embrace{not only pointwise}.
\end{lem}
\begin{proof}[Proof sketch]
	You can directly verify $\Hom_{\Dd_{y/}}(L_y(R(y)\rightarrow x),-)\simeq \Hom_{\Cc_{R(y)/}}(R(y)\rightarrow x,R(-))$. To do so, plug in \cref{cor:HomPreservesColimits}, \cref{cor:HomInSliceCategories}, and the given adjunction $L\dashv R$, then perform a formal manipulation of pullbacks. Since adjoints can be constructed pointwise (\cref{lem:Adjunction}), this proves the existence of $L_y$. With a little more care, one can make the pullback manipulation functorial in $R(y)\rightarrow x$ as well, and then the claimed description of $L_y$ follows.
\end{proof}

\newpage

\sectionappendix{Presentable \texorpdfstring{$\infty$}{Infinity}-categories}
%\addcontentsline{toc}{sectionappendix}{Appendix to \texorpdfstring{\cref{sec:InftyCategoryTheory}}{\S6}. Presentable \texorpdfstring{$\infty$}{Infinity}-categories}
%\sectionmark{Toast}

Suppose $\Cc$ is an $\infty$-category with all colimits and let $F\colon \Cc\rightarrow\Dd$ be a colimit-preserving functor of $\infty$-categories. Then the only thing preventing $F$ from having a right adjoint is set theory. Indeed, the values of a hypothetical right adjoint $G\colon \Dd\rightarrow\Cc$ would be given by $G(y)\simeq \colimit(\Cc_{/y}\rightarrow \Cc)$ for all $y\in\Dd$ (as we'll see in the proof of \cref{thm:AdjointFunctorTheorem}\cref{enum:AdjointFunctorTheoremLeft}), except that this colimit usually doesn't exist, even though $\Cc$ has all colimits. The problem ist that $\Cc_{/y}$ is usually not an \emph{essentially small} $\infty$-category in the sense of \cref{def:KappaSmall}\cref{enum:Small} below. So far, we have ignored these smallness issues. Still, \crefrange{subsec:Adjunctions}{subsec:KanExtensions} can be made set-theoretically sound. As a rule of thumb, whenever a limit or colimit is considered, the indexing $\infty$-category should be assumed small (or at least admit a coinitial/initial functor from an essentially small $\infty$-category) and whenever we consider $\PSh(\Cc)$, we should assume that $\Cc$ is essentially small. The only time this gets hairy is in the proof of \cref{lem:KanExtensionFormula}, where we should allow $\Dd$ to be large, but also consider $\PSh(\Dd)$. Nevertheless, this can be fixed too.\footnote{For example, by using universes, but Fabian proposed a trick to get away with ZFC: Instead of $\PSh(\Dd)$, consider the $\infty$-category of right fibrations $\Uu\rightarrow \Dd$, for which $\Uu$ admits a coinitial functor from an essentially small $\infty$-category. This $\infty$-category contains $\Dd_{/y}\rightarrow\Dd$ for all $y\in \Dd$, so all Yoneda arguments go through.}

However, a more thorough analysis is needed to save our adjoint functor argument. In fact, $\infty$-categories $\Cc$ with all colimits are very seldomly essentially small, and so neither is $\Cc_{/y}$. However, often there exists an essentially small sub-$\infty$-category $\Cc_0\subseteq\Cc$ that generates $\Cc$ under colimits, and in this case one can replace $\Cc_{/y}$ by a coinitial essentially small sub-$\infty$-category, so that the required colimits do exist. The theory of accessible and presentable $\infty$-categories makes these ideas precise and allows to prove an incredibly powerful \emph{adjoint functor theorem}.

In \crefrange{subsec:EssentiallySmall}{subsec:AdjointFunctorTheorem}, we'll give the necessary definitions and prove Lurie's adjoint functor theorem (\cref{thm:AdjointFunctorTheorem}). After that, we'll discuss some supplements in \cref{subsec:PrL}. Naturally, this means that \crefrange{subsec:EssentiallySmall}{subsec:PrL} will be very technical. If you're mainly interested in spectra and willing to take the adjoint functor theorem on faith, you can safely skip ahead to \cref{sec:TowardsSpectra} at this point. If instead you're looking for a much more detailed exposition, you should consult \cite[\S\href{https://people.math.harvard.edu/~lurie/papers/HTT.pdf\#page=332}{5}]{HTT}.


\subsection{Essentially small and locally small \texorpdfstring{$\infty$}{Infinity}-categories}\label{subsec:EssentiallySmall}

First we'll explain how to put cardinality bounds on $\infty$-categories.
\begin{defi}\label{def:KappaSmall}
	Let $\kappa$ be a regular cardinal and let $\Cc$ be an $\infty$-category.
	\begin{alphanumerate}
		\item If $\kappa=\aleph_0$, then $\Cc$ is called \emph{essentially $\aleph_0$-small} if it is contained in the full sub-$\infty$-category of $\cat{Cat}_\infty$ generated under pushouts by $\emptyset$ and $\Delta^n$ for all $n\geqslant 0$. If $\kappa$ is uncountable, then $\Cc$ is called \emph{essentially $\kappa$-small} if $\pi_0\core \Cc$ as well as $\pi_0\Hom_\Cc(x,y)$ and $\pi_n(\Hom_\Cc(x,y),\alpha)$ are sets of cardinality $<\kappa$ for all $x,y\in\Cc$, all $\alpha\colon x\rightarrow y$, and all $n\geqslant 1$.\label{enum:KappaSmallCategory}%$\pi_0\Hom_{\cat{Cat}_\infty}(\Delta^n,\Cc)$ has cardinality $<\kappa$ for all $n\geqslant 0$. \label{enum:KappaSmallCategory}
		\item $\Cc$ is called \emph{essentially small} if it is essentially $\kappa$-small for some regular cardinal $\kappa$, and \emph{large} otherwise. $\Cc$ is called \emph{locally small} if $\Hom_\Cc(x,y)$ is essentially small for all $x,y\in\Cc$.\label{enum:Small}
		\item A colimit or a limit over a functor $F\colon \Ii\rightarrow \Cc$ is called \emph{$\kappa$-small} if $\Ii$ is essentially $\kappa$-small. Instead of \emph{$\aleph_0$-small}, we often say that a limit or colimit is \emph{finite}.\label{enum:KappaSmallLimit}
	\end{alphanumerate}
\end{defi}
\begin{rem}\label{rem:FunLocallySmall}
	If $\Cc$ is a small $\infty$-category and $\Dd$ is locally small, then $\Fun(\Cc,\Dd)$ is again locally small, as can be seen by \cref{cor:HomInFunctorCats}. In particular, $\PSh(\Cc)$ and the $\infty$-categories $\cat{Ind}_\kappa(\Cc)$ from \cref{con:Ind} below will be locally small.
\end{rem}
For practical applications, it will, unfortunately, be necessary to translate our nice model independent \cref{def:KappaSmall}\cref{enum:KappaSmallCategory} into the language of simplicial sets.
%\cref{def:KappaSmall}\cref{enum:KappaSmallCategory} is rather cumbersome to work with. We chose it because we're striving for model-independent notions, but 
\begin{lem}\label{lem:KappaSmall}
	Let $\kappa$ be an uncountable regular cardinal and let $\Cc$ be an $\infty$-category. Then the following are equivalent:
	\begin{alphanumerate}
		\item $\Cc$ is essentially $\kappa$-small.\label{enum:KappaSmallA}
		\item $\Cc$ is equivalent to a quasi-category with $<\kappa$ simplices across all dimensions.\label{enum:KappaSmallB}
		\item \!There exists a simplicial set $K$ with $<\kappa$ simplices across all dimensions and a Joyal equivalence $K\rightarrow \Cc$ \embrace{that is, a weak equivalence in the Joyal model structure from \cref{exm:JoyalModelStructure}}.\label{enum:KappaSmallC}
	\end{alphanumerate}
	Furthermore, if $K$ is a finite simplicial set \embrace{that is, a simplicial set with only finitely many non-degenerate simplices} and $K\rightarrow \Cc$ is a Joyal equivalence, then $\Cc$ is $\aleph_0$-small.
\end{lem}
\begin{proof}[Proof sketch]
	The implications \cref{enum:KappaSmallB} $\Rightarrow$ \cref{enum:KappaSmallA} and \cref{enum:KappaSmallB} $\Rightarrow$ \cref{enum:KappaSmallC} are trivial. For \cref{enum:KappaSmallC} $\Rightarrow$ \cref{enum:KappaSmallB} let $K\rightarrow \Cc'$ be the inner anodyne map into a quasi-category provided by the proof of \cref{lem:SmallObjectArgument}. Then $\Cc'$ has again $<\kappa$ simplices across all dimensions, because we're attaching $<\kappa$ new simplices countably many times. For the additional assertion, use induction on the dimension and write $K$ as a sequence of pushouts against $\coprod\partial\Delta^n\rightarrow \coprod\Delta^n$, where the disjoint union is finite. Replacing everything by quasi-categories and using model category fact~\cref{par:HomotopyPushout}, we conclude that $\Cc$ is contained in the full sub-$\infty$-category of $\cat{Cat}_\infty$ generated under pushouts by $\emptyset$ and $\Delta^n$ for all $n\geqslant 0$, as desired.
	
	It remains to show \cref{enum:KappaSmallA} $\Rightarrow$ \cref{enum:KappaSmallB}. We build a sub-simplicial set $\Cc'\subseteq\Cc$ as follows: Start with $\Cc'=\emptyset$. Choose $<\kappa$ representatives for every equivalence class in $\pi_0\core (\Cc)$ and add them to $\Cc'$. For all $x,y\in\Cc'$ and every equivalence class in $\pi_0\Hom_\Cc(x,y)$, we add a representative $\alpha\colon x\rightarrow y$. Furthermore, for every $n\geqslant 1$ and every class in $\pi_n(\Hom_\Cc(x,y),\alpha)$, we choose a representative $\Delta^n/\partial\Delta^n\rightarrow \Hom_\Cc(x,y)$ and add the simplices in the image of the corresponding map $\Delta^n/\partial\Delta^n\times\Delta^1\rightarrow\Cc$ to $\Cc'$. Then $\Cc'$ still has $<\kappa$ simplices. Mimicking the proof of \cref{lem:SmallObjectArgument}, we can add $<\kappa$ further simplices to $\Cc'$ to ensure that $\Cc'$ is a quasi-category. By construction, $\Cc'\rightarrow\Cc$ is essentially surjective and the map $\Hom_{\Cc'}(x,y)\rightarrow\Hom_\Cc(x,y)$ is a surjection on all $\pi_n$ for all $x,y\in\Cc'$. To make it injective, for every class in the kernel,  choose a homotopy $\Delta^n/\partial\Delta^n\times\Delta^1\rightarrow \Hom_\Cc(x,y)$ to $\const\alpha$. This homotopy corresponds to a map $(\Delta^n/\partial\Delta^n\times\Delta^1)\times\Delta^1\rightarrow \Cc$ and we add its image to $\Cc'$. Then we add $<\kappa$ simplices to make $\Cc'$ into a quasi-category again. Clearly, $\Cc'\rightarrow\Cc$ is still essentially surjective; furthermore, all elements in the previous kernel of $\pi_n(\Hom_{\Cc'}(x,y),\alpha)\rightarrow\pi_n(\Hom_\Cc(x,y),\alpha)$ have been killed now. But there could be new ones. So we simply repeat this process countably many times. Then $\Cc'\rightarrow \Cc$ is fully faithful too and thus an equivalence by \cref{thm:EquivalenceFullyFaithfulEssentiallySurjective}.%
	%
	%By adding another $<\kappa$ simplices to $\Cc'$, we can ensure that for all $x,y\in\Cc'$, every equivalence class in $\pi_0\F(\partial\Delta^n,\Hom_\Cc(x,y))$ and $\pi_0\F(\Delta^n,\Hom_\Cc(x,y))$ has a representative in $\Cc'$. Indeed, this can be done by applying the above observation to $K=(\partial\Delta^n\times\Delta^1)/(\partial\Delta^n\times\{0,1\})$ and $K=(\Delta^n\times\Delta^1)/(\Delta^n\times\{0,1\})$. Mimicking the proof of \cref{lem:SmallObjectArgument}, we can add $<\kappa$ further simplices to $\Cc'$ to ensure that $\Cc'$ is a quasi-category. Then $\Cc'$ has $<\kappa$ simplices across all dimensions. Furthermore, $\Cc'\rightarrow\Cc$ is essentially surjective by construction and for all $x,y\in\Cc'$ the map $\Hom_{\Cc'}(x,y)\rightarrow\Hom_\Cc(x,y)$ is a bijection on $\pi_0$ and an isomorphism on $\pi_n$ for all $n\geqslant 1$ and all basepoints. Hence $\Cc'\rightarrow\Cc$ is fully faithful too and thus an equivalence by \cref{thm:EquivalenceFullyFaithfulEssentiallySurjective}.%First, a simple induction over the number of simplices shows that $\pi_0\core\F(K,\Cc)$ has cardinality $<\kappa$ for every finite simplicial set $K$. For the inductive step, write $K\cong K'\sqcup_{\partial\Delta^n}\Delta^n$, so that $\F(K,\Cc)\cong \F(K',\Cc)\times_{\F(\partial\Delta^n,\Cc)}\F(\Delta^n,\Cc)$. Let $S\subseteq \F(K',\Cc)_0$ be a set of $<\kappa$ representatives for every equivalence class in $\pi_0\core\F(K',\Cc)$ and define $T\subseteq \F(\Delta^n,\Cc)_0$ similarly. For every $\sigma\in S$ and $\tau\in T$ such that the images of $\sigma$ and $\tau$ in $\F(\partial\Delta^n,\Cc)$ are equivalent, we can find $\tau'\in \F(\Delta^n,\Cc)$ such that $\tau\simeq \tau'$ and the images of  $\sigma$ and $\tau'$ in $\F(\partial\Delta^n,\Cc)$ are equal. Indeed, claim~\cref{claim:Pullback} from the proof of \cref{thm:EquivalenceFullyFaithfulEssentiallySurjective} allows us to lift equivalences. Then $\{(\sigma,\tau')\}$ is a set of $<\kappa$ representatives for every equivalence class in $\pi_0\core\F(K,\Cc)$, finishing the induction.
\end{proof}
\begin{rem}\label{rem:KappaSmallClosedUnderPushouts}
	If $\kappa$ is an uncountable regular cardinal, then pushouts or pullbacks of essentially $\kappa$-small $\infty$-categories are essentially $\kappa$-small again. Indeed, this follows from \cref{lem:KappaSmall}\cref{enum:KappaSmallB} together with~\cref{par:HomotopyPushout} and a cardinality bound on \cref{lem:SmallObjectArgument}: A functor between quasi-categories with $<\kappa$ simplices across all dimensions can be factored into a cofibration followed by a trivial fibration or into a Joyal equivalence followed by an isofibration in such a way that the new quasi-category in the middle has again $<\kappa$ simplices across all dimensions. Combining this observation with \cref{lem:KappaSmallColimits} below, we see that the full sub-$\infty$-category $\cat{Cat}_\infty^{<\kappa}$ of essentially $\kappa$-small $\infty$-categories is closed under $\kappa$-small limits and colimits.
	
	In the case $\kappa=\aleph_0$ it's obvious that $\aleph_0$-small $\infty$-categories are closed under pushouts and thus under finite colimits by \cref{lem:KappaSmallColimits} below. The same can be shown for finite products, but I don't know if it works for pullbacks too.
\end{rem}
\begin{lem}\label{lem:KappaSmallColimits}
	Let $\kappa$ be a regular cardinal. An $\infty$-category $\Cc$ has all $\kappa$-small colimits if and only if $\Cc$ has pushouts and $\kappa$-small coproducts. A functor $F\colon \Cc\rightarrow\Dd$ of $\infty$-categories preserves colimits if and only if it preserves pushouts and $\kappa$-small coproducts. A dual assertion holds for limits.
\end{lem}
\begin{proof}[Proof sketch]
	Repeat the proof of \cref{lem:ColimitsIffCoproductsAndPushouts} and use \cref{lem:KappaSmall} together with model category fact~\cref{par:HomotopyPushout} to see that pushouts of $\kappa$-small $\infty$-categories are still $\kappa$-small.
\end{proof}
%This finishes our discussion of cardinality bounds on $\infty$-categories. Our next goal is to study filtered colimits in the $\infty$-setting.
\subsection{Filtered colimits}

In this subsection, we'll study filtered colimits in $\infty$-categories and prove a version of the well-known fact that filtered colimits commute with finite limits (\cref{lem:FilteredColimitsPreserveFiniteLimits}).

\begin{con}\label{con:ConeCategory}
	Let $\Ii$ be an $\infty$-category. We define the \emph{cone $\Ii^\triangleleft$ over $\Ii$} and the \emph{cocone $\Ii^\triangleright$ under $\Ii$} as the following pushouts in $\cat{Cat}_\infty$:
	\begin{equation*}
		\begin{tikzcd}
			\Ii\times\{0\}\rar\dar\drar[pushout] & \Ii\times\Delta^1\dar\\
			*\rar & \Ii^\triangleleft
		\end{tikzcd}\quad\text{and}\quad
		\begin{tikzcd}
			\Ii\times\{1\}\rar\dar\drar[pushout] &\Ii\times\Delta^1\dar\\
			*\rar & \Ii^\triangleright
		\end{tikzcd}
	\end{equation*}
	It's tempting to use the procedure from model category fact~\cref{par:HomotopyPushout} to compute these pushouts explicitly, but this is a little tricky. Steps \cref{enum:PushoutStepA}, \cref{enum:PushoutStepB}, and \cref{enum:PushoutStepC} are easy though: $\Ii\times \{0\}\rightarrow \Ii\times\Delta^1$ and $\Ii\times\{1\}\rightarrow \Ii\times\Delta^1$ are already cofibrations, so we can simply take the pushout on the nose. The tricky step, however, is \cref{enum:PushoutStepD}, in which one has to replace the pushout in   $\cat{sSet}$ by a quasi-category. One can show that the \emph{joins} $\{0\}\star\Ii$ and $\Ii\star\{1\}$, which we didn't introduce, are such replacements; see \cite[Proposition~\HTTthm{4.1.2.1}]{HTT} or \cite[Proposition~2.5.19]{Land}. We won't need this explicit description and work with the abstract construction exclusively.
	
	Note that $*\rightarrow \Ii^\triangleleft$ is an initial object and $*\rightarrow \Ii^\triangleright$ is a terminal object. This is obvious in the simplicial models, but there's also a model-independent argument: We must show that $*\rightarrow\Ii^\triangleleft$ is left adjoint to the unique functor $\Ii^\triangleleft\rightarrow *$. This can be done via \cref{lem:TriangleIdentities} by constructing the unit and counit by hand. The unit is clear, as there are not that many functors from $*$ to itself (in fact, there's only one). For the counit, we must construct a natural transformation $c\colon \Ii^\triangleleft\times\Delta^1\rightarrow\Ii^\triangleleft$ from $\const *$ to $\id_{\Ii^\triangleleft}$. Using that $-\times\Delta^1\colon \cat{Cat}_\infty\rightarrow \cat{Cat}_\infty$ commutes with pushouts (since $\Fun(\Delta^1,-)$ is a right adjoint by \cref{exm:Adjunctions}\cref{enum:Currying}), this boils down to constructing a natural transformation $\Delta^1\times\Delta^1\rightarrow\Delta^1$ from $\const 0$ to $\id_{\Delta^1}$, which is easy. In the same way, verifying the triangle identities reduces to a question about $\Delta^1$. In particular, we deduce $\abs{\Ii^\triangleleft}\simeq *$ and $\abs{\Ii^\triangleright}\simeq *$, as $\infty$-categories with an initial or terminal object are always weakly constractible.
	
	As in ordinary category theory, cones and cocones are closely related to limits and colimits, respectively. Concretely, if $F\colon \Ii\rightarrow \Cc$ is a functor and $y\in\Cc$ is an object, then an easy calculation using \cref{cor:HomPreservesLimits} shows
	\begin{equation*}
		\{F\}\times_{\Hom_{\cat{Cat}_\infty}(\Ii,\Cc)}\Hom_{\cat{Cat}_\infty}(\Ii^\triangleright,\Cc)\times_{\Hom_{\cat{Cat}_\infty}(*,\Cc)}\{y\}\simeq \Hom_{\Fun(\Ii,\Cc)}(F,\const y)\,.
	\end{equation*}
	Informally, an extension of $F$ to a functor $F^\triangleright \colon \Ii^\triangleright\rightarrow \Cc$ that sends the tip $*\in \Cc^\triangleright$ to $y$ is the same as a natural transformation $F\Rightarrow \const y$. If $F$ admits a colimit, such a natural transformation is the same as a morphsm $\colimit_{i\in\Ii}F(i)\rightarrow y$, and the right-hand side above is equivalent to $\Hom_\Cc(\colimit_{i\in\Ii}F(i),y)$.
\end{con}

\begin{defi}\label{def:KappaFiltered}
	Let $\kappa$ be a regular cardinal and let $\Jj$, $\Cc$ be $\infty$-categories.
	\begin{alphanumerate}
		\item $\Jj$ is called \emph{$\kappa$-filtered} if every functor $\Ii\rightarrow\Jj$ from an essentially $\kappa$-small $\infty$-category extends to a functor $\Ii^\triangleright\rightarrow\Jj$ from the cocone under $\Cc$, or in other words, if the restriction $\Fun(\Ii^\triangleright,\Jj)\rightarrow\Fun(\Ii,\Jj)$ is essentially surjective. In the case $\kappa=\aleph_0$, we usually just say $\Jj$ is \emph{filtered}.\label{enum:KappaFilteredCategory}
		\item A colimit over a functor $F\colon \Jj\rightarrow\Cc$ is called \emph{$\kappa$-filtered} if $\Jj$ is $\kappa$-filtered, and \emph{filtered} if $\Jj$ is filtered.\label{enum:KappaFilteredColimit}
		\item An object $x\in\Cc$ is called \emph{$\kappa$-compact} or \emph{compact} if $\Hom_\Cc(x,-)\colon \Cc\rightarrow\cat{An}$ commutes with $\kappa$-filtered or filtered colimits, respectively.\label{enum:KappaCompact} 
	\end{alphanumerate}
\end{defi}
\begin{rem}\label{rem:LuriesFilteredness}
	We'll explain why Lurie's definition of $\kappa$-filteredness in \cite[Definition~\HTTthm{5.3.1.7}]{HTT} is equivalent to ours. Let $\Jj$ be a $\kappa$-filtered quasi-category as in \cref{def:KappaFiltered}\cref{enum:KappaFilteredCategory}. Furthermore, let $\Ii$ be an essentially $\kappa$-small quasi-category and choose the simplicial model $\Ii\star\{1\}$ for $\Ii^\triangleright$ (as Lurie does). Then any functor $\Ii\rightarrow\Jj$ can not only be extended to $\Ii^\triangleright\rightarrow\Jj$ up to equivalence, but even on the nose. The reason is that $\Ii\rightarrow\Ii^\triangleright$ is a cofibration and thus $\core\F(\Ii^\triangleright,\Jj)\rightarrow \core\F(\Ii,\Jj)$ has lifting against $\{0\}\rightarrow\Delta^1$ by claim~\cref{claim:Pullback} in the proof of \cref{thm:EquivalenceFullyFaithfulEssentiallySurjective}. Then \cref{lem:KappaSmall} easily implies that Lurie's definition of $\kappa$-filteredness is equivalent to ours in the case where $\kappa$ is uncountable.
	
	If $\kappa=\aleph_0$, then \cref{lem:KappaSmall} shows that any filtered $\infty$-category $\Jj$ in the sense of \cref{def:KappaFiltered}\cref{enum:KappaFilteredCategory} is also filtered in Lurie's sense. The converse is true as well, but not as obvious (at least to me), since I don't know if the converse of the additional assertion in \cref{lem:KappaSmall} is true (I'd guess it's not). So here's a different argument: If $\Jj$ is filtered in Lurie's sense, then $\colimit\colon \Fun(\Jj,\cat{An})\rightarrow\cat{An}$ preserves finite limits (by \cite[Proposition~\HTTthm{5.3.3.3}]{HTT} or by observing that the proof of \cref{lem:FilteredColimitsPreserveFiniteLimits} still goes through). Hence \cref{lem:FilteredColimitsPreserveFiniteLimits}, which we'll prove next, implies that $\Jj$ is filtered in our sense. 
\end{rem}
\begin{thm}\label{lem:FilteredColimitsPreserveFiniteLimits}
	Let $\kappa$ be a regular cardinal. Then an $\infty$-category is $\kappa$-filtered if and only if the functor $\colimit\colon \Fun(\Jj,\cat{An})\rightarrow\cat{An}$ preserves $\kappa$-small limits.
\end{thm}
Before we can prove \cref{lem:FilteredColimitsPreserveFiniteLimits}, we need to send four more lemmas in advance.%First, we need a general coinitiality result for $\kappa$-filtered $\infty$-categories.
\begin{lem}\label{lem:FilteredCofinal}
	Let $\kappa$ be a regular cardinal and let $\Jj$ be a $\kappa$-filtered $\infty$-category. Then $\abs*{\Jj}\simeq *$. Furthermore, for every $j\in\Jj$ the slice $\Jj_{j/}$ is $\kappa$-filtered again and $\Jj_{j/}\rightarrow \Jj$ is coinitial.
\end{lem}
\begin{proof}[Proof sketch]
	To see $\abs{\Jj}\simeq *$, unfortunately, we need to use simplicial methods. It's enough to show that every map $\sigma\colon \partial\Delta^n\rightarrow\abs{\Jj}$ is nullhomotopic, because then the same argument as in the proof of \cref{lem:ContractibleKanComplex} shows that $\abs{\Jj}\rightarrow*$ is a trivial fibration. We'll show that for every $\sigma$ there is a functor $\alpha\colon\Ii\rightarrow \Jj$ from $\aleph_0$-small $\infty$-category $\Ii$ such that $\sigma$ factors through $\abs{\alpha}\colon\abs{\Ii}\rightarrow \abs{\Jj}$. This will be enough since then $\sigma$ also factors through $*\simeq \abs{\Ii^\triangleright}\rightarrow\abs{\Jj}$ by filteredness of $\Jj$. To construct $\alpha$, recall that $\Jj\rightarrow\abs{\Jj}$ can be constructed as an anodyne map into a Kan complex via \cref{lem:SmallObjectArgument}. Accordingly, as simplicial sets, $\abs{\Jj}\cong \colimit_{i\geqslant 0}\Jj_i$, where $\Jj_0=\Jj$ and $\Jj_{i+1}$ is obtained from $\Jj_i$ by attaching solutions to horn filling problems. All the finitely many simplices in the image of $\sigma\colon\partial\Delta^n\rightarrow\abs{\Jj}$ must already be contained in $\Jj$ or occur in $\Jj_i$ as a solution to some horn filling problem. If the latter is the case, all the finitely many simplices involved in that horn filling problem must already occur in $\Jj$ or in some $\Jj_k$ for $k<i$. Continuing in this way, we can trace back $\sigma$ to a finite number of simplices in $\Jj$. Completing these finitely many simplices to a sub-quasi-category $\Ii\subseteq\Jj$ as in the proof of \cref{lem:KappaSmall} yields the desired functor $\alpha\colon \Ii\rightarrow\Jj$. %use not only simplicial methods, but \cref{thm:SimplicialApproximation} (please tell me if you know a better argument): Let $X$ be the CW complex obtained as the geometric realisation of the Kan complex $\abs*{\Jj}$. Then $X$ is homotopy equivalent to the geometric realisation of $\Jj$, because $\Jj\rightarrow \abs*{\Jj}$ is anodyne by \cref{con:Localisation}. So it suffices to prove that every map $\alpha\colon \partial D^n\rightarrow X$ from the boundary of a topological $n$-disk extends all of $D^n$, up to homotopy. This follows from simplicial approximation in the form of \cite[Theorem~\href{https://pi.math.cornell.edu/~hatcher/AT/AT.pdf\#page=186}{2C.1}]{Hatcher}: Write $\partial D^n\cong \abs*{\partial \Delta^n}$; up to homotopy, we may assume that $\alpha$ is the geometric realisation of some map $\alpha_0\colon \operatorname{sd}^m(\partial\Delta^n)\rightarrow \Jj$ from some barycentric subdivision of $\partial\Delta^n$. Since $\Jj$ is filtered, $\alpha_0$ extends to $\alpha_0^\triangleright\colon\operatorname{sd}^m(\partial\Delta^n)^\triangleright\rightarrow\Jj$. Since $\abs*{\operatorname{sd}^m(\partial\Delta^n)^\triangleright}\simeq D^n$, we've proved that $\alpha$ extends to all of $D^n$ up to homotopy. This proves $\abs*{\Jj}\simeq *$.
	
	For the other assertions, let $\Ii\rightarrow\Jj_{j/}$ be a map from an essentially $\kappa$-small $\infty$-category. By unravelling the respective universal properties, such a map is equivalent to a map $\Ii^\triangleleft\rightarrow \Jj$ sending the tip of the cone to $j$. Since $\Ii^\triangleleft$ is still essentially $\kappa$-small, we get an extension $(\Ii^\triangleleft)^\triangleright\rightarrow\Jj$. Since $(\Ii^\triangleleft)^\triangleright\simeq (\Ii^\triangleright)^\triangleleft$, this defines a map $\Ii^\triangleright\rightarrow\Jj_{j/}$, proving that $\Jj_{j/}$ is $\kappa$-filtered. An analogous argument shows that $\Jj_{j/}\times_\Jj\Jj_{j'/}$ is $\kappa$-filtered for every $j'\in \Jj$. Hence $\abs{\Jj_{j/}\times_\Jj\Jj_{j'/}}\simeq *$ by the first part. Thus $\Jj_{j/}\rightarrow\Jj$ is coinitial by \cref{thm:JoyalsQuillenA}\cref{enum:WeaklyContractible}.
\end{proof}
%Next, we need a result about limits and colimits in slice $\infty$-categories.
\begin{lem}\label{lem:ColimitsInSliceCategory}
	Let $\Dd$ be an $\infty$-category and $y\in\Dd$ an object.
	\begin{alphanumerate}
		\item $\Dd_{y/}\rightarrow\Dd$ reflects and creates limits. That is, if $\alpha\colon\Ii\rightarrow\Dd_{y/}$ is a diagram such that the underlying diagram $\ov\alpha\colon\Ii\rightarrow \Dd_{y/}\rightarrow\Dd$ has a limit in $\Dd$, then $\alpha$ has a limit which is preserved under $\Dd_{y/}\rightarrow\Dd$.\label{enum:LimitsInSlice}
		\item If $\abs{\Ii}\simeq *$, then $\Dd_{y/}\rightarrow\Dd$ reflects and creates $\Ii$-shaped colimits. In particular, this applies to pushouts \embrace{since $\abs{\Lambda_0^2}\simeq*$} and filtered colimits \embrace{by \cref{lem:FilteredCofinal}}.\label{enum:ColimitsInSlice}
		\item In general, let $\alpha\colon \Ii\rightarrow \Dd_{y/}$ be a diagram in $\Dd_{y/}$ and let $\ov\alpha\colon \Ii\rightarrow \Dd_{y/}\rightarrow\Dd$ be the underlying diagram in $\Dd$. If the colimits $\colimit_{i\in\Ii}\ov\alpha(i)$ and $\colimit_{i\in\Ii}y$ as well as the pushout\label{enum:ColimitsInSliceGeneral}
		\begin{equation*}
			\begin{tikzcd}
				\colimit_{i\in\Ii}y\dar\rar\drar[pushout] & \colimit_{i\in\Ii}\ov\alpha(i)\dar\\
				y\rar & c
			\end{tikzcd}
		\end{equation*}
		exist in $\Dd$, then $(y\rightarrow c)\in\Dd_{y/}$ is the colimit of $\alpha\colon \Ii\rightarrow \Dd_{y/}$.
	\end{alphanumerate}
	%Dual assertions hold for the other slice projection $\Dd_{/y}\rightarrow\Dd$.
\end{lem}
\begin{proof}[Proof sketch]
	For \cref{enum:LimitsInSlice}, first consider the case where $\Dd$ has all limits. The functor $\alpha\colon\Ii\rightarrow\Dd_{y/}$ defines a natural transformation $\const y\Rightarrow \ov\alpha$, hence a morphism $y\rightarrow \lim_{i\in\Ii}\ov\alpha(i)$. We claim that $(y\rightarrow \limit_{i\in\Ii}\ov\alpha(i))$ is the limit of $\alpha$. Indeed, using \cref{cor:HomInSliceCategories} and the fact that limits commute with limits by the dual of \cref{lem:ColimitManipulations}, we immediately verify the condition from \cref{cor:HomPreservesColimits}. This concludes the case where $\Dd$ has all limits. The general case can be reduced to this special case by considering a fully faithful limit-preserving functor $i\colon \Dd\rightarrow\Dd'$ into an $\infty$-category with all limits; for example, $\Yo_\Dd\colon \Dd\rightarrow\Fun(\Dd^\op,\cat{An})$ does it by \cref{cor:HomPreservesLimits}.
	
	Assertion \cref{enum:ColimitsInSliceGeneral} follows from \cref{lem:SliceAdjunction}, using $\Fun(\Ii,\Dd_{y/})\simeq \Fun(\Ii,\Dd)_{\const y/}$. To prove \cref{enum:ColimitsInSlice}, first assume that $\Dd$ has all colimits. Then the assumptions from \cref{enum:ColimitsInSliceGeneral} are satisfied and $\colimit_{i\in\Ii}\alpha(i)$ exists. If $\abs{\Ii}\simeq *$, then \cref{lem:ContractibleColimit} below implies that the canonical morphism $\colimit_{i\in\Ii}y\rightarrow y$ is an equivalence. Hence the pushout from \cref{enum:ColimitsInSliceGeneral} becomes an equivalence $\colimit_{i\in\Ii}\ov\alpha(i)\simeq c$. This proves \cref{enum:ColimitsInSlice} in the case where $\Dd$ has all colimits. For the general case, choose a fully faithful colimit-preserving functor $i\colon \Dd\rightarrow\Dd'$ into an $\infty$-category $\Dd$ with all colimits; for example, the mutilated Yoneda embedding $(\Yo_{\Dd^\op})^\op\colon (\Dd^\op)^\op\rightarrow\Fun(\Dd,\cat{An})^\op$ does it by \cref{cor:HomPreservesLimits}.
	%
	%, we use a general fact: If $L\colon \Cc\shortdoublelrmorphism\Dd\noloc R$ is an adjunction of $\infty$-categories and the counit $c_y\colon LR(y)\rightarrow y$ is an equivalence, then we get an induced adjunction $L\colon \Cc_{R(y)/}\shortdoublelrmorphism \Dd_{y/}\noloc R$ on slice $\infty$-categories. The proof is another straightforward application of \cref{cor:HomInSliceCategories}.
	%
	%First assume that $\Dd$ has $\Ii$-shaped colimits. Applying the general fact to the adjunction $\colimit_\Ii\colon \Fun(\Ii,\Dd)\shortdoublelrmorphism \Dd\noloc \const$ and using $\Fun(\Ii,\Dd_{y/})\simeq \Fun(\Ii,\Dd)_{\const y/}$, we see that it suffices to check $\colimit_\Ii\const y\simeq y$. This follows from \cref{lem:ContractibleColimit} below. This settles the case where $\Dd$ has $\Ii$-shaped colimits. For the general case, choose a fully faithful colimit-preserving functor $i\colon \Dd\rightarrow\Dd'$ into an $\infty$-category $\Dd$ with all colimits; for example, the mutilated Yoneda embedding $(\Yo_{\Dd^\op})^\op\colon (\Dd^\op)^\op\rightarrow\Fun(\Dd,\cat{An})^\op$ does it by \cref{cor:HomPreservesLimits}.
\end{proof}
\begin{lem}\label{lem:ContractibleColimit}
	Let $\Dd$ be an $\infty$-category, $y\in\Dd$ an object, and $\Ii$ be an $\infty$-category satisfying $\abs*{\Ii}\simeq *$. Then $\colimit_{i\in\Ii} y\simeq y$; in particular, this colimit always exists.
\end{lem}
\begin{proof}
	Clearly, $\const y\colon \Ii\rightarrow \Dd$ factors through $\Ii\rightarrow\abs*{\Ii}$. This functor is coinitial by \cref{exm:Cofinal}\cref{enum:LocalisationsCofinal}, and since $ \abs*{\Ii}\simeq *$, it follows that the colimit is indeed given by $y$. 
\end{proof}
The following lemma is the crucial step in the proof of \cref{lem:FilteredColimitsPreserveFiniteLimits}:
\begin{lem}\label{lem:HomotopyGroupsFilteredColimits}
	The functor $\pi_0\colon \cat{An}\rightarrow \cat{Set}$ commutes with products and all colimits. The functors $\pi_1\colon \cat{An}_{*/}\rightarrow\cat{Grp}$, and $\pi_n\colon\cat{An}_{*/}\rightarrow\cat{Ab}$ for all $n\geqslant 2$ commute with products and filtered colimits.
\end{lem}
\begin{proof}[Proof sketch]
	Since $\pi_0\colon\cat{An}\rightarrow\cat{Set}$ is left adjoint to the inclusion $\cat{Set}\subseteq \cat{An}$, \cref{lem:AdjointsPreserveColimits} shows that $\pi_0$ preserves colimits. By a simple inspection $\pi_0$ also preserves products. This immediately implies that $\pi_n$ preserves products for all $n\geqslant 1$, since $\pi_n(X,x)\cong \pi_0\Hom_{\cat{An}_{*/}}((S^n,*),(X,x))$ and $\Hom_{\cat{An}_{*/}}((S^n,*),-)\colon \cat{An}_{*/}\rightarrow \cat{An}$ preserves limits by \cref{cor:HomPreservesLimits}.
	
	The assertion about $\pi_n$ needs simplicial methods (and two black boxes), unfortunately. Let $\Jj$ be a filtered $\infty$-category. For every ordinary category $\Cc$, we have $\Fun(\Jj,\Cc)\simeq \Fun(\operatorname{ho}(\Jj),\Cc)$ by \cref{lem:SimplicialHoNerveAdjunction}, and so $\Jj$-shaped colimits in $\Cc$ agree with $\operatorname{ho}(\Jj)$-shaped colimits. It's straightforward to see that $\operatorname{ho}(\Jj)$ is filtered in the usual sense. So for filtered colimits in an ordinary category we can replace the indexing diagram by an ordinary filtered category. But a stronger assertion is true, which we'll need later:
	\begin{alphanumerate}\itshape
		\item[\blacksquare_1] For every filtered $\infty$-category there exists a directed partially ordered set $J$ and a coinitial functor $J\rightarrow \Jj$.\label{blackbox:Cofinal}
	\end{alphanumerate}
	For a proof of \cref{blackbox:Cofinal} see \cite[Proposition~\HTTthm{5.3.1.18}]{HTT} or \cite[Tag~\href{https://kerodon.net/tag/02QA}{02QA}]{Kerodon} (the Kerodon proof is relatively short and only uses methods that we have already available).
	
	Next, observe that $\pi_1\colon \cat{Kan}_{*/}\rightarrow\cat{Grp}$ and $\pi_n\colon \cat{Kan}_{*/}\rightarrow \cat{Ab}$ for $n\geqslant 2$ commute with filtered colimits in the ordinary category $\cat{Kan}_{*/}$. This follows essentially from the fact that $\square^n$ and $\partial\square^n$ are finite simplicial sets, using an argument as near the end of the proof of \cref{lem:SmallObjectArgument}. It follows that for every filtered $\infty$-category $\Jj$, the functor $\colimit\colon \Fun(\Jj,\cat{Kan})\rightarrow\cat{Kan}$ sends pointwise homotopy equivalences to homotopy equivalences. Indeed, let $X_{(-)}\Rightarrow X'_{(-)}$ be a natural transformation in $\Fun(\Jj,\cat{Kan})$ such that $X_j\rightarrow X'_j$ is a homotopy equivalence for all $j\in\Jj$. We can check on homotopy groups whether $\colimit_{j\in\Jj}X_j\rightarrow\colimit_{j\in\Jj}X_j$ is a homotopy equivalence. By the argument above, we get a bijection on $\pi_0$. Now let $x\in \pi_0(\colimit_{j\in\Jj}X_j)$ be a point. Since $\pi_0$ commutes with colimits, we must have $x\in \pi_0(X_{j_0})$ for some $j_0\in \Jj$. By \cref{lem:FilteredCofinal}, we may replace $\Jj$ by $\Jj_{j_0/}$, so we may assume $j_0$ is initial in $\Jj$. Then $\{x\}\rightarrow X_{j_0}\rightarrow X_j$ for all $j\in \Jj$ turns $X_{(-)}$ into a functor $(X_{(-)},x)\colon \Jj\rightarrow\cat{Kan}_{*/}$. The same works for $X'_{(-)}$. Since $X_{(-)}\Rightarrow X'_{(-)}$ is a pointwise homotopy equivalence and $\pi_n$ commutes with filtered colimits in $\cat{Kan}_{*/}$, we conclude that $\pi_n(\colimit_{j\in\Jj}X_j,x)\cong \pi_n(\colimit_{j\in\Jj}X_j',x)$. This finishes the proof that $\colimit\colon \Fun(\Jj,\cat{Kan})\rightarrow\cat{Kan}$ sends pointwise homotopy equivalences to homotopy equivalences. At this point, we need the second black box:
	\begin{alphanumerate}\itshape
		\item[\blacksquare_2] If $J$ is a directed partially ordered set, then there is an equivalence of $\infty$-categories\label{blackbox:Localisation}
		\begin{equation*}
			\Fun(J,\cat{Kan})\left[\{\text{pointwise homotopy equivalences}\}^{-1}\right]\overset{\simeq}{\longrightarrow}\Fun(J,\cat{An})\,.
		\end{equation*}
	\end{alphanumerate}
	The proof of \cref{blackbox:Localisation} is similar to that of \cref{thm:AnAsALocalisation}: First, one defines a simplicial model structure on $\Fun(J,\cat{sSet})$ in such a way that $\N^\Delta((\Fun(J,\cat{sSet})^\Delta)^\mathrm{cf})\simeq \Fun(J,\cat{An})$.  
	In the proof of \cite[Proposition~\HTTthm{5.3.3.3}]{HTT}, Lurie explains how to do this. Then one uses \cref{rem:SimplicialModelCategory,rem:ModelCategoryUnderlyingInftyCategory} to identify the simplicial nerve $\N^\Delta(\Fun(J,\cat{sSet})_\Delta^\mathrm{cf})$ with the localisation above.
	
	Now let $p\colon\cat{Kan}\rightarrow\cat{An}$ and $p_J\colon \Fun(J,\cat{Kan})\rightarrow\Fun(J,\cat{An})$ denote the canonical functors. By \cref{thm:AnAsALocalisation} and \cref{blackbox:Localisation}, both $p$ and $p_J$ are localisations. As we've seen above, $\colimit\colon \Fun(J,\cat{Kan})\rightarrow\cat{Kan}$ sends pointwise homotopy equivalences to homotopy equivalences. Hence $p\circ \colimit\colon \Fun(J,\cat{Kan})\rightarrow \cat{An}$ factors uniquely through the localisation $p_J$ by \cref{lem:Localisation}. Let $c\colon \Fun(J,\cat{An})\rightarrow\cat{An}$ be the induced functor; we claim that $c$ is simply the colimit functor. To this end, consider $\const\colon \cat{Kan}\rightarrow\Fun(J,\cat{Kan})$; since it sends homotopy equivalences to pointwise homotopy equivalences, the same argument as above shows that $p_J\circ \const$ factors uniquely through the localisation $p$. That factorisation is necessarily $\const\colon \cat{An}\rightarrow\Fun(J,\cat{An})$. It's straightforward to verify that the adjunction $\colimit\colon \Fun(J,\cat{Kan})\shortdoublelrmorphism \cat{Kan}\noloc \const$ descends to an adjunction $c\colon \Fun(J,\cat{An})\shortdoublelrmorphism \cat{An}\noloc \const$ on the localisations. Indeed, one can show using \cref{lem:Localisation} that the unit and counit transformations as well as the triangle identities get inherited, so we may appeal to \cref{lem:TriangleIdentities}. This shows that $c$ is left adjoint to $\const$, hence it must be the colimit functor, as claimed.
	
	Finally, we can finish the proof. Let $(X_{(-)},x_{(-)})\colon\Jj\rightarrow \cat{An}_{*/}$ be a functor from a filtered $\infty$-category into pointed animae. By \cref{blackbox:Cofinal}, we may assume that $\Jj\simeq J$ is a directed partially ordered set. By \cref{lem:FilteredCofinal}, we may assume that $J$ contains an initial object $j_0$. Then for every $(X_j,x_j)$, the point $\{x_j\}\rightarrow X_j$ agrees with $\{x_{j_0}\}\rightarrow X_{j_0}\rightarrow X_j$. Since $\cat{An}_{*/}\rightarrow\cat{An}$ preserves filtered colimits by \cref{lem:ColimitsInSliceCategory}, it follows that the pointed anima $\colimit_{j\in J}(X_j,x_j)$ is given by the unpointed colimit $\colimit_{j\in J}X_j$ together with the point $x_{j_0}\rightarrow X_{j_0}\rightarrow \colimit_{j\in J}X_j$. By \cref{blackbox:Localisation}, $\Fun(J,\cat{Kan})\rightarrow\Fun(J,\cat{An})$ is essentially surjective. So we may assume that $X_{(-)}$ comes from a functor $X_{(-)}\colon J\rightarrow\cat{Kan}$. As argued above, we may then as well take the colimit in $\cat{Kan}$ instead of $\cat{An}$. So the fact that $\pi_n\colon \cat{An}_{*/}\rightarrow \cat{Set}$ preserves filtered colimits reduces to the same assertion about $\pi_n\colon\cat{Kan}_{*/}\rightarrow\cat{Set}$, which we already know.
\end{proof}
\begin{proof}[Proof of \cref{lem:FilteredColimitsPreserveFiniteLimits}]
	First assume $\Jj$ is $\kappa$-filtered. By \cref{lem:KappaSmallColimits}, it's enough to show that $\colimit\colon\Fun(\Jj,\cat{An})\rightarrow\cat{An}$ preserves pullbacks and $\kappa$-small products. Using \cref{lem:LongExactFibrationSequence} and the five lemma (plus \cref{rem:ExactnessInLowDegrees}), we can further reduce pullbacks to fibre sequences (in the sense of \cref{def:Cofibre}).
	
	Let's do $\kappa$-small products first. We have to show that for every set $I$ of cardinality $<\kappa$, every $\kappa$-filtered $\infty$-category $\Jj$, and every functor $X_{(-,-)}\colon I\times\Jj\rightarrow\cat{An}$, the natural map
	\begin{equation*}
		\colimit_{j\in\Jj}\prod_{i\in I}X_{i,j}\longrightarrow\prod_{i\in I}\colimit_{j\in\Jj}X_{i,j}
	\end{equation*}
	is an equivalence. This can be checked on homotopy groups. We get a bijection on $\pi_0$, since $\pi_0\colon\cat{An}\rightarrow\cat{Set}$ preserves products and colimits and in $\cat{Set}$, $\kappa$-filtered colimits commute with $\kappa$-small products. For higher homotopy groups, fix some $x\in \pi_0\bigl(\colimit_{j\in\Jj}\prod_{i\in I}X_{i,j}\bigr)$. Since $\pi_0$ commutes with colimits, we must have $x\in \pi_0\bigl(\prod_{i\in I}X_{i,j_0}\bigr)$ for some $j_0\in \Jj$. By \cref{lem:FilteredCofinal}, we may replace $\Jj$ by $\Jj_{j_0/}$ to assume that $j_0$ is initial in $\Jj$. In this case, the composition $\{x\}\rightarrow \prod_{i\in I}X_{i,j_0}\rightarrow X_{i,j_0}\rightarrow X_{i,j}$ for all $(i,j)\in I\times\Jj$ turns $X_{(-,-)}$ into a functor $X_{(-,-)}\colon I\times\Jj\rightarrow\cat{An}_{*/}$. Then $\pi_n\bigl(\colimit_{j\in\Jj}\prod_{i\in I}X_{i,j},x\bigr)\cong\pi_n\bigl(\prod_{i\in I}\colimit_{j\in\Jj}X_{i,j},x\bigr)$ follows from \cref{lem:HomotopyGroupsFilteredColimits} and the fact that $\kappa$-filtered colimits in $\cat{Grp}$ or $\cat{Ab}$ commute with $\kappa$-small products.%; the latter is straightforward to verify, observing that $\operatorname{ho}(\Jj)$ is $\kappa$-filtered in the ordinary sense.
	
	The case of fibre sequences is similar. Let $F_{(-)}\Rightarrow X_{(-)}\Rightarrow Y_{(-)}$ be a fibre sequence in $\Fun(\Jj,\cat{An})$; by \cref{lem:ColimitsInFunctorCategories}, this is equivalent to $F_j\rightarrow X_j\rightarrow Y_j$ being a fibre sequences for every $j\in\Jj$. We must show that
	\begin{equation*}
		\colimit_{j\in \Jj}F_j\longrightarrow\fib\Bigl(\colimit_{j\in\Jj}X_j\rightarrow \colimit_{j\in\Jj}Y_j\Bigr)
	\end{equation*}
	is an equivalence. This follows from a comparison of long exact sequences, using \cref{lem:LongExactFibrationSequence} and the five lemma together with the fact that filtered colimits preserve exact sequences. This finishes the proof that $\colimit\colon \Fun(\Jj,\cat{An})\rightarrow\cat{An}$ preserves $\kappa$-small limits.
	
	Conversely, assume this is the case; we must show that $\Jj$ is $\kappa$-filtered. Let $\alpha\colon \Ii\rightarrow \Jj$ be a functor from a essentially $\kappa$-small $\infty$-category. Consider the composition
	\begin{equation*}
		\Ii^\op\xrightarrow{\alpha^\op} \Jj^\op\xrightarrow{\Yo_{\Jj^{\op}}} \Fun(\Jj,\cat{An})\,.
	\end{equation*}
	Since $\Fun(\Jj,\cat{An})$ has all limits by \cref{lem:ColimitsInFunctorCategories}, we can put $E\coloneqq \lim(\Ii^\op\rightarrow \Fun(\Jj,\cat{An}))$ and extend the functor above to a limit cone $(\alpha^\op)^\triangleleft\colon (\Ii^\op)^\triangleleft\rightarrow\Fun(\Jj,\cat{An})$. Suppose there is an object $j\in\Jj^\op$ together with a natural transformation $\eta\colon \Hom_\Jj(j,-)\Rightarrow E$ in $\Fun(\Jj,\cat{An})$. We may view $(\alpha^\op)^\triangleleft$ as a natural transformation $\const E\Rightarrow \Yo_{\Jj^\op}\circ\alpha^\op$. Composing with $\const\eta\colon \const \Hom_\Jj(j,-)\Rightarrow \const E$ yields another natural transformation, which we may again view as a functor $(\beta^\op)^\triangleleft\colon (\Ii^\op)^\triangleleft\rightarrow\Fun(\Jj,\cat{An})$. Then $(\beta^\op)^\triangleleft$ lands in the essential image of $\Yo_{\Jj^\op}$. Since the Yoneda embedding is fully faithful by \cref{cor:YonedaEmbeddingFullyFaithful}, we obtain a functor $\beta^\triangleright\colon \Ii^\triangleright\rightarrow\Jj$, as desired.
	
	So assume on the contrary that there exists no $\eta\colon \Hom_\Jj(j,-)\Rightarrow E$ as above. By Yoneda's lemma, \cref{thm:Yoneda}, this implies $E(j)\simeq \emptyset$ for all $j\in \Jj$. Since initial objects are preserved under arbitrary colimits, $\colimit_{j\in\Jj}E(j)\simeq \emptyset$. On the other hand, \cref{lem:ColimitsInAnima} implies $\colimit_{j\in\Jj}\Hom_\Jj(j_0,j)\simeq \abs{\Jj_{j_0/}}\simeq*$ for every $j_0\in\Jj$. Since $\colimit\colon \Fun(\Jj,\cat{An})\rightarrow\cat{An}$ preserves $\kappa$-small limits by assumption, it follows that $\colimit_{j\in\Jj}E(j)\simeq \limit_{i\in\Ii^\op}*\simeq *$, as terminal objects are preserved under arbitrary limits. Since $\emptyset\not\simeq *$, we get a contradiction.
\end{proof}

%This finishes our lengthy discussion of filtered colimits in $\infty$-categories. Next, we'll introduce $\cat{Ind}$-categories in the world of $\infty$-categories and we'll finally define what a presentable $\infty$-category is supposed to be.

\subsection{Accessible and presentable \texorpdfstring{$\infty$}{Infinity}-categories}\label{subsec:Presentable}

We can now introduce a class of large $\infty$-categories that are generated by a small sub-$\infty$-category.

\begin{con}\label{con:Ind}
	Let $\kappa$ be a regular cardinal and let $\Cc$ be an essentially small $\infty$-category. We let $\cat{Ind}_\kappa(\Cc)\subseteq \PSh(\Cc)$ be the full sub-$\infty$-category spanned by those presheaves $E\colon \Cc^\op\rightarrow\cat{An}$ for which the unstraightening $\operatorname{Un}^{\mathrm{right}}(E)$ is $\kappa$-filtered. In the case $\kappa=\aleph_0$, we often write $\cat{Ind}(\Cc)\coloneqq\cat{Ind}_{\aleph_0}(\Cc)$.
	
	Note that the Yoneda embedding $\Yo_\Cc\colon \Cc\rightarrow\PSh(\Cc)$ factors through $\cat{Ind}_\kappa(\Cc)$. Indeed, for every $x\in\Cc$, the unstraightening of $\Hom_\Cc(-,x)\colon \Cc^\op\rightarrow\Cc$ is the right fibration $\Cc_{/x}\rightarrow\Cc$ by the dual of \cref{exm:Straightening}\cref{enum:SliceLeftFibration}. Now $\Cc_{/x}$ has a terminal object $\id_x\colon x\rightarrow x$, hence it is $\kappa$-filtered for any $\kappa$. Indeed, composing any functor $\Ii\rightarrow\Cc_{/x}$ with $\id_{\Cc_{/x}}\Rightarrow \const \id_x$ yields an extension $\Ii^\triangleright\rightarrow\Cc_{/x}$, as desired. Alternatively, we could have used \cref{lem:FilteredColimitsPreserveFiniteLimits}: Since $\Cc_{/x}$ has a terminal object, every colimit over $\Cc_{/x}$ is just given by evaluating at that object. Therefore, it follows from \cref{lem:ColimitsInFunctorCategories} that $\colimit\colon \Fun(\Cc_{/x},\cat{An})\rightarrow \cat{An}$ preserves arbitrary limits. We'll denote the factorisation of $\Yo_\Cc$ by
	\begin{equation*}
		\Yo_\Cc^\kappa\colon \Cc\longrightarrow\cat{Ind}_\kappa(\Cc)\,.
	\end{equation*}
	If no confusion can occur, we'll usually drop the superscript and just write $\Yo_\Cc$.
	
	More generally, we have $\operatorname{Un}^{\mathrm{right}}(E)\simeq \Cc_{/E}$ for all $E\in \PSh(\Cc)$, so $E$ is contained in $\cat{Ind}_\kappa(\Cc)$ if and only if $\Cc_{/E}$ is filtered. Indeed, the right fibration $\PSh(\Cc)_{/E}\rightarrow\PSh(\Cc)$ is the unstraightening of $\Hom_{\PSh(\Cc)}(-,E)\colon \PSh(\Cc)^\op\rightarrow\cat{An}$. By Yoneda's lemma (combined with \cref{par:YonedaFunctorial}), we have an equivalence $E\simeq \Hom_{\PSh(\Cc)}(\Yo_\Cc(-),E)$ of presheaves. Hence the unstraightening of $E$ is the pullback of $\PSh(\Cc)_{/E}\rightarrow\PSh(\Cc)$ along $\Yo_\Cc\colon \Cc\rightarrow\PSh(\Cc)$, which is $\Cc_{/E}$.
\end{con}
\begin{defi}\label{def:Presentable}
	Let $\kappa$ be a regular cardinal. A \embrace{not necessarily essentially small} $\infty$-category $\Cc$ is \emph{$\kappa$-accessible} if $\Cc\simeq\cat{Ind}_\kappa(\Cc_0)$ for some small $\infty$-category $\Cc_0$. We call $\Cc$ \emph{accessible} if it is $\kappa$-accessible for some regular cardinal $\kappa$. We call $\Cc$ \emph{presentable} if it is accessible and has all colimits.
\end{defi}
%We'll now prove several useful properties and characterisations of accessible $\infty$-categories, including an analogue of \cref{thm:PShFreeCocompletion}.%After that, we'll spend a few pages studying presentability before we finally get to the adjoint functor theorem.
\begin{lem}\label{lem:Ind}
	Let $\kappa$ be a regular cardinal and let $\Cc$ be a small $\infty$-category.
	\begin{alphanumerate}
		\item A presheaf $E\in\PSh(\Cc)$ belongs to $\cat{Ind}_\kappa(\Cc)$ if and only if $E$ can be written as a $\kappa$-filtered colimit of representable presheaves. Furthermore, $\cat{Ind}_\kappa(\Cc)\subseteq \PSh(\Cc)$ is closed under $\kappa$-filtered colimits.\label{enum:IndGeneratedUnderFilteredColimits}
		\item If $\Cc$ has $\kappa$-small colimits, then a presheaf $E\in\PSh(\Cc)$ belongs to $\cat{Ind}_\kappa(\Cc)$ if and only if $E\colon \Cc^\op\rightarrow\cat{An}$ preserves $\kappa$-small limits. \label{enum:IndLimits}
		\item If $\Dd$ is an $\infty$-category which has all $\kappa$-filtered colimits, then restriction along the Yoneda embedding induces an equivalence\label{enum:IndFreelyGenerated}
		\begin{equation*}
			\Yo_\Cc^*\colon \Fun^{\kappa\mhyph\mathrm{filt}}\bigl(\cat{Ind}_\kappa(\Cc),\Dd\bigr)\overset{\simeq}{\longrightarrow}\Fun(\Cc,\Dd)\,.
		\end{equation*}
		Here $\Fun^{\kappa\mhyph\mathrm{filt}}(\cat{Ind}_\kappa(\Cc),\Dd)\subseteq\Fun(\cat{Ind}_\kappa(\Cc),\Dd)$ is spanned by those functors that preserve $\kappa$-filtered colimits.
	\end{alphanumerate}
\end{lem}
\begin{proof}
	We begin with \cref{enum:IndGeneratedUnderFilteredColimits}. By \cref{lem:PresheafColimitOfRepresentables}, every presheaf $E$ can be written as a colimit of representables, with $\Cc_{/E}$ as indexing $\infty$-category. If $E\in\cat{Ind}_\kappa(\Cc)$, then $\Cc_{/E}$ is $\kappa$-filtered by \cref{con:Ind}, hence $E$ is a $\kappa$-filtered colimit of representables. Conversely, assume $E$ can be written as such a $\kappa$-filtered colimit, say, $E\simeq\colimit_{j\in\Jj}\Hom_\Cc(-,x_j)$. Since the unstraightening $\operatorname{Un}^{\mathrm{right}}\colon \PSh(\Cc)\rightarrow\cat{Right}(\Cc)$ is an equivalence of $\infty$-categories, it preserves colimits. Recall from \cref{lem:KanExtensionForRight}\cref{enum:RightCofinalLeftAdjoint} that the inclusion $\cat{Right}(\Cc)\subseteq\Cat_{\infty/\Cc}$ has a left adjoint $c\colon \Cat_{\infty/\Cc}\rightarrow\cat{Right}(\Cc)$. Furthermore, by the dual of \cref{lem:ColimitsInSliceCategory}, $\cat{Cat}_{\infty/\Cc}\rightarrow\operatorname{Cat}_\infty$ preserves colimits. Hence a colimit in $\cat{Right}(\Cc)$ is computed by taking the colimit in $\cat{Cat}_\infty$ and then applying $c$. Therefore
	\begin{equation*}
		\operatorname{Un}^{\mathrm{right}}(E)\simeq c\Bigl(\colimit_{j\in\Jj}\Cc_{/x_j}\Bigr)\,.
	\end{equation*}
	Now a $\kappa$-filtered colimit of $\kappa$-filtered $\infty$-categories is $\kappa$-filtered again, which follows by combining \cref{lem:ColimitManipulations}\cref{claim:AssembleColimits} with the characterisation of $\kappa$-filteredness from \cref{lem:FilteredColimitsPreserveFiniteLimits}. So $\colimit_{j\in\Jj}\Cc_{/x_j}$ is $\kappa$-filtered. By \cref{lem:FilteredColimitsPreserveFiniteLimits} again it's clear that being $\kappa$-filtered is preserved under coinitial functors. Since $\colimit_{j\in\Jj}\Cc_{/x_j}\rightarrow c(\colimit_{j\in\Jj}\Cc_{/x_j})$ is coinitial by \cref{lem:KanExtensionForRight}\cref{enum:RightCofinalLeftAdjoint}, we've shown that $\operatorname{Un}^{\mathrm{right}}(E)$ is $\kappa$-filtered, as desired. The same argument shows that $\cat{Ind}_\kappa(\Cc)\subseteq \PSh(\Cc)$ is closed under $\kappa$-filtered colimits.
	
	For \cref{enum:IndLimits}, let's temporarily denote $\PSh^\kappa(\Cc)\subseteq\PSh(\Cc)$ the full sub-$\infty$-category of presheaves $E\colon \Cc^\op\rightarrow\cat{An}$ that preserve $\kappa$-small limits. Every representable presheaf preserves all limits by \cref{cor:HomPreservesLimits}, in particular, $\kappa$-small ones. By \cref{lem:FilteredColimitsPreserveFiniteLimits}, $\PSh^\kappa(\Cc)\subseteq \PSh(\Cc)$ is stable under $\kappa$-filtered colimits. By \cref{enum:IndGeneratedUnderFilteredColimits}, every $E\in\cat{Ind}_\kappa(\Cc)$ is a $\kappa$-filtered colimit of representables, hence $E\in \PSh^\kappa(\Cc)$. Conversely, assume $E\in\PSh^\kappa(\Cc)$. To show that $\Cc_{/E}$ is $\kappa$-filtered, we claim:
	\begin{alphanumerate}\itshape
		\item[\boxtimes] The restricted Yoneda embedding $\Yo_\Cc\colon \Cc\rightarrow\PSh^\kappa(\Cc)$ preserves $\kappa$-small colimits.\footnote{Note that this is completely false for the unrestricted Yoneda embedding: $\Yo_\Cc\colon \Cc\rightarrow \PSh(\Cc)$ preserves limits (by \cref{cor:HomPreservesLimits}), but not colimits. That is why, whenever we want to choose a fully faithful colimit-preserving functor $i\colon \Cc\rightarrow \Cc'$ into an $\infty$-category with all colimits, we have to take the awkward construction $(\Yo_{\Cc^\op})^\op\colon (\Cc^\op)^\op\rightarrow \PSh(\Cc^\op)^\op$.}\label{claim:YonedaPreservesColimits}
	\end{alphanumerate}
	If $\alpha\colon\Ii\rightarrow \Cc_{/E}$ is any functor from an essentially $\kappa$-small $\infty$-category, and $\ov\alpha\colon \Ii\rightarrow\Cc_{/E}\rightarrow\Cc$ denotes the composition of $\alpha$ with the projection to $\Cc$, then $x\coloneqq \colimit(\ov\alpha\colon\Ii\rightarrow\Cc)$ exists by assumption on $\Cc$. Now $\alpha$ corresponds to a natural transformation $\eta\colon\Yo_\Cc\circ\ov\alpha\Rightarrow\const E$ in $\Fun(\Ii,\PSh(\Cc))$. If \cref{claim:YonedaPreservesColimits} holds, then we can use the universal property of colimits to show that $\eta$ factors uniquely through $\Yo_\Cc\circ\ov\alpha\Rightarrow\const\Yo_\Cc(x)$. Since $\Yo_\Cc$ is fully faithful, this yields an extension $\alpha^\triangleright\colon \Ii\rightarrow\Cc_{/E}$, as desired.
	
	To prove \cref{claim:YonedaPreservesColimits}, let $x_{(-)}\colon \Ii\rightarrow\Cc$ be a functor from an essentially $\kappa$-small $\infty$-category $\Cc$ and let $E\in\PSh^\kappa(\Cc)$. Then $\Hom_{\PSh(\Cc)}(\Yo_\Cc(\colimit_{i\in\Ii}x_i),E)\simeq E(\colimit_{i\in\Ii}x_i)$ by Yoneda's lemma. Since $E$ preserves $\kappa$-small limits (and limits in $\Cc^\op$ correspond to colimits in $\Cc$), we can use Yoneda's lemma again to see
	\begin{equation*}
		E\Bigl(\colimit_{i\in\Ii}x_i\Bigr)\simeq \limit_{i\in\Ii}E(x_i)\simeq \limit_{i\in\Ii}\Hom_{\PSh(\Cc)}\bigl(\Yo_\Cc(x_i),E\bigr)\,.
	\end{equation*}
	Using \cref{cor:HomPreservesColimits}, this proves \cref{claim:YonedaPreservesColimits}.
	
	To prove \cref{enum:IndFreelyGenerated}, we can more or less copy the proof of \cref{thm:PShFreeCocompletion}: Let $F\colon \Cc\rightarrow\Dd$ be any functor. Since $\Dd$ has filtered colimits and $\Cc_{/E}$ is filtered for every $E\in\cat{Ind}_\kappa(\Cc)$, the Kan extension $\Lan_{\Yo_\Cc^\kappa}F\colon \cat{Ind}_\kappa(\Cc)\rightarrow\Dd$ exists by \cref{lem:KanExtensionFormula}. We must show that the Kan extension $\Lan_{\Yo_\Cc^\kappa}F$ preserves $\kappa$-filtered colimits. Let's first assume that $\Dd$ has all colimits. Consider the Kan extension $\Lan_{\Yo_\Cc}F\colon \PSh(\Cc)\rightarrow\Dd$. For formal reasons, $\Lan_{\Yo_\Cc}F$ is the left Kan extension of $\Lan_{\Yo_\Cc^\kappa}F$ along $\cat{Ind}_\kappa(\Cc)\subseteq\PSh(\Cc)$. Since the latter is fully faithful, \cref{cor:KanExtensionAlongFullyFaithful} shows $\Lan_{\Yo_\Cc^\kappa}F\simeq(\Lan_{\Yo_\Cc}F)|_{\cat{Ind}_\kappa(\Cc)}$. Now $\Lan_{\Yo_\Cc}F\colon\PSh(\Cc)\rightarrow\Dd$ preserves colimits by \cref{lem:LanAlongYonedaHasRightAdjoint} and $\cat{Ind}_\kappa(\Cc)\subseteq \PSh(\Cc)$ preserves $\kappa$-filtered colimits by \cref{enum:IndGeneratedUnderFilteredColimits}, so $\Lan_{\Yo_\Cc^\kappa}F$ preserves $\kappa$-filtered colimits, as desired. This concludes the case where $\Dd$ has all colimits. The general case can be reduced to this as follows: As in the proof of \cref{lem:ColimitsInSliceCategory}, we can choose a fully faithful colimit-preserving functor $i\colon \Dd\rightarrow\Dd'$ into an $\infty$-category with all colimits. The formula from \cref{lem:KanExtensionFormula} combined with \cref{thm:EquivalencePointwise} show that the canonical natural transformation $\Lan_{\Yo_\Cc^\kappa}(i\circ F)\Rightarrow i\circ \Lan_{\Yo_\Cc^\kappa}F$ is an equivalence, and so it suffices that $\Lan_{\Yo_\Cc^\kappa}(i\circ F)$ preserves $\kappa$-filtered colimits, which we did above.
	
	So the Kan extension functor $\Lan_{\Yo_\Cc^\kappa}\colon \Fun(\Cc,\Dd)\rightarrow\Fun(\cat{Ind}_\kappa(\Cc),\Dd)$ lands in the full sub-$\infty$-category $\Fun^\kappa(\cat{Ind}_\kappa(\Cc),\Dd)$. Therefore, we obtain an adjunction
	\begin{equation*}
		\Lan_{\Yo_\Cc}\colon \Fun(\Cc,\Dd)\doublelrmorphism \Fun^{\kappa\mhyph\mathrm{filt}}\bigl(\cat{Ind}_\kappa(\Cc),\Dd\bigr)\noloc \Yo_\Cc^*
	\end{equation*}
	By the same arguments as in the proof of \cref{thm:PShFreeCocompletion}, the unit $u\colon \id_{\Fun(\Cc,\Dd)}\Rightarrow\Yo_\Cc^*\circ \Lan_{\Yo_\Cc}$ is an equivalence and $\Yo_\Cc^*$ is conservative, hence the adjunction above is a pair of inverse equivalences by \cref{lem:FullyFaithfulConservativeAdjunction}.
\end{proof}
It's surprisingly common for an $\infty$-category to be accessible.% in the sense of \cref{def:Presentable}.
\begin{lem}\label{lem:KappaCompactlyGenerated}
	Let $\kappa$ be a regular cardinal and let $\Dd$ be a locally small $\infty$-category. Then the following are equivalent:
	\begin{alphanumerate}
		\item $\Dd$ is of the form $\Dd\simeq \cat{Ind}_\kappa(\Cc)$ for some essentially small $\infty$-category $\Cc$.\label{enum:DIsIndC}
		\item $\Dd$ admits $\kappa$-filtered colimits and there exists a set $S$ of $\kappa$-compact objects such that every object from $\Dd$ can be written as a $\kappa$-filtered colimit of objects from $S$.\label{enum:DGeneratedUnderFilteredColimits}
	\end{alphanumerate}
	In this case automatically $\Dd\simeq \cat{Ind}_\kappa(\Dd^\kappa)$, where $\Dd^\kappa\subseteq \Dd$ is the full sub-$\infty$-category spanned by the $\kappa$-compact objects. Furthermore, if $\Dd$ has $\kappa$-small colimits, then there is another equivalent condition:
	\begin{alphanumerate}[resume]
		\item $\Dd$ admits $\kappa$-filtered colimits and has a set of $\kappa$-compact generators; that is, a set $S\subseteq \Dd$ of $\kappa$-compact objects such that $\Hom_\Dd(s,-)\colon \Dd\rightarrow\cat{An}$, $s\in S$, are jointly conservative.\label{enum:CompactGenerators}
	\end{alphanumerate}%
	In the case where $\Dd$ has $\kappa$-small colimits and \cref{enum:CompactGenerators} holds, $\Dd$ is automatically presentable.
\end{lem}
\begin{proof}
	Assume \cref{enum:DIsIndC} is true. Then $\Dd$ has $\kappa$-filtered colimits by \cref{lem:Ind}\cref{enum:IndGeneratedUnderFilteredColimits}. We claim that $S\coloneqq\{\Yo_\Cc(x)\ \vert\ x\in\Cc\}$ generates $\Dd$ under $\kappa$-filtered colimits and forms a set of compact generators in the sense of \cref{enum:CompactGenerators}. The first assertion is \cref{lem:Ind}\cref{enum:IndGeneratedUnderFilteredColimits}. For the second assertion, Yoneda's lemma says $\Hom_{\PSh(\Cc)}(\Yo_\Cc(x),E)\simeq E(x)$ for every $E\in\cat{Ind}_\kappa(\Cc)$. Hence $\Hom_{\PSh(\Cc)}(\Yo_\Cc(x),-)$ for $x\in\Cc$ are jointly conservative by \cref{thm:EquivalencePointwise}. Furthermore, $\Yo_\Cc(x)$ is $\kappa$-compact since colimits in presheaf $\infty$-categories are computed pointwise by \cref{lem:ColimitsInFunctorCategories} and $\cat{Ind}_\kappa(\Cc)\subseteq\PSh(\Cc)$ preserves $\kappa$-filtered colimits by \cref{lem:Ind}\cref{enum:IndGeneratedUnderFilteredColimits}. This proves \cref{enum:DIsIndC} $\Rightarrow$ \cref{enum:DGeneratedUnderFilteredColimits} and \cref{enum:DIsIndC} $\Rightarrow$ \cref{enum:CompactGenerators} (even without the assumption that $\Dd$ has $\kappa$-small colimits).
	
	Now assume \cref{enum:DGeneratedUnderFilteredColimits}. Let's first sketch why $\Dd^\kappa$ is essentially small. We'll show that every $x\in\Dd^\kappa$ is a retract of some $s\in S$ and then leave it to you to verify that $S$ can't have \enquote{too many} retracts in the locally small $\infty$-category $\Dd$. Write $x\simeq \colimit_{j\in\Jj}s_j$ for some $s_j\in S$ and some $\kappa$-filtered $\infty$-category $\Jj$. Since $x$ is $\kappa$-compact and $\pi_0$ commutes with colimits by \cref{lem:HomotopyGroupsFilteredColimits}, we get $\colimit_{j\in\Jj}\pi_0\Hom_\Cc(x,s_j)\cong \pi_0\Hom_\Dd(x,x)$. Choosing a preimage of $\id_x$ yields a morphism $x\rightarrow s_j$ for some $j\in\Jj$, which exhibits $x$ as a retract of $s_j$, as desired.
	
	By \cref{lem:Ind}\cref{enum:IndFreelyGenerated}, the inclusion $\Dd^\kappa\subseteq \Dd$ extends uniquely to a functor $L^\kappa\colon \cat{Ind}_\kappa(\Dd^\kappa)\rightarrow\Dd$ that preserves $\kappa$-filtered colimits. Let's first construct a right adjoint $R^\kappa$. To this end, choose a fully faithful colimit-preserving functor $i\colon \Dd\rightarrow\Dd'$ into an $\infty$-category $\Dd'$ with all colimits; this can be done as in the proof of \cref{lem:Ind}\cref{enum:IndFreelyGenerated}. Furthermore, $i\circ L^\kappa$ extends uniquely to a colimit-preserving functor $L\colon \PSh(\Dd^\kappa)\rightarrow \Dd'$, which has a right adjoint $R$ by \cref{thm:PShFreeCocompletion}. We claim that $R\circ i\colon \Dd\rightarrow\PSh(\Dd)$ lands in $\cat{Ind}_\kappa(\Dd^\kappa)$. Indeed, let $y\in\Dd$ and write $y\simeq \colimit_{j\in\Jj}x_j$ where $\Jj$ is $\kappa$-filtered and $x_j\in\Dd^\kappa$; we could even choose $x_j\in S$. By the formula from \cref{lem:LanAlongYonedaHasRightAdjoint}, $R(i(y))$ is the presheaf $\Hom_\Dd(-,\colimit_{j\in\Jj}x_j)\colon (\Dd^\kappa)^\op\rightarrow\cat{An}$. By definition of $\Dd^\kappa$, this presheaf agrees with $\colimit_{j\in\Jj}\Hom_\Dd(-,x_j)$. Hence $R(i(y))$ is a $\kappa$-filtered colimit of representable presheaves and thus contained in $\cat{Ind}_\kappa(\Dd^\kappa)$ by \cref{lem:Ind}\cref{enum:IndGeneratedUnderFilteredColimits}. Thus, putting $R^\kappa\coloneqq R\circ i$, we obtain the desired adjunction $L^\kappa\colon \cat{Ind}_\kappa(\Dd^\kappa)\shortdoublelrmorphism\Dd\noloc R^\kappa$. Moreover, our argument shows that $R^\kappa$ commutes with $\kappa$-filtered colimits of objects from $\Dd^\kappa$. By inspection, the counit $c\colon L^\kappa\circ R^\kappa\Rightarrow \id_\Dd$ is an equivalence for objects from $\Dd^\kappa$. Since both sides commute with $\kappa$-filtered colimits of objects from $\Dd^\kappa$ and every object of $\Dd$ can be written as such a colimit, we see that $c$ is an equivalence. An analogous argument shows that $u\colon \id_{\cat{Ind}_\kappa(\Dd^\kappa)}\Rightarrow R^\kappa\circ L^\kappa$ is an equivalence. This proves \cref{enum:DGeneratedUnderFilteredColimits} $\Rightarrow$ \cref{enum:DIsIndC}.
	
	It remains to show \cref{enum:CompactGenerators} $\Rightarrow$ \cref{enum:DIsIndC}. First observe that if $\Dd$ has $\kappa$-small and $\kappa$-filtered colimits, then $\Dd$ has all colimits. Indeed, according to \cref{lem:ColimitsIffCoproductsAndPushouts}, we only need to check that $\Dd$ has arbitrary coproducts. This follows from the following claim:
	\begin{alphanumerate}\itshape
		\item[\boxtimes] Let $T$ be a discrete set and let $\Pp^{\kappa}(T)\subseteq \Pp(T)$ be the partially ordered set of all subsets $S\subseteq T$ of cardinality $\abs*{S}<\kappa$. Then $\Pp^{\kappa}(T)$ is $\kappa$-filtered and for every collection $(x_t)_{t\in T}$ of objects of $\Dd$ we have\label{claim:FilteredCoproduct}
		\begin{equation*}
			\colimit_{S\in\Pp^{\kappa}(T)}\coprod_{s\in S}x_s\overset{\simeq}{\longrightarrow}\coprod_{t\in T}x_t\,.
		\end{equation*}
	\end{alphanumerate}
	Using \cref{lem:SimplicialHoNerveAdjunction}, $\kappa$-filteredness of $\Pp^\kappa(T)$ reduces to a question about ordinary categories, which is easy. Now consider the tautological functor $U\colon \Pp^\kappa(T)\rightarrow\cat{Set}$ sending $S\mapsto S$ and let $\Uu$ be its unstraightening, which is an ordinary category and easy to describe.\footnote{Here we use that for functors into $\cat{Set}$ or $\cat{Grpd}^{(2)}$, Lurie's unstraightening recovers the \emph{Grothendieck construction} from classical category theory. We've seen something similar in \cref{par:Stacks}; in particular, compare the description of the unstraightening of $U\colon \Pp^\kappa(T)\rightarrow \cat{Set}$ to the unstraightening of $[S/G]\colon (\cat{Sch}_{/S})^\op\rightarrow \cat{Grpd}^{(2)}$.} Namely, $\Uu$ is the category of pairs $(S,s)$, where $S\subseteq T$ is a subset and $s\in S$ is an element. Morphisms can be described as follows: Fix $(S,s)$ and $(S',s')$. If $S\subseteq S'$ and $s=s'$, there exists a unique morphism $(S,s)\rightarrow(S',s')$ in $\Uu$; otherwise $\Hom_\Uu((S,s),(S',s'))=\emptyset$. There is a tautological natural transformation $U\Rightarrow\const T$, which induces a functor $\Uu\rightarrow \Pp^\kappa(T)\times T$ on unstraightenings. Note that $\Uu\rightarrow \Pp^\kappa(T)\times T\rightarrow T$ is coinitial. Indeed, for every $t\in T$, the slice $\Uu\times_TT_{t/}$ can be identified with the sub-partially ordered set $\Pp_t^\kappa(T)\subseteq \Pp^\kappa(T)$ of those $S$ such that $t\in S$. Then $\Pp_t^\kappa(T)$ has an initial object, namely $\{t\}$, and so $\mathopen|\Pp_t^\kappa(T)\mathclose|\simeq *$, whence \cref{thm:JoyalsQuillenA}\cref{enum:WeaklyContractible} is satisfied. Thus, if $T\rightarrow \Dd$ corresponds to the collection $(x_t)_{t\in T}$, then $\colimit(T\rightarrow\Dd)\simeq \colimit(\Uu\rightarrow T\rightarrow\Dd)$. Using \cref{lem:ColimitManipulations}\cref{claim:SliceColimits}, the right-hand side can be identified with $\colimit_{S\in\Pp^\kappa(T)}\coprod_{s\in S}x_s$. This finishes the proof of \cref{claim:FilteredCoproduct}
	
	Now let $S\subseteq \Dd$ be a set of $\kappa$-compact generators and let $\Cc\subseteq \Dd$ be the full sub-$\infty$-category generated by $S$ under $\kappa$-small colimits. Since $\Dd$ is locally small, one can verify that $\Cc$ is essentially small (there are \enquote{not too many} $\kappa$-small diagrams); we leave this to you. Since $\Dd$ has all colimits, we can apply \cref{thm:PShFreeCocompletion} to see that $\Cc\subseteq\Dd$ extends uniquely to a colimit-preserving functor $L\colon \PSh(\Cc)\rightarrow\Dd$, which has a right adjoint $R$. Observe that $R$ factors through $\cat{Ind}_\kappa(\Cc)$. Indeed, according to \cref{lem:LanAlongYonedaHasRightAdjoint}, for every $y\in\Dd$, the presheaf $R(y)$ is given by $\Hom_\Dd(-,y)\colon\Cc^\op\rightarrow\cat{An}$. This functor preserves arbitrary limits by \cref{cor:HomPreservesLimits}, in particular, $\kappa$-small ones, and so \cref{lem:Ind}\cref{enum:IndLimits} implies $R(y)\in\cat{Ind}_\kappa(\Cc)$. Restricting $L$, we thus obtain an adjunction $L\colon\cat{Ind}_\kappa(\Cc)\shortdoublelrmorphism\Dd\noloc R$. Observe that $R$ preserves $\kappa$-filtered colimits. Indeed, let $y_{(-)}\colon \Jj\rightarrow\Dd$ be a functor from a $\kappa$-filtered $\infty$-category. By \cref{thm:EquivalencePointwise} and \cref{lem:ColimitsInFunctorCategories}, it suffices to show that $\colimit_{j\in\Jj}\Hom_\Dd(x,y_j)\rightarrow \Hom_\Dd(x,\colimit_{j\in\Jj}y_j)$ is an equivalence for all $x\in\Cc$. But \cref{lem:FilteredColimitsPreserveFiniteLimits} easily implies that $\kappa$-compact objects are closed under $\kappa$-small colimits and so $x$ must be $\kappa$-compact, whence we get an equivalence as desired. Now we can apply the same argument as in the proof of \cref{enum:DGeneratedUnderFilteredColimits} $\Rightarrow$ \cref{enum:DIsIndC} to show that the unit $u\colon \id_{\cat{Ind}_\kappa(\Cc)}\Rightarrow R\circ L$ is an equivalence. Furthermore, $R$ is conservative. Indeed, if $\alpha\colon y\rightarrow z$ in $\Dd$ induces an equivalence $\alpha_*\colon \Hom_\Dd(-,y)\Rightarrow \Hom_\Dd(-,z)$ of presheaves, then, in particular, $\alpha_*\colon\Hom_\Dd(s,y)\rightarrow\Hom_\Dd(s,z)$ must be an equivalence for all $s\in S$. But $\Hom_\Dd(s,-)\colon \Dd\rightarrow\cat{An}$ for $s\in S$ are jointly conservative by assumption. Now \cref{lem:FullyFaithfulConservativeAdjunction}\cref{enum:Conservative} finishes the proof of the implication \cref{enum:CompactGenerators} $\Rightarrow$ \cref{enum:DIsIndC}.
\end{proof}
This finishes our discussion of accessibility. Next, we'll characterise presentable $\infty$-categories.

\begin{lem}\label{lem:Presentable}
	For a locally small $\infty$-category $\Dd$, the following are equivalent:
	\begin{alphanumerate}
		\item $\Dd$ is presentable.\label{enum:DIsPresentable}
		\item $\Dd$ is $\kappa$-accessible and has $\kappa$-small colimits for some regular cardinal $\kappa$.\label{enum:DHasKappaSmallColimits}
		\item \!There exists an essentially small $\infty$-category $\Cc$ and an adjunction $L\colon \PSh(\Cc)\shortdoublelrmorphism \Dd\noloc R$ such that $R$ is fully faithful and preserves $\kappa$-filtered colimits for some regular cardinal $\kappa$.\label{enum:AccessibleLocalisation}
		\item $\Dd$ is of the form $\Dd\simeq\cat{Ind}_\kappa(\Cc)$ for some essentially small $\infty$-category $\Cc$ which has $\kappa$-small colimits.\label{enum:DCHasKappaSmallColimits}
	\end{alphanumerate}
	In this case, $\Dd^\kappa$ automatically has all $\kappa$-small colimits \embrace{so that we may choose $\Cc\simeq \Dd^\kappa$ in \cref{enum:DCHasKappaSmallColimits} by \cref{lem:KappaCompactlyGenerated}}.
\end{lem}
\begin{proof}
	The implication \cref{enum:DIsPresentable} $\Rightarrow$ \cref{enum:DHasKappaSmallColimits} is trivial and \cref{enum:DIsPresentable} $\Rightarrow$ \cref{enum:AccessibleLocalisation} follows from the proof of \cref{lem:KappaCompactlyGenerated}. For \cref{enum:DHasKappaSmallColimits} $\Rightarrow$ \cref{enum:DCHasKappaSmallColimits}, we use \cref{lem:KappaCompactlyGenerated} to see that $\Dd\simeq \cat{Ind}_\kappa(\Dd^\kappa)$. It follows easily from \cref{lem:FilteredColimitsPreserveFiniteLimits} and \cref{cor:HomPreservesLimits} that $\kappa$-compact objects are closed under $\kappa$-small colimits. Therefore, if $\Dd$ has all $\kappa$-small colimits, then so has $\Dd^\kappa$. This proves \cref{enum:DHasKappaSmallColimits} $\Rightarrow$ \cref{enum:DCHasKappaSmallColimits}.
	
	For \cref{enum:AccessibleLocalisation} $\Rightarrow$ \cref{enum:DIsPresentable}, first note that $\Dd$ has all colimits. Indeed, given a diagram $\alpha\colon \Ii\rightarrow\Dd$, we can form the colimit $c\simeq\colimit_{i\in\Ii}R(\alpha(i))$ in $\PSh(\Cc)$. Since $L$ preserves colimits by \cref{lem:AdjointsPreserveColimits} and $L\circ R\simeq \id_\Dd$ by the dual of \cref{lem:FullyFaithfulConservativeAdjunction}\cref{enum:FullyFaithfulIffUnitEquivalence}, we see that $L(c)\simeq \colimit_{i\in\Ii}\alpha(i)$, as desired. Now consider the objects $L(\Yo_\Cc(x))$, where $x$ runs through a set of representatives for every equivalence class in $\Cc$. Then $\Hom_\Dd(L(\Yo_\Cc(x)),-)\simeq \Hom_{\PSh(\Cc)}(\Yo_\Cc(x),R(-))$. By Yoneda's lemma, $\Hom_{\PSh(\Cc)}(\Yo_\Cc(x),-)\colon \PSh(\Cc)\rightarrow\cat{An}$ are jointly conservative, and they preserve all colimits by \cref{lem:ColimitsInFunctorCategories}. Since $R$ is fully faithful and preserves $\kappa$-filtered colimits, it follows that $\Hom_\Dd(L(\Yo_\Cc(x)),-)\colon \Dd\rightarrow\cat{An}$ are jointly conservative and preserve $\kappa$-filtered colimits. So the set $\{L(\Yo_\Cc(x))\}$ satisfies the conditions from \cref{lem:KappaCompactlyGenerated}\cref{enum:CompactGenerators} and it follows that $\Dd$ is presentable. This proves \cref{enum:AccessibleLocalisation} $\Rightarrow$ \cref{enum:DIsPresentable}
	
	It remains to show \cref{enum:DCHasKappaSmallColimits} $\Rightarrow$ \cref{enum:DIsPresentable}. We need to show that $\cat{Ind}_\kappa(\Cc)$ has all colimits. We know from \cref{lem:Ind}\cref{enum:IndGeneratedUnderFilteredColimits} that $\cat{Ind}_\kappa(\Cc)$ has $\kappa$-filtered colimits, so by claim~\cref{claim:FilteredCoproduct} in the proof of \cref{lem:KappaCompactlyGenerated}, it's enough to show that $\cat{Ind}_\kappa(\Cc)$ has $\kappa$-small colimits. By \cref{lem:KappaSmallColimits}, it's enough to construct pushouts and $\kappa$-small coproducts. Also, we've seen in the proof of \cref{lem:Ind} that $\Yo_\Cc\colon \Cc\rightarrow\cat{Ind}_\kappa(\Cc)$ preserves $\kappa$-small colimits. Since $\Cc$ itself has all $\kappa$-small colimits by assumption, we see that $\cat{Ind}_\kappa(\Cc)$ has $\kappa$-small colimits of representable presheaves.
	
	Let's first construct the coproduct $\coprod_{s\in S}y_s$ for a discrete set $S$ of cardinality $\abs*{S}<\kappa$ and $y_s\in\cat{Ind}_\kappa(\Cc)$. Write $y_s\simeq \colimit_{j\in \Jj_s}x_{j,s}$ for some filtered $\infty$-category $\Jj_s$ and representable presheaves $x_{j,s}$. Observe that arbitrary products of $\kappa$-filtered $\infty$-categories is $\kappa$-filtered again. Furthermore, if $\Ii$ is any $\kappa$-filtered $\infty$-category, then $\colimit_{i\in\Ii}y_s\simeq y_s$ by \cref{lem:ContractibleColimit} and \cref{lem:FilteredCofinal}. Hence  $\colimit_{(i,j)\in \Ii\times\Jj_s}x_{j,s}\simeq y_s$ by \cref{lem:ColimitManipulations}. So we may replace $\Jj_s$ by $\Ii\times\Jj_s$ for any $\kappa$-filtered $\Ii$. In particular, we may replace $\Jj_s$ by $\Jj\coloneqq \prod_{s\in S}\Jj_s$ and thus we may assume that the diagrams $\Jj_s$ coincide for all $s\in S$. Then \cref{lem:ColimitManipulations} shows
	\begin{equation*}
		\colimit_{j\in\Jj}\coprod_{s\in S}x_{j,s}\simeq \coprod_{s\in S}\colimit_{j\in\Jj}x_{j,s}\simeq \coprod_{s\in S}y_s\,,
	\end{equation*}
	provided any of these colimits exists. But $\coprod_{s\in S}x_{j,s}$ exists for all $j\in \Jj$ because $\cat{Ind}_\kappa(\Cc)$ has $\kappa$-small coproducts of representable presheaves, and then $\colimit_{j\in\Jj}\coprod_{s\in S}x_{j,s}$ exists because $\cat{Ind}_\kappa(\Cc)$ has all $\kappa$-filtered colimits. This shows that $\cat{Ind}_\kappa(\Cc)$ has $\kappa$-small coproducts.
	
	It remains to construct pushouts. Fix a span $y\leftarrow x\rightarrow z$. Let's first construct the pushout in the case where $x$ is representable. Write $y\simeq \colimit_{j\in\Jj} y_j$ and $z\simeq \colimit_{k\in\Kk}z_k$, where $\Jj$ and $\Kk$ are $\kappa$-filtered and $y_j$, $z_k$ are representable presheaves. Since $x$ is representable and thus $\kappa$-compact, \cref{lem:HomotopyGroupsFilteredColimits} implies  $\pi_0\Hom_{\cat{Ind}_\kappa(\Cc)}(x,y)\simeq \colimit_{j\in\Jj}\pi_0\Hom_{\cat{Ind}_\kappa(\Cc)}(x,y_j)$. Hence $x\rightarrow y$ factors through $x\rightarrow y_{j_0}$ for some $j_0\in \Jj$. By \cref{lem:FilteredCofinal}, we can replace $\Jj$ by $\Jj_{j_0/}$ and thus assume that $\Jj$ contains an initial element $j_0$ such that $x\rightarrow y$ is induced by a map $x\rightarrow y_{j_0}$. The same argument applies to $x\rightarrow z$. Furthermore, as above, we can replace $\Jj$ and $\Kk$ by $\Jj\times \Kk$ and thus assume $\Jj=\Kk$. Finally, we have $\colimit_{j\in\Jj}x\simeq x$ by \cref{lem:ContractibleColimit} and \cref{lem:FilteredCofinal}. Hence, using \cref{lem:ColimitManipulations}, we can construct the desired pushout as
	\begin{equation*}
		\colimit_{j\in\Jj}(y_j\sqcup_xz_j)\simeq \colimit_{j\in \Jj}y_j\sqcup_{\colimit_{j\in\Jj}x}\colimit_{j\in\Jj}z_j\simeq y\sqcup_xz\,.
	\end{equation*}
	Here $y_j\sqcup_xz_j$ exists since $\cat{Ind}_\kappa(\Cc)$ has pushouts of representable presheaves, as we've noted above, and then $\colimit_{j\in\Jj}(y_j\sqcup_xz_j)$ exists because $\cat{Ind}_\kappa(\Cc)$ has $\kappa$-filtered colimits. This finishes the case where $x$ is representable. In the general case, write $x\simeq\colimit_{j\in\Jj}x_j$, where $\Jj$ is $\kappa$-filtered and $x_j$ are representable presheaves. By an argument we've seen several times, $y\simeq \colimit_{j\in\Jj}y$ and $z\simeq \colimit_{j\in\Jj}z$. Then
	\begin{equation*}
		\colimit_{j\in\Jj}(y\sqcup_{x_j}z)\simeq \colimit_{j\in \Jj}y\sqcup_{\colimit_{j\in\Jj}x_j}\colimit_{j\in\Jj}z\simeq y\sqcup_xz\,.
	\end{equation*}
	Here the pushouts $y\sqcup_{x_j}z$ exist by the representable case and then $\colimit_{j\in\Jj}(y\sqcup_{x_j}z)$ exists because $\cat{Ind}_\kappa(\Cc)$ has $\kappa$-filtered colimits. This finishes the proof that $\cat{Ind}_\kappa(\Cc)$ has pushouts.
\end{proof}


\begin{cor}\label{cor:AnPresentable}
	The $\infty$-categories $\cat{An}$ and $\cat{Cat}_\infty$ are presentable.
\end{cor}
\begin{proof}[Proof sketch]
	It's clear that $\cat{An}$ and $\cat{Cat}_\infty$ are locally small and they have all colimits by \cref{lem:ColimitsInAnima}. So it suffices to check that both are accessible. In fact, we'll show that both are $\aleph_0$-accessible, by verifying the condition from \cref{lem:KappaCompactlyGenerated}\cref{enum:CompactGenerators}. For $\cat{An}$, it's clear that $*$ is a compact generator as $\Hom_{\cat{An}}(*,X)\simeq X$ for all $X\in\cat{An}$. For $\cat{Cat}_\infty$, we claim that the $\infty$-categories $*$ and $\Delta^1$ are compact generators. 
	
	Let's first argue that $\Hom_{\cat{Cat}_\infty}(*,-)$ and $\Hom_{\cat{Cat}_\infty}(\Delta^1,-)$ are jointly conservative. To this end, recall from \cref{thm:CordierPorter} that $\Hom_{\cat{Cat}_\infty}(*,\Cc)\simeq \core(\Cc)$ and $\Hom_{\cat{Cat}_\infty}(\Delta^1,\Cc)\simeq \core\Ar(\Cc)$ for every $\infty$-category $\Cc$. Now if $F\colon \Cc\rightarrow\Dd$ is a functor such that $\core(F)\colon \core(\Cc)\rightarrow\core(\Dd)$ is an equivalence, then $F$ is essentially surjective. If furthermore $\core \Ar(\Cc)\rightarrow\core\Ar(\Dd)$ is an equivalence, then $F$ is fully faithful. Indeed, for all $x,y\in\Cc$ we can write $\Hom_\Cc(x,y)$ as a pullback of $\core\Ar(\Cc)\rightarrow\core(\Cc)\times\core(\Cc)$ by \cref{par:HomInQuasiCategories} plus the fact that $\core\colon \cat{Cat}_\infty\rightarrow\cat{An}$ preserves pullbacks, since it is a right adjoint by \cref{exm:Adjunctions}\cref{enum:AnToCatInfty}.\footnote{We're also implicitly using that the pullback diagram from \cref{par:HomInQuasiCategories}, which lived in simplicial sets, is also a pullback of $\infty$-categories. See model category fact~\cref{par:HomotopyPushout}.} This proves that $\Hom_{\cat{Cat}_\infty}(*,-)$ and $\Hom_{\cat{Cat}_\infty}(\Delta^1,-)$ are jointly conservative.
	
	We'll only sketch the argument why $*$ and $\Delta^1$ are compact in $\cat{Cat}_\infty$. The crucial observation is that equivalences of quasi-categories are preserved under filtered colimits in the ordinary category $\cat{QCat}$. Indeed, $\cat{QCat}\subseteq \cat{sSet}$ is closed under filtered colimits, because $\Lambda_i^n$ and $\Delta^n$ are finite simplicial sets and so every horn filling problem in a filtered colimit can be solved at some finite stage. So filtered colimits in $\cat{QCat}$ can be computed in $\cat{sSet}$ instead. Then it's straightforward to check that a filtered colimit of fully faithful and essentially surjective maps of quasi-categories is again fully faithful and essentially surjective. Now we can use the same arguments as in the proof of \cref{lem:HomotopyGroupsFilteredColimits} (including the black box \cref{blackbox:Cofinal} and an analogue of \cref{blackbox:Localisation}) to see that filtered colimits in $\cat{Cat}_\infty$ can be computed as ordinary filtered colimits in $\cat{QCat}$. So it remains to show that $ \colimit_{j\in J}\core \F(*,\Cc_j)\cong \core \F(*,\colimit_{j\in\Jj}\Cc_j)$ and $\colimit_{j\in J}\F(\Delta^1,\Cc_j)\cong \core\F(\Delta^1,\colimit_{j\in J}\Cc_j)$ holds for every filtered category $J$ and every diagram $\Cc_{(-)}\colon J\rightarrow\cat{QCat}$. This is straightforward.
\end{proof}
\begin{cor}\label{cor:FunctorCategoriesPresentable}
	If $\Dd$ is a presentable $\infty$-category, then for every $y\in\Dd$ the slice $\infty$-categories $\Dd_{y/}$ and $\Dd_{/y}$ are presentable again. Furthermore, if $\Cc$ is an essentially small $\infty$-category, then $\Fun(\Cc,\Dd)$ is presentable. In particular, $\PSh(\Cc)$ and $\Fun(\Cc,\cat{Cat}_\infty)$ are presentable.
\end{cor}
The same results are true for accessible $\infty$-categories, but this requires significantly more effort. In practice, the results about presentable $\infty$-categories are usually sufficient and so we refer to \cite[\S\href{https://people.math.harvard.edu/~lurie/papers/HTT.pdf\#section.5.4}{5.4}]{HTT} for the accessible case.
\begin{proof}[Proof of \cref{cor:FunctorCategoriesPresentable}]
	It follows from \cref{lem:ColimitsInSliceCategory} and its dual that if $\Dd$ has all colimits, then $\Dd_{y/}$ and $\Dd_{/y}$ have all colimits again. \cref{lem:ColimitsInFunctorCategories} shows the same for $\Fun(\Cc,\Dd)$. So it's enough to check accessibility in each case.
	
	By \cref{lem:KappaCompactlyGenerated}\cref{enum:CompactGenerators}, we can choose a set $S$ of $\kappa$-compact generators for $\Dd$. Since $\Dd$ has coproducts, one easily verifies via \cref{lem:Adjunction} that $\Dd_{y/}\rightarrow\Dd$ has a left adjoint, sending $z\in\Dd$ to $(y\rightarrow y\sqcup z)\in\Dd_{y/}$. Then $\Hom_{\Dd_{y/}}(y\rightarrow y\sqcup s,y\rightarrow z)\simeq \Hom_\Dd(s,z)$ and so $\Hom_{\Dd_{y/}}(y\rightarrow y\sqcup s,-)\colon \Dd_{y/}\rightarrow\cat{An}$ for $s\in S$ are jointly conservative. Using the adjunction property plus the fact that $\Dd_{y/}\rightarrow \Dd$ preserves $\kappa$-filtered colimits by \cref{lem:ColimitsInSliceCategory}\cref{enum:ColimitsInSlice}, we see that every $(y\rightarrow y\sqcup s)$ is $\kappa$-compact again. So $\Dd_{y/}$ satisfies the condition from \cref{lem:KappaCompactlyGenerated}\cref{enum:CompactGenerators} and is therefore accessible.
	
	For $\Fun(\Cc,\Dd)$, consider the functors $F_{x,s}\coloneqq\Lan_{\{x\}\rightarrow \Cc}(\const s)\colon \Cc\rightarrow\Dd$, where $s\in S$ and $x$ runs through a set of representatives of the equivalence classes of objects in $\Cc$. These Kan extensions exist by \cref{lem:KanExtensionFormula} since $\Dd$ has all colimits. The universal property of Kan extensions shows $\Hom_{\Fun(\Cc,\Dd)}(F_{x,s},G)\simeq \Hom_{\Fun(\{x\},\,\Dd)}(\const s,G|_{\{x\}})\simeq \Hom_\Dd(s,G(x))$ for every functor $G\in\Fun(\Cc,\Dd)$. Since colimits in functor categories are computed pointwise by \cref{lem:ColimitsInFunctorCategories} and $s$ is $\kappa$-compact by assumption, it follows that $F_{x,s}$ is $\kappa$-compact. Since equivalences of functors can be detected pointwise by \cref{thm:EquivalencePointwise}, it follows that $\Hom_{\Fun(\Cc,\Dd)}(F_{x,s},-)\colon \Fun(\Cc,\Dd)\rightarrow\cat{An}$ for $s\in S$ and $x$ running through all equivalence classes in $\Cc$ are jointly conservative. So $\Fun(\Cc,\Dd)$ satisfies the condition from \cref{lem:KappaCompactlyGenerated}\cref{enum:CompactGenerators} and is therefore accessible.
	
	For $\Dd_{/y}$, we will instead verify the condition from \cref{lem:KappaCompactlyGenerated}\cref{enum:DGeneratedUnderFilteredColimits}. First observe that $\Dd^\kappa_{/y}$ is essentially small. Indeed, $\Dd^\kappa_{/y}\simeq \Dd^\kappa\times_\Dd\Dd_{/y}\rightarrow \Dd_{/y}$ is fully faithful, hence $\Dd^\kappa_{/y}$ is locally small, because $\Dd_{/y}$ is locally small by the assumption on $\Dd$ and \cref{cor:HomInSliceCategories}. So its enough to show that $\pi_0\core(\Dd^\kappa_{/y})$ is a set. This follows from $\Dd^\kappa$ being essentially small (as we've seen in the proof of \cref{lem:KappaCompactlyGenerated}) and $\Dd$ being locally small, so that there can't be \enquote{too many} equivalence classes of morphisms $z\rightarrow y$ where $z\in\Dd^\kappa$. Since $\Dd_{/y}\rightarrow\Dd$ preserves arbitrary colimits by the dual of \cref{lem:ColimitsInSliceCategory}\cref{enum:LimitsInSlice} and $\kappa$-filtered colimits are preserved under pullbacks by \cref{lem:FilteredColimitsPreserveFiniteLimits}, we can use \cref{cor:HomInSliceCategories} to show that the objects in $\Dd^\kappa_{/y}$ are $\kappa$-compact in $\Dd_{/y}$. It remains to show that they generate $\Dd_{/y}$ under $\kappa$-filtered colimits. Pick some $(z\rightarrow y)\in\Dd_{/y}$ and write $z\simeq\colimit_{j\in\Jj}z_j$ for some $\kappa$-filtered $\infty$-category $\Jj$ and some diagram $z_{(-)}\colon\Jj\rightarrow\Dd^\kappa$. Composing the colimit transformation $u\colon z_{(-)}\Rightarrow \const z$ with $\const z\Rightarrow \const y$ yields a transformation $z_{(-)}\Rightarrow \const y$, which in turn defines a functor $(z_{(-)}\rightarrow y)\colon \Jj\rightarrow \Dd_{/y}$. Then $(z\rightarrow y)\simeq \colimit_{j\in\Jj}(z_j\rightarrow y)$ in $\Dd_{/y}$, as desired.
\end{proof}

\subsection{The adjoint functor theorem}\label{subsec:AdjointFunctorTheorem}
Finally, we can state and prove the adjoint functor theorem. The original version is of course Lurie's \cite[Corollary~\HTTthm{5.5.2.9}]{HTT}. Our version is slightly more general and is taken from Markus Land's book \cite[Theorems~5.2.2 and~5.2.14]{Land}, who in turn took them from \cite{AdjointFunctorTheorems}.
\begin{satanicthm}[Adjoint functor theorem]\label{thm:AdjointFunctorTheorem}
	Let $F\colon \Cc\rightarrow\Dd$ be a functor between locally small $\infty$-categories.
	\begin{alphanumerate}
		\item Assume that $\Cc$ and $\Dd$ have all colimits and $\Cc$ is generated under colimits by an essentially small sub-$\infty$-category $\Cc_0\subseteq\Cc$. Then $F$ admits a right adjoint $G\colon \Dd\rightarrow\Cc$ if and only if $F$ preserves colimits.\label{enum:AdjointFunctorTheoremLeft}
		\item Assume that $\Cc$ and $\Dd$ have all limits, that $\Cc$ is accessible, and that for every object $y\in\Dd$ there exists a regular cardinal $\kappa_y$ such that $y$ is $\kappa_y$-compact. If there exists a regular cardinal $\kappa$ such that $F$ preserves limits as well as $\kappa$-filtered colimits, then $F$ admits a left adjoint $G\colon \Dd\rightarrow\Cc$. The converse is true as well provided that $\Dd$ is accessible too.\label{enum:AdjointFunctorTheoremRight}
		%\item Assume that $\Dd$ is presentable and that $F$ is fully faithful. Then $F$ admits a left adjoint $G\colon \Dd\rightarrow \Cc$ if and only if $F$ preserves limits and there exists a regular cardinal $\kappa$ such $F$ preserves $\kappa$-filtered colimits.\label{enum:AdjointFunctorTheoremFullSubcategory}
	\end{alphanumerate}
	Furthermore, in both \cref{enum:AdjointFunctorTheoremLeft} and \cref{enum:AdjointFunctorTheoremRight}, the conditions on $\Cc$ and $\Dd$ are automatically satisfied if $\Cc$ and $\Dd$ are presentable.\hfill\smash{\GrothendieckRightDevil}
\end{satanicthm}


Before we embark on the proof of \cref{thm:AdjointFunctorTheorem}, we'll draw a somewhat surprising corollary and discuss a useful supplement.
\begin{cor}\label{cor:PresentableComplete}
	Let $\Cc$ be a locally small $\infty$-category.
	\begin{alphanumerate}
		\item If $\Cc$ has all colimits and is generated under colimits by an essentially small sub-$\infty$-category $\Cc_0\subseteq \Cc$, then $\Cc$ has all limits too. In particular, presentable $\infty$-categories have all limits.\label{enum:PresentableComplete}
		\item If $\Cc$ is accessible and has all limits, then $\Cc$ has all colimits too. In particular, $\Cc$ is presentable.\label{enum:AccessibleCocomplete}
	\end{alphanumerate}
\end{cor}
\begin{proof}
	Let $\Ii$ be an essentially small $\infty$-category. Then $\Fun(\Ii,\Cc)$ is locally small by \cref{rem:FunLocallySmall}. It follows from \cref{lem:ColimitsInFunctorCategories} that $\const\colon \Cc\rightarrow\Fun(\Ii,\Cc)$ preserves all limits and colimits. In the situation of \cref{enum:PresentableComplete}, we may apply \cref{thm:AdjointFunctorTheorem}\cref{enum:AdjointFunctorTheoremLeft} to see that $\const$ has a right adjoint $\limit_\Ii\colon \Fun(\Ii,\Cc)\rightarrow\Cc$, as desired.
	
	In the situation of \cref{enum:AccessibleCocomplete}, we only need to check that every element of $\Fun(\Ii,\Cc)$ is $\tau$-compact for some sufficiently large regular cardinal $\tau$, for then $\const$ will have a left adjoint $\colimit_\Ii\colon \Fun(\Ii,\Cc)\rightarrow\Cc$ by \cref{thm:AdjointFunctorTheorem}\cref{enum:AdjointFunctorTheoremRight}. Say $\Cc$ is $\kappa$-compact. Then every $x\in\Cc$ can be written as a colimit of $\kappa$-compact objects by \cref{lem:KappaCompactlyGenerated}\cref{enum:DGeneratedUnderFilteredColimits}. If $\kappa_x$ is larger than $\kappa$ and the cardinality of the indexing diagram, then $x$ will be $\kappa_x$-compact, because $\kappa_x$-compact objects are closed under $\kappa_x$-small colimits (we've seen this argument several times in the proofs of \cref{lem:KappaCompactlyGenerated,lem:Presentable}). Now let $\alpha\colon \Ii\rightarrow\Cc$ be a functor. Let $\tau_\alpha$ be a regular cardinal such that $\TwAr(\Cc)$ is $\tau_\alpha$-small and $\tau_\alpha\geqslant \kappa_{\alpha(i)}$ for every $i\in\Ii$. Using that $\tau_\alpha$-small limits commute with $\tau_\alpha$-filtered colimits by \cref{lem:FilteredColimitsPreserveFiniteLimits} and that colimits in functor categories are computed pointwise by \cref{lem:ColimitsInFunctorCategories}, the formula from \cref{cor:HomInFunctorCats} shows that $\alpha$ is $\tau_\alpha$-compact.
\end{proof}
A useful supplement to the adjoint functor theorem is the \emph{reflection theorem:}
\begin{thm}[Reflection theorem]\label{thm:ReflectionTheorem}
	Let $\Dd$ be a presentable $\infty$-category and let $\Cc\subseteq\Dd$ be a full sub-$\infty$-category such that $\Cc$ is closed under limits in $\Dd$ and there exists a regular cardinal $\kappa$ such that $\Cc$ is closed under $\kappa$-filtered colimits in $\Dd$. Then the inclusion $\Cc\subseteq \Dd$ has a left adjoint and $\Cc$ is presentable too.\hfill$\blacksquare$
\end{thm}
We won't prove the reflection theorem. A proof for ordinary categories can be found in Adamek and Rosicky's book \cite[Reflection Theorem~2.48]{AdamekRosicky}; the $\infty$-categorical version was only recently proven in \cite{ReflectionTheorem}.%
\footnote{\label{footnote:ReflectionTheorem}Interestingly, the proof for ordinary categories can not entirely be carried over. The step that fails is related to the following two important caveats:
\begin{alphanumerate}
	\item Let $\Cc$ be an $\infty$-category and suppose there are morphisms $\alpha\colon x\rightarrow y$ and $\beta\colon y\rightarrow x$ satisfying $\beta\circ\alpha\simeq \id_x$, so that $x$ is a retract of $y$. If $\Cc$ is an ordinary category, then $x$ can be expressed as the equaliser (and also as the coequaliser) of $\id_y$ and $\alpha\circ\beta$. However, this doesn't work in general $\infty$-categories---it already fails in $\cat{An}$, and even more spectacularly in the $\infty$-category of spectra. We can still express $x$ in terms of $y$, for example, as\label{enum:Retracts}
	\begin{equation*}
		x\simeq \colimit\left(y\xrightarrow{\alpha\circ\beta}y\xrightarrow{\alpha\circ\beta}y\xrightarrow{\alpha\circ\beta}\dotsb\right)\simeq \limit\left(\dotsb\xrightarrow{\alpha\circ\beta}y\xrightarrow{\alpha\circ\beta}y\xrightarrow{\alpha\circ\beta}y\right)
	\end{equation*}
	(to see this, just observe that these diagrams can be (co)initially replaced by constant $x$-valued diagrams). But a finite limit or colimit will never suffice.
	\item There is a notion of monomorphism in $\infty$-categories (see \cite[\S\href{https://people.math.harvard.edu/~lurie/papers/HTT.pdf\#subsection.5.5.6}{5.5.6}]{HTT}) and these allow for manipulation of equalisers as in ordinary categories. But the inclusion of a retract is usually not a monomorphism! Again, this already fails in $\cat{An}$---monomorphisms are inclusions of path components, but there are many more retracts---and even more spectacularly in $\cat{Sp}$, where there are no monomorphisms at all.\label{enum:Monomorphisms}
\end{alphanumerate}
If you'd like to read up on the proof of the $\infty$-categorical reflection theorem, I'd suggest you first read the proof for ordinary categories and identify where the above issues occur. Then check that all other arguments can be carried over. Finally, check out \cite{ReflectionTheorem} to see how the issues can be circumvented.
}
The only thing one has to show is that in the given situation $\Cc$ is automatically accessible. Indeed, if that's true, then $\Cc$ is presentable by \cref{cor:PresentableComplete}\cref{enum:AccessibleCocomplete}. Furthermore, \cref{cor:PresentableComplete}\cref{enum:PresentableComplete} shows that \cref{thm:AdjointFunctorTheorem}\cref{enum:AdjointFunctorTheoremRight} is applicable, producing the desired left adjoint. But proving that $\Cc$ is automatically accessible is surprisingly hard (and rather surprising altogether).


We start off the proof of \cref{thm:AdjointFunctorTheorem} with two preparatory lemmas.
\begin{lem}\label{lem:TerminalInSlice}
	Let $F\colon \Cc\rightarrow\Dd$ be a functor between $\infty$-categories and let $y\in\Dd$ be an object. Then $y$ admits a right adjoint object $x\in\Cc$ under $F$ if and only if the slice $\infty$-category $\Cc_{/y}\simeq \Cc\times_{\Dd}\Dd_{/y}$ has a terminal object.
\end{lem}
\begin{proof}[Proof sketch]
	Let $x\in \Cc$ be an object and $c\colon F(x)\rightarrow y$ a morphism in $\Cc$. Then $x$ is a right adjoint object to $y$ under $F$, with counit $c$, if and only if the composition
	\begin{equation*}
		\Hom_\Cc(-,x)\overset{F}{\Longrightarrow} \Hom_\Dd\bigl(F(-),F(x)\bigr)\overset{c_*}{\Longrightarrow}\Hom_\Dd\bigl(F(-),y\bigr)
	\end{equation*}
	is an equivalence of functors. By \cref{thm:EquivalencePointwise}, this can be checked on objects. So choose $x'\in\Cc$. To check that $\Hom_\Cc(x,x')\rightarrow \Hom_\Dd(F(x'),y)$, it's enough by \cref{thm:Whitehead} to check that the fibres over every $\alpha\in\Hom_\Dd(F(x'),y)$ are contractible. So fix some $\alpha\colon F(x')\rightarrow y$. Using the fact that $\Hom$ animae in pullbacks of $\infty$-categories are the pullbacks of the respective $\Hom$ animae (which we'll prove in more generality in \cref{lem:HomInLimits}\cref{enum:HomInLimits}), one easily computes that the fibre $\Hom_\Cc(x',x)\times_{\Hom_\Dd(F(x'),y)}\{\alpha\}$ is equivalent to $\Hom_{\Cc_{/y}}((x',\alpha\colon F(x')\rightarrow y),(x,c\colon F(x)\rightarrow y))$. So the fibres are all contractible if and only if $(x,c\colon F(x)\rightarrow y)$ is a terminal object of $\Cc_{/y}$.
\end{proof}
\begin{lem}\label{lem:TerminalObjectColimit}
	Let $\Cc$ be any \embrace{possibly large} $\infty$-category. Then $\Cc$ has a terminal object if and only if $\id_\Cc\colon \Cc\rightarrow\Cc$ has a colimit, in which case the terminal object is that colimit.
\end{lem}
\begin{proof}[Proof sketch]
	If $y\in\Cc$ is terminal, then $\{y\}\rightarrow\Cc$ is a right adjoint, hence coinitial by \cref{exm:Cofinal}\cref{enum:RightAdjointCofinal}. Hence $\colimit(\id_\Cc\colon \Cc\rightarrow\Cc)\simeq y$; in particular, the colimit exists.
	
	Conversely, assume the colimit exists, and let $u\colon \id_\Cc\Rightarrow \const y$ be the natural transformation exhibiting $y$ has the colimit. We wish to prove that $\Cc\shortdoublelrmorphism \{y\}$ is an adjunction. To this end, by \cref{lem:TriangleIdentities}, it suffices to construct the unit and the counit as well as to verify the triangle identities. We take $u$ to be our unit. The counit as well as the first triangle identity come for free since $\Fun(\Cc,\{y\})\simeq *$ and $\Fun(\{y\},\{y\})\simeq *$. By a quick unravelling, the second triangle identity comes down to proving that $u_y\colon y\rightarrow y$ is the identity on $y$. To this end, consider $u$ as a functor $u\colon \Delta^1\times\Cc\rightarrow\Cc$ and consider the composition $\sigma\coloneqq u\circ(\id_{\Delta^1}\times\Cc)\colon \Delta^1\times(\Delta^1\times\Cc)\rightarrow \Delta^1\times\Cc\rightarrow \Cc$. By \enquote{currying}, $\sigma$ corresponds to a functor $\Delta^1\times\Delta^1\rightarrow\Fun(\Cc,\Cc)$, or in other words, to a commutative square in $\Fun(\Cc,\Cc)$. By a somewhat confusing unravelling, that commutative square is
	\begin{equation*}
		\begin{tikzcd}[column sep=large]
			\id_\Cc\doublear["u"{black,above=0.1em}]{r}\doublear["u"'{black,left=0.1em}]{d}\drar[commutes] & \const y\doublear["\const u_y"{black,right=0.1em}]{d}\\
			\const y\doublear["\id_{\const y}"{black,above=0.1em}]{r} & \const y
		\end{tikzcd}
	\end{equation*}
	Thus, in the equivalence $\Hom_\Cc(y,y)\simeq \Hom_\Cc(\colimit_\Cc\id_\Cc,y)\simeq \Hom_{\Fun(\Cc,\Cc)}(\id_\Cc,y)$, both $\id_y$ and $u_y$ are mapped to $u\in\Hom_{\Fun(\Cc,\Cc)}(\id_\Cc,y)$. This proves $\id_y\simeq u_y$, as desired.
\end{proof}

\begin{proof}[Proof sketch of \cref{thm:AdjointFunctorTheorem}\cref{enum:AdjointFunctorTheoremLeft}]
	If $F$ admits a right adjoint, then $F$ preserves colimits by \cref{lem:AdjointsPreserveColimits}. Conversely, assume $F$ preserves colimits. Adjoints can be constructed pointwise by \cref{lem:Adjunction}, and thus by \cref{lem:TerminalInSlice}, it's enough to show that the slice $\infty$-category $\Cc_{/y}\simeq \Cc\times_\Dd\Dd_{/y}$ has a terminal object for every $y\in\Dd$. A straightforward generalisation of the arguments in the proof of \cref{cor:FunctorCategoriesPresentable} shows that $\Cc_{/y}$ has again all (small) colimits and is generated under colimits by its full sub-$\infty$-category $(\Cc_0)_{/y}\simeq \Cc_0\times_\Cc\Cc_{/y}$; this is the only time we use that $F$ preserves colimits. So we can replace $\Cc$ by $\Cc_{/y}$ and are thus reduced to showing that $\Cc$ has a terminal object.		
	
	%$F$ preserves colimits, we can use a similar (but dual) argument as in the proof of \cref{lem:ColimitsInSliceCategory}\cref{enum:LimitsInSlice} to show that $\Cc_{/y}$ has all (small) colimits. Furthermore, as we'll now explain a variation of the argument shows that $\Cc_{/y}$ is generated under colimits by its full sub-$\infty$-category $(\Cc_0)_{/y}\simeq \Cc_0\times_\Cc\Cc_{/y}$. Indeed, pick some $(x,\alpha\colon F(x)\rightarrow y)\in\Cc_{/y}$ and write $x\simeq\colimit_{i\in\Ii}x_i$ for some diagram $x_{(-)}\colon\Ii\rightarrow\Cc_0$. Applying $F$ to the natural transformation $u\colon x_{(-)}\Rightarrow \const x$ and composing with $\const\alpha\colon \const F(x)\Rightarrow \const y$ yields a transformation $\const\alpha\circ F(u)\colon F(x_{(-)})\Rightarrow \const y$, which in turn defines functors $(F(x_{(-)})\rightarrow y)\colon \Ii\rightarrow \Dd_{y/}$ and $(x_{(-)},F(x_{(-)}\rightarrow y))\colon \Ii\rightarrow \Cc_0\times_\Dd\Dd_{y/}$. Then a straightforward argument shows $(x,\alpha\colon F(x)\rightarrow y)\simeq \colimit_{i\in\Ii}(x_i,F(x_i)\rightarrow y)$ in $\Cc_{/y}$, as desired. Finally, it's easy to show that $(\Cc_0)_{/y}$ is essentially small. Clearly, $(\Cc_0)_{/y}$ is locally small, because \cref{cor:HomInSliceCategories} and the upcoming \cref{lem:HomInLimits} show that $\Hom_{(\Cc_0)_{/y}}$ can be written as a pullback involving $\Hom_\Cc$ and $\Hom_\Dd$, both of which are essentially small by the assumption that $\Cc$ and $\Dd$ are locally small. So its enough to show that $\pi_0\core((\Cc _0)_{/y})$ is a set. This easily follows from $\Cc$ and $\Dd$ being locally small, so that there can't be \enquote{too many} equivalence classes of pairs $(x_0,\alpha_0\colon F(x_0)\rightarrow y)$ where $x_0\in\Cc_0$. We leave the argument to you.
	
	By \cref{lem:TerminalObjectColimit}, we must show that $\id_\Cc\colon \Cc\rightarrow\Cc$ admits a colimit. Since $\Cc$ has all small colimits, it will be enough to show that $\Cc$ admits a coinitial functor from a small $\infty$-category. Note that this step requires some set-theoretic care, since it's not so clear why \cref{thm:JoyalsQuillenA} would be applicable to colimits with potentially large indexing $\infty$-categories. This problem can be solved by considering universes, and with some more effort even in ZFC; we'll ignore it in the following.
	
	Since $\Cc$ has all small colimits, the colimit $t\coloneqq\colimit(\Cc_0\rightarrow\Cc)$ exists. For every $x\in\Cc$ there exists a morphism $x\rightarrow t$. Indeed, since we assume $\Cc$ to be generated under colimits by $\Cc_0$, we can write $x$ as a colimit $x\simeq \colimit(\Ii\rightarrow\Cc_0\rightarrow\Cc)$ and then we can consider the morphism $x\simeq \colimit(\Ii\rightarrow\Cc_0\rightarrow\Cc)\rightarrow\colimit(\Cc_0\rightarrow\Cc)\simeq t$ using functoriality of colimits, see \cref{lem:ColimitsFunctorial}. Now let $\Tt\subseteq \Cc$ be the full sub-$\infty$-category spanned by $t$ (note that $\Tt$ is not just $\{t\}$, since we include all non-identity endomorphisms of $t$ as well). Since $\Cc$ is locally small, $\Tt$ must be essentially small. We claim that $\Tt\rightarrow\Cc$ is coinitial. To this end, we'll show that $\Tt\times_\Cc\Cc_{x/}$ is filtered; then \cref{lem:FilteredCofinal} will show that the condition from \cref{thm:JoyalsQuillenA}\cref{enum:WeaklyContractible} is satisfied. Let $\alpha\colon \Ii\rightarrow\Tt\times_\Cc\Cc_{x/}$ be a functor from any small $\infty$-category. If $\ov\alpha\colon \Ii\rightarrow\Tt\times_\Cc\Cc_{x/}\rightarrow\Cc$ denotes the underlying functor, then $\alpha$ is equivalently given by a natural transformation $\const x\Rightarrow \ov\alpha$ such that $\ov\alpha$ takes values in the full sub-$\infty$-category $\Tt\subseteq\Cc$. Since $\Cc$ has small colimits, $x\simeq\colimit_{i\in\Ii}\ov\alpha(i)$ exists in $\Cc$. As observed above, there exists a morphism $x\rightarrow t$. Composing the colimit transformation $\ov\alpha\Rightarrow \const x$ with $\const x\Rightarrow t$ yields a natural transformation $\ov\alpha\Rightarrow \const t$, or equivalently, a functor $\ov\alpha^\triangleright\colon \Ii^\triangleright \rightarrow \Cc$. By construction, $\ov\alpha^\triangleright$ takes values in the full sub-$\infty$-category $\Tt\subseteq\Cc$. Composing with $\const x\Rightarrow \ov\alpha$ provides a functor $\alpha^\triangleright\colon \Ii^\triangleright\rightarrow\Tt\times_\Cc\Cc_{x/}$, as desired. This finishes the proof that $\Tt\times_\Cc\Cc_{x/}$ is filtered.%So we're done!
\end{proof}
Our proof of \cref{thm:AdjointFunctorTheorem}\cref{enum:AdjointFunctorTheoremRight} will again be preceded by two preparatory lemmas.
\begin{lem}[\enquote{Right adjoints preserve sufficiently filtered colimits}]\label{lem:RightAdjointsAccessible}
	Let $G\colon \Dd\rightarrow\Cc$ be a functor between accessible $\infty$-categories. If $G$ admits a left adjoint $F$, then $G$ preserves $\tau$-filtered colimits for sufficiently large regular cardinals $\tau$.
\end{lem}
\begin{proof}
	Choose regular cardinals $\kappa$ and $\lambda$ such that $\Cc$ is $\kappa$-accessible and $\Dd$ is $\lambda$-accessible. By \cref{lem:KappaCompactlyGenerated}, we may identify $\Cc$ and $\Dd$ with $\cat{Ind}_\kappa(\Cc^\kappa)$ and $\cat{Ind}_\lambda(\Dd^\lambda)$, respectively. First note that for every $y\in\Dd$ there exists a regular cardinal $\lambda_y$ such that $y$ is $\lambda_y$-compact. Indeed, we may write $y$ has a colimit of $\lambda$-compact objects, and then it suffices to choose $\lambda_y\geqslant \lambda$ sufficiently large so that the indexing diagram of the colimit is $\lambda_y$-small. Since $\Cc^\kappa$ is essentially small, as we've seen in the proof of \cref{lem:KappaCompactlyGenerated}, we may choose a regular cardinal $\tau\geqslant \kappa$ such that $F(x)$ is $\tau$-compact for all $x\in\Cc^\kappa$. We claim that $G$ preserves $\tau$-filtered colimits. Since $\Cc\simeq\cat{Ind}_\kappa(\Cc^\kappa)\subseteq \PSh(\Cc^\kappa)$, the functors $\Hom_\Cc(x,-)\colon \Cc\rightarrow \cat{An}$ for $x\in\Cc^\kappa$ are jointly conservative and preserve $\kappa$-filtered and thus also $\tau$-filtered colimits. So it's enough to show that $\Hom_\Cc(x,G(-))$ preserves $\tau$-filtered colimits. But $\Hom_\Cc(x,G(-))\simeq \Hom_\Dd(F(x),-)$ and $F(x)$ is $\tau$-compact by construction.
\end{proof}
\begin{lem}\label{lem:Accessible}
	Let $\Cc$ be a $\kappa$-accessible $\infty$-category. Then $\Cc$ is also $\tau$-accessible for every sufficiently large regular cardinal $\tau$.
\end{lem}
\begin{rem}\label{rem:Accessible}
	It's usually \emph{not} true that a $\kappa$-accessible $\infty$-category $\Cc$ is $\tau$-accessible for all $\tau >\kappa$. However, this works if $\Cc$ is presentable. Indeed, it's immediately clear from \cref{lem:KappaCompactlyGenerated}\cref{enum:CompactGenerators} that any set of $\kappa$-compact generators is also a set of $\tau$-compact generators 
\end{rem}
\begin{proof}[Proof sketch of \cref{lem:Accessible}]
	By \cref{lem:KappaCompactlyGenerated}\cref{enum:DGeneratedUnderFilteredColimits} will be enough to show that $\Cc$ is generated under $\tau$-filtered colimits by $\Cc^\tau$, where $\tau$ is a sufficiently large regular cardinal that will be chosen at the end of the proof. Every $x\in\Cc$ can be written as $x\simeq \colimit_{j\in\Jj}x_j$, where $\Jj$ is $\kappa$-filtered and $x_j\in\Cc^\kappa$. We'll rewrite this as a $\tau$-filtered colimit of $\tau$-compact objects. First, by \cref{blackbox:Cofinal} in the proof of \cref{lem:HomotopyGroupsFilteredColimits}, we find a coinitial functor $J\rightarrow\Jj$ from a directed partially ordered set $J$. Note that $J$ is automatically a $\kappa$-filtered $\infty$-category by the criterion from \cref{lem:FilteredColimitsPreserveFiniteLimits}. We'll show that $J$ can be written as a colimit $J\simeq \colimit_{i\in I}J_i$ in $\cat{Cat}_\infty$, where $I$ is a $\tau$-filtered directed partially ordered set and $J_i\subseteq J$ are essentially $\tau$-small $\kappa$-filtered partially ordered subsets. If we can do this, we're done. Indeed, by \cref{lem:ColimitManipulations}\cref{claim:AssembleColimits}, we may then write $x\simeq \colimit_{i\in I}\colimit_{j\in J_i}x_j$. Each $\colimit_{j\in J_i}x_j$ exists, as $\Cc$ admits $\kappa$-filtered colimits by \cref{lem:Ind}\cref{enum:IndGeneratedUnderFilteredColimits}. Furthermore, $\colimit_{j\in J_i}x_j$ is $\tau$-compact because each $x_j$ is $\kappa$-compact, hence $\tau$-compact, and $\tau$-compact objects are stable under $\tau$-small colimits by an easy application of \cref{lem:FilteredColimitsPreserveFiniteLimits}.
	
	To write $J$ as such a colimit, let $\Pp^\tau(J)$ be the partially ordered set of subsets $S\subseteq J$ of cardinality $\abs*{S}<\tau$. Note that $\Pp^\tau(J)$ is $\tau$-filtered as an $\infty$-category. Indeed, using \cref{lem:SimplicialHoNerveAdjunction}, it's enough to show that $\Pp^\tau(J)$ is $\tau$-filtered as an ordinary category, which is true since we can just take unions of $<\tau$ subsets of cardinality $<\tau$. Each $S\in\Pp^\tau(J)$ can be identified with the full subcategory $J[S]\subseteq J$ spanned by $S$ and we have $J\simeq \colimit_{S\in\Pp^\tau(J)}J[S]$ in $\cat{Cat}_\infty$. One way to prove this would be to use that filtered colimits in $\cat{Cat}_\infty$ can be computed on the level of simplicial sets (see the proof of \cref{cor:AnPresentable}); then the desired equivalence is straightforward. For an alternative, model-independent argument, let $\Uu$ be the unstraightening of the functor $\Pp^\tau(J)\rightarrow \cat{Cat}_\infty$ sending $S\mapsto J[S]$. Then $\Uu$ is an ordinary category and can be easily described explicitly. The same argument as in the proof of claim~\cref{claim:FilteredCoproduct} in the proof of~\cref{lem:KappaCompactlyGenerated} shows that $\Uu\rightarrow J$ is coinitial. By \cref{lem:ColimitsInAnima}, $\colimit_{S\in\Pp^\tau(J)}J[S]$ is a localisation of $\Uu$. Since localisations are coinitial by  \cref{exm:Cofinal}\cref{enum:LocalisationsCofinal}, we conclude that $\colimit_{S\in\Pp^\tau(J)}J[S]\rightarrow J$ is coinitial too. This is not quite what we wanted, but it's enough for our purposes. Now we claim:
	\begin{alphanumerate}\itshape
		\item[\boxtimes] \!There exists a partially ordered subset $I\subseteq \Pp^\tau(J)$ such that $J[S]$ is $\kappa$-filtered for every $S\in I$ and such that the inclusion $I\rightarrow\Pp^\tau(J)$ has a left adjoint $L\colon \Pp^\tau(J)\rightarrow I$.\label{claim:Filterification}
	\end{alphanumerate}
	Since right adjoints are coinitial by \cref{exm:Cofinal}\cref{enum:RightAdjointCofinal}, we also get $J\simeq \colimit_{S\in I}J[S]$. Furthermore, this coinitiality implies that $I$ is $\tau$-filtered again, because it satisfies the criterion from \cref{lem:FilteredColimitsPreserveFiniteLimits}. So once we know \cref{claim:Filterification}, we're done.
	
	For every equivalence class of functors $\alpha\colon \Ii\rightarrow J$ from an essentially $\kappa$-small $\infty$-category $\Ii$, choose an extension $\alpha^\triangleright\colon \Ii^\triangleright\rightarrow J$. Let $S_0\in \Pp^\tau(J)$. Let $S_1\subseteq J$ be obtained from $S_0$ by adjoining the \enquote{tip of the cone} for every $\alpha^\triangleright \colon \Ii^\triangleright\rightarrow J$ such that $\alpha\colon \Ii \rightarrow J$ factors through $J[S_0]\rightarrow J$. If $\tau$ is larger than the set of equivalence classes of essentially $\kappa$-small $\infty$-categories, then $S_1$ will have cardinality $\abs*{S_1}<\tau$ again. By transfinite induction, we can repeat this construction $\kappa$ many times. The result is a subset $S_\kappa\subseteq J$ such that $\abs*{S_\kappa}<\tau$ and $J[S_\kappa]$ is $\kappa$-filtered. If we put $L(S_0)\coloneqq S_\kappa$, then $L\colon \Pp^\tau(J)\rightarrow \Pp^\tau(J)$ is a functor satisfying $L\circ L=L$ (we really get an equality, not just an equivalence). Thus, if $I\subseteq \Pp^\tau(J)$ is the image of $L$, then an easy argument shows that $L\colon \Pp^\tau(J)\rightarrow I$ is indeed left adjoint to the inclusion (bear in mind that we're working with ordinary categories here, so constructing functors and adjunctions can be done by hand). Therefore, the conditions from \cref{claim:Filterification} are satisfied.
\end{proof}
\begin{proof}[Proof sketch of \cref{thm:AdjointFunctorTheorem}\cref{enum:AdjointFunctorTheoremRight}]
	By the dual of \cref{lem:TerminalInSlice}, it's enough to show that the slice $\infty$-category $\Cc_{y/}\simeq \Cc\times_\Dd\Dd_{y/}$ has an initial object for all $y\in\Dd$. By the dual of \cref{lem:TerminalObjectColimit}, this is equivalent to showing that $\id_{\Cc_{y/}}\colon \Cc_{y/}\rightarrow\Cc_{y/}$ admits a limit. A straightforward generalisation of \cref{lem:ColimitsInSliceCategory}\cref{enum:LimitsInSlice} shows that $\Cc_{y/}$ has all (small) limits; this argument crucially uses that $F$ preserves limits. So it will be enough to construct an initial functor from an essentially small $\infty$-category into $\Cc_{y/}$.
	
	By assumption and \cref{lem:Accessible} we may choose a sufficiently large regular cardinal $\kappa$ such that $\Cc$ is $\kappa$-accessible, $F$ preserves $\kappa$-filtered colimits, and $y$ is $\kappa$-compact. Let $\Tt\subseteq\Cc_{y/}$ be the full sub-$\infty$-category spanned by those $(x, \alpha\colon y\rightarrow F(x))$ where $x$ is $\kappa$-compact. By an easy argument, the likes of which we've seen several times by now, $\Tt$ is essentially small. We claim that for every $z\in \Cc_{y/}$ there is an element $t\in \Tt$ and a morphism $t\rightarrow z$ in $\Cc_{y/}$. If we can show this, then a similar argument as in the proof of \cref{enum:AdjointFunctorTheoremLeft} shows that $\Tt\rightarrow\Cc_{y/}$ is initial. Indeed, we'll show that $\Tt\times_{\Cc_{y/}}(\Cc_{y/})_{/w}$ is cofiltered for every $w\in \Cc_{y/}$, which will imply initiality by the dual of \cref{thm:JoyalsQuillenA}\cref{enum:WeaklyContractible} and the dual of \cref{lem:FilteredCofinal}. So let $\alpha\colon \Ii\rightarrow \Tt\times_{\Cc_{y/}}(\Cc_{y/})_{/w}$ be a functor from a small $\infty$-category $\Ii$. Since $\Cc_{y/}$ admits small limits, the underlying functor $\ov\alpha\colon \Ii\rightarrow \Cc_{y/}$ admits a limit $z\simeq \limit_{i\in\Ii}\ov\alpha(i)$. Choosing a morphism $t\rightarrow z$ for some $t\in \Tt$, we get natural transformations $\const t\Rightarrow\const z\Rightarrow\ov\alpha$. The composition $\const t\Rightarrow \ov\alpha$ induces an extension $\alpha^\triangleleft\colon \Ii^\triangleleft\rightarrow \Tt\times_{\Cc_{y/}}(\Cc_{y/})_{/w}$ of $\alpha$, as desired.
	
	It remains to show our claim that for every $z\in\Cc_{y/}$ there exists a moprhism $t\rightarrow z$ for some $t\in\Tt$. Write $z$ as a pair $(x,\beta\colon y\rightarrow F(x))$ for some $x\in \Cc$. Since $\Cc$ is $\kappa$-accessible, we can write $x$ as a $\kappa$-filtered colimit $x\simeq \colimit_{j\in\Jj}x_j$ for some $x_j\in \Cc^\kappa$. Since $F$ preserves $\kappa$-filtered colimits, $F(x)\simeq \colimit_{j\in\Jj}F(x_j)$. Since $y$ is $\kappa$-compact by assumption and $\pi_0$ commutes with colimits by \cref{lem:HomotopyGroupsFilteredColimits}, the canonical map
	\begin{equation*}
		\colimit_{j\in\Jj}\pi_0\Hom_\Dd\bigl(y,F(x_j)\bigr)\overset{\cong}{\longrightarrow}\pi_0\Hom_\Dd\bigl(y,F(x)\bigr)
	\end{equation*}
	is a bijection. Hence $\beta\colon y\rightarrow F(x)$ factors over a map $\beta_j\colon y\rightarrow F(x_j)$ for some $j\in\Jj$. Then $x_j\rightarrow\colimit_{j\in\Jj}x_j\simeq x$ induces a morphism $(x_j,\beta_j\colon y\rightarrow F( x_j))\rightarrow (x,\beta\colon y\rightarrow F(x))$ in $\Cc_{y/}$. Since $x_j$ is $\kappa$-compact, we see that $(x_j,\beta_j\colon y\rightarrow F( x_j))\in\Tt$. This proves that there exists a morphism $t\rightarrow z$ for some $t\in\Tt$ and thus we've proved that $F\colon \Cc\rightarrow \Dd$ has a left adjoint.
	
	To prove the converse in the case where $\Cc$ and $\Dd$ are both accessible, just observe that if $F$ admits a left adjoint, then $F$ preserves all limits by \cref{lem:AdjointsPreserveColimits} and also all sufficiently filtered colimits by \cref{lem:RightAdjointsAccessible}. Finally, to show that the conditions in \cref{enum:AdjointFunctorTheoremLeft} and \cref{enum:AdjointFunctorTheoremRight} are satisfied in the case where $\Cc$ and $\Dd$ are presentable, the only non-obvious assertion is that for every $y\in\Dd$ there exists a regular cardinal $\kappa_y$ such that $y$ is $\kappa_y$-compact. But we've seen this in the proof of \cref{lem:RightAdjointsAccessible} already.
\end{proof}

\subsection{Lurie's magic \texorpdfstring{$\infty$}{Infinity}-category \texorpdfstring{$\cat{Pr}^\L$}{PrL}}\label{subsec:PrL}

To finish this appendix to \cref{sec:InftyCategoryTheory}, we'll introduce Lurie's $\infty$-category $\cat{Pr}^\L$. At first, it'll probably not be obvious to you why this construction is so useful, but hopefully you'll come to appreciate it more and more. Without further ado, here's the \enquote{definition}.
\begin{numpar}[\enquote{Definition}.]\label{par:PrL}
	The \emph{$\infty$-category of presentable $\infty\text{-}$categories and left adjoint functors} $\cat{Pr}^\L$ is the non-full sub-$\infty$-category of $\cat{Cat}_\infty$ spanned by the presentable $\infty$-categories and the left adjoint, or equivalently (by \cref{thm:AdjointFunctorTheorem}\cref{enum:AdjointFunctorTheoremLeft}), colimit-preserving functors.
\end{numpar}
As stated, \enquote{Definition}~\cref{par:PrL} doesn't make sense: $\cat{Cat}_\infty$ only contains small $\infty$-categories, but presentable $\infty$-categories usually aren't small. So to make \enquote{Definition}~\cref{par:PrL} work, we would need to assume two nested universes (of \enquote{small} and \enquote{large} sets) and define $\cat{Pr}^\L$ as a  non-full sub-$\infty$-category of the $\infty$-category $\widehat{\cat{Cat}}_\infty$ of all large $\infty$-categories (which is neither small nor large). But there's an alternative construction of $\Pr^\L$ that stays within ZFC; see \cref{par:PrLInZFC} below. For the moment, let's work with universes and assume $\widehat{\cat{Cat}}_\infty$ exists. Also note that $\cat{Pr}^\L$ is not even locally small: We have $\Hom_{\cat{Pr}^\L}(\Cc,\Dd)\simeq \core \Fun^\L(\Cc,\Dd)$; this is usually not an essential small anima.\footnote{However, $\Fun^\L(\Cc,\Dd)$ is at least locally small. To see this, write $\Cc\simeq \cat{Ind}_\kappa(\Cc^\kappa)$ for some regular cardinal $\kappa$. Using a similar argument as in \cref{lem:PrLKappa}\cref{enum:FunLKappa}, $\Fun^\L(\Cc,\Dd)$ can be identified with the full sub-$\infty$-category of $\Fun(\Cc^\kappa,\Dd)$ spanned by those functors that preserve $\kappa$-small colimits. This is clearly a locally small $\infty$-category.} %we'll ignore these set-theoretic issues and pretend that $\cat{Pr}^\L$ is a non-full sub-$\infty$-category of $\cat{Cat}_\infty$.

Let's begin by studying limits and colimits in $\cat{Pr}^\L$. To this end, we also consider the $\infty$-category $\cat{Pr}^\R$ of all presentable $\infty$-categories and right adjoint functors.

%The reason why people care so much about $\cat{Pr}^\L$ is that one can do algebra in it! In \cref{subsec:SymmetricMonoidal}, we'll define what a \emph{symmetric monoidal $\infty$-category} is. Lurie constructs a symmetric monoidal structure on $\cat{Pr}^\L$, called the \emph{Lurie tensor product}, which has a bunch of amazing properties. In particular, the Lurie tensor product allows us to do algebra not only on the level of rings~$R$, but also on the level of their derived $\infty$-categories $\Dd(R)$. We'll say a lot more about the Lurie tensor product in [TODO]. For now, as a preparation, let's study limits and colimits in $\cat{Pr}^\L$. Let $\cat{Pr}^\R$ be the $\infty$-category of presentable $\infty$-categories and right adjoint functors. Then \cref{cor:ExtractingAdjoints}\cref{enum:CatLCatR} restricts to an equivalence $\cat{Pr}^\L\simeq (\cat{Pr}^\R)^\op$. In particular, colimits in $\cat{Pr}^\L$ are just limits in $\cat{Pr}^\R$ and so it suffices to study limits.
\begin{lem}\label{lem:PrLColimits}
	The $\infty$-categories $\cat{Pr}^\L$ and $\cat{Pr}^\R$ have all small limits and colimits. The forgetful functors $\cat{Pr}^\L\rightarrow \widehat{\cat{Cat}}_\infty$ and $\cat{Pr}^\R\rightarrow \widehat{\cat{Cat}}_\infty$ preserve all small limits.
\end{lem}
To prove \cref{lem:PrLColimits}, we need two more technical lemmas. The first one has already been referenced several times before.
\begin{lem}\label{lem:HomInLimits}
	Let $\Cc_{(-)}\colon\Ii\rightarrow\cat{Cat}_\infty$ \embrace{or $\Cc_{(-)}\colon\Ii\rightarrow\widehat{\cat{Cat}}_\infty$} be a diagram of $\infty$-categories.
	\begin{alphanumerate}
		\item For every pair of objects $x,y\in\limit_{i\in\Ii}\Cc_i$ and their images $x_i,y_i\in \Cc_i$ under the projections $\pr_i\colon \limit_{i\in\Ii}\Cc_i\rightarrow \Cc_i$ there is a canonical equivalence\label{enum:HomInLimits}
		\begin{equation*}
			\Hom_{\limit_{i\in\Ii}\Cc_i}\left(x,y\right)\overset{\simeq}{\longrightarrow}\limit_{i\in\Ii}\Hom_{\Cc_i}\left(x_i,y_i\right)\,.
		\end{equation*}
		\item Let $F\colon \Jj\rightarrow \limit_{i\in\Ii}\Cc_i$ be a functor. Assume that all compositions ${\pr_i}\circ F\colon \Jj\rightarrow \Cc_i$ admit a colimit and that these colimits are preserved under $\Cc_i\rightarrow\Cc_j$ for all morphisms $i\rightarrow j$ in $\Ii$. Then $F$ admits a colimit and that colimit is preserved under $\pr_i\colon\limit_{i\in \Ii}\Cc_i\rightarrow \Cc_i$ for all $i\in \Ii$. A similar assertion is true for limits.\label{enum:ColimitsInLimits}
	\end{alphanumerate}
\end{lem}
\begin{proof}
	Let $\Cc$ be an $\infty$-category and $x,y\in\Cc$. Using $\Hom_{\cat{Cat}_\infty}(-,\Cc)\simeq\core\Fun(-,\Cc)$, we get a pullback diagram
	\begin{equation*}
		\begin{tikzcd}
			\Hom_\Cc(x,y)\rar\dar\drar[pullback] & \Hom_{\cat{Cat}_\infty}\bigl(\Delta^1,\Cc\bigr)\dar\\
			\{x\}\times\{y\}\rar & \Hom_{\cat{Cat}_\infty}(*\ \,*,\Cc)
		\end{tikzcd}
	\end{equation*}
	in $\cat{An}$. Now for every $\infty$-category $\Dd$, the functor $\Hom_{\cat{Cat}_\infty}(\Dd,-)\colon \cat{Cat}_\infty\rightarrow\cat{An}$ preserves arbitrary limits by \cref{cor:HomPreservesLimits}. Applying this for $\Dd\simeq \Delta^1$ and $\Dd\simeq *\ \,*$ and using that pullbacks commute with limits by the dual of \cref{lem:ColimitManipulations}, we deduce~\cref{enum:HomInLimits}.
	
	For~\cref{enum:ColimitsInLimits}, we only have to show that the colimit cocones $\Jj^\triangleright \rightarrow \Cc_i$ for all $i\in \Ii$ assemble into a functor $F^\triangleright\colon \Jj^\triangleright\rightarrow \limit_{i\in \Ii}\Cc_i$. If we can do this, then~\cref{enum:HomInLimits} combined with \cref{cor:HomPreservesColimits} and the fact that limits commute with limits (by the dual of \cref{lem:ColimitManipulations}) will show that $F^\triangleright$ is a colimit cocone. To construct $F^\triangleright$, we'll show a slightly stronger assertion: Consider the slice-$\infty$-category $(\cat{Cat}_\infty)_{\Jj/}$ and let $(\cat{Cat}_\infty)_{\Jj/}^{\colimit}$ be the non-full sub-$\infty$-category spanned by those objects $(\Jj\rightarrow \Cc)$ that admit a colimit and those morphisms that preserve this colimit. We wish to show that $(\cat{Cat}_\infty)_{\Jj/}^{\colimit}$ has all limits and that $(\cat{Cat}_\infty)_{\Jj/}^{\colimit}\rightarrow(\cat{Cat}_\infty)_{\Jj/}$ preserves all limits. By \cref{lem:ColimitsIffCoproductsAndPushouts}, it's enough to check this for products and pullbacks. So we can reduce the construction of $F^\triangleright\colon \Jj^\triangleright\rightarrow \limit_{i\in\Ii}\Cc_i$ to the case where $\limit_{i\in\Ii}\Cc_i$ is a product or a pullback.%
	\footnote{\label{footnote:LimitNonFullSubcategory}After constructing $F^\triangleright\colon \Jj^\triangleright\rightarrow \limit_{i\in\Ii}\Cc_i$ there's still something to show before we can conclude that $\limit_{i\in\Ii}\Cc_i$ (taken in $\cat{Cat}_\infty$ or $(\cat{Cat}_\infty)_{\Jj/}$, this doesn't matter by the dual of \cref{lem:ColimitsInSliceCategory}\cref{enum:LimitsInSlice}) is also the limit in $(\cat{Cat}_\infty)_{\Jj/}^{\colimit}$. The problem is that $(\cat{Cat}_\infty)_{\Jj/}^{\colimit}$ is only a \emph{non-full} sub-$\infty$-category of $(\cat{Cat}_\infty)_{\Jj/}$. But this is easily fixed. Using \cref{lem:NonFullSubcategory} and the universal property of $\limit_{i\in\Ii}\Cc_i$ in $(\cat{Cat}_\infty)_{\Jj/}$, verifying the corresponding universal property in $(\cat{Cat}_\infty)_{\Jj/}^{\colimit}$ reduces to a matching of path components. For example, in the case of a product, we have to show that the projections $\pr_i\colon \prod_{i\in I}\Cc_i\rightarrow \Cc_i$ preserve the colimit over $\Jj$; and furthermore, that a functor $\Dd\rightarrow \prod_{i\in I}\Cc_i$ in $(\cat{Cat}_\infty)_{\Jj/}$ preserves the colimit over $\Jj$ if and only if the same is true for each $\Dd\rightarrow \Cc_i$. A similar assertion would be to show for pullbacks. Both are straightforward.}
	
	In the product case it's clear how to construct $F^\triangleright\colon \Jj^\triangleright \rightarrow \prod_{i\in I}\Cc_i$. So let's consider a pullback $\Cc_0\times_{\Cc_2}\Cc_1$ of $\alpha_{0}\colon \Cc_0\rightarrow \Cc_2$ and $\alpha_{1}\colon \Cc_1\rightarrow\Cc_2$. Choose colimit cocones $F_0^\triangleright\colon \Jj^\triangleright \rightarrow \Cc_0$ and $F_1^\triangleright\colon \Jj^\triangleright \rightarrow \Cc_1$. Choose a composition $F_2^\triangleright\coloneqq \alpha_0\circ F_0^\triangleright$ and a composition $\alpha_{1}\circ F_1^\triangleright$. Then $F_2^\triangleright$ and $\alpha_{1}\circ F_1^\triangleright$ are both colimit cocones of the given functor ${\pr_2}\circ F\colon \Jj\rightarrow \Cc_2$. Since colimit cocones are unique up to equivalence, there must be a natural equivalence $\eta\colon F_2^\triangleright\overset{\simeq}{\Longrightarrow}\alpha_{1}\circ F_1^\triangleright$. Thus, we obtain a commutative diagram
	\begin{equation*}
		\begin{tikzcd}
			\Jj^\triangleright\eqar[r]\drar[commutes]\dar["F_0^\triangleright"'] & \Jj^\triangleright\eqar[r]\drar[commutes]\dar["F_2^\triangleright"'] & \Jj^\triangleright\dar["F_1^\triangleright"]\\
			\Cc_0\rar["\alpha_0"] & \Cc_2 & \Cc_1\lar["\alpha_1"']
		\end{tikzcd}
	\end{equation*}
	in $\cat{Cat}_\infty$ ($\eta$ is precisely what makes the right square commute). This diagram constitutes a natural transformation $\const \Jj^\triangleright \Rightarrow \Cc_{(-)}$ in $\Fun(\Lambda_2^2,\Cc)$, which induces the desired functor $\Jj^\triangleright\rightarrow \Cc_0\times_{\Cc_2}\Cc_2$. As argued above, this is automatically a colimit cone.
\end{proof}
\begin{lem}\label{lem:AccessibilityOfFibreProducts}
	Let $\cat{Acc}\subseteq \widehat{\cat{Cat}}_\infty$ be the non-full sub-$\infty$-category spanned by accessible $\infty$-categories and functors that preserve sufficiently filtered colimits. Then $\cat{Acc}$ has all limits and $\cat{Acc}\rightarrow \widehat{\cat{Cat}}_\infty$ preserves all limits.
\end{lem}
\begin{proof}[Proof sketch]
	By \cref{lem:ColimitsIffCoproductsAndPushouts} we can reduce to products and pullbacks. We start with products. Let $(\Cc_i)_{i\in I}$ be an collection of accessible $\infty$-categories. We must to show that $\Cc\coloneqq \prod_{i\in I}\Cc_i$ is accessible again and that the projections $\pr_i\colon \Cc\rightarrow \Cc_i$ preserve sufficiently filtered colimits. The latter is immediate from \cref{lem:HomInLimits}\cref{enum:ColimitsInLimits}. To prove that $\Cc$ is accessible, we use \cref{lem:Accessible} to choose a sufficiently large regular cardinal $\kappa$ such that $\kappa >\abs*{I}$ and all $\Cc_i$ are $\kappa$-accessible. If $x=(x_i)_{i\in I}$ is an object of $\Cc$ such that each $x_i\in \Cc_i$ is $\kappa$-compact, then $x$ is $\kappa$-compact too: Indeed, this follows from \cref{lem:HomInLimits}\cref{enum:HomInLimits} and the fact that $\kappa$-filtered colimits commute with $I$-indexed products in $\cat{An}$ by \cref{lem:FilteredColimitsPreserveFiniteLimits}. Now let $y=(y_i)_{i\in I}$ be another object of $\Cc$. For all $i\in I$, we can write $y_i\simeq \colimit_{j\in\Jj_i}x_{i}(j)$ as a $\kappa$-filtered colimit of $\kappa$-compact objects. By the same trick as in the proof of \cref{lem:Presentable}, we can replace each $\Jj_i$ by $\Jj\coloneqq \prod_{i\in I}\Jj_i$. Putting $x(j)\coloneqq (x_i(j))_{i\in I}\in\Cc$ we deduce that $y\simeq \colimit_{j\in \Jj}x(j)$ is expressible as a $\kappa$-filtered colimit of $\kappa$-compact objects by \cref{lem:HomInLimits}\cref{enum:ColimitsInLimits}. This shows that $\Cc$ is accessible.
	
	Now consider a pullback $\Cc\coloneqq \Cc_0\times_{\Cc_2}\Cc_1$ of accessible $\infty$-categories along functors $\alpha_0\colon \Cc_0\rightarrow \Cc_2$ and $\alpha_1\colon \Cc_1\rightarrow \Cc_2$ that preserve sufficiently filtered colimits. Again, \cref{lem:HomInLimits}\cref{enum:ColimitsInLimits} ensures that $\pr_i\colon \Cc\rightarrow \Cc_i$ preserve sufficiently filtered colimits, so it's enough to show that $\Cc$ is accessible. Choose a sufficiently large regular cardinal $\kappa$ such that $\Cc_0$, $\Cc_1$, $\Cc_2$ are $\kappa$-accessible and $\alpha_0$, $\alpha_1$ preserve $\kappa$-filtered colimits. Then $\Cc_0^\kappa$ and $\Cc_1^\kappa$ are small $\infty$-categories. Choose $\tau$ large enough such that the images of $\alpha_0|_{\Cc_0^\kappa}\colon \Cc_0^\kappa\rightarrow \Cc_2$ and $\alpha_1|_{\Cc_1^\kappa}\colon \Cc_1^\kappa\rightarrow \Cc_2$ land inside $\tau$-compact objects of $\Cc_2$. Let $\ov\Cc_0\subseteq \Cc_0$ be the full sub-$\infty$-category spanned by all retracts of objects that can be written as a $\tau$-small $\kappa$-filtered colimit of objects in $\Cc_0^\kappa$. Enlarging $\tau$ if necessary, we can ensure $\ov\Cc_0\simeq \Cc_0^\tau$. Indeed, it's clear that all objects of $\ov\Cc_0$ are $\tau$-compact in $\Cc_0$. Conversely, the proof of \cref{lem:Accessible} shows that, for sufficiently large $\tau$, every object of $\Cc_0$ can be written as a $\tau$-filtered colimit of objects in $\ov\Cc_0$. By the argument in the proof of \cref{lem:KappaCompactlyGenerated}, all $\tau$-compact objects of $\Cc_0$ must then be retracts of objects in $\ov\Cc_0$, as desired.
	
	By assumption, $\alpha_0\colon \Cc_0\rightarrow \Cc_1$ preserves $\kappa$-filtered colimits and retracts. Furthermore, $\Cc_2^\tau$ is closed under $\tau$-small colimits and under retracts. We deduce that for sufficiently large $\tau$, the functor $\alpha_0$ restricts to a functor $\alpha_0\colon \Cc_0^\tau\rightarrow \Cc_2^\tau$ and likewise $\alpha_1$ restricts to a functor $\alpha_1\colon \Cc_1^\tau\rightarrow \Cc_2^\tau$. Now consider the pullback $\ov\Cc\coloneqq \Cc_0^\tau\times_{\Cc_2^\tau}\Cc_1^\tau$. We wish to show that $\Cc\simeq \cat{Ind}_\tau(\ov\Cc)$. By construction, $\ov\Cc$ is a full sub-$\infty$-category of $\Cc$, hence according to \cref{lem:KappaCompactlyGenerated}\cref{enum:DGeneratedUnderFilteredColimits} it will be enough to show that every object in $\Cc$ can be written as a $\tau$-filtered colimit of objects in $\ov\Cc$. So let $x\in \Cc$ and let $x_i\coloneqq \pr_i(x)$ be its projections to $\Cc_i$ for $i=0,1,2$. We claim:
	\begin{alphanumerate}\itshape
		\item[\boxtimes_1] $\ov\Cc_{/x}$ is $\tau$-filtered.\label{claim:tauFiltered}
		\item[\boxtimes_2] The projections $\pr_i\colon \ov\Cc_{/x}\rightarrow \Cc_{i/x_i}^\tau$ for $i=0,1$ are coinitial functors.\label{claim:ProjectionsCoinitial}
	\end{alphanumerate}
	Together, these claims imply that the canonical morphism $\colimit(s\colon \ov\Cc_{/x}\rightarrow \Cc)\rightarrow x$ is an equivalence, so that $x$ can be written as a $\tau$-filtered colimits of objects in $\ov\Cc$, as desired. Indeed, by definition of $\Cc$ as a pullback, the projections $\pr_i$ for $i=0,1$ are jointly conservative and they preserve $\kappa$-filtered colimits by \cref{lem:HomInLimits}\cref{enum:ColimitsInLimits}. In particular, if \cref{claim:tauFiltered} holds, then $\pr_i$ preserves $\ov\Cc$-indexed colimits. Using \cref{claim:ProjectionsCoinitial}, we conclude
	\begin{equation*}
		\pr_i\Bigl(\colimit\left(\ov\Cc_{/x}\rightarrow \Cc\right)\Bigr)\simeq  \colimit\bigl(\Cc_{i/x_i}^\tau\rightarrow \Cc_i\bigr)\simeq x_i
	\end{equation*}
	for $i=0,1$, as desired.
	
	It remains to prove the two claims. It's straightforward to check that taking slice-$\infty$-categories commutes with pullbacks. Therefore we have a pullback diagram
	\begin{equation*}
		\begin{tikzcd}
			\ov\Cc_{/x}\rar["\pr_1"]\dar["\pr_0"']\drar[pullback] & \Cc_{1/x_1}^\tau\dar["\alpha_1"]\\
			\Cc_{0/x_0}^\tau\rar["\alpha_0"] & \Cc_{2/x_2}^\tau
		\end{tikzcd}
	\end{equation*}
	Since $\Cc_i\simeq \cat{Ind}_\tau(\Cc_i^\tau)$ for $i=0,1,2$, the argument from \cref{con:Ind} shows that each $\Cc_{i/x_i}^\tau$ is $\tau$-filtered. Furthermore, the legs of the pullback $\alpha_i\colon \Cc_{i/x_i}^\tau\rightarrow \Cc_{2/x_2}^\tau$ for $i=0,1$ are coinitial functors. Indeed, in general, if $F\colon \Uu\rightarrow \Vv$ is a functor of small $\infty$-categories, $u\in\cat{Ind}_\tau(\Uu)$ and $v$ is the image of $u$ under $\cat{Ind}_\tau(\Uu)\rightarrow \cat{Ind}_\tau(\Vv)$, then $\Uu_{/u}\rightarrow \Vv_{/v}$ is coinitial. To see this, observe that $\cat{Ind}_\tau(\Uu)\rightarrow \cat{Ind}_\tau(\Vv)$ fits into a diagram
	\begin{equation*}
		\begin{tikzcd}
			\Uu\dar["F"']\rar & \cat{Ind}_\tau(\Uu)\dar\rar & \PSh(\Uu)\dar["F_!"]\\
			\Vv\rar & \cat{Ind}_\tau(\Vv)\rar & \PSh(\Vv)
		\end{tikzcd}
	\end{equation*}
	(this follows from \cref{lem:Ind}\cref{enum:IndFreelyGenerated}, \cref{thm:PShFreeCocompletion}, and the fact that $\cat{Ind}_\tau(\Vv)\rightarrow \PSh(\Vv)$ preserves $\tau$-filtered colimits by \cref{lem:Ind}\cref{enum:IndGeneratedUnderFilteredColimits}). By \cref{con:Ind}, $\Uu_{/u}\rightarrow \Uu$ is the right fibration associated to the presheaf $u\colon \Uu^\op\rightarrow \cat{An}$. Hence coinitiality of $\Uu_{/u}\rightarrow \Vv_{/v}$ follows from \cref{lem:KanExtensionForRight}\cref{enum:RightPullbackLeftAdjoint}.
	
	To prove \cref{claim:tauFiltered}, let $\varphi\colon \Ii\rightarrow \ov\Cc_{/x}$ be any functor from a $\tau$-small $\infty$-category. Since $\Cc_{i/x_i}^\tau$ are $\tau$-filtered, we may extend ${\pr_i}\circ\varphi$ to functors $\varphi_i\colon \Ii^\triangleright \rightarrow \Cc_{i/x_i}^\tau$ for $i=0,1$. The problem is that we don't necessarily have an equivalence $\alpha_1\circ\varphi_0\simeq \alpha_0^\tau\circ \varphi_1$, so $\varphi_0$ and $\varphi_1$ can't necessarily be combined into an extension $\varphi^\triangleright\colon \Ii^\triangleright \rightarrow \ov\Cc_{/x}$ of $\varphi$. Instead, $\alpha_1\circ\varphi_0$ and $\alpha_0\circ \varphi_1$ define a functor $\psi\colon \Ii^\triangleright\sqcup_\Ii\Ii^\triangleright\rightarrow \Cc_{2/x_2}^\tau$. Since the target is $\tau$-filtered, we can extend $\psi$ to a functor $\psi^\triangleright\colon (\Ii^\triangleright\sqcup_\Ii\Ii^\triangleright)^\triangleright\rightarrow \Cc_{2/x_2}^\tau$. Restricting along $\Ii\rightarrow \Ii^\triangleright\sqcup_\Ii\Ii^\triangleright$ yields a functor $\varphi_2\colon \Ii^\triangleright \rightarrow \Cc_{2/x_2}^\tau$ which extends ${\pr_2}\circ\varphi$. The problem with this extension is that it doesn't necessarily factor through $\Cc_{0/x_0}^\tau$ and $\Cc_{1/x_1}^\tau$. So it seems that no matter what we do, we either end up with $\varphi_1$ and $\varphi_2$, which factor as desired but need not be compatible, or with $\psi$, which is compatible (with itself) but need not factor as desired. The crucial idea is now that we can get the best of both worlds by iterating!
	
	To make this work, first observe that $\psi^\triangleright$ defines natural transformations $\eta_0\colon\alpha_1\circ\varphi_0\Rightarrow \eta_2$ and $\eta_1\colon\alpha_0\circ\varphi_1\Rightarrow \varphi_2$. For $i=0,1,2$ let $y_i$ be the image of the tip $*\in\Ii^\triangleright$ under $\varphi_i$. Then $\psi^\triangleright$ defines morphisms $\alpha_i(y_i)\rightarrow y_2$ for $i=0,1$ and the natural transformations $\eta_i$ are necessarily given by postcomposition with these morphisms. Since $\alpha_i\colon \Cc_{i/x_i}^\tau\rightarrow \Cc_{2/x_2}^\tau$ is coinitial, we find $y_i'\in \Cc_{i/x_i}^\tau$ and a morphism $y_2\rightarrow \alpha_i(y_i')$ in $\Cc_{2/x_2}^\tau$ for $i=0,1$. Applying the same argument to $\alpha_i\colon(\Cc_{i/x_i}^\tau)_{y_i/}\rightarrow (\Cc_{2/x_2}^\tau)_{\alpha_i(y_i)/}$, which is still coinitial by \cref{lem:FilteredCofinal}, we can even arrange that a morphism $y_i\rightarrow y_i'$ exists in such a way that its image under $\alpha_i$ is the composition $\alpha_i(y_i)\rightarrow y_2\rightarrow \alpha_i(y_i')$. Postcomposition with $y_i\rightarrow y_i'$ yields another extension $\varphi_i'\colon \Ii^\triangleright\rightarrow \Cc_{i/x_i}^\tau$ of ${\pr_i}\circ\varphi$ together with a natural transformation $\vartheta_i\colon \varphi_i\Rightarrow \varphi_i'$. So now we are in the same situation as before, except that $\varphi_1$, $\varphi_2$ have been replaced by $\varphi_1'$, $\varphi_2'$. Which means we can iterate!
	
	Iterating the construction $\varphi_i\mapsto \varphi_i'$ transfinitely%
	%
	\footnote{The above construction only takes care of the transfinite induction step for successor ordinals. But the case of limit ordinals is entirely analogous.}
	many times, we obtain $\varphi_i^{(\gamma)}\colon \Ii^\triangleright\rightarrow \Cc_{i/x_i}^\tau$ for all ordinals $\gamma<\kappa$. Now define
	\begin{equation*}
		\varphi_i^\triangleright\coloneqq \colimit_{\gamma<\kappa}\varphi_i^{(\gamma)}\colon \Ii^\triangleright \rightarrow \Cc_{i/x_i}^\tau\,.
	\end{equation*}
	This colimit exists in $\Fun(\Ii^\triangleright,\Cc_{i/x_i}^\tau)$ since $\{\gamma<\kappa\}$ is $\kappa$-filtered and $\tau$-small. Furthermore, using that $\alpha_i\colon \Cc_{i/x_i}^\tau\rightarrow \Cc_{2/x_2}^\tau$ preserve $\kappa$-filtered $\tau$-small colimits, it's straightforward to check that now indeed $\alpha_1\circ\varphi_0^\triangleright\simeq \alpha_0\circ\varphi_1^\triangleright$ holds, so we obtain our desired extension $\varphi^\triangleright\colon \Ii^\triangleright\rightarrow \ov\Cc_{/x}$. This finishes the proof of \cref{claim:tauFiltered}.
	
	To show that $\pr_0\colon \ov\Cc_{/x}\rightarrow \Cc_{0/x_0}^\tau$ is coinitial, we use \cref{thm:JoyalsQuillenA}\cref{enum:WeaklyContractible}: For all $z_0\in\Cc_{0/x_0}^\tau$, we wish to show that $\ov\Cc_{/x}\times_{\Cc_{0/x_0}^\tau}(\Cc_{0/x_0}^\tau)_{z_0/}$ is weakly contractible. We'll even show that it is $\tau$-filtered, which is enough by \cref{lem:FilteredCofinal}. To this end, consider any functor
	\begin{equation*}
		\varphi\colon \Ii\longrightarrow \ov\Cc_{/x}\times_{\Cc_{0/x_0}^\tau}\bigl(\Cc_{0/x_0}^\tau\bigr)_{z_0/}
	\end{equation*}
	from a $\tau$-small $\infty$-category. Equivalently, $\varphi$ is a functor $\ov\varphi\colon \Ii\rightarrow \ov\Cc_{/x}$ together with an extension of ${\pr_0}\circ \ov\varphi$ to a functor $\varphi_0^\triangleleft\colon \Ii^\triangleleft\rightarrow \Cc_{0/x_0}^\tau$ that sends the tip $*\in \Ii^\triangleleft$ to $z_0$. By \cref{claim:tauFiltered}, we find an extension $\ov\varphi^\triangleright \colon \Ii^\triangleright\rightarrow \ov\Cc_{/x}$. The pushout of ${\pr_0}\circ\ov\varphi^\triangleright$ and $\ov\varphi_0^\triangleleft$ defines a functor $(\varphi_0^\triangleright)^\triangleleft\colon (\Ii^\triangleright)^\triangleleft \rightarrow \Cc_{0/x_0}^\tau$. Together, $\ov\varphi^\triangleright$ and $(\varphi_0^\triangleright)^\triangleleft$ define the desired extension $\varphi^\triangleright$ of $\varphi$. This shows that $\pr_0\colon \ov\Cc_{/x}\rightarrow \Cc_{0/x_0}^\tau$ is coinitial. The argument for $\pr_1\colon \ov\Cc_{/x}\rightarrow \Cc_{1/x_1}^\tau$ is analogous and thus we've shown \cref{claim:ProjectionsCoinitial}.
\end{proof}
\begin{proof}[Proof sketch of \cref{lem:PrLColimits}]
	The equivalence from \cref{cor:ExtractingAdjoints}\cref{enum:CatLCatR} (applied to $\widehat{\cat{Cat}}_\infty$ rather than $\cat{Cat}_\infty$) restricts to an equivalence $\cat{Pr}^\L\simeq (\cat{Pr}^\R)^\op$. In particular, colimits in $\cat{Pr}^\L$ are just limits in $\cat{Pr}^\R$ and vice versa. So it suffices to study limits in either case. By \cref{lem:ColimitsIffCoproductsAndPushouts} we can reduce to products and pullbacks.
	
	We start with products. Let $(\Cc_i)_{i\in I}$ be an collection of presentable $\infty$-categories and let $\Cc\coloneqq \prod_{i\in I}\Cc_i$. From \cref{lem:AccessibilityOfFibreProducts} we know that $\Cc$ is accessible and from \cref{lem:HomInLimits}\cref{enum:ColimitsInLimits} (and its dual) we know that $\Cc$ has all colimits and that the projections $\pr_i\colon \Cc\rightarrow \Cc_i$ preserve all limits and colimits. It's then straightforward to verify that $\Cc$ satisfies the universal property of the product in both $\cat{Pr}^\L$ and $\cat{Pr}^\R$ (compare this to footnote~\cref{footnote:LimitNonFullSubcategory} on page~\cpageref{footnote:LimitNonFullSubcategory}).
	
	It remains to do pullbacks. If $\Cc_0\rightarrow \Cc_2\leftarrow \Cc_1$ are functors in $\cat{Pr}^\L$ or $\cat{Pr}^\R$, then they preserve sufficiently filtered colimits (for $\cat{Pr}^\R$ this needs \cref{lem:RightAdjointsAccessible}) and so the pullback $\Dd\coloneqq\Cc_0\times_{\Cc_2}\Cc_1$ is accessible by \cref{lem:AccessibilityOfFibreProducts}. In the case of $\cat{Pr}^\L$, \cref{lem:HomInLimits}\cref{enum:ColimitsInLimits} shows that $\Cc$ has all colimits, so it is presentable, and that the projections $\pr_i\colon \Cc\rightarrow \Cc_i$ preserve all colimits, so they are functors in $\cat{Pr}^\L$. The required universal property in $\cat{Pr}^\L$ is then straightforward to verify. In the case of $\cat{Pr}^\R$, \cref{lem:HomInLimits}\cref{enum:ColimitsInLimits} shows that $\Cc$ has all limits, hence it is presentable by \cref{cor:PresentableComplete}\cref{enum:AccessibleCocomplete}, and that the projections $\pr_i\colon \Cc\rightarrow \Cc_i$ preserve all limits and sufficiently filtered colimits, so they are functors in $\cat{Pr}^\R$ by \cref{thm:AdjointFunctorTheorem}\cref{enum:AdjointFunctorTheoremRight}. Again, the required universal property in $\cat{Pr}^\R$ is then straightforward to verify.
\end{proof}
Let us now explain how to construct $\cat{Pr}^\L$ in ZFC. To this end, we need to introduce a variant of $\cat{Pr}^\L$. As we'll see in \cref{lem:Mindblow}, this variant has a truly mindblowing property, which makes it quite interesting on its own.
\begin{defi}\label{def:PrKappa}
	Let $\kappa$ be a regular cardinal.
	\begin{alphanumerate}
		\item A presentable $\infty$-category is called \emph{$\kappa$-compactly generated} if it is $\kappa$-accessible, that is, of the form $\Cc\simeq \cat{Ind}_\kappa(\Cc^\kappa)$, where $\Cc^\kappa\subseteq \Cc$ is the full sub-$\infty$-category spanned by the $\kappa$-compact objects.
		\item We let $\cat{Pr}_\kappa^\L$ be the non-full sub-$\infty$-category of $\cat{Pr}^\L$ spanned by the $\kappa$-compactly generated $\infty$-categories and those left adjoint functors that also preserve $\kappa$-compact objects.
	\end{alphanumerate} 
\end{defi}
\begin{lem}\label{lem:LeftAdjointPreservesCompactsRightAdjointPreservesFiltered}
	A left adjoint functor $F\colon \Cc\rightarrow \Dd$ between presentable $\infty$-categories preserves $\kappa$-compact objects if the right adjoint $G\colon \Dd\rightarrow \Cc$ preserves $\kappa$-filtered colimits. The converse is true as well, provided $\Cc$ is $\kappa$-compactly generated.
\end{lem}
\begin{proof}
	Let $x\in \Cc$ be $\kappa$-compact and let $y_{(-)}\colon \Ii\rightarrow \Dd$ be a $\kappa$-filtered diagram. If $G$ preserves $\kappa$-filtered colimits, then
	\begin{align*}
		\Hom_\Dd\Bigl(F(x),\colimit_{i\in\Ii}y_i\Bigr)&\simeq  \Hom_\Cc\Bigl(x,\colimit_{i\in\Ii}G(y_i)\Bigr)\\
		&\simeq \colimit_{i\in\Ii}\Hom_\Cc\bigl(x,G(y_i)\bigr)\\
		&\simeq \colimit_{i\in\Ii}\Hom_\Dd\bigl(F(x),y_i\bigr)\,,
	\end{align*}
	proving that $F(x)$ is $\kappa$-compact. Conversely, if $F$ preserves $\kappa$-compact objects, then the same calculation run backwards shows that the canonical morphism $\colimit_{i\in\Ii}G(y_i)\rightarrow G(\colimit_{i\in \Ii}y_i)$ becomes an equivalence after applying the functors $\Hom_\Cc(x,-)\colon \Cc\rightarrow \cat{An}$ for every $\kappa$-compact object $x\in \Cc^\kappa$. These functors are jointly conservative if $\Cc$ is $\kappa$-compactly generated. 
\end{proof}
\begin{numpar}[Constructing $\cat{Pr}^\L$ in ZFC.]\label{par:PrLInZFC}
	By \cref{rem:Accessible}, we have an inclusion $\cat{Pr}_\kappa^\L\subseteq \cat{Pr}_\lambda^\L$ of non-full sub-$\infty$-categories of $\cat{Pr}^\L$ for all regular cardinals $\lambda>\kappa$. By \cref{lem:Accessible}, every presentable $\infty$-category is $\kappa$-compactly generated for all sufficiently large regular cardinals $\kappa$. Furthermore, by \cref{lem:LeftAdjointPreservesCompactsRightAdjointPreservesFiltered} and \cref{lem:RightAdjointsAccessible}, every left adjoint functor $F\colon \Cc\rightarrow \Dd$ between presentable $\infty$-categories preserves $\kappa$-compact objects for sufficiently large $\kappa$. Therefore we can write
	\begin{equation*}
		\cat{Pr}^\L\simeq \bigcup_{\kappa}\cat{Pr}^\L_\kappa\,,
	\end{equation*}
	where the union is taken over all cardinals which are small with respect to our two nested universes. The union can be made precise as a colimit in $\widehat{\cat{Cat}}_\infty$, the $\infty$-category of all large $\infty$-categories, but it can also be viewed as simply a union of simplices in every degree.  In any case, we see that every functor $T\colon \Ii\rightarrow \cat{Pr}^\L$ from a small $\infty$-category $\Ii$ factors through $\cat{Pr}_\kappa^\L$ for all sufficiently large cardinals $\kappa$.
	
	So it suffices to explain how $\cat{Pr}_\kappa^\L$ can be constructed in ZFC. In \cref{lem:PrLKappa}\cref{enum:PrLKappaInZFC} below, we see that, for uncountable $\kappa$, $\cat{Pr}_\kappa^\L$ can be identified with a non-full sub-$\infty$-category of $\cat{Cat}_\infty$ in a non-trivial way. This can be viewed as an alternative construction of $\cat{Pr}_\kappa^\L$, thus providing a construction within the confines of ZFC.
\end{numpar}
\begin{lem}\label{lem:PrLKappa}
	Let $\kappa$ be an uncountable regular cardinal and let $\cat{Cat}_\infty^{\kappa\mhyph\!\colimit}\subseteq \cat{Cat}_\infty$ be the non-full sub-$\infty$-category spanned by those small $\infty$-categories that have all $\kappa$-small colimits and those functors that preserve all $\kappa$-small colimits. Let $(-)^\kappa$ be the functor that sends a $\kappa$-compactly generated presentable $\infty$-category $\Dd$ to its full sub-$\infty$-category $\Dd^\kappa$ spanned by the $\kappa$-compact objects.%
	%
	\footnote{We should explain how to construct this functor. In general, given a functor $F\colon \Cc\rightarrow \cat{Cat}_\infty$, it's easy to construct \emph{subfunctors} of $F$. Indeed, suppose for every $x\in \Cc$ we're given a sub-$\infty$-category $F_0(x)\subseteq F(x)$ in the sense of \cref{par:SubQuasiCategories}. We don't assume $F_0(x)$ to be a full sub-$\infty$-category, although in the case at hand it is. If, for every morphism $\alpha\colon x\rightarrow y$ in $F$, the functor $F(\alpha)\colon F(x)\rightarrow F(y)$ restricts to a functor $F_0(x)\rightarrow F_0(y)$, then we automatically get a functor $F_0\colon \Cc\rightarrow\cat{Cat}_\infty$ together with a natural transformation $F_0\Rightarrow F$. Indeed, let $p\colon \Uu\rightarrow \Cc$ be the unstraightening of $F$ and let $S_1$ be the set of morphisms that can be written as $\varphi_0\circ \varphi$, where $\varphi_0$ is a morphism in some $F_0(x)$ and $\varphi$ is a cocartesian morphism in $\Uu$. Then $S_1$ contains all identities and is closed under equivalences. Furthermore, our assumption on $F_0$ guarantees that $S_1$ is closed under compositions. Let $\Uu_0\coloneqq \Uu[S_1]\subseteq \Uu$ be the (not necessarily full) sub-$\infty$-category spanned by $S_1$. Then $p_0\colon \Uu_0\rightarrow\Cc$ given as the restriction of $p$ is still a cocartesian fibration, since we've ensured that $\Uu_0$ is closed under $p$-cocartesian lifts in $\Uu$. Thus, we can define $F_0$ as the straightening of $p_0$.
		
	In the case at hand, the identity on $\cat{Pr}_\kappa^\L$ can be viewed as a functor $\cat{Pr}_\kappa^\L\rightarrow \widehat{\cat{Cat}}_\infty$ and we can construct $(-)^\kappa$ as a subfunctor of it.}%
	%
	\begin{alphanumerate}
		\item If $\Cc$ is a small $\infty$-category with all $\kappa$-small colimits and $\Dd$ is a presentable $\infty$-category, then restriction along $\Yo_\Cc\colon \Cc\rightarrow \cat{Ind}_\kappa(\Cc)$ induces an equivalence of $\infty$-categories\label{enum:FunLKappa}
		\begin{equation*}
			\Yo_\Cc^*\colon \Fun_\kappa^\L\bigl(\cat{Ind}_\kappa(\Cc),\Dd\bigr)\overset{\simeq}{\longrightarrow} \Fun^{\kappa\mhyph\!\colimit}(\Cc,\Dd^\kappa)\,.
		\end{equation*}
		Here $\Fun^\L_\kappa\subseteq \Fun$ is the full sub-$\infty$-category spanned by left adjoint functors that preserve $\kappa$-compact objects and $\Fun^{\kappa\mhyph\!\colimit}\subseteq \Fun$ is spanned by $\kappa$-small colimit-preserving functors.
		\item If $\Cc$ is a small $\infty$-category with all $\kappa$-small colimits, then $\Cc\simeq \cat{Ind}_\kappa(\Cc)^\kappa$.\label{enum:KappaCompactObjects}
		\item The functor $(-)^\kappa$ induces an equivalence of $\infty$-categories\label{enum:PrLKappaInZFC}
		\begin{equation*}
			(-)^\kappa\colon \cat{Pr}_\kappa^\L\overset{\simeq}{\longrightarrow}\cat{Cat}_\infty^{\kappa\mhyph\!\colimit}\,.
		\end{equation*} 
	\end{alphanumerate}
\end{lem}
\begin{proof}
	We begin with~\cref{enum:KappaCompactObjects}. It's clear that the objects $\{\Yo_\Cc(x)\ \vert\ x\in\Cc\}$ form a set of $\kappa$-compact generators of $\cat{Ind}_\kappa(\Cc)$. As we've seen in the proof of \cref{lem:KappaCompactlyGenerated}, this means that every $\kappa$-compact object of $\cat{Ind}_\kappa(\Cc)$ is a retract of an object in $\{\Yo_\Cc(x)\ \vert\ x\in\Cc\}$. As we've seen in footnote~\cref{footnote:ReflectionTheorem} on \cpageref{footnote:ReflectionTheorem}, retracts can be written as countable colimits. Furthermore, we've seen in the proof of \cref{lem:Ind} that $\Yo_\Cc\colon \Cc\rightarrow\cat{Ind}_\kappa(\Cc)$ preserves $\kappa$-small colimits; in particular, $\Yo_\Cc$ preserves countable colimits, as $\kappa$ is assumed uncountable. Since we assume that $\Cc$ has all $\kappa$-small colimits, it follows that $\{\Yo_\Cc(x)\ \vert\ x\in\Cc\}$ is closed under retracts and thus comprises all $\kappa$-compact objects. The claim follows.
	
	To prove \cref{enum:FunLKappa}, our starting point is the equivalence $\Fun^{\kappa\mhyph\mathrm{filt}}(\cat{Ind}_\kappa(\Cc),\Dd)\simeq \Fun(\Cc,\Dd)$ from \cref{lem:Ind}\cref{enum:IndFreelyGenerated}. So we only have to match full sub-$\infty$-categories on either side. Suppose a functor $F\colon \cat{Ind}_\kappa(\Cc)\rightarrow \Dd$ preserves $\kappa$-compact objects. Then the associated functor $\Cc\rightarrow \Dd$ factors through a functor $G\colon \Cc\rightarrow \Dd^\kappa$. Furthermore, $\Dd^\kappa\subseteq \Dd$ is closed under $\kappa$-small colimits and so is $\Cc\simeq \cat{Ind}_\kappa(\Cc)^\kappa\subseteq \cat{Ind}_\kappa(\Cc)$ by \cref{enum:KappaCompactObjects}. Thus, if $F$ preserves all colimits, then $G$ preserves $\kappa$-small colimits. Conversely, suppose we're given a functor $G\colon \Cc\rightarrow \Dd^\kappa$ that preserves $\kappa$-small colimits. Let $F\colon \cat{Ind}_\kappa(\Cc)\rightarrow \Dd$ be the associated functor. By construction, $F$ preserves $\kappa$-compact objects and $\kappa$-filtered colimits. Furthermore, we've seen in the proof of \cref{lem:Presentable} that arbitrary colimits in $\cat{Ind}_\kappa(\Cc)$ can be built from $\kappa$-filtered colimits as well as $\kappa$-small colimits of objects in the image of $\Yo_\Cc\colon \Cc\rightarrow \cat{Ind}_\kappa(\Cc)$. Thus $F$ preserves all colimits. This finishes the proof of \cref{enum:FunLKappa}.
	
	Finally, \cref{enum:PrLKappaInZFC} is a formal consequence:~\cref{enum:KappaCompactObjects} shows that $(-)^\kappa$ is essentially surjective, and~\cref{enum:FunLKappa}, together with \cref{lem:Presentable}\cref{enum:DCHasKappaSmallColimits}, shows that $(-)^\kappa$ is fully faithful.
\end{proof}
\begin{cor}\label{cor:PrLKappaColimits}
	Let $\lambda >\kappa$ be regular cardinals.
	\begin{alphanumerate}
		\item The inclusion $\cat{Pr}_\kappa^\L\subseteq \cat{Pr}_\lambda^\L$ admits a right adjoint. On objects, it sends $\Dd\in\cat{Pr}_\lambda^\L$ to $\cat{Ind}_\kappa(\Dd^\lambda)$.\label{enum:PrLKappaLambda}
		\item The $\infty$-category $\cat{Pr}_\kappa^\L$ has all small limits and colimits. The forgetful functor $\cat{Pr}_\kappa^\L\rightarrow \cat{Pr}^\L$ preserves all small colimits.\label{enum:PrLKappaColimits}%
		%
		\footnote{\label{footnote:ProductsInPrLKappa}It's true that any product of $\kappa$-compactly generated $\infty$-categories in $\cat{Pr}^\L$ is $\kappa$-compactly generated again; we'll see this in the proof of \cref{cor:PrLKappaColimits}\cref{enum:PrLKappaColimits}. But, confusingly, the product in $\cat{Pr}^\L$ is usually not the product in $\cat{Pr}_\kappa^\L$. The reason is that preservation of limits or colimits can be detected factor-wise, but not preservation of $\kappa$-compact objects.}
	\end{alphanumerate}
\end{cor}
\begin{proof}
	We start with \cref{enum:PrLKappaLambda}. Let $\Dd\in \cat{Pr}_\lambda^\L$. By \cref{lem:Ind}\cref{enum:IndFreelyGenerated}, the identity functor $\id_{\Dd^\lambda}\colon \Dd^\lambda\rightarrow \Dd^\lambda$ induces a $\kappa$-filtered colimit-preserving functor $c_\Dd\colon \cat{Ind}_\kappa(\Dd^\lambda)\rightarrow \Dd$. Let's first argue why $c_\Dd$ is a functor in $\cat{Pr}_\lambda^\L$. Since $\Dd^\lambda$ has all $\kappa$-small colimits and $\id_{\Dd^\lambda}$ preserves them, the same argument as in the proof of \cref{lem:PrLKappa}\cref{enum:FunLKappa} shows that $c_\Dd$ preserves all colimits. Furthermore, the $\lambda$-compact objects of $\cat{Ind}_\kappa(\Dd^\lambda)$ are precisely those generated under $\lambda$-small colimits by objects in the image of $\Yo_{\Dd^\lambda}\colon \Dd^\lambda\rightarrow\cat{Ind}_\kappa(\Dd^\lambda)$. Indeed, if $\overline{\Dd}$ denotes the full sub-$\infty$-category spanned by these objects, then the proof of \cref{lem:KappaCompactlyGenerated}\cref{enum:CompactGenerators} yields an equivalence $\cat{Ind}_\lambda(\overline{\Dd})\simeq \cat{Ind}_\kappa(\Dd^\lambda)$; now apply \cref{lem:PrLKappa}\cref{enum:KappaCompactObjects}. Since $c_\Dd$ preserves all colimits and $\Dd^\lambda\subseteq \Dd$ is closed under $\lambda$-small colimits, it follows that all $\lambda$-compact objects of $\cat{Ind}_\kappa(\Dd^\lambda)$ land in $\Dd^\lambda$. This proves that $c_\Dd$ is indeed a functor in $\cat{Pr}_\lambda^\L$.
	
	To construct the desired right adjoint, it's now enough by \cref{lem:Adjunction} to show that the functor $(c_\Dd)_*\colon \Fun^\L_\kappa(\Cc,\cat{Ind}_\kappa(\Dd^\lambda))\longrightarrow\Fun^\L_\lambda(\Cc,\Dd)$ for all $\Cc\in \cat{Pr}_\kappa^\L$, given by postcomposition with $c_\Dd$, is an equivalence of $\infty$-categories. This functor fits into the following diagram:
	\begin{equation*}
		\begin{tikzcd}
			\Fun^\L_\kappa\bigl(\Cc,\cat{Ind}_\kappa(\Dd^\lambda)\bigr)\rar["(c_\Dd)_*"]\dar\drar[commutes] & \Fun^\L_\lambda(\Cc,\Dd)\dar\\
			\Fun\bigl(\Cc^\kappa,\cat{Ind}_\kappa(\Dd^\lambda)^\kappa\bigr)\rar["\simeq"] & \Fun(\Cc^\kappa,\Dd^\lambda)
		\end{tikzcd}
	\end{equation*}
	The vertical arrows are given by restriction along $\Cc^\kappa\subseteq \Cc$. By \cref{lem:PrLKappa}, the left vertical arrow is fully faithful and the bottom arrow is an equivalence. The right vertical arrow is fully faithful by \cref{lem:Ind}\cref{enum:IndFreelyGenerated}, using $\Cc\simeq \cat{Ind}_\kappa(\Cc^\kappa)$. It follows that $(c_\Dd)_*$ must be fully faithful too. Furthermore, by \cref{lem:PrLKappa}, the essential image of $\Fun_\kappa^\L(\Cc,\cat{Ind}_\kappa(\Dd^\lambda))\rightarrow \Fun(\Cc^\kappa,\Dd^\lambda)$ is spanned by those functors that preserve $\kappa$-small colimits. Since $\Cc^\kappa\subseteq \Cc$ and $\Dd^\lambda\subseteq \Dd$ are closed under $\kappa$-small colimits, the essential image of $\Fun_\lambda^\L(\Cc,\Dd)\rightarrow \Fun(\Cc^\kappa,\Dd^\lambda)$ must also be contained in the $\kappa$-small colimit-preserving functors. This shows that $(c_\Dd)_*$ is essentially surjective and we've finished the proof of \cref{enum:PrLKappaLambda}.
	
	The existence of limits in $\cat{Pr}_\kappa^\L$ follows from \cref{lem:Mindblow} below combined with \cref{cor:PresentableComplete}\cref{enum:PresentableComplete}. For colimits, let $\cat{Pr}_\kappa^\R$ be the $\infty$-category of all $\kappa$-compactly generated presentable $\infty$-categories and right adjoint functors that preserve $\kappa$-filtered colimits. Then \cref{cor:ExtractingAdjoints}\cref{enum:CatLCatR} and \cref{lem:LeftAdjointPreservesCompactsRightAdjointPreservesFiltered} show $\cat{Pr}_\kappa^\L\simeq (\cat{Pr}_\kappa^\R)^\op$. Thus it's enough to check that $\cat{Pr}_\kappa^\R$ is closed under limits in $\cat{Pr}^\R$.\footnote{As in the proof of \cref{lem:PrLColimits}, a straightforward extra-argument is needed since $\cat{Pr}_\kappa^\L$ is not a full sub-$\infty$-category of $\cat{Pr}^\R$. As preservation of limits and $\kappa$-filtered colimits can be checked factor-wise, we don't run into the same issue as in footnote~\cref{footnote:ProductsInPrLKappa}} As usual, it's enough to do products and pullbacks. In either case, we know from \cref{lem:PrLColimits} that the limit in $\widehat{\cat{Cat}}_\infty$ is also the limit in $\cat{Pr}^\R$, so by \cref{lem:KappaCompactlyGenerated}\cref{enum:CompactGenerators} we only need to construct a set of $\kappa$-compact generators. For a product $\prod_{i\in I}\Cc_i$, choose a set $S_i$ of $\kappa$-compact generators of $\Cc_i$ for all $i\in I$. Furthermore, choose an initial object $\emptyset_i\in \Cc_i$. For all $s_i\in S_i$ let $e_i(s_i)\in \prod_{i\in I}\Cc_i$ be the object given by $e_i(s_i)_i=s_i$ and $e_i(s_i)_j=\emptyset_j$ for $j\neq i$. Then $\left\{e_i(s_i)\ \middle|\ s_i\in S_i\right\}$ are $\kappa$-compact and jointly detect equivalences in the $i$\textsuperscript{th} component. Hence the union $\bigcup_{i\in I}\left\{e_i(s_i)\ \middle|\ s_i\in S_i\right\}$ is a set of $\kappa$-compact generators of $\prod_{i\in I}\Cc_i$.
	
	The argument for pullbacks is similar. Let $\Cc\coloneqq\Cc_0\times_{\Cc_2}\Cc_1$ be a pullback in $\cat{Pr}^\R$, where the underlying diagram is already contained in $\cat{Pr}_\kappa^\R$. By definition of $\cat{Pr}^\R$, the pullback projections $\pr_0\colon \Cc\rightarrow \Cc_0$ and $\pr_1\colon \Cc\rightarrow \Cc_1$ admit left adjoints $L_0\colon \Cc_0\rightarrow\Cc$ and $L_1\colon \Cc_1\rightarrow \Cc$. By \cref{lem:HomInLimits}\cref{enum:ColimitsInLimits}, the projections $\pr_0$ and $\pr_1$ preserve $\kappa$-filtered colimits, hence \cref{lem:LeftAdjointPreservesCompactsRightAdjointPreservesFiltered} shows that $L_0$ and $L_1$ preserve $\kappa$-compact objects. Now choose sets $S_0$ and $S_1$ of $\kappa$-compact generators of $\Cc_0$ and $\Cc_1$. Then $\left\{L_0(s_0)\ \middle|\ s_0\in S_0\right\}$ are $\kappa$-compact objects of $\Cc$ and jointly detect equivalences in the first factor. Similarly, $\left\{L_1(s_1)\ \middle|\ s_1\in S_1\right\}$ jointly detect equivalences in the second factor. Hence the union $\left\{L_0(s_0)\ \middle|\ s_0\in S_0\right\}\cup \left\{L_1(s_1)\ \middle|\ s_1\in S_1\right\}$ is a set of $\kappa$-compact generators of $\Cc$. This finishes the proof of \cref{enum:PrLKappaColimits}.
\end{proof}

To finish this rather lengthy subsection, we'll show the aforementioned mindblowing property of $\cat{Pr}_\kappa^\L$. If you're in a situation where you can fix an uncountable regular cardinal $\kappa$ and only work with $\kappa$-compactly generated $\infty$-categories (for most practical applications, $\kappa=\aleph_1$ is enough), \cref{lem:Mindblow} allows you to bypass all set-theoretic problems.
\begin{thm}[\enquote{$\text{Russel's paradox}=\text{skill issue}$}]\label{lem:Mindblow}
	Let $\kappa$ be an uncountable regular cardinal. Then $\cat{Pr}_\kappa^\L$ is an object of $\cat{Pr}_\kappa^\L$.
\end{thm}
\begin{proof}[Proof sketch]
	We already know from \cref{cor:PrLKappaColimits}\cref{enum:PrLKappaColimits} that $\cat{Pr}_\kappa^\L$ has all colimits. Thus, by \cref{lem:KappaCompactlyGenerated}\cref{enum:CompactGenerators} it's enough to find a set of $\kappa$-compact generators. We'll show that $\{\cat{An},\PSh(\Delta^1)\}$ does it. Let's first show that the functors $\Hom_{\cat{Pr}_\kappa^\L}(\cat{An},-)$ and $\Hom_{\cat{Pr}_\kappa^\L}(\PSh(\Delta^1),-)$ are jointly conservative. To this end, we claim more generally:
	\begin{alphanumerate}\itshape
		\item[\boxtimes_1] If $\Cc$ is a small $\infty$-category and $\Dd$ is a $\kappa$-compactly generated presentable $\infty$-category, then restriction along $\Yo_\Cc\colon \Cc\rightarrow \PSh(\Cc)$ induces an equivalence of $\infty$-categories\label{claim:FunLKappaPSh}
		\begin{equation*}
			\Yo_\Cc^*\colon \Fun_\kappa^\L\bigl(\PSh(\Cc),\Dd\bigr)\overset{\simeq}{\longrightarrow}\Fun(\Cc,\Dd^\kappa)\,.
		\end{equation*}
	\end{alphanumerate}
	Believing \cref{claim:FunLKappaPSh}, we find $\Hom_{\cat{Pr}_\kappa^\L}(\cat{An},\Dd)\simeq \core\Dd^\kappa$ and $\Hom_{\cat{Pr}_\kappa^\L}(\PSh(\Delta^1),\Dd)\simeq \core\Ar(\Dd^\kappa)$. Now $(-)^\kappa\colon \cat{Pr}_\kappa^\L\rightarrow \cat{Cat}_\infty$ is conservative by \cref{lem:PrLKappa}\cref{enum:PrLKappaInZFC}. Furthermore, $\core(-)\colon \cat{Cat}_\infty\rightarrow \cat{An}$ and $\core\Ar(-)\colon \cat{Cat}_\infty\rightarrow \cat{An}$ are jointly conservative, as we've seen in the proof of \cref{cor:AnPresentable}. It follows that $\cat{An}$ and $\PSh(\Delta^1)$ are generators of $\cat{Pr}_\kappa^\L$, as desired.
	
	To prove \cref{claim:FunLKappaPSh}, recall $\Fun^\L(\PSh(\Cc),\Dd)\simeq \Fun(\Cc,\Dd)$ from \cref{thm:PShFreeCocompletion}, so we only have to find out to which full sub-$\infty$-category $\Fun_\kappa^\L\subseteq \Fun^\L$ corresponds in $\Fun(\Cc,\Dd)$. Since $\{\Yo_\Cc(x)\ \vert\ x\in\Cc\}$ is a set of generators for $\PSh(\Cc)$, the same argument as in the proof of \cref{lem:PrLKappa}\cref{enum:PrLKappaLambda} shows that the $\kappa$-compact objects in $\PSh(\Cc)$ are precisely those generated under $\kappa$-small colimits from representable presheaves. Thus, a colimit-preserving functor $\PSh(\Cc)\rightarrow \Dd$ also preserves $\kappa$-compact objects if and only if it restricts to a functor $\Cc\rightarrow \Dd^\kappa$. This proves \cref{claim:FunLKappaPSh}.
	
	It remains to show that $\cat{An}$ and $\PSh(\Delta^1)$ are $\kappa$-compact in $\cat{Pr}_\kappa^\L$. As explained above, \cref{claim:FunLKappaPSh} shows $\Hom_{\cat{Pr}_\kappa^\L}(\cat{An},-)\simeq \core((-)^\kappa)$ and $\Hom_{\cat{Pr}_\kappa^\L}(\PSh(\Delta^1),-)\simeq \core\Ar((-)^\kappa)$. We know from \cref{lem:PrLKappa}\cref{enum:PrLKappaInZFC} that $(-)^\kappa\colon \cat{Pr}_\kappa^\L\rightarrow \cat{Cat}_\infty^{\kappa\mhyph\!\colimit}$ is an equivalence of $\infty$-categories, so it preserves all $\kappa$-filtered colimits, and we've sketched in the proof of \cref{cor:AnPresentable} that $\core(-)\colon \cat{Cat}_\infty\rightarrow \cat{An}$ and $\core\Ar(-)\colon \cat{Cat}_\infty\rightarrow \cat{An}$ preserve filtered colimits. Therefore, to finish the proof it's enough to show the following claim about $\kappa$-filtered colimits in $\cat{Cat}_\infty^{\kappa\mhyph\!\colimit}$.
	\begin{alphanumerate}\itshape
		\item[\boxtimes_2] The forgetful functor $\cat{Cat}_\infty^{\kappa\mhyph\!\colimit}\rightarrow\cat{Cat}_\infty$ preserves $\kappa$-filtered colimits.\label{claim:CatInftyKappaColimits}
	\end{alphanumerate}
	To prove~\cref{claim:CatInftyKappaColimits}, let $\Cc_{(-)}\colon \Jj\rightarrow \cat{Cat}_\infty^{\kappa\mhyph\!\colimit}$ be a $\kappa$-filtered diagram. We have to show that the colimit $\colimit_{j\in\Jj}\Cc_j$ in $\cat{Cat}_\infty$ also has all $\kappa$-small colimits and constitutes a colimit in $\cat{Cat}_\infty^{\kappa\mhyph\!\colimit}$. So let $\Ii$ be a $\kappa$-small $\infty$-category and let $T\colon \Ii\rightarrow \colimit_{j\in\Jj}\Cc_j$ be an $\Ii$-shaped diagram in $\colimit_{j\in\Jj}\Cc_j$. It's not hard to check that for uncountable regular cardinals $\kappa$ the $\kappa$-compact objects in $\cat{Cat}_\infty$ are precisely the $\kappa$-small $\infty$-categories.%
	%
	\footnote{Since $\cat{Cat}_\infty$ is generated by the compact objects $*$ and $\Delta^1$, hence the $\kappa$-compact objects are precisely those $\infty$-categories generated under $\kappa$-small colimits by $*$ and $\Delta^1$. Everyone of them is $\kappa$-small by \cref{rem:KappaSmallClosedUnderPushouts}. Conversely, for all $n\geqslant 0$, the $\infty$-category $\Delta^n$ can be written as a finite colimit in $*$ and $\Delta^1$ (namely, as the iterated pushout $\Delta^{\{0,1\}}\sqcup_{\{1\}}\dotsb \sqcup_{\{n-1\}}\Delta^{\{n-1,n\}}$). Using \cref{lem:KappaSmall}\cref{enum:KappaSmallC}, every other $\kappa$-small $\infty$-category is contained in the full sub-$\infty$-category of $\cat{Cat}_\infty$ generated under $\kappa$-small colimits by $\{\Delta^n\ \vert\ n\geqslant 0\}$.}
	%
	Hence $\Ii$ is $\kappa$-compact and so $T$ factors through a functor $T_0\colon \Ii\rightarrow \Cc_{j_0}$ for some $j_0\in\Jj$. By assumption, $\Cc_{j_0}$ has all $\kappa$-small colimits and so $T_0$ extends to a colimit cone $T_0^\triangleright\colon \Ii^\triangleright\rightarrow \Cc_{j_0}$. Furthermore, for every morphism $j_0\rightarrow j$ in $\Jj$ the functor $\Cc_{j_0}\rightarrow \Cc_j$ preserves $\kappa$-small colimits, hence $\Ii^\triangleright\rightarrow \Cc_{j_0}\rightarrow \Cc_j$ is still a colimit cone. We claim that then also $\Ii^\triangleright\rightarrow \Cc_{j_0}\rightarrow \colimit_{j\in \Jj}\Cc_j$ is a colimit cone. To show this, we need an analogue of \cref{lem:HomInLimits}\cref{enum:HomInLimits} for filtered colimits; this can be obtained in the exact same way, using that $\Hom_{\cat{Cat}_\infty}(\Delta^1,-)$ and $\Hom_{\cat{Cat}_\infty}(*\ \,*,-)$ also preserve filtered colimits and that pullbacks in $\cat{An}$ commute with filtered colimits by \cref{lem:FilteredColimitsPreserveFiniteLimits}. Then \cref{cor:HomPreservesColimits} shows that $\Ii^\triangleright\rightarrow \Cc_{j_0}\rightarrow \colimit_{j\in \Jj}\Cc_j$ is indeed a colimit cone, as claimed. This proves that $\colimit_{j\in \Jj}\Cc_j$ again has all $\kappa$-small colimits.
	
	To finish the proof, it remains to argue why $\colimit_{j\in\Jj}\Cc_j$ is also a colimit in the non-full sub-$\infty$-category $\cat{Cat}_\infty^{\kappa\mhyph\!\colimit}\subseteq \cat{Cat}_\infty$. Unravelling the definitions (and using \cref{lem:NonFullSubcategory}), we must check that for every natural transformation $\eta\colon \Cc_{(-)}\Rightarrow \const \Dd$ in $\cat{Cat}_\infty^{\kappa\mhyph\!\colimit}$ the induced functor $\colimit_{j\in\Jj}\Cc_j\rightarrow \Dd$ in $\cat{Cat}_\infty$ preserves $\kappa$-small colimits. But we've seen above that every $\kappa$-small colimit in $\colimit_{j\in \Jj}\Cc_j$ is inherited from $\Cc_{j_0}$ for some $j_0\in \Jj$ and the functor $\eta_{j_0}\colon \Cc_{j_0}\rightarrow \Dd$ preserves $\kappa$-small colimits by assumption on $\eta$.
\end{proof}

\postsectionappendix

	\section{Towards spectra}\label{sec:TowardsSpectra}
The goal of this section is to introduce the stable $\infty$-category $\cat{Sp}$ of \emph{spectra}. Along the way we'll be able to deduce many classical topological results.
\subsection{Suspensions and loop animae}
\begin{defi}\label{def:Loop}
	Let $X\in\cat{An}$ be an anima or $(X,x)\in\cat{An}_{*/}$ be a pointed anima. We define $\Sigma X$, the \emph{suspension of $X$}, and $\Omega_xX$, the \emph{loop anima of $X$ with basepoint $x$}, via
	\begin{equation*}
		\begin{tikzcd}
			X\rar\dar\drar[pushout] & *\dar\\
			*\rar & \Sigma X
		\end{tikzcd}\quad\text{and}\quad
		\begin{tikzcd}
			\Omega_xX\rar\dar\drar[pullback] & \{x\}\dar\\
			\{x\}\rar & X
		\end{tikzcd}
	\end{equation*}
	respectively, where the pushout and the pullback are taken in $\cat{An}$. If the basepoint is clear from the context, we often simply write $\Omega X$. Note that $\Sigma X$ is canonically a pointed anima via $*\rightarrow \Sigma X$ and $\Omega_xX$ is canonically a pointed anima since the pullback can be taken in $\cat{An}_{*/}$ instead by \cref{lem:ColimitsInSliceCategory}\cref{enum:LimitsInSlice}.
\end{defi}
\begin{rem}
	By model category fact~\cref{par:HomotopyPushout}, to compute $\Sigma X$, we have to replace one $X\rightarrow *$ by a cofibration, then take the usual pushout of simplicial sets, and finally replace the result by a Kan complex.  Such a replacement by a cofibration could be $X\rightarrow X^\triangleright\rightarrow CX$, where $X^\triangleright\rightarrow CX$ is an anodyne map from the cone $X^\triangleright$ from \cref{con:ConeCategory} into a Kan complex (which exists thanks to \cref{lem:SmallObjectArgument}); then $CX$ is contractible because $CX\simeq \left|X^\triangleright\right|\simeq *$. From this description, we see that $\Sigma$ is compatible with the topological suspension functor $\Sigma^{\cat{Top}} \colon \cat{Top}\rightarrow\cat{Top}$ (reduced or unreduced doesn't matter) in the sense that
	\begin{equation*}
		\begin{tikzcd}
			\operatorname{ho}(\cat{An})\rar["\Sigma"]\dar["\left|\,\cdot\,\right|"']\drar[commutes] & \operatorname{ho}(\cat{An})\dar["\left|\,\cdot\,\right|"]\\
			\operatorname{ho}(\cat{Top})\rar["\Sigma^{\cat{Top}}"] & \operatorname{ho}(\cat{Top})
		\end{tikzcd}
	\end{equation*}
	commutes; here $\left|\,\cdot\,\right|\colon \operatorname{ho}(\cat{An})\rightarrow \operatorname{ho}(\cat{Top})$ denotes the geometric realisation functor. So the suspension functor $\Sigma\colon \cat{An}\rightarrow\cat{An}$ deserves its name.
\end{rem}
Next, we'll show that the loop functor $\Omega\colon \cat{An}_{*/}\rightarrow\cat{An}_{*/}$ deserves its name as well.
%	\begin{lem}\label{lem:PullbackInPointedAnimae}
	%		The forgetful functor $\cat{An}_{*/}\rightarrow\cat{An}$ commutes with all limits and with $\Ii$-shaped colimits if $\left|\Ii\right|\simeq *$. In particular, it commutes with pushouts \embrace{since $\left|\Lambda_0^2\right|\simeq *$}.
	%	\end{lem}
%	\begin{proof}[Proof sketch]
	%		For limits, simply observe that $\cat{An}_{*/}\rightarrow\cat{An}$ has a left adjoint $(-)_+\colon \cat{An}\rightarrow\cat{An}_{*/}$ sending $X$ to the disjoint union $X_+\coloneqq X\sqcup *$; this follows immediately from \cref{cor:HomPreservesColimits} and \cref{cor:HomInSliceCategories}. Then \cref{lem:AdjointsPreserveColimits} can be applied.
	%		
	%		For colimits, we use a general fact: If $L\colon \Cc\shortdoublelrmorphism\Dd\noloc R$ is an adjunction of $\infty$-categories and $y\in\Dd$ is an element for which the counit $c_y\colon LR(y)\rightarrow y$ is an equivalence, then we get an induced adjunction $L\colon \Cc_{R(y)/}\shortdoublelrmorphism \Dd_{y/}\noloc R$ on slice $\infty$-categories. Applying this fact to the adjunction $\colimit_\Ii\colon \Fun(\Ii,\cat{An})\shortdoublelrmorphism \cat{An}\noloc \const$ and using $\Fun(\Ii,\cat{An}_{*/})\simeq \Fun(\Ii,\cat{An})_{\const */}$, we see that it suffices to check $\colimit_\Ii\const *\simeq *$. This follows from \cref{lem:ColimitsInAnima}: We have $\colimit_\Ii\const *\simeq \mathopen|\operatorname{Un}^{(\mathrm{left})}(\const *)\mathclose|\simeq \left|\Ii\right|$, and $ \left|\Ii\right|\simeq *$ holds by assumption. 
	%	\end{proof}
\begin{lem}\label{lem:SuspensionLoopAdjunction}
	Suspension and loop form an adjunction $\Sigma\colon \cat{An}_{*/}\shortdoublelrmorphism \cat{An}_{*/}\noloc \Omega$. In particular, for every pointed anima $(X,x)$, the following hold:
	\begin{alphanumerate}
		\item $\pi_n(\Omega_xX,x)\cong \pi_{n+1}(X,x)$ for all $n\geqslant 0$.\label{enum:LoopShiftsHomotopyGroups}
		\item $\Omega_xX\simeq \Hom_{\cat{An}_{*/}}((S^1,*),(X,x))\simeq \Hom_X(x,x)$.\label{enum:LoopIsHom}
	\end{alphanumerate}
\end{lem}
\begin{proof}[Proof sketch]
	The adjunction $\Sigma\dashv\Omega$ follows immediately from \cref{cor:HomPreservesColimits} and the fact that the pushout and pullback diagrams in \cref{def:Loop} can be taken in $\cat{An}_{*/}$ as well by \cref{lem:ColimitsInSliceCategory}.
	
	Part \cref{enum:LoopShiftsHomotopyGroups} follows immediately from the suspension-loop adjunction and $S^{n+1}\simeq \Sigma S^n$. The latter is clear if we define $S^n\coloneqq \Sigma(*\ \,*)$ as the $n$-fold suspension of two points; for any other construction, it is a straightforward check.
	
	The first equivalence in \cref{enum:LoopIsHom} follows from $\Omega_xX\simeq \Hom_{\cat{An}}\left(*,\Omega_xX\right)\simeq \Hom_{\cat{An}_{*/}}\left(*\ \,*,(\Omega_xX,x)\right)$ and $\Sigma(*\ \,*)\simeq S^1$. For the second equivalence, note $\Ar(X)\simeq X$. Indeed, $\Ar(X)\simeq\Fun(\Delta^1,X)$ is already an anima and so $\Fun(\Delta^1,X)\simeq \core \Fun(\Delta^1,X)\simeq \Hom_{\Cat_\infty}(\Delta^1,X)$. Note that $\mathopen|\Delta^1\mathclose|\simeq *$, since $\Delta^1$ has an initial object, and so \cref{exm:Adjunctions}\cref{enum:AnToCatInfty} implies the desired equivalence $\Hom_{\Cat_\infty}(\Delta^1,X)\simeq \Hom_{\cat{An}}(*,X)\simeq X$. Therefore, $(s,t)\colon\Ar(X)\rightarrow X\times X$ is homotopic to the diagonal $\Delta\colon X\rightarrow X\times X$ and so $\Hom_X(x,x)\simeq (\{x\}\times\{x\})\times_{X\times X,\Delta}X$. Now consider the following diagram:
	\begin{equation*}
		\begin{tikzcd}
			%\Omega_xX\rar\dar\drar[pullback]& \{x\}\times\{x\}\dar & \\
			\{x\}\dar\rar\drar[pullback] & \{x\}\times X\rar\dar\drar[pullback] & \{x\}\dar\\
			X\rar["\Delta"] & X\times X\rar["\pr_1"] & X
		\end{tikzcd}
	\end{equation*}
	The right square is a pullback by inspection and the outer rectangle is a pullback because the bottom row $\pr_1\circ \Delta\colon X\rightarrow X$ is the identity on $X$. It follows formally that the left square must be a pullback as well. Finally, consider the following diagram:
	\begin{equation*}
		\begin{tikzcd}
			%\Omega_xX\rar\dar\drar[pullback]& \{x\}\times\{x\}\dar & \\
			\Omega_xX\dar\rar\drar[pullback] & \{x\}\rar\dar\drar[pullback] & X\dar["\Delta"]\\
			\{x\}\rar & \{x\}\times X\rar & X\times X
		\end{tikzcd}
	\end{equation*}
	The right square is a pullback as argued above and the left square is a pullback by \cref{def:Loop}. Now the outer square is a pullback again, which proves $\Omega_xX\simeq \Hom_X(x,x)$, as desired.
\end{proof}
\begin{exm}\label{exm:EilenbergMacLaneAnima}
	For every $n\geqslant 0$, the following is a pullback diagram in $\Dd_{\geqslant 0}(\IZ)$:
	\begin{equation*}
		\begin{tikzcd}
			A[n]\rar\dar\drar[pullback] & 0\dar\\
			0\rar & A[n+1]
		\end{tikzcd}
	\end{equation*}
	(this may seem weird at first, but will become more clear once we discuss stable $\infty$-categories in \cref{subsec:Spectra}; the proof is similar to \cref{lem:ColimitsInDR}\cref{enum:CofibresInDR}). Since the Eilenberg--MacLane anima functor $\K\colon\Dd_{\geqslant 0}(\IZ)\rightarrow\cat{An}$ from \cref{con:Homology} is a right adjoint, it preserves pullbacks by \cref{lem:AdjointsPreserveColimits}, which shows $\K(A,n)\simeq \Omega\! \K(A,n+1)$. This fits prefectly with the fact that the loop functor shifts homotopy groups down by \cref{lem:SuspensionLoopAdjunction}\cref{enum:LoopShiftsHomotopyGroups}.
\end{exm}

\subsection{\texorpdfstring{$\IE_1$}{E-1}-monoids and \texorpdfstring{$\IE_1$}{E1}-groups}\label{subsec:E1}
\begin{numpar}[Associahedra.]\label{par:AssociahedraI}
	What's an associative monoid in the $\infty$-category $\cat{An}$? Clearly, part of the data should be an anima $M$ together with a multiplication $\mu\colon M\times M\rightarrow M$. We'll often write we put $a\cdot b\coloneqq \mu(a,b)$ for convenience.
	
	Intuitively, associativity means that for every $n\geqslant 3$ and all $a_1,\dotsc,a_n\in M$, every way of bracketing the product $a_1\dotsb a_n$ should be equivalent. What does this mean concretely? In the case $n=3$, another part of the data should be a homotopy $\eta_3\colon \mu(-,\mu(-,-))\Rightarrow \mu(\mu(-,-),-)$ in $\Hom_{\cat{An}}(M^3,M)$, witnessing $a\cdot(b\cdot c)\simeq (a\cdot b)\cdot c$ for all $a,b,c\in M$. If $M$ were a monoid in $\cat{Set}$ (or in any ordinary category), then the case $n=3$ would already guarantee associativity for arbitrary $n$. However, in an $\infty$-category, this no longer works. For example, in the case $n=4$, we need additional data---a homotopy $\eta_4$ in $\Hom_{\cat{An}}(M^4,M)$ that witnesses commutativity of the diagram
	\begin{equation*}
		\begin{tikzcd}
			&[-5em] &[-4.125em] a\cdot\left(b\cdot(c\cdot d)\right)\ar[dll,"a\cdot \eta_{b,c,d}"',"\simeq"]\ar[drr,"\eta_{a,b,(c\cdot d)}","\simeq"']\ar[drr,phantom,""{name=A}]\arrow[from=A,to=3-2,end anchor=center,draw=none,"\Longleftarrow"{sloped,marking,pos=0.5},"\eta_4"] &[-4.125em] &[-5em] \\[-0.5em]
			a\cdot \left((b\cdot c)\cdot d\right)\drar["\eta_{a,(b\cdot c),d}"',"\simeq"] & & & & (a\cdot b)\cdot (c\cdot d)\dlar["\eta_{(a\cdot b),c,d}","\simeq"'] \\[0.5em]
			& \left(a\cdot (b\cdot c)\right)\cdot d\ar[rr,"\eta_{a,b,c}\cdot d"',"\simeq"] & & \left((a\cdot b)\cdot c\right)\cdot d & 
		\end{tikzcd}
	\end{equation*}
	Then $\eta_4$ needs to satisfy another compatibility in $\Hom_{\cat{An}}(M^5,M)$ and so on. In general, Stasheff \cite{Stasheff} introduced $(d-2)$-dimensional polytopes $K_d$, called \emph{associahedra}, such that associativity up to $n=d-1$ induces a map $\partial K_d\rightarrow \Hom_{\cat{An}}(M^d,M)$ and associativity up to $n=d$ amounts to extending this to a map $K_d\rightarrow\Hom_{\cat{An}}(M^d,M)$.
	
	A similar story exists for unitality. This leads to a notion of \emph{$\IA_n$-monoids}, and in the limit case, \emph{$\IA_\infty$-monoids}. Fortunately, $\infty$-category theory provides a way to package all this unwieldy data into a much cleaner definition.
\end{numpar}
\begin{defi}\label{def:E1Monoids}
	Let $\Cc$ be an $\infty$-category with finite products (so in particular, the empty product exists, so $\Cc$ has a terminal object $*$).
	\begin{alphanumerate}
		\item An \emph{$\IA_\infty$-monoid} or \emph{$\IE_1$-monoid in $\Cc$} is a functor $M\colon \IDelta^\op\rightarrow \Cc$ satisfying $M_0\simeq *$ as well as the \emph{Segal condition}: The \emph{Segal maps} $e_i\colon [1]\rightarrow [n]$ that send $[1]$ bijectively to $\{i,i+1\}$ induce an equivalence\label{enum:E1Monoid}
		\begin{equation*}
			M_n\overset{\simeq}{\longrightarrow}M_1^n\,.
		\end{equation*}
		We call $M_1$ the \emph{underlying object of $M$}; we'll often don't distinguish between $M$ and $M_1$. Let $\cat{Mon}(\Cc)\subseteq\Fun(\IDelta^\op,\Cc)$ denote the full sub-$\infty$-category spanned by the $\IE_1$-monoids.
		\item For an $\IE_1$-monoid $M$ in $\Cc$, we get a multiplication map $\mu\colon M_1\times M_1\simeq M_2\overset{d_1^*}{\longrightarrow}M_1$ using the Segal condition. Then $M$ is called an \emph{$\IE_1$-group in $\Cc$} if the \emph{shearing map}\label{enum:E1Group}
		\begin{equation*}
			(\pr_1,\mu)\colon M_1\times M_1\overset{\simeq}{\longrightarrow}M_1\times M_1
		\end{equation*}
		is an equivalence. We let $\cat{Grp}(\Cc)\subseteq \cat{Mon}(\Cc)$ denote the full sub-$\infty$-category spanned by $\IE_1$-groups.
	\end{alphanumerate}
\end{defi}
\begin{numpar}[Associahedra revisited.]\label{par:AssociahedraII}
	Let's unravel what happens in \cref{def:E1Monoids}. Let $M\colon \IDelta^\op\rightarrow\Cc$ be an $\IE_1$-monoid in an $\infty$-category $\Cc$. We've already seen that $d_1^*\colon M_2\rightarrow M_1$ encodes the multiplication on $M$. In general, if we identify $M_n\simeq M_1^n$ and $M_{n-1}\simeq M_1^{n-1}$ via \cref{def:E1Monoids}\cref{enum:E1Monoid}, then the face map $d_i^*\colon M_n\rightarrow M_{n-1}$ for $0<i<n$ can be interpreted as the map that sends $(a_1,\dotsc,a_n)$ to $(a_1,\dotsc,a_{i-1},a_i\cdot a_{i+1}, a_{i+2},\dotsc,a_n)$. More precisely, if $e_{i,i+1}^{(2)}\colon [2]\rightarrow [n]$ is the map that sends $[2]$ bijectively to $\{i-1,i,i+1\}$, then the diagram
	\begin{equation*}
		\begin{tikzcd}
			M_n\rar["{(e_1,\dotsc,e_{i-1})\times e_{i,i+1}^{(2)}\times(e_{i+2},\dotsc,e_n)}","\simeq"']\dar["d_i^*"']\drar[commutes] &[8.5em] M_1^{i-1}\times M_2\times M_1^{n-i-1}\dar["\rlap{$\id \times d_1^*\times \id $}"]\rar["\id\times{(e_1,e_2)}\times\id","\simeq"'] &[2.5em] M_1^{i-1}\times M_1^2\times M_1^{n-i-1}\dlar[bend left=18.5,end anchor=0,"\id\times\mu\times\id"{name=A}]\arrow[phantom,from=1-3,to=2-2,commutes,xshift=-0.75em]\\
			M_{n-1}\rar["{(e_1,\dotsc,e_{i-1})\times e_{i}\times(e_{i+1},\dotsc,e_{n-1})}","\simeq"'] & M_1^{i-1}\times M_1\times M_1^{n-i-1}
		\end{tikzcd}
	\end{equation*}
	commutes. Indeed, the square on the left can be reduced to certain commutative squares in the ordinary category $\IDelta^\op$; we leave the details to you. The triangle on the right commutes by definition of $\mu$. In a similar way, one can show that the \enquote{outer} face maps $d_0^*$ and $d_n^*$ simply forget $a_1$ and $a_n$, respectively. 
	
	So the face maps in $\IDelta$ encode the multiplication, including its associativity, of the $\IE_1$-monoid $M\colon \IDelta^\op\rightarrow \Cc$. Likewise, the degeneracy maps encode unitality. The image of $*\simeq M_0$ under $s_0\colon M_0\rightarrow M_1$ is a point $1\in M_1$ which plays the role of the identity element of $M$ in the sense that the left and right multiplication maps
	\begin{equation*}
		M_1\simeq\{1\}\times M_1\overset{\mu }{\longrightarrow}M_1\quad\text{and}\quad M_1\simeq M_1\times \{1\}\overset{\mu }{\longrightarrow}M_1
	\end{equation*}
	are both homotopic to the identity $\id_{M_1}\colon M_1\rightarrow M_1$. Indeed, this follows from the identities $s_0\circ d_1=\id_{[1]}=s_1\circ d_1$ in $\IDelta$ via the commutative diagrams
	\begin{equation*}
		\begin{tikzcd}
			M_1\dar["\simeq"']\rar["s_0^*"]\drar[commutes] & M_2\rar["d_1^*"]\dar["{(e_1,e_2)}"'] & M_1\\
			\{1\}\times M_1\rar & M_1\times M_1\urar["\mu"',bend right=25]\urar[commutes,xshift=-0.375em] &
		\end{tikzcd}\quad\text{and}\quad
		\begin{tikzcd}
			M_1\dar["\simeq"']\rar["s_1^*"]\drar[commutes] & M_2\rar["d_1^*"]\dar["{(e_1,e_2)}"'] & M_1\\
			M_1\times \{1\}\rar & M_1\times M_1\urar["\mu"',bend right=25]\urar[commutes,xshift=-0.375em] &
		\end{tikzcd}
	\end{equation*}
	In general, $s_j^*\colon M_{n-1}\rightarrow M_n$ can be interpreted as the map that sends an $(n-1)$-tuple $(a_1,\dotsc,a_n)\in M_1^{n-1}\simeq M_{n-1}$ to the $n$-tuple $(a_1,\dotsc,a_{j-1},1,a_j,\dotsc,a_n)\in M_1^n$.
	
	These considerations lead to a nice conceptual description of Stasheff's associahedra $K_d$ from \cref{par:AssociahedraI}. We've seen that the \enquote{inner} face maps $d_i\colon [n]\rightarrow [n-1]$ for $0<i<n$ encode the multiplication on $M$. The (non-full) sub-category of $\IDelta$ spanned by $d_i\colon [n]\rightarrow [n-1]$ for $0<i<n$ and $1<n\leqslant d$ is equivalent to $\square^{d-1}\coloneqq (\Delta^1)^{d-1}$. Since $(\square^{d-1})^\op\simeq \square^{d-1}$, we get a (faithful but not fully faithful) functor $\square^{d-1}\rightarrow\IDelta^\op$. The restriction $M|_{\square^{d-1}}\colon \square^{d-1}\rightarrow \Cc$ of $M$ then encodes the multiplication $\mu$ on $M$ plus the fact that $\mu$ is associative for up to $d$ factors. But what does this have to do with Stasheff's associahedra? In the case $\Cc\simeq \cat{An}=\N_\Delta(\cat{Kan}_\Delta)$, a functor $\square^{d-1}\rightarrow\N^\Delta(\cat{Kan}^\Delta)$ is equivalently given by a simplicially enriched functor $\CC[\square^{d-1}]\rightarrow\cat{Kan}^\Delta$ by \cref{con:SimplicialNerve}. Thus, an anima $M_1$ together with a multiplication that's associative for up to $d$ factors is encoded by a simplicially enriched functor $M^\Delta\colon\CC[\square^{d-1}]\rightarrow\cat{Kan}^\Delta$ such that $M^\Delta$ sends $(0,\dotsc,0)$ to $M_1^d$ and $(1,\dotsc,1)$ to $M_1$. In particular, we get a morphism
	\begin{equation*}
		\F_{\CC[\square^{d-1}]}\left((0,\dotsc,0),(1,\dotsc,1)\right)\longrightarrow \Hom_{\cat{An}}\bigl(M_1^d,M_1\bigr)\,.
	\end{equation*}
	This is precisely the kind of structure we've seen in \cref{par:AssociahedraI}: a map from a polytope, modelled here as a simplicial set, into $\Hom_{\cat{An}}(M^d,M)$! And indeed, $\F_{\CC[\square^{d-1}]}((0,\dotsc,0),(1,\dotsc,1))$ turns out to be a model for Stasheff's associahedron $K_d$. In a similar way, $\partial K_d$ arises as a $\Hom$-simplicial set in $\CC[\partial\square^{d-1}]$. For a greatly expanded version of this explanation see \cite[\S\href{https://people.math.harvard.edu/~lurie/papers/HA.pdf\#subsection.4.1.6}{4.1.6}]{HA}.
\end{numpar}
\begin{lem}[\enquote{Equivalences of $\IE_1$-monoids can be checked on underlying objects}]\label{lem:E1MonoidsEquivalenceOnUnderlyingObjects}
	Let $\Cc$ be an $\infty$-category with finite products. Then a morphism $f\colon M\rightarrow N$ in $\cat{Mon}(\Cc)$ is an equivalence if and only if $f_1\colon M_1\rightarrow N_1$ is an equivalence. 
\end{lem}
\begin{proof}
	This follows immediately from \cref{thm:EquivalencePointwise} and the Segal condition.
\end{proof}
\begin{lem}\label{lem:E1Groups}
	For an $\IE_1$-monoid $M$ in animae, the following conditions are equivalent:
	\begin{alphanumerate}
		\item $M$ is an $\IE_1$-group.\label{enum:MIsE1Group}
		\item For every $a\in M_1$, the \enquote{left multiplication map} $a\cdot (-)\colon M_1\simeq \{a\}\times M_1\overset{\mu}{\longrightarrow}M_1$ is an equivalence.\label{enum:MLeftMultiplication}
		\item For every $a\in M_1$, the \enquote{right multiplication map} $(-)\cdot a\colon M_1\simeq M_1\times\{a\}\overset{\mu}{\longrightarrow}M_1$ is an equivalence.\label{enum:MRightMultiplication}
		\item The ordinary monoid $\pi_0(M)\in\cat{Mon}(\cat{Set})$ is a group.\label{enum:MGroupOnPi0}
	\end{alphanumerate}
\end{lem}
\begin{proof}
	We prove \cref{enum:MIsE1Group} $\Leftrightarrow$ \cref{enum:MLeftMultiplication} first. Using \cref{thm:Whitehead}, \cref{lem:LongExactFibrationSequence}, and the five lemma (plus \cref{rem:ExactnessInLowDegrees}), we see that the shearing map $(\pr_1,\mu)\colon M_1\times M_1\rightarrow M_1\times M_1$ is an equivalence if and only if it induces equivalences on all fibres of $\pr_1\colon M_1\times M_1\rightarrow M_1$. The induced map on fibres over $a\in M_1$ is precisely $a\cdot (-)$. This already proves \cref{enum:MIsE1Group} $\Leftrightarrow$ \cref{enum:MLeftMultiplication}.
	
	The implication \cref{enum:MLeftMultiplication} $\Rightarrow$ \cref{enum:MGroupOnPi0} is clear, since the condition from \cref{enum:MLeftMultiplication} implies that for every equivalence class $[a]\in\pi_0(M)$, left multiplication with $[a]$ is a bijection. The same argument shows \cref{enum:MRightMultiplication} $\Rightarrow$ \cref{enum:MGroupOnPi0}. For \cref{enum:MGroupOnPi0} $\Rightarrow$ \cref{enum:MLeftMultiplication}, note that associativity of the multiplication of $M$ implies
	\begin{equation*}
		\bigl(b\cdot(-)\bigr)\circ \bigl(c\cdot (-)\bigr)\simeq \bigl((b\cdot c)\cdot (-)\bigr) 
	\end{equation*}
	for all $b,c\in M_1$. Since $\pi_0(M)$ is assumed to be a group, there exists an element $b\in M_1$ such that $a\cdot b\simeq 1\simeq b\cdot a$, where $1\in M_1$ is the identity element, that is, the image of $*\simeq M_0$ under $s_0\colon M_0\rightarrow M_1$. Since $1\cdot (-)\colon M_1\rightarrow M_1$ is homotopic to $\id_{M_1}$, as we've seen in \cref{par:AssociahedraII}, the equivalence above shows that $b\cdot (-)$ is both a left inverse and a right inverse to $a\cdot (-)$. So $a\cdot (-)\colon M_1\rightarrow M_1$ is an equivalence. This finishes the proof of \cref{enum:MGroupOnPi0} $\Rightarrow$ \cref{enum:MLeftMultiplication}. An analogous argument shows \cref{enum:MGroupOnPi0} $\Rightarrow$ \cref{enum:MRightMultiplication}.
\end{proof}
The main theorem of this subsection is Stasheff's \emph{recognition principle} for loop spaces:
\begin{thm}[\enquote{$\IE_1$-groups are the same as loop animae}]\label{thm:E1Loop}
	Let $((\cat{Cat}_\infty)_{*/})_{\geqslant 1}\subseteq (\cat{Cat}_\infty)_{*/}$ be the full sub-$\infty$-category of all \embrace{small} pointed $\infty$-categories $(\Cc,x)$ for which $\pi_0\core(\Cc)\cong *$ and let $(\cat{An}_{*/})_{\geqslant 1}\subseteq ((\cat{Cat}_\infty)_{*/})_{\geqslant 1}$ be the full sub-$\infty$-category spanned those pointed animae $(X,x)$ where $\pi_0(X)\cong *$.
	\begin{alphanumerate}
		\item There is an equivalence of $\infty$-categories\label{enum:E1LoopMon}
		\begin{equation*}
			\cat{Mon}(\cat{An})\overset{\simeq }{\longrightarrow}\bigl((\cat{Cat}_\infty)_{*/}\bigr)_{\geqslant 1}\,.%\equationblackbox
		\end{equation*}
		\item There is an adjunction $\B\colon \cat{Mon}(\cat{An})\shortdoublelrmorphism \cat{An}_{*/}\noloc \Omega$ which induces a pair of inverse equivalences\label{enum:E1LoopGrp}
		\begin{equation*}
			\B\colon \cat{Grp}(\cat{An})\underset{\simeq}{\mathrel{\smash{\underset{\smash{\raisebox{0.35em}{$\longleftarrow$}}}{\overset{\smash{\raisebox{-0.35em}{$\overset{\simeq}{\longrightarrow}$}}}{\phantom{\longrightarrow}}}}}} \bigl(\cat{An}_{*/}\bigr)_{\geqslant 1}\noloc \Omega\,.
		\end{equation*}
	\end{alphanumerate}
\end{thm}
\begin{rem}\label{rem:E1Loop}
	The intuition behind \cref{thm:E1Loop} is easy to explain: If $(\Cc,x)$ is a pointed $\infty$-category, such that $\pi_0\core(\Cc)\cong *$, then $\Hom_\Cc(x,x)$ is an $\IE_1$-monoid via composition. Coversely, if $M$ is an $\IE_1$-monoid, then we can build an $\infty$-category $\B^+ M$ with only one object $*$ and $\Hom_{\B^+M}(*,*)\simeq M$; the composition is dictated by the multiplication on $M$. Hence \cref{thm:E1Loop}\cref{enum:E1LoopMon}. Furthermore, $\Cc$ is an anima if and only if every morphism in $\Hom_\Cc(x,x)$ is invertible, which is equivalent to $\Hom_\Cc(x,x)$ being an $\IE_1$-group by \cref{lem:E1Groups}. Hence \cref{thm:E1Loop}\cref{enum:E1LoopGrp}. Unfortunately, making this intuition formal requires a lot more work.
\end{rem}
The proof of \cref{thm:E1Loop} will be rather lengthy. We'll first show \cref{thm:E1Loop}\cref{enum:E1LoopMon}, up to a pretty serious black box (\cref{thm:RezkNerve}). \cref{thm:E1Loop}\cref{enum:E1LoopGrp} could then be obtained as a simple consequence, but instead, we'll give a proof that avoids the aforementioned black box.

Our first goal on our way towards \cref{thm:E1Loop}\cref{enum:E1LoopMon} is to construct an $\IE_1$-monoid structure on the anima $\operatorname{End}_\Cc(x)\coloneqq\Hom_\Cc(x,x)$ of endomorphisms of $x$. This requires a construction which is quite interesting in its own right.
%This effort will not be wasted since the adjunction from \cref{thm:E1Loop}\cref{enum:E1LoopGrp} will be an immediate consequence. In total, the proof will consist of two parts: a formal nonsense part, in which we construct the adjoints from general principles, and a concrete part, in which we compute $\Omega\B G$ for every $\IE_1$-group $G$ to establish the equivalence from \cref{thm:E1Loop}\cref{enum:E1LoopGrp}.	
\begin{con}\label{con:RezkNerve}
	Consider the functor $U\colon \IDelta\rightarrow \cat{Cat}_\infty$ that sends $[n]\mapsto\Delta^n$ (or, if you want, $[n]\mapsto [n]$, since we suppress writing nerves). To construct $U$ formally, observe that it already exists as a functor $\IDelta\rightarrow \cat{QCat}$ of ordinary categories and use \cref{thm:AnAsALocalisation}. Alternatively, one can write down the unstraightening explicitly; it will be an ordinary category over $\IDelta$. Using \cref{thm:PShFreeCocompletion}, $U$ induces an adjunction
	\begin{equation*}
		\operatorname{asscat}\colon \Fun\left(\IDelta^\op,\cat{An}\right)\doublelrmorphism \cat{Cat}_\infty\noloc \N^\mathrm{Rezk}\,.
	\end{equation*}
	Here $\operatorname{asscat}$ stands for \emph{associated category}\footnote{\ldots and it has nothing to do with asinine felines (or worse). Why would you think that?!}, $\N^{\mathrm{Rezk}}$ is the \emph{Rezk nerve}. According to \cref{lem:LanAlongYonedaHasRightAdjoint}, the Rezk nerve is given by $\N^{\mathrm{Rezk}}(\Cc)_n\simeq \Hom_{\cat{Cat}_\infty}(\Delta^n,\Cc)$ for every $\infty$-category $\Cc$ and all $n\geqslant 0$.
\end{con}
To prove \cref{thm:E1Loop}\cref{enum:E1LoopMon}, we'll need the following black box. Fortunately, a relatively short proof in model-independent language has recently been found by Fabian and Jan Steinebrunner \cite{FabianRezkNerve}. The original proof due to Joyal and Tierney is in \cite{JoyalTierney}; Lurie has given another proof in \cite{LurieGoodwillieCalculus}.
\begin{thm}\label{thm:RezkNerve}
	The Rezk nerve $\N^\mathrm{Rezk}\colon \cat{Cat}_\infty\rightarrow \Fun(\IDelta^\op,\cat{An})$ is fully faithful and its image is given by the complete Segal animae. Here, a simplicial anima $X\colon \IDelta^\op\rightarrow \cat{An}$ is called Segal if the Segal maps $e_i\colon [1]\rightarrow [n]$ induce equivalences
	\begin{equation*}
		X_n\overset{\simeq}{\longrightarrow} \underbrace{X_1\times_{X_0}\dotsb\times_{X_0}X_1}_{n\text{ factors}}\,.
	\end{equation*}
	Furthermore, $X$ is called complete, if $s_0^*\colon X_0\rightarrow X_1$ is an equivalence onto the collection of path components $X_1^\sim\subseteq X_1$ given by those $\alpha\in X_1$ for which there exist $\sigma,\tau\in X_2$ such that $d_0^*(\sigma)\simeq \alpha\simeq d_2^*(\tau)$ and both $d_1^*(\sigma)$, $d_1^*(\tau)$ lie in the image of $s_0^*\colon X_0\rightarrow X_1$.\hfill$\blacksquare$
\end{thm}
Let us now construct the desired $\IE_1$-monoid structure on $\End_\Cc(x)$.
\begin{con}\label{con:EndomorphismE1Structure}
	If $M\in\Fun(\IDelta^\op,\cat{An})$ is an $\IE_1$-monoid, then $M_0\simeq *$. Via Yoneda's lemma, this induces a canonical morphism $\Yo_{\IDelta}([0])\simeq \const *\rightarrow M$ of $\IE_1$-monoids. Accordingly, we get a canonical morphism $\operatorname{asscat}(\const *)\rightarrow M$. Since $\operatorname{asscat}(\const *)\simeq \operatorname{asscat}(\Yo_{\IDelta}([0]))\simeq U([0])\simeq *$, the morphism above canonically turns $\operatorname{asscat}(M)$ into a pointed $\infty$-category and so $\operatorname{asscat}$ upgrades to a functor $\B^+\colon \cat{Mon}(\cat{An})\rightarrow (\cat{Cat}_\infty)_{*/}$.%
	%
	\footnote{Note that $\B^+$ is non-standard notation; there doesn't seem to be any standard notation.}
	%
	For a pointed $\infty$-category $(\Cc,x)$, let, temporarily, $\operatorname{End}_\Cc(x)\in\Fun(\IDelta^\op,\cat{An})$ be redefined as the pullback
	\begin{equation*}
		\begin{tikzcd}
			\operatorname{End}_\Cc(x)\doublear{d}\doublear{r}\drar[pullback] & \N^\mathrm{Rezk}(\Cc)\doublear["u"{black,right=0.1em}]{d}\\
			\const \{x\}\doublear{r} & \Ran_{\{[0]\}\rightarrow \IDelta^\op}\N^\mathrm{Rezk}(\Cc)\big|_{\{[0]\}}
		\end{tikzcd}
	\end{equation*}
	in $\Fun(\IDelta^\op,\cat{An})$. The right vertical arrow $u$ is the unit transformation from a functor to the right Kan extension of its restriction. For the bottom horizontal arrow, note that since $(\Cc,x)$ is a pointed $\infty$-category, there is a canonical morphism $\{x\}\rightarrow \core (\Cc)\simeq \N^\mathrm{Rezk}(\Cc)_0$; then the desired natural transformation $\const \{x\}\Rightarrow \Ran_{\{[0]\}\rightarrow \IDelta^\op}\N^\mathrm{Rezk}(\Cc)|_{\{[0]\}}$ is induced by the universal property of right Kan extension. It's straightforward to check that the right vertical and bottom horizontal arrows are functorial. Since taking pullbacks is functorial too, we get a functor $\operatorname{End}\colon (\cat{Cat}_\infty)_{*/}\rightarrow \Fun(\IDelta^\op,\cat{An})$, as desired.
\end{con}
\begin{lem}\label{lem:B+EndAdjunction}
	The simplicial anima $\operatorname{End}_\Cc(x)$ from \cref{con:EndomorphismE1Structure} is an $\IE_1$-monoid and its underlying anima $\operatorname{End}_\Cc(x)_1$ is the anima $\Hom_\Cc(x,x)$ of endomorphisms of $x$. Furthermore, the functors from \cref{con:EndomorphismE1Structure} fit into an adjunction
	\begin{equation*}
		\B^+\colon \cat{Mon}(\cat{An})\doublelrmorphism (\cat{Cat}_\infty)_{*/}\noloc \operatorname{End}\,.
	\end{equation*}
\end{lem}
\begin{proof}
	Let's check first that $\operatorname{End}$ takes values in $\cat{Mon}(\cat{An})\subseteq\Fun(\IDelta^\op,\cat{An})$ and that the underlying anima of $\operatorname{End}_\Cc(x)$ is indeed $\Hom_\Cc(x,x)$. To this end, fix $n\geqslant 0$; we'll compute $\operatorname{End}_\Cc(x)_n$. Recall that $\N^\mathrm{Rezk}(\Cc)_n\simeq \Hom_{\cat{Cat}_\infty}(\Delta^n,\Cc)$. To compute the right-Kan extension, we use the formula from \cref{lem:KanExtensionFormula}: The slice $\infty$-category $\{[0]\}_{/[n]}$ has $n+1$ objects, namely the morphisms $[0]\rightarrow\{j\}\rightarrow [n]$ for $0\leqslant j\leqslant n$, and there are no non-identity morphisms in $\{[0]\}_{/[n]}$. So the Kan extension formula is just a limit over a discrete diagram with $n+1$ objects, which leads to $(\Ran_{\{[0]\}\rightarrow \IDelta^\op}\N^\mathrm{Rezk}(\Cc)|_{\{[0]\}})_n\simeq \core (\Cc)^{n+1}$. Furthermore, a quick unravelling shows that the morphism $u$ from \cref{con:EndomorphismE1Structure} can be identified with 
	\begin{equation*}
		\Hom_{\cat{Cat}_\infty}\bigl(\Delta^n,\Cc\bigr)\rightarrow\Hom_{\cat{Cat}_\infty}\bigl(\{0\}\sqcup\dotsb\sqcup\{n\},\Cc\bigr)\simeq \core(\Cc)^{n+1}\,.
	\end{equation*}
	Now observe that $\Delta^n$ can be written as $\Delta^n\simeq \Delta^{\{0,1\}}\sqcup_{\{1\}}\Delta^{\{1,2\}}\sqcup_{\{2\}}\dotsb\sqcup_{\{n-1\}}\Delta^{\{n-1,n\}}$ in $\cat{Cat}_\infty$.\footnote{One way would to see this is to observe that the pushout in $\cat{sSet}$ would just be $I^n$ from the proof of \cref{thm:EquivalenceFullyFaithfulEssentiallySurjective} and that $I^n\subseteq \Delta^n$ is inner anodyne, so that $\Delta^n$ is the pushout in $\cat{Cat}_\infty$ by model category fact~\cref{par:HomotopyPushout}. Another way would be to use \cref{lem:ColimitsInAnima} and think hard about the localisation.} Identifying $\Hom_{\cat{Cat}_\infty}(\Delta^{\{i-1,i\}},\Cc)\simeq \core \Ar(\Cc)$ via \cref{thm:CordierPorter}, we obtain
	\begin{equation*}
		\Hom_{\cat{Cat}_\infty}(\Delta^n,\Cc)\simeq \underbrace{\core \Ar(\Cc)\times_{t,\core(\Cc),s}\dotsb\times_{t,\core(\Cc),s}\core \Ar(\Cc)}_{n\text{ factors}}
	\end{equation*}
	from \cref{cor:HomPreservesColimits}. Recall from \cref{lem:ColimitsInFunctorCategories} that pullbacks in $\Fun(\IDelta^\op,\cat{An})$ are computed degree-wise. So $\operatorname{End}_\Cc(x)_n$ is the pullback $\{x\}\times_{\core(\Cc)^{n+1}}\Hom_{\cat{Cat}_\infty}(\Delta^n,\Cc)$. Plugging in the formula above, we see
	\begin{equation*}
		\operatorname{End}_\Cc(x)_n\simeq \Hom_\Cc(x,x)^n
	\end{equation*}
	by a simple manipulation of pullbacks. So we've achieved two things at once: We've shown that $\operatorname{End}_\Cc(x)$ satisfies the conditions from \cref{def:E1Monoids}, so that it is an $\IE_1$-monoid, and that the underlying anima of that $\IE_1$-monoid is indeed $\Hom_\Cc(x,x)$.
	
	It remains to show that $\operatorname{End}$ is right adjoint to $\B^+$. So let $M\in\cat{Mon}(\cat{An})$. The universal property of right Kan extensions combined with $M_0\simeq *$ shows 
	\begin{equation*}
		\Hom_{\Fun(\IDelta^\op,\cat{An})}\left(M,\Ran_{\{[0]\}\rightarrow \IDelta^\op}\N^\mathrm{Rezk}(\Cc)\big|_{\{[0]\}}\right)\simeq \Hom_{\cat{An}}(M_0,\core (\Cc))\simeq \core (\Cc)
	\end{equation*}
	This allows us to compute
	\begin{align*}
		\Hom_{\Fun(\IDelta^\op,\cat{An})}\bigl(M,\operatorname{End}_\Cc(x)\bigr)&\simeq \Hom_{\Fun(\IDelta^\op,\cat{An})}\bigl(M,\N^\mathrm{Rezk}(\Cc)\bigr)\times_{\core(\Cc)}\{x\}\\
		&\simeq \Hom_{\cat{Cat}_\infty}\bigl(\operatorname{asscat}(M),\Cc\bigr)\times_{\Hom_{\cat{Cat}_\infty}(*,\Cc)}\{x\}\\
		&\simeq \Hom_{(\cat{Cat}_\infty)_{*/}}\bigl(\B^+M,(\Cc,x)\bigr)\,.
	\end{align*}
	In the first step we use that $\Hom_{\Fun(\IDelta^\op,\cat{An})}(M,-)$ commutes with pullbacks by \cref{cor:HomPreservesLimits} together with the above simplification. In the second step, we use the adjunction $\operatorname{asscat}\dashv \N^\mathrm{Rezk}$ as well as $\core(\Cc)\simeq \Hom_{\cat{Cat}_\infty}(*,\Cc)$. In the third step we use \cref{cor:HomInSliceCategories}. It's easy to make all steps functorial in $M$ and $(\Cc,x)$ and so the proof is finished.
\end{proof}
\begin{proof}[Proof sketch of \cref{thm:E1Loop}\cref{enum:E1LoopMon}]
	Observe that a morphism $(\Cc,x)\rightarrow (\Dd,y)$ in $((\cat{Cat}_\infty)_{*/})_{\geqslant 1}$ is automatically essentially surjective. Hence any such morphism is an equivalence if and only if $\Hom_\Cc(x,x)\rightarrow \Hom_\Dd(y,y)$ is an equivalence. This immediately shows that the right adjoint $\End\colon ((\cat{Cat}_\infty)_{*/})_{\geqslant 1}\rightarrow \cat{Mon}(\cat{An})$ is conservative. It follows from \cref{thm:RezkNerve}, or more precisely, from \cite[Corollary~\href{https://arxiv.org/pdf/2312.09889\#equation.3.15}{3.15}]{FabianRezkNerve}, that $\Hom_{\B^+M}(*,*)\simeq M$ holds for all $M\in \cat{Mon}(\cat{An})$. Using \cref{lem:E1MonoidsEquivalenceOnUnderlyingObjects}, it follows that the unit $u_M\colon M\rightarrow \End_{\B^+M}(*)$ is an equivalence. Hence $\B^+\colon \cat{Mon}(\cat{An})\rightarrow ((\cat{Cat}_\infty)_{*/})_{\geqslant 1}$ is fully faithful by \cref{lem:FullyFaithfulConservativeAdjunction}\cref{enum:FullyFaithfulIffUnitEquivalence}. Then \cref{lem:FullyFaithfulConservativeAdjunction}\cref{enum:Conservative} shows that $\B^+$ and $\End$ are inverse equivalences.
\end{proof}



Let us now turn to \cref{thm:E1Loop}\cref{enum:E1LoopGrp}. The proof will consist of two parts: a formal part, in which we effortlessly deduce the adjunction $\B\colon \cat{Mon}(\cat{An})\shortdoublelrmorphism \cat{An}_{*/}\noloc \Omega$, and a hard part, in which we compute $\Omega\B G$ for every $\IE_1$-group $G$ to establish the equivalence $\cat{Grp}(\cat{An})\simeq (\cat{An}_{*/})_{\geqslant 1}$.

\begin{proof}[Proof sketch of \cref{thm:E1Loop}\cref{enum:E1LoopGrp}, formal part]
	Let $\Omega\coloneqq \operatorname{End}|_{\cat{An}_{*/}}\colon \cat{An}_{*/}\rightarrow \cat{Mon}(\cat{An})$ denote the restriction of $\operatorname{End}$ from \cref{con:EndomorphismE1Structure} to $\cat{An}_{*/}\subseteq(\cat{Cat}_\infty)_{*/}$. It follows from \cref{lem:SuspensionLoopAdjunction}\cref{enum:LoopIsHom} that the underlying anima of $\Omega X\in \cat{Mon}(\cat{An})$ is indeed the eponymous $\Omega X$ from \cref{def:Loop}. Now we claim:
	\begin{alphanumerate}\itshape
		\item[\boxtimes_1] The functor $\Omega\colon \cat{An}_{*/}\rightarrow \cat{Mon}(\cat{An})$ factors through $\cat{Grp}(\cat{An})\subseteq \cat{Mon}(\cat{An})$. Furthermore, $\Omega$ admits a left adjoint $\B\colon \cat{Mon}(\cat{An})\rightarrow \cat{An}_{*/}$, which factors through $(\cat{An}_{*/})_{\geqslant 1}\subseteq \cat{An}_{*/}$.\label{claim:BOmegaAdjunction}
	\end{alphanumerate}
	To see that $\Omega X$ is an $\IE_\infty$-group, one can use \cref{lem:E1Groups}\cref{enum:MGroupOnPi0} for example: $\pi_0(\Omega X)\cong \pi_1(X,x)$ is a group by \cref{lem:SuspensionLoopAdjunction}\cref{enum:LoopShiftsHomotopyGroups}.
	
	Next, let's construct $\B$. Using \cref{cor:HomInSliceCategories} and $\abs{*}\simeq *$, it's straightforward to check that $\abs{\,\cdot\,}\colon \cat{Cat}_\infty\rightarrow \cat{An}$ induces a functor $\abs{\,\cdot\,}\colon(\cat{Cat}_\infty)_{*/}\rightarrow\cat{An}_{*/}$ which is left adjoint to the inclusion $\cat{An}_{*/}\subseteq(\cat{Cat}_\infty)_{*/}$. We then let $\B\coloneqq \abs{\B^+(-)}\colon \cat{Mon}(\cat{An})\rightarrow\cat{An}_{*/}$ denote the \emph{delooping} functor. From the diagram
	\begin{equation*}
		\begin{tikzcd}[column sep=large]
			\cat{Mon}(\cat{An})\rar[shift left=0.2em,"B^+"]\drar[end anchor=175,shorten <=0.4ex,shorten >=0.1ex,bend right=15.5,shift left=0.2em,"\B"] & (\cat{Cat}_\infty)_{*/}\lar[shift left=0.2em,"\operatorname{End}"]\dar[shift left=0.2em,"\abs{\,\cdot\,}"]\dar[phantom,""{name=A}]\arrow[from=1-1,to=A,commutes,pos=0.7]\\%start anchor=325,end anchor=175,
			{ } & \cat{An}_{*/}\ular[start anchor=175,bend left=15,shift left=0.2em,"\Omega"]\uar[shift left=0.2em] %end anchor=325,start anchor=175,
		\end{tikzcd}
	\end{equation*}
	it's immediate that $\B$ and $\Omega$ are adjoints. To show that $\B$ lands in $(\cat{An}_{*/})_{\geqslant 1}$, we need an alternative description of $\B$. By construction, the composition of $\B\colon \cat{Mon}(\cat{An})\rightarrow\cat{An}_{*/}$ with $\cat{An}_{*/}\rightarrow\cat{An}$ agrees with $\abs{\operatorname{asscat}(-)}\colon \Fun(\IDelta^\op,\cat{An})\rightarrow\cat{An}$. Note that this functor preserves all colimits, because so do $\operatorname{asscat}\colon \Fun(\IDelta^\op,\cat{An})\rightarrow \cat{Cat}_\infty$ and $\abs{\,\cdot\,}\colon \cat{Cat}_\infty\rightarrow\cat{An}$. By \cref{thm:PShFreeCocompletion}, $\abs{\operatorname{asscat}(-)}$ must be the unique colimit-preserving extension of the functor $\IDelta\rightarrow \cat{An}$ sending $[n]\mapsto \left|\Delta^n\right|\simeq *$; that is, $\left|\operatorname{asscat}(-)\right|$ is the unique colimit-preserving extension of the constant functor $\const *\colon \IDelta\rightarrow \cat{An}$. On the other hand, $\colimit_{\IDelta^\op}\colon \Fun(\IDelta^\op,\cat{An})\rightarrow \cat{An}$ also preserves colimits, since it is a left adjoint by definition. Moreover, $\colimit_{\IDelta^\op}\Yo_{\IDelta}([n])\simeq \mathopen|\IDelta_{/[n]}\mathclose|\simeq *$ by \cref{lem:ColimitsInAnima} and the fact that $\IDelta_{/[n]}$ has a final object. So $\colimit_{\IDelta^\op}\colon \Fun(\IDelta^\op,\cat{An})\rightarrow\cat{An}$ is also the unique colimit-preserving extension of $\const *$. It follows that if $M\in\cat{Mon}(\cat{An})$, then the underlying unpointed anima of $\B M$ is $\colimit_{[n]\in\IDelta^\op}M_n$, and the point $*\rightarrow \B M$ comes via $*\simeq M_0\rightarrow \colimit_{[n]\in\IDelta^\op}M_n$. We've seen in \cref{lem:HomotopyGroupsFilteredColimits} that $\pi_0\colon \cat{An}\rightarrow \cat{Set}$ commutes with arbitrary colimits. So we get a bijection of sets $\pi_0(\B M)\cong \colimit_{[n]\in \IDelta^\op}\pi_0(M_n)$. Using $\pi_0(M_0)\cong *$, it's straightforward to check that the colimit must be $*$ as well.%
	%
	\footnote{In fact, if $S\colon\IDelta^\op\rightarrow\Cc$ is any functor into an ordinary category, then the colimit of $S$ is given by the coequaliser
	\begin{equation*}
		\colimit_{[n]\in \IDelta^\op}S_n\cong \coeq\biggl(S_1\overset{\smash{d_0^*}}{\underset{\smash{d_1^*}}{\doublemorphism}}S_0\biggr)\,.
	\end{equation*}
	(assuming either colimit exists). This formula is wildly false in general $\infty$-categories, as already evidenced by $\B M\simeq \colimit_{[n]\in\IDelta^\op}M_n$ in $\cat{An}$.}
	%
	This finishes the proof of \cref{claim:BOmegaAdjunction}.
	
	In particular, we obtain a restricted adjunction $\B\colon \cat{Grp}(\cat{An})\shortdoublelrmorphism (\cat{An}_{*/})_{\geqslant 1}\noloc \Omega$. To show that this is a pair of inverse equivalences, it's enough to show that $\B$ is fully faithful and that $\Omega$ is conservative; see \cref{lem:FullyFaithfulConservativeAdjunction}\cref{enum:Conservative}. The latter is easy. If $\ev_{[1]}\colon \cat{Mon}(\cat{An})\rightarrow \cat{An}$ is the functor  that sends an $\IE_1$-monoid to its underlying anima is conservative, then already
	\begin{equation*}
		\cat{An}_{*/}\overset{\Omega}{\longrightarrow} \cat{Mon}(\cat{An})\xrightarrow{\ev_{[1]}}\cat{An}
	\end{equation*}
	is conservative. Indeed, this composition is the loop functor $\Omega\colon \cat{An}_{*/}\rightarrow \cat{An}$. Since any morphism $(X,x)\rightarrow (Y,y)$ in $(\cat{An}_{*/})_{\geqslant 1}$ is automatically a bijection on $\pi_0$, \cref{thm:Whitehead} and \cref{lem:SuspensionLoopAdjunction}\cref{enum:LoopShiftsHomotopyGroups} show that such a morphism is an equivalence if and only if $\Omega_xX\rightarrow\Omega_yY$ is an equivalence. This proves that $\Omega$ is indeed conservative. To prove that $\B$ is fully faithful, we will need another claim:
	\begin{alphanumerate}\itshape
		\item[\boxtimes_2] For every $\IE_1$-group $G\in\cat{Grp}(\cat{An})$, the unit transformation $u_G\colon G\rightarrow \Omega\B G$ is an equivalence on underlying animae.\label{claim:HomBG}
	\end{alphanumerate}
	If we can show \cref{claim:HomBG}, then $u_G$ will also be an equivalence of $\IE_1$-groups by \cref{lem:E1MonoidsEquivalenceOnUnderlyingObjects}. So $\B$ is fully faithful by \cref{lem:FullyFaithfulConservativeAdjunction}\cref{enum:FullyFaithfulIffUnitEquivalence} and we would be done. The proof of \cref{claim:HomBG} requires some further tools, and we postpone it for now.
\end{proof}
The main difficulty in the proof of \cref{claim:HomBG} is the fact that $\B G$ is defined as a colimit, whereas $\Omega$ is a pullback. So we need to commute pullbacks and (non-filtered) colimits. Fortunately, there's a relatively simple criterion due to Charles Rezk \cite[Proposition~\href{https://rezk.web.illinois.edu/i-hate-the-pi-star-kan-condition.pdf\#page=3}{2.4}]{RezkEquifibrancy} that allows us to do this in certain situations.
\begin{lem}\label{lem:RezkEquifibrancy}
	Let $\Jj$ be an $\infty$-category. A natural transformation $q\colon B\Rightarrow D$ in $\Fun(\Jj,\cat{An})$ is called equifibred if for every morphism $\alpha\colon i\rightarrow j$ in $\Jj$, the induced diagram
	\begin{equation*}
		\begin{tikzcd}
			B_i\rar["{B(\alpha)}"]\dar["q_i"']\drar[pullback] & B_j\dar["q_j"]\\
			D_i\rar["{D(\alpha)}"] & D_j
		\end{tikzcd}
	\end{equation*}
	is a pullback square in $\cat{An}$. Then the colimit functor $\colimit_\Jj\colon \Fun(\Jj,\cat{An})\rightarrow\cat{An}$ preserves pullback squares in which one leg is equifibred. That is, if we're given a pullback square
	\begin{equation*}
		\begin{tikzcd}
			A\doublear{r}\doublear["p"{black,left=0.1em}]{d}\drar[pullback] & B\doublear["q"{black,right=0.1em}]{d}\\
			C\doublear{r} & D
		\end{tikzcd}
	\end{equation*}
	in $\Fun(\Jj,\cat{An})$ such that $q\colon B\Rightarrow D$ is equifibred, then $\colimit_\Jj\colon \Fun(\Jj,\cat{An})\rightarrow\cat{An}$ sends this diagram to a pullback square in $\cat{An}$.
\end{lem}
The idea to prove \cref{lem:RezkEquifibrancy} is to interpret $q_i\colon B_i\rightarrow D_i$ as the unstraightenings of certain functors $G_i\colon D_i\rightarrow \cat{An}$ and then to use \cref{lem:ColimitsInAnima} backwards. To this end, we need to study the straightening equivalence from \cref{thm:Straightening} a little more.
\begin{numpar}[The universal unstraightening.]\label{par:UniversalUnstraightening}
	Let $\kappa$ be a regular cardinal and let $\cat{An}^{<\kappa}\subseteq \cat{An}$ be the full sub-$\infty$-category of essentially $\kappa$-small animae as in \cref{def:KappaSmall}. Then $\cat{An}^{<\kappa}$ is essentially small itself (albeit not necessarily essentially $\kappa$-small) and so we can consider the unstraightening $p_\mathrm{univ}^{<\kappa}\colon \Uu_\mathrm{univ}^{<\kappa}\rightarrow \cat{An}^{<\kappa}$ of $\cat{An}^{<k}\rightarrow \cat{An}$ and we can regard $\cat{An}^{<\kappa}$ as an object in $\cat{Cat}_\infty$. If $p\colon \Uu\rightarrow \Cc$ is any left fibration with essentially $\kappa$-small fibres over an $\infty$-category $\Cc$ and $F\simeq \operatorname{St}^{(\mathrm{left})}(p)\colon \Cc\rightarrow \cat{An}$ is the associated functor, then $F$ factors through $\cat{An}^{<\kappa}$. Since precompositions are sent to pullbacks by \cref{thm:Straightening}\cref{enum:CocartesianStraightening}, it follows that there must be a pullback diagram
	\begin{equation*}
		\begin{tikzcd}
			\Uu\rar\dar["p"']\drar[pullback] & \Uu_\mathrm{univ}^{<\kappa}\dar["p_\mathrm{univ}^{<\kappa}"]\\
			X\rar["F"] & \cat{An}^{<\kappa}
		\end{tikzcd}
	\end{equation*}
	in animae. So $p_\mathrm{univ}^{<\kappa}$ acts as a \emph{universal unstraightening}, whence the notation. Of course, what we would really like to do here is to consider the unstraightening $p_\mathrm{univ}\colon \Uu_\mathrm{univ}\rightarrow\cat{An}$ of $\id_{\cat{An}}\colon \cat{An}\rightarrow \cat{An}$ and regard $\cat{An}$ as an object in $\cat{Cat}_\infty$. The only way to do this without any set theorist suffering a stroke would be to consider universes, which amounts to choosing a strongly inaccessible cardinal bound. It turns out that any cardinal bound $\kappa$ does it, so we can get away without using universes.
\end{numpar}
\begin{lem}\label{lem:StraighteningFunctorial}
	Consider the slice $\infty$-category $\cat{An}_{/\cat{An}^{<\kappa}}\simeq \cat{An}\times_{\cat{Cat}_\infty}(\cat{Cat}_\infty)_{/\cat{An}^{<\kappa}}$ and let $\Ar_{\pullbacksign \vphantom{^t} }^{<\kappa}(\cat{An})\subseteq \Ar(\cat{An})$ be the \embrace{non-full} sub-$\infty$-category, in the sense of \cref{par:SubQuasiCategories}, spanned by those objects $(\alpha\colon X\rightarrow Y)\in \Ar(\cat{An})$ for which the fibres of $\alpha$ are essentially $\kappa$-small and those morphisms $(\alpha\colon X\rightarrow Y)\rightarrow (\alpha'\colon X'\rightarrow Y')$ that represent pullback squares in $\cat{An}$. Then there is an equivalence of $\infty$-categories
	\begin{equation*}
		(p_\mathrm{univ}^{<\kappa})^*\colon \cat{An}_{/\cat{An}^{<\kappa}}\overset{\simeq}{\longrightarrow} \Ar_{\pullbacksign\vphantom{^t}}^{<\kappa}(\cat{An})
	\end{equation*}
	that sends an object $(F\colon X\rightarrow \cat{An}^{<\kappa})\in\cat{An}_{/\cat{An}^{<\kappa}}$ to the pullback $(X\times_{\cat{An}^{<\kappa}}\Uu_\mathrm{univ}^{<\kappa}\rightarrow X)$, or equivalently \embrace{by \cref{par:UniversalUnstraightening}} to the unstraightening of $F$.
\end{lem}
\begin{proof}[Proof sketch]
	It's easy to construct $(p_\mathrm{univ}^{<\kappa})^*$ formally and we'll only sketch the necessary steps. First, one constructs a functor $\cat{An}_{/\cat{An}^{<\kappa}}\rightarrow \Fun(\Lambda_2^2,\cat{Cat}_\infty)$ that sends $(X\rightarrow \cat{An}^{<\kappa})$ to the span $(X\rightarrow \cat{An}^{<\kappa}\leftarrow \Uu_\mathrm{univ}^{<\kappa})$. To do so, let $\Xx\coloneqq\Fun(\Lambda_2^2,\cat{Cat}_\infty)\times_{\Fun(\Delta^{\{1,2\}},\cat{Cat}_\infty)}\{p_\mathrm{univ}^{<\kappa}\}$ be the $\infty$-category of those spans whose second leg is $p_\mathrm{univ}^{<\kappa}$. There's a functor
	\begin{equation*}
		\Xx\rightarrow \Fun\bigl(\Delta^{\{0,2\}},\cat{Cat}_\infty\bigr)\times_{\Fun(\{2\},\cat{Cat}_\infty)}\{\cat{An}^{<\kappa}\}\simeq (\cat{Cat}_\infty)_{/\cat{An}^{<\kappa}}
	\end{equation*}
	sending a span whose second leg is $p_\mathrm{univ}^{<\kappa}$ to its first leg. This functor is clearly essentially surjective, and one easily checks that it is fully faithful too, using the formulas from \cref{cor:HomInSliceCategories} and \cref{lem:HomInLimits}\cref{enum:HomInLimits}. Hence we get an equivalence by \cref{thm:EquivalenceFullyFaithfulEssentiallySurjective}. Choosing an inverse of this equivalence yields the desired functor $\cat{An}_{/\cat{An}^{<\kappa}}\rightarrow \Fun(\Lambda_2^2,\cat{Cat}_\infty)$.
	
	Next, one constructs a functor $\Fun(\Lambda_2^2,\cat{Cat}_\infty)\rightarrow \Ar(\cat{Cat}_\infty)$ that sends a span $(\Cc\rightarrow {\Dd}\leftarrow \Dd')$ to the pullback $\Cc\times_\Dd\Dd'\rightarrow \Cc$. To do so, let $\Yy\subseteq \Fun(\square^2,\cat{Cat}_\infty)$, where $\square^2\simeq \Delta^1\times\Delta^1$, be the full sub-$\infty$-category spanned by the pullback squares. Then $\Yy\rightarrow \Fun(\Lambda_2^2,\cat{Cat}_\infty)$ is essentially surjective, since $\cat{Cat}_\infty$ has pullbacks, and fully faithful by an easy application of \cref{cor:HomInFunctorCats} and \cref{cor:HomPreservesColimits}. Hence it is an equivalence by \cref{thm:EquivalenceFullyFaithfulEssentiallySurjective}. Choosing an inverse and composing it with the projection $\Fun(\square^2,\cat{Cat}_\infty)\rightarrow \Fun(\Delta^1\times\{0\},\cat{Cat}_\infty)\simeq \Ar(\cat{Cat}_\infty)$ yields the desired functor $\Fun(\Lambda_2^2,\cat{Cat}_\infty)\rightarrow \Ar(\cat{Cat}_\infty)$.
	
	Putting everything together yields a functor $\cat{An}_{/\cat{An}^{<\kappa}}\rightarrow \Ar(\cat{Cat}_\infty)$, which, on objects, sends $(X\rightarrow \cat{An}^{<\kappa})$ to $(X\times_{\cat{An}^{<\kappa}}\Uu_\mathrm{univ}^{<\kappa}\rightarrow X)$. By inspection, our functor factors through the non-full sub-$\infty$-category $\Ar_{\pullbacksign \vphantom{^t}}^{<\kappa}(\cat{An})\rightarrow \Ar(\cat{Cat}_\infty)$ and we obtain a functor $(p_\mathrm{univ}^{<\kappa})^*$, as desired.
	
	To show that $(p_\mathrm{univ}^{<\kappa})^*$ is an equivalence, we'll once again verify that it is essentially surjective and fully faithful. Essential surjectivity reduces to the assertion that every morphism $\alpha\colon X\rightarrow Y$ is equivalent to a left fibration in $\Ar(\cat{An})$. Using the dual of \cref{lem:KanExtensionForRight}\cref{enum:RightCofinalLeftAdjoint}, this reduces to checking that every final morphism $X\rightarrow X'$ of animae is an equivalence. For \emph{cofinal} morphisms, this follows from $X\simeq \left|X\right|\simeq \colimit_{x\in X}*\simeq \colimit_{x'\in X'}*\simeq \left|X'\right|\simeq X'$ using \cref{lem:ColimitsInAnima}. For final morphisms, we can use the same argument to show that $X^\op\rightarrow (X')^\op$ is an equivalence and then $X\rightarrow X'$ must be an equivalence too.
	
	To show that $p_\mathrm{univ}^{<\kappa}$ is fully faithful, let $(F\colon X\rightarrow \cat{An}^{<\kappa})$ and $(G\colon Y\rightarrow \cat{An}^{<\kappa})$ be elements in $\cat{An}_{/\cat{An}^{<\kappa}}$ and let $p\colon \Uu\rightarrow X$ and $q\colon \Vv\rightarrow Y$ be the unstraightenings of $F$ and $G$, respectively. For brevity, let us put
	\begin{align*}
		\Hom(F,G)&\coloneqq \Hom_{\cat{An}_{/\cat{An}^{<\kappa}}}\left((F\colon X\rightarrow \cat{An}^{<\kappa}),(G\colon Y\rightarrow \cat{An}^{<\kappa})\right)\\
		\Hom(p,q)&\coloneqq \Hom_{\Ar_{ \pullbacksign \vphantom{^n} }^{<\kappa}(\cat{An})}\bigl((p\colon \Uu\rightarrow X),(q\colon \Vv\rightarrow Y)\bigr)
	\end{align*}
	By \cref{cor:HomInSliceCategories}, $\Hom(F,G)$ is the pullback $\Hom_{\cat{Cat}_\infty}\left(X,Y\right)\times_{\Hom_{\cat{Cat}_\infty}(X,\cat{An}^{<\kappa})}\{F\}$. By \cref{lem:HomInArrowCategories} and \cref{lem:NonFullSubcategory}, $\Hom(p,q)$ is a collection of path components of the pullback $\Hom_{\cat{Cat}_\infty}\left(\Uu,\Vv\right)\times_{\Hom_{\cat{Cat}_\infty}(\Uu,Y)}\Hom_{\cat{Cat}_\infty}\left(X,Y\right)$. By \cref{thm:Whitehead}, \cref{lem:LongExactFibrationSequence}, and the five lemma (plus \cref{rem:ExactnessInLowDegrees}), it's enough to check that $\Hom(F,G)\rightarrow \Hom(p,q)$ induces an equivalence on fibres over $\Hom_{\cat{Cat}_\infty}(X,Y)$. So fix $f\colon X\rightarrow Y$. If $F\not\simeq G\circ f$, then both fibres are empty by \cref{thm:Straightening}. If $F\simeq G\circ f$, then the fibre $\{f\}\times_{\Hom_{\cat{Cat}_\infty}(X,Y)}\Hom\left(F,G\right)$ is given by $\{f\}\times_{\Hom_{\cat{Cat}_\infty}(X,Y)}\left(\Hom_{\cat{Cat}_\infty}(X,Y\right)\times_{\Hom_{\cat{Cat}_\infty}(X,\cat{An}^{<\kappa})}\{F\})$, which can be simplified to
	\begin{equation*}
		\{G\circ f\}\times_{\Hom_{\cat{Cat}_\infty}(X,\cat{An}^{<\kappa})}\{F\}\simeq \Hom_{\core \Fun(X,\cat{An})}\left(G\circ f,F\right)
	\end{equation*}
	using \cref{thm:CordierPorter} and \cref{lem:SuspensionLoopAdjunction}\cref{enum:LoopIsHom}. Likewise, $\Hom(p,q)\times_{\Hom_{\cat{Cat}_\infty}(X,Y)}\{f\}$ is a collection of path components in $(\Hom_{\cat{Cat}_\infty}\left(\Uu,\Vv\right)\times_{\Hom_{\cat{Cat}_\infty}(\Uu,Y)}\Hom_{\cat{Cat}_\infty}\left(X,Y\right))\times_{\Hom_{\cat{Cat}_\infty}(X,Y)}\{f\}$. This pullback can be simplified to
	\begin{equation*}
		\Hom_{\cat{Cat}_\infty}\left(\Uu,\Vv\right)\times_{\Hom_{\cat{Cat}_\infty}(\Uu,Y)}\{p\circ f\}\simeq \Hom_{\cat{Cat}_{\infty/Y}}\left(\Uu,\Vv\right)\simeq \Hom_{\cat{Left}(X)}\left(\Uu,X\times_Y\Vv\right)\,.
	\end{equation*}
	In the first step we use \cref{cor:HomInSliceCategories} and in the second step we use the dual of \cref{lem:KanExtensionForRight}\cref{enum:RightPullbackLeftAdjoint} combined with the fact that $\cat{Left}(X)\subseteq \cat{Cat}_{\infty/X}$ is a full sub-$\infty$-category. Thus, the fibre $\Hom(p,q)\times_{\Hom_{\cat{Cat}_\infty}(X,Y)}\{f\}$ is a collection of path components of $\Hom_{\cat{Left}(X)}\left(\Uu,X\times_Y\Vv\right)$. A quick unravelling shows that the relevant path components are precisely those morphisms $\Uu\rightarrow X\times_Y\Vv$ that are equivalences. Since $\cat{Left}(X)\simeq \Fun(X,\cat{An})$ by \cref{thm:Straightening}\cref{enum:LeftStraightening}, this agrees with the collection of path components of $\Hom_{\Fun(X,\cat{An})}(G\circ f,F)$ spanned by the equivalences. By \cref{lem:NonFullSubcategory}, this is precisely $\Hom_{\core \Fun(X,\cat{An})}\left(G\circ f,F\right)$. This finishes the proof that $(p_\mathrm{univ}^{<\kappa})^*$ is fully faithful.
\end{proof}
\begin{proof}[Proof sketch of \cref{lem:RezkEquifibrancy}]
	Choose a cardinal $\kappa$ in such a way that all $A_j$, $B_j$, $C_j$, $D_j$, and the colimits are essentially $\kappa$-small in the sense of \cref{def:KappaSmall}. The natural transformation $q\colon B\Rightarrow D$ in $\Fun(\Jj,\cat{An})$ can be viewed as a functor $q\colon \Jj\rightarrow \cat{Ar}(\cat{An})$. The assumption that $q$ is equifibred and our choice of $\kappa$ guarantee that $q$ factors through $\Ar_{\pullbacksign\vphantom{^t}}^{<\kappa}(\cat{An})$. Applying the equivalence of $\infty$-categories $\Ar_{\pullbacksign\vphantom{^t}}^{<\kappa}(\cat{An})\simeq \cat{An}_{/\cat{An}^{<\kappa}}$ from \cref{lem:StraighteningFunctorial}, we see that $q$ corresponds to a functor $F\colon \Jj\rightarrow \cat{An}_{/\cat{An}^{<\kappa}}$. On objects, $F$ sends $j\in\Jj$ to $(F_j\colon D_j\rightarrow \cat{An}^{<\kappa})$ in $\cat{An}_{/\cat{An}^{<\kappa}}$ such that $(q_j\colon B_j\rightarrow D_j)$ is the unstraightening of $F_j$. By definition of the slice-$\infty$-category $\cat{An}_{/\cat{An}^{<\kappa}}$ we can view $F$ as a natural transformation $\eta\colon D\Rightarrow \const \cat{An}^{<\kappa}$ in $\Fun(\Jj,\cat{Cat}_\infty)$, hence it induces a functor $F_\infty\colon \colimit_{i\in\Ii}D_i\rightarrow \cat{An}^{<\kappa}$. We claim:
	\begin{alphanumerate}\itshape
		\item[\boxtimes] The unstraightening of $F_\infty$ is $\colimit_{j\in\Jj}B_j\rightarrow \colimit_{j\in\Jj}D_j$. In particular, we obtain the following pullback square:\label{claim:UnstraighteningColimitPullback}
		\begin{equation*}
			\begin{tikzcd}
				\colimit_{j\in\Jj}B_j\rar\dar\drar[pullback] & \Uu_\mathrm{univ}^{<\kappa}\dar["p_\mathrm{univ}^{<\kappa}"]\\
				\colimit_{j\in\Jj}D_j\rar["F_\infty"] & \cat{An}^{<\kappa}
			\end{tikzcd}
		\end{equation*}
	\end{alphanumerate}
	If we know \cref{claim:UnstraighteningColimitPullback}, then we're done. Indeed, our construction of $F$ above exhibits $q\colon B\Rightarrow D$ as a pullback of $\const p_\mathrm{univ}^{<\kappa}\colon \const \Uu_\mathrm{univ}^{<\kappa}\Rightarrow \const \cat{An}^{<\kappa}$. Then $p\colon A\Rightarrow C$ must be a pullback of $\const p_\mathrm{univ}^{<\kappa}$ as well, hence $p$ is equifibred again. The same reasoning as above then shows that $\colimit_{j\in\Jj}A_j\rightarrow \colimit_{j\in\Jj}C_j$ must too be a pullback of $p_\mathrm{univ}^{<\kappa}\colon \Uu_\mathrm{univ}^{<\kappa}\rightarrow\cat{An}^{<\kappa}$. Hence the square formed by the colimits must be a pullback as well.
	
	To prove \cref{claim:UnstraighteningColimitPullback}, note that $B_j\simeq \left|B_j\right|\simeq \colimit (F_j\colon D_j\rightarrow \cat{An}^{<\kappa})$ follows from \cref{lem:ColimitsInAnima}. So \cref{lem:ColimitManipulations}\cref{claim:AssembleColimits} shows
	\begin{equation*}
		\colimit_{j\in\Jj}B_i\simeq \colimit_{j\in\Jj}\left(\colimit(F_j\colon D_j\rightarrow \cat{An}^{<\kappa})\right)\simeq \colimit\Bigl(F_\infty\colon \colimit_{j\in\Jj}D_j\rightarrow\cat{An}^{<\kappa}\Bigr)
	\end{equation*}
	But the colimit on the right-hand side is the unstraightening of $F_\infty$, again by \cref{lem:ColimitsInAnima}. However, there's a subtlety: To make this argument work, we have to show that the \emph{functor} $B\colon \Jj\rightarrow \cat{An}^{<\kappa}$ agrees with the \emph{functor} $(j\mapsto \colimit F_j)\colon \Jj\rightarrow\cat{An}^{<\kappa}$ constructed in the proof of \cref{lem:ColimitManipulations}\cref{claim:AssembleColimits}; let's temporarily denote this functor by $B'$. So far, we've only verified that the values of $B$ and $B'$ coincide! 
	
	To fix this, let $\Bb\rightarrow \Jj$ and $\Dd\rightarrow \Jj$ be the unstraightenings of the functors $B\colon \Jj\rightarrow \cat{An}^{<\kappa}$ and $D\colon \Jj\rightarrow \cat{An}^{<\kappa}$. Let's first recall the construction of $B'$: By the proof of \cref{lem:ColimitManipulations}\cref{claim:AssembleColimits}, we have a diagram
	\begin{equation*}
		\begin{tikzcd}
			\Dd\rar["d"]\dar &\left|\Dd\right|\simeq \colimit_{j\in\Jj}D_j\rar["F_\infty"]\dlar[phantom,start anchor=center,end anchor=center,"\Longleftarrow"{sloped,pos=0.5}] & \cat{An}^{<\kappa}\\
			\Jj\ar[urr,bend right=15,dashed,"B'"'] & & 
		\end{tikzcd}
	\end{equation*}
	that exhibits $B'$ as the left Kan extension of $F_\infty\circ d\colon \Dd\rightarrow \cat{An}^{<\kappa}$ along $\Dd\rightarrow \Jj$. We know from the dual of \cref{lem:KanExtensionForRight}\cref{enum:RightPullbackLeftAdjoint} how left Kan extensions for functors into $\cat{An}$ interact with unstraightening. Namely, the unstraightening $\Bb'\rightarrow \Jj$ of $B'\colon \Jj\rightarrow \cat{An}^{<\kappa}$ is given by factoring the unstraightening of $F_\infty\circ d\colon \Dd\rightarrow \cat{An}^{<\kappa}$ into a final functor followed by a left fibration. In particular, if we can show that $\Bb\rightarrow\Dd$ is the unstraightening of $F_\infty\circ d\colon \Dd\rightarrow \cat{An}^{<\kappa}$, then we're done, because $\Bb\rightarrow\Jj$ is already a left fibration and so we would be able to deduce $\Bb\simeq \Bb'$.
	
	To show that $\Bb\rightarrow \Dd$ is the desired unstraightening, recall that $F\colon \Jj\rightarrow \cat{An}_{/\cat{An}^{<\kappa}}$ is equivalently given by a natural transformation $\eta\colon D\Rightarrow \const \cat{An}^{<\kappa}$ in $\Fun(\Jj,\cat{An})$. So we obtain a morphism $\Dd\rightarrow \Jj\times\cat{An}^{<\kappa}$ of cocartesian fibrations over $\Jj$. Now consider the diagram
	\begin{equation*}
		\begin{tikzcd}
			\Bb\rar\dar\drar[pullback] & \Jj\times\Uu_\mathrm{univ}^{<\kappa}\rar\dar\drar[pullback] & \Uu_\mathrm{univ}^{<\kappa}\dar["p_\mathrm{univ}^{<\kappa}"]\\
			\Dd\rar & \Jj\times\cat{An}^{<\kappa}\rar & \cat{An}^{<\kappa}
		\end{tikzcd}
	\end{equation*}
	The right square is a pullback for obvious reasons. To see that the left square is a pullback too, observe that the natural transformation $q\colon B\Rightarrow D$ is the pullback of $\eta\colon D\Rightarrow\const \cat{An}^{<\kappa}$ along $\const p_\mathrm{univ}^{<\kappa}\colon \const \Uu_\mathrm{univ}^{<\kappa}\Rightarrow \const \cat{An}^{<\kappa}$; this follows by construction of $\eta$, unravelling the proof of \cref{lem:StraighteningFunctorial} and using that pullbacks in functor categories can be computed pointwise by \cref{lem:ColimitsInFunctorCategories}. Since unstraightening is an equivalence of $\infty$-categories, it preserves pullbacks, and so the left square must be a pullback too.\footnote{We've used similar arguments in the proofs of \cref{lem:HomRealityCheck,lem:HomFunctorial}, except back then we couldn't talk about pullbacks in $\infty$-categories yet} It follows that the outer rectangle in the diagram above must be a pullback too. But then $\Bb\rightarrow \Dd$ is a pullback of the universal unstraightening and thus $\Bb$ is the unstraightening of the bottom composition $\Dd\rightarrow \Jj\times\cat{An}^{<\kappa}\rightarrow \cat{An}^{<\kappa}$ by \cref{par:UniversalUnstraightening}. So it remains to identify that composition with $F_\infty\circ d\colon \Dd\rightarrow \cat{An}^{<\kappa}$. This follows from a closer investigation of the proof of \cref{lem:ColimitsInAnima}.
\end{proof}
With Rezk's equifibrancy condition from \cref{lem:RezkEquifibrancy}, we have obtained one of the two ingredients in the proof of \cref{claim:HomBG}. The other one is a general construction for simplicial objects.
\begin{con}\label{con:Decalage}
	Let $X\colon \IDelta^\op\rightarrow \Cc$ be a simplicial object in an arbitrary $\infty$-category $\Cc$. We picture $X$ as
	\begin{equation*}
		X\simeq \Biggl(\begin{tikzcd}[cramped,column sep=\the\longarrowlength,shorten=0.18ex]
			X_0 \rar & \lar[shift left=0.4em,"d_1^*"]\lar[shift right=0.4em,"d_0^*"']X_1\rar[shift left=0.4em]\rar[shift right=0.4em] & \lar[shift left=0.8em,"d_2^*"]\lar\lar[shift right=0.8em,"d_0^*"']X_2\ \dotsb
		\end{tikzcd}\Biggr)
	\end{equation*}
	(for typographical reasons, we couldn't label the degeneracy maps nor the inner face maps). The \emph{décalage of $X$} is another simplicial object $\operatorname{d\acute{e}c}(X)\colon \IDelta^\op\rightarrow \Cc$ given by \enquote{shifting} $X$, thus \enquote{forgetting} $X_0$ as well as all the face maps $d_0^*$ and all the degeneracy maps $s_0^*$. In pictures:
	\begin{equation*}
		\operatorname{d\acute{e}c}(X)\simeq \Biggl(\begin{tikzcd}[cramped,column sep=\the\longarrowlength,shorten=0.18ex]
			X_1 \rar & \lar[shift left=0.4em,"d_2^*"]\lar[shift right=0.4em,"d_1^*"']X_2\rar[shift left=0.4em]\rar[shift right=0.4em] & \lar[shift left=0.8em,"d_3^*"]\lar\lar[shift right=0.8em,"d_1^*"']X_3\ \dotsb
		\end{tikzcd}\Biggr)\,.
	\end{equation*}
	More precisely, there's a functor $\sigma\colon \IDelta\rightarrow \IDelta$ given by $\sigma([n])\coloneqq[n+1]$ on objects. A morphism $\alpha\colon [m]\rightarrow [n]$ is sent to $\sigma(\alpha)\colon [m+1]\rightarrow [n+1]$ given by $\sigma(\alpha)(0)\coloneqq 0$ and $\sigma(\alpha)(i)\coloneqq\alpha(i-1)+1$ for all $1\leqslant i\leqslant n+1$. Then $\operatorname{d\acute{e}c}(X)$ is simply the composition $X\circ \sigma^\op\colon \IDelta^\op\rightarrow \Cc$. The décalage sits inside a diagram
	\begin{equation*}
		\begin{tikzcd}
			\const X_0\doublear{r}\doublear["\id_{\const X_0}"{black,swap}]{dr} & \operatorname{d\acute{e}c}(X)\doublear["d_\mathrm{last}^*"{black,right=0.1em}]{d}\arrow[from=1-2,to=2-1,commutes,pos=0.3] \doublear["d_0^*"{black,above=0.1em}]{r} & X\\
			{ } & \const X_0 &
		\end{tikzcd}
	\end{equation*}
	The transformation $\const X_0\Rightarrow \operatorname{d\acute{e}c}(X)$ is induced by the unique transformation $\sigma\Rightarrow \const {[0]}$ in $\Fun(\IDelta,\IDelta)$. This transformation has a left inverse $d_\mathrm{last}\colon \const {[0]}\Rightarrow \sigma$ given object-wise by the maps $[0]\rightarrow \sigma([n])=[n+1]$ that send $0\mapsto 0$. These maps can be written as compositions $d_{n+1}\circ d_n\circ \dotsb\circ d_1\colon [0]\rightarrow [1]\rightarrow \dotsb\rightarrow [n]\rightarrow [n+1]$, whence the notation $d_\mathrm{last}$. The natural transformation $d_\mathrm{last}$ induces a transformation $d_\mathrm{last}^*\colon \operatorname{d\acute{e}c}(X)\Rightarrow \const X_0$. Finally, the maps $d_0\colon [n]\rightarrow [n+1]=\sigma([n])$ induce a natural transformation $d_0\colon \id_{\IDelta}\Rightarrow \sigma$, which in turn induces a transformation $d_0^*\colon \operatorname{d\acute{e}c}(X)\Rightarrow X$.
\end{con}
\begin{lem}\label{lem:DecalageColimit}
	If $\Cc$ is any $\infty$-category and $X\colon \IDelta^\op\rightarrow\Cc$ is a simplicial object in $\Cc$, then the diagram from \cref{con:Decalage} induces equivalences
	\begin{equation*}
		\colimit_{[n]\in\IDelta^\op}\operatorname{d\acute{e}c}(X)_n\overset{\simeq}{\longrightarrow}\colimit_{[n]\in\IDelta^\op}X_0\simeq X_0\,.
	\end{equation*}
	In particular, these colimits always exist in $\Cc$.
\end{lem}
\begin{proof}[Proof sketch]
	The equivalence $\colimit_{[n]\in\IDelta^\op}X_0\simeq X_0$ follows from \cref{lem:ContractibleColimit} and the fact that $\left|\IDelta^\op\right|\simeq *$, since $\IDelta$ has a terminal object, namely $[0]$. To show $\colimit_{[n]\in\IDelta^\op}\operatorname{d\acute{e}c}(X)_n\simeq X_0$, note that $\sigma\colon \IDelta\rightarrow \IDelta$ can be identified with the inclusion $\IDelta_{\geqslant 1}\rightarrow\IDelta$ of the (non-full) subcategory spanned by $[n+1]$ for all $n\geqslant 0$ and all morphisms $\alpha\colon [m+1]\rightarrow [n+1]$ satisfying $\alpha^{-1}\{0\}=\{0\}$. Furthermore, let $\IDelta_0\rightarrow \IDelta$ be the (non-full) subcategory spanned by all objects but only those morphisms that send $0\mapsto 0$.
	
	Via this reinterpretation, $\colimit_{[n]\in\IDelta^\op}\operatorname{d\acute{e}c}(X)_n\simeq \colimit_{[n]\in\IDelta_{\geqslant 1}^\op}X_n$.
	On the other hand, it's straightforward to check that $[0]\in \IDelta_0$ is an initial object; therefore, $\colimit_{[n]\in \IDelta_0^\op}X_n\simeq X_0$. So it would be enough to show that $\IDelta_{\geqslant 1}^\op\rightarrow \IDelta_0^\op$ is cofinal, or equivalently, that $\IDelta_{\geqslant 1}\rightarrow \IDelta_0$ is final. By the dual of \cref{thm:JoyalsQuillenA}\cref{enum:WeaklyContractible}, we must show that $\mathopen|\IDelta_{\geqslant 1}\times_{\IDelta_0}\IDelta_{0/[n]}\mathclose|\simeq 0$ for all $n\geqslant 0$. The case $n=0$ is clear: It's straightforward to see that $[0]\in \IDelta_{0}$ is also a terminal object, so that $\IDelta_{0/[0]}\simeq \IDelta_0$ and thus $\mathopen|\IDelta_{\geqslant 1}\times_{\IDelta_0}\IDelta_{0/[0]}\mathclose|\simeq \mathopen|\IDelta_{\geqslant 1}\mathclose|\simeq *$, since $[1]\in \IDelta_{\geqslant 1}$ is terminal. Now let $n\geqslant 1$ and consider the full subcategory $\Xx\subseteq \IDelta_{\geqslant 1}\times_{\IDelta_0}\IDelta_{0/[n]}$ spanned by those $\alpha\colon [m+1]\rightarrow [n]$ such that $\alpha$ maps $\alpha^{-1}\{1,\dotsc,n\}$ bijectively to $\{1,\dotsc,n\}$. It's straightforward to check that this inclusion has a left adjoint $\IDelta_{\geqslant 1}\times_{\IDelta_0}\IDelta_{0/[n]}\rightarrow \Xx$.%
	\footnote{The left adjoint can be constructed as follows: Let $(\alpha\colon [m+1]\rightarrow [n])\in \IDelta_{\geqslant 1}\times_{\IDelta_0}\IDelta_{0/[n]}$. Then there exists some $0\leqslant k\leqslant m$ such that $\alpha^{-1}\{0\}=\{0,1,\dotsc,k+1\}$. Let $\ov\alpha\colon [k+n+1]\rightarrow[n]$ be defined by $\ov\alpha(i)=0$ for $i=0,1,\dotsc,k+1$ and $\ov\alpha(i)=i-(k+1)$ for $i\geqslant k+2$. Then $\ov\alpha\in \Xx$. Furthermore, there's a canonical morphism $u_\alpha\colon \alpha\rightarrow \ov\alpha$ in $\IDelta_{\geqslant 1}\times_{\IDelta_0}\IDelta_{0/[n]}$, given by the identity on $\{0,1,\dotsc,k+1\}$ and $u_\alpha(i)=\alpha(i)+k+1$ for $i\geqslant k+2$. Then $\alpha\mapsto \ov\alpha$ is the desired left adjoint and $u_\alpha$ is the unit of the adjunction.}
	%
	Since adjunctions induce equivalences after $\left|\,\cdot\,\right|$, it's enough to show $\left|\Xx\right|\simeq *$. But now it's straightforward to check that $(\id_{[n]}\colon [n]\rightarrow [n])$ is an inital object of $\Xx$.
\end{proof}
Now we can finally finish the proof of \cref{thm:E1Loop}\cref{enum:E1LoopGrp}.
\begin{proof}[Proof sketch of \cref{thm:E1Loop}\cref{enum:E1LoopGrp}, claim~\cref{claim:HomBG}]
	Let $G\in\cat{Grp}(\cat{An})$ be an $\IE_1$-group. Using the Segal condition from \cref{def:E1Monoids}\cref{enum:E1Monoid}, one verifies that the following is a the pullback square in $\Fun(\IDelta^\op,\cat{An})$:
	\begin{equation*}
		\begin{tikzcd}
			\const G_1\doublear{r}\doublear{d}\drar[pullback] & \operatorname{d\acute{e}c}(G)\doublear["d_0^*"{black,right=0.1em}]{d}\\
			\const *\doublear{r} & G
		\end{tikzcd}
	\end{equation*}
	Note that $G$ being an $\IE_1$-group as opposed to merely an $\IE_1$-monoid implies that $d_0^*\colon \operatorname{d\acute{e}c}(G)\Rightarrow G$ from \cref{con:Decalage} is equifibred in the sense of \cref{lem:RezkEquifibrancy}. Indeed, being an $\IE_1$-group means that $(\pr_1,\mu)\colon G_1\times G_1\rightarrow G_1\times G_1$ is an equivalence, so that all occurences of the multiplication map $\mu$ in \cref{par:AssociahedraII} can be replaced by simple projections, and then equifibrancy is straightforward to check. \cref{lem:RezkEquifibrancy} now implies that the central square of the diagram
	\begin{equation*}
		\begin{tikzcd}
			G_1\dar\rar["\simeq"]\drar[commutes] &\colimit_{[n]\in\IDelta^\op}G_1\dar\rar\drar[pullback] & \colimit_{[n]\in\IDelta^\op}\operatorname{d\acute{e}c}(G)_n\dar\drar[commutes] & G_0\lar["\simeq"']\dar\\
			*\rar["\simeq"] & \colimit_{[n]\in\IDelta^\op}*\rar & \colimit_{[n]\in\IDelta^\op}G_n & \B G\lar["\simeq"']
		\end{tikzcd}
	\end{equation*}
	is a pullback. The equivalences on the left follows from \cref{lem:ContractibleColimit}. The top right equivalence is due to \cref{lem:DecalageColimit}. The bottom right equivalence is the definition of $\B G$. Since $G_0\simeq *$, this diagram shows $G_1\simeq \Omega\B G$, which is precisely what we claimed in \cref{claim:HomBG}.
\end{proof}
Here are some immediate consequences of \cref{thm:E1Loop}\cref{enum:E1LoopGrp}:
\begin{cor}[\enquote{$\Omega\B$ is group completion}]\label{cor:GroupCompletion}
	The inclusion $\cat{Grp}(\cat{An})\subseteq\cat{Mon}(\cat{An})$ of $\IE_1$-groups into $\IE_1$-monoids has a left adjoint, given by $\Omega\B\colon \cat{Mon}(\cat{An})\rightarrow\cat{Grp}(\cat{An})$.\hfill$\qedsymbol$
\end{cor}
\begin{cor}[\enquote{$\Omega\Sigma X_+$ is the free $\IE_1$-group on $X$}]\label{cor:FreeE1Group}
	The functor $\ev_{[1]}\colon\cat{Grp}(\cat{An})\rightarrow \cat{An}$ sending an $\IE_1$-group to its underlying anima has a left adjoint, sending an anima $X$ to $\Omega\Sigma X_+$, where $X_+\coloneqq X\sqcup *$, regarded as a pointed anima.
\end{cor}
\begin{proof}
	It's straightforward to check (for example, using \cref{cor:HomPreservesLimits} and \cref{cor:HomInSliceCategories}) that $(-)_+\coloneqq (-)\sqcup *\colon \cat{An}\rightarrow \cat{An}_{*/}$ is a left adjoint to the forgetful functor $\cat{An}_{*/}\rightarrow\cat{An}$. Combining this observation with \cref{lem:SuspensionLoopAdjunction} and \cref{thm:E1Loop}\cref{enum:E1LoopGrp} yields a diagram of adjunctions
	\begin{equation*}
		\begin{tikzcd}
			\cat{An}\rar[shift left=0.2em,"{(-)_+}"]\ar[drr,bend right=15.5,shorten <=0.4ex,shorten >=0.1ex,shift left=0.2em,"{\Omega\Sigma(-)_+}"{pos=0.45},start anchor=300,end anchor=175] & \lar[shift left=0.2em]\cat{An}_{*/} \rar[shift left=0.2em,"\Sigma"]\drar[commutes,pos=0.45,xshift=0.5em] & \lar[shift left=0.2em,"\Omega"](\cat{An}_{*/})_{\geqslant 1}\dar[shift left=0.2em,"\Omega"]\\
			& & \cat{Grp}(\cat{An})\uar[shift left=0.2em,"\B"] \ar[ull,bend left=15,shift left=0.2em,"\ev_{[1]}",end anchor=300,start anchor=175]
		\end{tikzcd}
	\end{equation*}
	which shows that $\Omega\Sigma(-)_+\colon \cat{An}\shortdoublelrmorphism \cat{Grp}(\cat{An})\noloc \ev_{[1]}$ must be an adjunction too.
\end{proof}
Another immediate consequence of \cref{thm:E1Loop}\cref{enum:E1LoopGrp} is the Seifert--van Kampen theorem.
\begin{thm}[Seifert--van Kampen]
	The functor $\pi_1\colon (\cat{An}_{*/})_{\geqslant 1}\rightarrow\cat{Grp}$ preserves pushouts. That is, the fundamental group of a pushout of pointed connected animae is given by the pushout of fundamental groups, taken in the category $\cat{Grp}$ of groups.
\end{thm}
\begin{proof}
	Let $(X,x)$ be a pointed anima. By \cref{lem:SuspensionLoopAdjunction}, we have $\pi_1(X,x)\cong \pi_0(\Omega X)$. By \cref{thm:E1Loop}\cref{enum:E1LoopGrp}, the functor $\Omega\colon (\cat{An}_{*/})_{\geqslant 1}\rightarrow\cat{Grp}(\cat{An})$ is an equivalence of $\infty$-categories, so it preserves pushouts. The functor $\pi_0\colon \cat{An}\rightarrow \cat{Set}$ is left adjoint to the inclusion $i\colon \cat{Set}\rightarrow\cat{An}$ given by regarding sets as discrete animae. By \cref{cor:FunctorCategoryAdjunctions}, this implies that there is an adjunction $(\pi_0)_*\colon \Fun(\IDelta^\op,\cat{An})\shortdoublelrmorphism \Fun(\IDelta^\op,\cat{Set})\noloc i_*$. Note that $\pi_0$ and $i$ both preserve products. Hence $(\pi_0)_*$ and $i_*$ preserve the conditions from \cref{def:E1Monoids} and so the adjunction above restricts to an adjunction $\pi_0\colon \cat{Grp}(\cat{An})\shortdoublelrmorphism \cat{Grp}(\cat{Set})\noloc i$. In particular, $\pi_0\colon \cat{Grp}(\cat{An})\rightarrow \cat{Grp}(\cat{Set})\simeq \cat{Grp}$ is a left adjoint and so it preserves pushouts. It follows that $\pi_1\cong \pi_0\circ \Omega$ preserves pushouts too, as claimed.
\end{proof}

\subsection{\texorpdfstring{$\IE_\infty$}{E-inftinity}-monoids and \texorpdfstring{$\IE_\infty$}{E-infinity}-groups}\label{subsec:Einfty}
Our next goal is to study the analogue of commutative monoids and commutative groups in animae. The definition is quite similar to \cref{def:E1Monoids}, except that we have to replace $\IDelta^\op$ by a category that encodes commutativity as well.
\begin{defi}\label{def:EinftyMonoid}
	Let $\cat{Fin}$ be the ordinary category of finite sets $\langle n\rangle =\{1,\dotsc,n\}$ for $n\geqslant 0$ and partially defined (!) maps. Let $\Cc$ be an $\infty$-category with finite products.
	\begin{alphanumerate}
		\item An \emph{$\IE_\infty$-monoid in $\Cc$} is a functor $M\colon \cat{Fin}\rightarrow \Cc$ satisfying $M_0\simeq *$ as well as the \emph{Segal condition}: The \emph{Segal maps} $e_i\colon \langle n\rangle\rightarrow \langle 1\rangle$, where $e_i$ is everywhere undefined except at $i$, induce an equivalence\label{enum:EinftyMonoid}
		\begin{equation*}
			M_n\overset{\simeq}{\longrightarrow}M_1^n\,.
		\end{equation*}
		We call $M_1$ the \emph{underlying object of $M$}; we'll often don't distinguish between $M$ and $M_1$. Let $\cat{CMon}(\Cc)\subseteq\Fun(\cat{Fin},\Cc)$ denote the full sub-$\infty$-category spanned by the $\IE_\infty$-monoids.
		\item An $\IE_\infty$-monoid $M$ in $\Cc$ is called an \emph{$\IE_\infty$-group} if its underlying $\IE_1$-monoid in the sense of \cref{con:EinftyUnderlyingE1} below is an $\IE_1$-group. We let $\cat{CGrp}(\Cc)\subseteq \cat{CMon}(\Cc)$ denote the full sub-$\infty$-category spanned by $\IE_\infty$-groups. \label{enum:EinftyGroup}
	\end{alphanumerate}
\end{defi}
\begin{con}\label{con:EinftyMultiplication}
	Let's unravel how \cref{def:EinftyMonoid}\cref{enum:EinftyMonoid} encodes a commutative multiplication on $M_1$. The unique everywhere defined map $f_2\colon \langle 2\rangle\rightarrow \langle 1\rangle$ induces a morphism
	\begin{equation*}
		\mu\colon M_1\times M_1\simeq M_2\longrightarrow M_1\,.
	\end{equation*}
	This is our multiplication. Now let's see why it is commutative: Let $\operatorname{flip}\colon \langle 2\rangle\rightarrow \langle2\rangle$ be the everywhere defined map that sends $1\mapsto 2$ and $2\mapsto 1$. Then $f_2\circ \operatorname{flip}=f_2$ and so the following diagram commutes in $\Cc$:
	\begin{equation*}
		\begin{tikzcd}[row sep=small]
			M_1\times M_1\ar[dd,"\operatorname{flip}"']\drar[bend left=20,"\mu"] & \\
			\phantom{X}\rar[commutes,pos=0.4] & M_1\\
			M_1\times M_1\urar[bend right=20, "\mu"'] &
		\end{tikzcd}
	\end{equation*}
	Here $\operatorname{flip}\colon M_1\times M_1\rightarrow M_1\rightarrow M_1$ is the morphism that flips the two factors; under the Segal isomorphism $M_1\times M_1\simeq M_2$, this really corresponds to $\operatorname{flip}\colon M_2\rightarrow M_2$, so the notational overload checks out. 
\end{con}
\begin{con}\label{con:EinftyUnderlyingE1}
	Let us construct an underlying $\IE_1$-monoid to every $\IE_\infty$-monoid $M$. To this end, we'll construct a functor $\operatorname{Cut}\colon \IDelta^\op\rightarrow\cat{Fin}$. On objects, $\operatorname{Cut}$ is given by $\operatorname{Cut}([n])\coloneqq\langle n\rangle $. A map $\alpha\colon [m]\rightarrow [n]$ in $\IDelta$, which corresponds to a morphism $[n]\rightarrow [m]$ in $\IDelta^\op$, is sent to $\operatorname{Cut}(\alpha)\colon \langle n\rangle \rightarrow\langle m\rangle$ given by the formula
	\begin{equation*}
		\operatorname{Cut}(\alpha)(i)\coloneqq \begin{cases*}
			j & if $\alpha(j-1)< i\leqslant \alpha(j)$\\
			\text{undefined} & else
		\end{cases*}\,.
	\end{equation*}
	A more conceptual way of saying this is that $\operatorname{Cut}$ sends $[n]$ to its set of \emph{Dedekind cuts}, that is, to the set of all partitions of $[n]$ into two non-empty intervals (of which there are exactly $n$, so $\operatorname{Cut}([n])=\langle n\rangle $). The map $\operatorname{Cut}(\alpha)\colon \operatorname{Cut}([n])\rightarrow\operatorname{Cut}([m])$ sends such a partition of $[n]$ to its preimage under $\alpha$, which is again a partition of $[m]$ into intervals. However, it may happen that one of the intervals is empty; if this is the case, we define the value of $\operatorname{Cut}(\alpha)$ as undefined.
	
	Now $\operatorname{Cut}$ induces a precomposition functor $\operatorname{Cut}^*\colon \Fun(\cat{Fin},\Cc)\rightarrow\Fun(\IDelta^\op,\Cc)$. It's straightforward to check that $\operatorname{Cut}(e_i)=e_i$, that is, $\operatorname{Cut}$ sends the Segal maps in $\IDelta^\op$ to the Segal maps in $\cat{Fin}$. Hence $\operatorname{Cut}^*$ preserves the Segal condition from and therefore restricts to a functor
	\begin{equation*}
		\operatorname{Cut}^*\colon \cat{CMon}(\Cc)\longrightarrow\cat{Mon}(\Cc)\,.
	\end{equation*}
	For an $\IE_\infty$-monoid $M$, we call $\operatorname{Cut}^*(M)$ the \emph{underlying $\IE_1$-monoid of $M$}. As with the underlying object, we often abuse notation and identify $M$ with its underlying $\IE_1$-monoid.
\end{con}
Our eventual goal in this subsection is to prove an analogue of \cref{thm:E1Loop}\cref{enum:E1LoopGrp} for $\IE_\infty$-monoids/groups. This needs some preparations.
\begin{defi}\label{def:Additive}
	Let $\Cc$ be an $\infty$-category  with finite coproducts and finite products (in particular, it has both an initial and a terminal object).
	\begin{alphanumerate}
		\item $\Cc$ is called \emph{semi-additive} if the initial object, which we denote $0\in\Cc$, is also terminal, and for all $x,y\in \Cc$ the natural map\label{enum:SemiAdditive}
		\begin{equation*}
			\begin{pmatrix}
				\id_x & 0\\
				0 & \id_y
			\end{pmatrix}\colon x\sqcup y\overset{\simeq}{\longrightarrow} x\times y
		\end{equation*}
		is an equivalence. Here $0\colon x\rightarrow 0\rightarrow y$ denotes the unique (up to contractible choice) morphism in $\Hom_\Cc(x,y)$ factoring through $0$. If $\Cc$ is semi-additive, we usually write $x\sqcup y\simeq x\oplus y\simeq x\times y$.
		\item $\Cc$ is called \emph{additive} if it is semi-additive and additionally for all $x\in \Cc$ the \emph{shearing morphism} is an equivalence:\label{enum:Additive}
		\begin{equation*}
			\begin{pmatrix}
				\id_x & \id_x\\
				0 & \id_x
			\end{pmatrix}\colon x\oplus x\overset{\simeq}{\longrightarrow} x\oplus x\,.
		\end{equation*}	
	\end{alphanumerate}
\end{defi}
\begin{lem}[\enquote{Every object in an additive $\infty$-category is canonically an $\IE_\infty$-group}]\label{lem:CGrpIsC}
	If $\Cc$ is a semi-additive $\infty$-category, then $\cat{CMon}(\Cc)\simeq \cat{Mon}(\Cc)\simeq \Cc$. If $\Cc$ is an additive $\infty$-category, then also $\cat{CGrp}(\Cc)\simeq \cat{Grp}(\Cc)\simeq \Cc$.
\end{lem}
\begin{proof}[Proof sketch]
	Let $\cat{Fin}_{\leqslant 1}\simeq \{\InlineFin\}$ denotes the full subcategory of $\cat{Fin}$ spanned by $\langle 0\rangle$ and $\langle 1\rangle$ and let $\cat{Fin}_{\leqslant 1}^\circ \subseteq \cat{Fin}_{\leqslant1}$ denote the non-full subcategory given by $\{\begin{tikzcd}[cramped, column sep=small,ampersand replacement=\&]\langle0\rangle\&\lar\langle 1\rangle\end{tikzcd}\}$. The proof rests upon the following two crucial observations:
	\begin{alphanumerate}\itshape
		\item[\boxtimes_1] A functor $F\colon \cat{Fin}\rightarrow \Cc$ with $F(\langle 0\rangle)\simeq 0$ satisfies the Segal condition from \cref{def:EinftyMonoid}\cref{enum:EinftyMonoid} if and only if $F$ is the left Kan extension of its own restriction along $i\colon\cat{Fin}_{\leqslant 1}\rightarrow \cat{Fin}$.\label{claim:SegalConditionLeftKan}
		\item[\boxtimes_2] If $\Fun_*\subseteq \Fun$ denotes the full sub-$\infty$-category spanned by those functors that send $\langle 0\rangle\mapsto 0$, then restriction along $j\colon \cat{Fin}_{\leqslant 1}^\circ \rightarrow \cat{Fin}_{\leqslant 1}$ and evaluation at $\langle 1\rangle$ induces equivalences\label{claim:LeftKanEasier}
		\begin{equation*}
			\Fun_*\left(\cat{Fin}_{\leqslant 1},\Cc\right)\overset{\simeq}{\longrightarrow}\Fun_*\left(\cat{Fin}_{\leqslant 1}^\circ,\Cc\right)\overset{\simeq}{\longrightarrow}\Cc\,.
		\end{equation*}
	\end{alphanumerate}
	We begin with \cref{claim:LeftKanEasier}. Since $0\in\Cc$ is terminal, it's clear that $\ev_{\langle 1\rangle}\colon \Fun_*\left(\cat{Fin}_{\leqslant 1}^\circ,\Cc\right)\rightarrow \Cc$ is essentially surjective. By an easy application of \cref{cor:HomInFunctorCats}, using that $\cat{Fin}_{\leqslant 1}^\circ$ is an ordinary category and so we understand its twisted arrow category $\TwAr(\cat{Fin}_{\leqslant 1})$, $\ev_{\langle 1\rangle}$ is also fully faithful. Alternatively one could also use \cref{lem:HomInArrowCategories} (which amounts to the same). Hence $\ev_{\langle1\rangle}$ is an equivalence by \cref{thm:EquivalenceFullyFaithfulEssentiallySurjective}.
	
	To show that $j^*\colon \Fun_*\left(\cat{Fin}_{\leqslant 1},\Cc\right)\rightarrow\Fun_*\left(\cat{Fin}_{\leqslant 1}^\circ,\Cc\right)$ is an equivalence, we consider left Kan extension along $j$. To this end, let $F\colon \cat{Fin}_{\leqslant1}^\circ\rightarrow \Cc$ be a functor satisfying $F(\langle0\rangle)\simeq 0$. We unravel the Kan extension formula from \cref{lem:KanExtensionFormula}: We have $\cat{Fin}_{\leqslant 1}^\circ\times_{\cat{Fin}_{\leqslant 1}}(\cat{Fin}_{\leqslant 1})_{/\langle 0\rangle}\simeq (\cat{Fin}_{\leqslant 1}^\circ)_{/\langle 0\rangle}$, and so the colimit describing $\Lan_jF(\langle 0\rangle)$ exists and is given by evaluating at the terminal object $(\id_{\langle 0\rangle}\colon \langle 0\rangle\rightarrow \langle 0\rangle)\in (\cat{Fin}_{\leqslant 1}^\circ)_{/\langle 0\rangle}$. Hence $\Lan_jF(\langle 0\rangle)\simeq F(\langle 0\rangle)\simeq 0$. In a similar way, we can analyse $\cat{Fin}_{\leqslant 1}^\circ\times_{\cat{Fin}_{\leqslant 1}}(\cat{Fin}_{\leqslant 1})_{/\langle 1\rangle}$. This category is a disjoint union $\Tt_0\sqcup \Tt_1$ of two components: $\Tt_1$ is simply $\{\id_{\langle 1\rangle}\colon \langle 1\rangle\rightarrow \langle 1\rangle\}$. On the other hand, $\Tt_0$ is a category with two objects, namely the nowhere defined maps $(\langle 1\rangle \rightarrow \langle 1\rangle)$ and $(\langle 0\rangle \rightarrow \langle 1\rangle)$, as well as precisely one non-identity morphism $(\langle 1\rangle \rightarrow \langle 1\rangle)\rightarrow (\langle 0\rangle \rightarrow \langle 1\rangle)$. In particular, $(\langle 0\rangle\rightarrow \langle 1\rangle)$ is terminal in $\Tt_0$. Hence the colimit describing $\Lan_jF(\langle 1\rangle)$ exists and is given by $F(\langle 1\rangle)\oplus F(\langle 0\rangle)\simeq F(\langle 1\rangle)\oplus 0\simeq F(\langle 1\rangle)$.
	
	In summary, \cref{lem:KanExtensionFormula} shows that $\Lan_jF$ exists and satisfies $\Lan_jF(\langle 0\rangle)\simeq 0$ and so we get an adjunction 
	\begin{equation*}
		\Lan_j\colon \Fun_*\left(\cat{Fin}_{\leqslant 1}^\circ,\Cc\right)\doublelrmorphism  \Fun_*\left(\cat{Fin}_{\leqslant 1},\Cc\right)\noloc j^*\,.
	\end{equation*}
	It follows from our calculations above that for all functors $G\in\Fun_*(\cat{Fin}_{\leqslant 1},\Cc)$ the counit $c_G\colon \Lan_j(G\circ j)\Rightarrow G$ is a pointwise equivalence and thus an equivalence by \cref{thm:EquivalencePointwise}. By \cref{lem:FullyFaithfulConservativeAdjunction}\cref{enum:FullyFaithfulIffUnitEquivalence}, this implies that $\Lan_j$ is fully faithful, even though $j$ itself is not. Furthermore, since $j$ is essentially surjective, it's clear that $j^*$ must be conservative. Hence $\Lan_j$ and $j^*$ are inverse equivalences by \cref{lem:FullyFaithfulConservativeAdjunction}\cref{enum:Conservative} and we've finished the proof of \cref{claim:LeftKanEasier}. 
	
	To prove \cref{claim:SegalConditionLeftKan}, first observe that by \cref{claim:LeftKanEasier} we can replace $i$ by $i\circ j$. Then we use \cref{lem:KanExtensionFormula} once again to compute the values of $
	\Lan_{i\circ j}(F\circ i\circ j)$. To this end, one analyses the category $\cat{Fin}_{\leqslant 1}^\circ\times_{\cat{Fin}}\cat{Fin}_{/\langle n\rangle}$: This category is a disjoint union $\Tt_0\sqcup \Tt_1\sqcup\dotsb\sqcup \Tt_n$, where $\Tt_0$ is as above and $\Tt_i$ is given by $\{s_i\colon \langle 1\rangle \rightarrow\langle n\rangle\}$, where $s_i(1)\coloneqq i$. Hence the colimit describing $\Lan_{i\circ j}(F\circ i\circ j)(\langle n\rangle)$ evaluates to $F(\langle 0\rangle)\oplus F(\langle 1\rangle)\oplus\dotsb\oplus F(\langle 1\rangle)\simeq 0\oplus F(\langle 1\rangle)^{\oplus n}\simeq F(\langle 1\rangle)^{\oplus n}$. This shows that $F$ satisfies the Segal condition if and only if $c_F\colon \Lan_{i\circ j}(F\circ i\circ j)\Rightarrow F$ is an equivalence of functors and thus proves \cref{claim:SegalConditionLeftKan}.
	
	%		at $\langle n\rangle$: It is given as a colimit with indexing $\infty$-category $(\cat{Fin}_{\leqslant 1})_{/\langle n\rangle}$, which is an ordinary category. An easy manipulation, which involves identifying a cofinal subcategory, shows that the colimit is given by the $n$-fold pushout $F(\langle 1\rangle)\sqcup_{F(\langle 0\rangle)}\dotsb \sqcup_{F(\langle 0\rangle)}F(\langle 1\rangle)$. Since $F(\langle 0\rangle)\simeq 0$, this pushout agrees with $F(\langle 1\rangle)^{\oplus n}$. But coproducts in $\Cc$ are the same as products by \cref{def:Additive}\cref{enum:SemiAdditive}, and so \cref{claim:SegalConditionLeftKan} follows. For more details see  \cite[Proposition~\href{https://florianadler.github.io/AlgebraBonn/KTheory.pdf\#dummy.2.16}{II.16}]{KTheory}.
	%		
	%		Let $\Fun_*\subseteq \Fun$ denote those functors that send $\langle 0\rangle\mapsto 0$. Then sending $F\mapsto F(\langle 1\rangle)$ induces an equivalence
	%		\begin{equation*}
		%			\ev_{\langle 1\rangle}\colon \Fun_*\left(\cat{Fin}_{\leqslant 1},\Cc\right)\overset{\simeq}{\longrightarrow}\Cc
		%		\end{equation*}
	%		of $\infty$-categories. Indeed, using that $0$ is both initial and terminal, it's straightforward to see that $\ev_{\langle 1\rangle}$ is essentially surjective. To see that $\ev_{\langle 1\rangle}$ is fully faithful, one uses the formula from \cref{cor:HomInFunctorCats}, which is completely explicit in this case since $\cat{Fin}_{\leqslant1}$ is an ordinary category and so we understand its twisted arrow category $\TwAr(\cat{Fin}_{\leqslant 1})$ via \cref{con:HomTwAr}.
	
	To finish the proof, observe that since $i\colon \cat{Fin}_{\leqslant 1}\rightarrow \cat{Fin}$ is fully faithful, the left Kan extension functor $\Lan_{i}\colon \Fun(\cat{Fin}_{\leqslant 1},\Cc)\rightarrow \Fun(\cat{Fin},\Cc)$ must be fully faithful too by \cref{cor:KanExtensionAlongFullyFaithful}. So
	\begin{equation*}
		\Cc\simeq \Fun_*\left(\cat{Fin}_{\leqslant 1},\Cc\right)\rightarrow \Fun\left(\cat{Fin}_{\leqslant 1},\Cc\right)\xrightarrow{\Lan_i}\Fun\left(\cat{Fin},\Cc\right)
	\end{equation*}
	is fully faithful, and its essential image is $\cat{CMon}(\Cc)$ by \cref{claim:SegalConditionLeftKan}. It follows that $\cat{CMon}(\Cc)\simeq \Cc$. Replacing $\cat{Fin}$ by $\IDelta^\op$ everywhere, the same argument shows $\cat{Mon}(\Cc)\simeq \Cc$. Finally, if $\Cc$ is additive, then \cref{def:Additive}\cref{enum:Additive} shows that the $\IE_\infty$-monoid in $\Cc$ associated to $x\in\Cc$ is automatically an $\IE_\infty$-group, so that $\cat{CGrp}(\Cc)\simeq \cat{CMon}(\Cc)$ and $\cat{Grp}(\Cc)\simeq \cat{Mon}(\Cc)$.
\end{proof}
\begin{lem}\label{lem:CGrpAdditive}
	If $\Cc$ is any $\infty$-category with finite products, then $\cat{CMon}(\Cc)$ is semi-additive and $\cat{CGrp}(\Cc)$ is an additive $\infty$-category.
\end{lem}
For the proof we need a criterion to decide when an $\infty$-category is semi-additive. %This is a version of \cite[Proposition~\HAthm{2.4.3.19}]{HA}.
\begin{lem}\label{lem:SemiAddCriterion}
	Let $\Cc$ be an $\infty$-category with finite products. Then $\Cc$ is semi-additive if the following two conditions are satisfied:
	\begin{alphanumerate}
		\item The terminal object $*\in \Cc$ is also initial.\label{enum:InitialTerminal}
		\item Let $\Delta\colon \Cc\rightarrow \Cc$ be the functor that sends $x\mapsto x\times x$. Then there exists a natural transformation $\mu\colon \Delta\Rightarrow\id_\Cc$ such that both compositions\label{enum:Multiplication}
		\begin{align*}
			x&\simeq x\times *\xrightarrow{{\id_x}\times 0} x\times x\overset{\mu_x}{\longrightarrow}x\\
			x&\simeq *\times x\xrightarrow{0\times{\id_x}} x\times x\overset{\mu_x}{\longrightarrow}x
		\end{align*}
		are homotopic to $\id_x$ for all $x\in \Cc$, and the following diagram commutes for all $x,y\in\Cc$:
		\begin{equation*}
			\begin{tikzcd}[column sep=0.9em]
				(x\times x)\times (y\times y)\drar["\mu_x\times \mu_y"']\ar[rr,"\id_x\times\mathrm{flip}\times \id_y"] & {}\dar[commutes,pos=0.45] & (x\times y)\times (x\times y)\dlar["\mu_{x\times y}"]\\
				& x\times y & 
			\end{tikzcd}
		\end{equation*}
	\end{alphanumerate}
	All conditions on $\mu$ in \cref{enum:Multiplication} are pointwise; so for example, we don't need to assume that the diagram above commutes functorially in $x$ and $y$.\hfill$\blacksquare$
\end{lem}
The proof of \cref{lem:SemiAddCriterion} is rather straightforward: One proves that the morphisms $x\simeq x\times *\rightarrow x\times y$ and $y\simeq *\times y\rightarrow x\times y$ exhibit $x\times y$ as a coproduct of $x$ and $y$. However, the details become rather tedious, and so we skip the proof. You can find a full argument in \cite[Lemma~\href{https://florianadler.github.io/AlgebraBonn/KTheory.pdf\#dummy.2.20}{II.20}]{KTheory} and another variant in \cite[Proposition~\HAthm{2.4.3.19}]{HA}.
\begin{proof}[Proof sketch of \cref{lem:CGrpAdditive}]
	We use the criterion from \cref{lem:SemiAddCriterion} to check that $\cat{CMon}(\Cc)$ is semi-additive. First observe that if $*\in\Cc$ is a terminal object, then $\const *$ is terminal in $\cat{CMon}(\Cc)$ (even in $\Fun(\cat{Fin},\Cc)$ by \cref{lem:ColimitsInFunctorCategories}). But it is also initial. Indeed, if $M\colon \cat{Fin}\rightarrow\Cc$ is any functor, then $\Hom_{\cat{Fun}(\cat{Fin},\Cc)}(\const *,M)\simeq \Hom_\Cc(*,\limit_{\langle n\rangle\in\cat{Fin}} M_n)$. However, $\langle 0\rangle \in\cat{Fin}$ is an initial object and so $\limit_{\langle n\rangle\in\cat{Fin}} M_n\simeq M_0$; in particular, the limit always exists. Now if $M\in \cat{CMon}(\Cc)$, then $M_0\simeq *$. Thus
	\begin{equation*}
		\Hom_{\cat{CMon}(\Cc)}(\const *,M)\simeq \Hom_\Cc(*,*)\simeq *\,,
	\end{equation*}
	as desired. So \cref{lem:SemiAddCriterion}\cref{enum:InitialTerminal} is satisfied.\footnote{The same argument works for $\cat{Mon}(\Cc)$, since $[0]\in \IDelta^\op$ is initial too. So $\cat{Mon}(\Cc)$ also satisfies \cref{lem:SemiAddCriterion}\cref{enum:InitialTerminal}.}
	
	To construct $\mu$, consider the functor $\times\colon  \cat{Fin}\times \cat{Fin}\rightarrow \cat{Fin}$ sending a pair $(\langle m\rangle,\langle n\rangle)$ to the product $\langle m\rangle \times\langle n\rangle \coloneqq \langle mn\rangle$.%
	%
	\footnote{Here we crucially use that we're working with $\cat{Fin}$; for $\IDelta^\op$, such a functor wouldn't exist. Thus, $\cat{Mon}(\Cc)$ doesn't satisfy \cref{lem:SemiAddCriterion}\cref{enum:Multiplication}.}
	%
	Precomposition with $\times$ then induces a functor $\Fun\left(\cat{Fin},\Cc\right)\rightarrow \Fun\left(\cat{Fin}\times\cat{Fin},\Cc\right)\simeq \Fun\left(\cat{Fin},\Fun(\cat{Fin},\Cc)\right)$. It's straightforward to check that the Segal condition is preserved, and so we obtain a functor
	\begin{equation*}
		\operatorname{Double}\colon \cat{CMon}(\Cc)\longrightarrow\cat{CMon}\left(\cat{CMon}(\Cc)\right)\,.
	\end{equation*}
	Unravelling the definitions, we find that
	\begin{align*}
		\operatorname{Double}(-)_1\colon \cat{CMon}(\Cc) &\longrightarrow \cat{CMon}\left(\cat{CMon}(\Cc)\right)\xrightarrow{\ev_{\langle 1\rangle}}\cat{CMon}(\Cc)\\
		\operatorname{Double}(-)_2\colon \cat{CMon}(\Cc) &\longrightarrow \cat{CMon}\left(\cat{CMon}(\Cc)\right)\xrightarrow{\ev_{\langle 2\rangle}} \cat{CMon}(\Cc)
	\end{align*}
	are equivalent to $\id_{\cat{CMon}(\Cc)}$ and $\Delta$, respectively. The everywhere defined map $f_2\colon \langle 2\rangle\rightarrow \langle 1\rangle$ from \cref{con:EinftyMultiplication} induces a natural transformation $\ev_{\langle 2\rangle}\Rightarrow \ev_{\langle 1\rangle}$, which yields a natural transformation $\mu\colon \Delta\Rightarrow \id_{\cat{CMon}(\Cc)}$ in $\Fun(\cat{CMon}(\Cc),\cat{CMon}(\Cc))$, as desired. It's straightforward to check that $\mu$ satisfies the conditions from \cref{lem:SemiAddCriterion}\cref{enum:Multiplication}. This finishes the proof that $\cat{CMon}(\Cc)$ is semi-additive.
	
	Since $\cat{CGrp}(\Cc)\subseteq \cat{CMon}(\Cc)$ is closed under products, it follows that $\cat{CGrp}(\Cc)$ must be semi-additive too. But then every $G\in\cat{CGrp}(\Cc)$ also satisfies the condition from \cref{def:Additive}\cref{enum:Additive}, by definition of $G$ being an $\IE_\infty$-group. Hence $\cat{CGrp}(\Cc)$ is additive.
\end{proof} 
Now we're ready to approach the desired analogue of \cref{thm:E1Loop}\cref{enum:E1LoopGrp}.
\begin{con}\label{con:BOmegaAdjunction}
	By \cref{cor:FunctorCategoryAdjunctions}, the adjunction $\B\colon \cat{Mon}(\cat{An})\shortdoublelrmorphism \cat{An}_{*/}\noloc \Omega$ from \cref{thm:E1Loop}\cref{enum:E1LoopGrp} induces an adjunction $\B_*\colon \Fun(\cat{Fin},\cat{Mon}(\cat{An}))\shortdoublelrmorphism \Fun(\cat{Fin},\cat{An}_{*/})\noloc \Omega_*$. We claim that this restricts to another adjunction
	\begin{equation*}
		\B\colon \cat{CMon}\left(\cat{Mon}(\cat{An})\right)\doublelrmorphism \cat{CMon}\bigl(\cat{An}_{*/}\bigr)\noloc \Omega\,.
	\end{equation*}
	To see this, we must show that the Segal condition is preserved under $\B_*$ and $\Omega_*$. This in turn reduces to checking that $\B\colon \cat{Mon}(\cat{An})\rightarrow \cat{An}_{*/}$ and $\Omega\colon \cat{An}_{*/}\rightarrow\cat{Mon}(\cat{An})$ preserve finite products. For $\Omega$, this is obvious, since right adjoints preserve all limits. For $\B$, this follows from \cref{lem:BPreservesProducts} below (plus \cref{lem:ColimitsInSliceCategory}\cref{enum:LimitsInSlice}).
	
	Now the currying equivalence $\Fun(\cat{Fin},\Fun(\IDelta^\op,\cat{An}))\simeq \Fun(\IDelta^\op,\Fun(\cat{Fin},\cat{An}))$ restricts to an equivalence $\cat{CMon}(\cat{Mon}(\cat{An}))\simeq \cat{Mon}(\cat{CMon}(\cat{An}))$ by a straightforward check. Furthermore, \cref{lem:CGrpIsC,lem:CGrpAdditive} show $\cat{Mon}(\cat{CMon}(\cat{An}))\simeq \cat{CMon}(\cat{An})$. In a similar way, the equivalence $\Fun(\cat{Fin},\cat{An}_{*/})\simeq \Fun(\cat{Fin},\cat{An})_{\const */}$ restricts to $\cat{CMon}(\cat{An}_{*/})\simeq \cat{CMon}(\cat{An})_{\const */}$. But $\const *\in \cat{CMon}(\cat{An})$ is an initial object, as we've seen in the proof of \cref{lem:CGrpAdditive}. Thus $\cat{CMon}(\cat{An})_{\const */}\simeq \cat{CMon}(\cat{An})$. Putting everything together, we can rewrite the adjunction above as
	\begin{equation*}
		\B\colon \cat{CMon}(\cat{An})\doublelrmorphism \cat{CMon}(\cat{An})\noloc \Omega\,.
	\end{equation*}
\end{con}
\begin{lem}\label{lem:BPreservesProducts}
	The functor $\colimit_{\IDelta^\op}\colon \Fun(\IDelta^\op,\cat{An})\rightarrow \cat{An}$ preserves finite products.
\end{lem}
\begin{proof}[Proof sketch]
	The crucial step is to show that the diagonal $\IDelta^\op\rightarrow \IDelta^\op\times\IDelta^\op$ is cofinal. This is another application of \cref{thm:JoyalsQuillenA}\cref{enum:WeaklyContractible}, of course, but it's not completely obvious and we leave it as a not quite easy exercise. For a full proof, consult \cite[Tag~\href{https://kerodon.net/tag/02QP}{02QP}]{Kerodon} or \cite[Exercise~\href{https://florianadler.github.io/AlgebraBonn/KTheory.pdf\#smallerdummy.2.18.1}{II.18$a$}]{KTheory}.
	
	It will be enough to show that $\colimit_{\IDelta^\op}$ preserves empty products and binary products. First note that $\colimit_{[n]\in \IDelta^\op}*\simeq *$ follows from \cref{lem:ContractibleColimit}, since  $\left|\IDelta^\op\right|\simeq *$ (which follows, for example, from the fact that $[0]\in\IDelta^\op$ is an initial object). This shows that $\colimit_{\IDelta^\op}$ preserves empty products. For binary products, let $X,Y\colon \IDelta^\op\rightarrow\cat{An}$ be functors. Since $\IDelta^\op\rightarrow\IDelta^\op\times\IDelta^\op$ is cofinal, we can rewrite $\colimit_{[n]\in\IDelta^\op}\left(X_n\times Y_n\right)$ as
	\begin{equation*}
		\colimit_{([m],[n])\in \IDelta^\op\times \IDelta^\op}\left(X_m\times Y_n\right)\simeq \colimit_{[m]\in\IDelta^\op}\biggl(X_m\times \colimit_{[n]\in\IDelta^\op}Y_n\biggr)
		\simeq \biggl(\colimit_{[m]\in\IDelta^\op}X_m\biggr)\times \biggl(\colimit_{[n]\in\IDelta^\op}Y_n\biggr)\,.
	\end{equation*}
	The first equivalence follows from \cref{lem:ColimitManipulations} together with the fact that $X_m\times -\colon \cat{An}\rightarrow\cat{An}$ commutes with arbitrary colimits, because it is a left adjoint by \cref{exm:Adjunctions}\cref{enum:Currying}. Applying the same argument to $-\times \colimit_{\IDelta^\op}Y_m$ gives the third equivalence.
\end{proof}
\begin{thm}\label{thm:EinftyBOmegaEquivalence}
	The adjunctions from \cref{con:BOmegaAdjunction} and \cref{thm:E1Loop}\cref{enum:E1LoopGrp} fit into a commutative diagram
	\begin{equation}\label{eq:BOmegaAdjunction}\tag{$*$}
		\begin{tikzcd}
			\cat{CMon}(\cat{An})\rar[shift left=0.2em,"\B"]\dar["\operatorname{Cut}^*"']\drar[commutes] & \lar[shift left=0.2em,"\Omega"] \cat{CMon}(\cat{An})\dar["\ev_{\langle 1\rangle}"]\\
			\cat{Mon}(\cat{An}) \rar[shift left=0.2em,"\B"] & \lar[shift left=0.2em,"\Omega"]\cat{An}_{*/}
		\end{tikzcd}
	\end{equation}
	\embrace{note that $\ev_{\langle 1\rangle}\colon \cat{CMon}(\cat{An})\rightarrow \cat{An}$ factors canonically over $\cat{An}_{*/}\rightarrow \cat{An}$ since we have $\cat{CMon}(\cat{An})\simeq \cat{CMon}(\cat{An}_{*/})$ by \cref{con:BOmegaAdjunction}}. Furthermore:
	\begin{alphanumerate}
		\item Both $\B\colon \cat{CMon}(\cat{An})\rightarrow \cat{CMon}(\cat{An})$ and $\Omega\colon \cat{CMon}(\cat{An})\rightarrow \cat{CMon}(\cat{An})$ factor through the full sub-$\infty$-category $\cat{CGrp}(\cat{An})\subseteq \cat{CMon}(\cat{An})$ and they induce inverse equivalences\label{enum:BOmegaEquivalence}
		\begin{equation*}
			\B\colon \cat{CGrp}(\cat{An}) \underset{\simeq}{\mathrel{\smash{\underset{\smash{\raisebox{0.35em}{$\longleftarrow$}}}{\overset{\smash{\raisebox{-0.35em}{$\overset{\simeq}{\longrightarrow}$}}}{\phantom{\longrightarrow}}}}}} \cat{CGrp}(\cat{An})_{\geqslant 1}\noloc \Omega\,.
		\end{equation*}
		Here $\cat{CGrp}(\cat{An})_{\geqslant 1}\subseteq \cat{CGrp}(\cat{An})$ is the full sub-$\infty$-category spanned by those $\IE_\infty$-groups $G$ for which $\pi_0(G)\cong *$.
		\item The inclusion $\cat{CGrp}(\cat{An})\subseteq\cat{CMon}(\cat{An})$ has a left adjoint, namely $\Omega \B$. So $\Omega\B$ is not only the \enquote{group completion} for $\IE_1$-monoids \embrace{see \cref{cor:GroupCompletion}}, but for $\IE_\infty$-monoids too. \label{enum:EinftyGroupCompletion}
	\end{alphanumerate}
\end{thm}
\begin{proof}[Proof sketch]
	Commutativity of \cref{eq:BOmegaAdjunction} is a straightforward unravelling of definitions. Let's proceed with \cref{enum:BOmegaEquivalence}. Let $M\in \cat{CMon}(\cat{An})$. To show that $\B$ factors through $\cat{CGrp}(\cat{An})\subseteq \cat{CMon}(\cat{An})$, simply observe $\pi_0(\B M)\cong *$. This ordinary monoid is a group and so the underlying $\IE_1$-monoid of $\B M$ must be an $\IE_1$-group by \cref{lem:E1Groups}\cref{enum:MGroupOnPi0}. To show that $\Omega$ factors through $\cat{CGrp}(\cat{An})\subseteq \cat{CMon}(\cat{An})$, we must show that the underlying $\IE_1$-monoid $\operatorname{Cut}^*(\Omega M)$ of $\Omega M$ is an $\IE_1$-group. But commutativity of \cref{eq:BOmegaAdjunction} shows $\operatorname{Cut}^*(\B M)\simeq \Omega M_1$  $\operatorname{Cut}^*(\Omega M)\simeq \Omega M_1$ and $\Omega\colon \cat{An}_{*/}\rightarrow \cat{Mon}(\cat{An})$ factors through $\operatorname{Grp}(\cat{An})\subseteq \cat{Mon}(\cat{An})$, as we've seen in the proof of \cref{thm:E1Loop}\cref{enum:E1LoopGrp}.
	
	It remains to show that $\B$ and $\Omega$ induce inverse equivalences $\cat{CGrp}(\cat{An})\simeq \cat{CGrp}(\cat{An})_{\geqslant 1}$. We've already seen that the functor $\B\colon \cat{CGrp}(\cat{An})\rightarrow \cat{CMon}(\cat{An})$ factors through the inclusion $\cat{CGrp}(\cat{An})_{\geqslant 1}\subseteq \cat{CMon}(\cat{An})$, so at least we get an adjunction
	\begin{equation*}
		\B\colon \cat{CGrp}(\cat{An}) \doublelrmorphism \cat{CGrp}(\cat{An})_{\geqslant 1}\noloc \Omega\,.
	\end{equation*}
	We use the criterion from \cref{lem:FullyFaithfulConservativeAdjunction}\cref{enum:Conservative}. Observe that equivalences of $\IE_\infty$-monoids can be checked on underlying animae by the same argument as in \cref{lem:E1MonoidsEquivalenceOnUnderlyingObjects}. Thus, the questions whether the unit $u\colon \id_{\cat{CGrp}(\cat{An})}\Rightarrow \Omega \B$ is an equivalence and whether $\Omega$ is conservative can be reduced to the analogous questions for the adjunction $\B\colon \cat{Grp}(\cat{An})\shortdoublelrmorphism (\cat{An}_{*/})_{\geqslant 1}\noloc \Omega$. But \cref{thm:E1Loop}\cref{enum:E1LoopGrp} says that this adjunction is a pair of inverse equivalences. This finishes the proof of \cref{enum:BOmegaEquivalence}. Part~\cref{enum:EinftyGroupCompletion} is a formal consequence of \cref{enum:BOmegaEquivalence}. 
	%		
	%		For \cref{enum:EinftyGroupCompletion}, we combine the observations from \cref{enum:BOmegaEquivalence} to see that we have a chain of adjunctions
	%		\begin{equation*}
		%			\begin{tikzcd}[cramped,column sep=\the\longarrowlength,shorten=0.18ex]
			%				\cat{CMon}(\cat{An})\rar[shift left=0.2em,"\B"] & \lar[shift left=0.2em,"\Omega"] \cat{CGrp}(\cat{An})_{\geqslant 1}\rar[shift left=0.2em,"\Omega"] & \lar[shift left=0.2em,"\B"]\cat{CGrp}(\cat{An})
			%			\end{tikzcd}
		%		\end{equation*}
	%		(for the adjunction on the right, note that $\Omega\colon \cat{CGrp}(\cat{An})_{\geqslant 1}\rightarrow \cat{CGrp}(\cat{An})_{\geqslant 1}$ is also a left adjoint of $\B\colon \cat{CGrp}(\cat{An})\rightarrow \cat{CGrp}(\cat{An})_{\geqslant 1}$, since these functors are inverse equivalences). It follows that $\Omega \B\colon \cat{CMon}(\cat{An})\rightarrow \cat{CGrp}(\cat{An})$ is a left adjoint of $\Omega\B\colon \cat{CGrp}(\cat{An})\rightarrow \cat{CMon}(\cat{An})$. But the latter functor is equivalent to the inclusion $\cat{CGrp}(\cat{An})\rightarrow \cat{CMon}(\cat{An})$, since $u\colon \id_{\cat{CGrp}(\cat{An})}\Rightarrow \Omega \B$ is an equivalence, as we've seen in \cref{enum:BOmegaEquivalence}. This finishes the proof of \cref{enum:EinftyGroupCompletion}.
\end{proof}



\subsection{Spectra and stable \texorpdfstring{$\infty$}{Infinity}-categories}\label{subsec:Spectra}
We've seen in \cref{thm:E1Loop}\cref{enum:E1LoopGrp} that $\IE_1$-groups in $\cat{An}$ are essentially the same as loop animae. Furthermore, we've seen in \cref{cor:FreeE1Group} that $\Omega\Sigma X_+$ is the free $\IE_1$-group on an anima $X$. Of course, these observations should have analogues for $\IE_\infty$-groups, but it's not immediately clear how such analogues would look like, nor how they would follow from \cref{thm:EinftyBOmegaEquivalence}.  In this subsection, we'll introduce the $\infty$-category of \emph{spectra}, which will eventually lead us to answers for both questions (\cref{rem:InfiniteLoopAnimae} and \cref{cor:FreeEInftyGroup}), but also to many more applications.
\begin{con}\label{con:Binfty}
	We've seen in \cref{thm:EinftyBOmegaEquivalence}\cref{enum:BOmegaEquivalence} that $\Omega\B\colon \cat{CGrp}(\cat{An})\rightarrow \cat{CGrp}(\cat{An})$ is homotopic to the identity. Therefore, the following diagram commutes in $\cat{Cat}_\infty$ (or really, in $\widehat{\cat{Cat}}_\infty$, since we're dealing with large $\infty$-categories, but we'll ignore this issue here):
	\begin{equation*}
		\begin{tikzcd}
			\dotsb\eqar[r]& \cat{CGrp}(\cat{An})\eqar[r]\drar[commutes]\dar["\B\circ \B"] & \cat{CGrp}(\cat{An})\eqar[r]\drar[commutes]\dar["\B"] & \cat{CGrp}(\cat{An})\eqar[d]\\
			\dotsb\rar["\Omega"] & \cat{CGrp}(\cat{An})\rar["\Omega"] & \cat{CGrp}(\cat{An})\rar["\Omega"] & \cat{CGrp}(\cat{An})
		\end{tikzcd}
	\end{equation*}
	This diagram yields a functor%
	%
	\footnote{Here we would like to point out a subtlety that only the extraordinarily careful reader will have noticed: Let $\IN$ denote the partially ordered set $(\dotsb\rightarrow 2\rightarrow 1\rightarrow0)$. Then to construct a functor $\IN\rightarrow \Dd$ into an arbitrary $\infty$-category $\Dd$, it's enough to specify objects $y_n\in\Dd$ together with morphisms $y_{n+1}\rightarrow y_{n}$ for all $n\in\IN$. This is because the inclusion $\operatorname{sk}_1\N(\IN)\rightarrow \N(\IN)$ of the $1$-skeleton of the nerve of $\IN$ is inner anodyne, so that $\F(\N(\IN),\Dd)\rightarrow\F(\operatorname{sk}_1\N(\IN),\Dd)$ is a trivial fibration (and thus an equivalence of quasi-categories) by \cref{cor:FKanFibration}. In the situation above, we implicitly used this observation in the case $\Dd\simeq \Ar(\cat{Cat}_\infty)$ to turn the commutative diagram, which a priori only encodes a sequence of morphisms
	\begin{equation*}
		\bigl(\B^{\circ(n+1)}\colon \cat{CGrp}(\cat{An}\bigr)\rightarrow \Cc_{*/})\longrightarrow \bigl(\B^{\circ n}\colon \cat{CGrp}(\cat{An})\rightarrow \Cc_{*/}\bigr)
	\end{equation*}
	in $\Ar(\cat{Cat}_\infty)$, into a functor $\IN\rightarrow \Ar(\cat{Cat}_\infty)$, which by currying encodes a natural transformation in $\Fun(\IN,\cat{Cat}_\infty)$ and thus a functor $\B^\infty$ by the universal property of limits. Also note that this subtle observation was implicitly used to even write down the limit above: We can't just take the limit of a sequence of morphisms, we must turn that sequence into a functor $\IN\rightarrow \cat{Cat}_\infty$!}
	%
	\begin{equation*}
		\B^\infty\colon \cat{CGrp}(\cat{An})\longrightarrow \limit\left(\dotsb\overset{\Omega}{\longrightarrow} \cat{CGrp}(\cat{An})\overset{\Omega}{\longrightarrow} \cat{CGrp}(\cat{An})\overset{\Omega}{\longrightarrow} \cat{CGrp}(\cat{An})\right)\,.
	\end{equation*}
	Observe that for all $n\geqslant 0$, $\B$ and $\Omega$ induce inverse equivalences $\cat{CGrp}(\cat{An})_{\geqslant n+1}\simeq \cat{CGrp}(\cat{An})_{\geqslant n}$, where $\cat{CGrp}(\cat{An})_{\geqslant n}\subseteq \cat{CGrp}(\cat{An})$ is the full sub-$\infty$-category spanned by those $\IE_\infty$-groups $G$ that satisfy $\pi_i(G)\cong0$ for all $0\leqslant i<n$. Indeed, the case $n=0$ follows from \cref{thm:EinftyBOmegaEquivalence}\cref{enum:BOmegaEquivalence}. Since $\Omega$ \enquote{shifts homotopy groups down by one} (see \cref{lem:SuspensionLoopAdjunction}\cref{enum:LoopShiftsHomotopyGroups}), its inverse $\B$ must \enquote{shift homotopy groups up by one}. This implies that $\B\colon \cat{CGrp}(\cat{An})\rightarrow \cat{CGrp}(\cat{An})_{\geqslant 1}$ must map $\cat{CGrp}(\cat{An})_{\geqslant n}$ into $\cat{CGrp}(\cat{An})_{\geqslant n+1}$; similarly, $\Omega$ must map $\cat{CGrp}(\cat{An})_{\geqslant n+1}$ into $\cat{CGrp}(\cat{An})_{\geqslant n}$. Hence the equivalence from \cref{thm:EinftyBOmegaEquivalence}\cref{enum:BOmegaEquivalence} must restrict to an equivalence $\cat{CGrp}(\cat{An})_{\geqslant n+1}\simeq \cat{CGrp}(\cat{An})_{\geqslant n}$ for all $n\geqslant 0$, as claimed.
	
	Combining this observation with \cref{lem:HomInLimits}\cref{lem:HomInLimits} shows that $\B^\infty$ is fully faithful, with essential image given by
	\begin{equation*}
		\B^\infty\colon \cat{CGrp}(\cat{An})\overset{\simeq}{\longrightarrow}\limit\left(\dotsb\overset{\Omega}{\longrightarrow} \cat{CGrp}(\cat{An})_{\geqslant 2}\overset{\Omega}{\longrightarrow} \cat{CGrp}(\cat{An})_{\geqslant 1}\overset{\Omega}{\longrightarrow} \cat{CGrp}(\cat{An})\right)\,.
	\end{equation*}
	Let us now turn this construction into a definition.
\end{con}
\begin{defi}\label{def:Spectra}
	Let $\Cc$ be an $\infty$-category with finite limits (in the sense of \cref{def:KappaSmall}\cref{enum:KappaSmallLimit}); in particular, $\Cc$ has a terminal object $*\in\Cc$. The \emph{$\infty$-category of spectra in $\Cc$} is defined as the following limit in $\cat{Cat}_\infty$:
	\begin{equation*}
		\cat{Sp}(\Cc)\coloneqq \limit\left(\dotsb\overset{\Omega_\Cc}{\longrightarrow} \Cc_{*/}\overset{\Omega_\Cc}{\longrightarrow} \Cc_{*/}\overset{\Omega_\Cc}{\longrightarrow} \Cc_{*/}\right)\,.
	\end{equation*}
	Here $\Omega_\Cc\colon \Cc_{*/}\rightarrow \Cc_{*/}$ is defined by the same pullback diagram as in \cref{def:Loop}. In the case $\Cc\simeq \cat{An}$, we write $\cat{Sp}\coloneqq \cat{Sp}(\cat{An})$ for brevity, and we call $\cat{Sp}$ simply the \emph{$\infty$-category of spectra}.
\end{defi}
\begin{lem}\label{lem:SpCGrpIsSp}
	Let $\Cc$ be an $\infty$-category with finite limits. Then $\ev_{\langle 1\rangle}\colon \cat{CGrp}(\Cc)\rightarrow \Cc$ induces an equivalence $\cat{Sp}(\cat{CGrp}(\Cc))\simeq \cat{Sp}(\Cc)$. In particular, the first limit from \cref{con:Binfty} agrees with $\cat{Sp}$.
\end{lem}
\begin{proof}[Proof sketch]
	Let's address the \enquote{in particular} first. Since $0\coloneqq \const *\in\cat{CGrp}(\cat{An})$ is both initial and terminal by \cref{lem:CGrpAdditive}, we have $\cat{CGrp}(\cat{An})\simeq \cat{CGrp}(\cat{An})_{0/}$. Hence the first limit from \cref{con:Binfty} agrees with $\cat{Sp}(\cat{CGrp})(\cat{An})$ and thus with $\cat{Sp}$, as claimed.
	
	To show $\cat{Sp}(\cat{CGrp}(\Cc))\simeq \cat{Sp}(\Cc)$ in general, first observe that $\Fun(\cat{Fin},-)\colon \cat{Cat}_\infty\rightarrow \cat{Cat}_\infty$ commutes with limits, since it is a right adjoint by \cref{exm:Adjunctions}\cref{enum:Currying}. Hence we obtain $\Fun(\cat{Fin},\Cc_{*/})\simeq \Fun(\cat{Fin},\Cc)_{\const */}$ and  $\Fun(\cat{Fin},\cat{Sp}(\Cc))\simeq \cat{Sp}(\Fun(\cat{Fin},\Cc))$. It's straightforward to check that the latter equivalence restricts to $\cat{Sp}(\cat{CGrp}(\Cc))\simeq \cat{CGrp}(\cat{Sp}(\Cc))$. Thus, by \cref{lem:CGrpIsC}, it's enough to show that $\cat{Sp}(\Cc)$ is additive. We'll use \cref{lem:SemiAddCriterion}. Note that $*\in\Cc_{*/}$ is both initial and terminal. By \cref{lem:HomInLimits}\cref{enum:HomInLimits}, it follows that $0\coloneqq (\dotsc,*,*,*)\in \cat{Sp}(\Cc)$ is initial and terminal too. So the condition from \cref{lem:SemiAddCriterion}\cref{enum:InitialTerminal} is satisfied. To construct a natural transformation $\mu\colon \Delta\Rightarrow \id_{\cat{Sp}(\Cc)}$ as in \cref{lem:SemiAddCriterion}\cref{enum:Multiplication}, we observe:
	\begin{alphanumerate}\itshape
		\item[\boxtimes_1] $\Omega_\Cc$ induces an equivalence $\Omega_\Cc\colon \cat{Sp}(\Cc)\overset{\simeq}{\longrightarrow}\cat{Sp}(\Cc)$.\label{claim:OmegaEquivalence}
		\item[\boxtimes_2] $\Omega_\Cc$ can be factored into $\Omega_\Cc\colon\cat{Sp}(\Cc)\rightarrow \cat{Grp}(\cat{Sp}(\Cc))\xrightarrow{\ev_{[1]}}\cat{Sp}(\Cc)$. More generally, the same is true if $\cat{Sp}(\Cc)$ is replaced by any $\infty$-category $\Dd$ with finite limits, whose terminal object $*\in\Dd$ is also initial.\label{claim:OmegaGrp}
	\end{alphanumerate}
	Observation~\cref{claim:OmegaEquivalence} is clear from \cref{def:Spectra}. Observation~\cref{claim:OmegaGrp} can be shown by hand. Alternatively, first observe \cref{claim:OmegaGrp} is true in the case $\Dd\simeq \cat{An}_{*/}$ by \cref{thm:E1Loop}\cref{enum:E1LoopGrp}, using that $\cat{Grp}(\cat{An}_{*/})\simeq \cat{Grp}(\cat{An})_{*/}\simeq \cat{Grp}(\cat{An})$ since $*\in\cat{Grp}(\cat{An})$ is both initial and terminal by the arguments from the proof of \cref{lem:CGrpAdditive}. The general case can be reduced to this special case using the Yoneda embedding $\Yo_\Dd\colon \Dd\rightarrow \Fun(\Dd^\op,\cat{An})$, which preserves all limits by \cref{cor:HomPreservesLimits} and thus factors through $\Fun(\Dd^\op,\cat{An})_{\const */}\simeq \Fun(\Dd^\op,\cat{An}_{*/})$. A full argument can be found in \cite[Remark*~\href{https://florianadler.github.io/AlgebraBonn/KTheory.pdf\#smallerdummy.2.23.1}{II.23$a$}]{KTheory}.
	
	Now \cref{claim:OmegaEquivalence} and \cref{claim:OmegaGrp} imply that every $X\in\cat{Sp}(\Cc)$ can be written as $X\simeq\Omega_\Cc(\Omega_\Cc^{-1}(X))$, and so $X$ can be functorially upgraded to an $\IE_1$-group $X\in \cat{Grp}(\cat{Sp}(\Cc))$. We then define $\mu\colon X\times X\rightarrow X$ to be the multiplication on $X$. All conditions from \cref{lem:SemiAddCriterion}\cref{enum:InitialTerminal} are easily verified.
\end{proof}
\begin{con}\label{con:HomotopyGroupsOfSpectra}
	We regard $\IN$ and $\IZ$ as partially ordered sets and $\IN\subseteq \IZ$ as the inclusion $(\dotsb\rightarrow 2\rightarrow 1\rightarrow0)\subseteq (\dotsb\rightarrow 2\rightarrow 1 \rightarrow 0\rightarrow (-1)\rightarrow (-2)\rightarrow\dotsb)$. Then $\IN\rightarrow \IZ$ is a final functor of $\infty$-categories. Indeed, this is immediate from the dual of \cref{thm:JoyalsQuillenA}\cref{enum:WeaklyContractible}, or from the dual of \cref{exm:Cofinal}\cref{enum:RightAnodyneCofinal}. Hence we can rewrite $\cat{Sp}(\Cc)$ as
	\begin{equation*}
		\cat{Sp}(\Cc)\simeq \limit\left(\dotsb\overset{\Omega_\Cc}{\longrightarrow} \Cc_{*/}\overset{\Omega_\Cc}{\longrightarrow} \Cc_{*/}\overset{\Omega_\Cc}{\longrightarrow} \Cc_{*/}\overset{\Omega_\Cc}{\longrightarrow}\dotsb\right)\,.
	\end{equation*}
	For all $n\in\IZ$, we let $\Omega_\Cc^{\infty-n}\colon \cat{Sp}(\Cc)\rightarrow \Cc_{*/}$ denote the projection to the $n$\textsuperscript{th} component of the limit. This notation is chosen in such a way that $\Omega_\Cc(\Omega_\Cc^{\infty-n}X)\simeq \Omega_\Cc^{\infty-(n-1)}X$, as one would expect. In the case $\Cc\simeq \cat{An}$, we drop the index and just write $\Omega^{\infty-n}$.
	
	In the case $\Cc\simeq \cat{An}$, we define $\pi_n(X)\coloneqq \pi_0(\Omega^{\infty+n}X)$, the \emph{$n$\textsuperscript{th} homotopy group of the spectrum $X$}. Since $\Omega^{\infty+n}X\simeq \Omega^i(\Omega^{\infty+n-i}X)$ and $\Omega$ shifts homotopy groups down by \cref{lem:SuspensionLoopAdjunction}\cref{enum:LoopShiftsHomotopyGroups}, we see $\pi_n(X)\cong \pi_i(\Omega^{\infty+n-i}X)$ for all $i\geqslant 0$ (we don't have to specify a base point since $\Omega^{\infty+n-i}X\in\cat{An}_{*/}$ by construction). In particular, choosing $i\geqslant 2$ and using \cref{lem:HomotopyGroups}\cref{enum:EckmannHilton}, we see that $\pi_n(X)$ is an abelian group for all $n\in\IZ$.
	
	A spectrum $X$ is called \emph{connective} if $\pi_n(X)\cong 0$ for all $n<0$, and \emph{coconnective} if $\pi_n(X)\cong 0$ for all $n>0$. It's customary to denote by $\cat{Sp}_{\geqslant 0}\subseteq \cat{Sp}$ and $\cat{Sp}_{\leqslant 0}\subseteq \cat{Sp}$ the full sub-$\infty$-categories spanned by the connective and the coconnective spectra, respectively.
\end{con}
\begin{lem}\label{lem:SpCHasLimits}
	Let $\Cc$ be an $\infty$-category with finite limits. Then $\cat{Sp}(\Cc)$ has all finite limits and $\Omega_\Cc^{\infty-n}\colon \cat{Sp}(\Cc)\rightarrow \Cc_{*/}$ preserves all finite limits for all $n\in\IZ$. Furthermore, in the special case $\Cc\simeq \cat{An}$, the following is true:
	\begin{alphanumerate}
		\item The $\infty$-category $\cat{Sp}$ has all small limits and colimits. For all $n\in\IZ$, the functors $\Omega^{\infty-n}\colon \cat{Sp}\rightarrow \cat{An}_{*/}$ commute with all limits and with filtered colimits.\label{enum:SpHasAllColimits}
		\item For all $n\in\IZ$, the functors $\pi_n\colon \cat{Sp}\rightarrow \cat{Ab}$ commute with all products \embrace{in particular, with finite products and thus with finite coproducts too} and with filtered colimits.\label{enum:HomotopyGroupsOfSpectraCommuteWithFilteredColimits}
		\item A morphism $f\colon X\rightarrow Y$ of spectra is an equivalence if and only if it induces isomorphisms $\pi_n(X)\cong \pi_n(Y)$ for all $n\in\IZ$. Furthermore, if $X\rightarrow Y\rightarrow Z$ is a fibre sequence in $\cat{Sp}$ \embrace{in the sense of \cref{def:Cofibre}}, then there is a long exact sequence of abelian groups\label{enum:WhiteheadForSpectra}
		\begin{equation*}
			\dotsb\longrightarrow \pi_{n+1}(Z)\overset{\partial}{\longrightarrow} \pi_n(X)\longrightarrow \pi_n(Y)\longrightarrow \pi_n(Z)\overset{\partial}{\longrightarrow} \pi_{n-1}(X)\longrightarrow\dotsb\,.
		\end{equation*}
	\end{alphanumerate}
\end{lem}
\begin{proof}
	The first assertion is an immediate consequence of \cref{lem:HomInLimits}\cref{enum:ColimitsInLimits}. The same argument also proves that $\cat{Sp}$ has all limits and that $\Omega^{\infty-n}\colon \cat{Sp}\rightarrow\cat{An}_{*/}$ commutes with limits. To prove the existence of colimits in $\cat{Sp}$, it's enough to show that pushouts, finite coproducts, and filtered colimits exist, because infinite coproducts can be written as filtered colimits of finite coproducts (see claim~\cref{claim:FilteredCoproduct} in the proof of \cref{lem:KappaCompactlyGenerated}). Since $\Omega\colon \cat{An}_{*/}\rightarrow \cat{An}_{*/}$ preserves filtered colimits by \cref{lem:FilteredColimitsPreserveFiniteLimits}, we can apply \cref{lem:HomInLimits}\cref{enum:ColimitsInLimits} again to deduce that $\cat{Sp}$ has all filtered colimits and that $\Omega^{\infty-n}\colon \cat{Sp}\rightarrow\cat{An}_{*/}$ commutes with filtered colimits. The existence of finite coproducts follows from the fact that $\cat{Sp}$ is additive, as observed in \cref{lem:SpCGrpIsSp}. Finally, pushouts will be constructed in \cref{lem:Stable} below. This finishes the proof of \cref{enum:SpHasAllColimits}.
	
	Part~\cref{enum:HomotopyGroupsOfSpectraCommuteWithFilteredColimits} follows immediately from \cref{enum:SpHasAllColimits} and the fact that $\pi_0\colon \cat{An}\rightarrow \cat{Set}$ preserves all products and filtered colimits by \cref{lem:HomotopyGroupsFilteredColimits} (plus the fact that $\cat{Ab}\rightarrow\cat{Set}$ is conservative and preserves all products and filtered colimits).
	
	The long exact sequence from \cref{enum:WhiteheadForSpectra} follows immediately from \cref{lem:LongExactFibrationSequence} and the fact that $\Omega^{\infty-n}X\rightarrow \Omega^{\infty-n}Y\rightarrow \Omega^{\infty-n}Z$ is a fibre sequence in $\cat{An}_{*/}$ for all $n\in\IZ$ by \cref{enum:SpHasAllColimits}. It's clear that any equivalence $f\colon X\rightarrow Y$ induces isomorphisms $\pi_n(X)\cong \pi_n(Y)$ for all $n\in\IZ$. The converse follows essentially from \cref{thm:Whitehead}; the only non-obvious point is that \cref{thm:Whitehead} requires isomorphisms on homotopy groups \emph{for all basepoints}, whereas $\pi_n(X)\cong \pi_i(\Omega^{\infty+n-i}X)$ only uses the preferred base point of $\Omega^{\infty+n-i}X\in\cat{An}_{*/}$. However, $\Omega^{\infty+n-i}X$ upgrades canonically to an $\IE_\infty$-group in $\cat{An}$ by \cref{lem:SpCGrpIsSp}. In an $\IE_\infty$-group, all path components are homotopy equivalent, and so it doesn't matter which basepoint we use.
\end{proof}
\begin{rem}\label{rem:SpPresentable}
	Using the formalism from \cref{subsec:PrL}, we can give a slick proof of \cref{lem:SpCHasLimits}\cref{enum:SpHasAllColimits}: Suppose $\Cc$ is a presentable $\infty$-category. The loop functor $\Omega_\Cc\colon \Cc_{*/}\rightarrow \Cc_{*/}$ admits a left adjoint $\Sigma_\Cc\colon \Cc_{*/}\rightarrow \Cc_{*/}$ given by the same pushout diagram as in \cref{def:Loop}. Thus $\Omega_\Cc$ is a functor in $\cat{Pr}^\R$. We know from \cref{lem:PrLColimits} that $\cat{Pr}^\R\rightarrow \widehat{\cat{Cat}}_\infty$ preserves limits, and so the limit defining $\cat{Sp}(\Cc)$ can also be viewed as a limit in $\cat{Pr}^\R$.%
	%
	\footnote{Since we can construct $\cat{Pr}^\R$ in ZFC, see~\cref{par:PrLInZFC}, this also provides a way to construct $\cat{Sp}(\Cc)$ without enlarging our universe and talking about $\widehat{\cat{Cat}}_\infty$.}
	%
	This immediately shows that $\cat{Sp}(\Cc)$ is presentable, so in particular, it has all colimits. If $\Cc_{*/}$ is $\aleph_0$-compactly generated (which is true for $\Cc\simeq \cat{An}$, as  the pointed $0$-dimensional sphere $(S^0,*)$ is a compact generator of $\cat{An}_{*/}$; this is clear from \cref{lem:KappaCompactlyGenerated}\cref{enum:CompactGenerators}), the limit defining $\cat{Sp}(\Cc)$ can also be interpreted as a limit in $\cat{Pr}_{\aleph_0}^\R$, because $\cat{Pr}_{\aleph_0}^\R\rightarrow \cat{Pr}^\R$ also preserves all limits by the dual of \cref{cor:PrLKappaColimits}\cref{enum:PrLKappaColimits}. This shows that the projections $\Omega_\Cc^{\infty-n}\colon \cat{Sp}(\Cc)\rightarrow \Cc_{*/}$ preserve filtered colimits.
	
	We can take these considerations one step further: Recall that there's an equivalence of $\infty$-categories $\cat{Pr}^\L\simeq (\cat{Pr}^\R)^\op$ given by extracting adjoints. Thus, $\cat{Sp}(\Cc)$ can also be described as a \emph{colimit} in $\cat{Pr}^\L$, namely
	\begin{equation*}
		 \cat{Sp}\simeq \colimit\left(\Cc_{*/}\overset{\Sigma_\Cc}{\longrightarrow}\Cc_{*/}\overset{\Sigma_\Cc}{\longrightarrow}\Cc_{*/}\overset{\Sigma_\Cc}{\longrightarrow}\dotsb\right)\,.
	\end{equation*}
	Thus, $\cat{Sp}(\Cc)$ is the terminal $\infty$-category over $\Cc_{*/}$ such that $\Omega_\Cc$ becomes invertible, but it's also the initial \emph{presentable} $\infty$-category under $\Cc_{*/}$ such that $\Sigma_\Cc$ becomes invertible. And we get for free that $\Omega_\Cc^\infty\colon \cat{Sp}(\Cc)\rightarrow \Cc_{*/}$ admits a left adjoint $\Sigma_\Cc^\infty\colon \Cc_{*/}\rightarrow \cat{Sp}(\Cc)$. In \cref{lem:Spectrification} and \cref{cor:SigmaInfty}, we'll give an explicit construction of $\Sigma_\Cc^\infty$ that works in greater generality (in particular, not only for presentable $\Cc$), but it's nice to see a first instance where Lurie's magical $\infty$-category $\cat{Pr}^\L$ becomes really useful.
\end{rem}
\begin{cor}[\enquote{$\IE_\infty$-groups are connective spectra}]\label{cor:ConnectiveSpectraCGrp}
	The functor $\B^\infty$ from \cref{con:Binfty} fits into an adjunction
	\begin{equation*}
		\B^\infty\colon \cat{CGrp}(\cat{An})\doublelrmorphism \cat{Sp}\noloc \Omega^\infty\,.
	\end{equation*}
	Furthermore, $\B^\infty$ is fully faithful and its essential image is the full sub-$\infty$-category $\cat{Sp}_{\geqslant 0}\subseteq\cat{Sp}$ of connective spectra.
\end{cor}
\begin{proof}
	This follows immediately from \cref{con:Binfty} together with \cref{lem:SpCGrpIsSp} and \cref{con:HomotopyGroupsOfSpectra}.
\end{proof}
\begin{rem}\label{rem:InfiniteLoopAnimae}
	\cref{cor:ConnectiveSpectraCGrp} implies that an anima $Y$ can be equipped with an $\IE_\infty$-group structure if and only if $Y$ can be written as $\Omega^\infty X$ for some spectrum $X$. Equivalently, $Y$ must admit a compatible sequence $(\dotsc,Y_2,Y_1,Y_0)$ of \emph{deloopings}, satisfying $Y_0\simeq Y$ and $\Omega Y_{n+1}\simeq Y_n$ for all $n\geqslant 0$. This can be regarded as an analogue of \cref{thm:E1Loop}\cref{enum:E1LoopGrp}: Just as $\IE_1$-groups are precisely the loop animae, that is, those animae that can be delooped once, $\IE_\infty$-groups are precisely the \emph{infinite loop animae}, that is, those animae that can be delooped arbitrarily often, in a compatible way. This is the \emph{recognition principle} for infinite loop spaces due to Boardman--Vogt, May, and Segal.
	
	Furthermore, \cref{cor:ConnectiveSpectraCGrp} implies that if $Y\simeq \Omega^\infty X$, then the spectrum $X$ can always be chosen to be connective. In other words, if $Y$ admits a compatible sequence $(\dotsc,Y_2,Y_1,Y_0)$ of deloopings, then we may always assume that $Y_n$ is \emph{$n$-connected} for all $n\geqslant 0$, that is, $\pi_i(Y_n)=0$ for all $i<n$ and all basepoints. The intuitive reason for this is that upon writing $Y\simeq \Omega^nY_n$, all information about $\pi_*(Y_n)$ below degree $n$ will be lost, so we may as well assume these homotopy groups vanish. If we work with spectra (not necessarily connective), this information is instead remembered in the form of negative homotopy groups. In general, working with $\cat{Sp}$ rather than $\cat{CGrp}(\cat{An})$ has a number of advantages, due to the excellent categorical properties of $\cat{Sp}$. These properties are axiomatised in the notion of a \emph{stable $\infty$-category}.
\end{rem}

\begin{defi}\label{def:Stable}
	An $\infty$-category $\Cc$ is called \emph{stable} if it satisfies the equivalent conditions from \cref{lem:Stable} below.
\end{defi}
\begin{lem}\label{lem:Stable}
	Suppose $\Cc$ is an $\infty$-category with an object $0\in \Cc$ that's both initial and terminal. Then the following conditions are equivalent:
	\begin{alphanumerate}
		\item $\Cc$ has finite limits and $\Omega_\Cc\colon \Cc\rightarrow\Cc$ is an equivalence of $\infty$-categories. Here $\Omega_\Cc$ is defined by an analogous pullback diagram as in \cref{def:Loop}.\label{enum:OmegaEquivalence}
		\item $\Cc$ has finite colimits and $\Sigma_\Cc\colon \Cc\rightarrow \Cc$ is an equivalence of $\infty$-categories. Here $\Sigma_\Cc$ is defined by an analogous pushout diagram as in \cref{def:Loop}.\label{enum:SigmaEquivalence}
		\item $\Cc$ has finite limits and finite colimits and a commutative square in $\Cc$ is a pushout square if and only if it is a pullback square.\label{enum:PushoutPullback}
		\item The functor $\Omega_\Cc^\infty\colon \cat{Sp}(\Cc)\rightarrow \Cc$ is an equivalence of $\infty$-categories.\label{enum:SpCisC}
		\item There exists an $\infty$-category $\Dd$ and an equivalence of $\infty$-categories $\cat{Sp}(\Dd)\simeq\Cc$.\label{enum:SpDisC}
	\end{alphanumerate}
	%In this case we call $\Cc$ a stable $\infty$-category.
\end{lem}
\begin{proof}%[Proof of \cref{lem:Stable}]
	The implication \cref{enum:OmegaEquivalence} $\Rightarrow$ \cref{enum:SpCisC} is clear: Since $0\in\Cc$ is both initial and terminal, we have $\Cc_{0/}\simeq \Cc$, and so $\Omega_\Cc\colon \Cc_{0/}\rightarrow \Cc_{0/}$ is an equivalence too. It follows that the limit defining $\cat{Sp}(\Cc)$ is taken along equivalences and thus equivalent to $\Cc$ by (a dual variant of) \cref{lem:ContractibleColimit}. The implication \cref{enum:SpCisC} $\Rightarrow$ \cref{enum:SpDisC} is trivial, as is the implication \cref{enum:SpDisC} $\Rightarrow$ \cref{enum:OmegaEquivalence}: $\Omega_{\cat{Sp}(\Dd)}\colon \cat{Sp}(\Dd)\rightarrow \cat{Sp}(\Dd)$ is an equivalence for obvious reasons (for example, using \cref{con:HomotopyGroupsOfSpectra}, $\Omega_{\cat{Sp}(\Dd)}$ just corresponds to a shift in the index category $\IZ$, which is clearly an equivalence). Furthermore, the implications \cref{enum:PushoutPullback} $\Rightarrow $ \cref{enum:OmegaEquivalence}, \cref{enum:SigmaEquivalence} are also clear: Applying the pushout-pullback condition to the pushout square defining $\Sigma_\Cc$ and the pullback square defining $\Omega_\Cc$ shows that the unit $u\colon\id_{\Cc}\Rightarrow\Omega_\Cc\Sigma_\Cc$ and counit $c\colon \Sigma_\Cc\Omega_\Cc\Rightarrow\id_\Cc$ are natural equivalences, so $\Sigma_\Cc$ and $\Omega_\Cc$ must be equivalences of $\infty$-categories.
	
	It remains to show \cref{enum:OmegaEquivalence} $\Rightarrow$ \cref{enum:PushoutPullback}; the implication \cref{enum:SigmaEquivalence} $\Rightarrow$ \cref{enum:PushoutPullback} will follow from a dual argument. The same argument as in the proof of \cref{lem:SpCGrpIsSp} shows that $\Cc$ is additive  (write $X\simeq \Omega_\Cc (\Omega_\Cc^{-1}X)$ to lift $X$ to an $\IE_1$-group in $\Cc$ and then apply \cref{lem:SemiAddCriterion}). So we only need to check that pushouts exist and coincide with pullbacks. Let $\Xx\subseteq \Fun(\square^2,\Cc)$, where $\square^2\simeq \Delta^1\times\Delta^1$, be the full subcategory spanned by pullback squares. We claim:
	\begin{alphanumerate}\itshape
		\item[\boxtimes]  The restriction $r\colon \Xx\longrightarrow\Fun(\Lambda_0^2,\Cc)$ is an equivalence of $\infty$-categories.\label{claim:PushoutPullbacks}
	\end{alphanumerate}
	To prove \cref{claim:PushoutPullbacks}, we construct a functor $s\colon \Fun(\Lambda_0^2,\Cc)\rightarrow\Xx$ satisfying $r\circ s\simeq (\Omega_\Cc)_*$ and $s\circ r\simeq (\Omega_\Cc)_*$, where $(\Omega_\Cc)_*\colon \Fun(\square^2,\Cc)\rightarrow \Fun(\square^2,\Cc)$ is postcomposition with $\Omega_\Cc$. Since $\Omega_\Cc$ is an equivalence, so is $(\Omega_\Cc)_*$, and \cref{claim:PushoutPullbacks} will be proved. Given a functor $F\colon \Lambda_0\rightarrow \Cc$, which we can view as a span $c\leftarrow a\rightarrow b$ in $\Cc$, we construct the following moderately large diagram:
	\begin{equation*}
		\begin{tikzcd}
			\Omega_\Cc (a)\rar\dar\drar[pullback] & \Omega_\Cc (c)\rar\dar\drar[pullback] & 0\dar & \\
			\Omega_\Cc (b)\rar\dar\drar[pullback] & x \rar\dar\drar[pullback] & f \rar\dar\drar[pullback] & 0\dar\\
			0\rar & g\rar\dar\drar[pullback] & a\dar\rar & b\\
			& 0\rar & c & 
		\end{tikzcd}
	\end{equation*}
	All squares are pullbacks as indicated. The fact that $\Omega_\Cc (a)$, $\Omega_\Cc (b)$, and $\Omega_\Cc (c)$ appear in the top left corner follows by combining suitable pullback squares into larger pullback rectangles. The functor $s\colon \Fun(\Lambda_0^2,\Cc)\rightarrow \Xx$ now sends%
	\newlength{\HeightOfOmega}\settoheight{\HeightOfOmega}{$\Omega$}%
	\newlength{\HeightOfy}\settototalheight{\HeightOfy}{$y$}%
	\begin{equation*}
		F=\mathopen\vast{3.05}(\begin{tikzcd}[column sep=scriptsize,row sep=scriptsize,baseline=(mid.base)]
			\vphantom{\Omega}a\dar[shorten >=1ex-\HeightOfOmega]\rar\drar[phantom,""{name=mid}] & b\\
			\vphantom{\Omega}c & {}
		\end{tikzcd}\mathclose\vast{3.05})\longmapsto\!
		\mathopen\vast{3.05}(\begin{tikzcd}[column sep=scriptsize,row sep=scriptsize,baseline=(mid.base)]
			\Omega_\Cc(a)\dar\rar\drar[pullback]\drar[phantom,""{name=mid}] & \Omega_\Cc (b)\dar\\
			\Omega_\Cc (c)\rar & x
		\end{tikzcd}\mathclose\vast{3.05})
	\end{equation*}
	(technically we have only defined $s$ on objects, but its clear how to make it functorial since limits are functorial). This proves \cref{claim:PushoutPullbacks}.
	
	To construct pushouts, let $F\colon \Lambda_0^2\rightarrow\Cc$ be a span $c\leftarrow a\rightarrow b$ as above. We know from \cref{claim:PushoutPullbacks} that $F$ can be uniquely (up to contractible choice) extended to a pullback square, where the bottom right corner is some object $d\in \Cc$. The same goes for the trivial span consisting of identities $y=y=y$ for some $y\in \Cc$, and in this case the object we have to add in the bottom right corner has to be $y$ by uniqueness. Hence
	\begin{equation*}
		\Hom_{\Fun(\Lambda_0^2,\,\Cc)}\mathopen\vast{3.05}(\begin{tikzcd}[column sep=scriptsize,row sep=scriptsize,baseline=(mid.base),ampersand replacement=\&]
			a\vphantom{y}\dar[shorten <=1ex-\HeightOfy]\rar\drar[phantom,""{name=mid}] \& b\\
			c\vphantom{y} \& {}
		\end{tikzcd},\begin{tikzcd}[column sep=scriptsize,row sep=scriptsize,baseline=(mid.base),ampersand replacement=\&]
			y\eqar[r]\eqar[d]\drar[phantom,""{name=mid}] \& y\\
			y \& {}
		\end{tikzcd}\mathclose\vast{3.05})\simeq \Hom_{\Fun(\Delta^1\times\Delta^1,\,\Cc)}\mathopen\vast{3.05}(\begin{tikzcd}[column sep=scriptsize,row sep=scriptsize,baseline=(mid.base),ampersand replacement=\&]
			a\vphantom{y}\dar[shorten <=1ex-\HeightOfy,shorten >=1ex-\HeightOfOmega]\rar\drar[pullback]\&  \smash{b}\dar\\
			c\vphantom{d}\vphantom{y}\rar \& d
		\end{tikzcd},\begin{tikzcd}[column sep=scriptsize,row sep=scriptsize,baseline=(mid.base),ampersand replacement=\&]
			y\eqar[r]\eqar[d,shorten >=1ex-\HeightOfOmega]\drar[pullback] \& y\eqar[d,shorten >=1ex-\HeightOfOmega]\\
			y\vphantom{d}\eqar[r] \& \smash{y}\vphantom{d}
		\end{tikzcd}\mathclose\vast{3.05})\,.%\\
		%&\simeq \Hom_\Cc\mathopen\vast{3.25}(\colimit\mathopen\vast{3.05}(\begin{tikzcd}[column sep=scriptsize,row sep=scriptsize,baseline=(mid.base),ampersand replacement=\&]
		%	a\vphantom{y}\dar[shorten <=1ex-\HeightOfy,shorten >=1ex-\HeightOfOmega]\rar\drar[pullback]\&  \smash{b}\dar\\
		%	c\vphantom{d}\vphantom{y}\rar \& d
		%\end{tikzcd}\mathclose\vast{3.05}),y\mathclose\vast{3.25})\,.
	\end{equation*}
	By the universal property of colimits, this means that the colimit over the span $c\leftarrow a\rightarrow b$ agrees with the colimit over the commutative square formed by $a$, $b$, $c$, and $d$, provided that at least one of these colimits exists. But $\Delta^1\times \Delta^1$ has a terminal object, namely the vertex $\{1\}\times\{1\}$, and so the colimit over any commutative square exists and is given by the bottom right corner. This shows that $d$ is a pushout of the span $c\leftarrow a\rightarrow b$. Simultaneously, we've also shown that pushout squares agree with pullback squares. This finishes the proof of \cref{enum:OmegaEquivalence} $\Rightarrow$ \cref{enum:PushoutPullback} and so we're done.
\end{proof}
\begin{cor}\label{cor:Exact}
	Let $F\colon \Cc\rightarrow\Dd$ be a functor between stable $\infty$-categories. Then $F$ preserves finite colimits if and only if it preserves finite limits.
\end{cor}
\begin{proof}
	This is an immediate consequence of \cref{lem:KappaSmallColimits}: Since $\Cc$ and $\Dd$ are additive (as we've seen in the proof of \cref{lem:SpCGrpIsSp}), $F$ preserves finite coproducts if and only it preserves finite products. By \cref{lem:Stable}\cref{enum:PushoutPullback}, $F$ preserves pushouts if and only if it preserves pullbacks.
\end{proof}

\begin{defi}\label{def:Exact}
	A functor $F\colon \Cc\rightarrow\Dd$ between stable $\infty$-categories is called \emph{exact} if it preserves finite colimits, or equivalently, finite limits. We let $\cat{Cat}_{\infty}^\mathrm{st}\subseteq\cat{Cat}_\infty$ denote the (non-full) sub-$\infty$-category spanned by stable $\infty$-categories and exact functors between them.
\end{defi}

In the remainder of this subsection, we'll explain how the \emph{derived $\infty$-category} $\Dd(R)$ and its variant $\Dd_{\geqslant 0}(R)$ from crash course~\cref{con:DerivedCategoryI} fit into the framework of stable $\infty$-categories.
\begin{lem}\label{lem:DRStable}
	Let $R$ be a \embrace{not necessarily commutative} ring. Then there exists an equivalence of $\infty$-categories $\Dd(R)\simeq \cat{Sp}(\Dd_{\geqslant 0}(R))$, given on objects by
	\begin{equation*}
		M\longmapsto\Bigl(\dotsc,(\tau_{\geqslant -2}M)[2],(\tau_{\geqslant -1}M)[1],\tau_{\geqslant 0}M\Bigr)\,.
	\end{equation*}
	\embrace{here $\tau_{\geqslant -n}(-)$ are the smart truncations and $(-)[n]$ are the shift functors from crash course~\cref{con:DerivedCategoryI}}. In particular, $\Dd(R)$ is a stable $\infty$-category.
\end{lem}
\begin{proof}[Proof sketch]
	Let's first explain how to construct the desired functor $\Dd(R)\rightarrow \cat{Sp}(\Dd_{\geqslant 0}(R))$ formally. The crucial observation is that $\Omega_{\Dd(R)}\colon \Dd(R)\rightarrow \Dd(R)$ can be identified with the shift functor $(-)[-1]$; we've seen an instance of this \cref{exm:EilenbergMacLaneAnima}, the general case follows from similar arguments as in \cref{lem:ColimitsInDR}\cref{enum:CofibresInDR}. Since $\tau_{\geqslant0}\colon \Dd(R)\rightarrow \Dd_{\geqslant 0}(R)$ is right adjoint to the inclusion $\Dd_{\geqslant 0}(R)\subseteq \Dd(R)$, it follows formally that $\Omega_{\Dd_{\geqslant 0}(R)}\colon \Dd_{\geqslant 0}(R)\rightarrow\Dd_{\geqslant 0}(R)$ is given by $\tau_{\geqslant 0}((-)[-1])$. Then we get an equivalence of functors
	\begin{equation*}
		\tau_{\geqslant -n}(-)[n]\simeq \Omega_{\Dd_{\geqslant 0}(R)}\circ \tau_{\geqslant -(n+1)}(-)[n+1]
	\end{equation*}
	in $\Fun(\Dd(R),\Dd_{\geqslant 0}(R))$ for all $n\geqslant 0$. Indeed, substituting $\tau_{\geqslant 0}((-)[-1])$ for $\Omega_{\Dd_{\geqslant 0}(R)}$, this equivalence is straightforward to verify in $\Fun(\Ch(R),\Ch_{\geqslant 0}(R))$; after that, \cref{lem:Localisation} does the rest. Thus, the functors $\tau_{\geqslant -n}(-)[n]\colon \Dd(R)\rightarrow \Dd_{\geqslant 0}(R)$ for all $n\geqslant 0$ assemble into a functor $\Dd(R)\rightarrow \cat{Sp}(\Dd_{\geqslant 0}(R))$, as desired.
	
	Now we'll verify that this functor is fully faithful and essentially surjective. For fully faithfulness, we employ \cref{lem:HomInLimits}\cref{enum:HomInLimits} to compute $\Hom_{\cat{Sp}(\Dd_{\geqslant 0}(R))}$; we must then show that $\Hom_{\Dd(R)}(M,N)\simeq \limit_{n\in\IN}\Hom_{\Dd_{\geqslant 0}(R)}((\tau_{\geqslant -n}M)[n],(\tau_{\geqslant -n}N)[n])$ for all $M,N\in\Dd(R)$. Clearly, we can get rid of the shifts and instead write $\limit_{n\in\IN}\Hom_{\Dd_{\geqslant -n}(R)}(\tau_{\geqslant -n}M,\tau_{\geqslant -n}N)$, where the transition morphisms are induced by applying the functor $\tau_{\geqslant -n}$. This functor can also be viewed as a right adjoint $\tau_{\geqslant -n}\colon \Dd(R)\rightarrow \Dd_{\geqslant -n}(R)$ of the inclusion $\Dd_{\geqslant -n}(R)\subseteq \Dd(R)$. Therefore, we get an adjunction equivalence $\Hom_{\Dd_{\geqslant -n}(R)}(\tau_{\geqslant -n}M,\tau_{\geqslant -n}N)\simeq \Hom_{\Dd(R)}(\tau_{\geqslant -n}M,N)$. We claim:
	\begin{alphanumerate}\itshape
		\item[\boxtimes] Under these adjunction equivalences, the transition morphisms, which were originally induced by $\tau_{\geqslant -n}$, get identified with the precomposition morphisms\label{claim:TransitionMorphisms}
		\begin{equation*}
			c_n^*\colon \Hom_{\Dd(R)}\left(\tau_{\geqslant -(n+1)}M,N\right)\longrightarrow \Hom_{\Dd(R)}\left(\tau_{\geqslant -n}M,N\right)
		\end{equation*}
		induced by the canonical morphisms $c_n\colon \tau_{\geqslant -n}M\rightarrow \tau_{\geqslant -(n+1)}M$.
	\end{alphanumerate}
	To prove~\cref{claim:TransitionMorphisms}, recall from the proof of \cref{lem:TriangleIdentities} that any adjunction equivalence can, at least pointwise, be obtained by applying the right adjoint and then precomposing with the unit transformation. In our case, we see that $\Hom_{\Dd_{\geqslant -n}(R)}(\tau_{\geqslant -n}M,\tau_{\geqslant -n}N)\simeq \Hom_{\Dd(R)}(\tau_{\geqslant -n}M,N)$ is simply given by applying $\tau_{\geqslant -n}$, since the unit $u_{\tau_{\geqslant -n} M}\colon \tau_{\geqslant -n}M\rightarrow \tau_{\geqslant -n}(\tau_{\geqslant -n}M)$ is just the identity. To show~\cref{claim:TransitionMorphisms}, we now simply observe that $\tau_{\geqslant -n}\circ\tau_{\geqslant -(n+1)}\simeq \tau_{\geqslant -n}$ and that $\tau_{\geqslant -n}(c_n)$ is the identity on $\tau_{\geqslant -n}M$.
	
	Using \cref{claim:TransitionMorphisms} and \cref{cor:HomPreservesColimits}, we see that to show the desired equivalence
	\begin{equation*}
		\Hom_{\Dd(R)}(M,N)\overset{\simeq}{\longrightarrow}\limit_{n\in\IN}\Hom_{\Dd(R)}\left(\tau_{\geqslant -n}M,N\right)\,,
	\end{equation*}
	it'll be enough to show $\colimit_{n\in\IN}\tau_{\geqslant -n}M\simeq M$. To prove this, observe that filtered colimits in $\Ch(R)$ preserve quasi-isomorphisms. Through \cref{lem:Localisation}, this formally implies that $\Ch(R)\rightarrow \Dd(R)$ preserves filtered colimits (we've seen analogous arguments in the proofs of \cref{lem:HomotopyGroupsFilteredColimits} and \cref{cor:AnPresentable}). So $\colimit_{n\in\IN}\tau_{\geqslant -n}M\simeq M$ can be checked on the level of chain complexes, where it becomes obvious. This finishes the proof that $\Dd(R)\rightarrow \cat{Sp}(\Dd_{\geqslant 0}(R))$ is fully faithful.
	
	To show essential surjectivity, observe that objects in $\cat{Sp}(\Dd_{\geqslant 0}(R))$ are given by sequences $(\dotsc,M_2,M_1,M_0)$ in $\Dd_{\geqslant 0}(R)$ together with equivalences $M_n\simeq \Omega_{\Dd_{\geqslant 0}(R)}M_{n+1}\simeq \tau_{\geqslant 0}(M_{n+1}[-1])$. These equivalences induce morphisms $M_n\rightarrow M_{n+1}[-1]$ in $\Dd(R)$ and we can form the colimit $M\coloneqq \colimit_{n\in\IN}M_n[-n]$. Using once again that filtered colimits in $\Dd(R)$ are well-understood, one checks that $M$ is a preimage of $(\dotsc,M_2,M_1,M_0)$ up to equivalence.
\end{proof}


\begin{numpar}[Eilenberg--MacLane spectra]\label{exm:EilenbergMacLaneSpectra}
	In the special case $R=\IZ$, we have the Eilenberg--MacLane functor $\K\colon \Dd_{\geqslant 0}(\IZ)\rightarrow \cat{An}$ from \cref{con:Homology}, which preserves all limits (being a right adjoint) and thus commutes with $\Omega$. Therefore, $\K$ induces a functor $\H\colon \Dd(\IZ)\rightarrow \cat{Sp}$ via the commutative diagram
	\begin{equation*}
		\begin{tikzcd}
			\cat{Sp}\bigl(\Dd_{\geqslant 0}(\IZ)\bigr)\rar["{\cat{Sp}(\cat{K})}"]\dar["\simeq"']\drar[commutes] & \cat{Sp}(\cat{An})\eqar[d]\\
			\Dd(\IZ)\rar["\H"] & \cat{Sp}
		\end{tikzcd}
	\end{equation*}
	We call $\H$ the \emph{Eilenberg--MacLane spectrum functor}. Unravelling the equivalence $\Dd(\IZ)\simeq \cat{Sp}(\Dd_{\geqslant 0}(\IZ))$ from \cref{lem:DRStable} and the construction of $\cat{Sp}(\K)\colon \cat{Sp}(\Dd_{\geqslant 0}(\IZ))\rightarrow \cat{Sp}$, we see that for all abelian groups $A$ and all $n\geqslant 0$, the spectrum $\H A\coloneqq \H (A[0])$ is explicitly given by the sequence of animae
	\begin{equation*}
		\H A\simeq \bigl(\dotsc,\K(A,2),\K(A,1),\K(A,0)\bigr)\,.
	\end{equation*}
	This fits perfectly with the homotopy equivalences $\K(A,n)\simeq \Omega\! \K(A,n+1)$ from \cref{exm:EilenbergMacLaneAnima}. We call $\H A$ the \emph{Eilenberg--MacLane spectrum of $A$}.
	
	The Eilenberg--MacLane functor induces an equivalence of $\infty$-categories $\H\colon \cat{Ab}\overset{\simeq}{\longrightarrow}\cat{Sp}^\heartsuit$ from the (ordinary) category of abelian groups onto the $\infty$-category $\cat{Sp}^\heartsuit\coloneqq \cat{Sp}_{\geqslant 0}\cap \cat{Sp}_{\leqslant 0}$ of spectra concentrated in degree $0$. Indeed, an inverse functor is provided via $\cat{Sp}^\heartsuit\rightarrow \cat{Sp}_{\geqslant 0}\simeq \cat{CGrp}(\cat{An})$ (\cref{cor:ConnectiveSpectraCGrp}) and $\pi_0\colon \cat{CGrp}(\cat{An})\rightarrow \cat{CGrp}(\cat{Set})\simeq \cat{Ab}$. In the modern point of view, abelian groups \emph{are} just spectra concentrated in degree $0$. Following this, we'll often suppress $\H$ in the notation and write the Eilenberg--MacLane spectrum just as $A$.\footnote{In the words of Robert Burklund: \enquote{Why would you give a name to the functor that sends an abelian group to itself?}}
\end{numpar}
In the classical theory of derived categories, much emphasis is placed on the fact that $D(R)$ can be equipped with a \emph{triangulated structure}. Let us now explain how this structure is captured and radically simplified by the fact that the derived $\infty$-category $\Dd(R)$ is stable.

\begin{numpar}[Stable $\infty$-categories and triangulated categories.]
	One striking feature of stable $\infty$-categories is that their homotopy category admits a canonical triangulated structure. If you haven't see triangulated categories before, \cite[Definition~\HAthm{1.1.2.5}]{HA} has a nice review (but you can also safely skip this remark). Moreover, \cite[Theorem~\HAthm{1.1.2.14}]{HA} explains the triangulated structure in much more detail than we'll do below.
	
	Let $\Cc$ be a stable $\infty$-category. We choose $(-)[1]\coloneqq \operatorname{ho}(\Sigma_\Cc)\colon \operatorname{ho}(\Cc)\rightarrow \operatorname{ho}(\Cc)$ to be the shift functor in our emerging triangulated structure. By \cref{lem:Stable}\cref{enum:SigmaEquivalence}, $(-)[1]$ is an equivalence of categories. We say that $x\rightarrow y\rightarrow z\rightarrow x[1]$ is a \emph{distinguished triangle} if $x\rightarrow y\rightarrow z$ is a cofibre sequence in $\Cc$ in the sense of \cref{def:Cofibre}. Then we can form the following pushout diagram
	\begin{equation*}
		\begin{tikzcd}
			x\rar\dar\drar[pushout] & y\rar\dar\drar[pushout] & 0\dar\\
			0\rar & z\rar & \Sigma_\Cc(x)
		\end{tikzcd}
	\end{equation*}
	(to see why $\Sigma_\Cc(x)$ appears in the bottom left corner, just observe that the outer rectangle must be a pushout too). This shows several things at once: First it explains where the morphism $z\rightarrow x[1]$ in a distinguished triangle comes from. Second, a closer investigation of the diagram shows that $x\rightarrow y\rightarrow z$ is a cofibre sequence if and only if $y\rightarrow z\rightarrow \Sigma_\Cc(x)$ is a cofibre sequence.%
	\footnote{Here's the argument: We've already seen that $x\rightarrow y\rightarrow z$ being a cofibre sequence implies the same for $y\rightarrow z\rightarrow \Sigma_\Cc(x)$. Conversely,  if $y\rightarrow z\rightarrow \Sigma_\Cc(x)$ is a cofibre sequence, then the right square in the diagram is a pushout, hence a pullback by \cref{lem:Stable}\cref{enum:PushoutPullback}. Similarly, the outer square must be a pullback. It follows formally that the left square must be a pullback too, hence a pushout, and so $x\rightarrow y\rightarrow z$ is a cofibre sequence too.}
	%
	Hence $x\rightarrow y\rightarrow z\rightarrow x[1]$ is a distinguished triangle if and only if $y\rightarrow z\rightarrow x[1]\rightarrow y[1]$ is a distinguished triangle. In other words, Verdier's axiom (TR\textsubscript{2}) is satisfied.
	
	Furthermore, it's immediately clear that every morphism $x\rightarrow y$ can be extended to a distinguished triangle (just form the cofibre), that distinguished triangles are closed under isomorphisms in $\cat{Ho}(\Cc)$, and that for every $x\in \Cc$, the identity $\id_x\colon x\rightarrow x$ fits into a distinguished triangle $x\rightarrow x\rightarrow 0\rightarrow x[1]$. So (TR\textsubscript{1}) is satisfied.
	
	Next, we'll tackle axiom (TR\textsubscript{3}). Since taking cofibres is functorial, for every commutative diagram in $\Cc$ as below there is a unique dashed arrow (up to contractible choice):
	\begin{equation*}
		\begin{tikzcd}
			x\rar["\alpha"]\dar["\beta"']\drar[commutes] & y\rar\dar["\gamma"] &\cofib(\alpha)\dar[dashed]\\
			x'\rar["\alpha'"] & y'\rar & \cofib(\alpha')
		\end{tikzcd}
	\end{equation*}
	The crucial detail here is \enquote{$\scriptscriptstyle/\!/\!/$}: A commutative diagram in $\Cc$ is a functor $\ov\sigma\colon\square^2\rightarrow \Cc$, whereas a commutative diagram in $\operatorname{ho}(\Cc)$ is a functor $\sigma\colon\partial\square^2\rightarrow \Cc$, which can be extended to a functor $\ov\sigma\colon\square\rightarrow \Cc$; however, \emph{the choice of $\ov\sigma$ is not part of the data!} In particular, there could be several non-homotopic choices, corresponding to the fact that $\pi_1(\Hom_\Cc(x,y'),\gamma\circ \alpha)$ may not be trivial. So taking cofibres is \emph{not} functorial in commutative diagrams in $\operatorname{ho}(\Cc)$. If we start with a commutative diagram in $\operatorname{ho}(\Cc)$, then a dashed arrow will exist, but it will not necessarily be unique; the uniqueness only comes about once a filler $\ov\sigma\colon \square^2\rightarrow \Cc$ has been chosen, which we indicate by writing \enquote{$\scriptscriptstyle/\!/\!/$} in a diagram as above. This shows axiom (TR\textsubscript{3}) and it offers a nice conceptual explanation of the non-uniqueness statement in that axiom.
	
	Finally, let's talk about the dreaded \emph{octahedron axiom} (TR\textsubscript{4}): Given morphisms $\alpha\colon x\rightarrow y$ and $\beta\colon y\rightarrow z$ in $\Cc$, we can form a pushout diagram
	\begin{equation*}
		\begin{tikzcd}
			x\rar["\alpha"]\dar\drar[pushout] & y\rar["\beta"]\dar\drar[pushout] & z\dar\\
			0\rar & \cofib(\alpha)\rar\dar\drar[pushout] & \cofib(\beta\circ\alpha)\dar\\
			& 0\rar & \cofib(\beta)
		\end{tikzcd}
	\end{equation*}
	which shows that $\cofib(\alpha)\rightarrow \cofib(\beta\circ\alpha)\rightarrow \cofib(\beta)$ is a cofibre sequence in $\Cc$. And that's already the octahedron axiom!
	
	Not every triangulated category arises as the homotopy category of a stable $\infty$-category. However, every triangulated category encountered in nature does, the primordial example being the derived category $D(R)$ of a ring $R$, which arises as the homotopy category of $\Dd(R)$, which is stable by \cref{exm:EilenbergMacLaneSpectra} and \cref{lem:Stable}\cref{enum:SpDisC}. The point we're trying to make here is that whenever you would work with triangulated categories, you should use stable $\infty$-categories instead: It is both conceptually simpler and more powerful! For a concrete example, you might have seen the \emph{filtered derived category} of a ring $R$ before. In the classical approach, you run into annoying technical subtleties when you try to define it in full generality; this is the reason why the Stacks Project only considers degree-wise finite filtrations in \cite[\stackstag{05RX}]{Stacks}. But on the level of $\infty$-categories, everything works as expected: We simply define $\Fil(\Dd(R))\coloneqq \Fun(\IZ,\Dd(R))$. This is a stable $\infty$-category again%
	%
	\footnote{In general, if $\Cc$ is stable, then so is $\Fun(\Ii,\Cc)$ for any $\infty$-category $\Ii$; to see this, use that limits and colimits in functor $\infty$-categories can be computed pointwise (\cref{lem:ColimitsInFunctorCategories}) to verify your favourite condition from \cref{lem:Stable}.}
	and so its homotopy category $\operatorname{ho}\Fil(\Dd(R))$ is canonically a triangulated category. This is the \enquote{right} definition of the filtered derived category. It also explains where the technical subtleties come from: The homotopy category $\operatorname{ho}\Fun(\IZ,\Dd(R))$ is in general not the same as $\Fun(\IZ,\operatorname{ho}\Dd(R))$
\end{numpar}
As our final application to the theory of derived $\infty$-categories, we would like to explain the relationship between $\Hom_{\Dd(R)}(M,N)$ and $\RHom_R(M,N)$. This needs a general construction, which is pretty important on its own.
\begin{lem}\label{cor:hom}
	If $\Cc$ is a stable $\infty$-category, then the $\Hom$ animae in $\Cc$ can be refined to spectra. More precisely, there is a unique \embrace{up to equivalence} functor $\hom_\Cc\colon \Cc^\op\times\Cc\rightarrow\cat{Sp}$ fitting into the following diagram:
	\begin{equation*}
		\begin{tikzcd}
			& \cat{Sp}\dar["\Omega^\infty"]\\
			\Cc^\op\times\Cc\urar[dashed,"\hom_\Cc"]\rar["\Hom_\Cc"] & \cat{An} 
		\end{tikzcd}
	\end{equation*}
\end{lem}
\begin{proof}
	The Yoneda embedding $\Yo_\Cc\colon \Cc\rightarrow \Fun(\Cc^\op,\cat{An})$ preserves limits by \cref{cor:HomPreservesLimits}. In particular, it commutes with $\Omega$ and thus induces a functor $\cat{Sp}(\Yo_\Cc)\colon \cat{Sp}(\Cc)\rightarrow \cat{Sp}(\Fun(\Cc^\op,\cat{An}))$, which is uniquely (up to equivalence) characterised by the fact that $\Omega^\infty \Yo_\Cc^\mathrm{st}\simeq \Yo_\Cc$. Now $\Fun(\Cc^\op,-)\colon \cat{Cat}_\infty\rightarrow \cat{Cat}_\infty$ commutes with limits, since it has a left adjoint given by $-\times\Cc^\op$. Furthermore, $\Fun(\Cc^\op,\cat{An}_{*/})\simeq \Fun(\Cc^\op,\cat{An})_{\const */}$. Hence $\cat{Sp}(\Fun(\Cc^\op,\cat{An}))\simeq \Fun(\Cc^\op,\cat{Sp})$ and so we've upgraded the Yoneda embedding to a functor
	\begin{equation*}
		\Yo_\Cc^\mathrm{st}\colon \Cc\longrightarrow \Fun(\Cc^\op,\cat{Sp})\,.
	\end{equation*}
	After currying, this induces the desired functor $\hom_\Cc\colon \Cc^\op\times\Cc\rightarrow \cat{Sp}$. Uniqueness follows from uniqueness of $\Yo_\Cc^\mathrm{st}$.
\end{proof}
\begin{cor}\label{cor:RHom}
	For any ring $R$, the spectra-enriched hom in the derived $\infty$-category $\Dd(R)$ is given by
	\begin{equation*}
		\hom_{\Dd(R)}(M,N)\simeq \RHom_R(M,N)\,,
	\end{equation*}
	that is, the Eilenberg--MacLane spectrum associated to $\RHom_R(M,N)\in\Dd(R)$ as in \cref{exm:EilenbergMacLaneSpectra}, but we suppress writing $\H$.
\end{cor}
\begin{proof}[Proof sketch]
	By the uniqueneness statement from \cref{cor:hom}, it's enough to construct a functorial equivalence $\Omega^\infty\!\RHom_R(M,N)\simeq \Hom_\Dd(R)(M,N)$. Unravelling the construction in \cref{exm:EilenbergMacLaneSpectra}, we see that $\Omega^\infty\!\RHom_R(M,N)\simeq \K(\tau_{\geqslant 0}\RHom_R(M,N))$ is the Eilenberg--MacLane anima associated to the trunctation $\tau_{\geqslant}\RHom_R(M,N)$. Now any anima $X$ satisfies $X\simeq \Hom_{\cat{An}}(*,X)$ and then we can compute
	\begin{align*}
		\Hom_{\cat{An}}\left(*,\K(\tau_{\geqslant 0}\RHom_R(M,N))\right)&\simeq \Hom_{\Dd_{\geqslant 0}(\IZ)}\left(\IZ[0],\tau_{\geqslant 0}\RHom_R(M,N)\right)\\
		&\simeq \Hom_{\Dd(\IZ)}\left(\IZ[0],\RHom_R(M,N)\right)\\
		&\simeq\Hom_{\Dd(R)}\bigl(\IZ[0]\lotimes_\IZ M,N\bigr)\\
		&\simeq\Hom_{\Dd(R)}(M,N)\,.
	\end{align*}
	In the first step, we use the adjunction $\C\colon \cat{An}\shortdoublelrmorphism \Dd_{\geqslant 0}(\IZ)\noloc \K$ from \cref{con:Homology}. In the second step, we use that $\tau_{\geqslant 0}\colon \Dd(\IZ)\rightarrow \Dd_{\geqslant 0}(\IZ)$ is right adjoint to the inclusion $\Dd_{\geqslant 0}(\IZ)\subseteq \Dd(\IZ)$, as we've seen in crash course~\cref{con:DerivedCategoryI}. In the third step, we use the \enquote{derived tensor-$\Hom$ adjunction}. To construct this, using the description from crash course~\cref{con:DerivedCategoryIII}, we only need to verify that the ordinary tensor-$\Hom$ adjunction refines to an adjunction of Kan-enriched categories, which is straightforward. The fourth step is obvious.
	
	Putting everything together, we get $\Omega^\infty\!\RHom_R(M,N)\simeq \Hom_\Dd(R)(M,N)$, as desired. Since all steps can easily be made functorial, we're done.
\end{proof}

\subsection{Spectra and excisive functors}
In this subsection, we'll explain an alternative construction of $\cat{Sp}(\Cc)$ that more closely resembles the \enquote{Segal models} for $\IE_1$-groups and $\IE_\infty$-groups from \cref{def:E1Monoids,def:EinftyMonoid}. This alternative model will be needed in \cref{sec:TensorProduct} to construct the tensor product of spectra, but we'll also use it to show that $\Omega^\infty \colon \cat{Sp}\rightarrow \cat{An}$ has a left adjoint and to construct the famous \emph{sphere spectrum} $\IS$.
\begin{defi}\label{def:Excisive}
	Let $\FinAn\subseteq \cat{An}_{*/}$ be the $\infty$-category of \emph{finite pointed animae}, defined as smallest full sub-$\infty$-category that contains ${S^0}\simeq *\ \,*$ and is closed under finite colimits.\footnote{$\FinAn$ looks like it could be the full sub-$\infty$-category $(\cat{An}_{*/})^{\aleph_0}\subseteq \cat{An}_{*/}$ spanned by the compact objects (in the sense of \cref{def:KappaFiltered}\cref{enum:KappaCompact}), but it's not: $(\cat{An}_{*/})^{\aleph_0}$ also contains all retracts of objects in $\FinAn$.} Furthermore, let $\Cc$ be an $\infty$-category with all finite limits, so that, in particular, $\Cc$ contains a terminal object $*\in\Cc$.
	\begin{alphanumerate}
		\item A functor $F\colon \FinAn\rightarrow \Cc_{*/}$ is called \emph{reduced} if $F(*)\simeq *$.\label{enum:Reduced}
		\item A functor $F\colon \FinAn\rightarrow \Cc_{*/}$ is called \emph{excisive} if $F$ sends pushout squares to pullback squares.\label{enum:Excisibe}
	\end{alphanumerate}
	Furthermore, we let $\Fun_*(\FinAn,\Cc_{*/})\subseteq \Fun_*^\mathrm{exc}(\FinAn,\Cc_{*/})\subseteq\Fun(\FinAn,\Cc_{*/})$ denote full sub-$\infty$-categories spanned by the reduced functors or the reduced excisive functors, respectively.
\end{defi}
\begin{lem}\label{lem:FunExcStable}
	If $\Cc$ has finite limits, then $\Fun_*^\mathrm{exc}(\FinAn,\Cc_{*/})$ is a stable $\infty$-category.
\end{lem}
\begin{proof}
	We'll verify the conditions from \cref{lem:Stable}\cref{enum:OmegaEquivalence}. Since limits in colimits in functor categories are computed pointwise by \cref{lem:ColimitsInFunctorCategories}, it follows that $\Fun(\FinAn,\Cc_{*/})$ has all finite limits and that the terminal object $\const *$ is also initial. Furthermore, reduced excisive functors are closed under all limits, and so $\Fun_*^\mathrm{exc}(\FinAn,\Cc_{*/})$ still has all finite limits and its terminal object is initial too. It remains to show that the loop functor $\Omega$ on $\Fun_*^\mathrm{exc}(\FinAn,\Cc_{*/})$ is an equivalence. We'll show that precomposition with $\Sigma\colon \FinAn\rightarrow \FinAn$ provides an inverse. To this end, let $F\colon \FinAn\rightarrow\Cc_{*/}$ be a reduced excisive functor. By definition of $\Sigma$, we have a diagram of natural transformations
	\begin{equation*}
		\begin{tikzcd}
			F(-)\doublear{d}\doublear{r}\drar[commutes] & F(\const *)\doublear{d}\\
			F(\const *)\doublear{r} & F\left(\Sigma(-)\right)
		\end{tikzcd}
	\end{equation*}
	Since $F(*)\simeq *$, we have $F(\const *)\simeq \const *$. Therefore, this diagram induces a natural transformation $\eta_F\colon F(-)\Rightarrow \Omega F(\Sigma(-))$. Since $F$ sends pushout squares to pullbacks, $\eta_F$ is a pointwise equivalence, hence an equivalence of functors by \cref{thm:EquivalencePointwise}. Furthermore, it's clear from the construction that $\eta_F$ is also functorial in $F$.
	
	Now observe that the loop functor $\Omega$ on $\Fun_*^\mathrm{exc}(\FinAn,\Cc_{*/})$ is given by postcomposition with the loop functor $\Omega_\Cc\colon \Cc_{*/}\rightarrow \Cc_{*/}$, as follows from \cref{lem:ColimitsInFunctorCategories}. Since, in general, postcomposition commutes with precomposition, our functorial equivalence $F(-)\simeq \Omega F(\Sigma(-))$ thus shows that precomposition with $\Sigma$ is both a left and a right inverse of $\Omega$.
\end{proof}
\begin{lem}\label{lem:FunExcIsSp}
	For every $n\geqslant 0$, let $\ev_{S^n}\colon \Fun_*^\mathrm{exc}(\FinAn,\Cc_{*/})\rightarrow \Cc_{*/}$ be given by evaluation at the $n$-sphere $S^n$. Then $\ev_{S^{n+1}}\simeq \Omega_\Cc\circ \ev_{S^n}$ holds for all $n\geqslant 0$ and the induced functor
	\begin{equation*}
		\Fun_*^\mathrm{exc}\bigl(\FinAn,\Cc_{*/}\bigr)\overset{\simeq}{\longrightarrow}\cat{Sp}(\Cc)
	\end{equation*}
	is an equivalence of $\infty$-categories.
\end{lem}
\begin{proof}[Proof sketch]
	The condition $\ev_{S^{n+1}}\simeq \Omega_\Cc\circ \ev_{S^n}$ follows immediately from $S^{n+1}\simeq \Sigma S^n$ and the fact that excisive functors send pushouts to pullbacks, so the hard part will be to show that we get an equivalence. Let's first consider the case where $\Cc$ is a stable $\infty$-category. We'll show that
	\begin{equation*}
		\ev_{S^0}\colon \Fun_*^\mathrm{exc}\bigl(\FinAn,\Cc_{*/}\bigr)\overset{\simeq}{\longrightarrow} \Cc
	\end{equation*}
	is an equivalence of $\infty$-categories. This special case will occupy the majority of the proof; the general case is an easy consequence, as we'll see below. If $\Cc$ is stable, then $\Cc_{*/}\simeq \Cc$, since the terminal object is also initial. This also shows that a functor is reduced if and only if it preserves initial objects. Furthermore, since pushout squares and pullback squares agree in any stable $\infty$-category, a functor $F\colon \FinAn\rightarrow \Cc$ is excisive if and only if it preserves pushouts. Any finite colimit can be built from initial objects and pushouts. Indeed, by \cref{lem:KappaSmallColimits}, we only have to show that finite coproducts can be built that way, but coproducts are pushouts over the initial object. In summary, we obtain
	\begin{equation*}
		\Fun_*^\mathrm{exc}\bigl(\FinAn,\Cc_{*/}\bigr)\simeq \Fun^{\mathrm{fin}\mhyph\mathrm{colim}}\bigl(\FinAn,\Cc\bigr)\,,
	\end{equation*}
	where $\Fun^{\mathrm{fin}\mhyph\mathrm{colim}}\subseteq \Fun$ denotes the full sub-$\infty$-category of functors that preserve finite colimits.
	
	Now let $\Ii\coloneqq \{\InlineS\}$ be the full sub-$\infty$-category of $\FinAn$ spanned by $*$ and $S^0$. Observe that $\Ii\simeq \cat{Fin}_{\leqslant 1}$, where $\cat{Fin}_{\leqslant 1}$ is the (ordinary) category from the proof of \cref{lem:CGrpIsC}.\footnote{Indeed, it's straightforward to verify that $\operatorname{ho}(\Ii)\simeq \cat{Fin}_{\leqslant 1}$. But one also easily verifies that the $\Hom$ animae in $\Ii$ are discrete, and so the canonical functor $\Ii\rightarrow \operatorname{ho}(\Ii)$ is an equivalence by \cref{thm:EquivalenceFullyFaithfulEssentiallySurjective}.} Applying claim~\cref{claim:LeftKanEasier} from the proof of \cref{lem:CGrpIsC}, we obtain that evaluation at $S^0$ induces an equivalence of $\infty$-categories
	\begin{equation*}
		\ev_{S^0}\colon \Fun_*\left(\Ii,\Cc\right)\overset{\simeq}{\longrightarrow}\Cc\,.
	\end{equation*}
	So it remains to show that restriction along the inclusion $i\colon \Ii\rightarrow \FinAn$  induces an equivalence of $\infty$-categories
	\begin{equation*}
		i^*\colon \Fun^{\mathrm{fin}\mhyph\mathrm{colim}}\bigl(\FinAn,\Cc\bigr)\overset{\simeq}{\longrightarrow} \Fun_*\left(\Ii,\Cc\right)\,.
	\end{equation*}
	To prove this, we'll study left Kan extension along the inclusion $i$. By \cref{lem:Smash} below, every reduced functor $F\colon \Ii\rightarrow \Cc$ admits a left Kan extension; furthermore, that lemma provides an explicit formula (in form of a pushout diagram) for $\Lan_iF(X,x)$. Combining this formula with \cref{lem:ColimitManipulations}\cref{claim:AssembleColimits} shows that $\Lan_iF\colon \FinAn\rightarrow \Cc$ preserves pushouts. $\Lan_iF$ also preserves initial objects since $F$ was assumed to be reduced. Hence $\Lan_iF$ preserves all finite colimits. Therefore, usual left Kan extension adjunction $\Lan_i\dashv i^*$ restricts to an adjunction
	\begin{equation*}
		\Lan_i\colon \Fun_*\left(\Ii,\Cc\right)\doublelrmorphism \Fun^{\mathrm{fin}\mhyph\mathrm{colim}}\bigl(\FinAn,\Cc\bigr)\noloc i^*\,.
	\end{equation*}
	Since $i$ is fully faithful, so is $\Lan_i$ by \cref{cor:KanExtensionAlongFullyFaithful}. Furthermore, since $\FinAn$ is generated under finite colimits by $*$ and $S^0$, it's clear that $i^*$ is conservative. So $\Lan_i$ and $i^*$ are inverse equivalences by \cref{lem:FullyFaithfulConservativeAdjunction}\cref{enum:Conservative}, which is what we wanted to show.
	
	It remains to deduce the general case. So let $\Cc$ again be an arbitrary $\infty$-category with finite limits. Then an equivalence $\cat{Sp}(\Cc)\simeq\Fun_*^\mathrm{exc}(\FinAn,\Cc_{*/})$ can be obtained as follows:
	\begin{equation*}
		\cat{Sp}(\Cc)\simeq \Fun_*^\mathrm{exc}\bigl(\FinAn,\cat{Sp}(\Cc)\bigr)\simeq \cat{Sp}\Bigl(\Fun_*^\mathrm{exc}\bigl(\FinAn,\Cc_{*/}\bigr)\Bigr)\simeq \Fun_*^\mathrm{exc}\bigl(\FinAn,\Cc_{*/}\bigr)
	\end{equation*}
	The first equivalence follows from what we've just shown. The third equivalence follows from $\Fun_*^\mathrm{exc}(\FinAn,\Cc_{*/})$ being stable by \cref{lem:FunExcStable}. So let's explain where the second equivalence comes from: The functor $\Fun(\FinAn,-)\colon \cat{Cat}_\infty\rightarrow \cat{Cat}_\infty$ commutes with limits since it is a right adjoint by \cref{exm:Adjunctions}\cref{enum:Currying}. Hence $\Fun(\FinAn,\cat{Sp}(\Cc))\simeq \cat{Sp}(\Fun(\FinAn,\Cc_{*/}))$. Now $\Fun_*^\mathrm{exc}(\FinAn,\cat{Sp}(\Cc))$ and $\cat{Sp}(\Fun_*^\mathrm{exc}(\FinAn,\Cc_{*/}))$ can be regarded as full sub-$\infty$-categories of the left- and the right-hand side, respectively, and we only have to check that they match. To see this, recall that limits in $\cat{Sp}(\Cc)$ are formed degree-wise by \cref{lem:HomInLimits}\cref{enum:ColimitsInLimits}, and so a functor $F\colon \FinAn\rightarrow \cat{Sp}(\Cc)$ is reduced and excisive if and only if $\Omega^{\infty-n}_\Cc \circ F\colon \FinAn\rightarrow \Cc_{*/}$ is reduced and excisive for all $n\in\IZ$. This is precisely what we need.
	
	So we've constructed an equivalence $\cat{Sp}(\Cc)\simeq\Fun_*^\mathrm{exc}(\FinAn,\Cc_{*/})$. By a straightforward unravelling, this equivalence is really induced by $\ev_{S^n}$ for all $n\geqslant 0$.
\end{proof}
\begin{lem}\label{lem:Smash}
	Let $\Cc$ be an $\infty$-category with finite colimits; in particular, $\Cc$ contains an initial object $0\in \Cc$. Let $i\colon \Ii\rightarrow \FinAn$ be as in the proof of \cref{lem:FunExcIsSp} and let $F\colon \Ii\rightarrow \Cc$ be a functor such that $F(*)\simeq 0$. Then $\Lan_iF\colon \FinAn\rightarrow \Cc$ exists and its value on a pointed anima $(X,x)$ is given as the pushout
	\begin{equation*}
		\begin{tikzcd}
			F\bigl(S^0\bigr)\dar \rar\drar[pushout] & \colimit\bigl(\const F(S^0)\colon X\rightarrow \Cc\bigr)\dar\\
			0\rar & \Lan_iF(X,x)
		\end{tikzcd}
	\end{equation*}
	in $\Cc$ \embrace{where the top horizontal arrow is induced by $\{x\}\rightarrow X$}. 
\end{lem}
%As we'll see in \cref{cor:SigmaInfty}, what \cref{lem:Smash} is trying to say is that $\Lan_iF(X,x)$ is given as the \enquote{smash product $X\wedge F(S^0)$}, even though this is only literally true for $\Cc\simeq \cat{An}_{*/}$
\begin{proof}%[Proof of \cref{lem:Smash}]
	First note that the pushout above exists in $\Cc$. Indeed, since $\Cc$ is stable, it has all finite colimits by \cref{lem:Stable}\cref{enum:PushoutPullback}. In particular, since $X$ is a finite anima, $\colimit(\const F(S^0)\colon X\rightarrow \Cc)$ exists, and then so does the pushout.
	
	Showing that $\Lan_iF(X,x)$ is indeed given by the pushout in question is essentially a lengthy unravelling of the Kan extension formula from \cref{lem:KanExtensionFormula}. We've seen in the proof of \cref{lem:FunExcIsSp} that $\Ii\simeq \cat{Fin}_{\leqslant 1}$. Under this equivalence, $\cat{Fin}_{\leqslant1}^\circ$ corresponds to the non-full sub-$\infty$-category $\Jj\coloneqq \{\begin{tikzcd}[cramped, column sep=small,ampersand replacement=\&]*\&\lar S^0\end{tikzcd}\}$ of $\FinAn$. Let $j\colon \Jj\rightarrow \FinAn$ be the inclusion of $\Jj$. By claim~\cref{claim:LeftKanEasier} in the proof of \cref{lem:CGrpIsC} we may replace $\Ii$ by $\Jj$ and analyse the left Kan extension $\Lan_jF$ of a reduced functor $F\colon \Jj\rightarrow \FinAn$ instead. This will make our life much easier.
	
	Fix some pointed anima $(X,x)$ and consider the slice $\infty$-category
	\begin{equation*}
		\Yy\coloneqq \Jj\times_{\FinAn}\bigl(\FinAn\bigr)_{/(X,x)}
	\end{equation*}
	together with its usual slice projection $s\colon \Yy\rightarrow \Jj$. The Kan extension formula from \cref{lem:KanExtensionFormula} asserts that $\Lan_jF(X,x)\simeq \colimit(F\circ s\colon\Yy\rightarrow \Cc)$, provided this colimits exists. So let's analyse the $\infty$-category $\Yy$. The objects of $\Yy$ come in two flavours: First there are pointed morphisms $*\rightarrow (X,x)$, of which there's only one, which by abuse of notation we'll also denote $*$. Second, there are pointed morphisms $S^0\rightarrow (X,x)$. Every such morphism is uniquely given by where it sends the non-basepoint, and we let $y\colon S^0\rightarrow (X,x)$ denote the morphism that sends the non-basepoint to $y\in X$. Next, let's compute morphism animae. For $y,z\in X$, we can use \cref{cor:HomInSliceCategories} and \cref{lem:HomInLimits}\cref{enum:HomInLimits} to see that $\Hom_\Yy(y,z)$ sits in a pullback square
	\begin{equation*}
		\begin{tikzcd}
			\Hom_\Yy(y,z)\dar\rar\drar[pullback] & \{y\}\dar\\
			\Hom_{\Jj}\bigl(S^0,S^0\bigr)\rar["z_*"] &  \Hom_{\cat{An}_{*/}}\bigl(S^0,(X,x)\bigr)
		\end{tikzcd}
	\end{equation*}
	Since $\Hom_\Jj(S^0,S^0)\simeq \id_{S^0}$ and $\Hom_{\cat{An}_{*/}}(S^0,(X,x))\simeq \Hom_{\cat{An}}(*,X)\simeq X$, this pullback can be identified with $\{y\}\times_X\{z\}$, and then an argument as in \cref{lem:SuspensionLoopAdjunction}\cref{enum:LoopIsHom} shows
	\begin{equation*}
		\Hom_\Yy(y,z)\simeq\Hom_X(y,z)\,.
	\end{equation*}
	In a similar way, we obtain $\Hom_\Yy(y,*)\simeq \Hom_X(y,x)$ as well as $\Hom_\Yy(*,*)\simeq *$ and $\Hom_\Yy(*,z)\simeq \emptyset$. This finishes our description of $\Yy$. 
	
	Now let $\Xx$ be the pushout in $\cat{Cat}_\infty$ of $\{x\}\rightarrow X$ along $\{x\}\simeq \{0\}\rightarrow \Delta^1$. We wish to construct a functor $\vartheta\colon \Xx\rightarrow \Yy$ and then to show that $\vartheta$ is an equivalence of $\infty$-categories. To this end, first consider the functor
	\begin{equation*}
		\varphi\colon X\simeq \{S^0\}\times_{\FinAn}\bigl(\FinAn\bigr)_{/(X,x)}\longrightarrow \Jj\times_{\FinAn}\bigl(\FinAn\bigr)_{/(X,x)}\simeq \Yy
	\end{equation*}
	(the equivalence on the left follows from the fact that the right fibration $\bigl(\FinAn\bigr)\vphantom{)}^\mathrm{\phantom{fin}}_{/(X,x)}\rightarrow \FinAn$ parametrises the functor $\Hom_{\FinAn}(-,(X,x))\colon \bigl(\FinAn\bigr)^{\smash{\op}}\rightarrow \cat{An}$ and so its fibre over $S^0$ is given by $\Hom_{\cat{An}_{*/}}(S^0,X)\simeq X$). Secondly, consider the functor $\psi\colon \Delta^1\rightarrow \Yy$ corresponding to the morphism $\psi\colon x\rightarrow *$ in $\Yy$ which in turn corresponds to $\id_x\in \Hom_X(x,x)\simeq \Hom_\Yy(x,*)$. By construction, $\varphi|_{\{x\}}\simeq \psi|_{\{0\}}$ and so by the universal property of pushouts, $\varphi$ and $\psi$ together determine a functor $\vartheta\colon \Xx\rightarrow \Yy$.\footnote{More precisely, every \emph{choice} of an equivalence $\varphi|_{\{x\}}\simeq \psi|_{\{0\}}$ determines a natural transformation between the span $X \leftarrow \{x\}\simeq \{0\}\rightarrow \Delta^1$ and the span $\const \Yy$ in $\Fun(\Lambda_0^2,\cat{Cat}_\infty)$. And every such transformation determines a viable $\vartheta$ by the universal property of colimits.}
	
	If we can show that $\vartheta$ is an equivalence, we're done. Indeed, using \cref{lem:ColimitManipulations}\cref{claim:AssembleColimits} and our assumption $F(*)\simeq 0$, we see that $\colimit(F\circ s\circ \vartheta\colon \Xx\rightarrow \Cc)$ is precisely the pushout we're looking for! It's obvious that $\vartheta$ is essentially surjective, so we only need to prove that $\vartheta$ is fully faithful, and for that, we must understand $\Hom$ animae in $\Xx$. In general, there's no nice way to describe $\Hom$ in a pushout, but here we can use a trick: The inclusion $\iota\colon X\rightarrow \Xx$ is left adjoint to the functor $r\colon \Xx\simeq X\sqcup_{\{x\}}\Delta^1\rightarrow X\sqcup_{\{x\}}\{x\}\simeq X$ defined by $\Delta^1\rightarrow \{x\}$! To see this, first note that $\{0\}\shortdoublelrmorphism \Delta^1$ is an adjunction (which is obvious, as these are ordinary categories), and recall from \cref{lem:TriangleIdentities} that to construct an adjunction, it's enough to construct unit and counit as well as the triangle identities. Since $-\times\Delta^1\colon \cat{Cat}_\infty\rightarrow\cat{Cat}_\infty$ commutes with pushouts, as it is a left adjoint by \cref{exm:Adjunctions}\cref{enum:Currying}, we can construct the counit $c\colon \id_\Xx\rightarrow r\circ \iota$ by taking the pushout of the identity transformation on $X$ with the counit of the adjunction $\{0\}\shortdoublelrmorphism \Delta^1$. In the same way, we can construct the unit, and then the triangle identities will still be satisfied.
	
	Using this adjunction, we see that $\vartheta$ induces equivalences $\Hom_\Xx(y,z)\simeq \Hom_\Yy(y,z)$ for all $y,z\in X$. Furthermore, if $*\in \Xx$ denotes the image of $1\in \Delta^1$, then $\vartheta(*)\simeq *$ and we have $\Hom_\Xx(y,*)\simeq \Hom_X(y,r(*))\simeq \Hom_X(y,x)$, so $\Hom_\Xx(y,*)\simeq \Hom_\Yy(y,*)$ for all $y\in X$. Finally, we have $\Hom_\Xx(*,*)\simeq *$ and $\Hom_\Xx(*,z)\simeq \emptyset$ for all $z\in X$. For the latter, simply note that $X\rightarrow\{x\}$ defines a functor $\Xx\simeq X\sqcup_{\{x\}}\Delta^1\rightarrow \{x\}\sqcup_{\{x\}}\Delta^1\simeq \Delta^1$ and then there's a morphism $\Hom_\Xx(*,z)\rightarrow \Hom_{\Delta^1}(1,0)\simeq \emptyset$. For the former, we use model category fact~\cref{par:HomotopyPushout}: $\Xx$ is given by choosing an inner anodyne map of the pushout $X\sqcup_{\{x\}}\Delta^1$ in $\cat{sSet}$ into a quasi-category. If we use the recipe from the proof of \cref{lem:SmallObjectArgument}, we won't ever add any simplex whose vertices are all $*$, hence $\Hom_\Xx(*,*)\simeq \Hom_{\Delta^1}(1,1)$. Alternatively, for a model-independent argument, one could use \cref{lem:ColimitsInAnima} and a general formula for $\Hom$ in localisations, but this is much more difficult. This shows that $\vartheta$ is fully faithful and we're done!
\end{proof}
%	\begin{proof}
	%		Throughout the proof, we fix some pointed anima $(X,x)$. First note that the pushout above exists in $\Cc$. Indeed, since $\Cc$ is stable, it has all finite colimits by \cref{lem:Stable}\cref{enum:PushoutPullback}. In particular, since $X$ is a finite anima, $\colimit(\const F(S^0)\colon X\rightarrow \Cc)$ exists, and then so does the pushout. Now let
	%		\begin{equation*}
		%			\Yy\coloneqq \{\InlineS\}\times_{\FinAn}\bigl(\FinAn\bigr)_{/(X,x)}
		%		\end{equation*}
	%		and let $s\colon \Yy\rightarrow \{\InlineS\}$ denote usual slice $\infty$-category projection. Recall the Kan extension formula from \cref{lem:KanExtensionFormula}, asserting that $\Lan_iF(X,x)\simeq \colimit(F\circ s\colon\Yy\rightarrow \Cc)$, provided this colimits exists. Our goal now is to massage that colimit until it agrees with the pushout above; this will show that $\Lan_i F$ exists and that its values are given by said pushout.
	%		
	%		So let's analyse $\Yy$. The objects of $\Yy$ come in two kinds: First there are morphisms $*\rightarrow (X,x)$, of which there's only one, which by abuse of notation we'll also denote $*$. Second, there are morphisms $S^0\rightarrow (X,x)$. Every such morphism is uniquely given by where it sends the non-basepoint, and we let $y\colon S^0\rightarrow (X,x)$ denote the morphism that sends the non-basepoint to $y\in X$. Next, let's compute morphism animae. For $y,z\in X$, we can use \cref{cor:HomInSliceCategories} and \cref{lem:HomInLimits} to see that $\Hom_\Yy(y,z)$ sits in a pullback square
	%		\begin{equation*}
		%			\begin{tikzcd}
			%				\Hom_\Yy(y,z)\dar\rar\drar[pullback] & \{y\}\dar\\
			%				\Hom_{\cat{An}_{*/}}\bigl(S^0,S^0\bigr)\rar["z_*"] &  \Hom_{\cat{An}_{*/}}\bigl(S^0,(X,x)\bigr)
			%			\end{tikzcd}
		%		\end{equation*}
	%		Now observe that $\Hom_{\cat{An}_{*/}}(S^0,S^0)\simeq \{\id_{S^0},\varepsilon\}$ is a discrete anima on two elements, where $\varepsilon\colon S^0\rightarrow S^0$ is the morphism that sends everything to the basepoint. Furthermore, we have $\Hom_{\cat{An}_{*/}}(S^0,(X,x))\simeq \Hom_{\cat{An}}(*,X)\simeq X$. Plugging this into the pullback and using \cref{lem:SuspensionLoopAdjunction}\cref{enum:LoopIsHom}, we obtain an equivalence
	%		\begin{equation*}
		%			\Hom_\Yy\left(y,z\right)\simeq \Hom_X\left(y,z\right)\sqcup \Hom_X\left(y,x\right)\,.
		%		\end{equation*}
	%		We call a morphism $\alpha\colon y\rightarrow z$ in $\Yy$ \emph{good} if it is contained in the first component of the disjoint union, that is, if $\alpha$ lies over $\id_{S^0}$, and we call $\alpha$ \emph{evil} if it is contained in the second component, that is, if $\alpha$ lies over $\varepsilon$. Finally, we let $\Hom_\Yy(y,z)\simeq \Hom_\Yy^\mathrm{good}(y,z)\sqcup \Hom_\Yy^\mathrm{evil}(y,z)$ denote the decomposition above.
	%		
	%		In a similar way, using $\Hom_{\cat{An}_{*/}}(S^0,*)\simeq *$, we see $\Hom_\Yy(y,*)\simeq \Hom_X(y,x)$ for all $y\in X$, and since $\Hom_{\cat{An}_{*/}}(*,(X,x))\simeq *$, we obtain $\Hom_\Yy(*,z)\simeq *\simeq\Hom_\Yy(*,*)$ for all $z\in X$. This finishes our description of $\Yy$. 
	%		
	%		Now let $\Xx$ be the pushout of $\{x\}\rightarrow X$ along $\{x\}\simeq \{0\}\rightarrow \Delta^1$, taken in $\cat{Cat}_\infty$. We wish to construct a functor $\omega\colon \Xx\rightarrow \Yy$. To this end, first consider the functor
	%		\begin{equation*}
		%			\varphi\colon X\simeq \{S^0\}\times_{\FinAn}\bigl(\FinAn\bigr)_{/(X,x)}\longrightarrow \{\InlineS\}\times_{\FinAn}\bigl(\FinAn\bigr)_{/(X,x)}\simeq \Yy
		%		\end{equation*}
	%		(the equivalence on the left follows from the fact that the right fibration $\bigl(\FinAn\bigr)\vphantom{)}^\mathrm{\phantom{fin}}_{/(X,x)}\rightarrow \FinAn$ parametrises the functor $\Hom_{\FinAn}(-,(X,x))\colon \bigl(\FinAn\bigr)^{\smash{\op}}\rightarrow \cat{An}$ and so its fibre over $S^0$ is given by $\Hom_{\cat{An}_{*/}}(S^0,X)\simeq X$). Secondly, consider the functor $\psi\colon \Delta^1\rightarrow \Yy$ corresponding to the morphism $\psi\colon x\rightarrow *$ in $\Yy$ which in turn corresponds to $\id_x\in \Hom_X(x,x)\simeq \Hom_\Yy(x,*)$. By the universal property of pushouts, $\varphi$ and $\psi$ together determine a functor $\omega\colon \Xx\rightarrow \Yy$, as desired. Using \cref{lem:ColimitManipulations}\cref{claim:AssembleColimits} and our assumption $F(*)\simeq 0$, we see that $\colimit(F\circ s\circ \omega\colon \Xx\rightarrow \Cc)$ is precisely the pushout we're looking for! So it suffices to show:
	%		\begin{alphanumerate}\itshape
		%			\item[\boxtimes_1] $F\circ s\colon \Yy\rightarrow \Cc$ is the left Kan extension of $F\circ s\circ \omega\colon \Xx\rightarrow \Cc$ along $\omega$. More precisely, the canonical \embrace{counit} transformation $c_{F\circ s}\colon \Lan_\omega(F\circ s\circ \omega)\Rightarrow F\circ s$ is an equivalence.\label{claim:LeftKanExtension}
		%		\end{alphanumerate}
	%		If we know \cref{claim:LeftKanExtension}, then
	%		\begin{equation*}
		%			\colimit\left(F\circ s\circ \omega\colon \Xx\rightarrow \Cc\right)\simeq \colimit\left(F\circ s\colon \Yy\rightarrow \Cc\right)
		%		\end{equation*}
	%		will hold for formal reasons (indeed, taking colimits is itself a left Kan extension, namely along $\Xx\rightarrow 
	%		*$ or $\Yy\rightarrow *$, respectively, and left Kan extensions compose).
	%		
	%		To prove \cref{claim:LeftKanExtension}, we must understand $\Hom$ animae in $\Xx$. In general, there's no nice way to describe $\Hom$ in a pushout, but here we can use a trick: The inclusion $j\colon X\rightarrow \Xx$ is left adjoint to the functor $r\colon \Xx\rightarrow X$ defined by $\Delta^1\rightarrow \{x\}$! To see this, first note that $\{0\}\shortdoublelrmorphism \Delta^1$ is an adjunction (which is obvious, as these are ordinary categories), and recall from \cref{lem:TriangleIdentities} that to construct an adjunction, it's enough to construct unit and counit as well as the triangle identities. Since $-\times\Delta^1\colon \cat{Cat}_\infty\rightarrow\cat{Cat}_\infty$ commutes with pushouts (as it admits a right adjoint, namely $\Fun(\Delta^1,-)$), we can construct the counit $c\colon \id_\Xx\rightarrow r\circ j$ by taking the pushout of the identity transformation on $\Xx$ with the counit of the adjunction $\{0\}\shortdoublelrmorphism \Delta^1$. In the same way, we can construct the unit, and then the triangle identities will obviously still be satisfied.
	%		
	%		Using this adjunction, we see that $\Hom_\Xx(y,z)\simeq \Hom_X(y,z)\rightarrow \Hom_\Yy(y,z)$ is an equivalence onto the component $\Hom_\Yy^\mathrm{good}(y,z)$. Furthermore, if $*\in \Xx$ denotes the image of $1\in \Delta^1$, then $\omega(*)\simeq *$ and we have $\Hom_\Xx(y,*)\simeq \Hom_X(y,r(*))\simeq \Hom_X(y,x)$, so $\Hom_\Xx(y,*)\rightarrow \Hom_\Yy(y,*)$ is an equivalence for all $y\in X$. Finally, we have $\Hom_\Xx(*,*)\simeq *$ and $\Hom_\Xx(*,z)\simeq \emptyset$ for all $z\in X$. For the latter, simply note that $X\rightarrow\{x\}$ defines a functor $\Xx\rightarrow \Delta^1$ and then there's a morphism $\Hom_\Xx(*,z)\rightarrow \Hom_{\Delta^1}(1,0)\simeq \emptyset$. For the former, we use model category fact~\cref{par:HomotopyPushout}: $\Xx$ is given by choosing an inner anodyne map of the pushout $X\sqcup_{\{x\}}\Delta^1$ in $\cat{sSet}$ into a quasicategory. If we use the recipe from the proof of \cref{lem:SmallObjectArgument}, we won't ever add any simplex whose vertices are all $*$, hence $\Hom_\Xx(*,*)\simeq \Hom_{\Delta^1}(1,1)$. Alternatively, for a model-independent argument, one could use \cref{lem:ColimitsInAnima} and a general formula for $\Hom$ in localisations, but this is much more difficult.
	%		
	%		To show that the canonical transformation $c_{F\circ s}\colon \Lan_\omega(F\circ s\circ \omega)\Rightarrow F\circ s$ is an equivalence (and that the left-hand side even exists), we use the formula from \cref{lem:KanExtensionFormula} again: We need to show that the following morphisms are equivalences:
	%		\begin{align*}
		%			\colimit\left(\Xx\times_\Yy\Yy_{/z}\rightarrow \Yy\xrightarrow{F\circ s} \Cc\right)&\longrightarrow F\left(s(z)\right)\simeq F\bigl(S^0\bigr)\quad\text{for all $z\in \Yy$}\,,\\
		%			\colimit\left(\Xx\times_\Yy\Yy_{/*}\rightarrow \Yy\xrightarrow{F\circ s}\Cc\right)&\longrightarrow F\left(s(*)\right)\simeq F(*)\,.
		%		\end{align*}
	%		The second one is easy: Our description of $\Xx$ shows that $\Xx_{/*}\rightarrow \Xx\times_\Yy\Yy_{/*}$ is fully faithful and essentially surjective, hence an equivalence by \cref{thm:EquivalenceFullyFaithfulEssentiallySurjective}. So the colimit on the left-hand side is given by evaluating at the terminal object $(\id_*\colon*\rightarrow *)\in\Xx_{/*}$. For the other colimit, we claim:
	%		\begin{alphanumerate}\itshape
		%			\item[\boxtimes_2] Let $\Tt^\mathrm{good}\subseteq \Xx\times_\Yy\Yy_{/z}$ be the full sub-$\infty$-category spanned by the good morphisms $\alpha\colon y\rightarrow z$, and let $\Tt^\mathrm{evil}\subseteq \Xx\times_\Yy\Yy_{/z}$ be spanned by the evil morphisms $\beta\colon y\rightarrow z$ as well as the unique morphism $*\rightarrow z$. Then $\Xx_{/z}\rightarrow \Xx\rightarrow \Xx\times_\Yy\Yy_{/z}$ is fully faithful, with essential image $\Tt^\mathrm{good}$. Furthermore,\label{claim:GoodEvilDecomposition}
		%			\begin{equation*}
			%				\Xx\times_\Yy\Yy_{/z}\simeq \Tt^\mathrm{good}\sqcup \Tt^\mathrm{evil}\,,
			%			\end{equation*}
		%			and $(*\rightarrow z)\in\Tt^\mathrm{evil}$ is a terminal object.
		%		\end{alphanumerate}
	%		Using our explicit description of $\Xx$, it's straightforward to see that $\Xx_{/z}\rightarrow \Xx\rightarrow \Xx\times_\Yy\Yy_{/z}$ is fully faithful, with essential image $\Tt^\mathrm{good}$. Moreover, if $\alpha\colon y\rightarrow z$ is good and $\beta\colon y'\rightarrow z$ is evil, then \cref{cor:HomInSliceCategories} combined with \cref{lem:HomInLimits} shows
	%		\begin{align*}
		%			\Hom_{\Xx\times_\Yy\Yy_{/z}}\left((\alpha\colon y\rightarrow z),(\beta\colon y'\rightarrow z)\right)&\simeq \Hom_\Yy^\mathrm{good}(y,y')\times_{\Hom_\Yy(y,z)}\{\alpha\}\simeq \emptyset\,,\\
		%			\Hom_{\Xx\times_\Yy\Yy_{/z}}\left((\beta\colon y'\rightarrow z),(\alpha\colon y\rightarrow z)\right)&\simeq \Hom_\Yy^\mathrm{good}(y',y)\times_{\Hom_\Yy(y',z)}\{\beta\}\simeq \emptyset\,.
		%		\end{align*}
	%		Indeed, in the first case, postcomposition with the evil morphism $\beta\colon y'\rightarrow z$ restricts to $\beta_*\colon \Hom_\Yy^{\mathrm{good}}(y,y')\rightarrow \Hom_\Yy^\mathrm{evil}(y,z)$, and so the first pullback is empty. Similarly, composition with the good morphism $\alpha\colon y\rightarrow z$ restricts to $\alpha_*\colon \Hom_\Yy^\mathrm{good}(y',y)\rightarrow \Hom_\Yy^\mathrm{good}(y',z)$ and so the second pullback is empty. Finally, for every $y''\in \Yy$, postcomposition with the unique morphism $*\rightarrow z$ defines an equivalence $\Hom_\Yy(y'',*)\simeq \Hom_\Yy^\mathrm{evil}(y'',z)$. Hence for every morphism $\gamma\colon y''\rightarrow z$, we have that
	%		\begin{equation*}
		%			\Hom_{\Xx\times_\Yy\Yy_{/z}}\left((\gamma\colon y''\rightarrow z),(*\rightarrow z)\right)\simeq \Hom_\Yy(y'',*)\times_{\Hom_\Yy(y'',z)}\{\gamma\}
		%		\end{equation*}
	%		is $\simeq \emptyset$ if $\gamma$ is good, and $\simeq *$ if $\gamma$ is evil.
	%		This shows at once that $\Xx\times_\Yy\Yy_{/z}\simeq \Tt^\mathrm{good}\sqcup \Tt^\mathrm{evil}$ and that $(*\rightarrow z)\in \Tt^\mathrm{evil}$ is terminal. So we've proved \cref{claim:GoodEvilDecomposition}
	%		
	%		Using \cref{lem:ColimitManipulations}\cref{claim:AssembleColimits}, we see that the second colimit above is the coproduct of the colimits over $\Tt^\mathrm{good}$ and $\Tt^\mathrm{evil}$, respectively. The colimit over $\Tt^\mathrm{good}\simeq \Xx_{/z}$ is given by evaluation at the final object $(\id_z\colon z\rightarrow z)\in  \Xx_{/z}$, hence by $F(s(z))\simeq F(S^0)$. The colimit over $\Tt^\mathrm{evil}$ is given by evaluation at the terminal object $(*\rightarrow z)$, hence by $F(*)\simeq 0$. We deduce that the desired colimit is given by $F(S^0)\sqcup 0\simeq F(S^0)$, and so we're finally done.
	%	\end{proof}
This finishes the proof that $\cat{Sp}(\Cc)\simeq\Fun_*^\mathrm{exc}(\FinAn,\Cc_{*/})$. Now we'll use this alternative description to define a left adjoint of $\Omega^\infty\colon \cat{Sp}\rightarrow \cat{An}$ and to construct the sphere spectrum $\IS$.
\begin{lem}\label{lem:Spectrification}
	Let $\Cc$ be an $\infty$-category with finite limits; in particular, $\Cc$ has a terminal object $*\in\Cc$. Assume furthermore that $\Cc_{*/}$ admits sequential colimits and that $\Omega_\Cc\colon \Cc_{*/}\rightarrow \Cc_{*/}$ commutes with them. Then $\Fun_*^\mathrm{exc}(\FinAn,\Cc_{*/})\subseteq \Fun_*(\FinAn,\Cc_{*/})$ has a left adjoint, which sends a reduced functor $F\colon \FinAn\rightarrow \Cc_{*/}$ to
	\begin{equation*}
		F^\mathrm{sp}\coloneqq \colimit_{n\geqslant 0}\Omega_\Cc^nF\bigl(\Sigma^n(-)\bigr)
	\end{equation*}
	\embrace{in the proof of \cref{lem:FunExcStable} we've constructed a transformation $F\Rightarrow \Omega_\Cc F(\Sigma(-))$; the colimit on the right-hand side is given by iterating this construction}.
\end{lem}
To prove \cref{lem:Spectrification}, we need a general lemma about adjunctions:
\begin{lem}\label{lem:FormalInclusionAdjunction}
	Let $L\colon \Cc\rightarrow \Cc$ be an endofunctor of an $\infty$-category and $u\colon \id_\Cc\Rightarrow L$ be a natural transformation. Suppose that both $Lu\colon L\Rightarrow L\circ L$ and $uL\colon L\Rightarrow L\circ L$ are equivalences. Then, if $i\colon \Cc_L\rightarrow\Cc$ denotes the inclusion of the full sub-$\infty$-category spanned by the essential image of $L$, we have an adjunction
	\begin{equation*}
		L\colon \Cc\doublelrmorphism \Cc_L\noloc i\,.
	\end{equation*}
\end{lem}
\begin{proof}
	By \cref{lem:TriangleIdentities}, it's enough to construct the unit as well as the counit and to verify the triangle identities. This will be so tautological that it becomes confusing again. As the notation suggests, we take $u$ to be our unit. Restricting $u$ along $i\colon \Cc_L\rightarrow \Cc$ defines a natural transformation $ui\colon i\Rightarrow L\circ i$ in $\Fun(\Cc_L,\Cc)$. By assumption, $u_{L(x)}\colon L(x)\rightarrow L(L(x))$ is an equivalence for all $x\in \Cc$. This shows that $ui$ is a pointwise equivalence, hence it admits an inverse by \cref{thm:EquivalencePointwise}. Furthermore $ui$ takes values in $\Cc_L$, so we can regard it as a natural transformation $\id_{\Cc_L}\Rightarrow L\circ i$ in $\Fun(\Cc_L,\Cc_L)$. Its inverse can then also be regarded as a natural transformation $c\colon L\circ i\Rightarrow \id_{\Cc_L}$ in $\Fun(\Cc_L,\Cc_L)$. This will be our counit.
	
	Let's now verify the triangle identities. The second one from \cref{lem:TriangleIdentities} is trivially satisfied, since, by construction, $ic$ is an inverse of $ui$ and so $ic\circ ui\simeq \id_i$. For the first triangle identity (in its weak form, where we only require $cL\circ uL$ to be an equivalence), we use that $Lu\colon L\Rightarrow L\circ L=L\circ i\circ L$ is an equivalence by assumption, so we only need to check that $cL$ is an equivalence. But $c$ itself is, by construction, already an equivalence.
\end{proof}
\begin{proof}[Proof sketch of \cref{lem:Spectrification}]
	It's clear that the construction of $F^\mathrm{sp}$ can be made into an endofunctor $(-)^\mathrm{sp}\colon \Fun_*(\FinAn,\Cc_{*/})\rightarrow \Fun_*(\FinAn,\Cc_{*/})$. By construction, for every $F$ there is a natural transformation $u_F\colon F\Rightarrow F^\mathrm{sp}$ in $\Fun_*(\FinAn,\Cc_{*/})$. This is clearly natural in $F$ as well, hence defines a natural transformation $\id_{\Fun_*(\FinAn,\Cc_{*/})}\Rightarrow (-)^\mathrm{sp}$. We'll verify the conditions from \cref{lem:FormalInclusionAdjunction} and show that the image of $(-)^\mathrm{sp}$ are precisely the reduced and excisive functors.
	
	Let's start with the first condition: To show that $u^\mathrm{sp}\colon (-)^\mathrm{sp}\Rightarrow ((-)^\mathrm{sp})^\mathrm{sp}$ is an equivalence, we must show that
	\begin{equation*}
		u_F^\mathrm{sp}\colon \colimit_{m\geqslant 0}\Omega_\Cc^m F\bigl(\Sigma^m(-)\bigr)\overset{\simeq}{\Longrightarrow} \colimit_{m\geqslant 0}\colimit_{n\geqslant 0}\Omega_\Cc^{m+n}F\bigl(\Sigma^{m+n}(-)\bigr)
	\end{equation*}
	is an equivalence for all $F$. This follows from a formal manipulation of colimits.
	
	To show the second condition, observe that if $F$ is already reduced and excisive, then $F\Rightarrow \Omega_\Cc F(\Sigma(-))$ is an equivalence, and so $u_F\colon F\Rightarrow F^\mathrm{sp}$ must be an equivalence too. Thus, to show that $u(-)^\mathrm{sp}\colon (-)^\mathrm{sp}\Rightarrow ((-)^\mathrm{sp})^\mathrm{sp}$ is a pointwise equivalence, it's enough to check that $(-)^\mathrm{sp}$ takes values in reduced and excisive functors. This has to be done anyway, since we have to identify the image of $(-)^\mathrm{sp}$. Also, our observation  that $u_F$ is an equivalence whenever $F$ is reduced and excisive already shows that the essential image of $(-)^\mathrm{sp}$ contains all reduced and excisive functors. Thus, once we show that the $F^\mathrm{sp}$ is reduced and excisive, we'll be done.
	
	To show this, it's clear that $F^\mathrm{sp}$ is reduced again. For excisivity, observe $F^\mathrm{sp}\simeq \Omega_\Cc F^\mathrm{sp}(\Sigma(-))$. Indeed, precomposition with $\Sigma$ commutes with all colimits, and postcomposition with $\Omega_\Cc$ commutes with sequential colimits by our assumption, so the colimit defining $F^\mathrm{sp}$ just gets transformed into itself. Now consider an arbitrary pushout diagram in $\FinAn$ and extend it as follows:
	\begin{equation*}
		\begin{tikzcd}
			A\rar\dar\drar[pushout] & C\dar\rar\drar[pushout] & *\dar & \\
			B\rar\dar\drar[pushout] & D\rar\dar\drar[pushout] & Q\rar\dar\drar[pushout] & *\dar\\
			*\rar & P\rar\dar\drar[pushout] & \Sigma A\rar\dar\drar[pushout] & \Sigma B\dar\\
			& *\rar & \Sigma C\rar & \Sigma D
		\end{tikzcd}
	\end{equation*}
	The top left $2\times 2$-square induces a morphism $F^\mathrm{sp}(B)\times_{F^\mathrm{sp}(D)}F^\mathrm{sp}(C)\rightarrow \Omega_\Cc F^\mathrm{sp}(\Sigma A)\simeq F^\mathrm{sp}(A)$. It's straightforward to check that this morphism is an inverse to the canonical morphism in the other direction. This proves that $F^\mathrm{sp}$ turns pushouts into pullbacks, as required.
\end{proof}
\begin{cor}\label{cor:SigmaInfty}
	The functor $\Omega^\infty\colon \cat{Sp}\rightarrow \cat{An}_{*/}$ admits a left adjoint $\Sigma^\infty\colon \cat{An}_{*/}\rightarrow \cat{Sp}$. If $(X,x)$ is a pointed anima, then $\Omega^\infty\Sigma^\infty(X,x)\simeq \colimit_{n\geqslant 0}\Omega^n\Sigma^nX$ together with its basepoint $x$. In particular,
	\begin{equation*}
		\pi_*\Sigma^\infty(X,x)\cong \colimit_{n\geqslant 0}\pi_{*}(\Omega^n\Sigma^nX)\cong \colimit_{n\geqslant 0}\pi_{*+n}(\Sigma^n X)
	\end{equation*}
	are the stable homotopy groups of $X$.
\end{cor}
\begin{proof}
	To prove that $\Sigma^\infty$ exists and is given as above, let $\Ii\coloneqq \{\InlineS\}$ be the full sub-$\infty$-category of $\FinAn$ spanned by $*$ and $S^0$ and recall the chain of equivalences and adjunctions
	\begin{equation*}
		\cat{An}_{*/}\xleftarrow[\ev_{S_0}]{\simeq}\Fun_*\left(\Ii,\cat{An}_{*/}\right)\overset{\Lan_i}{\underset{i^*}{\doublelrmorphism}} \Fun_*\left(\FinAn,\cat{An}_{*/}\right)
	\end{equation*}
	from the proof of \cref{lem:FunExcIsSp}. Thus, $\ev_{S_0}\colon \Fun_*(\FinAn,\cat{An}_{*/})\rightarrow \cat{An}_{*/}$ has a left adjoint. Furthermore, according to \cref{lem:FunExcIsSp,lem:Spectrification}, $\cat{Sp}\simeq \Fun_*^\mathrm{exc}(\FinAn,\cat{An}_{*/})\subseteq \Fun_*(\FinAn,\cat{An}_{*/})$ has a left adjoint too. This shows that $\Sigma^\infty$ exists.
	
	To show the desired formula for $\Sigma^\infty$, fix a pointed anima $(Y,y)$ and let $-\wedge Y\colon \FinAn\rightarrow \cat{An}_{*/}$ denote the associated functor. Let $(X,x)$ be a finite pointed anima; we wish to compute the value $X\wedge Y$ of $-\wedge Y$ on $(X,x)$. It will turn out that $X\wedge Y$ agrees with the smash product you know from topology, so the suggestive notation is justified. But for the moment, let's forget what we know about smash products and regard $X\wedge Y$ as the value of our functor. We use \cref{lem:Smash} to compute it. By definition, $S^0\wedge Y\simeq Y$. The colimit of the constant functor $\const S^0\wedge Y\colon X\rightarrow \cat{An}$ is therefore $X\times Y$ by \cref{lem:ColimitsInAnima}. If we take the colimit of $\const S^0\wedge Y\colon X\rightarrow \cat{An}_{*/}$ in pointed animae, we get $(X\times Y)/(X\times \{y\})$ instead, see \cref{lem:ColimitsInSliceCategory}\cref{enum:ColimitsInSliceGeneral}. Plugging this into \cref{lem:Smash}, we get a pushout diagram
	\begin{equation*}
		\begin{tikzcd}
			X\times\{y\}\rar\dar\drar[pushout] & (X\times Y)/\bigl(X\times \{y\}\bigr)\dar\\
			*\rar & X\wedge Y
		\end{tikzcd}
	\end{equation*}
	(in $\cat{An}$ or $\cat{An}_{*/}$, this doesn't matter by \cref{lem:ColimitsInSliceCategory}\cref{enum:ColimitsInSlice}). So $X\wedge Y$ is indeed the usual smash product from topology.%
	%
	\footnote{It's easy to turn the usual definition from topology into a functor $-\wedge Y\colon \cat{An}_{*/}\rightarrow \cat{An}_{*/}$ (more on that in [TODO]). This functor agrees with the functor we've constructed above. So far, we only know this on objects, but the equivalence as functors is not hard to check. Since both definitions of $S^0\smash X$ agree, both functors must agree in $\Fun_*(\Ii,\cat{An}_{*/})$. From the universal property of left Kan extension, we then get a natural transformation between them for free. So knowing that they agree object-wise is enough by \cref{thm:EquivalencePointwise}.}
	
	Now $\Omega^\infty\Sigma^\infty(Y,y)$ can be described as the value of $(-\wedge Y)^\mathrm{sp}$ on $S^0$. According to the formula from \cref{lem:Spectrification}, this value is given by
	\begin{equation*}
		\colimit_{n\geqslant 0}\Omega^n(\Sigma^n S^0\wedge Y)\simeq \colimit_{n\geqslant 0}\Omega^n(S^n\wedge Y)\simeq \colimit_{n\geqslant 0}\Omega^n\Sigma^nY\,.
	\end{equation*}
	It remains to show the \enquote{in particular} about the homotopy groups of the spectrum $\Sigma^\infty(Y,y)$. The same argument as above shows that $\Omega^{\infty-i}\Sigma^\infty(Y,y)$ is given by the value of $(-\wedge Y)^\mathrm{sp}$ on $S^i$, which is $\colimit_{n\geqslant 0}\Omega^n(\Sigma^{n}S^i\wedge Y)\simeq \colimit_{n\geqslant 0}\Omega^n\Sigma^{n+i}Y$. Hence \cref{lem:HomotopyGroupsFilteredColimits,lem:SuspensionLoopAdjunction}\cref{enum:LoopShiftsHomotopyGroups} show $\pi_*\Sigma^\infty(Y,y)\cong \pi_0(\Omega^{\infty-*}\Sigma^\infty(Y,y))\cong \colimit_{n\geqslant 0}\pi_{*+n}(\Sigma^n X)$, as desired. 
\end{proof}
As an immediate consequence, we get an analogue of \cref{cor:FreeE1Group} for $\IE_\infty$-groups.
\begin{cor}[\enquote{$\Omega^\infty\Sigma^\infty X_+$ is the free $\IE_\infty$-group on $X$}]\label{cor:FreeEInftyGroup}
	The forgetful functor $\ev_{\langle 1\rangle}\colon\cat{CGrp}(\cat{An})\rightarrow \cat{An}$ sending an $\IE_\infty$-group to its underlying anima has a left adjoint, sending an anima $X$ to $\Omega^\infty\Sigma^\infty X_+$, where $X_+\coloneqq X\sqcup *$, regarded as a pointed anima.
\end{cor}
\begin{proof}
	Since $\pi_*\Sigma^\infty(X,x)$ is given by the stable homotopy groups of $X$, $\Sigma^\infty$ takes values in the full sub-$\infty$-category $\cat{Sp}_{\geqslant 0}$ of connective spectra. Therefore, we get a diagram of adjunctions
	\begin{equation*}
		\begin{tikzcd}[column sep=large]
			\cat{An}\rar[shift left=0.2em,"{(-)_+}"]\ar[drr,bend right=15.5,shorten <=0.4ex,shorten >=0.1ex,shift left=0.2em,"{\Omega^\infty\Sigma^\infty(-)_+}"{pos=0.45},start anchor=300,end anchor=175] & \lar[shift left=0.2em]\cat{An}_{*/} \rar[shift left=0.2em,"\Sigma^\infty"]\drar[commutes,pos=0.45,xshift=1em] & \lar[shift left=0.2em,"\Omega^\infty"]\cat{Sp}_{\geqslant 0}\dar[shift left=0.2em,"\Omega^\infty"]\\
			& & \cat{CGrp}(\cat{An})\uar[shift left=0.2em,"\B^\infty"] \ar[ull,bend left=15,shift left=0.2em,"\ev_{[1]}",end anchor=300,start anchor=175]
		\end{tikzcd}
	\end{equation*}
	which shows that $\Omega^\infty\Sigma^\infty(-)_+\colon \cat{An}\shortdoublelrmorphism \cat{CGrp}(\cat{An})\noloc \ev_{\langle1\rangle}$ must be an adjunction too.
\end{proof}
And finally, we can define the legendary \emph{sphere spectrum}.
\begin{defi}\label{def:SphereSpectrum}
	The \emph{reduced suspension spectrum functor} functor $\Sigma^\infty\colon \cat{An}_{*/}\rightarrow\cat{Sp}$. The \emph{\embrace{unreduced} suspension spectrum functor}
	%
	\footnote{In the old literature, and still in much of the modern one, the (unreduced) suspension spectrum of $X$ is denoted $\Sigma^\infty_+X$ rather than $\IS[X]$. However, in the modern mathematics, we think of spectra as \enquote{modules over the sphere spectrum} (a point of view that will be much elaborated on in \cref{sec:TensorProduct}), and so it seems only natural that the \enquote{free $\IS$-module on $X$} should be denoted $\IS[X]$, just as $\IZ[S]$ usually denotes the free abelian group on a set $S$.}%
	is the composition
	\begin{equation*}
		\IS[-]\colon \cat{An}\xrightarrow{(-)_+}\cat{An}_{*/}\xrightarrow{\Sigma^\infty} \cat{Sp}\,;
	\end{equation*}
	it is a left adjoint of $\Omega^\infty\colon \cat{Sp}\rightarrow \cat{An}$. The spectrum $\IS\coloneqq \IS[*]$ is called the \emph{sphere spectrum}.
\end{defi}
%In the old literature, and still in much of the modern one, the (unreduced) suspension spectrum of $X$ is denoted $\Sigma^\infty_+X$ rather than $\IS[X]$. However, in the modern mathematics, we think of spectra as \enquote{modules over the sphere spectrum} (a point of view that will be much elaborated on in \cref{sec:TensorProduct}), and so it seems only natural that the \enquote{free $\IS$-module on $X$} should be denoted $\IS[X]$, just as $\IZ[S]$ usually denotes the free abelian group on a set $S$.
	
	
	\newpage
	
	
	
	
	\newpage
	\section{The tensor product of spectra}\label{sec:TensorProduct}
	The overarching theme of these notes is to do topology without doing topology. So far, we've seen that many classical results are entirely formal consequences of abstract $\infty$-category theory. From now on, we'll show that many more classical results can be proved by doing \emph{algebra} in the stable $\infty$-category $\cat{Sp}$. We already know that $\cat{Sp}$ is additive (see the proof of \cref{lem:SpCGrpIsSp}) and so spectra can be viewed as homotopical generalisations of abelian groups. But to make the analogy between $\cat{Ab}$ and $\cat{Sp}$ really powerful, we need to be able to talk about \emph{algebras} and \emph{modules} in $\cat{Sp}$. This requires the construction of a tensor product on $\cat{Sp}$.
	
	In \cref{subsec:SymmetricMonoidal}, we'll study symmetric monoidal structures on arbitrary $\infty$-categories. In \cref{subsec:DayConvolution}, we'll construct many interesting examples, including the tensor product of spectra. In \cref{subsec:Homology}, we'll take the theory of algebras and modules in $\cat{Sp}$ for granted and use it to give the \enquote{correct} construction of homology and cohomology. Finally, there'll be a lengthy appendix. In \cref{subsec:InfinityOperads}, we'll sketch the missing theory of algebras and modules. In \cref{subsec:EnAlgebras}, we'll introduce the notion of \emph{$\IE_n$-algebras} for all $0\leqslant n\leqslant \infty$, which generalises the notions of $\IE_1$- and $\IE_\infty$-monoids that we already know. Finally, in \cref{subsec:LurieTensorProduct}, we'll prove more cool stuff about Lurie's magical $\infty$-category $\cat{Pr}^\L$ and sketch another construction of the tensor product on $\cat{Sp}$.
	
	
	\subsection{Symmetric monoidal \texorpdfstring{$\infty$}{Infinity}-categories}\label{subsec:SymmetricMonoidal}
	
	\subsection{Day convolution}\label{subsec:DayConvolution}
	
	\subsection{Homology and cohomology}\label{subsec:Homology}
	\begin{thm}\label{thm:DRIsModulesOverHR}
		The Eilenberg--MacLane functor $\Dd(\IZ)\rightarrow \cat{Sp}$ from \cref{exm:EilenbergMacLaneSpectra} upgrades to an equivalence of stable $\infty$-categories
		\begin{equation*}
			\Dd(R)\overset{\simeq}{\longrightarrow} \cat{LMod}_{R}(\cat{Sp})
		\end{equation*}
		for every ordinary ring $R$. If $R$ is commutative, then $\cat{LMod}_{R}(\cat{Sp})\simeq \cat{Mod}_R(\cat{Sp})$ admits a canonical symmetric monoidal structure and the above equivalence can be made strictly monoidal if we equip $\Dd(R)$ with the symmetric monoidal structure induced by $-\lotimes_R-$.\hfill$\blacksquare$
	\end{thm}
	\begin{cor}\label{cor:Homology}
		If $X\in\cat{An}$ is an anima, then the unreduced and reduced homology and cohomology of $X$ with coefficients in an abelian group $A$ are given by
		\begin{align*}
			\H_*(X,A)&\cong \pi_*\bigl(\IS[X]\otimes A\bigr)\,,& \widetilde{\H}_*(X,A)&\cong \pi_*\Bigl(\fib\bigl(\IS[X]\rightarrow \IS[*]\bigr)\otimes A\Bigr)\,,\\
			\H^*(X,A)&\cong \pi_{-*}\hom_\IS\bigl(\IS[X], A\bigr)\,,& \widetilde{\H}^*(X,A)&\cong \pi_{-*}\hom_\IS\Bigl(\fib\bigl(\IS[X]\rightarrow \IS[*]\bigr), A\Bigr)\,.
		\end{align*}
	\end{cor}
	
	\newpage
	
	\sectionappendix{A glimpse of higher algebra}
	\subsection{\texorpdfstring{$\infty$}{Infinity}-Operads}\label{subsec:InfinityOperads}
	
	\subsection{\texorpdfstring{$\IE_n$}{En}-Algebras and iterated loop animae}\label{subsec:EnAlgebras}
	
	\subsection{The Lurie tensor product}\label{subsec:LurieTensorProduct}
	
	
	\newpage
	\section[Cool topology applications]{Cool topology applications \coolemoji}\label{sec:CoolTopologyApplications}
	
	

	\newpage
	\renewcommand{\ParagraphOrNot}{}
	
	\setkomafont{section}{\rmfamily\bfseries\centering\Large}
	\renewcommand{\bibfont}{\small}
	
	\printbibliography
\end{document}